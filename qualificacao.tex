\documentclass[12pt]{article}
\usepackage[utf8]{inputenc}
\usepackage[brazil]{babel}
\usepackage{fancyvrb}
\usepackage[alf]{abntex2cite}
\usepackage[top=2cm,left=0.5in,right=0.5in,bottom=2cm]{geometry}
\usepackage{graphicx}
\DeclareGraphicsExtensions{.pdf}

\title{
  Caracterização da qualidade interna de ferrramentas de análise estática de
  código fonte
}
\author{Joenio Marques da Costa\\
  {\small Universidade Federal da Bahia (UFBA)} \\
  {\small joenio@colivre.coop.br}
}
\date{\today}

\begin{document}

\maketitle

\section{Introdução}

(à fazer)

\subsection{Contribuições esperadas}

(à fazer)

\section{Fundamentação teórica}

Qualidade de software é assunto presente em diversos estudos em engenharia de
software, ela diz respeito à quão bem um software é projetado e quão bem o
software está em conformidade com este design, embora existam várias
definições \cite{Kitchenham1996} \cite{McConnell2004} \cite{Iso25022}
\cite{WikiBooksSoftwareEngineering} \cite{Staines2015} podemos concluir que
existem duas dimensões básicas para medir qualidade de software,
estas dimensões estão relacionadas a características de qualidade interna e
externa de um software.

\subsection{Qualidade interna}

Segundo \citeonline{McConnell2004}, qualidade interna são aquelas
características que preocupam o desenvolvedor, como por exemplo:
manutenabilidade, flexibilidade, portabilidade, reusabilidade, readability,
testabilidade, compreensão. São questões relacionadas ao código-fonte e em
como o software foi construído, como design, boas práticas, etc.

\subsection{Qualidade externa}

Qualidade externa são aquelas características que afetam o usuário, como por
exemplo: correctiness, usability, eficiência, reliability, integridade,
adaptabilidade, acurácia, robustez. Estas características impactam
extritamente no uso do software e não em como o software foi construído.

Estes características, sejam internas ou externas, segundo a ISO 25010 (???)
podem ser divididas em subcaracterísticas, usabilidade por exemplo é dividida
em: understandability, learnability, operability, attractiveness, compliance.

Esta característica está relacionada a facilidade com que usuários aprendem e
utilizam um sistema, ela passa por questões como facilidade de instalação,
aprendizado, e uso. \citeonline{WikiBooksSoftwareEngineering} enumera algumas
questões à respeito da usabilidade úteis para mensurar tal característica:

\begin{enumerate}
  \item A interface de usuário é intuitica (auto-explicativa/auto-documentada)?
  \item É fácil de executar uma operação simples?
  \item É possível executar operações complexas?
  \item O software dá mensagens de erro compreensíveis?
  \item Os elementos se comportam como esperado?
  \item O software é bem documentado?
  \item A interface do usuário é responsiva ou muito lenta?
\end{enumerate}

\subsection{Qualidade interna x Qualidade externa}

Sabe-se que, em algum nível, as caractarísticas de qualidade interna afetam as
características de qualidade externa \cite{McConnell2004}, softwares que não
possuem boa manutenabilidade por exemplo afetam a habilidade de correção de
defeitos, que por sua vez afetam as características de exatidão (correctness)
e confiabilidade (reliability).

% Podemos afirmar então que podemos traçar algum
% nível de relação entre características de qualidade interna e externa.

% P. Clements, L. Bass, R. Kazman and G. Abowd, “Predicting Software
% Quality by Architecture-Level Evaluation”,5 th Int. conf. on Software
% Quality, vol 5, no 0, Austin, TX, pp. 485-497, 1995. ( QICID: 11205,
% ASQC)

\section{Metodologia}

%Com base nestas referências irei tomar alguns pontos básicos para contrastar
%com as ferramentas sendo anaalisadas a fim de caracterizar as mesmas em
%relação à um subcaracteristica básica externa, aquela à qual impacta em
%possibilitar qualquer outra, que é a capacidade de instalar e executar o
%software, partirei das seguintes questões:
%
%\begin{enumerate}
%  \item O software é fácil de instalar? (operability)
%  \item O software possui alguma instrução de instalação? (understandability ou learnability)
%  \item O software é facilmente encontrado para obtenção? (download, requisito para operability)
%  \item O software ao ser instalado devidamente executa suas funções mínimas sem erros? (compliance)
%\end{enumerate}
%
%As métricas de qualidade interna darão indícios de sua qualidade externa,
%especificamente em relação a caracteristicas de usabilidade, eu
%posso "ser cobaia" e tentar instalar cada ferramenta a ponto de medir a
%facilidade de instalação e uso. E isto será possível de correlacionar com as
%métricas de qualidade. Será?

Neste capítulo será apresentada a metodologia utilizada no estudo como meio
de validar as seguintes hipóteses:

\begin{enumerate}
  \item[{\bf H1:}] {\em Existem publicações sobre ferramentas de análise
    estática com disponibilidade de código-fonte}
  \item[{\bf H2:}] {\em Existem ferramentas de análise estática disponíveis
    livremente na indústria com disponibilidade de código-fonte}
  \item[{\bf H3:}] {\em Existem valores de referência para métricas de
    código-fonte para ferramentas de análise estática}
  \item[{\bf H4:}] {\em Ferramentas da indústria possuem melhores valores de
    métricas de código-fonte}
  \item[{\bf H5:}] {\em É possível relacionar a característica de qualidade
    externa usabilidade à valores de métricas de qualidade interna}
\end{enumerate}

As seções à seguir descrevem as atividades de cada etapa da metodologia.

\subsection{Planejamento do estudo}

\subsubsection{Seleção das métricas}

(à fazer - ler tese de Paulo \cite{Meirelles2013})

\subsubsection{Seleção das fontes de ferramentas de análise estática}

Para ser possível validar as hipóteses aqui levantadas é necessário realizar
uma busca por ferramentas de análise estática desenvolvidas no contexto da
academia e da indústria, para isso, será feito um planejamento detalhado para
realizar a seleção de ferramentas em cada um destes contextos.

\paragraph{Academia} No contexto acadêmica a busca por ferramentas será feita
através de artigos publicados em conferências que tenham histórico de
publicação sobre ferramentas de análise estática de código fonte. Estes
artigos serão analisados e aqueles com publicação de ferramenta de análise
estática serão selecionados.

\paragraph{Indústria} Na indústria a busca por ferramentas será feita a partir
de referências encontradas na internet, algumas organizações mantém listas de
ferramentas para análise de código-fonte, a Wikipedia também mantém uma lista
de ferramentas, estas referências serão utilizadas como ponto de partida e
cada ferramenta será analisada a fim de validar se são da indústria ou
surgiram em contexto acadêmico.

Uma vez que as ferramentas tenham sido selecionadas inicia-se a extração de
seus atributos de qualidade interna.

\subsubsection{Seleção da ferramenta de análise estática de código-fonte}

Para realizar a caracterização das ferramentas através dos seus atributos de
qualidade interna é necessário uma ferramenta capaz de analisar estaticamente
o código-fonte destas ferramentas e extrair atributos relacionados à sua
qualidade interna. Para isto utilizaremos o Analizo\cite{Terceiro2010}. Falta
Justificar! Quais vantagens? Referencias?

\subsection{Coleta de dados}

A partir das fontes selecionadas na etapa anterior serão realizadas duas
atividades para identificar e mapear as ferramentas de análise estática com
código-fonte disponível, uma atividade relacionada ao levantamento de
ferramentas da academia, outra atividade relacionada ao levantamento de
ferramentas da indústria.

\subsubsection{Ferramentas da academia}

A seleção de ferramentas será relizada através de uma revisão estruturada dos
artigos selecionados a partir das seguintes conferências:

\begin{itemize}
  \item ASE - Automated Software
    Engineering\footnote{http://ase-conferences.org}
  \item CSMR\footnote{A conferência CSMR tornou-se SANER - Software Analysis,
    Evolution, and Reengineering a partir da edição 2015.} - Conference on
    Software Maintenance and
    Reengineering\footnote{http://ansymore.uantwerpen.be/csmr-wcre}
  \item SCAM - Source Code Analysis and Manipulation Working
    Conference\footnote{http://www.ieee-scam.org}
\end{itemize}

Chamamos de revisão estruturada um processo disciplinado para seleção de
artigos a partir de critérios bem definidos de forma que seja possível a
reprodução do estudo por parte de pesquisadores interessados. Alguns
resultados preliminares podem ser consultados na Tabela \ref{artigos-do-scam}
da Seção \ref{resultados}.

\subsubsection{Ferramentas da indústria}

A seleção de ferramentas da indústria será feita a partir de uma busca manual
em fontes encontradas na Internet sobre ferramentas de análise estática.

O projeto SAMATE\footnote{
http://samate.nist.gov} - {\em Software Assurance Metrics and Tool Evaluation}
disponível em \citeonline{SamateAnalysers} mantém uma lista de
ferramentas de análise estática mantida, mais sobre o projeto SAMATE pode
ser encontrado em \citeonline{Ribeiro2015}.

O software Spin mantém em seu site uma lista de ferramentas comerciais e de
pesquisa para análise estática de código-fonte para C em
\citeonline{spinSourceAnalysisTools}.

O Instituto de Engenharia de Software do CERT mantém uma lista de ferramentas
de análise estática em \citeonline{certSecureCodingTools}.

O software Flawfinder oferece em seu site um link com referências para
inúmeras ferramentas livres, proprietárias, gratuitas mas não-livres de
ferramentas de análise estática e outros tipos de análise em
\citeonline{wheelerStaticAnalysisTools}.

Uma outra fonte contendo uma relação expressiva de ferramentas é mantida na
Wikipedia em \citeonline{wikipediaListStaticCodeAnalysis}.

Estas fontes serão pesquisadas manualmente em busca de ferramentas de análise
estática que tenham sido desenvolvidas no contexto da indústria, algums
resultados preliminares podem ser encontrados nas Tabelas
\ref{ferramentas-do-nist-com-codigo} e \ref{ferramentas-do-nist-sem-codigo}.

\subsection{Caracterização dos artigos}

Caracterização dos papers analisados na revisão estruturada e caracterização
teórica do ecosistema das ferramentas da academia.

\subsection{Caracterização das ferramentas}

Será realizada uma caracterização prática das ferramentas, tanto acadêmica
quando da indústria, através da análise e extração de métricas de código-fonte
das mesmas.

% Resultado = Documentação com métricas de referência para ferramentas de análise estática.

\subsection{Exemplo de uso}

Por fim, os valores de métricas de referência encontradas serão utilizadas
como guia para refatorar a ferramenta Analizo.

(à fazer)

\section{Conclusão}

\subsection{Resultados preliminares}\label{resultados}

A Tabela \ref{artigos-do-scam} apresenta um resumo do número de artigos em
cada edição do SCAM e quantos artigos trazem publicação de ferramenta de análise
estática com código fonte disponível.

\begin{table}
\caption{Total de artigos analisados por edições do SCAM}
\centering
\begin{tabular}{| l | c | c |}
\hline
Edição    & Total de artigos & Artigos com ferramenta \\
\hline
SCAM 2001 & 23               & -                      \\
SCAM 2002 & 18               & -                      \\
SCAM 2003 & 21               & -                      \\
SCAM 2004 & 17               & -                      \\
SCAM 2005 & 19               & -                      \\
SCAM 2006 & 22               & 2                      \\
SCAM 2007 & 23               & 1                      \\
SCAM 2008 & 29               & -                      \\
SCAM 2009 & 20               & -                      \\
SCAM 2010 & 21               & 1                      \\
SCAM 2011 & 21               & 1                      \\
SCAM 2012 & 22               & 4                      \\
SCAM 2013 & 24               & -                      \\
SCAM 2014 & 35               & 1                      \\
SCAM 2015 & ?? (pendente)    & ?                      \\
\hline
Total     & 315              & 10                     \\
\hline
\end{tabular}
\label{artigos-do-scam}
\end{table}

As Tabelas \ref{ferramentas-do-nist-com-codigo} e
\ref{ferramentas-do-nist-sem-codigo} apresentam ferramentas do NIST após
avaliação inicial sobre disponibilidade do código-fonte. Das 54 ferramentas
apenas 19 tinham código fonte disponível.

% Todos foram baixados em "dataset/NIST".

\begin{table}
\caption{Lista de ferramentas do SAMATE - NIST com código fonte não disponível}
\centering
\begin{tabular}{| l | l |}
\hline
Ferramenta & Avaliacao  \\
\hline
ABASH                     & código não disponível \\
ApexSec Security Console  & código não disponível \\
Astrée                    & código não disponível \\
bugScout                  & código não disponível \\
C/C++test®                & código não disponível \\
dotTEST™                  & código não disponível \\
Jtest®                    & código não disponível \\
HP Code Advisor (cadvise) & código não disponível \\
Checkmarx CxSAST          & código não disponível \\
CodeCenter                & código não disponível \\
CodePeer                  & código não disponível \\
CodeSecure                & site offline \\
CodeSonar                 & código não disponível \\
Coverity SAVE™            & código não disponível \\
Csur                      & código não disponível \\
DoubleCheck               & código não disponível \\
Fluid                     & código não disponível \\
Goanna Studio and Goanna Central & código não disponível \\
HP QAInspect              & código não disponível \\
Insight                   & código não disponível \\
ObjectCenter              & código não disponível \\
Parfait                   & código não disponível \\
PLSQLScanner 2008         & código não disponível \\
PHP-Sat                   & link para código offline \\
PolySpace                 & código não disponível \\
PREfix and PREfast        & código não disponivel \\
QA-C, QA-C++, QA-J        & código não disponível \\
Qualitychecker            & código não disponível \\
Rational AppScan Source Edition & código não disponível \\
Resource Standard Metrics (RSM) & código não disponível \\
SCA                       & código não disponível \\
SPARK tool set            & código não disponível \\
TBmisra®, TBsecure®       & código não disponível \\
PVS-Studio                & código não disponível \\
xg++                      & código não disponível \\
\hline
\end{tabular}
\label{ferramentas-do-nist-sem-codigo}
\end{table}

\begin{table}
\caption{Lista de ferramentas do SAMATE - NIST com código fonte disponível}
\centering
\begin{tabular}{| l | l |}
\hline
Ferramenta & Avaliacao  \\
\hline
BOON                      & código disponível \\
Clang Static Analyzer     & código disponível \\
Closure Compiler          & código disponível \\
Cppcheck                  & código disponível \\
CQual                     & código disponível \\
FindBugs                  & código disponível \\
FindSecurityBugs          & código disponível \\
Flawfinder                & código disponível \\
Jlint                     & código disponível \\
LAPSE                     & código disponível \\
Pixy                      & código disponível \\
PMD                       & código disponível \\
pylint                    & codigo disponivel \\
RATS (Rough Auditing Tool for Security) & código disponível \\
Smatch                    & código disponível \\
Splint                    & código disponível \\
UNO                       & código disponível \\
Yasca                     & código disponível \\
WAP                       & código disponível \\
\hline
\end{tabular}
\label{ferramentas-do-nist-com-codigo}
\end{table}

Assim, temos um total de 19 ferramentas da indústria com código-fonte
disponível e ??? da academia com código fonte disponível, é preciso avaliar em
qual linguagem de programação foi escrita cada ferramentas pois só iremos
analisar aquelas em C, C++ o Java que são suportadas pelo Analizo.

Ferramenta utilizada para isto foi a sloccount, uma ferramenta livre para
contagem de linhas de código fonte, dá estatística de em qual linguagem
é escrita em porcentagem. Abaixo destaco a linguagem de programação que
tem maior porção em porcentagem.

Após analise ficamos com um total de 25 ferramentas, 15 da indústria e 10 da
academia.

Segue a lista de todos os projetos e qual a fonte (industria ou academia):

\begin{table}
\caption{Lista com total de ferramentas a ser analisadas}
\centering
\begin{tabular}{| l | c | c |}
\hline
Ferramenta & Linguagem & Fonte \\
\hline
BOON                  & ansic                & industria \\
CQual                 & ansic                & industria \\
RATS                  & ansic                & industria \\
Smatch                & ansic                & industria \\
Splint                & ansic                & industria \\
UNO                   & ansic                & industria \\
Clang Static Analyzer & cpp                  & industria \\
Cppcheck              & cpp                  & industria \\
Jlint                 & cpp                  & industria \\
WAP                   & java                 & industria \\
Closure Compiler      & java                 & industria \\
FindBugs              & java                 & industria \\
FindSecurityBugs      & java                 & industria \\
Pixy                  & java                 & industria \\
PMD                   & java                 & industria \\
Indus                 & java                 & academia  \\
TACLE                 & java                 & academia  \\
JastAdd               & java                 & academia  \\
WALA                  & java                 & academia  \\
error-prone           & java                 & academia  \\
AccessAnalysis        & java                 & academia  \\
Bakar Alir            & java/ada/python      & academia  \\
InputTracer           & ansic                & academia  \\
srcML                 & cpp/cs (cs = C\# ?)   & academia  \\
Source Meter          & java                 & academia  \\
\hline
\end{tabular}
\label{total-de-ferramentas}
\end{table}

\subsection{Cronograma}

(à fazer)

\bibliography{bibliografia}

\end{document}
