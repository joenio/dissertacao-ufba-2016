\documentclass[12pt]{article}
\usepackage[utf8]{inputenc}
\usepackage[brazil]{babel}
\usepackage{fancyvrb}
\usepackage[alf]{abntex2cite}
\usepackage[top=2cm,left=0.5in,right=0.5in,bottom=2cm]{geometry}
\usepackage{graphicx}
\DeclareGraphicsExtensions{.pdf}

\title{
  Caracterização da qualidade interna de ferrramentas de análise estática de
  código fonte
}
\author{Joenio Marques da Costa\\
  {\small Universidade Federal da Bahia (UFBA)} \\
  {\small joenio@colivre.coop.br}
}
\date{\today}

\begin{document}

\maketitle

\section{Introdução}

(à fazer)

% falar aqui de ferramentas de análise estática e o porque escolhi elas? artigo
% com um reumo geral de ferramentas de analise: Source Code Analysis: A Road Map.pdf
% 
% Reprodutibilidade é a habilidade de duplicar todo um experimento ou estudo,
% seja pelo pesquisador original ou por outros interessados. Reproduzir um
% experimento é chamado de replicá-lo. A reprodutibilidade é um dos principais
% princípios do método científico.

\subsection{Contribuições esperadas}

(à fazer)

\section{Fundamentação teórica}

% 1 possibilitar replicar estudos é importante, referencia ciencia aberta, etc
% 
% 2 dentre as inúmeros requisitos para replicar temos o software/ferramenta que
% precisa estar disponível e ter qualidade mínima para ser re-utilizado
% 
% 3 para medir o quanto as ferramentas podem ser reutilizadas iremos medir a
% qualidade de software, interna e externa
% 
% 4 a qualidade interna será medida através de métricas de código-fonte
% 
% 5 os valores encontrados serão associados de alguma forma a caracteristicas de
% qualidade externa

\subsection{Engenharia de Software}

Sistemas de software são utilizados em praticamente todas as áreas do
conhecimento humano e têm exercido um papel essencial em nossa sociedade
\cite{Mafra2006}. A dependência crescente de serviços oferecidos por tais
sistemas evidencia a necessidade de produzir software de qualidade,
contornando os  desafios relacionados a funcionalidades incompletas ou
incorretas, custos acima do esperado ou prazos não cumpridos.

Diante destes desafios, surge a Engenharia de Software, uma disciplina
centrada no desenvolvimento de sistemas de software \cite{Wohlin2012} através
de uma abordagem sistemática, disciplinada, e quantificável para o
desenvolvimento, operação e manutenção \cite{SWEBOK2014}.

Nas últimas décadas, o foco em estudos empíricos na área de Engenharia de
Software tem crescido significantemente \cite{Stol2015}, resultando no uso
crescente de métodos como surveys, estudos de caso, experimentos e revisões
sistemáticas de literatura. Através destes estudos empíricos, pesquisadores
transformam a Engenharia de Software em uma disciplina mais científica e
controlável -- a  Engenharia de Software Experimental -- provendo meios para
avaliar e validar métodos, técnicas, linguagens e ferramentas.

O crescimento no número de pesquisas e publicações em Engenharia de Software
Experimental desperta a atenção para a necessidade de verificar a validade dos
estudos empíricos realizados -- um ponto central em qualquer pesquisa
científica. A validade de um estudo empírico deve ser averiguada com o intuito
de aumentar o nível de confiança em seus resultados, replicação costuma ser
citado como um importante meio para atingir tal objetivo \cite{Almqvist2006}.

Um dos primeiros artigos discutindo replicação de experimentos em Engenharia
de Software foi publicado por Basili et al. \cite{Mantyla2010} e sugere
replicação não apenas como uma escolha, mas como um possível "próximo passo" a
ser tomado após o experimento original ser concluído. Apesar do conceito
replicação de estudos empíricos em Engenharia de Software estar usualmente
associado à experimentação, argumenta-se que ele deve ser estendido para
incluir ao menos estudos de caso e surveys \cite{Basili1986}.

Em diversas linhas de pesquisa da Computação e, em especial, em Engenharia de
Software, é bastante comum que novos sistemas de software sejam desenvolvidos,
tais sistemas costumam ser utilizados como meio para atingir os resultados da
pesquisa ou, em alguns casos, são o próprio fim do estudo realizado. Neste
trabalho, tais ferramentas de software são nosso objeto de pesquisa e serão
chamados de "software científico" \ -- \citeonline{Portillo12} utiliza o
termo "research tool" para designar este mesmo tipo de software.

Softwares científicos são produtos de software e, em geral, precisam ser
avaliados com uso de métodos científicos adequados, e, é de fundamental
importância que estejam disponíveis e em funcionamento \cite{Kon2011}. A
disponibilidade destes softwares é peça fundamental para possibilitar a
reprodutibilidade dos estudos relacionados, proporcionando assim, meios para
validar pesquisas. Dentre as inúmeras iniciativas voltadas à reprodutibilidade
de pesquisas científicas, destaco aqui o movimento chamado "Ciência Aberta" e
a sua grande área "Open Reproducible Research" voltada a disseminar práticas
que garantam a capacidade de replicação de estudos.

% Reprodutibilidade é a habilidade de duplicar todo um experimento ou estudo,
% seja pelo pesquisador original ou por outros interessados. Reproduzir um
% experimento é chamado de replicá-lo. A reprodutibilidade é um dos principais
% princípios do método científico.

\subsection{Ciência Aberta}

Ciência Aberta é um movimento que tem por objetivo tornar a pesquisa
científica, seus dados e sua disseminação acessíveis à todos os interessados,
sejam amadores ou profissionais \cite{WikipediaOpenScience}. Sua principal
motivação está em possibilitar a reprodução dos resultados de pesquisas e em
garantir transparência das metodologias utilizadas, isto aumenta o impacto
social das pesquisas e gera economia de tempo e dinheiro para os pesquisadores
e para as instituições \cite{Nesta2010}.

Este movimento é guiado por princípios básicos de transparência,
acessibilidade e reusabilidade universais, disseminadas via ferramentas
online, ele é dividido em quatro grandes áreas: (1) Open Access, (2) Open
Data, (3) Open Source e (4) Open Reproducible Research. Dentre elas destaca-se
a Open Reproducible Research por preocupar-se com a reprodutibilidade dos
resultados de pesquisas de forma independente \cite{Stodden2009} e aberta, no
entanto, esta área tem recebido ainda pouca atenção da comunidade de pesquisa
\cite{Nancy2015} \cite{Grand2010Open} apesar do aumento geral do interesse
pelas práticas da Ciência Aberta \cite{Grand2010}.

Enquanto pesquisadores publicam artigos descrevendo e divulgando seus
resultados, é raro que façam o mesmo com toda a produção gerada durante a
pesquisa. A maioria dos componentes necessários para a reprodução dos
resultados de uma pesquisa -- por exemplo, códigos fonte e dados -- usualmente
permanecem não publicados. Este é um problema sério já que um dos fundamentos
da ciência é que novas descobertas sejam reproduzidas antes de serem
consideradas parte da base de conhecimento \cite{Stodden2009}.

Neste sentido, \citeonline{Prlic2012} dão dicas para o desenvolvimento aberto de software
científico e citam que disponibilizar o código criado durante pesquisas não
apenas aumenta o impacto como também se torna essencial para outros
reproduzirem os resultados encontrados. Eles citam ainda que manutenabilidade
e disponibilidade do software após a publicação é o maior problema enfrentado
pelos pesquisadores que desenvolvem tais softwares, e é aí que a
participação no desenvolvimento aberto desde o início pode trazer maior
benefício.

Dentro deste contexto, e considerando que pesquisas em engenharia de software
produzem bastante softwares científicos, surge a preocupação de avaliar a
qualidade de tais softwares a partir de métodos científicos adequados,
especialmente em relação a sua manutenabilidade e disponibilidade.

\subsection{Qualidade de software}

Qualidade de software é assunto presente em diversos estudos em engenharia de
software, ela diz respeito à quão bem um software é projetado e o quanto este
software está em conformidade com o projeto, embora existam inúmeras definições
\cite{Kitchenham1996} \cite{McConnell2004} \cite{Iso25022}
\cite{WikiBooksSoftwareEngineering} \cite{Staines2015} pode-se afirmar que
existem duas dimensões básicas para medir a qualidade de um software, estas
dimensões estão relacionadas à características de qualidade interna e de
qualidade externa de um software.

\subsubsection{Qualidade interna}

Segundo \citeonline{McConnell2004}, qualidade interna são aquelas
características que preocupam o desenvolvedor, como por exemplo:
manutenabilidade, flexibilidade, portabilidade, reusabilidade, readability,
testabilidade, compreensão. São questões relacionadas ao código-fonte e em
como o software foi construído, como design, boas práticas, etc.

\subsubsection{Qualidade externa}

Qualidade externa são aquelas características que afetam o usuário, como por
exemplo: correctiness, usability, eficiência, reliability, integridade,
adaptabilidade, acurácia, robustez. Estas características impactam
extritamente no uso do software e não em como o software foi construído.

Estes características, sejam internas ou externas, segundo a ISO 25010 (???)
podem ser divididas em subcaracterísticas, usabilidade por exemplo é dividida
em: understandability, learnability, operability, attractiveness, compliance.
Esta característica está relacionada a facilidade com que usuários aprendem e
utilizam um sistema, ela passa por questões como facilidade de instalação,
aprendizado, e uso. \citeonline{WikiBooksSoftwareEngineering} enumera algumas
questões à respeito da usabilidade úteis para mensurar tal característica:

\begin{enumerate}
  \item A interface de usuário é intuitica (auto-explicativa/auto-documentada)?
  \item É fácil de executar uma operação simples?
  \item É possível executar operações complexas?
  \item O software dá mensagens de erro compreensíveis?
  \item Os elementos se comportam como esperado?
  \item O software é bem documentado?
  \item A interface do usuário é responsiva ou muito lenta?
\end{enumerate}

\subsubsection{Qualidade interna x Qualidade externa}

Sabe-se que, em algum nível, as caractarísticas de qualidade interna afetam as
características de qualidade externa \cite{McConnell2004}, softwares que não
possuem boa manutenabilidade por exemplo afetam a habilidade de correção de
defeitos, que por sua vez afetam as características de exatidão (correctness)
e confiabilidade (reliability).

\subsection{Métricas de código-fonte}

Uma métrica, segundo definição da ISO/IEC 25010 (2001), é a composição de
procedimentos para a definição de escalas e métodos para medidas
\cite{Meirelles2013}.

Métricas são classificadas quanto aos critérios utilizados para determiná-las,
quanto ao método de obtenção \cite{Meirelles2013}. Entre as classificações
possíveis temos aquelas consideradas objetivas que tratam de características
do código-fonte, muitos trabalhos tentam correlacionar métricas de software
com qualidade de software.  (Subramanyam e Krishnan, 2003) Briand et al.
(2000) (Zuse, 1990).

Em suma, há uma variedade de métricas baseadas análise estática do código-fonte que permitem
a avaliação de produtos de software (Gousios et al., 2007).

% Podemos afirmar então que podemos traçar algum
% nível de relação entre características de qualidade interna e externa.

\subsubsection{Análise de Código Fonte}

O rápido crescimento no número de softwares nas últimas décadas leva a uma
crescente demanda por mecanismos e ferramentas de apoio à compreensão de,
desenvolvedores necessitam entender em profundade a implementação de um
determinado software antes de realizar atividades de correção ou refatoração
de forma eficiente \cite{Kirkov2010}.

Isto é evidenciado ao perceber que a complexidade dos softwares vem crescendo
à cada dia \cite{Kirkov2010}, tornando assim, extremamente útil a existência
de ferramentas de análise automática de código-fonte, tais ferramentas
auxiliam os desenvolvedores e engenheiros à compreender a implementação de um
determinado software de forma profunda e abrangente.

Essas ferramentas geram modelos de alto nível representando entidades,
relacionamentos, métricas, características ou outra informação qualquer
extraída diretamente do código-fonte, e permitem que sejam geradas
visualizações em diversos níveis de abstração representando o código-fonte.

Dentre as inúmeras sub-áreas da engenharia de software, este trabalho irá
focar do domínio de ferramentas de análise de código-fonte, esta escolha tem
por base o domínio do pesquisador nesta área e a escolha de um domínio é
necessário para reduzir o escopo e viabilizar a revisão de estudos e
ferramentas de um domínio específico, do contrário o trabalho seria muito
extenso e a revisão sistemática seria inviável dentro do prazo de um trabalho
de pesquisa num mestrado.

\section{Metodologia}

%Com base nestas referências irei tomar alguns pontos básicos para contrastar
%com as ferramentas sendo anaalisadas a fim de caracterizar as mesmas em
%relação à um subcaracteristica básica externa, aquela à qual impacta em
%possibilitar qualquer outra, que é a capacidade de instalar e executar o
%software, partirei das seguintes questões:
%
%\begin{enumerate}
%  \item O software é fácil de instalar? (operability)
%  \item O software possui alguma instrução de instalação? (understandability ou learnability)
%  \item O software é facilmente encontrado para obtenção? (download, requisito para operability)
%  \item O software ao ser instalado devidamente executa suas funções mínimas sem erros? (compliance)
%\end{enumerate}
%
%As métricas de qualidade interna darão indícios de sua qualidade externa,
%especificamente em relação a caracteristicas de usabilidade, eu
%posso "ser cobaia" e tentar instalar cada ferramenta a ponto de medir a
%facilidade de instalação e uso. E isto será possível de correlacionar com as
%métricas de qualidade. Será?

Neste capítulo será apresentada a metodologia utilizada no estudo como meio
de validar as seguintes hipóteses:

\begin{enumerate}
  \item[{\bf H1:}] {\em Existem publicações sobre ferramentas de análise
    estática com disponibilidade de código-fonte}
  \item[{\bf H2:}] {\em Existem ferramentas de análise estática disponíveis
    livremente na indústria com disponibilidade de código-fonte}
  \item[{\bf H3:}] {\em Existem valores de referência para métricas de
    código-fonte para ferramentas de análise estática}
  \item[{\bf H4:}] {\em Ferramentas da indústria possuem melhores valores de
    métricas de código-fonte}
  \item[{\bf H5:}] {\em É possível relacionar a característica de qualidade
    externa usabilidade à valores de métricas de qualidade interna}
\end{enumerate}


% +\begin{itemize}
% +  \item Existe publicação sobre ferramentas de análise estática
% +  \item As publicações sobre ferramentas de análise estática disponibilizam o código fonte
% +  \item Existem ferramentas de análise estática disponíveis livremente na indústria
% +  \item O código fonte disponibilizado está (de fato) disponível livremente
% +  \item Existe similaridade em termos de métricas entre as ferramentas
% +  \item Existem valores de referência para métricas de código fonte
% +  \item As boas (?) ferramentas tem uma arquitetura semelhante
% +  \item É possível extrair uma arquitetura de referência a partir das ferramentas analisadas
% +  \item A arquitetura de referência gera um conjunto de métricas de referência
% +\end{itemize}
% +\subsection{Trabalhos relacionados}
%  
% +Pesquisas sobre o recente tópico, Ciência Aberta, \citeonline{Prlic2012} dão
% +dicas para o desenvolvimento aberto de software científico e citam que
% +disponibilizar o código criado durante pesquisas não apenas aumenta o impacto
% +como também se torna essencial para outros reproduzirem os resultados
% +encontrados. Eles citam ainda que manutenabilidade e disponibilidade do
% +software após a publicação é o maior problema enfrentado pelos pesquisadores
% +que desenvolvem tais softwares, e é aí que a participação no desenvolvimento
% +aberto desde o início pode trazer maior benefício.
%  
% +\citeonline{Portillo12} faz uma revisão sistemática caracterizando ferramentas
% +usadas em engenharia de software global, designa tais feramentas como
% +"research tool".
%  
% +\citeonline{Kon2011} cita que oftwares científicos são produtos de software e,
% +em geral, precisam ser avaliados com uso de métodos científicos adequados, e,
% +é de fundamental importância que estejam disponíveis e em funcionamento.


As seções à seguir descrevem as atividades de cada etapa da metodologia.

\subsection{Planejamento do estudo}

\subsubsection{Seleção das métricas}

\citeonline{Meirelles2013} realizou um estudo onde associou-se
qualidade do produto de software à qualidade de código através de métricas de
código-fonte como indicador para o sucesso de projetos de software livre,
este estudo selecionou algumas métricas de código-fonte como base, a partir de
critérios como existência de valores de referência para as métricas, além de X
Y e B. Aqui deste trabalho utilizaremos como base às mesmas métricas:

\paragraph{Métricas de tamanho:}
LOC (Lines of Code), AMLOC (Average Method LOC), Total Number of Modules or Classes

\paragraph{Indicadores estruturais:}
NOA (Number of Attributes), NOM (Number of Methods), NPA (Number of Public
Attributes), NPM (Number of Public Methods), ANPM (Average Number of
Parameters per Method), DIT (Depth of Inheritance Tree), NOC (Number of
Children), RFC (Response For a Class), ACCM (Average Cyclomatic Complexity per
Method)

\paragraph{Métricas de acoplamento:}
ACC (Afferent Connections per Class), CBO (Coupling Between Objects), COF
(Coupling Factor)

\paragraph{Métricas de coesão:}
LCOM (Lack of Cohesion in Methods), SC (Structural Complexity)

\subsubsection{Seleção das fontes de ferramentas de análise estática}

Para ser possível validar as hipóteses aqui levantadas é necessário realizar
uma busca por ferramentas de análise estática desenvolvidas no contexto da
academia e da indústria, para isso, será feito um planejamento detalhado para
realizar a seleção de ferramentas em cada um destes contextos.

\paragraph{Academia} No contexto acadêmica a busca por ferramentas será feita
através de artigos publicados em conferências que tenham histórico de
publicação sobre ferramentas de análise estática de código fonte. Estes
artigos serão analisados e aqueles com publicação de ferramenta de análise
estática serão selecionados.

\paragraph{Indústria} Na indústria a busca por ferramentas será feita a partir
de referências encontradas na internet, algumas organizações mantém listas de
ferramentas para análise de código-fonte, a Wikipedia também mantém uma lista
de ferramentas, estas referências serão utilizadas como ponto de partida e
cada ferramenta será analisada a fim de validar se são da indústria ou
surgiram em contexto acadêmico.

Uma vez que as ferramentas tenham sido selecionadas inicia-se a extração de
seus atributos de qualidade interna.

\subsubsection{Seleção da ferramenta de análise estática de código-fonte}

Para realizar a caracterização das ferramentas através dos seus atributos de
qualidade interna é necessário uma ferramenta capaz de analisar estaticamente
o código-fonte destas ferramentas e extrair atributos relacionados à sua
qualidade interna. Para isto utilizaremos o Analizo\cite{Terceiro2010}. Falta
Justificar! Quais vantagens? Referencias?

\subsection{Coleta de dados}

A partir das fontes selecionadas na etapa anterior serão realizadas duas
atividades para identificar e mapear as ferramentas de análise estática com
código-fonte disponível, uma atividade relacionada ao levantamento de
ferramentas da academia, outra atividade relacionada ao levantamento de
ferramentas da indústria.

\subsubsection{Ferramentas da academia}

A seleção de ferramentas será relizada através de uma revisão estruturada dos
artigos selecionados a partir das seguintes conferências:

\begin{itemize}
  \item ASE - Automated Software
    Engineering\footnote{http://ase-conferences.org}
  \item CSMR\footnote{A conferência CSMR tornou-se SANER - Software Analysis,
    Evolution, and Reengineering a partir da edição 2015.} - Conference on
    Software Maintenance and
    Reengineering\footnote{http://ansymore.uantwerpen.be/csmr-wcre}
  \item SCAM - Source Code Analysis and Manipulation Working
    Conference\footnote{http://www.ieee-scam.org}
  \item ICSME - International Conference on Software Maintenance and
    Evolution\footnote{http://www.icsme.org}
\end{itemize}

Chamamos de revisão estruturada um processo disciplinado para seleção de
artigos a partir de critérios bem definidos de forma que seja possível a
reprodução do estudo por parte de pesquisadores interessados. Alguns
resultados preliminares podem ser consultados na Tabela \ref{artigos-do-scam}
da Seção \ref{resultados}.

\subsubsection{Ferramentas da indústria}

A seleção de ferramentas da indústria será feita a partir de uma busca manual
em fontes encontradas na Internet sobre ferramentas de análise estática.

O projeto SAMATE\footnote{
http://samate.nist.gov} - {\em Software Assurance Metrics and Tool Evaluation}
disponível em \citeonline{SamateAnalysers} mantém uma lista de
ferramentas de análise estática mantida, mais sobre o projeto SAMATE pode
ser encontrado em \citeonline{Ribeiro2015}.

O software Spin mantém em seu site uma lista de ferramentas comerciais e de
pesquisa para análise estática de código-fonte para C em
\citeonline{spinSourceAnalysisTools}.

O Instituto de Engenharia de Software do CERT mantém uma lista de ferramentas
de análise estática em \citeonline{certSecureCodingTools}.

O software Flawfinder oferece em seu site um link com referências para
inúmeras ferramentas livres, proprietárias, gratuitas mas não-livres de
ferramentas de análise estática e outros tipos de análise em
\citeonline{wheelerStaticAnalysisTools}.

Uma outra fonte contendo uma relação expressiva de ferramentas é mantida na
Wikipedia em \citeonline{wikipediaListStaticCodeAnalysis}.

Estas fontes serão pesquisadas manualmente em busca de ferramentas de análise
estática que tenham sido desenvolvidas no contexto da indústria, algums
resultados preliminares podem ser encontrados nas Tabelas
\ref{ferramentas-do-nist-com-codigo} e \ref{ferramentas-do-nist-sem-codigo}.

\subsection{Caracterização dos artigos}

Caracterização dos papers analisados na revisão estruturada e caracterização
teórica do ecosistema das ferramentas da academia.

\subsection{Caracterização das ferramentas}

Será realizada uma caracterização prática das ferramentas, tanto acadêmica
quando da indústria, através da análise e extração de métricas de código-fonte
das mesmas.

% Resultado = Documentação com métricas de referência para ferramentas de análise estática.

\subsection{Exemplo de uso}

Por fim, os valores de métricas de referência encontradas serão utilizadas
como guia para refatorar a ferramenta Analizo.

Ler o TCC "Código Limpo e seu Mapeamento para Métricas de Código Fonte" onde
foi feito trabalho similar.

(à fazer)

\section{Conclusão}

\subsection{Resultados preliminares}\label{resultados}

A Tabela \ref{artigos-do-scam} apresenta um resumo do número de artigos em
cada edição do SCAM e quantos artigos trazem publicação de ferramenta de análise
estática com código fonte disponível.

\begin{table}
\caption{Total de artigos analisados por edições do SCAM}
\centering
\begin{tabular}{| l | c | c |}
\hline
Edição    & Total de artigos & Artigos com ferramenta \\
\hline
SCAM 2001 & 23               & -                      \\
SCAM 2002 & 18               & -                      \\
SCAM 2003 & 21               & -                      \\
SCAM 2004 & 17               & -                      \\
SCAM 2005 & 19               & -                      \\
SCAM 2006 & 22               & 2                      \\
SCAM 2007 & 23               & 1                      \\
SCAM 2008 & 29               & -                      \\
SCAM 2009 & 20               & -                      \\
SCAM 2010 & 21               & 1                      \\
SCAM 2011 & 21               & 1                      \\
SCAM 2012 & 22               & 4                      \\
SCAM 2013 & 24               & -                      \\
SCAM 2014 & 35               & 1                      \\
SCAM 2015 & ?? (pendente)    & ?                      \\
\hline
Total     & 315              & 10                     \\
\hline
\end{tabular}
\label{artigos-do-scam}
\end{table}

As Tabelas \ref{ferramentas-do-nist-com-codigo} e
\ref{ferramentas-do-nist-sem-codigo} apresentam ferramentas do NIST após
avaliação inicial sobre disponibilidade do código-fonte. Das 54 ferramentas
apenas 19 tinham código fonte disponível.

% Todos foram baixados em "dataset/NIST".

\begin{table}
\caption{Lista de ferramentas do SAMATE - NIST com código fonte não disponível}
\centering
\begin{tabular}{| l | l |}
\hline
Ferramenta & Avaliacao  \\
\hline
ABASH                     & código não disponível \\
ApexSec Security Console  & código não disponível \\
Astrée                    & código não disponível \\
bugScout                  & código não disponível \\
C/C++test®                & código não disponível \\
dotTEST™                  & código não disponível \\
Jtest®                    & código não disponível \\
HP Code Advisor (cadvise) & código não disponível \\
Checkmarx CxSAST          & código não disponível \\
CodeCenter                & código não disponível \\
CodePeer                  & código não disponível \\
CodeSecure                & site offline \\
CodeSonar                 & código não disponível \\
Coverity SAVE™            & código não disponível \\
Csur                      & código não disponível \\
DoubleCheck               & código não disponível \\
Fluid                     & código não disponível \\
Goanna Studio and Goanna Central & código não disponível \\
HP QAInspect              & código não disponível \\
Insight                   & código não disponível \\
ObjectCenter              & código não disponível \\
Parfait                   & código não disponível \\
PLSQLScanner 2008         & código não disponível \\
PHP-Sat                   & link para código offline \\
PolySpace                 & código não disponível \\
PREfix and PREfast        & código não disponivel \\
QA-C, QA-C++, QA-J        & código não disponível \\
Qualitychecker            & código não disponível \\
Rational AppScan Source Edition & código não disponível \\
Resource Standard Metrics (RSM) & código não disponível \\
SCA                       & código não disponível \\
SPARK tool set            & código não disponível \\
TBmisra®, TBsecure®       & código não disponível \\
PVS-Studio                & código não disponível \\
xg++                      & código não disponível \\
\hline
\end{tabular}
\label{ferramentas-do-nist-sem-codigo}
\end{table}

\begin{table}
\caption{Lista de ferramentas do SAMATE - NIST com código fonte disponível}
\centering
\begin{tabular}{| l | l |}
\hline
Ferramenta & Avaliacao  \\
\hline
BOON                      & código disponível \\
Clang Static Analyzer     & código disponível \\
Closure Compiler          & código disponível \\
Cppcheck                  & código disponível \\
CQual                     & código disponível \\
FindBugs                  & código disponível \\
FindSecurityBugs          & código disponível \\
Flawfinder                & código disponível \\
Jlint                     & código disponível \\
LAPSE                     & código disponível \\
Pixy                      & código disponível \\
PMD                       & código disponível \\
pylint                    & codigo disponivel \\
RATS (Rough Auditing Tool for Security) & código disponível \\
Smatch                    & código disponível \\
Splint                    & código disponível \\
UNO                       & código disponível \\
Yasca                     & código disponível \\
WAP                       & código disponível \\
\hline
\end{tabular}
\label{ferramentas-do-nist-com-codigo}
\end{table}

Assim, temos um total de 19 ferramentas da indústria com código-fonte
disponível e ??? da academia com código fonte disponível, é preciso avaliar em
qual linguagem de programação foi escrita cada ferramentas pois só iremos
analisar aquelas em C, C++ o Java que são suportadas pelo Analizo.

Ferramenta utilizada para isto foi a sloccount, uma ferramenta livre para
contagem de linhas de código fonte, dá estatística de em qual linguagem
é escrita em porcentagem. Abaixo destaco a linguagem de programação que
tem maior porção em porcentagem.

Após analise ficamos com um total de 25 ferramentas, 15 da indústria e 10 da
academia.

Segue na Tabela \ref{total-de-ferramentas} a lista de todos os projetos e qual
a fonte (industria ou academia):

\begin{table}
\caption{Lista com total de ferramentas a serem analisadas}
\centering
\begin{tabular}{| l | c | c |}
\hline
Ferramenta & Linguagem & Fonte \\
\hline
BOON                  & ansic                & industria \\
CQual                 & ansic                & industria \\
RATS                  & ansic                & industria \\
Smatch                & ansic                & industria \\
Splint                & ansic                & industria \\
UNO                   & ansic                & industria \\
Clang Static Analyzer & cpp                  & industria \\
Cppcheck              & cpp                  & industria \\
Jlint                 & cpp                  & industria \\
WAP                   & java                 & industria \\
Closure Compiler      & java                 & industria \\
FindBugs              & java                 & industria \\
FindSecurityBugs      & java                 & industria \\
Pixy                  & java                 & industria \\
PMD                   & java                 & industria \\
Indus                 & java                 & academia  \\
TACLE                 & java                 & academia  \\
JastAdd               & java                 & academia  \\
WALA                  & java                 & academia  \\
error-prone           & java                 & academia  \\
AccessAnalysis        & java                 & academia  \\
Bakar Alir            & java/ada/python      & academia  \\
InputTracer           & ansic                & academia  \\
srcML                 & cpp/cs (cs = C\# ?)   & academia  \\
Source Meter          & java                 & academia  \\
\hline
\end{tabular}
\label{total-de-ferramentas}
\end{table}

\subsection{Cronograma}

(à fazer)

\bibliography{bibliografia}

\end{document}
