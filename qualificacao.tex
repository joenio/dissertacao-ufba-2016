\documentclass[qual, classic, a4paper]{ufbathesis}
\usepackage[utf8]{inputenc}
\usepackage[brazil]{babel}
\usepackage{fancyvrb}
\usepackage[alf]{abntex2cite}
\usepackage{graphicx}
\DeclareGraphicsExtensions{.pdf}

%\date{?? de Junho de 2016}
\adviser[f]{Profa. Dra. Christina von Flach G. Chavez}
\coadviser{Prof. Dr. Paulo Roberto Miranda Meirelles}

\title{
  Caracterização da qualidade interna de ferramentas de análise estática de
  código fonte
}
\author{Joenio Marques da Costa\\
  {\small joenio@joenio.me}
}

\begin{document}
\frontpage
\frontmatter
\presentationpage

%\acknowledgements
%DIGITE OS AGRADECIMENTOS AQUI
%
%\resumo
%DIGITE O RESUMO AQUI
%
%\begin{keywords}
%DIGITE AS PALAVRAS-CHAVE AQUI
%\end{keywords}
%
%\abstract
%RESUMO EM INGLÊS
%
%\begin{keywords}
%DIGITE AS PALAVRAS-CHAVE AQUI
%\end{keywords}

\tableofcontents
\listoffigures
\listoftables
\mainmatter

\chapter{Introdução}

(à fazer)

% falar aqui de ferramentas de análise estática e o porque escolhi elas? artigo
% com um reumo geral de ferramentas de analise: Source Code Analysis: A Road Map.pdf

\section{Contribuições esperadas}

(à fazer)

\chapter{Fundamentação teórica}

\section{Engenharia de Software}

Sistemas de software são utilizados em praticamente todas as áreas do
conhecimento humano e têm exercido um papel essencial em nossa sociedade
\cite{Mafra2006}. A dependência crescente de serviços oferecidos por tais
sistemas evidencia a necessidade de produzir software de qualidade,
contornando os  desafios relacionados a funcionalidades incompletas ou
incorretas, custos acima do esperado ou prazos não cumpridos.

Diante destes desafios, surge a Engenharia de Software, uma disciplina
centrada no desenvolvimento de sistemas de software através
de uma abordagem sistemática, disciplinada, e quantificável para o
desenvolvimento, operação e manutenção \cite{SWEBOK2014}.

Nas últimas décadas, o foco em estudos empíricos na área de Engenharia de
Software tem crescido significantemente \cite{Stol2015}, resultando no uso
crescente de métodos como surveys, estudos de caso, experimentos e revisões
sistemáticas de literatura. Através destes estudos empíricos, pesquisadores
transformam a Engenharia de Software em uma disciplina mais científica e
controlável -- a  Engenharia de Software Experimental -- provendo meios para
avaliar e validar métodos, técnicas, linguagens e ferramentas.

Não raro, muitos destes estudos criam novos sistemas de software, tais
sistemas costumam ser utilizados como meio para atingir os resultados da
pesquisa ou, em alguns casos, são o próprio fim do estudo realizado. Neste
trabalho, tais ferramentas de software são nosso objeto de pesquisa e serão
chamados de "software científico" \ -- \citeonline{Portillo12} utiliza o termo
"research tool" para designar este mesmo tipo de software.

\section{Software científico}

Softwares científicos são ferramentas de software desenvolvidas no decorrer de
pesquisas científicas como parte de um estudo, podem ser pequenos scripts,
protótipos, ou mesmo produtos de software completos que demonstram ou refletem
os resultados de uma pesquisa. Em Engenharia de Software este tipo de software
desempenha um papel essencial e sua importância pode ser notada através do
grande número de conferências com sessões específicas sobre publicação de
ferramentas.

\citeonline{Kon2011} em um estudo sobre como pesquisas em Engenharia de
Software podem se beneficiar do ecosistema de Software Livre faz uma análise
de 10 edições do SBES\footnote{Simpósio Brasileiro de Engenharia de Software}
e conclui que apesar do aumento do interesse por parte dos pesquisadores em
disponibilizar o código-fonte de suas ferramentas isto ainda é uma minoria. O
que confirma a preocupação de \citeonline{Krishnamurthi2015} em um estudo
sobre repetibilidade de pesquisas científicas, onde chamam atenção para o papel
central que os artefatos de software possuem em pesquisas de ciência da
computação e questionam: "Onde está o software nas pesquisas sobre linguagem
de programação?".

A partir daí podemos afirmar que softwares científicos são peça fundamental
para que pesquisadores independentes possam reproduzir, validar ou expandir os
resultados encontrados em estudos anteriores e assim aumentar o rigor e a
qualidade científica de tais pesquisas \cite{Vitek2011}.

\section{Reprodutibilidade}

Reprodutibilidade é a habilidade de replicar um experimento ou estudo em sua
totalidade a fim de confirmar suas hipóteses e resultados, apesar de ser uma
prática central do método científico ainda é um grande obstáculo em muitos
estudos. Enquanto pesquisadores publicam artigos descrevendo e divulgando seus
resultados, é raro que façam o mesmo com toda a produção gerada durante a
pesquisa. A maioria dos componentes necessários para a reprodução dos
resultados de uma pesquisa -- por exemplo, código-fonte e dados -- usualmente
permanecem não publicados.

Isto se configura como uma barreira para a reprodutibilidade, e
consequentemente para a repetição, replicação e variaçãode estudos
\cite{Feitelson2015} já que a disponibilidade de código-fonte é o mínimo
necessário para isto, como pode ser visto no espectro de
reprodutibilidade de \citeonline{Peng2011} reproduzido aqui na Figura
\ref{reproducibility-spectrum}.

\begin{figure}[h]
  \center
  \includegraphics[scale=0.25]{imagens/reproducibility-spectrum.png}
  \caption{The spectrum of reproducibility\cite{Peng2011}}
  \label{reproducibility-spectrum}
\end{figure}

Dentro deste contexto, e considerando que muitos estudos ainda sofrem com
dificuldades de repetição \cite{Tang2016}, surge a preocupação de avaliar a
qualidade dos softwares científicos a partir de métodos adequados,
especialmente em relação a sua manutenabilidade e disponibilidade por serem
problemas comuns enfrentados pelos pesquisadores \cite{Prlic2012}.

\section{Qualidade de software}

Qualidade de software diz respeito à quão bem um software é projetado e o
quanto este software está em conformidade com o projeto, embora existam
inúmeras definições há um concenso de que existem duas dimensões para medir a
qualidade de um software, estas dimensões estão relacionadas à características
de qualidade interna e de qualidade externa.

Segundo \citeonline{McConnell2004}, qualidade interna são aquelas
características que preocupam o desenvolvedor, como: manutenabilidade,
flexibilidade, portabilidade, reusabilidade, legibilidade, testabilidade e
compreensão. São questões relacionadas ao código-fonte e em como o software
foi construído, como design e boas práticas por exemplo.

Qualidade externa são aquelas características que afetam o usuário, como por
exemplo: exatidão, usabilidade, eficiência, confiabilidade, integridade,
adaptabilidade, acurácia e robustez. Estas características impactam
extritamente no uso do software e não em como o software foi construído.

Segundo a ISO/IEC 25010 \cite{iso2011iec25010} estas características podem ser
divididas em subcaracterísticas. A característica usabilidade por exemplo é
dividida nas seguintes subcaracterísticas: capacidade de compreensão,
capacidade de aprendizado, operabilidade, atratividade e conformidade. Esta
caracteristica está relacionada, por exemplo, a facilidade com que usuários aprendem e
utilizam um sistema, e passa por questões como facilidade de instalação,
aprendizado, e uso.

Sabe-se que, em algum nível, as caractarísticas de qualidade interna afetam as
características de qualidade externa \cite{McConnell2004}, softwares que não
possuem boa manutenabilidade por exemplo afetam a habilidade de correção de
defeitos, que por sua vez afetam as características de exatidão e
confiabilidade.

Dito isto podemos perceber que é possível inferir características de qualidade
externa de um software a partir de suas qualidades internas, o que pode ser
realizado a partir da análise de suas métricas de código-fonte.

\section{Métricas de código-fonte}

Uma métrica, segundo a definição da ISO/IEC 25010 \cite{iso2011iec25010}, é a
composição de procedimentos para a definição de escalas e métodos para
medidas. Métricas de software podem ser classificadas em três categorias: métricas de
produto, métricas de processo e métricas de projeto.

Métricas de processo medem atributos relacionados
ao ciclo de desenvolvimento do software. Métricas de produto são aquelas que
descrevem as características de artefatos do desenvolvimento, como documentos,
diagramas, código-fonte e arquivos binários. Métricas de projeto são aquelas
que descrevem as características dos recursos disponíveis ao desenvolvimento.

Métricas de produto podem ser classificadas entre internas ou externas, ou
seja, aquelas que medem propriedades visíveis apenas aos desenvolvedores ou
que medem propriedades visíveis aos usuários, respectivamente.
Métricas internas podem ainda ser divididas de acordo com o artefato
analisado, seja documentos, diagramas ou código-fonte.

Neste trabalho iremos adotar métricas de design e métricas de código-fonte
para extrair propriedades internas dos softwares científicos a fim de medir a
sua qualidade. A seleção das métricas tomará como base o trabalho realizado
por \citeonline{Meirelles2013} onde um estudo associando qualidade de software
à qualidade de código-fonte foi conduzido através da observação de métricas de
código-fonte.

\subsection{Ferramentas de análise estática de código-fonte}

A extração de métricas de código-fonte, usualmente, ocorre através de ferramentas de
análise automática de software, segundo \citeonline{Kirkov2010} estas
ferramentas possuem uma anatomia comum, composta de quatro componentes básicos
- construção de modelos; algoritmos de análise e reconhecimento de padrões;
base de conhecimento de padrões; e representação final.

A construção de modelos de um programa é o primeiro passo e é feito por um
parser de código-fonte \cite{Binkley2007}. A base de conhecimento de padrões é usada para
representar e armazenar informações sobre potenciais problemas encontrados no
código-fonte. O objetivo do algoritmo de análise e reconhecumento de padroes é
classificar as informações encontradas no modelo a partir da base de
conhecumento de padroes. A representação final é um relatório ou outro tipo de
visualização gerado para para o usuário (desenvolvedor) através de uma
interface de usuário apropriada.

Com o uso de ferramentas de análise estática de código-fonte, seguindo o seu
funcionamento básico, é possível gerar como resultado final uma saída contendo
métricas de design e código-fonte represresentando características e atributos
de qualidade interna de um determinado software.

\chapter{Metodologia}

%Com base nestas referências irei tomar alguns pontos básicos para contrastar
%com as ferramentas sendo anaalisadas a fim de caracterizar as mesmas em
%relação à um subcaracteristica básica externa, aquela à qual impacta em
%possibilitar qualquer outra, que é a capacidade de instalar e executar o
%software, partirei das seguintes questões:
%
%O software pode ser instalado facilmente? (Portability -> Installability)
% * O software possui alguma instrução de instalação? (understandability ou learnability)
% * O software é facilmente encontrado para obtenção? (download, requisito para operability)
%O usuário consegue utilizar o sistema sem muito esforço? (Usability -> Operability)
% * O software ao ser instalado devidamente executa suas funções mínimas sem erros? (compliance)
%\cite{Padayachee2010}
%
%As métricas de qualidade interna darão indícios de sua qualidade externa,
%especificamente em relação a caracteristicas de usabilidade, eu
%posso "ser cobaia" e tentar instalar cada ferramenta a ponto de medir a
%facilidade de instalação e uso. E isto será possível de correlacionar com as
%métricas de qualidade. Será?

Neste capítulo será apresentada a metodologia utilizada no estudo como meio
de validar as seguintes hipóteses:

\begin{enumerate}
  \item[{\bf H1:}] {\em Existem publicações sobre ferramentas de análise
    estática com disponibilidade de código-fonte}
  \item[{\bf H2:}] {\em Existem ferramentas de análise estática disponíveis
    livremente na indústria com disponibilidade de código-fonte}
  \item[{\bf H3:}] {\em Existem valores de referência para métricas de
    código-fonte para ferramentas de análise estática}
  \item[{\bf H4:}] {\em Ferramentas da indústria possuem melhores valores de
    métricas de código-fonte}
  \item[{\bf H5:}] {\em É possível relacionar a característica de qualidade
    externa usabilidade à valores de métricas de qualidade interna}
\end{enumerate}

% +\section{Trabalhos relacionados}
%  
% +Pesquisas sobre o recente tópico, Ciência Aberta, \citeonline{Prlic2012} dão
% +dicas para o desenvolvimento aberto de software científico e citam que
% +disponibilizar o código criado durante pesquisas não apenas aumenta o impacto
% +como também se torna essencial para outros reproduzirem os resultados
% +encontrados. Eles citam ainda que manutenabilidade e disponibilidade do
% +software após a publicação é o maior problema enfrentado pelos pesquisadores
% +que desenvolvem tais softwares, e é aí que a participação no desenvolvimento
% +aberto desde o início pode trazer maior benefício.

As seções à seguir descrevem as atividades de cada etapa da metodologia.

\section{Planejamento do estudo}

\subsection{Seleção das métricas}

\citeonline{Meirelles2013} realizou um estudo onde associou-se
qualidade do produto de software à qualidade de código através de métricas de
código-fonte como indicador para o sucesso de projetos de software livre.

(definir quais métricas serão utilizadas e justificar a escolha)

Além dessa seleção baseada na sinergia e necessidades do projeto QualiPSo, selecionamos algu-
mas, entre as inúmeras métricas propostas pela literatura, tendo em vista medir os aspectos mais
relevantes à manutenibilidade do software, como sua complexidade interna, modularidade e grau de
dependência entre os módulos. Outro critério utilizado na escolha das métricas que detalharemos
nesta seção foi a existência de trabalhos que, para elas, sugerem valores teóricos de referência para
compararmos nossos estudos posteriores.

De acordo com nossas hipóteses, selecionamos as seguintes métricas calculadas pela Analizo:
número total de linhas de código (LOC), número total de módulos (NM), média do acoplamento
entre objetos (CBO) e média da falta de coesão entre métodos (LCOM4).

Também calculamos CBO e LCOM4 para obtermos o valor da complexidade estrutural
(SC) do projeto métrica apresentada na Seção 3.1.2.

\subsection{Seleção das fontes de ferramentas de análise estática}

Para ser possível validar as hipóteses aqui levantadas é necessário realizar
uma busca por ferramentas de análise estática desenvolvidas no contexto da
academia e da indústria, para isso, será feito um planejamento detalhado para
realizar a seleção de ferramentas em cada um destes contextos.

No contexto acadêmica a busca por ferramentas será feita
através de artigos publicados em conferências que tenham histórico de
publicação sobre ferramentas de análise estática de código fonte. Estes
artigos serão analisados e aqueles com publicação de ferramenta de análise
estática serão selecionados.

Na indústria a busca por ferramentas será feita a partir
de referências encontradas na internet, algumas organizações mantém listas de
ferramentas para análise de código-fonte, a Wikipedia também mantém uma lista
de ferramentas, estas referências serão utilizadas como ponto de partida e
cada ferramenta será analisada a fim de validar se são da indústria ou
surgiram em contexto acadêmico.

Uma vez que as ferramentas tenham sido selecionadas inicia-se a extração de
seus atributos de qualidade interna.

\subsection{Seleção da ferramenta de análise estática de código-fonte}

Para realizar a caracterização das ferramentas através dos seus atributos de
qualidade interna é necessário uma ferramenta capaz de analisar estaticamente
o código-fonte destas ferramentas e extrair atributos relacionados à sua
qualidade interna. Para isto utilizaremos o Analizo\cite{Terceiro2010}.

(falta justificar! quais vantagens? referencias?)

\section{Coleta de dados}

A partir das fontes selecionadas na etapa anterior serão realizadas duas
atividades para identificar e mapear as ferramentas de análise estática com
código-fonte disponível, uma atividade relacionada ao levantamento de
ferramentas da academia, outra atividade relacionada ao levantamento de
ferramentas da indústria.

\subsection{Ferramentas da academia}

A seleção de ferramentas será relizada através de uma revisão estruturada dos
artigos selecionados a partir das seguintes conferências:

\begin{itemize}
  \item ASE - Automated Software
    Engineering\footnote{http://ase-conferences.org}
  \item CSMR\footnote{A conferência CSMR tornou-se SANER - Software Analysis,
    Evolution, and Reengineering a partir da edição 2015.} - Conference on
    Software Maintenance and
    Reengineering\footnote{http://ansymore.uantwerpen.be/csmr-wcre}
  \item SCAM - Source Code Analysis and Manipulation Working
    Conference\footnote{http://www.ieee-scam.org}
  \item ICSME - International Conference on Software Maintenance and
    Evolution\footnote{http://www.icsme.org}
\end{itemize}

Chamamos de revisão estruturada um processo disciplinado para seleção de
artigos a partir de critérios bem definidos de forma que seja possível a
reprodução do estudo por parte de pesquisadores interessados. Alguns
resultados preliminares podem ser consultados na Tabela \ref{artigos-do-scam}
da Seção \ref{resultados}.

\subsection{Ferramentas da indústria}

A seleção de ferramentas da indústria será feita a partir de uma busca manual
em fontes encontradas na Internet sobre ferramentas de análise estática.

O projeto SAMATE\footnote{
http://samate.nist.gov} - {\em Software Assurance Metrics and Tool Evaluation}
disponível em \citeonline{SamateAnalysers} mantém uma lista de
ferramentas de análise estática mantida, mais sobre o projeto SAMATE pode
ser encontrado em \citeonline{Ribeiro2015}.

O software Spin mantém em seu site uma lista de ferramentas comerciais e de
pesquisa para análise estática de código-fonte para C em
\citeonline{spinSourceAnalysisTools}.

O Instituto de Engenharia de Software do CERT mantém uma lista de ferramentas
de análise estática em \citeonline{certSecureCodingTools}.

O software Flawfinder oferece em seu site um link com referências para
inúmeras ferramentas livres, proprietárias, gratuitas mas não-livres de
ferramentas de análise estática e outros tipos de análise em
\citeonline{wheelerStaticAnalysisTools}.

Uma outra fonte contendo uma relação expressiva de ferramentas é mantida na
Wikipedia em \citeonline{wikipediaListStaticCodeAnalysis}.

Estas fontes serão pesquisadas manualmente em busca de ferramentas de análise
estática que tenham sido desenvolvidas no contexto da indústria, algums
resultados preliminares podem ser encontrados nas Tabelas
\ref{ferramentas-do-nist-com-codigo} e \ref{ferramentas-do-nist-sem-codigo}.

\section{Caracterização dos artigos}

Caracterização dos papers analisados na revisão estruturada e caracterização
teórica do ecosistema das ferramentas da academia.

\section{Caracterização das ferramentas}

Será realizada uma caracterização prática das ferramentas, tanto acadêmica
quando da indústria, através da análise e extração de métricas de código-fonte
das mesmas.

% Resultado = Documentação com métricas de referência para ferramentas de análise estática.

\section{Exemplo de uso}

Por fim, os valores de métricas de referência encontradas serão utilizadas
como guia para refatorar a ferramenta Analizo.

(ler o TCC "Código Limpo e seu Mapeamento para Métricas de Código Fonte" onde
foi feito trabalho similar)

\chapter{Conclusão}

\section{Resultados preliminares}\label{resultados}

A Tabela \ref{artigos-do-scam} apresenta um resumo do número de artigos em
cada edição do SCAM e quantos artigos trazem publicação de ferramenta de análise
estática com código fonte disponível.

\begin{table}
\caption{Total de artigos analisados por edições do SCAM}
\centering
\begin{tabular}{| l | c | c |}
\hline
Edição    & Total de artigos & Artigos com ferramenta \\
\hline
SCAM 2001 & 23               & -                      \\
SCAM 2002 & 18               & -                      \\
SCAM 2003 & 21               & -                      \\
SCAM 2004 & 17               & -                      \\
SCAM 2005 & 19               & -                      \\
SCAM 2006 & 22               & 2                      \\
SCAM 2007 & 23               & 1                      \\
SCAM 2008 & 29               & -                      \\
SCAM 2009 & 20               & -                      \\
SCAM 2010 & 21               & 1                      \\
SCAM 2011 & 21               & 1                      \\
SCAM 2012 & 22               & 4                      \\
SCAM 2013 & 24               & -                      \\
SCAM 2014 & 35               & 1                      \\
SCAM 2015 & ?? (pendente)    & ?                      \\
\hline
Total     & 315              & 10                     \\
\hline
\end{tabular}
\label{artigos-do-scam}
\end{table}

As Tabelas \ref{ferramentas-do-nist-com-codigo} e
\ref{ferramentas-do-nist-sem-codigo} apresentam ferramentas do NIST após
avaliação inicial sobre disponibilidade do código-fonte. Das 54 ferramentas
apenas 19 tinham código fonte disponível.

% Todos foram baixados em "dataset/NIST".

\begin{table}
\caption{Lista de ferramentas do SAMATE - NIST com código fonte não disponível}
\centering
\begin{tabular}{| l | l |}
\hline
Ferramenta & Avaliacao  \\
\hline
ABASH                     & código não disponível \\
ApexSec Security Console  & código não disponível \\
Astrée                    & código não disponível \\
bugScout                  & código não disponível \\
C/C++test®                & código não disponível \\
dotTEST™                  & código não disponível \\
Jtest®                    & código não disponível \\
HP Code Advisor (cadvise) & código não disponível \\
Checkmarx CxSAST          & código não disponível \\
CodeCenter                & código não disponível \\
CodePeer                  & código não disponível \\
CodeSecure                & site offline \\
CodeSonar                 & código não disponível \\
Coverity SAVE™            & código não disponível \\
Csur                      & código não disponível \\
DoubleCheck               & código não disponível \\
Fluid                     & código não disponível \\
Goanna Studio and Goanna Central & código não disponível \\
HP QAInspect              & código não disponível \\
Insight                   & código não disponível \\
ObjectCenter              & código não disponível \\
Parfait                   & código não disponível \\
PLSQLScanner 2008         & código não disponível \\
PHP-Sat                   & link para código offline \\
PolySpace                 & código não disponível \\
PREfix and PREfast        & código não disponivel \\
QA-C, QA-C++, QA-J        & código não disponível \\
Qualitychecker            & código não disponível \\
Rational AppScan Source Edition & código não disponível \\
Resource Standard Metrics (RSM) & código não disponível \\
SCA                       & código não disponível \\
SPARK tool set            & código não disponível \\
TBmisra®, TBsecure®       & código não disponível \\
PVS-Studio                & código não disponível \\
xg++                      & código não disponível \\
\hline
\end{tabular}
\label{ferramentas-do-nist-sem-codigo}
\end{table}

\begin{table}
\caption{Lista de ferramentas do SAMATE - NIST com código fonte disponível}
\centering
\begin{tabular}{| l | l |}
\hline
Ferramenta & Avaliacao  \\
\hline
BOON                      & código disponível \\
Clang Static Analyzer     & código disponível \\
Closure Compiler          & código disponível \\
Cppcheck                  & código disponível \\
CQual                     & código disponível \\
FindBugs                  & código disponível \\
FindSecurityBugs          & código disponível \\
Flawfinder                & código disponível \\
Jlint                     & código disponível \\
LAPSE                     & código disponível \\
Pixy                      & código disponível \\
PMD                       & código disponível \\
pylint                    & codigo disponivel \\
RATS (Rough Auditing Tool for Security) & código disponível \\
Smatch                    & código disponível \\
Splint                    & código disponível \\
UNO                       & código disponível \\
Yasca                     & código disponível \\
WAP                       & código disponível \\
\hline
\end{tabular}
\label{ferramentas-do-nist-com-codigo}
\end{table}

Assim, temos um total de 19 ferramentas da indústria com código-fonte
disponível e ??? da academia com código fonte disponível, é preciso avaliar em
qual linguagem de programação foi escrita cada ferramentas pois só iremos
analisar aquelas em C, C++ ou Java que são suportadas pelo Analizo.

A ferramenta utilizada para identificar a linguagem de programação em que
estes softwares foram escritos foi a sloccount, uma ferramenta livre para
contagem de linhas de código fonte, onde se calcula em quais linguagens de
programação um software foi escrito.

%Abaixo destaco a linguagem de programação que tem maior
%porção em porcentagem.

Após analise ficamos com um total de 25 ferramentas, 15 da indústria e 10 da
academia, a Tabela \ref{total-de-ferramentas} traz um resumo de todos as
ferramentas analisadas.

\begin{table}
\caption{Lista com total de ferramentas a serem analisadas}
\centering
\begin{tabular}{| l | c | c |}
\hline
Ferramenta & Linguagem & Fonte \\
\hline
BOON                  & ansic                & industria \\
CQual                 & ansic                & industria \\
RATS                  & ansic                & industria \\
Smatch                & ansic                & industria \\
Splint                & ansic                & industria \\
UNO                   & ansic                & industria \\
Clang Static Analyzer & cpp                  & industria \\
Cppcheck              & cpp                  & industria \\
Jlint                 & cpp                  & industria \\
WAP                   & java                 & industria \\
Closure Compiler      & java                 & industria \\
FindBugs              & java                 & industria \\
FindSecurityBugs      & java                 & industria \\
Pixy                  & java                 & industria \\
PMD                   & java                 & industria \\
Indus                 & java                 & academia  \\
TACLE                 & java                 & academia  \\
JastAdd               & java                 & academia  \\
WALA                  & java                 & academia  \\
error-prone           & java                 & academia  \\
AccessAnalysis        & java                 & academia  \\
Bakar Alir            & java/ada/python      & academia  \\
InputTracer           & ansic                & academia  \\
srcML                 & cpp/cs (cs = C\# ?)   & academia  \\
Source Meter          & java                 & academia  \\
\hline
\end{tabular}
\label{total-de-ferramentas}
\end{table}

\section{Cronograma}

(à fazer)

\backmatter
\appendix
\bibliography{bibliografia}
\end{document}
