\documentclass[12pt]{article}
\usepackage[utf8]{inputenc}
\usepackage[brazil]{babel}
\usepackage{fancyvrb}
\usepackage[alf]{abntex2cite}
\usepackage[top=2cm,left=0.5in,right=0.5in,bottom=2cm]{geometry}
\usepackage{graphicx}
\DeclareGraphicsExtensions{.pdf}

\title{
  Caracterização da qualidade interna de ferrramentas de análise estática de
  código fonte
}
\author{Joenio Marques da Costa\\
  {\small Universidade Federal da Bahia (UFBA)} \\
  {\small joenio@colivre.coop.br}
}
\date{\today}

\begin{document}

\maketitle

\section{Introdução}

(à fazer)

\section{Fundamentação teórica}

(à fazer)

\section{Metodologia}

Neste capítulo será apresentada a metodologia utilizada no estudo como meio
de validar as seguintes hipóteses:

\begin{enumerate}
  \item[{\bf H1:}] Existem publicações sobre ferramentas de análise estática com
    disponibilidade de código-fonte
  \item[{\bf H2:}] Existem ferramentas de análise estática disponíveis livremente na
    indústria com disponibilidade de código-fonte
  \item[{\bf H3:}] Existem valores de referência para métricas de código-fonte para
    ferramentas de análise estática
\end{enumerate}

As seções à seguir descrevem as atividades de cada etapa da metodologia.

\subsection{Planejamento do estudo}

\begin{enumerate}
  \item Seleção das métricas
  \item Seleção das fontes de ferramentas de análise estática
  \item Seleção da ferramenta de análise estática de código-fonte
\end{enumerate}

\subsection{Coleta de dados}

A partir das fontes selecionadas ne etapa anterior serão realizadas duas
atividades para identificar e mapear ferramentas de análise estática com
código-fonte disponível.

\subsubsection{Identificar ferramentas da academia}

Seleção de ferramentas da academia através de revisão estruturada.

\subsubsection{Identificar ferramentas da indústria}

Seleção de ferramentas da indústria (através de ?).

\subsection{Caracterização dos artigos}

Caracterização dos papers analisados na revisão estruturada e caracterização
teórica do ecosistema das ferramentas da academia.

\subsection{Caracterização das ferramentas}

Será realizada uma caracterização prática das ferramentas, tanto acadêmica
quando da indústria, através da análise e extração de métricas de código-fonte
das mesmas.

% Resultado = Documentação com métricas de referência para ferramentas de análise estática.

\subsection{Exemplo de uso}

Por fim, os valores de métricas de referência encontradas serão utilizadas
como guia para refatorar a ferramenta Analizo.

\section{Contribuições esperadas}

(à fazer)

\section{Cronograma}

(à fazer)

%\bibliography{bibliografia}

\end{document}
