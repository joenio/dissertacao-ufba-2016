\chapter*{Glossário}

\begin{acronym}[Sustentabilidade de software]
  \acro{analise-estatica-de-software}[Análise estática de software]{
    É o processo de coleta de informações de um software a respeito dos seus
    componentes e sua organização interna sem necessidade de execução do código
    fonte.
  }
  \acro{disponibilidade}[Disponibilidade]{
    Capacidade de um softwares estar acessível publicamente para obtenção no
    presente.
  }
  \acro{manutenabilidade}[Manutenabilidade]{
    Característica de qualidade de software que indica o quão fácil é
    realizar atividades de evolução e manutenção em seu código fonte.
  }
  \acro{reprodutibilidade}[Reprodutibilidade]{
    Capacidade de reprodução de um dado estudo científico, ou seja, recriar
    o espírito do que outra pessoa fez, usando seus próprios artefatos.
  }
  \acro{software-academico}[Software acadêmico]{
    Softwares desenvolvidos durante pesquisas científicas, sejam pequenos
    scripts, protótipos ou produtos maduros, que tenham sido utilizados durante
    o estudo para coleta, extração ou análise dos dados (podem ser encontrados
    com o nome de {\it research-originated software} ou {\it research tool}).
  }
  \acro{sustentabilidade-de-software}[Sustentabilidade]{
    Conceito guarda chuva composto de múltiplas dimensões, social, individual,
    ambiental, econômica e técnica, que traz o tema sustentabilidade para o campo da
    ciência da computação exibindo preocupação com os impactos que os sistemas
    e a tecnologia da informação causam no futuro do planeta.
  }
  \acro{sustentabilidade-tecnica}[Sustentabilidade técnica]{
    Dimensão da sustentabilidade preocupada com a longevidade da informação,
    dos sistemas, e infraestrutura, e sua adequada evolução frente as condições
    do ambiente em constante mudança.
  }
\end{acronym}
