\xchapter{Introdução}{}

O domínio de aplicação de análise estática tem evoluído constantemente desde o
seu surgimento a mais de 40 anos, e apesar da sua evolução constante poucos
estudos demonstram a preocupação em avaliar ou comparar ferramentas de análise
estática de código-fonte em relação a seus atributos de qualidade interna, um
aspecto importante para guiar e auxiliar desenvolvedores em atividades de
evolução e manutenção de software.

Desta forma, estudar e comparar tais ferramentas a fim de compreender seus
atributos de qualidade interna e entender quais características arquiteturais
explicam tais atributos é de fundamental importância do ponto de vista de
desenvolvedores interessados em melhorar a qualidade interna de suas
ferramentas.

\section{Objetivos}

Diante disso, definimos como objetivo principal deste trabalho: compreender
as ferramentas de software para análise estática de código-fonte do ponto de
vista de sua manutenabilidade, a partir da análise de sua complexidade
estrutural, discutindo quais características arquiteturais explicam seus
atributos de qualidade interna.

\subsection{Questão de pesquisa}

Com este objetivo em mente, levantamos a seguinte questão de pesquisa:

\begin{enumerate}
  \item [{\bf Q1:}] {\em Como a complexidade estrutural pode ser interpretada
    e explicada para ferramentas de software do domínio de aplicação de
    análise estática de código-fonte?}
\end{enumerate}

\subsubsection{Hipóteses}

Para responder a questão colocada acima, definimos as seguintes hipóteses:

\begin{enumerate}
  \item[{\bf H1:}] {\em É possível calcular valores de referência de métricas
    de código-fonte para ferramentas de análise estática a partir de um
    conjunto de softwares da academia e da indústria.}
  \item[{\bf H2:}] {\em Ferramentas de análise estática tendem a ter uma
    maior complexidade estrutural do que ferramentas de outros domínios de
    aplicação.}
  \item[{\bf H3:}] {\em Dentre as ferramentas de análise estática de
    código-fonte, aquelas desenvolvidas na indústria apresentam uma menor
    complexidade estrutural.}
\end{enumerate}

A Seção \ref{hipoteses} traz detalhes sobre como cada hipótese será testada.

\subsection{Objetivos específicos}

São objetivos específicos deste trabalho:

\begin{itemize}
  \item Selecionar e obter código-fonte de ferramentas de análise estática
    desenvolvidas na academia, para coletar suas métricas de código-fonte.  A
    seleção terá como base o resultado de uma revisão estruturada realizada a
    partir de artigos publicados em conferências relacionadas. 
  \item Selecionar e obter código-fonte de ferramentas de análise estática
    desenvolvidas pela indústria, para coletar suas métricas de código-fonte.
  \item Propor intervalos de referência para a observação parametrizada da
    qualidade interna das ferramentas de análise estática, a partir de suas
    métricas de código-fonte.
  \item Calcular a distância Euclidiana entre valores de referência e os
    valores das ferramentas estudadas.
\end{itemize}

\section{Contribuições esperadas}

Ao final deste trabalho, as seguintes contribuições científicas ({\bf CC}) e
tecnológicas ({\bf CT}) são esperadas:

\begin{enumerate}
  \item [{\bf CC1:}] Um conjunto de intervalos de referência da frequência dos
    valores de métricas de código-fonte para o domínio de aplicação de
    análise estática de código-fonte.
  \item [{\bf CC2:}] Definição de argumentos que expliquem a alta complexidade
    estrutural em ferramentas de análise estática de código-fonte.
  \item [{\bf CT1:}] Evolução de uma ferramenta de análise estática de
    código-fonte.
\end{enumerate}

Lembrando que neste trabalho serão utilizadas métricas de produto,
especificamente, métricas de código-fonte, que cobrem aspectos de tamanho,
complexidade e qualidade.

\section{Estrutura do texto} 

O capítulo \ref{fundamentacao} apresenta conceitos sobre análise estática e
métricas de código-fonte necessários para compreensão do trabalho. O capítulo
\ref{metodologia} apresenta trabalhos relacionados, hipóteses do estudo,
planejamento sobre a coleta e análise dos dados
e o cronograma do estudo. O capítulo \ref{conclusoes} traz a
discussão e interpretação dos dados e apresenta os próximos passos do estudo.
