\xchapter{Introdução}
{À medida que o software se torna uma tecnologia generalizada em praticamente
todos os aspectos da condição humana, também é inserido firmemente no meio
acadêmico, software analisa dados, simula o mundo real, e visualiza
resultados.}

A Ciência moderna depende de software. Projetos de software resolvem
problemas comuns de metade dos cientistas de todas as áreas da Ciência durante
atividades de pesquisa \cite{wilson2014best}. Estes projetos, definidos neste
estudo como software acadêmico -- também referenciado na literatura como {\it
research-originated software} \cite{kon2011free}, {\it research tool}
\cite{portillo2012tools} ou {\it academic software} \cite{allen2017engineering}
-- quando desenvolvidos por cientistas carregam boa parte do conhecimento
empregado em suas pesquisas.

Novos projetos de software são constantemente criados e desenvolvidos durante
pesquisas científicas, definido neste estudo como software acadêmico

Em pesquisas da Ciência da Computação, é prática comum que
algum tipo de software seja desenvolvido para apoiar a pesquisa em andamento ou
mesmo como principal resultado da pesquisa. Estes projetos 

Falar MAIS sobre o contexto   ...
Por exemplo, um pesquisador pode desenvolver ...  ... exemplo,  ... Outros
pesquisadores podem usar tal software em suas pesquisas ... outros podem usá-lo
para comparar resultados com seus próprios resultados, etc ...

em ciência da computação,
particularmente em engenharia de software, tem-se notado um aumento constante
no número de novos softwares acadêmicos \cite{allen2017engineering}.

A comunidade tem refletido sobre os problemas relacionados ao
desenvolvimento, promoção e sustentabilidade desses softwares, e o
impacto que tais problemas causam no meio científico \cite{allen2017engineering}. Esta
reflexão tem mostrado, por exemplo, que muitos estudos em engenharia de
software sofrem de dificuldades de repetição \cite{tang2016worthiness}, e apontam
problemas específicos relacionados à manutenibilidade e a sustentabilidade
técnica dos softwares acadêmicos.

%Manutenabilidade é uma característica de qualidade que indica o quão fácil é
%realizar atividades de evolução e manutenção em softwares, um aspecto
%importante aos pesquisadores interessados em adaptar softwares acadêmicos, algo
%muitas vezes necessário ao reproduzir pesquisas anteriores \cite{peng2011reproducible}.

Sustentabilidade técnica diz respeito a longevidade dos softwares, ou seja, a
capacidade de continuar disponível no futuro. Muitos pesquisadores não
disponibilizam os seus softwares \cite{robles2010replicating,
amann2015software} ou quanto o fazem enfrentam problemas com disponibilidade e
manutenibilidade \cite{prlic2012ten}, isto leva a um corpo computacional
extramente difícil de reproduzir uma vez que mais da metade dos pesquisadores
desenvolvem seus próprios softwares \cite{hettrick2014uk}, além de ferir um dos
fundamentos da ciência de que novas descobertas sejam reproduzidas antes de
serem consideradas parte da base de conhecimento \cite{stodden2009enabling}.

Isto tem motivado a organização de conferências específicas sobre o tema, como
o RSE\footnote{Conference of Research Software Engineers
\url{http://rse.ac.uk/conf2017}}, WSSSPE\footnote{Workshop on Sustainable
Software for Science: Practice and Experiences
\url{http://wssspe.researchcomputing.org.uk}} e o RESER\footnote{Workshop on
Replication in Empirical Software Engineering Research
\url{http://sequoia.cs.byu.edu/reser}}, e tem contribuido para a compreensão
dos problemas relacionados aos softwares acadêmicos, abordando questões sobre
desenvolvimento, qualidade e sustentabiliade, sobre como citar softwares em
novas pesquisas, como promover e reconhecer o papel do pesquisador engenheiro
desenvolvedor de softwares acadêmicos, além de questões sobre infraestrutura,
ferramentas e práticas para o desenvolvimento de softwares acadêmicos de
forma sustentável.

Análise estática de software, uma área com uma longa e respeitável
tradição e que ainda sofre carência de estudos sobre avaliação e validação de
seus softwares \cite{li2010comparative, ilyas2016static}.

publicações usando softwares como método, mostrou que apenas 59 mencionavam o
uso de softwares de alguma forma, os demais 31 artigos, apesar de usar software
acadêmico, não mencionavam nada a respeito \cite{howison2016software}.

Isto gera um impacto negativo na visibilidade dos softwares acadêmicos e faz
surgir questionamentos sobre a sua qualidade, não apenas técnica, mas também a
capacidade de ser encontrado, compartilhado e co-desenvolvido, qualidades
importantes para a evolução do próprio software, mas também extremamente útil
para um uso eficiente dos limitados recursos da ciência \cite{howison2013incentives,
katz2014transitive}.

No entanto, parece ser regra geral não testar ou não documentar o próprio
software, pesquisadores geralmente não testam ou documentam seus softwares
acadêmicos, a maioria também não sabe o quão confiável seu software é,
ocasionando graves erros em conclusões centrais da literatura,
gerando retrabalho nas mais diversas áreas da ciência \cite{merali2010computational},
apesar de nem sempre ser possível, ou viável, ter tudo dentro de
padrões estritos, é preciso estar consciente das boas práticas ao
produzir e utilizar softwares acadêmicos, tanto para melhorar a própria
abordagem quanto para revisar outros trabalhos \cite{wilson2014best}.

A maior parte dos cientistas (90\%) no entretanto nunca tiveram treinamento
algum de como escrever software de forma eficiente, faltam práticas básicas de
desenvolvimento, como escrever código legível, revisão de código, controle de
versão, testes unitários, entre outros, como resultado, dados são perdidos,
análises levam mais tempo que o necessário e os pesquisadores não conseguem a
eficiência que poderiam ter ao trabalhar com softwares acadêmicos
\cite{wilson2017good}.

Isto contradiz as boas práticas de qualquer projeto experimental ({\it
laboratory
notebooks}\footnote{\url{https://en.wikipedia.org/wiki/Lab_notebook}}, dados
organizados, passos documentados, projeto estruturado para reprodutibilidade) e
torna praticamente impossível utilizar o método mais comum e cientificamente
produtivo de produzir conhecimento novo a partir de pesquisas anteriores, a
replicação, ou seja, seguir os mesmos passos do autor original com objetivo de
validar, melhorar ou estender seus dados e sua metodologia
\cite{king1995replication, stodden2010reproducible}.

Somado a isto temos ainda o fato de que pesquisadores raramente publicam seus
códigos \cite{robles2010replicating, amann2015software}, piorando ainda mais toda a situação, isto tem motivado a organização
de conferências específicas para discutir os problemas dos softwares
acadêmicos, como o RSE (Conference of Research Software Engineers)\footnote{
\url{http://rse.ac.uk/conf2017}}, WSSSPE (Workshop on Sustainable Software for
Science: Practice and Experiences)\footnote{
\url{http://wssspe.researchcomputing.org.uk}} e o RESER (Workshop on
Replication in Empirical Software Engineering Research)\footnote{
\url{http://sequoia.cs.byu.edu/reser}}.

%Independente da finalidade, tamanho ou motivação, todo software tem o potencial
%de voltar a ser útil em outros momentos ou lugares, para o autor original,
%ou para pesquisadores enfrentando problemas semelhantes aos dos autores originais.
%Não é difícil imaginar que o problema enfrentado por um pesquisador pode, em
%algum momento, ser enfrentado por outros pesquisadores, criando assim uma ótima
%oportunidade de colaboração entre pesquisas e pesquisadores, sendo o software
%um excelente vetor de ajuda mútua em ambas as direções.

% Software acadêmico sofre de {\it ``dysfunctional chaotic churn''}.
% O ecossistema de software acadêmico sofre de um fenômeno chamado de
% desordem caótica disfuncional ({\it ``dysfunctional chaotic churn''}).
%
% 1. Abnormal or impaired functioning of a bodily system or organ. 
% 2. Failure to achieve or sustain a behavioral norm or expected condition, as in a social relationship. dys·func′tion·al adj.

Apesar disso, cientistas têm percebido que os softwares desenvolvidos na academia 
sofrem de desordem caótica disfuncional ({\it ``dysfunctional chaotic churn''}), 
ou seja, a existência
de muitos projetos, com poucos usuários, com ciclos de vida curtos, que
terminam em paralelo ao financiamento inicial, comunidades desconectadas e
paralelas, incompatibilidades entre projetos, e tentativas aparentemente não
coordenadas de ``reiniciar'' tudo ({\it re-boots}).

%Este cenário, além de desacelerar o progresso geral da ciência gerando
%retrabalho, faz surgir questionamentos sobre as conclusões dessas pesquisas,
%especialmente quando grande parte dos pesquisadores não sabem o quão confiável
%seus softwares são.

Este problema, apesar de ser percebido por muitos cientistas, carece de
evidências. % \cite{chitaozinho & xororó}  ;-)  
Neste trabalho, investigamos como o problema ...  se manifesta em pesquisas da
engenharia de software, em especial, em publicações de análise estática, uma
área com uma longa e respeitável tradição em pesquisas sobre a criação de novas
ferramentas, métodos e algoritmos.

% papel pesquisador no ecossistema de soft academico
%
%Essas preocupações gerais sugerem um conjunto de questões específicas, com foco
%em padrões globais e padrões emergentes dentro do ecossistema, incluindo: Quais
%recursos foram destinados à produção de software? Quantos usuários ou
%comunidades de usuários têm projetos? Quais são os impactos científicos desse
%uso? Os números de usuários crescem? Os projetos possuem recursos e habilidades
%suficientes para gerenciar seu crescimento? Quais projetos possuem
%funcionalidades sobrepostas? Há quanto tempo os pedaços de software e projetos
%persistem? Nós desconectamos as comunidades de usuários e desenvolvedores? São
%componentes específicos, ou camadas de componentes, faltam? Que código
%geralmente é usado em conjunto; são os projetos e as pessoas que produzem esses
%componentes se comunicando adequadamente? Como podemos sustentar o software
%crítico?
%
%Aqui há uma clara tensão entre um desejo de flexibilidade e liberdade, ligado
%às expectativas de inovação científica e desejos de estruturas de autoridade e
%controle de coordenação. As questões de influência incluem: como os programas
%de financiamento e quais os requisitos em suas chamadas, resultaram em software
%amplamente utilizado e impacto científico substancial? Quais são as
%características dos campos que alcançaram maior coalescência? Quais jornais e
%conferências têm políticas exemplares? Como o trabalho de software é visto
%dentro das práticas de contratação e avaliação, como os casos de posse?
%
%\cite{howison2015understanding}


%e à medida
%que percebe-se que os softwares estão se tornando parte integrante dos
%processos, ferramentas e produção científicas, torna-se necessário e urgente
%discutir o seu desenvolvimento, visibilidade, qualidade e sustentabilidade.

\section{Escopo}

O que sabemos sobre a sustentabilidade técnica de projetos de software
acadêmico de análise estática publicados em conferências de Engenharia de
Software? Como estes projetos são mencionados na literatura acadêmica? Há
colaboração entre os cientistas no desenvolvimento destes projetos?

\subsection{Definição do Objetivo}

\begin{description}
  \item{\bf Objeto de estudo.}
    O objeto de estudo são projetos de software acadêmico de análise estática publicados
    em artigos científicos e sua publicização, menções e ciclo de vida.
  \item{\bf Propósito.}
    O propósito deste estudo é caracterizar a sustentabilidade técnica de cada
    software acadêmico de análise estática, trazendo informações que permitam
    compreender o problema de desordem disfuncional caótica no domínio de
    análise estática.
  \item{\bf Perspectiva.}
    A perspectiva considerada é a de cientistas e pesquisadores, isto é, o
    cientista ou pesquisador gostaria de conhecer ecossistemas de software
    acadêmico de análise estática em termos de sua sustentabilidade técnica.
    Além disso, pessoas da indústria podem estar interessadas em conhecer
    software acadêmico de análise estática para financiá-lo.
  \item{\bf Foco de qualidade.}
    O principal aspecto de qualidade estudado é a sustentabilidade técnica, com
    destaque para três aspectos: publicização, menções e estágio de evolução no
    ciclo de vida de cada projeto de software acadêmico de análise estática.
  \item{\bf Contexto.}
    O estudo foi conduzido com projetos de software acadêmico de análise
    estática publicados nas conferências de Engenharia de Software ASE e SCAM.
\end{description}

\subsection{Sumário da Definição}

Analisar os \textit{projetos de software acadêmico de análise estática e sua sustentabilidade técnica} %object of study
com o propósito de \textit{caracterizar} %purpose
com respeito a \textit{publicização, menções em artigos acadêmicos e estágio de evolução no ciclo de vida} %quality focus
na perspectiva do \textit{pesquisador} e do \textit{cientista} %perspective
no contexto das \textit{conferências de Engenharia de Software ASE e SCAM, e publicações nas bases ACM e IEEE}. %context

%\subsubsection{Objetivo Geral}
%
%Caracterizar o software acadêmico de análise estática com respeito à sua
%sustentatibilidade técnica.
%A caracterização será feita em um conjunto de software acadêmico de análise
%estática, com base em medidas para avaliar sua sustentabilidade técnica.

<<<<<<< HEAD
\begin{itemize}
  \item Há projetos similares, como poucos usuários e ciclos de vida curto?
  \item Os projetos sustentáveis tem contribuidores além dos autores iniciais?
  \item Os projetos sustentáveis tem mais contribuidores?
  \item Os artigos de projetos sustentáveis são mais lidos, mais citados?
  \item Os autores de projetos sustentáveis tem mais publicações com o uso de software do que os autores?
  \item Os projetos sustentáveis são mais fáceis de manter? de usar? 
\end{itemize}

Estas questões darão importantes indícios sobre a qualidade do software acadêmico de
análise estática desenvolvido na academia, especialmente sobre a capacidade de serem 
encontrados, compartilhados e co-desenvolvidos.

\section{Questão de Pesquisa}

Quão sustentável é o software acadêmico de análise estática?

Qual a relação entre sustentabilidade e reproducibilidade?

Qual o nível de DCD do domínio de análise estática?

\section{Metodologia de trabalho}

(ver arquivo capitulos/metodologia.tex)

\subsection{Objetivos}

O objetivo geral deste trabalho é caracterizar a relação entre sustentabilidade
e reproducibilidade no contexto de software acadêmico e, mais especificamente,
software de análise estática. 

=======
>>>>>>> 0d2599399bf175468e9ee7218a7507020b7aecf7
São objetivos específicos deste trabalho:

% Objetivo específico: caracterizar o tipo de citação / menção d
% Reflexão sobre os problemas de sustentabilidade sofridos pelo ecossistema de software acadêmico de análise estática
% Formulação de hipóteses sobre os problemas de sustentabilidade sofridos pelo ecossistema de software acadêmico de análise estática

\begin{description}
  \item [01]
    Caracterizar o software acadêmico de análise estática com respeito à sua
    publicização.
    A caracterização será feita em um conjunto de software acadêmico de análise
    estática, com base em medidas para avaliar a sua disponibilidade de
    download, código fonte, e presençã oficial online.
  \item [02]
    Caracterizar o software acadêmico de análise estática com respeito aos tipos
    de menções em artigos acadêmicos.
    A caracterização será feita em um conjunto de software acadêmico de análise
    estática, com base em uma análise de trabalhos científicos que o utiliza ou
    adapta.
  \item [03]
    Caracterizar o software acadêmico de análise estática com respeito ao
    estágio de evolução.
    A caracterização será feita em um conjunto de software acadêmico de análise
    estática, com base nas informaçoes sobre lançamentos e métricas de código
    fonte.
\end{description}

Estas questões darão importantes indícios sobre a qualidade do software acadêmico de
análise estática desenvolvido na academia, especialmente sobre a capacidade de serem 
encontrados, compartilhados e co-desenvolvidos.

\subsection{Questões de Pesquisa}

Neste estudo as seguintes questões de pesquisa, a respeito do ecossistema de
software acadêmico de análise estática, serão investigadas:

% Quão sustentável é o software acadêmico de análise estática?
% Os artigos de projetos sustentáveis são mais lidos, mais citados?
% Os autores de projetos sustentáveis tem mais publicações com o uso de software do que os autores?
\newcommand{\QuestaoUm}{Como as taxas de publicação contribuindo com novos projetos de software acadêmico nas conferências ASE e SCAM mudam ao longo do tempo?}
\newcommand{\QuestaoDois}{Os projetos estão disponíveis para obtenção hoje? Os projetos incentivam ativamente a contribuição? Os espaços do projeto são abertos e transparentes?}
\newcommand{\QuestaoTres}{Como a visibilidade dos artefatos de software publicados nas conferências ASE e SCAM mudam ao longo do tempo? Como os artefatos de software acadêmico publicados nas conferências ASE e SCAM são mencionados na literatura acadêmica ao longo do tempo?}

%* Como ocorre o co-desenvolvimento dos softwares
%* Como acontece colaboração na construção dos softwares
%* Como os softwares contribuem para a construcao de conhecimento novo em novas pesquisas derivadas
% * mais da metade desenvolvem seus próprios softwares
% * falta de visibilidade gera questionamentos sobre qualidade
% * falta de treinamento leva a produzir softwares sem qualidade
% * produtividade científica requer capacidade de replicação
% * capacidade de replicação depende de qualidade

\begin{description}
  \item [Q1:] \QuestaoUm
  \item [Q2:] \QuestaoDois
  \item [Q3:] \QuestaoTres
\end{description}

\subsection{Métricas}

Para responder às questões de pesquisas, as seguintes métricas serão usadas:

\begin{enumerate}
  \item Métricas relacionadas a publicização do software (número de projetos
  disponível para download, disponível em código fonte, tipo de licença);
  \item Métricas de publicação (número de citações, número de menções, número
  de usos);
  \item Métricas relacionadas ao ciclo de vida do software (número total de
  lançamentos, data e número de versão de cada lançamento);
  \item Métricas ao tamanho do software (número de módulos do código fonte).
\end{enumerate}

\section{Estratégia de Pesquisa}

Este trabalho apresenta um estudo de caso exploratório ({\it exploratory case
study}) \cite{stol2015holistic} sobre a sustentabilidade do ecossistema de
software acadêmico de análise estática.

Adotamos uma estratégia de pesquisa de trabalho de campo ({\it Field Studies}),
segundo o framework apresentado em \citeonline{stol2015holistic}, de
configuração natural ({\it Natural Settings}), com as seguintes características
principais:

\begin{itemize}
  \item Com o foco num fenômeno, organização ou sistema em particupar;
  \item Com um baixo nível de generalização e alto realismo do contexto; e
  \item Sem intervenção do pesquisador no ambiente.
\end{itemize}

A ``estratégia de pesquisa'' tem um impacto significativo sobre o que pode e
não pode ser alcançado em um estudo em termos de aquisição de novos
conhecimentos e uma compreensão mais profunda dos fenômenos investigados
\cite{stol2015holistic}.

\section{Organização do texto}

O capítulo \ref{fundamentacao} apresenta os fundamentos teóricos necessários
para a compreensão deste trabalho.

O capítulo \ref{estudo1} traz um estudo sobre a publicização de software
acadêmico de análise estática nas conferências de Engenharia de Software ASE e
SCAM.

O capítulo \ref{estudo2} descreve um estudo sobre o reconhecimento de software
acadêmico de análise estática em publicações nas bases ACM e IEEE.

O capítulo \ref{estudo3} apresenta um estudo sobre o estágio de evolução e o
ciclo de vida de software acadêmico de análise estática.

O capítulo \ref{discussao} discute os resultados em termos da sustentabilidade
técnica do ecossistema de software acadêmico de análise estática e traça
paralelos com trabalhos relacionados.

O capítulo \ref{conclusoes} apresenta as considerações finais da pesquisa e
aponta trabalhos futuros.
