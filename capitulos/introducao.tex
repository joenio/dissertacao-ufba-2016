\xchapter{Introdução}{}

%\section{Apresentação}

% Em diversas linhas de pesquisa da computação, em especial, a engenharia de software, é muito comum que novos softwares (não creio ser recomendável colocar software no plural, pois é um termo que vem do inglês e representa coletivo) sejam desenvolvidos durante trabalhos de pesquisa.
%Esses tem sido chamados na literatura de {\it research-originated software} \cite{Kon2011}, {\it research tool} \cite{Portillo12} ou {\it academic software} \cite{allen2017engineering}, e vêm ganhando atenção da comunidade devido ao papel que ocupam na reprodutibilidade de seus estudos \cite{Peng2011}.

Em diversas linhas de pesquisa da Ciência da Computação, é prática comum que algum tipo de software seja desenvolvido para apoiar a pesquisa em andamento ou mesmo como principal resultado da pesquisa.
Falar MAIS sobre o contexto   ... Por exemplo, um pesquisador pode desenvolver ...
... exemplo,  ... Outros pesquisadores podem usar tal software em suas pesquisas ... outros podem usá-lo para comparar resultados com seus próprios resultados, etc ...

Software acadêmico pode ser definido como o software originado e associado ao desenvolvimento da pesquisa científica -- também referenciado na literatura como {\it research-originated software}~\cite{Kon2011}, {\it research tool}~\cite{Portillo12} ou {\it academic software}~\cite{allen2017engineering}. 

A comunidade tem refletido sobre os problemas relacionados ao
desenvolvimento, promoção e sustentabilidade desses softwares, e o
impacto que tais problemas causam no meio científico \cite{allen2017engineering}. Esta
reflexão tem mostrado, por exemplo, que muitos estudos em engenharia de
software sofrem de dificuldades de repetição \cite{Tang2016}, e apontam
problemas específicos relacionados à manutenabilidade e a sustentabilidade
técnica dos softwares acadêmicos.

Manutenabilidade é uma característica de qualidade que indica o quão fácil é
realizar atividades de evolução e manutenção em softwares, um aspecto
importante aos pesquisadores interessados em adaptar softwares acadêmicos, algo
muitas vezes necessário ao reproduzir pesquisas anteriores \cite{Peng2011}.
Sustentabilidade técnica diz respeito a longevidade dos softwares, ou seja, a
capacidade de continuar disponível no futuro. Muitos pesquisadores não
disponibilizam os seus softwares \cite{robles2010replicating,
amann2015software} ou quanto o fazem enfrentam problemas com disponibilidade e
manutenabilidade \cite{Prlic2012}, isto leva a um corpo computacional
extramente difícil de reproduzir uma vez que mais da metade dos pesquisadores
desenvolvem seus próprios softwares \cite{hettrick_2014_14809}, além de ferir um dos
fundamentos da ciência de que novas descobertas sejam reproduzidas antes de
serem consideradas parte da base de conhecimento \cite{Stodden2009}.

Isto tem motivado a organização de conferências específicas sobre o tema, como
o RSE\footnote{Conference of Research Software Engineers
\url{http://rse.ac.uk/conf2017}}, WSSSPE\footnote{Workshop on Sustainable
Software for Science: Practice and Experiences
\url{http://wssspe.researchcomputing.org.uk}} e o RESER\footnote{Workshop on
Replication in Empirical Software Engineering Research
\url{http://sequoia.cs.byu.edu/reser}}, e tem contribuido para a compreensão
dos problemas relacionados aos softwares acadêmicos, abordando questões sobre
desenvolvimento, qualidade e sustentabiliade, sobre como citar softwares em
novas pesquisas, como promover e reconhecer o papel do pesquisador engenheiro
desenvolvedor de softwares acadêmicos, além de questões sobre infraestrutura,
ferramentas e práticas para o desenvolvimento de softwares acadêmicos de
forma sustentável.

Mas apesar desta crescente preocupação com os softwares acadêmicos ainda
sabe-se pouco sobre o quanto a sustentabilidade técnica e a manutenabilidade
impactam na reprodutibilidade de seus estudos, sobretudo em áreas específicas,
como a análise estática de software, uma área com uma longa e respeitável
tradição e que ainda sofre carência de estudos sobre avaliação e validação de
seus softwares \cite{Li2010, ilyas2016static}.

% (2) Tools to support systematic literature reviews in software engineering: A mapping study \cite{marshall2013tools}
%
% Cita um mapeamento feito sobre estudos que criam ferramentas para apoio a
% revisão sistemática no domínio de SE, 14 estudos foram selecionados, ao final
% apenas 8 tinham proposta de ferramentas, ao final conclui que as ferramentas
% encontradas estão em estado inicial de desenvolvimento. 
%
% (3) Tools used in Global Software Engineering: A systematic mapping review \cite{Portillo12}
%
% Cita um mapeamento sistemático com objetivo de encontrar ferramentas de
% comunicação e coordenação para suporte a times altamente distribuidos
% gograficamente, encontrou 132 ferramentas, para uso em projetos de software
% global. A maioria destas ferramentas foram desenvolvidas em centros de
% pesquisas, e apenas uma pequena porcentagem (18.9\%) foram testados fora do
% seu contexto onde foi desenvolvido.
%
% (5) Tools in mining software repositories \cite{chaturvedi2013tools}
%
% Faz uma revisão dos papers submetidos ao MSR desde 2007 até 2013 (?) e
% identifica data sets, ferramentas e técnicas utilizadas pelos autores, mais
% da metade dos papers usam ou criam ferramentas, categoriza as ferramentas em
% ferramentas novas, ferramentas tradicionais, protótipos e scripts para
% mineração de dados
%
% (6) A systematic literature review of software product line management tools \cite{pereira2015systematic}
%
% (???)
%
% (7) Software configuration management tools \cite{chan1997software}
%
% (???)
%
% (8) Comparison and evaluation of source code mining tools and techniques: A qualitative approach \cite{khatoon2013comparison}
%
% Lista ferramentas e técnicas para mineração de dados, estado da arte.
%
% (9) An overview of free software tools for general data mining \cite{jovic2014overview}
%
% Descreve característica dos 6 softwares livres mais usados para mineração de
% dados no geral.
%
% (10) Analyzing the State of Static Analysis: A Large-Scale Evaluation in Open Source Software \cite{beller2016analyzing}
%
% faz um estudo mostrando que analise estatica tem uma certa adocao em projetos livres
% e mostra onde pode-se melhorar nas ferramentas para aumentar a adoção
%
% Taming the Static Analysis Beast
% \cite{toman2017taming}
% Despite advances in tooling and mainstream success, static analysis development is still a
% painful process.

Dessa forma, definimos como objetivo geral deste trabalho avaliar o quanto a
sustentabilidade técnica e a manutenabilidade dos softwares acadêmicos de
análise estática impactam na reprodutibilidade de seus estudos.

Entre os objetivos da pesquisa, pretende-se:

\begin{enumerate}
  \item Encontrar e obter o código fonte dos softwares acadêmicos de análise
        estática.
  \item Medir e avaliar a sustentabilidade técnica e a manutenabilidade dos
        softwares acadêmicos de análise estática.
\end{enumerate}

\section{Metodologia de trabalho}

Nesta dissertação, foi investigado o quanto a sustentabilidade dos softwares
acadêmicos de análise estática impactam na reprodutibilidade dos seus estudos,
selecionamos softwares acadêmicos de análise estática, medimos a
sustentabilidade técnica e a manutenabilidade, e avaliamos o quanto essas medidas
impactam na reprodutibilidade das pesquisas onde os softwares foram criados.

A seleção de softwares acadêmicos foi realizada através de um procedimento
inspirado na revisão e no mapeamento sistemático de literatura, chamado de
revisão estruturada, composto de atividades para seleção e coleta de
informações sobre softwares acadêmicos de análise estática, essa revisão
avaliou o histórico de publicações de 25 anos da conferência ASE e 15 anos da
conferência SCAM.

As informações coletadas sobre cada software inclui nome, descrição e o
endereço onde obter uma cópia, normalmente página web ou repositório de código
fonte, esses endereços foram verificados para confirmar se os softwares estão,
de fato, disponíveis.

Os softwares disponíveis foram avaliados em relação à disponibilidade de código
fonte e à licença utilizada, essas informações, e as demais coletadas até aqui,
foram distribuídas cronologicamente, e interpretadas numa perspectiva histórica
sobre a sustentabilidade técnica dos softwares acadêmicos de análise estática.

No segundo estudo, os softwares com código fonte disponível foram avaliados em
relação a sua manutenabilidade através da métrica de complexidade estrutural. A
coleta dessa métrica para cada software foi realizada pelo Analizo, uma suíte
de ferramentas para análise de código fonte, e está sendo considerado como um
indicador de manutenabilidade.

Um conjunto de softwares de análise estática da indústria foi incluído nesta
etapa, todos os dados coletados para os softwares acadêmicos foram também
coletados para este novo conjunto. Esses softwares foram então caracterizados em
relação à frequencia de lançamentos, linguagem de programação e o tipo de
entrada suportado.

Todas estas características foram comparadas entre sí, por exemplo, softwares
com maior frequencia de lançamentos, escritos na mesma linguagem de
programação, apresentam maior complexidade estrutural? Eles são da academia ou
da indústria? Softwares da indústria apresentam melhor manutenabilidade do que
os softwares acadêmicos?

Essas perguntas serão respondidas através de uma análise exploratória dos
dados, essa análise apresenta também uma perspectiva evolutiva de alguns
softwares, aqueles com maior frequencia de lançamentos foram selecionados para
esta avaliação.

(continua...)

\section{Contribuições}

(pendente)

\section{Organização do texto}

O capítulo \ref{fundamentacao} apresenta os fundamentos teóricos necessários
para a compreensão deste trabalho.

O capítulo \ref{sustentabilidade-tecnica} traz um estudo sobre a
sustentabilidade técnica e a disponibilidade dos softwares acadêmicos de
análise estática.

O capítulo \ref{complexidade-ferramentas} descreve um estudo sobre a
manutenabilidade dos softwares acadêmicos de análise estática.

O capítulo \ref{conclusoes} apresenta as considerações finais e discute os
resultados deste trabalho.
