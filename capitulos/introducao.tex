\xchapter{Introdução}
{À medida que o software se torna uma tecnologia generalizada em praticamente
todos os aspectos da condição humana, também é inserido firmemente no meio
acadêmico: software analisa dados, simula o mundo real, e visualiza
resultados.}
\label{introducao}

A Ciência moderna depende de software. Software resolve problemas comuns de
pelo menos metade dos cientistas de todas as áreas do conhecimento
\cite{wilson2014best}. Cientistas não apenas utilizam software para fazer
Ciência como são também os seus principais desenvolvedores
\cite{goble2014better}.

Entretanto, ao desenvolver software, pesquisadores raramente publicam o seu
código fonte \cite{robles2010replicating, amann2015software}, ferindo um dos
princípios da Ciência, de que novas descobertas sejam reproduzidas antes de
serem consideradas parte da base de conhecimento \cite{stodden2009enabling}.
%
Além de desacelerar o progresso geral da Ciência, a indisponibilidade do
software acadêmico pode inviabilizar a replicação, considerado um  método comum
e cientificamente produtivo de produzir conhecimento novo a partir de pesquisas
anteriores \cite{king1995replication, stodden2010reproducible}, gerando
retrabalho e consumindo de maneira ineficiente os limitados recursos da Ciência
\cite{howison2013incentives, katz2014transitive}.

Este cenário remete ao fenômeno de desordem caótica disfuncional - DCD ({\it
``dysfunctional chaotic churn''}) do software acadêmico mencionado por
\citeonline{howison2015understanding}, caracterizado pela existência de muitos
projetos, com poucos usuários, com ciclos de vida curtos, que terminam em
paralelo ao financiamento inicial, comunidades desconectadas e paralelas,
incompatibilidades entre projetos, e tentativas aparentemente não coordenadas
de ``reiniciar'' tudo ({\it re-boots}) \cite{howison2015understanding}.

Como resultado da indisponibilidade do software acadêmico, dados são perdidos,
análises levam mais tempo que o necessário e os pesquisadores não conseguem a
eficiência que poderiam ter ao trabalhar com software \cite{wilson2017good},
ocasionando graves erros em conclusões centrais da Ciência
\cite{merali2010computational}.

Isto tem motivado a realização de conferências específicas sobre o tema, como
por exemplo o {\it RSE (Conference of Research Software
Engineers)}\footnote{\url{http://rse.ac.uk/conf2017}}, {\it WSSSPE (Workshop on
Sustainable Software for Science: Practice and
Experiences)}\footnote{\url{http://wssspe.researchcomputing.org.uk}} e {\it
RESER (Workshop on Replication in Empirical Software Engineering
Research)}\footnote{\url{http://sequoia.cs.byu.edu/reser}}.  Estes fóruns têm
contribuído para a compreensão dos problemas, abordando questões sobre
desenvolvimento, qualidade e sustentabilidade, sobre como citar software, como
promover e reconhecer o papel do pesquisador engenheiro desenvolvedor de
software ({\it Research Software Engineer}), além de questões sobre
infraestrutura, ferramentas e práticas para o desenvolvimento de software de
forma sustentável.

O desenvolvimento sustentável de software tem sido alvo de intenso debate e
pouco consenso. Sustentabilidade é um tema multidisciplinar, sistêmico e com
múltiplas dimensões, individual, social, econômica, ambiental e técnica
\cite{becker2014karlskrona}. A dimensão técnica, por exemplo, diz respeito a
capacidade do software de perdurar e de continuar sendo suportado ao longo do
tempo, o que implica em qualidades de longevidade e manutenção
\cite{venters2014software}.

Na dimensão técnica, o desenvolvimento de software para a Ciência, ou seja, o
desenvolvimento de {\it software acadêmico}, de forma sustentável, abre portas
para elevar a qualidade geral do software e da própria pesquisa científica,
proporcionando um ambiente de compartilhamento e colaboração em oposição ao
tradicional modelo de competição que permeia o sistema de reputação e crédito
científico.

Dessa forma, neste trabalho investigamos como o software acadêmico do domínio
de análise estática, desenvolvido e publicado em duas importantes conferências
de Engenharia de Software, {\it ASE (Automated Software
Engineering)}\footnote{\url{http://ase-conferences.org}} e {\it SCAM (Working
Conference on Source Code Analysis \&
Manipulation)}\footnote{\url{http://www.ieee-scam.org}}, se apresenta em termos
de sustentabilidade técnica, identificando problemas e abrindo caminho para a
melhoria geral da qualidade das ferramentas e das pesquisas neste campo.

\section{Escopo}

\subsection{Objetivo Geral}

Analisar os \textit{projetos de software acadêmico de análise estática e sua sustentabilidade técnica}
com o propósito de \textit{caracterizar}
com respeito a \textit{publicização, reconhecimento e ciclo de vida}
na perspectiva do \textit{cientista -- desenvolvedor ou usuário -- de software acadêmico}
no contexto das \textit{conferências de Engenharia de Software ASE e SCAM}.

\subsection{Objetivos Específicos}

São objetivos específicos deste trabalho:

\begin{description}
  \item [01]
    Caracterizar o software acadêmico de análise estática com respeito à sua
    publicização.
    A caracterização será feita em um conjunto de software acadêmico de análise
    estática, com base em medidas para avaliar a sua disponibilidade de
    download, código fonte, e presença oficial online.
  \item [02]
    Caracterizar o software acadêmico de análise estática com respeito ao
    reconhecimento na literatura acadêmica.
    A caracterização será feita em um conjunto de software acadêmico de análise
    estática, com base em uma análise de trabalhos científicos que o utiliza ou
    adapta.
  \item [03]
    Caracterizar o software acadêmico de análise estática com respeito ao
    seu ciclo de vida.
    A caracterização será feita em um conjunto de software acadêmico de análise
    estática, com base nas informaçoes sobre lançamentos ({\it releases}) e
    métricas de código fonte.
\end{description}

\subsection{Questão de Pesquisa}

\newcommand{\QuestaoGeralUm}{
  Como a desordem caótica disfuncional (DCD) pode explicar a sustentabilidade técnica
  dos projetos do ecossistema de software acadêmico de análise estática em
  termos de publicização, reconhecimento e estágio de evolução?
}

Neste estudo a seguinte questão de pesquisa, a respeito do software acadêmico
de análise estática, será investigada:

\begin{description}
  \item [Questão:] \QuestaoGeralUm
\end{description}

\subsection{Métricas}

Para responder à questão de pesquisa, as seguintes métricas serão usadas:

\begin{enumerate}
  \item Métricas relacionadas à publicização do software (número de projetos
  disponíveis para download, com código fonte disponível, tipo de licença);

  \item Métricas relacionadas ao reconhecimento do software (número de
  citações, número de menções, número de usos e contribuições);

  \item Métricas relacionadas ao ciclo de vida do software (número total de
  lançamentos, data e número de versão de cada lançamento, variação no número
  de módulos do código fonte).
\end{enumerate}

\section{Estratégia de Pesquisa}

Este trabalho apresenta um estudo de caso exploratório ({\it exploratory case
study}) sobre a sustentabilidade técnica do software acadêmico de análise
estática. O estudo adotou uma estratégia de pesquisa de trabalho de campo ({\it
field studies}) em ambiente natural ({\it natural settings})
\cite{stol2015holistic}, com as seguintes características principais:

\begin{itemize}
  \item Com o foco num fenômeno, organização ou sistema em particular;
  \item Com um baixo nível de generalização e alto realismo do contexto;
  \item Sem intervenção do pesquisador no ambiente.
\end{itemize}

Foram realizados três estudos sobre o software acadêmico de análise estática em
termos de sua publicização, reconhecimento e ciclo de vida.

No primeiro estudo, sobre publicização, um conjunto de projetos de software
publicados nas conferências de Engenharia de Software ASE e SCAM foi
selecionado. Os projetos foram caracterizados em termos de disponibilidade de
download, código fonte, forma de distribuição e licença.

A escolha destas duas conferências baseou-se no princípio de serem conferências
tradicionais e importantes para a área de Engenharia de Software, tendo
tradição em estudos sobre análise de software, e dessa forma, potencializando
aumentar o número de projetos de software acadêmico de análise estática
encontrados entre suas publicações.

No segundo estudo, sobre reconhecimento, os projetos selecionados pelo primeiro
estudo foram caracterizados em relação ao reconhecimento acadêmico em termos de
menções encontradas em publicações nas bases ACM e IEEE. As menções
encontradas foram classificadas por tipo, e cada artigo mencionando o software
selecionado foi caracterizado segundo os tipos encontrados.

No terceiro estudo, sobre ciclo de vida, os projetos de software acadêmico
foram caracterizados em relação ao estágio de ciclo de vida em que se
encontram, em termos de lançamentos, versões, disponibilidade de download,
disponibilidade de código fonte, e número de módulos para cada versão com
código fonte disponível.

\section{Organização do texto}

O Capítulo \ref{fundamentacao} apresenta os fundamentos teóricos necessários
para a compreensão deste trabalho. O Capítulo \ref{estudo1} traz um estudo
sobre a publicização de software acadêmico de análise estática nas conferências
de Engenharia de Software ASE e SCAM. O Capítulo \ref{estudo2} descreve um
estudo sobre o reconhecimento de software acadêmico de análise estática em
publicações nas bases ACM e IEEE. O Capítulo \ref{estudo3} apresenta um estudo
sobre o estágio de evolução e o ciclo de vida de software acadêmico de análise
estática. O Capítulo \ref{discussao} discute os resultados em termos da
sustentabilidade técnica do software acadêmico de análise estática e traça
paralelos com trabalhos relacionados. O Capítulo \ref{conclusoes} apresenta as
considerações finais da pesquisa e aponta trabalhos futuros.
