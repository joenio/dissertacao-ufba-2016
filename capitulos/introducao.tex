\xchapter{Introdução}
{``Ocorrências não reprodutíveis não têm significado para a ciência'' \cite{popper2004logica}.}

%\section{Apresentação}

A Ciência moderna depende de software. Projetos de software resolvem
problemas comuns de metade dos cientistas de todas as áreas da Ciência durante
atividades de pesquisa \cite{wilson2014best}. Estes projetos, definidos neste
estudo como software acadêmico -- também referenciado na literatura como {\it
research-originated software} \cite{kon2011free}, {\it research tool}
\cite{portillo2012tools} ou {\it academic software} \cite{allen2017engineering}
-- quando desenvolvidos por cientistas carregam boa parte do conhecimento
empregado em suas pesquisas.

Novos projetos de software são constantemente criados e desenvolvidos durante
pesquisas científicas, definido neste estudo como software acadêmico

Em pesquisas da Ciência da Computação, é prática comum que
algum tipo de software seja desenvolvido para apoiar a pesquisa em andamento ou
mesmo como principal resultado da pesquisa. Estes projetos 

Falar MAIS sobre o contexto   ...
Por exemplo, um pesquisador pode desenvolver ...  ... exemplo,  ... Outros
pesquisadores podem usar tal software em suas pesquisas ... outros podem usá-lo
para comparar resultados com seus próprios resultados, etc ...

em ciência da computação,
particularmente em engenharia de software, tem-se notado um aumento constante
no número de novos softwares acadêmicos \cite{allen2017engineering}.

A comunidade tem refletido sobre os problemas relacionados ao
desenvolvimento, promoção e sustentabilidade desses softwares, e o
impacto que tais problemas causam no meio científico \cite{allen2017engineering}. Esta
reflexão tem mostrado, por exemplo, que muitos estudos em engenharia de
software sofrem de dificuldades de repetição \cite{tang2016worthiness}, e apontam
problemas específicos relacionados à manutenibilidade e a sustentabilidade
técnica dos softwares acadêmicos.

Manutenabilidade é uma característica de qualidade que indica o quão fácil é
realizar atividades de evolução e manutenção em softwares, um aspecto
importante aos pesquisadores interessados em adaptar softwares acadêmicos, algo
muitas vezes necessário ao reproduzir pesquisas anteriores \cite{peng2011reproducible}.
Sustentabilidade técnica diz respeito a longevidade dos softwares, ou seja, a
capacidade de continuar disponível no futuro. Muitos pesquisadores não
disponibilizam os seus softwares \cite{robles2010replicating,
amann2015software} ou quanto o fazem enfrentam problemas com disponibilidade e
manutenibilidade \cite{prlic2012ten}, isto leva a um corpo computacional
extramente difícil de reproduzir uma vez que mais da metade dos pesquisadores
desenvolvem seus próprios softwares \cite{hettrick2014uk}, além de ferir um dos
fundamentos da ciência de que novas descobertas sejam reproduzidas antes de
serem consideradas parte da base de conhecimento \cite{stodden2009enabling}.

Isto tem motivado a organização de conferências específicas sobre o tema, como
o RSE\footnote{Conference of Research Software Engineers
\url{http://rse.ac.uk/conf2017}}, WSSSPE\footnote{Workshop on Sustainable
Software for Science: Practice and Experiences
\url{http://wssspe.researchcomputing.org.uk}} e o RESER\footnote{Workshop on
Replication in Empirical Software Engineering Research
\url{http://sequoia.cs.byu.edu/reser}}, e tem contribuido para a compreensão
dos problemas relacionados aos softwares acadêmicos, abordando questões sobre
desenvolvimento, qualidade e sustentabiliade, sobre como citar softwares em
novas pesquisas, como promover e reconhecer o papel do pesquisador engenheiro
desenvolvedor de softwares acadêmicos, além de questões sobre infraestrutura,
ferramentas e práticas para o desenvolvimento de softwares acadêmicos de
forma sustentável.

nálise estática de software, uma área com uma longa e respeitável
tradição e que ainda sofre carência de estudos sobre avaliação e validação de
seus softwares \cite{li2010comparative, ilyas2016static}.

publicações usando softwares como método, mostrou que apenas 59 mencionavam o
uso de softwares de alguma forma, os demais 31 artigos, apesar de usar software
acadêmico, não mencionavam nada a respeito \cite{howison2016software}.

Isto gera um impacto negativo na visibilidade dos softwares acadêmicos e faz
surgir questionamentos sobre a sua qualidade, não apenas técnica, mas também a
capacidade de ser encontrado, compartilhado e co-desenvolvido, qualidades
importantes para a evolução do próprio software, mas também extremamente útil
para um uso eficiente dos limitados recursos da ciência \cite{howison2013incentives,
katz2014transitive}.

No entanto, parece ser regra geral não testar ou não documentar o próprio
software, pesquisadores geralmente não testam ou documentam seus softwares
acadêmicos, a maioria também não sabe o quão confiável seu software é,
ocasionando graves erros em conclusões centrais da literatura,
gerando retrabalho nas mais diversas áreas da ciência \cite{merali2010computational},
apesar de nem sempre ser possível, ou viável, ter tudo dentro de
padrões estritos, é preciso estar consciente das boas práticas ao
produzir e utilizar softwares acadêmicos, tanto para melhorar a própria
abordagem quanto para revisar outros trabalhos \cite{wilson2014best}.

A maior parte dos cientistas (90\%) no entretanto nunca tiveram treinamento
algum de como escrever software de forma eficiente, faltam práticas básicas de
desenvolvimento, como escrever código legível, revisão de código, controle de
versão, testes unitários, entre outros, como resultado, dados são perdidos,
análises levam mais tempo que o necessário e os pesquisadores não conseguem a
eficiência que poderiam ter ao trabalhar com softwares acadêmicos
\cite{wilson2017good}.

Isto contradiz as boas práticas de qualquer projeto experimental ({\it
laboratory
notebooks}\footnote{\url{https://en.wikipedia.org/wiki/Lab_notebook}}, dados
organizados, passos documentados, projeto estruturado para reprodutibilidade) e
torna praticamente impossível utilizar o método mais comum e cientificamente
produtivo de produzir conhecimento novo a partir de pesquisas anteriores, a
replicação, ou seja, seguir os mesmos passos do autor original com objetivo de
validar, melhorar ou estender seus dados e sua metodologia
\cite{king1995replication, stodden2010reproducible}.

Somado a isto temos ainda o fato de que pesquisadores raramente publicam seus
códigos \cite{robles2010replicating, amann2015software}, piorando ainda mais toda a situação, isto tem motivado a organização
de conferências específicas para discutir os problemas dos softwares
acadêmicos, como o RSE (Conference of Research Software Engineers)\footnote{
\url{http://rse.ac.uk/conf2017}}, WSSSPE (Workshop on Sustainable Software for
Science: Practice and Experiences)\footnote{
\url{http://wssspe.researchcomputing.org.uk}} e o RESER (Workshop on
Replication in Empirical Software Engineering Research)\footnote{
\url{http://sequoia.cs.byu.edu/reser}}.
%>>>>>>> rascunho da introducao
%
%Independente da finalidade, tamanho ou motivação, todo software tem o potencial
%de voltar a ser útil em outros momentos ou lugares, para o autor original,
%ou para pesquisadores enfrentando problemas semelhantes aos dos autores originais.
%Não é difícil imaginar que o problema enfrentado por um pesquisador pode, em
%algum momento, ser enfrentado por outros pesquisadores, criando assim uma ótima
%oportunidade de colaboração entre pesquisas e pesquisadores, sendo o software
%um excelente vetor de ajuda mútua em ambas as direções.

% Software acadêmico sofre de {\it ``dysfunctional chaotic churn''}.
% O ecossistema de software acadêmico sofre de um fenômeno chamado de
% desordem caótica disfuncional ({\it ``dysfunctional chaotic churn''}).

Apesar disso, cientistas têm percebido que os softwares desenvolvidos na academia 
sofrem de desordem caótica disfuncional ({\it ``dysfunctional chaotic churn''}), 
ou seja, a existência
de muitos projetos, com poucos usuários, com ciclos de vida curtos, que
terminam em paralelo ao financiamento inicial, comunidades desconectadas e
paralelas, incompatibilidades entre projetos, e tentativas aparentemente não
coordenadas de ``reiniciar'' tudo ({\it re-boots}).

%Este cenário, além de desacelerar o progresso geral da ciência gerando
%retrabalho, faz surgir questionamentos sobre as conclusões dessas pesquisas,
%especialmente quando grande parte dos pesquisadores não sabem o quão confiável
%seus softwares são.

Este problema, apesar de ser percebido por muitos cientistas, carece de
evidências. % \cite{chitaozinho & xororó}  ;-)  
Neste trabalho, investigamos como o problema ...  se manifesta em pesquisas da
engenharia de software, em especial, em publicações de análise estática, uma
área com uma longa e respeitável tradição em pesquisas sobre a criação de novas
ferramentas, métodos e algoritmos.

\section{Escopo}

O objetivo geral desta pesquisa é explorar como o {\it ``dysfunctional chaotic
churn''} se manifesta entre os projetos de software de análise estática,
respondendo as seguintes questões:

\begin{itemize}
  \item Os projetos sustentáveis tem contribuidores além dos autores iniciais?
  \item Os projetos sustentáveis tem mais contribuidores?
  \item Os artigos de projetos sustentáveis são mais lidos, mais citados?
  \item Os autores de projetos sustentáveis tem mais publicações com o uso de software do que os autores?
  \item Os projetos são mais fáceis de manter? de usar? 
\end{itemize}

Estas questões darão importantes indícios sobre a qualidade do software acadêmico de
análise estática desenvolvido na academia, especialmente sobre a capacidade de serem 
encontrados, compartilhados e co-desenvolvidos.

\section{Questão de Pesquisa}

Quão sustentável é o software acadêmico de análise estática?

Qual a relação entre sustentabilidade e reproducibilidade?

\section{Metodologia de trabalho}

(apresentar aqui um resumo do capítulo \ref{metodologia})

\subsection{Objetivos}

O objetivo geral deste trabalho é caracterizar a relação entre sustentabilidade
e reproducibilidade no contexto de software acadêmico e, mais especificamente,
software de análise estática. 

São objetivos específicos deste trabalho:

\begin{description}
  \item[O1] Caracterizar o software acadêmico de análise estática com respeito à sua sustentatibilidade técnica.
A caracterização será feita em um conjunto de software acadêmico de análise estática, com base em medidas para avaliar
sua sustentabilidade técnica e manutenibilidade.
  \item[O2] Caracterizar o software acadêmico de análise estática com respeito à sua manutenibilidade.
A caracterização será feita em um conjunto de software acadêmico de análise estática, com base em uma análise de trabalhos científicos que o utiliza ou adapta.
  \item[03] Avaliar a relação sustentabilidade e reproducibilidade para software acadêmico de análise estática.
\end{description}

\section{Questão de Pesquisa}

Quão sustentável é o software acadêmico de análise estática?

Qual a relação entre sustentabilidade e reproducibilidade?

\section{Metodologia de trabalho}

(apresentar aqui um resumo do capítulo \ref{metodologia})

% Objetivo específico: caracterizar software acad publicado quanto a ...
% Objetivo específico: caracterizar o tipo de citação / menção d

\section{Contribuições}

(pendente)

A caracterização ... pode contribuir para a sustentabilidade do software acadêmico de análise estática.

\section{Organização do texto}

(pendente)

%O capítulo \ref{fundamentacao} apresenta os fundamentos teóricos necessários
%para a compreensão deste trabalho.
%
%O capítulo \ref{metodologia} apresenta a estratégia de pesquisa adotada nos
%estudos.
%
%O capítulo \ref{sustentabilidade-tecnica} traz um estudo sobre a
%sustentabilidade técnica e a disponibilidade dos softwares acadêmicos de
%análise estática.
%
%O capítulo \ref{complexidade-ferramentas} descreve um estudo sobre a
%manutenibilidade dos softwares acadêmicos de análise estática.
%
%O capítulo \ref{discussao} ...
%
%O capítulo \ref{conclusoes} apresenta as considerações finais e discute os
%resultados deste trabalho.

% A Retrospective Study of Academic Software Projects
% (estudo restropectivo parece um bom termo, uma busca rápida no google-scholar retornou muita coisa da área de saúde)

