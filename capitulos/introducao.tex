\xchapter{Introdução}{}

Em diversas linhas de pesquisa da computação, em especial, em engenharia de
software, é bastante comum que novos softwares sejam desenvolvidos durante
trabalhos de pesquisa, tais softwares tem sido chamados na literatura de {\it
research-originated software} \cite{Kon2011}, {\it research tool}
\cite{Portillo12} ou {\it academic software} \cite{allen2017engineering}, e vêm
ganhando atenção da academia devido a sua importância para o amadurecimento da
área.

Este amadurecimento inclui a capacidade de replicar pesquisas anteriores e
assim aumentar a validade em seus resultados, algo de extrema importância para
contribuir com a base de conhecimento da comunidade de pesquisa. Esses
softwares, chamados a partir daqui apenas de softwares acadêmicos, ocupam um
papel central na discussão sobre replicação \cite{Stodden2009, Peng2011} e
podem impactar de forma positiva ou negativa nesta questão.

Isto tem feito a comunidade voltar sua atenção para os softwares acadêmicos e a
refletir sobre os problemas relacionados ao seu desenvolvimento, promoção,
reconhecimento e sustentabilidade \cite{allen2017engineering}. Esta reflexão
tem mostrado, por exemplo, que muitos estudos sofrem de dificuldades de
repetição \cite{Tang2016}, e apontam problemas específicos relacionados à
disponibilidade e a qualidade dos softwares. Muitos pesquisadores não
disponibilizam os seus softwares \cite{robles2010replicating,
amann2015software} ou quanto o fazem enfrentam problemas com manutenabilidade
\cite{Prlic2012}, uma qualidade importante aos interessados em adaptar tais
softwares.

Apesar da crescente preocupação com os softwares acadêmicos, sabe-se pouco
sobre o quanto algumas de suas características afetam a capacidade de
reprodução de seus estudos, sobretudo em áreas específicas da ciência da
computaçao, como a área de análise estática de software, uma área com uma longa
e respeitável tradição, mas que ainda sofre com carência de estudos sobre
avaliação e validação de suas ferramentas \cite{Li2010, ilyas2016static}.

%softwares de análise estática de
%código fonte tem sido continuamente desenvolvidas e tem se tornado mais comuns
%no ciclo de desenvolvimento de software \cite{Novak2010}, e apesar da rápida e
%constante evolução da área, 
%
%A resposta à todas estas questões pode nos levar a inúmeros caminhos e
%resultados variados, dentre os quais avaliamos como uma boa oportunidade de
%investigação avaliar os softwares científicos do ponto de vista de sua
%qualidade interna a fim de compreender quanta atenção é dada ao seu
%desenvolvimento, uma vez que tradicionalmente os
% 
% Manutenabilidade é uma característica de qualidade externa que indica o quão
% fácil é realizar atividades de evolução e manutenção em componentes de
% software, ela pode ser medida através de características de qualidade interna,
% uma vez que grande parte dos engenheiros de software assumem que uma boa
% estrutura interna resulta em boa qualidade externa \cite{Fenton2014}. A
% estrutura interna de um software pode ser avaliada através da sua complexidade,
% uma característica bastante referenciada na literatura como um importante
% indicador de qualidade, estudos mostram que quanto maior a complexidade, maior
% é o esforço de manutenção \cite{hashim1996software, Darcy2005}, em especial a
% complexidade estrutural, uma medida definida em termos de acoplamento e coesão
% \cite{Terceiro2012}.
%
% Avaliar esses softwares, chamados a partir daqui apenas de {\it software
% científico}, deve ser uma preocupação da engenharia de software, uma disciplina
% centrada em avaliar e validar métodos, técnicas, linguagens e ferramentas.
% Estudos avaliando softwares são bastante comuns, apenas no domínio de aplicação
% de análise estática, por exemplo, é possível encontrar diversos estudos
% avaliando ferramentas, desde ferramentas de localização de bugs
% \cite{Rutar2004}, detecção de {\it buffer overflow} \cite{Kratkiewicz2005},
% segurança \cite{Okun2007, Johns2011}, detecção de falhas e refatorações
% \cite{Wedyan2009}, detecção de vulnerabilidades \cite{Li2010, Ataide2014},
% cálculo de métricas \cite{Alemerien2013}, até estudos comparando ferramentas de
% análise estática com compiladores comuns \cite{Emanuelsson2008} e estudos
% comparando ferramentas comerciais com ferramentas {\it open source}
% \cite{Al2010}.
% 
% Apesar dos inúmeros estudos avaliando softwares, poucos fazem isso com
% softwares científicos, algo de extrema importância para compreender o quanto
% estão contribuindo para a divulgação do conhecimento e para replicação dos
% resultados das pesquisas em engenharia de software.
%
% (2) Tools to support systematic literature reviews in software engineering: A mapping study \cite{marshall2013tools}
%
% Cita um mapeamento feito sobre estudos que criam ferramentas para apoio a
% revisão sistemática no domínio de SE, 14 estudos foram selecionados, ao final
% apenas 8 tinham proposta de ferramentas, ao final conclui que as ferramentas
% encontradas estão em estado inicial de desenvolvimento. 
%
% (3) Tools used in Global Software Engineering: A systematic mapping review \cite{Portillo12}
%
% Cita um mapeamento sistemático com objetivo de encontrar ferramentas de
% comunicação e coordenação para suporte a times altamente distribuidos
% gograficamente, encontrou 132 ferramentas, para uso em projetos de software
% global. A maioria destas ferramentas foram desenvolvidas em centros de
% pesquisas, e apenas uma pequena porcentagem (18.9\%) foram testados fora do
% seu contexto onde foi desenvolvido.
%
% (5) Tools in mining software repositories \cite{chaturvedi2013tools}
%
% Faz uma revisão dos papers submetidos ao MSR desde 2007 até 2013 (?) e
% identifica data sets, ferramentas e técnicas utilizadas pelos autores, mais
% da metade dos papers usam ou criam ferramentas, categoriza as ferramentas em
% ferramentas novas, ferramentas tradicionais, protótipos e scripts para
% mineração de dados
%
% (6) A systematic literature review of software product line management tools \cite{pereira2015systematic}
%
% (???)
%
% (7) Software configuration management tools \cite{chan1997software}
%
% (???)
%
% (8) Comparison and evaluation of source code mining tools and techniques: A qualitative approach \cite{khatoon2013comparison}
%
% Lista ferramentas e técnicas para mineração de dados, estado da arte.
%
% (9) An overview of free software tools for general data mining \cite{jovic2014overview}
%
% Descreve característica dos 6 softwares livres mais usados para mineração de
% dados no geral.
%
% (10) Analyzing the State of Static Analysis: A Large-Scale Evaluation in Open Source Software \cite{beller2016analyzing}
%
% faz um estudo mostrando que analise estatica tem uma certa adocao em projetos livres
% e mostra onde pode-se melhorar nas ferramentas para aumentar a adoção
%
% Taming the Static Analysis Beast
% \cite{toman2017taming}
% Despite advances in tooling and mainstream success, static analysis development is still a
% painful process.
%
% A resposta à todas estas questões pode nos levar a inúmeros caminhos e
% resultados variados, dentre os quais avaliamos como uma boa oportunidade de
% investigação avaliar os softwares científicos do ponto de vista de sua
% qualidade interna a fim de compreender quanta atenção é dada ao seu
% desenvolvimento, uma vez que tradicionalmente os autores de software científico
% enfrentam problemas com manutenabilidade e disponibilidade de tais softwares
% \cite{Prlic2012}.
%
% Manutenabilidade é uma característica de qualidade externa que indica o quão
% fácil é realizar atividades de evolução e manutenção em componentes de
% software, ela pode ser medida através de características de qualidade interna,
% uma vez que grande parte dos engenheiros de software assumem que uma boa
% estrutura interna resulta em boa qualidade externa \cite{Fenton2014}. A
% estrutura interna de um software pode ser avaliada através da sua complexidade,
% uma característica bastante referenciada na literatura como um importante
% indicador de qualidade, estudos mostram que quanto maior a complexidade, maior
% é o esforço de manutenção \cite{hashim1996software, Darcy2005}, em especial a
% complexidade estrutural, uma medida definida em termos de acoplamento e coesão
% \cite{Terceiro2012}.

Avaliar questões sobre sustentabilidade e desenvolvimento, especialmente sobre
a sustentabilidade técnica, ou, a capacidade de perdurar e estar disponível no
futuro e a qualidade interna, ou, a forma como os softwares estão estruturados
e organizados internamente, nos ajudará a compreender o quanto tais
características impactam na reprodutibilidade de seus estudos.

Desta forma, definimos como objetivo geral deste trabalho avaliar o quanto a
sustentabilidade técnica e a qualidade interna dos softwares acadêmicos de
análise estática impactam na reprodutibilidade de seus estudos.

São objetivos específicos deste trabalho:

\begin{enumerate}
  \item Medir e avaliar a sustentabilidade técnica dos softwares acadêmicos de
        análise estática de software.
  \item Medir e avaliar a qualidade interna dos softwares acadêmicos de análise
        estática de software.
  \item Compreender o quanto a sustentabilidade técnica e a qualidade interna
        dos softwares acadêmicos de análise estática de software impactam na
        reprodutibilidade de seus estudos.
\end{enumerate}

%% Além dos argumentos já citados estamos também em sintonia com a crescente
%% preocupação em relação ao tema {\it sustentabiliade de software}, 
%% 
%% com alguns dos compromissos do {\it Dagstuhl Manifesto}, um documento
%% que resume algumas perspectivas discutidas no workshop {\it Dagstuhl
%% Perspectives Workshop on ``Engineering Academic Software''} voltado a examinar
%% as fortalezas, fraquezas, riscos e oportunidades da engenharia de software
%% acadêmica, explora tópicos novos e emergentes na ciência da computação e produz
%% manifestos que capturam tendências e desenvolvimentos relacionados à estes
%% tópicos. A contribuição chave deste workshop é o {\it Dagstuhl Manifesto}, um
%% roteiro para o futuro da engenharia de software profissional e acadêmica, com
%% foco em instrumentos para pesquisas em software.  O manifesto é expresso em
%% termos de ações ``promessas'' destinados a usuários e desenvolvedores de
%% softwares acadêmicos, com passos concretos para melhorar o ambiente em que os
%% softwares são produzidos.
%% 
%% O compromisso (iii) do manifesto Dagstuhl \cite{allen2017engineering} de {\it
%% ``medir a qualidade e sustentabilidade do software durante e após o seu
%% desenvolvimento''} é aquele que está em sintonia com este trabalho,
%% especialmente em relação às questões após o desenvolvimento dos softwares.
%% Os tópicos levantados no manifesto e discutos nos capítulos 3 e 4 do manifesto
%% relacionados ao presente trabalho são os seguintes:
%% 
%% \begin{description}
%% 
%%   \item [Sustentabilidade e disponibilidade]
%% 
%%     O software que nós usamos e produzimos no contexto acadêmico é sustentável?
%%     Temos certeza de que podemos reproduzir métodos de pesquisas anteriores no
%%     futuro, tendo em vistas mudanças arbitrárias nos contextos tecnológicos
%%     (máquinas, sistemas operacionais, linguagens de programação, frameworks)?
%%     Podemos adaptar de forma incremental o software de pesquisa para
%%     oportunidades emergentes ao mesmo tempo, sem perda de reprodutibilidade e
%%     sem custos proibitivos?
%% 
%%   \item [Qualidade]
%% 
%%     Como podemos garantir a qualidade do software acadêmico? Como podemos
%%     monitorar, orientar, informar e revisar a qualidade do software acadêmico?
%%     Como podemos gerenciar e garantir confiança entre as equipes de pesquisa
%%     acadêmica considerando software desenvolvido para métodos de presquisa e/ou
%%     de produção?
%% 
%% \end{description}

\section{Metodologia de trabalho}

(pendente)

\section{Contribuições}

(pendente)

\section{Organização do texto}

(pendente)
