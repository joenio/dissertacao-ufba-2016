\xchapter{Introdução}
{À medida que o software se torna uma tecnologia generalizada em praticamente
todos os aspectos da condição humana, também é inserido firmemente no meio
acadêmico, software analisa dados, simula o mundo real, e visualiza
resultados.}
\label{introducao}

A Ciência moderna depende de software. Software resolve problemas comuns de
pelo menos metade dos cientistas de todas as áreas do conhecimento
\cite{wilson2014best}. Cientistas não apenas utilizam software para fazer Ciência como são também
os seus principais desenvolvedores
\cite{goble2014better}.

Entretanto, ao desenvolver software, pesquisadores raramente publicam o
seu código fonte \cite{robles2010replicating, amann2015software}, ferindo um dos
princípios da Ciência, de que novas descobertas sejam reproduzidas antes de
serem consideradas parte da base de conhecimento \cite{stodden2009enabling}.
%
Além de desacelerar o progresso geral da Ciência, a indisponibilidade 
do software acadêmico pode inviabilizar a replicação, considerado
um  método comum e cientificamente produtivo de produzir
conhecimento novo a partir de pesquisas anteriores
\cite{king1995replication, stodden2010reproducible}, 
gerando retrabalho e consumindo de maneira ineficiente os limitados recursos da Ciência
\cite{howison2013incentives, katz2014transitive}.

Este cenário remete ao fenômeno de desordem caótica disfuncional ({\it ``dysfunctional
chaotic churn''}) mencionado por \citeonline{howison2015understanding}, 
caracterizado pela existência de muitos projetos, com poucos
usuários, com ciclos de vida curtos, que terminam em paralelo ao financiamento
inicial, comunidades desconectadas e paralelas, incompatibilidades entre
projetos, e tentativas aparentemente não coordenadas de ``reiniciar'' tudo
({\it re-boots}) \cite{howison2015understanding}.


Como resultado, dados são perdidos,
análises levam mais tempo que o necessário e os pesquisadores não conseguem a
eficiência que poderiam ter ao trabalhar com software
\cite{wilson2017good},
ocasionando graves erros em conclusões centrais da Ciência \cite{merali2010computational}.

%especialmente quando grande parte dos pesquisadores não sabem o quão
%confiável seu software é,
%Neste cenário tem surgido questionamentos sobre conclusões centrais da Ciência,

Isto tem motivado a realização de conferências específicas sobre o tema,
como por exemplo o {\it RSE (Conference of Research
Software Engineers)}\footnote{ \url{http://rse.ac.uk/conf2017}}, {\it WSSSPE
(Workshop on Sustainable Software for Science: Practice and
Experiences)}\footnote{ \url{http://wssspe.researchcomputing.org.uk}} e {\it
RESER (Workshop on Replication in Empirical Software Engineering
Research)}\footnote{ \url{http://sequoia.cs.byu.edu/reser}}, algo que tem contribuido para
a compreensão dos problemas, abordando questões sobre desenvolvimento,
qualidade e sustentabilidade, sobre como citar software, como promover e
reconhecer o papel do pesquisador engenheiro desenvolvedor de software ({\it
Research Software Engineer}), além de questões sobre infraestrutura,
ferramentas e práticas para o desenvolvimento de software de forma sustentável.

%Esta
%reflexão tem mostrado, por exemplo, que muitos estudos em engenharia de
%software sofrem de dificuldades de repetição \cite{tang2016worthiness}, e apontam
%problemas específicos relacionados à manutenibilidade e a sustentabilidade
%técnica dos softwares acadêmicos.

%Estes projetos, definidos neste
%estudo como software acadêmico -- também referenciado na literatura como {\it
%research-originated software} \cite{kon2011free}, {\it research tool}
%\cite{portillo2012tools} ou {\it academic software} \cite{allen2017engineering}

O desenvolvimento sustentável de software tem sido alvo de intenso debate e
pouco consenso, sustentabilidade é um tema multidisciplinar, sistêmico e com
múltiplas dimensões, individual, social, econômica, ambiental e técnica
\cite{becker2014karlskrona}. A dimensão técnica, por exemplo, diz respeito a
capacidade do software de perdurar e de continuar sendo suportado ao longo do
tempo, o que implica em qualidades de longevidade e manutenção
\cite{venters2014software}.

Através desta dimensão, o desenvolvimento de software para a Ciência, ou seja,
o desenvolvimento de {\it software acadêmico}, de forma sustentável, abre portas para elevar a qualidade
geral do software e da própria pesquisa científica, proporcionando um
ambiente de compartilhamento e colaboração em oposição ao tradicional modelo de
competição que permeia o sistema de reputação e crédito científico.

%, resultando assim na melhoria da
%qualidade geral da própria Ciência, reduzindo custos e evitando retrabalho.

Dessa forma, neste trabalho investigamos como o software acadêmico do domínio
de análise estática, desenvolvido e publicado em duas importantes conferências
de Engenharia de Software, ASE e SCAM, se apresenta em termos de
sustentabilidade técnica, identificando problemas e abrindo caminho para a
melhoria geral da qualidade das ferramentas e das pesquisas neste campo.

%entre pesquisas e publicações de duas conferências
%importantes sobre a área de análise estática, ASE e SCAM, análise estática é
%uma área com uma longa e respeitável tradição em pesquisas sobre a criação de
%novas ferramentas, métodos e algoritmos e que ainda sofre carência de estudos
%sobre avaliação e validação de seus softwares \cite{li2010comparative,
%ilyas2016static}.

%No entanto, a compreensão da sustentabilidade técnica de um software depende
%fortemente do seu domínio de aplicação, assim, investigar um domínio específico
%tanto contribui para o debate sobre sustentabilidade de software, quanto para
%aumentar o nível de compreensão sobre este tema no domínio em questão.

%Apesar ... ainda não se sabe como domínios específicos são afetados pela falta
%de sustentabilidade técnica do software acadêmico, este problema, apesar de ser
%percebido por muitos cientistas, carece de evidências
%\cite{howison2015understanding}, 

%Ou ainda, em pesquisas da Engenharia de Software, onde tem se notado um
%crescimento constante no número de projetos de software acadêmico desenvolvidos
%durantes pesquisas \cite{allen2017engineering}.

%Falar MAIS sobre o contexto   ...
%Por exemplo, um pesquisador pode desenvolver ...  ... exemplo,  ... Outros
%pesquisadores podem usar tal software em suas pesquisas ... outros podem usá-lo
%para comparar resultados com seus próprios resultados, etc ...

\section{Escopo}

%O que sabemos sobre a sustentabilidade técnica de projetos de software
%acadêmico de análise estática publicados em conferências de Engenharia de
%Software? Como estes projetos são mencionados na literatura acadêmica? Há
%colaboração entre os cientistas no desenvolvimento destes projetos?

%\subsection{Definição do Objetivo}
%
%\begin{description}
%  \item{\bf Objeto de estudo.}
%    O objeto de estudo são projetos de software acadêmico de análise estática
%    publicados em artigos científicos e sua publicização, reconhecimento e
%    ciclo de vida.
%  \item{\bf Propósito.}
%    O propósito deste estudo é caracterizar a sustentabilidade técnica de cada
%    software acadêmico de análise estática, trazendo informações que permitam
%    compreender o problema de desordem disfuncional caótica no domínio de
%    análise estática.
%  \item{\bf Perspectiva.}
%    A perspectiva considerada é a de cientistas e pesquisadores, isto é, o
%    cientista ou pesquisador gostaria de conhecer o ecossistema de software
%    acadêmico de análise estática em termos de sua sustentabilidade técnica.
%    Além disso, pessoas da indústria podem estar interessadas em conhecer
%    software acadêmico de análise estática para financiá-lo.
%  \item{\bf Foco de qualidade.}
%    O principal aspecto de qualidade estudado é a sustentabilidade técnica, com
%    destaque para três aspectos: publicização, reconhecimento e ciclo de vida
%    de cada projeto de software acadêmico de análise estática.
%  \item{\bf Contexto.}
%    O estudo foi conduzido com projetos de software acadêmico de análise
%    estática publicados nas conferências de Engenharia de Software ASE e SCAM.
%\end{description}

%\subsection{Sumário da Definição}
\subsection{Objetivo Geral}

Analisar os \textit{projetos de software acadêmico de análise estática e sua sustentabilidade técnica} %object of study
com o propósito de \textit{caracterizar} %purpose
com respeito a \textit{publicização, reconhecimento e ciclo de vida} %quality focus
na perspectiva do \textit{cientista -- desenvolvedor ou usuário -- de software acadêmico} %perspective
no contexto das \textit{conferências de Engenharia de Software ASE e SCAM}. %context

%\subsubsection{Objetivo Geral}
%
%Caracterizar o software acadêmico de análise estática com respeito à sua
%sustentatibilidade técnica.
%A caracterização será feita em um conjunto de software acadêmico de análise
%estática, com base em medidas para avaliar sua sustentabilidade técnica.

\subsection{Objetivos Específicos}

São objetivos específicos deste trabalho:

% Objetivo específico: caracterizar o tipo de citação / menção d
% Reflexão sobre os problemas de sustentabilidade sofridos pelo ecossistema de software acadêmico de análise estática
% Formulação de hipóteses sobre os problemas de sustentabilidade sofridos pelo ecossistema de software acadêmico de análise estática

\begin{description}
  \item [01]
    Caracterizar o software acadêmico de análise estática com respeito à sua
    publicização.
    A caracterização será feita em um conjunto de software acadêmico de análise
    estática, com base em medidas para avaliar a sua disponibilidade de
    download, código fonte, e presença oficial online.
  \item [02]
    Caracterizar o software acadêmico de análise estática com respeito ao
    reconhecimento na literatura acadêmica.
    A caracterização será feita em um conjunto de software acadêmico de análise
    estática, com base em uma análise de trabalhos científicos que o utiliza ou
    adapta.
  \item [03]
    Caracterizar o software acadêmico de análise estática com respeito ao
    seu ciclo de vida.
    A caracterização será feita em um conjunto de software acadêmico de análise
    estática, com base nas informaçoes sobre lançamentos e métricas de código
    fonte.
\end{description}

%Estas questões darão importantes indícios sobre a qualidade do software acadêmico de
%análise estática desenvolvido na academia, especialmente sobre a capacidade de serem 
%encontrados, compartilhados e co-desenvolvidos.

\subsection{Questões de Pesquisa}

Neste estudo as seguintes questões de pesquisa, a respeito do software
acadêmico de análise estática, serão investigadas:

% Quão sustentável é o software acadêmico de análise estática?
\newcommand{\QuestaoUm}{
  Como evolui o reconhecimento ao software acadêmico de análise estática
  publicado nas conferências de Engenharia de Software ASE e SCAM?
  %Qual a taxa de crescimento no número de menções a software acadêmico de
  %análise estática publicado nas conferências de Engenharia de Software ASE e
  %SCAM?
}
\newcommand{\QuestaoDois}{
  A publicização do software acadêmico de análise estática publicado nas
  conferências de Engenharia de Software ASE e SCAM influencia o seu
  reconhecimento?
}
\newcommand{\QuestaoTres}{
  O ciclo de vida do software acadêmico de análise estática publicado nas
  conferências de Engenharia de Software ASE e SCAM influencia o seu
  reconhecimento?
  %Qual a idade média do software acadêmico de análise estática publicado nas conferências de
  %Engenharia de Software ASE e SCAM?
}
\newcommand{\QuestaoQuatro}{
  Qual o tamanho do software acadêmico de análise estática publicado nas
  conferências de Engenharia de Software ASE e SCAM?
  %e qual a relação com o seu estágio de evolução no ciclo de vida?
}
\newcommand{\QuestaoCinco}{
  Como evolui o tamanho do software acadêmico de análise estática publicado nas
  conferências de Engenharia de Software ASE e SCAM?
  %O software acadêmico de análise estática publicado nas conferências de
  %Engenharia de Software ASE e SCAM em estágio mais avançado de evolução
  %no ciclo de vida possui maior reconhecimento?
}
\newcommand{\QuestaoSeis}{
  É possível replicar ou reproduzir pesquisas que mencionam software
  acadêmico de análise estática publicado nas conferências de Engenharia de
  Software ASE e SCAM?
}
\newcommand{\QuestaoSete}{
  O software acadêmico de análise estática publicado nas conferências de
  Engenharia de Software ASE e SCAM é sustentável tecnicamente?
}
\newcommand{\QuestaoOito}{
  O software acadêmico de análise estática publicado nas conferências de
  Engenharia de Software ASE e SCAM é útil e maduro suficiente para ser
  utilizado em outras pesquisas?
}

\begin{description}
  \item [Q1] \QuestaoUm
  \item [Q2] \QuestaoDois
  \item [Q3] \QuestaoTres
  \item [Q4] \QuestaoQuatro
  \item [Q5] \QuestaoCinco
  \item [Q6] \QuestaoSeis
\end{description}

% * mais da metade desenvolvem seus próprios softwares
% * falta de visibilidade gera questionamentos sobre qualidade
% * falta de treinamento leva a produzir softwares sem qualidade
% * produtividade científica requer capacidade de replicação
% * capacidade de replicação depende de qualidade

%\begin{description}
%  \item Os projetos com presença oficial online são mais citados? São mais utilizados? Recebem mais contribuição?
%  \item Os artigos de projetos sustentáveis são mais lidos, mais citados?
%  \item Os projetos sustentáveis são usados em outras pesquisas?
%  \item Os projetos sustentáveis tem contribuidores além dos autores iniciais?
%  \item Os projetos sustentáveis tem mais contribuidores?
%  \item As datas das publicações com menções do tipo Contribui possuem relação com as datas de lançamentos do software?
%  \item Os projetos estão disponíveis para obtenção hoje? Os projetos incentivam ativamente a contribuição? Os espaços do projeto são abertos e transparentes? 
%  \item Como a visibilidade dos artefatos de software publicados nas conferências ASE e SCAM mudam ao longo do tempo? Como os artefatos de software acadêmico publicados nas conferências ASE e SCAM são mencionados na literatura acadêmica ao longo do tempo?
%  \item Há relaçao entre a publicização do software com o reconhecimento? Há menção 
%  \item [Q1:] \QuestaoUm
%  \item [Q2:] \QuestaoDois
%  \item [Q3:] \QuestaoTres
%\end{description}

\subsection{Métricas}

Para responder às questões de pesquisas, as seguintes métricas serão usadas:

\begin{enumerate}
  \item Métricas relacionadas a publicização do software (número de projetos
  disponível para download, disponível em código fonte, tipo de licença);

  \item Métricas relacionadas ao reconhecimento do software (número de
  citações, número de menções, número de usos e contribuições);

  \item Métricas relacionadas ao ciclo de vida do software (número total de
  lançamentos, data e número de versão de cada lançamento, variação no número
  de módulos do código fonte).
\end{enumerate}

\section{Estratégia de Pesquisa}

Este trabalho apresenta um estudo de caso exploratório ({\it exploratory case
study}) sobre a sustentabilidade técnica do software acadêmico de análise
estática, o estudo é conduzido através de uma estratégia de pesquisa de
trabalho de campo ({\it Field Studies}) e configuração natural ({\it Natural
Settings}) \cite{stol2015holistic}, com as seguintes características
principais:
%segundo o framework apresentado em \citeonline{stol2015holistic}, de

\begin{itemize}
  \item Com o foco num fenômeno, organização ou sistema em particular;
  \item Com um baixo nível de generalização e alto realismo do contexto;
  \item Sem intervenção do pesquisador no ambiente.
\end{itemize}

%A ``estratégia de pesquisa'' tem um impacto significativo sobre o que pode e
%não pode ser alcançado em um estudo em termos de aquisição de novos
%conhecimentos e uma compreensão mais profunda dos fenômenos investigados
%\cite{stol2015holistic}.

Foram realizados três estudos sobre o software acadêmico de análise estática em
termos de sua publicização, reconhecimento e ciclo de vida.

No primeiro estudo, sobre publicização, um conjunto de projetos de software publicados
nas conferências de Engenharia de Software ASE e SCAM foi selecionado. Os projetos foram
caracterizados em termos de disponibilidade de download, código fonte, forma de
distribuição e licença.

No segundo estudo, sobre reconhecimento, os projetos selecionados pelo primeiro estudo
foram caracterizados em relação ao
reconhecimento acadêmico em termos de menções encontradas em publicações nas
bases ACM e IEEE. As menções encontradas foram classificadas por tipo, e cada artigo
mencionando o software selecionado foi caracterizado segundo os tipos encontrados.

No terceiro estudo, sobre ciclo de vida, os projetos de software acadêmico foram caracterizados em
relação ao estágio de ciclo de vida em que se encontram, em termos de lançamentos, versões, disponibilidade de
download, disponibilidade de código fonte, e número de módulos
para cada versão com código fonte disponível.

\section{Organização do texto}

O capítulo \ref{fundamentacao} apresenta os fundamentos teóricos necessários
para a compreensão deste trabalho.

O capítulo \ref{estudo1} traz um estudo sobre a publicização de software
acadêmico de análise estática nas conferências de Engenharia de Software ASE e
SCAM.

O capítulo \ref{estudo2} descreve um estudo sobre o reconhecimento de software
acadêmico de análise estática em publicações nas bases ACM e IEEE.

O capítulo \ref{estudo3} apresenta um estudo sobre o estágio de evolução e o
ciclo de vida de software acadêmico de análise estática.

O capítulo \ref{discussao} discute os resultados em termos da sustentabilidade
técnica do software acadêmico de análise estática e traça
paralelos com trabalhos relacionados.

O capítulo \ref{conclusoes} apresenta as considerações finais da pesquisa e
aponta trabalhos futuros.

%No entanto, parece ser regra geral não testar ou não documentar o próprio
%software, pesquisadores geralmente não testam ou documentam seus softwares
%acadêmicos, a maioria também não sabe o quão confiável seu software é,
%ocasionando graves erros em conclusões centrais da literatura,
%gerando retrabalho nas mais diversas áreas da ciência \cite{merali2010computational},

%Isto gera um impacto negativo na visibilidade dos softwares acadêmicos e faz
%surgir questionamentos sobre a sua qualidade, não apenas técnica, mas também a
%capacidade de ser encontrado, compartilhado e co-desenvolvido, qualidades
%importantes para a evolução do próprio software, mas também extremamente útil
%para um uso eficiente dos limitados recursos da ciência \cite{howison2013incentives,
%katz2014transitive}.

%Mas, apesar da grande participação dos cientistas no desenvolvimento de
%software \cite{hettrick2014uk}, 

%O software utilizado na Ciência, ou {\it software acadêmico}
%\cite{allen2017engineering}, além de ser utilizado para coleta e análise de
%dados, pode, em algumas pesquisas, ser o principal resultado do processo científico.

%, carregando neste caso uma grande parte do conhecimento obtido durante a pesquisa.

%monografias, livros ou teses), incluindo desde protótipos escritos pelos
%próprios cientistas, a produtos completos desenvolvidos profissionalmente
%\cite{allen2017engineering}.

%A maior parte dos cientistas (90\%) no entretanto nunca tiveram treinamento
%algum de como escrever software de forma eficiente, faltam práticas básicas de
%desenvolvimento, como escrever código legível, revisão de código, controle de
%versão, testes unitários, entre outros, como resultado, dados são perdidos,
%análises levam mais tempo que o necessário e os pesquisadores não conseguem a
%eficiência que poderiam ter ao trabalhar com softwares acadêmicos
%\cite{wilson2017good}.

%Independente da finalidade, tamanho ou motivação, todo software tem o potencial
%de voltar a ser útil em outros momentos ou lugares, para o autor original,
%ou para pesquisadores enfrentando problemas semelhantes aos dos autores originais.
%Não é difícil imaginar que o problema enfrentado por um pesquisador pode, em
%algum momento, ser enfrentado por outros pesquisadores, criando assim uma ótima
%oportunidade de colaboração entre pesquisas e pesquisadores, sendo o software
%um excelente vetor de ajuda mútua em ambas as direções.

% papel pesquisador no ecossistema de soft academico
%
%Essas preocupações gerais sugerem um conjunto de questões específicas, com foco
%em padrões globais e padrões emergentes dentro do ecossistema, incluindo: Quais
%recursos foram destinados à produção de software? Quantos usuários ou
%comunidades de usuários têm projetos? Quais são os impactos científicos desse
%uso? Os números de usuários crescem? Os projetos possuem recursos e habilidades
%suficientes para gerenciar seu crescimento? Quais projetos possuem
%funcionalidades sobrepostas? Há quanto tempo os pedaços de software e projetos
%persistem? Nós desconectamos as comunidades de usuários e desenvolvedores? São
%componentes específicos, ou camadas de componentes, faltam? Que código
%geralmente é usado em conjunto; são os projetos e as pessoas que produzem esses
%componentes se comunicando adequadamente? Como podemos sustentar o software
%crítico?
%
%Aqui há uma clara tensão entre um desejo de flexibilidade e liberdade, ligado
%às expectativas de inovação científica e desejos de estruturas de autoridade e
%controle de coordenação. As questões de influência incluem: como os programas
%de financiamento e quais os requisitos em suas chamadas, resultaram em software
%amplamente utilizado e impacto científico substancial? Quais são as
%características dos campos que alcançaram maior coalescência? Quais jornais e
%conferências têm políticas exemplares? Como o trabalho de software é visto
%dentro das práticas de contratação e avaliação, como os casos de posse?
%
%\cite{howison2015understanding}

%Isto gera um impacto negativo na visibilidade dos softwares acadêmicos e faz
%surgir questionamentos sobre a sua qualidade, não apenas técnica, mas também a
%capacidade de ser encontrado, compartilhado e co-desenvolvido, qualidades
%importantes para a evolução do próprio software, mas também extremamente útil
%para um uso eficiente dos limitados recursos da ciência \cite{howison2013incentives,
%katz2014transitive}.

%A comunidade tem refletido sobre os problemas relacionados ao
%desenvolvimento, promoção e sustentabilidade desses softwares, e o
%impacto que tais problemas causam no meio científico \cite{allen2017engineering}.
