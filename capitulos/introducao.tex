\xchapter{Introdução}{}

%Modularity
%Modularity describes the logical partitioning of software into several parts, components, and modules.
%Software will be easy to understand and change when composed of independent modules.
%Simplicity
%Software simplicity to the extent that it lacks complexity in organization, language, and
%implementation techniques and reflects the use of singularity concepts and basic structures.
%sao dois faotes citados em TABLE-II: FACTORS REPORTED IN [21]
%21 = David E. Peercy, “A Software Maintainability Evaluation
%Methodology”, IEEE Transactions on Software Engineering,
%Vol. SE-7, NO. 4, July 1981.
%
%Complexity
%The complexity of a software affects its maintainability. It is supposed to reflect to a certain extent, the
%difficulty in comprehending or maintaining codes. é um dos fatores que afetam a manutenabilidade
%TABLE-III: FACTORS REPORTED IN [17]
%17 = Khairuddin Hashim and Elizabeth Key,” A Software
%Maintainability Attributes Model”, Malaysian Journal of
%Computer Science, Vol. 9 No. 2, December 1996, pp. 92-97
%
%Average
%Cyclomatic
%Complexity
%Average Cyclomatic Complexity can be expressed as the average of cyclomatic complexities of all
%modules.
%TABLE-IV: FACTORS REPORTED IN [18]
%18 = K.K. Aggarwal, Yogesh Singh, Pravin Chandra and
%Manimala Puri,” Measurement of Software Maintainability
%Using a Fuzzy Model”, Journal of Computer Sciences 1 (4):
%538-542, 2005 ISSN 1549-3636 © 2005 Science
%Publications
%
%
%Simplicity
%Software size and its complexity is a very critical issue. The code smell “DuplicateCode” is, perhaps,
%the worst [7]. It takes longer to identify a specific class when there are many classes. The presence of
%several classes that are almost empty is a sign of code that may possess low maintainability.
%TABLE-V: FACTORS REPORTED IN [12]
%12 = B. Anda, “Assessing Software System Maintainability using
%Structural Measures and Expert Assessments,” in IEEE
%International Conference Software Maintenance, 2007, pp.
%204–213.
%
%\cite{A Survey of Key Factors Affecting Software Maintainability}
%
%
%We will also
%investigate the maintainability measures taxonomy by
%Oman et al where 92 measures are listed and classified [24].
%[24] Oman, P., Hagemeister, J., and Ash, D., A Definition
%and Taxonomy for Software Maintainability, report
%SETL Report 91-08-TR, University of Idaho, 1991.
%
%\cite{Measurements of Software Maintainability}
%
%Taxonomia, metricas, etc ... sobre manutenabilidade de software.
%
%\cite{Metrics for Assessing a Software System’s Maintainability}
%
%Faz um experimento usando CBO LCOM e outras metricas como preditor de manutenabilidade...
%
%\cite{Predicting Maintainability with Object-Oriented Metrics - An Empirical Comparison}

Em diversas linhas de pesquisa da Computação, em especial, em Engenharia de
Software, é bastante comum que novos softwares sejam desenvolvidos durante
trabalhos de pesquisa, tais softwares tem sido chamados na literatura de {\it research tool}
\cite{Portillo12} ou {\it research-originated software} \cite{Kon2011}, e vêm
ganhando atenção por serem considerados artefatos importantes produzidos em tais
pesquisas \cite{Stodden2009}.

Diante desta importância e considerando que o trabalho de pesquisa em Engenharia de Software
é uma disciplina centrada em avaliar e validar métodos, técnicas, linguagens e
ferramentas, podemos considerar que {\it research tool} ou {\it
research-originated software}, chamados a partir daqui apenas de {\it software
científico}, também devem ser avaliados e validados com métodos
científicos adequados.

Não pode-se deixar de citar, no entanto, que estudos avaliando produtos de
software, seja ele {\it software científico} ou {\it software da indústria}, são bastante comuns, apenas
no domínio de aplicação de análise estática, por exemplo, encontramos diversos estudos
avaliando ferramentas de localização de bugs \cite{Rutar2004},
detecção de buffer overflow \cite{Kratkiewicz2005},
segurança \cite{Okun2007, Johns2011},
detecção de falhas e refatorações \cite{Wedyan2009},
detecção de vulnerabilidades \cite{Li2010, Ataide2014},
cálculo de métricas \cite{Alemerien2013}, e ainda, estudos
comparando ferramentas de análise estática com compiladores comuns \cite{Emanuelsson2008}
e estudos comparando ferramentas comerciais com ferramentas {\it open source} \cite{Al2010}.

Apesar dos inúmeros estudos avaliando softwares deste domínio,
poucos levam em consideração aspectos relevantes à sua
manutenabilidade, uma característica que indica o quão fácil é realizar
modificações em um sistema ou componente de software, compreender e avaliar os
fatores que influenciam esta característica é de fundamental importância já que
atividades de manutenção consomem boa parte do ciclo de vida de um software,
chegando a 75\% do tempo total de desenvolvimento \cite{aggarwal2002integrated, kumar2012survey}.

Um dos fatores que possivelmente influenciam a manutenabilidade de um sistema
ou componente de software é a sua complexidade, estudos mostram que quanto
maior a complexidade, maior é o esforço de manutenção \cite{Darcy2005}, em
especial, a complexidade estrutural, uma medida definida em termos de
acoplamento e coesão. Ainda, grande parte dos engenheiros de software assumem que uma
boa estrutura interna resulta em boa qualidade externa \cite{Fenton2014}.

Partindo deste fato e sabendo que o domínio de análise estática de código-fonte
tem carência de estudos sobre a compreensão e avaliação de sua manutenabilidade
\cite{Li2010}, expecialmente os {\it softwares científicos}, vemos como uma um
boa oportunidade de investigação compreender e explorar os aspectos que
influenciam na manutenabilidade de tais softwares, a partir da observação de
sua complexidade estrutural.

Ferramentas de análise estática de código-fonte tem se tornado mais e mais
comuns no ciclo de vida do desenvolvimento de software \cite{Novak2010}, a
variedade de categorias destas ferramentas é hoje enorme, elas podem ser
caracterizadas pelo tipo de tecnologia empregada, pela linguagem de programação
suportada, pelo formato de saída, ou ainda, pelo modo como a ferramenta é
integrada ao ambiente, entre outras.

Estas categorias potencialmente influenciam na medição da complexidade
estrutural dos softwares de análise estática de código-fonte, sabe-se que
fatores como linguagem de programação, domínio de aplicação ou o tamanho do
sistema influenciam nos valores das métricas de manutenabilidade
\cite{Zhang2013}. \citeonline{Zhang2013} realizaram um estudo exploratório com
mais de 300 softwares distintos, de 9 domínios de aplicação diferentes, e
forneceram evidências empíricas do impacto destes e de outros fatores na
distribuição dos valores das métricas de manutenabilidade, como acoplamento e
coesão por exemplo. 

% referencia sobre complexidade:
% Livro: Software Metrics, A Rigorous and Practical Approach [3rd 2015]
% 9.1.1 Structural Complexity Properties

Assim, definimos como objetivo geral deste trabalho compreender os
fatores, ou as características, que causam impacto na complexidade estrutural
das ferramentas de software de análise estática de código-fonte, com especial
atenção aos {\it softwares científicos} deste domínio.

E, como objetivo específico o seguinte:

\begin{enumerate}
  \item Caracterizar as ferramentas de análise estática.
  \item Medir a complexidade estrutural das ferramentas de análise estática.
  \item Compreender a relação entre as características e a complexidade estrutural
        das ferramentas de análise estática.
\end{enumerate}

\section{Metodologia de trabalho}

Nesta dissertação, foi conduzida uma investigado empírica das características
das ferramentas de análise estática e seu impacto na complexidade estrutural.
Os objetos de estudo foram ferramentas de análise estática de código-fonte,
ferramentas mais abrangentes do que apenas análise estática também foram
incluídas.  Estas ferramentas tiveram sua medida de complexidade estrutural
calculadas automaticamente pela ferramenta
Analizo\footnote{\url{http://www.analizo.org}}, um conjunto de ferramentas para
análise estática e cálculo de métricas de código-fonte.

\section{Resultados}

(pendente)

\section{Organização do texto}

(pendente)

% aproveitar perte destas referencias ao justificar o uso de percentis ao inves de média
%
%Observar métricas de código-fonte em nível de projetos de software leva
%ao seguinte desafio: como obter valores de métricas que representem todo o projeto sendo
%que métricas de código-fonte usualmente são calculadas para cada elemento do sistema, como arquivos ou classes?
%Este desafio tem sido amplamente discutido em estudos sobre definição de
%intervalos de referência ({\it thresholds}) para métricas de
%código-fonte \cite{Shatnawi2010, Kaur2013, Herbold2011}. Intervalos de
%referência são valores conhecidos para uma dada medida
%\cite[Chapter~2.1]{Lanza2007} com algum valor semantico, por exemplo, se
%medirmos a altura das pessoas e definirmos até 2 metros como alto, então
%pessoas acima de 2 metros serão classificadas como muito altas.
%
%Intervalos de referência podem ser definidos de diversas formas, desde
%abordagens baseadas em modelos estatísticos \cite{Shatnawi2010, Kaur2013}
%até aprendizado de máquina \cite{Herbold2011} e inteligência artificial.
%Entre as inúmeras abordagens, muitas partem de estudos empíricos
%usando softwares da indústria como objeto de estudo, geralmente com
%softwares de domínios específicos, parte-se da coleta de dados de
%métricas de código-fonte e com uso de uma abordagem, ou uma combinação entre
%elas, chega-se aos intervalos.
%
%Estes intervalos são também continuamente avaliados a fim de saber se são
%válidos ou não, as abordagens utilizadas para calcular os intervalos levam em
%consideração inúmeros aspectos na tentativa de validar os valores encontrados,
%como por exemplo a natureza dos dados, se seguem a lei de distribuição de
%potência
%\cite{Wheeldon2003,Potanin2005,Concas2007,Ferreira2009,Yao2009,Clauset2009} ou
%seguem uma distribuição normal
%\cite{Baxter2006,Lanza2007,Herraiz2011,Herraiz2012}, avaliam ainda se possuem
%cauda longa, se são livre de escala, entre outros aspectos.
