\xchapter{Introdução}{}

Em diversas linhas de pesquisa da Computação, em especial, em Engenharia de
Software, é bastante comum que novos softwares sejam desenvolvidos durante
trabalhos de pesquisa, tais softwares tem sido chamados na literatura de {\it
research tool} \cite{Portillo12} ou {\it research-originated software}
\cite{Kon2011}, e vêm ganhando atenção por ser artefatos importantes para a
divulgação do conhecimento e para replicação dos resultados de pesquisas
\cite{Stodden2009}.

Avaliar estes softwares, chamados a partir daqui apenas de {\it software
científico}, deve ser uma preocupação da Engenharia de Software, uma disciplina
centrada em avaliar e validar métodos, técnicas, linguagens e ferramentas.
Estudos avaliando produtos de software são bastante comuns, apenas no domínio
de aplicação de análise estática, por exemplo, encontra-se estudos
avaliando desde ferramentas de localização de bugs \cite{Rutar2004}, detecção
de buffer overflow \cite{Kratkiewicz2005}, segurança \cite{Okun2007,
Johns2011}, detecção de falhas e refatorações \cite{Wedyan2009}, detecção de
vulnerabilidades \cite{Li2010, Ataide2014}, cálculo de métricas
\cite{Alemerien2013}, até estudos comparando ferramentas de análise estática
com compiladores comuns \cite{Emanuelsson2008} e estudos comparando ferramentas
comerciais com ferramentas {\it open source} \cite{Al2010}.

Apesar dos inúmeros trabalhos avaliando produtos de software, poucos realizam
estudos avaliando {\it softwares científicos}, algo de extrema importância para
compreender como tais artefatos estão contribuindo para a divulgação do
conhecimento. A área de pesquisa em análise de código-fonte tem uma longa e
respeitável tradição, ferramentas de análise estática de código-fonte tem sido
continuamente desenvolvidas, e apesar da rápida e constante evolução, existe
carência de estudos sobre a compreensão e avaliação das ferramentas
desenvolvidas \cite{Li2010}.

Assim, vemos como boa oportunidade de investigação (a) explorar como estes
softwares são publicados e (b) avaliar atributos da sua qualidade interna,
como complexidade por exemplo.

A complexidade do
sistema é uma característica bastante referenciada na literatura como com
importante indicador de qualidade, sendo uma característica que influencia diretamente na
manutenabilidade, uma característica que indica o quão fácil é realizar
modificações em um sistema ou componente de software. Estudos mostram que
quanto maior a complexidade, maior é o esforço de manutenção
\cite{hashim1996software, Darcy2005}. Em especial a complexidade estrutural,
uma medida definida em termos de acoplamento e coesão. Além do mais, grande
parte dos engenheiros de software assumem que uma boa estrutura interna resulta
em boa qualidade externa \cite{Fenton2014}.

%manutenabilidade,  compreender e avaliar os fatores que influenciam esta característica
%é de fundamental importância já que atividades de manutenção consomem boa parte
%do ciclo de vida de um software, chegando a 75\% do tempo total de
%desenvolvimento \cite{aggarwal2002integrated, kumar2012survey}.
%Um dos fatores que possivelmente influenciam a manutenabilidade de um sistema
%ou componente de software é a sua complexidade, 

Ferramentas de análise estática de código-fonte tem se tornado mais e mais
comuns no ciclo de vida do desenvolvimento de software \cite{Novak2010}, a
importância destas ferramentas no ciclo de desenvolvimento de softwares tem
crescido ao longo do tempo, a variedade de categorias destas ferramentas é hoje
enorme, elas podem ser caracterizadas pelo tipo de tecnologia empregada, pela
linguagem de programação suportada, pelo formato de saída, ou ainda, pelo modo
como a ferramenta é integrada ao ambiente, entre outras.

Estas categorias potencialmente influenciam na medição da complexidade
estrutural dos softwares de análise estática de código-fonte, sabe-se que
fatores como linguagem de programação, domínio de aplicação ou o tamanho do
sistema influenciam nos valores das métricas de manutenabilidade
\cite{Zhang2013}. \citeonline{Zhang2013} realizaram um estudo exploratório com
mais de 300 softwares distintos, de 9 domínios de aplicação diferentes, e
forneceram evidências empíricas do impacto destes e de outros fatores na
distribuição dos valores das métricas de manutenabilidade, como acoplamento e
coesão por exemplo. 

Assim, definimos como objetivo geral deste trabalho compreender os
fatores, ou as características, que causam impacto na complexidade estrutural
das ferramentas de software de análise estática de código-fonte, com especial
atenção aos {\it softwares científicos} deste domínio.

E, como objetivo específico o seguinte:

\begin{enumerate}
  \item Caracterizar as ferramentas de análise estática.
  \item Medir a complexidade estrutural das ferramentas de análise estática.
  \item Compreender a relação entre as características e a complexidade estrutural
        das ferramentas de análise estática.
\end{enumerate}

\section{Metodologia de trabalho}

Nesta dissertação, foi conduzida uma investigado empírica das características
das ferramentas de análise estática e seu impacto na complexidade estrutural.
Os objetos de estudo foram ferramentas de análise estática de código-fonte,
ferramentas mais abrangentes do que apenas análise estática também foram
incluídas.  Estas ferramentas tiveram sua medida de complexidade estrutural
calculadas automaticamente pela ferramenta
Analizo\footnote{\url{http://www.analizo.org}}, um conjunto de ferramentas para
análise estática e cálculo de métricas de código-fonte.

\section{Resultados}

(pendente)

\section{Organização do texto}

(pendente)
