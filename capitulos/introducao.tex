\xchapter{Introdução}{}

Em diversas linhas de pesquisa da computação, em especial, em engenharia de
software, é bastante comum que novos softwares sejam desenvolvidos durante
trabalhos de pesquisa, tais softwares tem sido chamados na literatura de {\it
research-originated software} \cite{Kon2011}, {\it research tool}
\cite{Portillo12} ou {\it academic software} \cite{allen2017engineering}, e vêm
ganhando atenção da comunidade devido ao papel que ocupam na reprodutibilidade
de seus estudos \cite{Peng2011}.

Isto tem feito a comunidade refletir sobre os problemas relacionados ao
desenvolvimento, promoção e sustentabilidade dos softwares acadêmicos, e o
impacto que eles causam no meio científico \cite{allen2017engineering}. Esta
reflexão tem mostrado, por exemplo, que muitos estudos em engenharia de
software sofrem de dificuldades de repetição \cite{Tang2016}, e apontam
problemas específicos relacionados à manutenabilidade e a sustentabilidade técnica
dos softwares acadêmicos.

Manutenabilidade indica o quão fácil é realizar atividades de evolução e
manutenção em softwares, uma característica importante aos pesquisadores
interessados em adaptar softwares acadêmicos, uma atividade muitas vezes
necessária ao reproduzir estudos. Sustentabilidade técnica diz respeito a
longevidade dos softwares, ou seja, a capacidade de continuar disponível no
futuro. Muitos pesquisadores não disponibilizam os seus softwares
\cite{robles2010replicating, amann2015software} ou quanto o fazem enfrentam
problemas com disponibilidade e manutenabilidade \cite{Prlic2012}, isto leva a
um corpo computacional extramente difícil de reproduzir uma vez que mais da
metade dos pesquisadores desenvolvem seus prórios softwares
\cite{hettrick_2014_14809}, e fere um dos fundamentos da ciência de que novas
descobertas sejam reproduzidas antes de serem consideradas parte da base de
conhecimento \cite{Stodden2009}.

Isto tem motivado conferências específicas sobre o tema, como o
RSE\footnote{Conference of Research Software Engineers
\url{http://rse.ac.uk/conf2017}}, WSSSPE\footnote{Workshop on Sustainable
Software for Science: Practice and Experiences
\url{http://wssspe.researchcomputing.org.uk}} e o RESER\footnote{Workshop on
Replication in Empirical Software Engineering Research
\url{http://sequoia.cs.byu.edu/reser}}, e tem contribuido para a compreensão
dos problemas relacionados aos softwares acadêmicos e o impacto que eles causam
na reprodutibilidade dos seus estudos.

Mas apesar desta crescente preocupação com os softwares acadêmicos
ainda sabe-se pouco sobre o quanto a sustentabilidade técnica
e a manutenabilidade impactam na reprodutibilidade de seus estudos, sobretudo
em áreas específicas, como a análise estática de software, uma área com uma
longa e respeitável tradição e que ainda sofre carência de estudos sobre
avaliação e validação de seus softwares \cite{Li2010, ilyas2016static}.

% (2) Tools to support systematic literature reviews in software engineering: A mapping study \cite{marshall2013tools}
%
% Cita um mapeamento feito sobre estudos que criam ferramentas para apoio a
% revisão sistemática no domínio de SE, 14 estudos foram selecionados, ao final
% apenas 8 tinham proposta de ferramentas, ao final conclui que as ferramentas
% encontradas estão em estado inicial de desenvolvimento. 
%
% (3) Tools used in Global Software Engineering: A systematic mapping review \cite{Portillo12}
%
% Cita um mapeamento sistemático com objetivo de encontrar ferramentas de
% comunicação e coordenação para suporte a times altamente distribuidos
% gograficamente, encontrou 132 ferramentas, para uso em projetos de software
% global. A maioria destas ferramentas foram desenvolvidas em centros de
% pesquisas, e apenas uma pequena porcentagem (18.9\%) foram testados fora do
% seu contexto onde foi desenvolvido.
%
% (5) Tools in mining software repositories \cite{chaturvedi2013tools}
%
% Faz uma revisão dos papers submetidos ao MSR desde 2007 até 2013 (?) e
% identifica data sets, ferramentas e técnicas utilizadas pelos autores, mais
% da metade dos papers usam ou criam ferramentas, categoriza as ferramentas em
% ferramentas novas, ferramentas tradicionais, protótipos e scripts para
% mineração de dados
%
% (6) A systematic literature review of software product line management tools \cite{pereira2015systematic}
%
% (???)
%
% (7) Software configuration management tools \cite{chan1997software}
%
% (???)
%
% (8) Comparison and evaluation of source code mining tools and techniques: A qualitative approach \cite{khatoon2013comparison}
%
% Lista ferramentas e técnicas para mineração de dados, estado da arte.
%
% (9) An overview of free software tools for general data mining \cite{jovic2014overview}
%
% Descreve característica dos 6 softwares livres mais usados para mineração de
% dados no geral.
%
% (10) Analyzing the State of Static Analysis: A Large-Scale Evaluation in Open Source Software \cite{beller2016analyzing}
%
% faz um estudo mostrando que analise estatica tem uma certa adocao em projetos livres
% e mostra onde pode-se melhorar nas ferramentas para aumentar a adoção
%
% Taming the Static Analysis Beast
% \cite{toman2017taming}
% Despite advances in tooling and mainstream success, static analysis development is still a
% painful process.

Dessa forma, definimos como objetivo geral deste trabalho avaliar o quanto a
sustentabilidade técnica e a manutenabilidade dos softwares acadêmicos de
análise estática impactam na reprodutibilidade de seus estudos.

São objetivos específicos deste trabalho:

\begin{enumerate}
  \item Medir e avaliar a sustentabilidade técnica dos softwares acadêmicos de
        análise estática.
  \item Medir e avaliar a manutenabilidade dos softwares acadêmicos de análise
        estática.
  \item Compreender o quanto a sustentabilidade técnica e a manutenabilidade
        dos softwares acadêmicos de análise estática impactam na
        reprodutibilidade de seus estudos.
\end{enumerate}

\section{Metodologia de trabalho}

\begin{enumerate}
  \item Revisão estruturada para seleção de artigos com publicação de software
        acadêmico de análise estática.
  \item Ler artigos selecionados na revisão estruturada e coletar informações sobre
        os softwares acadêmicos de análise estática.
  \item Medir e quantificar a sustentabilidade técnica, ou, a disponibilidade, dos
        softwares acadêmicos de análise estática.
  \item Obter o código fonte dos softwares acadêmicos de análise estática.
  \item Extrair e coletar métricas de código fonte dos softwares acadêmicos de
        análise estática.
  \item Medir a manutenabilidade dos softwares acadêmicos de análise estática
        a partir de suas métricas de código fonte.
  \item Interpretar o impacto da disponibilidade e da manutenabilidade dos softwares
        acadêmicos de análise estática na reprodutibilidade de seus estudos.
\end{enumerate}

\section{Contribuições}

(pendente)

\section{Organização do texto}

O capítulo \ref{fundamentacao} apresenta os fundamentos teóricos necessários
para a compreensão deste trabalho.

O capítulo \ref{caracterizacao-ferramentas} traz um estudo sobre a
sustentabilidade técnica e a disponibilidade dos softwares acadêmicos de
análise estática.

O capítulo \ref{complexidade-ferramentas} descreve um estudo sobre a
manutenabilidade dos softwares acadêmicos de análise estática.

O capítulo \ref{conclusoes} apresenta as considerações finais e discute os
resultados deste trabalho.
