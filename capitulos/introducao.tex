\xchapter{Introdução}{}

Em diversas linhas de pesquisa da Computação, em especial, em Engenharia de
Software, é bastante comum que novos softwares sejam desenvolvidos durante
trabalhos de pesquisa, tais softwares tem sido chamados na literatura de {\it research tool}
\cite{Portillo12} ou {\it research-originated software} \cite{Kon2011}, e vêm
ganhando atenção por serem considerados artefatos importantes produzidos em tais
pesquisas \cite{Stodden2009}.

Diante desta importância e considerando que o trabalho de pesquisa em Engenharia de Software
é uma disciplina centrada em avaliar e validar métodos, técnicas, linguagens e
ferramentas, podemos considerar que {\it research tool} ou {\it
research-originated software}, chamados a partir daqui apenas de {\it software
científico}, também devem ser avaliados e validados com métodos
científicos adequados.

Não pode-se deixar de citar, no entanto, que estudos avaliando produtos de
software, seja ele {\it software científico} ou {\it software da indústria}, são bastante comuns, apenas
no domínio de aplicação de análise estática, por exemplo, encontramos diversos estudos
avaliando ferramentas de localização de bugs \cite{Rutar2004},
detecção de buffer overflow \cite{Kratkiewicz2005},
segurança \cite{Okun2007, Johns2011},
detecção de falhas e refatorações \cite{Wedyan2009},
detecção de vulnerabilidades \cite{Li2010, Ataide2014},
cálculo de métricas \cite{Alemerien2013}, e ainda, estudos
comparando ferramentas de análise estática com compiladores comuns \cite{Emanuelsson2008}
e estudos comparando ferramentas comerciais com ferramentas {\it open source} \cite{Al2010}.

Apesar dos inúmeros estudos avaliando softwares deste domínio,
poucos levam em consideração aspectos relevantes à sua
manutenabilidade, uma característica que indica o quão fácil é realizar
modificações em um sistema ou componente de software, compreender e avaliar os
fatores que influenciam esta característica é de fundamental importância já que
atividades de manutenção consomem boa parte do ciclo de vida de um software,
chegando a 75\% do tempo total de desenvolvimento \cite{aggarwal2002integrated, kumar2012survey}.

Um dos fatores que possivelmente influenciam a manutenabilidade de um sistema
ou componente de software é a sua complexidade, estudos mostram que quanto
maior a complexidade, maior é o esforço de manutenção \cite{hashim1996software, Darcy2005}, em
especial, a complexidade estrutural, uma medida definida em termos de
acoplamento e coesão. Ainda, grande parte dos engenheiros de software assumem que uma
boa estrutura interna resulta em boa qualidade externa \cite{Fenton2014}.

Partindo deste fato e sabendo que o domínio de análise estática de código-fonte
tem carência de estudos sobre a compreensão e avaliação de sua manutenabilidade
\cite{Li2010}, expecialmente os {\it softwares científicos}, vemos como uma um
boa oportunidade de investigação compreender e explorar os aspectos que
influenciam na manutenabilidade de tais softwares, a partir da observação de
sua complexidade estrutural.

Ferramentas de análise estática de código-fonte tem se tornado mais e mais
comuns no ciclo de vida do desenvolvimento de software \cite{Novak2010}, a
variedade de categorias destas ferramentas é hoje enorme, elas podem ser
caracterizadas pelo tipo de tecnologia empregada, pela linguagem de programação
suportada, pelo formato de saída, ou ainda, pelo modo como a ferramenta é
integrada ao ambiente, entre outras.

Estas categorias potencialmente influenciam na medição da complexidade
estrutural dos softwares de análise estática de código-fonte, sabe-se que
fatores como linguagem de programação, domínio de aplicação ou o tamanho do
sistema influenciam nos valores das métricas de manutenabilidade
\cite{Zhang2013}. \citeonline{Zhang2013} realizaram um estudo exploratório com
mais de 300 softwares distintos, de 9 domínios de aplicação diferentes, e
forneceram evidências empíricas do impacto destes e de outros fatores na
distribuição dos valores das métricas de manutenabilidade, como acoplamento e
coesão por exemplo. 

Assim, definimos como objetivo geral deste trabalho compreender os
fatores, ou as características, que causam impacto na complexidade estrutural
das ferramentas de software de análise estática de código-fonte, com especial
atenção aos {\it softwares científicos} deste domínio.

E, como objetivo específico o seguinte:

\begin{enumerate}
  \item Caracterizar as ferramentas de análise estática.
  \item Medir a complexidade estrutural das ferramentas de análise estática.
  \item Compreender a relação entre as características e a complexidade estrutural
        das ferramentas de análise estática.
\end{enumerate}

\section{Metodologia de trabalho}

Nesta dissertação, foi conduzida uma investigado empírica das características
das ferramentas de análise estática e seu impacto na complexidade estrutural.
Os objetos de estudo foram ferramentas de análise estática de código-fonte,
ferramentas mais abrangentes do que apenas análise estática também foram
incluídas.  Estas ferramentas tiveram sua medida de complexidade estrutural
calculadas automaticamente pela ferramenta
Analizo\footnote{\url{http://www.analizo.org}}, um conjunto de ferramentas para
análise estática e cálculo de métricas de código-fonte.

\section{Resultados}

(pendente)

\section{Organização do texto}

(pendente)
