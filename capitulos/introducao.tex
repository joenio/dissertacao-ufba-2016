\xchapter{Introdução}{}

\section{Apresentação}

Softwares acadêmicos\footnote{{\it research software}, {\it academic software},
{\it academic research software}} ocupam um papel central na ciência moderna,
eles são utilizados para coletar, processar ou analisar resultados em trabalhos
de pesquisa publicados na literatura acadêmica (seja em jornal, revista,
conferência, monografia, livro, dissertação ou tese), podem ser pequenos
scripts com poucas linhas de código, protótipos, ou mesmo produtos de software
completos que demonstram ou refletem os resultados de uma pesquisa.

%seu uso é crescente,

Cientistas de todas as áreas usam softwares em suas pesquisas, desde grupos
trabalhando exclusivamente com problemas computacionais até grupos em
laboratórios tradicionais ou em campo, todos fazem uso de softwares acadêmicos
de alguma forma. Isto põe os softwares numa posição de destaque no debate sobre
replicação, uma prática fundamental para tornar novas descobertas parte da base
de conhecimento científica \cite{Stodden2009}.  Replicação é também o método
mais comum e cientificamente produtivo de produzir conhecimento novo a partir
dos passos trilhados por outros cientistas \cite{king1995replication}.

%uma vez que, usualmente, ter acesso
%ao software, assim como aos dados, é requisito mínimo para tal.

Os softwares acadêmicos são tão importantes para a ciência quanto são os
telescópios ou tubos de ensaio, por exemplo \cite{wilson2014best}.  Mas apesar
disso não recebem o devido reconhecimento e muitas pesquisas nem ao menos
mencionam sua utilização \cite{howison2016software}, contradizendo as boas
práticas de qualquer projeto experimental, de ter {\it laboratory
notebooks}\footnote{\url{https://en.wikipedia.org/wiki/Lab_notebook}}, dados
organizados, passos documentados, e projeto estruturado para reprodutibilidade.


%estes softwares resolvem problemas comuns do cotidiano de pelo menos metade dos
%pesquisadores de todas as áreas do conhecimento, são considerados tão
%importantes para a ciência quanto qualquer outro aparato experimental, como por
%exemplo os telescópios ou tubos de ensaio, temos desde grupos trabalhando
%exclusivamente com problemas computacionais até grupos em laboratórios
%tradicionais ou em campo fazem uso de tais aparatos de software
%\cite{wilson2014best}.
%
%como qualquer outro aparato experimental, 
%
%, cientistas gastam mais tempo hoje utilizando e
%desenvolvendo softwares do que gastavam no passado,  \cite{wilson2014best}.
%
%podem ser desde pequenos scripts ou protótipos até softwares
%completos desenvolvidos profissionalmente.

%Cientistas gastam mais tempo hoje utilizando e desenvolvendo softwares do que
%gastavam no passado, softwares acadêmicos, assim como qualquer outro aparato
%experimental, são tão importantes para a ciência quanto são os telescópios ou
%tubos de ensaio, eles resolvem problemas comuns do cotidiano de metade dos
%pesquisadores de todas as áreas do conhecimento, desde grupos trabalhando
%exclusivamente com problemas computacionais até grupos em laboratórios
%tradicionais ou em campo \cite{wilson2014best}.
%
%Apesar disso, softwares acadêmicos ainda não recebem o devido reconhecimento,
%muitas pesquisas nem ao menos mencionam sua utilização, um estudo recente com
%90 artigos de diversas áreas da biologia, selecionados aleatoriamente entre
%publicações usando softwares como método, mostrou que apenas 59 mencionavam o
%uso de softwares de alguma forma, os demais 31 artigos, apesar de usar software
%acadêmico, não mencionavam nada a respeito \cite{howison2016software}.

Isto gera um impacto negativo na visibilidade dos softwares acadêmicos e faz
surgir questionamentos sobre a sua qualidade, não apenas técnica, mas também a
capacidade de ser encontrado, compartilhado e co-desenvolvido, qualidades
importantes para a evolução do próprio software, mas também extremamente útil
para um uso eficiente dos limitados recursos da ciência \cite{howison2013,
katz2014transitive}.

Este problema torna-se ainda mais aparente quando percebe-se que grande parte
destes softwares são desenvolvidos na própria academia
\cite{hettrick_2014_14809} e que pesquisadores geralmente não testam ou
documentam seus softwares acadêmicos, a maioria também não sabe o quão
confiável seu software é, somado a isto temos ainda o fato de que pesquisadores
raramente publicam seus códigos \cite{robles2010replicating,
amann2015software}.

%Especialmente quando nota-se que grande parte destes softwares são desenvolvidos na própria academia, um estudo
%entre cientistas do reino unido, por exemplo, mostrou que 56\% dos
%pesquisadores desenvolvem seus próprios softwares, pelo menos parcialmente
%\cite{hettrick_2014_14809}. Em outras áreas, como na astronomia, este número
%chega a 90\% \cite{momcheva2015software}, em ciência da computação,
%particularmente em engenharia de software, tem-se notado um aumento constante
%no número de novos softwares acadêmicos \cite{allen2017engineering}.

%No entanto, parece ser regra geral não testar ou não documentar o próprio
%software, pesquisadores geralmente não testam ou documentam seus softwares
%acadêmicos, a maioria também não sabe o quão confiável seu software é,

Como resultado, dados são perdidos, análises levam mais tempo que o necessário
e os pesquisadores não conseguem a eficiência que poderiam ter ao trabalhar com
softwares acadêmicos \cite{wilson2017good}.  Ocasionando graves erros em
conclusões centrais da literatura, gerando retrabalho nas mais diversas áreas
da ciência \cite{Merali2010Computational}.  Em ciência da computação,
particularmente em engenharia de software, tem-se notado um aumento constante
no número de novos softwares acadêmicos \cite{allen2017engineering}.

%Apesar de nem sempre ser possível, ou viável, ter tudo dentro de
%padrões estritos, é preciso estar consciente das boas práticas ao
%produzir e utilizar softwares acadêmicos, tanto para melhorar a própria
%abordagem quanto para revisar outros trabalhos \cite{wilson2014best}.
%
%A maior parte dos cientistas (90\%) no entretanto nunca tiveram treinamento
%algum de como escrever software de forma eficiente, faltam práticas básicas de
%desenvolvimento, como escrever código legível, revisão de código, controle de
%versão, testes unitários, entre outros, como resultado, dados são perdidos,
%análises levam mais tempo que o necessário e os pesquisadores não conseguem a
%eficiência que poderiam ter ao trabalhar com softwares acadêmicos
%\cite{wilson2017good}.
%
%Isto contradiz as boas práticas de qualquer projeto experimental ({\it
%laboratory
%notebooks}\footnote{\url{https://en.wikipedia.org/wiki/Lab_notebook}}, dados
%organizados, passos documentados, projeto estruturado para reprodutibilidade) e
%torna praticamente impossível utilizar o método mais comum e cientificamente
%produtivo de produzir conhecimento novo a partir de pesquisas anteriores, a
%replicação, ou seja, seguir os mesmos passos do autor original com objetivo de
%validar, melhorar ou estender seus dados e sua metodologia
%\cite{king1995replication, Stodden2010}.

%isto tem motivado a organização de conferências específicas
%para discutir os problemas dos softwares acadêmicos, como o RSE (Conference of
%Research Software Engineers)\footnote{ \url{http://rse.ac.uk/conf2017}}, WSSSPE
%(Workshop on Sustainable Software for Science: Practice and
%Experiences)\footnote{ \url{http://wssspe.researchcomputing.org.uk}} e o RESER
%(Workshop on Replication in Empirical Software Engineering Research)\footnote{
%\url{http://sequoia.cs.byu.edu/reser}}.
%
%mencionado em estudos sobre ecosistema de softwares acadêmicos é o 

Levando a um grave problema conhecido como ``dysfunctional chaotic churn'', ou
seja, muitos projetos, com poucos usuários, com ciclos de vida curtos, que terminam em
paralelo ao financiamento inicial, comunidades desconectadas e paralelas,
incompatibilidades entre projetos, e tentativas aparentemente não coordenadas
de "reiniciar" (re-boots) \cite{howison2015understanding}.

Este problema abranda o progresso da ciência uma vez que oportunidades de
colaboração são perdidas \cite{stewart2010cyberinfrastructure} e nos leva a
questionar como o problema ``dysfunctional chaotic churn'' se apresenta nas
pesquisas da engenharia de software, especialmente na análise estática de
programas, uma área com uma longa e respeitável tradição, e com um constante
crescimento no número de ferramentas publicadas ao longo do tempo \cite{Li2010,
ilyas2016static}.

Assim, neste trabalho temos como objetivo explorar o problema de
``dysfunctional chaotic churn'' entre os softwares acadêmicos de origem
científica para análise estática respondendo a seguinte questão geral:
Os projetos de software acadêmico de origem científica de análise estática
estão no caminho para sustentabilidade?

Podemos avaliar a saúde da comunidade
do projeto investigando as seguintes questões:

\begin{itemize}
  \item O projeto tem contribuidores além daqueles financiados diretamente?
  \item O projeto incentiva ativamente a contribuição e são oferecidas contribuições integradas?
  \item Os espaços do projeto são abertos e transparentes?
  \item Os projetos fazem um pedido explícito e claro de reconhecimento?
\end{itemize}

%o artigo abaixo faz um estudo e usa metrica para calcular o impacto
%das citacoes e mencoes ao software, usa um calculo baseado no numero
%de ocorrencias que o nome do software aparece
%Disciplinary differences of software use and impact in scientific literature
% eu fiz a mesma coisa mas numa avaliação qualitativa, meu peso é, 0 ou 1, menciona o software ou não menciona'
%se menciona, qual tipo, usa, contribui, etc... isto eh que vai dar o valor/peso/metrica

%artigo mostra o decaimento das URLs ao longo do tempo, fundamenta o assunto,
%mostra grafico com o caimento ao longo dos anos
%Use it or lose it: citations predict the continued online
%availability of published bioinformatics resources
%outro grafico muito foda do estudo acima é cruzar o efeito das citacoes
%por ano na availability rate (tenho dados suficiente para fazer um grafico igual)
%
%No entando, ainda não se sabe ao certo como os engenheiros de software
%desenvolvem e usam seus próprios softwares acadêmicos, um conhecimento
%necessário não apenas para tomar decisões sobre a necessidade de melhorias nas
%práticas atuais de desenvolvimento \cite{hannay2009scientists}, mas também
%imprescindível para compreender qual o impacto destes softwares na capacidade
%de replicação de seus estudos.
%
%e qual qualidade esses softwares apresentam ao longo do tempo.
%
%Dessa forma, definimos como objetivo geral deste trabalho explorar como são
%publicados os softwares acadêmicos de origem científica de análise estática do
%ponto de vista de alguém interessado em replicar os seus estudos, seja com o
%objetivo de avaliar, validar, refutar, melhorar ou extender os seus métodos ou
%dados.
%
%como essas taxas mudam ao longo do tempo.
%
%Dessa forma, definimos como objetivo geral deste trabalho medir a qualidade
%(técnica e não-técnica) dos softwares acadêmicos de origem científica da
%engenharia de software e explorar como essas taxas mudam ao longo do tempo.
%
%Questão de pesquisa:
%
%
%
%* Como ocorre o co-desenvolvimento dos softwares
%* Como acontece colaboração na construção dos softwares
%* Como os softwares contribuem para a construcao de conhecimento novo em novas pesquisas derivadas
%
% * mais da metade desenvolvem seus próprios softwares
% * falta de visibilidade gera questionamentos sobre qualidade
% * falta de treinamento leva a produzir softwares sem qualidade
% * produtividade científica requer capacidade de replicação
% * capacidade de replicação depende de qualidade
%
%Ainda assim, poucos estudos tem focado sua atenção nos softwares acadêmicos de
%origem científica\footnote{\it research-originated software \cite{Kon2011}},
%especialmente na engenharia de software, uma área com um potencial inato para a
%criação de novos softwares, como se pode notar em áreas como a análise estática
%de programas, uma área com uma longa e respeitável tradição, e com um constante
%crescimento no número de ferramentas publicadas ao longo do tempo \cite{Li2010,
%ilyas2016static}.
%
%Estes softwares, fazem parte do método empregado em suas pesquisas, ao mesmo
%tempo, são também artefatos produzidos como resultado pelos seus autores, e em
%muitos casos são também a contribuição ou resultado principal de uma
%determinada pesquisa. Além da já citada importancia dos softwares acadêmicos na
%capacidade de reproduzir o passo a passo do autor original numa descoberta, os
%softwares acadêmicos de origem científica publicado como resultado pelo seus
%autores possuem uma dupla importância neste cenário.
%
%Eles passam a fazer parte do conjunto de softwares acadêmicos disponíveis
%para a comunidade acadêmica, sendo
%
%+ software acadêmico: software para coleta e análise, ou resultado
%    + de origem científica
%        + resultado
%        + método
%Software is a critical part of modern research and yet there is little support across the
%scholarly ecosystem for its acknowledgement and citation. Inspired by the activities
%of the FORCE11 working group focused on data citation, this document
%summarizes the recommendations of the FORCE11 Software Citation Working
%Group and its activities between June 2015 and April 2016. Based on a review of
%existing community practices, the goal of the working group was to produce a
%consolidated set of citation principles that may encourage broad adoption of a
%consistent policy for software citation across disciplines and venues. Our work is
%presented here as a set of software citation principles, a discussion of the motivations
%for developing the principles, reviews of existing community practice, and a
%discussion of the requirements these principles would place upon different
%stakeholders. Working examples and possible technical solutions for how these
%principles can be implemented will be discussed in a separate paper.
%\cite{smith2016software}

%um caminho apontado como solução é acreditar que software deve evoluir para plataformas compartilhadas,
%com componentes reusáveis tanto quanto possível, tanto para usuário final, quanto
%para produtores de componentes (papel) agregando peças particulares de software,
%crença de que o software deve evoluir em direção a uma plataforma
%compartilhada, com componentes que são reutilizados o mais amplamente possível,
%já que os usuários finais e os produtores de componentes se agrupam em torno de
%peças específicas de software.
%
% artigo acima define ciberinfraestrutura, sustentabilidade, e faz um resumo
% ótimo de tudo que preciso, ler e referenciar aqui
%
%O ecosistema de software acadêmico é um sistema que consome tempo, dinheiro e
%atençao, e afeta a conduta e os resultados da ciência, tanto no geral, como em
%campos específicos
%
%chape para isto é acreditar que software deve evoluir para plataformas compartilhadas,
%com componentes reusáveis tanto quanto possível, tanto para usuário final, quanto
%para produtores de componentes (papel) agregando peças particulares de software
%
%A visão da ciberinfraestrutura, expressada no "Relatório Atkins" e instanciada
%para o ecossistema de software científico na NSF chamada de Infraestrutura de
%Software para Inovação Sustentada (NSF SI2),
%
%Os softwares vem não apenas atuando no avanço da ciência, mas atuando com uma
%eficiência crescente ao longo do tempo (Atkins 2003). A chave para isso é a
%crença de que o software deve evoluir em direção a uma plataforma
%compartilhada, com componentes que são reutilizados o mais amplamente possível,
%já que os usuários finais e os produtores de componentes se agrupam em torno de
%peças específicas de software.
%
%a literatura sobre plataformas de software fora da ciencia tem chamado isso de
%'coring' e 'tipping' (Gawer and Cusumano 2008),
%onde uma comunidade descobre sua funcionalidade compartilhada e se agrupa
%em pacotes que fornecem, levando ao uso eficiente de recursos através de
%economias de escala.
%
%coring também resulta em um aumento do uso sobreposto que facilita mais
%
%Isto também resulta em um aumento do uso sobreposto que facilita mais
%transparência na ciência, levando a uma maior qualidade e correctude (correctness), à medida
%que mais olhos e esforços são direcionados para os mesmos códigos que são
%sustentados e evoluem em longos períodos de utilidade científica.
%
%coring em direção às plataformas pode ser contrastado com o seu oposto, muitas
%vezes percebido por informantes: churn caótico disfuncional, com muitos
%projetos com poucos usuários, cada um tendo vidas curtas que terminam com o
%financiamento de concessão inicial, comunidades desconectadas e paralelas,
%incompatibilidades teimosamente imutáveis e periódicas e tentativas
%aparentemente não coordenadas de "reiniciar". Subjacente a isso é uma
%preocupação que as oportunidades são perdidas e que o progresso da ciência é
%abrandado (por exemplo, Stewart, Almes e Wheeler 2010).
%
%Essas preocupações gerais sugerem um conjunto de questões específicas, com foco
%em padrões globais e padrões emergentes dentro do ecossistema, incluindo: Quais
%recursos foram destinados à produção de software? Quantos usuários ou
%comunidades de usuários têm projetos? Quais são os impactos científicos desse
%uso? Os números de usuários crescem? Os projetos possuem recursos e habilidades
%suficientes para gerenciar seu crescimento? Quais projetos possuem
%funcionalidades sobrepostas? Há quanto tempo os pedaços de software e projetos
%persistem? Nós desconectamos as comunidades de usuários e desenvolvedores? São
%componentes específicos, ou camadas de componentes, faltam? Que código
%geralmente é usado em conjunto; são os projetos e as pessoas que produzem esses
%componentes se comunicando adequadamente? Como podemos sustentar o software
%crítico?
%
%Junto com estas questões estão as questões de como influenciar o ecossistema,
%incluindo questões de pontos de inflexão que levam ao uso coalescente, bem como
%a intervenções políticas diretas incentivando o uso de componentes específicos.
%Aqui há uma clara tensão entre um desejo de flexibilidade e liberdade, ligado
%às expectativas de inovação científica e desejos de estruturas de autoridade e
%controle de coordenação. As questões de influência incluem: como os programas
%de financiamento e quais os requisitos em suas chamadas, resultaram em software
%amplamente utilizado e impacto científico substancial? Quais são as
%características dos campos que alcançaram maior coalescência? Quais jornais e
%conferências têm políticas exemplares? Como o trabalho de software é visto
%dentro das práticas de contratação e avaliação, como os casos de posse?
%
%\cite{howison2015understanding}
%
% irei encontrar os problemas documentados e com propostas e evidencias
% de solução nos artigos abaixo, preciso ler para screver o parágrafo acima
%https://academic.oup.com/rev/article/24/4/454/1518466/Understanding-the-scientific-software-ecosystem
%artigo sobre ecosistema de software acadêmico, preciso citar! fala dos papeis,
%etc...
%
%data availability is one of
%the ‘critical issues for the evaluation of R&D programs and
%policies today’ (Rogers 2013: 2).
%
%In addition, the second and third authors each
%have over 15 years experience in the scientific software eco-
%system.
%
% fez exatamente o que pensei, vou ler para usar a metodologia "adaptada"
%In addition to these motivation studies, scientists have recently embarked on the issue of
%software use and impact. A study in 2013 has found that scientists tend to choose software
%that is widely used by others in their community and prefer software that is free for
%academic use (Huang et al. 2013). Studies on the scientific software ecosystem have
%suggested that the use of scientific software is influenced by its visibility, availability,
%sustainability, reproducibility, and citation (Howison and Herbsleb 2014; Howison et al.
%2015; Huang et al. 2013). Studies also have suggested that software developers are
%interested to know the use and impact of their software because ‘‘software use matters to
%them for funding purposes’’ (Howison et al. 2015; Trainer et al. 2015, p. 428).
%Recent studies on data impact have led to the discussions on software citation and
%evaluation, as a parallel can be drawn between software and data in scientific literature
%(Piwowar et al. 2011; Howison and Bullard 2016). It is suggested that the numbers of
%mentions and citations in literature can be used to measure the impact of software (Huang
%et al. 2013; Pan et al. 2015). Yet, it is argued that ‘‘the practices of citation to software vary
%considerably from field to field and appear to miss significant software’’ (Howison et al.
%2015, p. 478). One study examining the use of software in scientific articles in biology has
%found that more than half of the software mentions did not include references (Howison
%and Bullard 2016). Thus, it validates the need to use alternative metrics in addition to
%citations when assessing software impact, such as the numbers of downloads, registered
%users, subscribers, user reviews, and artifacts inserted in literature (Howison et al. 2015).
%
%, e tem agregado discussões das
%comunidades de ciência aberta, reprodutibilidade e sustentabilidade de
%software.
%
%Dessa forma, definimos como objetivo geral deste trabalho explorar como são

\section{Metodologia de trabalho}

(apresentar resumo do capítulo \ref{metodologia})

%Nesta dissertação, foi investigado o quanto a sustentabilidade dos softwares
%acadêmicos de análise estática impactam na reprodutibilidade dos seus estudos,
%selecionamos softwares acadêmicos de análise estática, medimos a
%sustentabilidade técnica e a manutenabilidade, e avaliamos o quanto essas medidas
%impactam na reprodutibilidade das pesquisas onde os softwares foram criados.
%
%A seleção de softwares acadêmicos foi realizada através de um procedimento
%inspirado na revisão e no mapeamento sistemático de literatura, chamado de
%revisão estruturada, composto de atividades para seleção e coleta de
%informações sobre softwares acadêmicos de análise estática, essa revisão
%avaliou o histórico de publicações de 25 anos da conferência ASE e 15 anos da
%conferência SCAM.
%
%As informações coletadas sobre cada software inclui nome, descrição e o
%endereço onde obter uma cópia, normalmente página web ou repositório de código
%fonte, esses endereços foram verificados para confirmar se os softwares estão,
%de fato, disponíveis.
%
%Os softwares disponíveis foram avaliados em relação à disponibilidade de código
%fonte e à licença utilizada, essas informações, e as demais coletadas até aqui,
%foram distribuídas cronologicamente, e interpretadas numa perspectiva histórica
%sobre a sustentabilidade técnica dos softwares acadêmicos de análise estática.
%
%No segundo estudo, os softwares com código fonte disponível foram avaliados em
%relação a sua manutenabilidade através da métrica de complexidade estrutural. A
%coleta dessa métrica para cada software foi realizada pelo Analizo, uma suíte
%de ferramentas para análise de código fonte, e está sendo considerado como um
%indicador de manutenabilidade.
%
%Um conjunto de softwares de análise estática da indústria foi incluído nesta
%etapa, todos os dados coletados para os softwares acadêmicos foram também
%coletados para este novo conjunto. Esses softwares foram então caracterizados em
%relação à frequencia de lançamentos, linguagem de programação e o tipo de
%entrada suportado.
%
%Todas estas características foram comparadas entre sí, por exemplo, softwares
%com maior frequencia de lançamentos, escritos na mesma linguagem de
%programação, apresentam maior complexidade estrutural? Eles são da academia ou
%da indústria? Softwares da indústria apresentam melhor manutenabilidade do que
%os softwares acadêmicos?
%
%Essas perguntas serão respondidas através de uma análise exploratória dos
%dados, essa análise apresenta também uma perspectiva evolutiva de alguns
%softwares, aqueles com maior frequencia de lançamentos foram selecionados para
%esta avaliação.
%
%(continua...)

\section{Contribuições}

(pendente)

\section{Organização do texto}

O capítulo \ref{fundamentacao} apresenta os fundamentos teóricos necessários
para a compreensão deste trabalho.

O capítulo \ref{metodologia} apresenta em detalhes o método utilizado nos
estudos.

O capítulo \ref{sustentabilidade-tecnica} traz um estudo sobre a
sustentabilidade técnica e a disponibilidade dos softwares acadêmicos de
análise estática.

O capítulo \ref{complexidade-ferramentas} descreve um estudo sobre a
manutenabilidade dos softwares acadêmicos de análise estática.

O capítulo \ref{conclusoes} apresenta as considerações finais e discute os
resultados deste trabalho.
