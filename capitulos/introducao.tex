\xchapter{Introdução}{}

% (paulo) Nesta seção manteremos a
%discussão nos argumentos do porquê monitorar métricas de código-fonte, bem como apresentaremos
%as métricas selecionadas (e implementadas) para nossas ferramentas estudos
%
%Qualquer que seja a metodologia de desenvolvimento, monitorar a qualidade do software é
%fundamental. Dentre as inúmeras características que fazem um bom software, várias delas podem
%ser percebidas no código-fonte, e algumas são exclusivas dele.
%
%Métricas de código-fonte foram propostas desde que os primeiros conceitos da engenharia de
%software surgiram. As métricas pioneiras foram rapidamente absorvidas pela indústria, que as usa
%com frequência. Métricas de complexidade e tamanho são as mais usuais, por exemplo LOC (linhas
%de código) e seus derivados (p.ex., erros/LOC, LOC/tempo). Essas métricas são reconhecidamente
%limitadas quando usadas isoladamente. Mas mesmo com o avanço da pesquisa em métricas desde
%então, a indústria continua usando as mesmas métricas simplistas e de forma isolada (Fenton e Neil,
%1999).
%
%Especialistas em métricas podem ser capazes de entender e usar métricas, mas todos desenvolve-
%dores deveriam saber como usá-las para monitorar e melhorar seu código. Facilitar o monitoramento
%pelos desenvolvedores em geral é uma das contribuições desta tese de doutorado ao mostrar (no
%Capítulo 4) como é possível monitorar e estudar um conjunto de métricas de código-fonte de proje-
%tos de software livre e acompanhar a sua evolução. Também, complementando isso, apresentamos
%neste contexto (no Capítulo 5) um conjunto de ferramentas (i.e., uma plataforma) que sistematiza a
%avaliação do código-fonte. Antes de chegarmos nesses pontos, apresentamos em detalhes o conjunto
%de métricas que nortearam os estudos nesta tese.
%
%Em um dos trabalhos mais reconhecidos da área de métricas de código-fonte, (Lanza e Marinescu,
%2006) coletaram diversas métricas de tamanho, complexidade e herança de 45 sistemas feitos em
%Java e 37 em C++. Seus valores são apresentados no livro de sua autoria, que sugere valores consi-
%derados como baixo, médio, alto e muito alto para cada métrica, obtidos através da média e desvio
%padrão dos dados coletados dos sistemas estudados.
%\cite{Lanza2007}

% (apresento parte da metodologia)
% apresentar parte da "antiga" metologia aqui
%
%* na introducao coloco questoes de pesquisa, etc, para alcancar os objetivos, usei esta metodologia
%* a outra parte da metodologia será em colcusão ou resultados

% Contexto. O contexto faz parte da motivação do trabalho.
\section{Contexto}
A análise estática de programas é a atividade de coletar informações de um
programa de software sem necessidade de execução, é uma atividade presente
especialmente nas primeiras etapas da compilação, realizando otimizações, ou
verificações de erros diversos, mas além dos compiladores esta atividade tem
se mostrado útil em diversas etapas do processo de desenvolvimento,
especialmente na observação de atributos de qualidade através do monitoramento
de aspectos qualidade, como por exemplo as métricas de código-fonte, isto pode
ser notado pelo uso crescente de ferramentas de análise estática em ...

% The Detailed Research Methodology which you intend to employ. 
% The methodology section should discuss what methods you are going to use in order 
% to address the research objectives of your dissertation. 
% You need to justify why the chosen methods were selected as the most appropriate 
% for your research, amongst the many alternative ones, given its specific objectives, 
% and constraints you may face in terms of access, time and so on. 
% Reference to general advantages and disadvantages of various methods and techniques 
% without specifying their relevance to your choice decision is unacceptable. 
% Remember to relate the methods back to the needs of your research question. 


% Problema. 
\section{Problema}
% Uma maneira de evidenciar a contribuição do trabalho é 
% definir bem o problema a ser resolvido. 
% Nesse caso, pode-se discutir: o problema em questão, a definição formal 
% do problema e sua importância, relevância, aplicações práticas.

A tecnologia de análise estática tem se desenvolvido rapidamente, mas a
comparação e avaliação de técnicas e ferramentas não tem acompanhado tal
velocidade \cite{Li2010}.

% Trabalhos Relacionados. 
\section{Trabalhos relacionados}
% Outra maneira de evidenciar a contribuição do trabalho 
% é discutir os trabalhos relacionados (resumidamente) ainda na introdução. 
% Esses trabalhos estão no mesmo contexto, não resolvem o problema ou 
% apresentam apenas soluções parciais. 
% Além disso, o trabalho atual pode ser a extensão ou continuação de um trabalho 
% anterior. Nesse caso, o trabalho original deve ser mencionado na introdução.

Diversos trabalhos avaliam ferramentas de análise estática de código-fonte, em
geral, que possuem um mesmo propósito (detecção de bugs, detecção de
vulnerabilidades, detecção de anomalias, etc.), sob a perspectiva de seus
usuários, e com foco em atributos de qualidade externa, tais como, desempenho,
precisão e cobertura de seus resultados. 

Entretanto, não foram encontrados estudos que avaliam ferramentas de análise
estática de código-fonte com foco em sua qualidade interna, considerando o
ponto de vista de desenvolvedores interessados não apenas em usar, mas também
em manter e evoluir tais ferramentas.

Em tal contexto, surge o problema de identificar, medir e analisar fatores
técnicos que influenciam na manutenção de uma ferramenta de análise estática. 

\citeonline{Meirelles2013} apresenta argumentos para se observar a qualidade
do software através das métricas de código-fonte, associa qualidade do
software à qualidade do código. Afirma que uma maneira objetiva de se observar
as características de um código-fonte é analisando os valores de suas métricas
e alerta para o pouco uso de métricas por parte dos desenvolvedores no ciclo
de desenvolvimento, ele indica que um dos motivos desta sub-utilização
é a falta de conhecimento de como coletar automaticamente
os valores das métricas, interpretar os seus resultados e os associar à
qualidade do código-fonte. Assim, desenvolveu uma abordagem para identificar
valores de métricas de forma a servirem como referência para projetos futuros,
onde analisa estatísticamente a correlação entre as métricas e define um
subconjunto reduzido que podem ser monitoradas ao longo do tempo e
ainda oferecer uma boa visão do projeto.

\citeonline{Terceiro2010} chamam a atenção para a importância de medir e
compreender aspectos de qualidade interna de software, uma vez que tais
questões impactam fortemente em atividades de evolução e manutenção. Eles
afirmam que quanto menor a qualidade de código-fonte, maior será o esforço de
mantê-lo. Uma das formas de medir qualidade interna é em função da
complexidade estrutural, uma medida que leva em conta a relação entre o
acoplamento e coesão dos módulos de um programa.

Complexidade estrutural representa um aspecto arquitetural importante e envolve
tanto a organização interna dos módulos quanto a relação entre eles. Uma alta
complexidade estrutural faz projetos de software mais difíceis de entender, e
por isto mesmo, mais difíceis de manter e evoluir.

Não podemos perder de vista, no entanto, que projetos diferentes são
influenciados por fatores diferentes, de forma que a observação dos aspectos
de qualidade interna devem ser observados e interpretados separadamente por
projeto \cite{Terceiro2012Understanding} ou ao menos por domínio de
aplicação \cite{Meirelles2013}.

O domínio de aplicação análise estática, por exemplo, carece de observação
detalhada sobre aspectos de qualidade interna de suas ferramentas, a
área tem se desenvolvido rapidamente com novos métodos, técnicas e ferramentas
para os mais diversos fins, no entanto a comparação e avaliação de técnicas e ferramentas
não tem acompanhado tal velocidade \cite{Li2010}, e, apesar de existirem
estudos avaliando ferramentas de análise estática de código-fonte, poucos
fazem isso do ponto de vista de vista de sua qualidade interna.

Por exemplo, \citeonline{Rutar2004} compara cinco ferramentas de localização
de bugs para Java e discute as técnicas utilizadas por cada uma e seu impacto
nos resultados. \citeonline{Kratkiewicz2005} avalia cinco ferramentas de
análise estática para determinar seus pontos fortes e fracos em detecção de
falhas de buffer overflow em código C. \citeonline{Okun2007} avalia os efeitos
de ferramentas de segurança com objetivo de identificar se estas ferramentas
melhoram de fato a segurança de programas. \citeonline{Emanuelsson2008}
avaliam três ferramentas de análise estática da indústria com o objetivo de
mapear funcionalidades significativas destas ferramentas, normalmente não
fornecidas por compiladores normais. \citeonline{Wedyan2009} avaliaram três
ferramentas para verificar a efetividade de detectar falhas e predizer
refatorações. \citeonline{Mantere2009} compararam três ferramentas de análise
estática de código-fonte em relação à performance e dão subsídios para apoiar
tomada de decisão sobre seleção de ferramentas de análise estática.
\citeonline{Al2010} avaliaram quatro ferramentas de análise estática em
relação à capacidade de detectar bugs em programas concorrentes em Java e
responde se ferramentas comerciais são melhores que ferramentas {\it open
source}. \citeonline{Li2010} comparam sete diferentes ferramentas de análise
estática com foco em detecção de vulnerabilidades, comparando suas
caracteristicas através de um experimento. \citeonline{Johns2011} avaliaram a
qualidade de ferramentas de análise estática de segurança a partir de uma
série de critérios. \citeonline{Alemerien2013} avaliaram duas ferramentas de
análise estática para cálculo de métricas com objetivo de entender as
diferenças nos resultados de cada uma. \citeonline{Ataide2014} analisa os
resultados gerados por três ferramentas de análise estática que podem ser
eficientemente usadas por programadores para remover vulnerabilidades comuns
em programas.

Dentre estes estudos, a grande maioria realiza avaliação ou comparação de
ferramentas de análise estática de código-fonte levando em conta aspectos e
características de qualidade externa, o que nos leva a entender que estudos
que joguem luz sobre aspectos da qualidade interna neste domínio de aplicação
são requeridos do ponto de vista principalmente dos desenvolvedores
interessados em atividades de manutenção e evolução.

\section{Relevância}
% ABAIXO: por que vale a pena tratar analisador estático
% como caso especial? Por que se concentrar nesse domínio?
% Por que abordagens independentes de domínio não atendem?

Estudar ferramentas de análise estática a fim de compreender seus atributos de
qualidade interna e entender quais características %arquiteturais 
explicam tais atributos é de fundamental importância do 
ponto de vista de usuários (gerentes, desenvolvedores) interessados
em adotar, usar, manter e evoluir ferramentas deste domínio de aplicação.
% sem afetar negativamente seus atributos de qualidade interna.

% compreensão, arquitetura;  falar de métricas de produto, valores de referência
A arquitetura de software tem papel importante na compreensão e evolução 
de [ ref ] ...
módulos e dependências entre eles ... coesão e acoplamento entre módulos ...
complexidade estrutural ...
 
Complexidade estrutural crescente em projetos de software pode resultar
no aumento no esforço necessário para atividades de manutenção.

A qualidade interna de ferramentas de análise estática pode ser 
por meio de ... métricas de ... 
Valores de referência são importantes ...
Em geral, valores de referência são encontrados para linguagens de programação.
E para domínios?

% ------------------

O domínio de aplicação de análise estática tem evoluído constantemente desde o
seu surgimento a mais de 40 anos, e apesar da sua evolução constante poucos
estudos demonstram a preocupação em avaliar ou comparar ferramentas de análise
estática de código-fonte em relação à seus atributos de qualidade interna, um
aspecto importante para guiar e auxiliar desenvolvedores em atividades de
evolução e manutenção de software.

Desta forma estudar e comparar tais ferramentas a fim de compreender seus
atributos de qualidade interna e entender quais características arquiteturais
explicam tais atributos é de fundamental importância do ponto de vista de
desenvolvedores interessados em melhorar a qualidade interna de suas
ferramentas.

\subsection{Questão de pesquisa}

Com este objetivo em mente, levantamos a seguinte questão de pesquisa:

\begin{enumerate}
  \item [{\bf Q1:}] {\em Como a complexidade estrutural pode ser interpretada
    e explicada para ferramentas de software do domínio de aplicação de
    análise estática de código-fonte?}
\end{enumerate}

% Aim, Main Research Goal
\section{Objetivo geral}

Diante disso, definimos como objetivo principal deste trabalho: compreender
as ferramentas de software para análise estática de código-fonte do ponto de
vista de sua manutenabilidade, a partir da observação e análise de sua complexidade
estrutural, discutindo quais características arquiteturais explicam seus
atributos de qualidade interna.

% Objectives
\section{Objetivos específicos}

São objetivos específicos deste trabalho:

\begin{itemize}
  \item Selecionar e obter código-fonte de ferramentas de análise estática
    desenvolvidas na academia, para coletar suas métricas de código-fonte.  A
    seleção terá como base o resultado de uma revisão estruturada feita a
    partir de artigos publicados em conferências relacionadas. 
  \item Selecionar e obter código-fonte de ferramentas de análise estática
    desenvolvidas pela indústria, para coletar suas métricas de código-fonte.
  \item Propor intervalos de referência para a observação parametrizada da
    qualidade interna das ferramentas de análise estática, a partir de suas
    métricas de código-fonte.
  \item Calcular a distância Euclidiana entre valores de referência e os
    valores das ferramentas estudadas.
\end{itemize}

\section{Hipóteses} \label{hipoteses}

Para responder a questão colocada acima, definimos as seguintes hipóteses:

\begin{enumerate}
  \item[{\bf H1:}] {\em É possível calcular valores de referência de métricas
    de código-fonte para ferramentas de análise estática a partir de um
    conjunto de softwares da academia e da indústria.}
  \item[{\bf H2:}] {\em Ferramentas de análise estática tendem a ter uma
    maior complexidade estrutural do que ferramentas de outros domínios de
    aplicação.}
  \item[{\bf H3:}] {\em Dentre as ferramentas de análise estática de
    código-fonte, aquelas desenvolvidas na indústria apresentam uma menor
    complexidade estrutural.}
\end{enumerate}

A hipótese {\bf H1} ({\em É possível calcular valores de referência de
métricas de código-fonte para ferramentas de análise estática a partir de um
conjunto de softwares da academia e da indústria}) será validada a partir da
análise das métricas calculadas para cada uma das ferramentas estudadas.  Esta
análise levará em consideração a caracterização das ferramentas
(Seção~\ref{caracterizacao-das-ferramentas}), em especial, em um subconjunto
de ferramentas com melhores valores de métricas.

A hipótese {\bf H2} ({\em Ferramentas de análise estática tendem a ter uma
maior complexidade estrutural do que ferramentas de outros domínios de
aplicação}) será validada a partir da comparação com os trabalhos relacionados
(Seção \ref{trabalhos-relacionados}) que realizaram estudos similares, com
cálculo e distribuição de métricas, mas que utilizam como objeto de estudo
conjuntos ferramentas de outros domínios de aplicação.

A hipótese {\bf H3} ({\em Dentre as ferramentas de análise estática de
código-fonte, aquelas desenvolvidas na indústria apresentam uma menor
complexidade estrutural}) será validada a partir do cálculo da distância das
métricas de cada ferramenta com os valores de referências encontrados neste
estudo (Seção \ref{distancia}).


\section{Contribuições esperadas}

Ao final deste trabalho, as seguintes contribuições científicas ({\bf CC}) e
tecnológicas ({\bf CT}) são esperadas:

\begin{enumerate}
  \item [{\bf CC1:}] Um conjunto de intervalos de referência da frequência dos
    valores de métricas de código-fonte para o domínio de aplicação de
    análise estática de código-fonte.
  \item [{\bf CC2:}] Definição de argumentos que expliquem a alta complexidade
    estrutural em ferramentas de análise estática de código-fonte.
  \item [{\bf CT1:}] Evolução de uma ferramenta de análise estática de
    código-fonte.
\end{enumerate}

Lembrando que neste trabalho serão utilizadas métricas de produto,
especificamente, métricas de código-fonte, que cobrem aspectos de tamanho,
complexidade e qualidade.

\section{Estrutura do texto} 

O capítulo \ref{fundamentacao} apresenta conceitos sobre análise estática e
métricas de código-fonte necessários para compreensão do trabalho. O capítulo
\ref{metodologia} apresenta trabalhos relacionados, hipóteses do estudo,
planejamento sobre a coleta e análise dos dados
e o cronograma do estudo. O capítulo \ref{conclusoes} traz a
discussão e interpretação dos dados e apresenta os próximos passos do estudo.

A seção \ref{trabalhos-relacionados} traz trabalhos relacionados à compreensão
e observação de atributos de qualidade interna de programas. A seção
\ref{hipoteses} detalha como as hipóteses serão testadas. A seção
\ref{planejamento} descreve o planejamento de estudo e os passos iniciais de
coleta de dados. A seção \ref{coleta} detalha a coleta de dados e a seção
\ref{analise} traz informações de como estes dados serão analisados e
interpretados.
