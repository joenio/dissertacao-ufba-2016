\xchapter{Introdução}{}

\section{Apresentação}

Software acadêmico\footnote{{\it research software}, {\it academic software},
{\it academic research software}} é todo software usado ou produzido para
coletar, processar ou analisar resultados em trabalhos publicados na literatura
acadêmica (seja em jornal, revista, conferência, monografia, livro, dissertação
ou tese), podem ser desde pequenos scripts ou protótipos até softwares
completos desenvolvidos profissionalmente.

Boa parte destes softwares são desenvolvidos na própria academia, um estudo
entre cientistas do reino unido mostrou, por exemplo, que 56\% desenvolvem seus
próprios softwares, pelo menos pacialmente \cite{hettrick_2014_14809}, em
outras áreas, como na astronomia, este número chega a 90\%
\cite{momcheva2015software}, em ciência da computação, particularmente em
engenharia de software, tem-se notado um aumento constante no número de novos
softwares acadêmicos \cite{allen2017engineering}.

Cientistas gastam mais tempo hoje utilizando e desenvolvendo software do que
gastavam no passado, softwares acadêmicos, assim como qualquer outro aparato
experimental, são tão importantes quanto são os telescópios e tubos de ensaio,
eles resolvem problemas comuns do cotidiano de metade dos pesquisadores de
todas as áreas do conhecimento, desde grupos trabalhando exclusivamente com
problemas computacionais até grupos em laboratórios tradicionais ou em campo
\cite{wilson2014best}.

Apesar disso, ainda há pouco reconhecimento aos softwares acadêmicos, muitas
pesquisas nem ao menos mencionam sua utilização, um estudo recente com 90
artigos de diversas áreas da biologia, selecionados aleatoriamente entre
publicações usando softwares acadêmicos, mostrou que apenas 59 mencionavam os
softwares de alguma forma, os demais 31 artigos, apesar de usar algum software
acadêmico, não mencionavam nada a respeito \cite{howison2016software}.

Isto gera impacto negativo na visibilidade dos softwares acadêmicos, e faz
surgir questionamentos sobre sua qualidade, não apenas técnica, mas também a
capacidade de ser encontrado, compartilhado e co-desenvolvido, qualidades
importantes para a evolução do próprio software, mas também extremamente úteis
para o uso eficiente dos limitados recursos da ciência \cite{howison2013,
katz2014transitive}.

No entanto, parece ser regra geral não testar ou não documentar o próprio
software, pesquisadores geralmente não testam ou documentam seus softwares
acadêmicos, a maioria também não sabe o quão confiável seu software é,
ocasionando sérios erros computacionais em conclusões centrais da literatura
acadêmica, gerando retrabalho para retratar tais erros nas mais diversas áreas
da ciência \cite{Merali2010Computational}. Sabe-se que nem sempre é possível,
ou viável, ter tudo exatamente dentro dos padrões, mas é preciso estar
consciente das boas práticas ao produzir e utilizar softwares acadêmicos, tanto
para melhorar a própria abordagem quanto para revisar trabalhos de outros
\cite{wilson2014best}.

A maior parte dos cientistas (90\%) no entretanto nunca tiveram treinamento
algum de como escrever software de forma eficiente, faltam práticas básicas de
desenvolvimento, como escrever código legível, revisão de código, controle de
versão, testes unitários, entre outros, como resultado, dados são perdidos,
análises levam mais tempo que o necessário e os pesquisadores não conseguem a
eficiência que poderiam ter ao trabalhar com softwares acadêmicos
\cite{wilson2017good}.

Isto contradiz as boas práticas de qualquer projeto experimental ({\it
laboratory
notebooks}\footnote{\url{https://en.wikipedia.org/wiki/Lab_notebook}}, dados
organizados, passos documentados, projeto estruturado para reprodutibilidade) e
torna praticamente impossível utilizar o método mais comum e cientificamente
produtivo de produzir conhecimento novo a partir de pesquisas anteriores, a
replicação, ou seja, seguir os mesmos passos do autor original com objetivo de
validar, melhorar ou estender seus dados e sua metodologia
\cite{king1995replication, Stodden2010}.

Somado a isto temos ainda o fato de que pesquisadores raramente publicam seus
códigos, piorando ainda mais toda a situação, isto tem motivado a organização
de conferências específicas para discutir os problemas dos softwares
acadêmicos, como o RSE (Conference of Research Software Engineers)\footnote{
\url{http://rse.ac.uk/conf2017}}, WSSSPE (Workshop on Sustainable Software for
Science: Practice and Experiences)\footnote{
\url{http://wssspe.researchcomputing.org.uk}} e o RESER (Workshop on
Replication in Empirical Software Engineering Research)\footnote{
\url{http://sequoia.cs.byu.edu/reser}}, e tem agregado discussões das
comunidades de ciência aberta, reprodutibilidade e sustentabilidade de
software.

Apesar disso, poucos estudos tem focado sua atenção aos softwares acadêmicos de
origem científica\footnote{\it research-originated software \cite{Kon2011}},
especialmente os da engenharia de software, uma área que, ao menos em teoria,
não sofre com a falta de treinamento prévio em práticas de desenvolvimento.

Dessa forma, definimos como objetivo geral deste trabalho explorar como os
softwares acadêmicos de origem científica publicados na literatura acadêmica da
engenharia de software estão visíveis hoje e como são mencionados ao longo do
tempo.

\section{Metodologia de trabalho}

Nesta dissertação, foi investigado o quanto a sustentabilidade dos softwares
acadêmicos de análise estática impactam na reprodutibilidade dos seus estudos,
selecionamos softwares acadêmicos de análise estática, medimos a
sustentabilidade técnica e a manutenabilidade, e avaliamos o quanto essas medidas
impactam na reprodutibilidade das pesquisas onde os softwares foram criados.

A seleção de softwares acadêmicos foi realizada através de um procedimento
inspirado na revisão e no mapeamento sistemático de literatura, chamado de
revisão estruturada, composto de atividades para seleção e coleta de
informações sobre softwares acadêmicos de análise estática, essa revisão
avaliou o histórico de publicações de 25 anos da conferência ASE e 15 anos da
conferência SCAM.

As informações coletadas sobre cada software inclui nome, descrição e o
endereço onde obter uma cópia, normalmente página web ou repositório de código
fonte, esses endereços foram verificados para confirmar se os softwares estão,
de fato, disponíveis.

Os softwares disponíveis foram avaliados em relação à disponibilidade de código
fonte e à licença utilizada, essas informações, e as demais coletadas até aqui,
foram distribuídas cronologicamente, e interpretadas numa perspectiva histórica
sobre a sustentabilidade técnica dos softwares acadêmicos de análise estática.

No segundo estudo, os softwares com código fonte disponível foram avaliados em
relação a sua manutenabilidade através da métrica de complexidade estrutural. A
coleta dessa métrica para cada software foi realizada pelo Analizo, uma suíte
de ferramentas para análise de código fonte, e está sendo considerado como um
indicador de manutenabilidade.

Um conjunto de softwares de análise estática da indústria foi incluído nesta
etapa, todos os dados coletados para os softwares acadêmicos foram também
coletados para este novo conjunto. Esses softwares foram então caracterizados em
relação à frequencia de lançamentos, linguagem de programação e o tipo de
entrada suportado.

Todas estas características foram comparadas entre sí, por exemplo, softwares
com maior frequencia de lançamentos, escritos na mesma linguagem de
programação, apresentam maior complexidade estrutural? Eles são da academia ou
da indústria? Softwares da indústria apresentam melhor manutenabilidade do que
os softwares acadêmicos?

Essas perguntas serão respondidas através de uma análise exploratória dos
dados, essa análise apresenta também uma perspectiva evolutiva de alguns
softwares, aqueles com maior frequencia de lançamentos foram selecionados para
esta avaliação.

(continua...)

\section{Contribuições}

(pendente)

\section{Organização do texto}

O capítulo \ref{fundamentacao} apresenta os fundamentos teóricos necessários
para a compreensão deste trabalho.

O capítulo \ref{sustentabilidade-tecnica} traz um estudo sobre a
sustentabilidade técnica e a disponibilidade dos softwares acadêmicos de
análise estática.

O capítulo \ref{complexidade-ferramentas} descreve um estudo sobre a
manutenabilidade dos softwares acadêmicos de análise estática.

O capítulo \ref{conclusoes} apresenta as considerações finais e discute os
resultados deste trabalho.
