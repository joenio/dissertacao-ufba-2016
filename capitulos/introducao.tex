\xchapter{Introdução}{}

Em diversas linhas de pesquisa da Computação, em especial, em Engenharia de
Software, é bastante comum que novos softwares sejam desenvolvidos durante
trabalhos de pesquisa, tais softwares tem sido chamados na literatura de {\it
research tool} \cite{Portillo12} ou {\it research-originated software}
\cite{Kon2011}, e vêm ganhando atenção por serem artefatos importantes para a
divulgação do conhecimento e para replicação dos resultados \cite{Stodden2009}.

Avaliar estes softwares, chamados a partir daqui apenas de {\it software
científico}, deve ser uma preocupação da Engenharia de Software, uma disciplina
centrada em avaliar e validar métodos, técnicas, linguagens e ferramentas.
Estudos avaliando softwares são bastante comuns, apenas no domínio
de aplicação de análise estática, por exemplo, é possível encontrar diversos estudos
avaliando ferramentas, desde ferramentas de localização de bugs \cite{Rutar2004}, detecção
de buffer overflow \cite{Kratkiewicz2005}, segurança \cite{Okun2007,
Johns2011}, detecção de falhas e refatorações \cite{Wedyan2009}, detecção de
vulnerabilidades \cite{Li2010, Ataide2014}, cálculo de métricas
\cite{Alemerien2013}, até estudos comparando ferramentas de análise estática
com compiladores comuns \cite{Emanuelsson2008} e estudos comparando ferramentas
comerciais com ferramentas {\it open source} \cite{Al2010}.

%Ferramentas de análise estática de código-fonte tem se tornado mais e mais
%comum no ciclo de vida do desenvolvimento de software \cite{Novak2010}, a
%importância destas ferramentas no ciclo de desenvolvimento de softwares tem
%crescido ao longo do tempo enorme.

Apesar dos inúmeros estudos avaliando softwares, poucos fazem isso com {\it
softwares científicos}, algo de extrema importância para compreender como estes
artefatos estão contribuindo para a divulgação do conhecimento e para
replicação dos resultados das pesquisas em engenharia de software. A área de
análise de código-fonte tem uma longa e respeitável tradição, ferramentas de
análise estática de código-fonte tem sido continuamente desenvolvidas, e apesar
da rápida e constante evolução da área, ainda há carência de estudos avaliando
estas ferramentas \cite{Li2010}, especialmente os {\it softwares científicos}.

Assim, avaliar estes softwares e explorar como são publicados pode jogar luz
sobre o quanto eles estão contribuindo para a divulgação do conhecimento em
pesquisas da engenharia de software e o quanto estão contribuindo para
proporcionar reprodução dos seus resultados. Avaliar os {\it softwares
científicos} do ponto de vista de sua qualidade pode também ajudar a
compreender quanta atenção é dada ao seu desenvolvimento, uma vez que
tradicionalmente os autores de {\it software científico} enfrentam problemas
com manutenabilidade e disponibilidade de tais softwares \cite{Prlic2012}.

Manutenabilidade é uma característica de qualidade externa que indica o quão
fácil é realizar atividades de evolução e manutenção em componentes de
software, ela pode ser medida através de características de qualidade interna,
uma vez que grande parte dos engenheiros de software assumem que uma boa
estrutura interna resulta em boa qualidade externa \cite{Fenton2014}. A
estrutura interna de um software pode ser avaliada através da sua complexidade,
uma característica bastante referenciada na literatura como um importante
indicador de qualidade, estudos mostram que quanto maior a complexidade, maior
é o esforço de manutenção \cite{hashim1996software, Darcy2005}, em especial a
complexidade estrutural, uma medida definida em termos de acoplamento e coesão
\cite{Terceiro2012}.

%Estas categorias potencialmente influenciam na medição da complexidade
%estrutural dos softwares de análise estática de código-fonte, sabe-se que
%fatores como linguagem de programação, domínio de aplicação ou o tamanho do
%sistema influenciam nos valores das métricas de manutenabilidade
%\cite{Zhang2013}. \citeonline{Zhang2013} realizaram um estudo exploratório com
%mais de 300 softwares distintos, de 9 domínios de aplicação diferentes, e
%forneceram evidências empíricas do impacto destes e de outros fatores na
%distribuição dos valores das métricas de manutenabilidade, como acoplamento e
%coesão por exemplo. 

Assim, definimos como objetivo geral deste trabalho caracterizar e avaliar os
{\it softwares científicos} para análise estática de código-fonte, a fim de
compreender como eles contribuem para divulgação e reprodução dos resultados de
pesquisas da área de engenharia de software.

São objetivos específicos deste trabalho:

\begin{enumerate}
  \item Caracterizar os {\it softwares científicos} de análise estática de código-fonte.
  \item Avaliar a qualidade dos {\it softwares científicos} a partir de sua complexidade estrutural.
  \item Compreender como os {\it software científicos} estão contribuindo para divulgação e reprodução dos resultados de pesquisas em engenharia de software.
\end{enumerate}

\section{Metodologia de trabalho}

(pendente)

%Nesta dissertação, foi conduzida uma investigado empírica das características
%das ferramentas de análise estática e seu impacto na complexidade estrutural.
%Os objetos de estudo foram ferramentas de análise estática de código-fonte,
%ferramentas mais abrangentes do que apenas análise estática também foram
%incluídas.  Estas ferramentas tiveram sua medida de complexidade estrutural
%calculadas automaticamente pela ferramenta
%Analizo\footnote{\url{http://www.analizo.org}}, um conjunto de ferramentas para
%análise estática e cálculo de métricas de código-fonte.

\section{Resultados}

(pendente)

\section{Organização do texto}

(pendente)
