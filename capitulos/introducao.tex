\xchapter{Introdução}{}

% Contexto. O contexto faz parte da motivação do trabalho.
\section{Contexto}
A análise estática de programas ...
tem se mostrado útil ... monitoramento da qualidade ...
adoção no processo de desenvolvimento de software ...
uso crescente de ferramentas ... 

% Problema. 
\section{Problema}
% Uma maneira de evidenciar a contribuição do trabalho é 
% definir bem o problema a ser resolvido. 
% Nesse caso, pode-se discutir: o problema em questão, a definição formal 
% do problema e sua importância, relevância, aplicações práticas.

A tecnologia de análise estática tem se desenvolvido rapidamente, mas a
comparação e avaliação de técnicas e ferramentas não tem acompanhado tal
velocidade \cite{Li2010}.

% Trabalhos Relacionados. 
\section{Trabalhos relacionados}
% Outra maneira de evidenciar a contribuição do trabalho 
% é discutir os trabalhos relacionados (resumidamente) ainda na introdução. 
% Esses trabalhos estão no mesmo contexto, não resolvem o problema ou 
% apresentam apenas soluções parciais. 
% Além disso, o trabalho atual pode ser a extensão ou continuação de um trabalho 
% anterior. Nesse caso, o trabalho original deve ser mencionado na introdução.

Diversos trabalhos avaliam ferramentas de análise estática de código-fonte,
em geral, que possuem um mesmo propósito (detecção de bugs, 
detecção de vulnerabilidades, detecção de anomalias, etc.),
sob a perspectiva de seus usuários, 
e com foco em atributos de qualidade externa, tais como, desempenho, 
precisão e cobertura de seus resultados. 

Por exemplo, \cite{Rutar2004} \cite{Kratkiewicz2005} \cite{Okun2007}
\cite{Emanuelsson2008} \cite{Wedyan2009} \cite{Mantere2009} \cite{Al2010}
\cite{Li2010} \cite{Johns2011} \cite{Alemerien2013} \cite{Ataide2014} realizam
estudos comparando ferramentas de análise estática através dos seus atributos
de qualidade externa.

Entretanto, não foram encontrados estudos que avaliam 
ferramentas de análise estática de código-fonte 
com foco em sua qualidade interna, 
considerando o ponto de vista de desenvolvedores interessados 
não apenas em usar, mas também em manter e evoluir tais ferramentas.

Em tal contexto, surge o problema de identificar, medir e analisar fatores
técnicos que influenciam na manutenção de uma ferramenta de análise estática. 

\section{Relevância}
% ABAIXO: por que vale a pena tratar analisador estático
% como caso especial? Por que se concentrar nesse domínio?
% Por que abordagens independentes de domínio não atendem?

Estudar ferramentas de análise estática a fim de compreender seus atributos de
qualidade interna e entender quais características %arquiteturais 
explicam tais atributos é de fundamental importância do 
ponto de vista de usuários (gerentes, desenvolvedores) interessados
em adotar, usar, manter e evoluir ferramentas deste domínio de aplicação.
% sem afetar negativamente seus atributos de qualidade interna.

% compreensão, arquitetura;  falar de métricas de produto, valores de referência
A arquitetura de software tem papel importante na compreensão e evolução 
de [ ref ] ...
módulos e dependências entre eles ... coesão e acoplamento entre módulos ...
complexidade estrutural ...
 
Complexidade estrutural crescente em projetos de software pode resultar
no aumento no esforço necessário para atividades de manutenção.

A qualidade interna de ferramentas de análise estática pode ser 
por meio de ... métricas de ... 
Valores de referência são importantes ...
Em geral, valores de referência são encontrados para linguagens de programação.
E para domínios?

% ------------------

O domínio de aplicação de análise estática tem evoluído constantemente desde o
seu surgimento a mais de 40 anos, e apesar da sua evolução constante poucos
estudos demonstram a preocupação em avaliar ou comparar ferramentas de análise
estática de código-fonte em relação à seus atributos de qualidade interna, um
aspecto importante para guiar e auxiliar desenvolvedores em atividades de
evolução e manutenção de software.

Desta forma estudar e comparar tais ferramentas a fim de compreender seus
atributos de qualidade interna e entender quais características arquiteturais
explicam tais atributos é de fundamental importância do ponto de vista de
desenvolvedores interessados em melhorar a qualidade interna de suas
ferramentas.

\subsection{Questão de pesquisa}

Com este objetivo em mente, levantamos a seguinte questão de pesquisa:

\begin{enumerate}
  \item [{\bf Q1:}] {\em Como a complexidade estrutural pode ser interpretada
    e explicada para ferramentas de software do domínio de aplicação de
    análise estática de código-fonte?}
\end{enumerate}

% Aim, Main Research Goal
\section{Objetivo geral}

Diante disso, definimos como objetivo principal deste trabalho: compreender
as ferramentas de software para análise estática de código-fonte do ponto de
vista de sua manutenabilidade, a partir da análise de sua complexidade
estrutural, discutindo quais características arquiteturais explicam seus
atributos de qualidade interna.

% Objectives
\section{Objetivos específicos}

São objetivos específicos deste trabalho:

\begin{itemize}
  \item Selecionar e obter código-fonte de ferramentas de análise estática
    desenvolvidas na academia, para coletar suas métricas de código-fonte.  A
    seleção terá como base o resultado de uma revisão estruturada feita a
    partir de artigos publicados em conferências relacionadas. 
  \item Selecionar e obter código-fonte de ferramentas de análise estática
    desenvolvidas pela indústria, para coletar suas métricas de código-fonte.
  \item Propor intervalos de referência para a observação parametrizada da
    qualidade interna das ferramentas de análise estática, a partir de suas
    métricas de código-fonte.
  \item Calcular a distância Euclidiana entre valores de referência e os
    valores das ferramentas estudadas.
\end{itemize}

\section{Contribuições esperadas}

Ao final deste trabalho, as seguintes contribuições científicas ({\bf CC}) e
tecnológicas ({\bf CT}) são esperadas:

\begin{enumerate}
  \item [{\bf CC1:}] Um conjunto de intervalos de referência da frequência dos
    valores de métricas de código-fonte para o domínio de aplicação de
    análise estática de código-fonte.
  \item [{\bf CC2:}] Definição de argumentos que expliquem a alta complexidade
    estrutural em ferramentas de análise estática de código-fonte.
  \item [{\bf CT1:}] Evolução de uma ferramenta de análise estática de
    código-fonte.
\end{enumerate}

Lembrando que neste trabalho serão utilizadas métricas de produto,
especificamente, métricas de código-fonte, que cobrem aspectos de tamanho,
complexidade e qualidade.

\section{Estrutura do texto} 

O capítulo \ref{fundamentacao} apresenta conceitos sobre análise estática e
métricas de código-fonte necessários para compreensão do trabalho. O capítulo
\ref{metodologia} apresenta trabalhos relacionados, hipóteses do estudo,
planejamento sobre a coleta e análise dos dados
e o cronograma do estudo. O capítulo \ref{conclusoes} traz a
discussão e interpretação dos dados e apresenta os próximos passos do estudo.
