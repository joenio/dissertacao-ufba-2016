\xchapter{Introdução}{}

Em diversas linhas de pesquisa da computação, em especial, em engenharia de
software, é bastante comum que novos softwares sejam desenvolvidos durante
trabalhos de pesquisa, tais softwares tem sido chamados na literatura de {\it
research tool} \cite{Portillo12}, {\it research-originated software}
\cite{Kon2011} ou {\it academic software}
\cite{allen2017engineering}, e vêm ganhando atenção por serem artefatos importantes para a
divulgação do conhecimento e para replicação dos resultados \cite{Stodden2009}.

Avaliar esses softwares, chamados a partir daqui apenas de {\it software
científico}, deve ser uma preocupação da engenharia de software, uma disciplina
centrada em avaliar e validar métodos, técnicas, linguagens e ferramentas.
Estudos avaliando softwares são bastante comuns, apenas no domínio
de aplicação de análise estática, por exemplo, é possível encontrar diversos estudos
avaliando ferramentas, desde ferramentas de localização de bugs \cite{Rutar2004}, detecção
de {\it buffer overflow} \cite{Kratkiewicz2005}, segurança \cite{Okun2007,
Johns2011}, detecção de falhas e refatorações \cite{Wedyan2009}, detecção de
vulnerabilidades \cite{Li2010, Ataide2014}, cálculo de métricas
\cite{Alemerien2013}, até estudos comparando ferramentas de análise estática
com compiladores comuns \cite{Emanuelsson2008} e estudos comparando ferramentas
comerciais com ferramentas {\it open source} \cite{Al2010}.

%Ferramentas de análise estática de código fonte tem se tornado mais e mais
%comum no ciclo de vida do desenvolvimento de software \cite{Novak2010}, a
%importância destas ferramentas no ciclo de desenvolvimento de softwares tem
%crescido ao longo do tempo enorme.

Apesar dos inúmeros estudos avaliando softwares, poucos fazem isso com
softwares científicos, algo de extrema importância para compreender como estes
artefatos estão contribuindo para a divulgação do conhecimento e para
replicação dos resultados das pesquisas em engenharia de software. A área de
análise de código fonte tem uma longa e respeitável tradição, ferramentas de
análise estática de código fonte tem sido continuamente desenvolvidas e tem se
tornado mais e mais comuns no ciclo de desenvolvimento de software
\cite{Novak2010}, e apesar da rápida e constante evolução da área, ainda há
carência de estudos avaliando estas ferramentas \cite{Li2010} e carência
de estudos com validação empírica \cite{ilyas2016static}, como {\it
surveys}, experimentos ou revisao sistemática de literatura, por exemplo, especialmente entre
os {\it softwares científicos}.

% (2) Tools to support systematic literature reviews in software engineering: A mapping study \cite{marshall2013tools}
%
% Cita um mapeamento feito sobre estudos que criam ferramentas para apoio a
% revisão sistemática no domínio de SE, 14 estudos foram selecionados, ao final
% apenas 8 tinham proposta de ferramentas, ao final conclui que as ferramentas
% encontradas estão em estado inicial de desenvolvimento. 
%
% (3) Tools used in Global Software Engineering: A systematic mapping review \cite{Portillo12}
%
% Cita um mapeamento sistemático com objetivo de encontrar ferramentas de
% comunicação e coordenação para suporte a times altamente distribuidos
% gograficamente, encontrou 132 ferramentas, para uso em projetos de software
% global. A maioria destas ferramentas foram desenvolvidas em centros de
% pesquisas, e apenas uma pequena porcentagem (18.9\%) foram testados fora do
% seu contexto onde foi desenvolvido.
%
% (5) Tools in mining software repositories \cite{chaturvedi2013tools}
%
% Faz uma revisão dos papers submetidos ao MSR desde 2007 até 2013 (?) e
% identifica data sets, ferramentas e técnicas utilizadas pelos autores, mais
% da metade dos papers usam ou criam ferramentas, categoriza as ferramentas em
% ferramentas novas, ferramentas tradicionais, protótipos e scripts para
% mineração de dados
%
% (6) A systematic literature review of software product line management tools \cite{pereira2015systematic}
%
% (???)
%
% (7) Software configuration management tools \cite{chan1997software}
%
% (???)
%
% (8) Comparison and evaluation of source code mining tools and techniques: A qualitative approach \cite{khatoon2013comparison}
%
% Lista ferramentas e técnicas para mineração de dados, estado da arte.
%
% (9) An overview of free software tools for general data mining \cite{jovic2014overview}
%
% Descreve característica dos 6 softwares livres mais usados para mineração de
% dados no geral.
%
% (10) Analyzing the State of Static Analysis: A Large-Scale Evaluation in Open Source Software \cite{beller2016analyzing}
%
% faz um estudo mostrando que analise estatica tem uma certa adocao em projetos livres
% e mostra onde pode-se melhorar nas ferramentas para aumentar a adoção



% Taming the Static Analysis Beast
% \cite{toman2017taming}
% Despite advances in tooling and mainstream success, static analysis development is still a
% painful process.

Assim, avaliar esses softwares e explorar como são publicados pode jogar luz
sobre o quanto eles estão contribuindo para a divulgação do conhecimento em
pesquisas da engenharia de software e o quanto estão contribuindo para
proporcionar reprodução dos seus resultados. Avaliar os {\it softwares
científicos} do ponto de vista de sua qualidade pode também ajudar a
compreender quanta atenção é dada ao seu desenvolvimento, uma vez que
tradicionalmente os autores de {\it software científico} enfrentam problemas
com manutenabilidade e disponibilidade de tais softwares \cite{Prlic2012}.

Manutenabilidade é uma característica de qualidade externa que indica o quão
fácil é realizar atividades de evolução e manutenção em componentes de
software, ela pode ser medida através de características de qualidade interna,
uma vez que grande parte dos engenheiros de software assumem que uma boa
estrutura interna resulta em boa qualidade externa \cite{Fenton2014}. A
estrutura interna de um software pode ser avaliada através da sua complexidade,
uma característica bastante referenciada na literatura como um importante
indicador de qualidade, estudos mostram que quanto maior a complexidade, maior
é o esforço de manutenção \cite{hashim1996software, Darcy2005}, em especial a
complexidade estrutural, uma medida definida em termos de acoplamento e coesão
\cite{Terceiro2012}.

Assim, definimos como objetivo geral deste trabalho caracterizar e avaliar os
{\it softwares científicos} para análise estática de código fonte, a fim de
compreender como eles contribuem para divulgação e reprodução dos resultados de
pesquisas da área de engenharia de software.

São objetivos específicos deste trabalho:

\begin{enumerate}
  \item Avaliar a divulgação e disponibilidade de código fonte dos {\it softwares científicos} de análise estática de código fonte.
  \item Avaliar a qualidade interna dos {\it softwares científicos} a partir de sua complexidade estrutural.
  \item Compreender como os {\it software científicos} estão contribuindo para divulgação e reprodução dos resultados de pesquisas em engenharia de software.
\end{enumerate}

% The personal pledges expressed in this Dagstuhl Manifesto 1 address three general con-
% cerns: (i) ensuring that research software is properly cited; (ii) promoting the careers of
% research software engineers who develop academic software; and (iii) ensuring the quality
% and sustainability of software during and following its development:

Além dos argumentos já citados como motivação deste estudo, estamos também em
sintonia com alguns dos compromissos do {\it Dagstuhl Manifesto}, um documento
que resume algumas perspectivas discutidas no workshop {\it Dagstuhl
Perspectives Workshop on ``Engineering Academic Software''} voltado a examinar
as fortalezas, fraquezas, riscos e oportunidades da engenharia de software
acadêmica, explora tópicos novos e emergentes na ciência da computação e produz
manifestos que capturam tendências e desenvolvimentos relacionados à estes
tópicos. A contribuição chave deste workshop é o {\it Dagstuhl Manifesto}, um
roteiro para o futuro da engenharia de software profissional e acadêmica, com
foco em instrumentos para pesquisas em software.  O manifesto é expresso em
termos de ações ``promessas'' destinados a usuários e desenvolvedores de
softwares acadêmicos, com passos concretos para melhorar o ambiente em que os
softwares são produzidos.

O compromisso (iii) do manifesto Dagstuhl \cite{allen2017engineering} de {\it
``medir a qualidade e sustentabilidade do software durante e após o seu
desenvolvimento''} é aquele que está em sintonia com este trabalho,
especialmente em relação às questões após o desenvolvimento dos softwares.
Os tópicos levantados no manifesto e discutos nos capítulos 3 e 4 do manifesto
relacionados ao presente trabalho são os seguintes:

\begin{description}

  \item [Sustentabilidade e disponibilidade]

    O software que nós usamos e produzimos no contexto acadêmico é sustentável?
    Temos certeza de que podemos reproduzir métodos de pesquisas anteriores no
    futuro, tendo em vistas mudanças arbitrárias nos contextos tecnológicos
    (máquinas, sistemas operacionais, linguagens de programação, frameworks)?
    Podemos adaptar de forma incremental o software de pesquisa para
    oportunidades emergentes ao mesmo tempo, sem perda de reprodutibilidade e
    sem custos proibitivos?

  \item [Qualidade]

    Como podemos garantir a qualidade do software acadêmico? Como podemos
    monitorar, orientar, informar e revisar a qualidade do software acadêmico?
    Como podemos gerenciar e garantir confiança entre as equipes de pesquisa
    acadêmica considerando software desenvolvido para métodos de presquisa e/ou
    de produção?

\end{description}

\section{Metodologia de trabalho}

(pendente)

\section{Contribuições}

(pendente)

\section{Organização do texto}

(pendente)
