\xchapter{Introdução}{}

"Ocorrências não reprodutíveis não têm significado para a ciência" \cite{popper2004logica}.

\section{Motivação}

Em 2010, ... contar o cenário que envolve Analizo.

Eu, como engenheiro de software pesquisador e parte do ecossistema do software
acadêmico de análise estática Analizo estou interessado em saber como melhor
investir recursos no ecossistema do projeto Analizo com o objetivo de fazê-lo
mais útil e acessível para todo o campo de pesquisa, uma vez que percebo que os
projetos de software do meu campo sofrem da existência de muitos projetos, com
poucos usuários, com ciclos de vida curtos, que terminam em paralelo ao
financiamento inicial, comunidades desconectadas e paralelas,
incompatibilidades entre projetos, e tentativas aparentemente não coordenadas
de ``reiniciar'' tudo ({\it re-boots}).

Mas antes de saber como atingir este objetivo é necessário compreender como
este problema ocorre e se ele realmente acontece no ecossistema de software
acadêmico de análise estática.

% é interessante realmente apresentar o Analizo como um case de sucesso

%software acadêmicos sofrem de {\it ``dysfunctional chaotic churn''}.
% O ecossistema de software acadêmico sofre de um fenômeno chamado de
% desordem caótica disfuncional ({\it ``dysfunctional chaotic churn''}).

\section{Apresentação}

A Ciência moderna depende de software. Novos softwares são constantemente
criados e desenvolvidos durante pesquisas científicas, seja pelo próprio pesquisador,
seja por colaboradores. 
Estes softwares resolvem problemas comuns de pelo menos metade dos cientistas 
de todas as áreas da Ciência, em atividades de coleta ou análise, 
e em outras atividades de pesquisa.

Estes softwares variam em tamanho e finalidade: alguns são simples
scripts para automatizar tarefas repetitivas, enquanto outros são utilizados para
coleta, análise ou transformação de dados. Alguns softwares são projetos maduros e
carregam boa parte do conhecimento desenvolvido ao longo da própria pesquisa, e
costumam ter uma maior promoção por parte dos seus autores e um maior
reconhecimento por parte da comunidade científica.

O domínio e objetivo da pesquisa também implicam em particularidades do software: 
um software para coleta de dados numa pesquisa em engenharia de
software certamente terá características e requisitos distintos de um software
para o mesmo fim em outro domínio, por exemplo,  antropologia ou medicina.
Em um mesmo domínio, softwares para análise de dados podem variar
enormemente entre sí, dependendo do objetivo da pesquisa e das atividades realizadas.
\'{E} natural acreditar, portanto, que existe demanda suficiente para a
criação de softwares para os mais diversos campos de atuação e aplicação da
Ciência.

A criação destes softwares tem motivações variadas:  alguns cientistas publicam
seus softwares juntamente com dados e outros artefatos, outros fazem um esforço
adicional para documentar e incentivar o seu uso e adoção, alguns não
consideram seus softwares dignos de publicação, outros não dão visibilidade a
sua existência mesmo quando são publicados sob as melhores práticas da
engenharia de software.

Independente da finalidade, tamanho ou motivação, todo software tem o potencial
de voltar a ser útil em outros momentos ou lugares, para o autor original,
ou para pesquisadores enfrentando problemas semelhantes aos dos autores originais.
Não é difícil imaginar que o problema enfrentado por um pesquisador pode, em
algum momento, ser enfrentado por outros pesquisadores, criando assim uma ótima
oportunidade de colaboração entre pesquisas e pesquisadores, sendo o software
um excelente vetor de ajuda mútua em ambas as direções.

Apesar disso, cientistas tem percebido que os softwares desenvolvidos na
academia sofrem de {\it ``dysfunctional chaotic churn''}, ou seja, a existência
de muitos projetos, com poucos usuários, com ciclos de vida curtos, que
terminam em paralelo ao financiamento inicial, comunidades desconectadas e
paralelas, incompatibilidades entre projetos, e tentativas aparentemente não
coordenadas de ``reiniciar'' tudo ({\it re-boots}).

%Este cenário, além de desacelerar o progresso geral da ciência gerando
%retrabalho, faz surgir questionamentos sobre as conclusões dessas pesquisas,
%especialmente quando grande parte dos pesquisadores não sabem o quão confiável
%seus softwares são.

Este problema apesar de ser percebido por muitos cientistas carece de
evidências, neste trabalho investigamos como ele se manifesta em pesquisas da
engenharia de software, em especial, em publicações de análise estática, uma
área com uma longa e respeitável tradição em pesquisas sobre a criação de novas
ferramentas, métodos e algoritmos.

\section{Escopo}

O objetivo geral desta pesquisa é explorar como o {\it ``dysfunctional chaotic
churn''} se manifesta entre os projetos de software de análise estática,
respondendo as seguintes questões:

\begin{itemize}
  \item Projetos de software acadêmico de análise estática tem contribuidores além dos autores iniciais?
  \item Os projetos incentivam ativamente a contribuição?
  \item Os espaços dos projetos são abertos e transparentes?
  \item Os projetos fazem um pedido explícito e claro de reconhecimento?
\end{itemize}

Que tal perguntas assim?
\begin{itemize}
  \item Os projetos sustentáveis tem contribuidores além dos autores iniciais?
  \item Os projetos sustentáveis tem mais contribuidores?
  \item Os artigos de projetos sustentáveis são mais lidos, mais citados?
  \item Os autores de projetos sustentáveis tem mais publicações com o uso de software do que os autores?
  \item Os projetos são mais fáceis de manter? de usar? 
\end{itemize}

Estas questões darão importantes indícios sobre a qualidade do software acadêmico de
análise estática desenvolvido na academia, especialmente sobre a capacidade de serem 
encontrados, compartilhados e co-desenvolvidos.

\section{Metodologia de trabalho}

(apresentar aqui um resumo do capítulo \ref{metodologia})

Objetivo específico: caracterizar software acad publicado quanto a ...
Objetivo específico: caracterizar o tipo de citação / menção d

\section{Contribuições}

(pendente)

\section{Organização do texto}

(pendente)

%O capítulo \ref{fundamentacao} apresenta os fundamentos teóricos necessários
%para a compreensão deste trabalho.
%
%O capítulo \ref{metodologia} apresenta a estratégia de pesquisa adotada nos
%estudos.
%
%O capítulo \ref{sustentabilidade-tecnica} traz um estudo sobre a
%sustentabilidade técnica e a disponibilidade dos softwares acadêmicos de
%análise estática.
%
%O capítulo \ref{complexidade-ferramentas} descreve um estudo sobre a
%manutenabilidade dos softwares acadêmicos de análise estática.
%
%O capítulo \ref{conclusoes} apresenta as considerações finais e discute os
%resultados deste trabalho.

% A Retrospective Study of Academic Software Projects
% (estudo restropectivo parece um bom termo, uma busca rápida no google-scholar retornou muita coisa da área de saúde)

