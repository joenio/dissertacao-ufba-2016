\xchapter{Introdução}{}

(pendente)

%Observação de métricas de código fonte tem sido uma prática útil em diversos
%estudos em engenharia de software, como por exemplo, predição de erros,
%visualização de código-fonte, definição e avaliação de intervalos de referência
%({\it thresholds}), estimativa de custos e prazos, entre outros.  Métricas tem
%sido útil ainda como uma maneira efetiva de se observar a qualidade interna de
%produtos de software \cite{Meirelles2013}, uma vez que são bons indicadores de
%qualidade \cite{Basili1996}.
%
%Medir e compreender aspectos de qualidade interna de software é útil em
%atividades de evolução e manutenção, uma vez que a qualidade interna produz
%forte impacto nestas atividades, quanto menor a qualidade de código-fonte,
%maior será o esforço de mantê-lo \cite{Terceiro2010}. Uma das formas de medir
%qualidade interna é em função de métricas de código-fonte, dentre as quais
%podemos citar a complexidade estrutural, uma medida que leva em conta a relação
%entre acoplamento e coesão dos módulos de um programa \cite{Darcy2005}. Os
%autores desta métrica demonstraram por meio de um experimento controlado que
%uma alta complexidade estrutural faz projetos de software mais difíceis de
%entender, e por isto mesmo, mais difíceis de manter e evoluir.
%
%A observação de métricas de código-fonte e as conclusões tomadas a partir
%delas, no entando, não podem ser generalizadas uma vez que projetos diferentes
%são influenciados por fatores diferentes, de forma que a observação dos
%aspectos de qualidade interna devem ser observados e interpretados
%separadamente por projeto \cite{Terceiro2012Understanding} ou ao menos por
%domínio de aplicação \cite{Meirelles2013}. Dessa forma, pode-se afirmar que
%realizar estudos em domínios específicos, especialmente em domínios ainda pouco
%estudados são extremamente importantes.
%
%Diante disso e sabendo que o domínio de aplicação de análise estática carece de
%observação detalhada sobre aspectos de qualidade interna de suas ferramentas
%podemos concluir que estudos que joguem luz sobre tais aspectos neste domínio
%são altamente desejáveis. A área de análise estática de código-fonte tem se
%desenvolvido rapidamente com novos métodos, técnicas e ferramentas para os mais
%diversos fins, no entanto a comparação e avaliação de técnicas e ferramentas
%não tem acompanhado tal velocidade \cite{Li2010}, e, apesar de existirem
%estudos avaliando ferramentas de análise estática de código-fonte
%\cite{Rutar2004, Kratkiewicz2005, Okun2007, Emanuelsson2008, Wedyan2009,
%Mantere2009, Al2010, Li2010, Johns2011, Alemerien2013, Ataide2014}, poucos
%fazem isso do ponto de vista de sua qualidade interna, em sua maioria observam
%apenas atributos de qualidade externa, como, desempenho, precisão, cobertura de
%resultados, entre outros.
%
%Observar métricas de código-fonte em nível de projetos de software leva
%ao seguinte desafio: como obter valores de métricas que representem todo o projeto sendo
%que métricas de código-fonte usualmente são calculadas para cada elemento do sistema, como arquivos ou classes?
%Este desafio tem sido amplamente discutido em estudos sobre definição de
%intervalos de referência ({\it thresholds}) para métricas de
%código-fonte \cite{Shatnawi2010, Kaur2013, Herbold2011}. Intervalos de
%referência são valores conhecidos para uma dada medida
%\cite[Chapter~2.1]{Lanza2007} com algum valor semantico, por exemplo, se
%medirmos a altura das pessoas e definirmos até 2 metros como alto, então
%pessoas acima de 2 metros serão classificadas como muito altas.
%
%Intervalos de referência podem ser definidos de diversas formas, desde
%abordagens baseadas em modelos estatísticos \cite{Shatnawi2010, Kaur2013}
%até aprendizado de máquina \cite{Herbold2011} e inteligência artificial.
%Entre as inúmeras abordagens, muitas partem de estudos empíricos
%usando softwares da indústria como objeto de estudo, geralmente com
%softwares de domínios específicos, parte-se da coleta de dados de
%métricas de código-fonte e com uso de uma abordagem, ou uma combinação entre
%elas, chega-se aos intervalos.
%
%Estes intervalos são também continuamente avaliados a fim de saber se são
%válidos ou não, as abordagens utilizadas para calcular os intervalos levam em
%consideração inúmeros aspectos na tentativa de validar os valores encontrados,
%como por exemplo a natureza dos dados, se seguem a lei de distribuição de
%potência
%\cite{Wheeldon2003,Potanin2005,Concas2007,Ferreira2009,Yao2009,Clauset2009} ou
%seguem uma distribuição normal
%\cite{Baxter2006,Lanza2007,Herraiz2011,Herraiz2012}, avaliam ainda se possuem
%cauda longa, se são livre de escala, entre outros aspectos.
%
%Apesar dos inúmeros estudos sobre métricas e intervalos de referência, e as
%várias abordagens utilizadas para validar os valores encontrados, nenhum trabalho
%compara tais valores com características arquiteturais,
%algo que pode vir a ser útil na compreensão de tais intervalos.
%Estudos sobre arquitetura de software tem uma longa e respeitável tradição, a
%área emergiu como uma importante disciplina da engenharia de software,
%particularmento em atividades de desenvolvimento de grandes sistemas.
%Arquitetura dá controle intelectual sobre a complexidade permitindo focar em
%partes essenciais do sistema e suas interações, ajudando a compreender como as
%diferentes partes de um sistema se relacionam entre sí \cite{Clements2002Book}.
%
%Traçar uma comparação entre intervalos de referência ({\it thresholds}) de
%métricas como complexidade estrutural, por exemplo, com métricas que
%representam visões arquiteturais de projetos de software pode ser útil como
%forma de avaliar se tais medidas estão consistentes entre sí, por exemplo, se
%numa dada medida de complexidade estrutural encontra-se valores considerados
%ruins indicando problemas arquiteturais, então medidas e métricas relacionadas
%arquitetura de software também deveriam confirmar tal indicação.
%
%Dentre os inúmeros estudos sobre arquitetura de software alguns tem feito uso
%de uma técnica chamada DSM ({\it Design Structure Matrix}) para observar e
%avaliar sistemas complexos, esta técnica foi concebida por
%\citeonline{Steward1981} e extendido por \citeonline{Eppinger1991} em um estudo
%sobre engenharia concorrente, ela é bastante útil em análises para mapear
%sistemas complexos, DSM provê uma representação compacta de um sistema complexo
%para visualização de interdependencia entre os seus vários elementos.
%
%DSM tem sido aplicada em diversas áreas do conhecimento, deste gestão de
%projetos \cite{Browning2016}, análise de sistemas da engenharia de energia,
%automotiva, construção civil, etc. Em engenharia de software alguns estudos
%sobre arquitetura de software foram conduzidos para analisar e comparar
%estruturas de produtos de software complexos, nestes estudos DSMs são
%utilizados para destacar a estrutura do design examinando as dependencias
%existentes entre os componentes através de uma matriz simétrica
%\cite{Steward1981}. DSM tem sido utilizada também como meio para calcular a
%métrica ``custo de mudança'' ({\it Change Cost}) \cite{Maccormack2006},
%uma medida para caracterizar a estrutura do design de software a partir da
%medida de acoplamento que ela exibe, capturando o nível com qual uma mudança em
%um elemento causa de impacto em outras partes do mesmo sitema, seja diretamente
%ou indiretamente através de chamadas aninhadas entre os elementos.
%
%A partir disso pode-se dizer que a métrica custo de mudança se mostra bastante
%útil em estudos sobre qualidade interna de produtos de software, uma vez que
%ela representa características arquiteturais importantes. Além disso, essa
%métrica é calculada em nível de projeto e não passa pelos problemas enfrentados
%por métricas tradicionais de código-fonte ao serem calculadas em nível de
%projeto, como a complexidade estrutural.  Algo que se mostra como uma enorme
%oportunidade de estudo sobre o cálculo de intervalos de referência, podendo
%servir de base de comparação e apoio na interpretação dos valores encontrados.
%
%Assim, iremos neste trabalho, analisar ferramentas do domínio de análise
%estática, extrair as métricas de complexidade estrutural e custo de mudança,
%calcular intervalos de referência para a métrica de complexidade estrutural, em
%seguida comparar tais intervalos com os valores da métrica custo de mudança a
%fim de compreender como estas medidas se comportam neste domínio de aplicação.

%% \section{Contribuições esperadas}
%% 
%% (pendente)
%% 
%Ao final deste trabalho, as seguintes contribuições científicas ({\bf CC}) e
%tecnológicas ({\bf CT}) são esperadas:
%
%\begin{enumerate}
%  \item [{\bf CC1:}] ...
%  \item [{\bf CC2:}] ...
%  \item [{\bf CT1:}] ...
%\end{enumerate}

%% \section{Estrutura do texto} 
%% 
%% (pendente)
%% 
%% O capítulo \ref{fundamentacao} apresenta conceitos sobre análise estática e
%% métricas de código-fonte necessários para compreensão do trabalho. O capítulo
%% \ref{metodologia} apresenta trabalhos relacionados, hipóteses do estudo,
%% planejamento sobre a coleta e análise dos dados
%% e o cronograma do estudo. O capítulo \ref{conclusoes} traz a
%% discussão e interpretação dos dados e apresenta os próximos passos do estudo.
%% 
%% A seção \ref{trabalhos-relacionados} traz trabalhos relacionados à compreensão
%% e observação de atributos de qualidade interna de programas. A seção
%% \ref{hipoteses} detalha como as hipóteses serão testadas. A seção
%% \ref{planejamento} descreve o planejamento de estudo e os passos iniciais de
%% coleta de dados. A seção \ref{coleta} detalha a coleta de dados e a seção
%% \ref{analise} traz informações de como estes dados serão analisados e
%% interpretados.
