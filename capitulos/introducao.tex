\xchapter{Introdução}{}

Em diversas linhas de pesquisa da computação, em especial, em engenharia de
software, é bastante comum que novos softwares sejam desenvolvidos durante
trabalhos de pesquisa, tais softwares tem sido chamados na literatura de {\it
research-originated software} \cite{Kon2011}, {\it research tool}
\cite{Portillo12} ou {\it academic software} \cite{allen2017engineering}, e vêm
ganhando atenção da academia devido a sua importância para o amadurecimento da
área.

Este amadurecimento inclui a capacidade de replicar pesquisas anteriores e
assim aumentar a validade em seus resultados, algo de extrema importância para
contribuir com a base de conhecimento da comunidade de pesquisa. Esses
softwares, chamados a partir daqui apenas de softwares acadêmicos, ocupam um
papel central na discussão sobre replicação \cite{Stodden2009, Peng2011} e
podem impactar de forma positiva ou negativa nesta questão.

Isto tem feito a comunidade voltar sua atenção para os softwares acadêmicos e a
refletir sobre os problemas relacionados ao seu desenvolvimento, promoção,
reconhecimento e sustentabilidade \cite{allen2017engineering}. Esta reflexão
tem mostrado, por exemplo, que muitos estudos sofrem de dificuldades de
repetição \cite{Tang2016}, e apontam problemas específicos relacionados à
disponibilidade e a qualidade dos softwares. Muitos pesquisadores não
disponibilizam os seus softwares \cite{robles2010replicating,
amann2015software} ou quanto o fazem enfrentam problemas com manutenabilidade
\cite{Prlic2012}, uma qualidade importante aos interessados em adaptar tais
softwares.

Apesar da crescente preocupação com os softwares acadêmicos, sabe-se pouco
sobre o quanto algumas de suas características afetam a capacidade de
reprodução de seus estudos, sobretudo em áreas específicas da ciência da
computaçao, como a área de análise estática de software, uma área com uma longa
e respeitável tradição, mas que ainda sofre com carência de estudos sobre
avaliação e validação de suas ferramentas \cite{Li2010, ilyas2016static}.

% (2) Tools to support systematic literature reviews in software engineering: A mapping study \cite{marshall2013tools}
%
% Cita um mapeamento feito sobre estudos que criam ferramentas para apoio a
% revisão sistemática no domínio de SE, 14 estudos foram selecionados, ao final
% apenas 8 tinham proposta de ferramentas, ao final conclui que as ferramentas
% encontradas estão em estado inicial de desenvolvimento. 
%
% (3) Tools used in Global Software Engineering: A systematic mapping review \cite{Portillo12}
%
% Cita um mapeamento sistemático com objetivo de encontrar ferramentas de
% comunicação e coordenação para suporte a times altamente distribuidos
% gograficamente, encontrou 132 ferramentas, para uso em projetos de software
% global. A maioria destas ferramentas foram desenvolvidas em centros de
% pesquisas, e apenas uma pequena porcentagem (18.9\%) foram testados fora do
% seu contexto onde foi desenvolvido.
%
% (5) Tools in mining software repositories \cite{chaturvedi2013tools}
%
% Faz uma revisão dos papers submetidos ao MSR desde 2007 até 2013 (?) e
% identifica data sets, ferramentas e técnicas utilizadas pelos autores, mais
% da metade dos papers usam ou criam ferramentas, categoriza as ferramentas em
% ferramentas novas, ferramentas tradicionais, protótipos e scripts para
% mineração de dados
%
% (6) A systematic literature review of software product line management tools \cite{pereira2015systematic}
%
% (???)
%
% (7) Software configuration management tools \cite{chan1997software}
%
% (???)
%
% (8) Comparison and evaluation of source code mining tools and techniques: A qualitative approach \cite{khatoon2013comparison}
%
% Lista ferramentas e técnicas para mineração de dados, estado da arte.
%
% (9) An overview of free software tools for general data mining \cite{jovic2014overview}
%
% Descreve característica dos 6 softwares livres mais usados para mineração de
% dados no geral.
%
% (10) Analyzing the State of Static Analysis: A Large-Scale Evaluation in Open Source Software \cite{beller2016analyzing}
%
% faz um estudo mostrando que analise estatica tem uma certa adocao em projetos livres
% e mostra onde pode-se melhorar nas ferramentas para aumentar a adoção
%
% Taming the Static Analysis Beast
% \cite{toman2017taming}
% Despite advances in tooling and mainstream success, static analysis development is still a
% painful process.

Avaliar questões sobre sustentabilidade e desenvolvimento, especialmente sobre
a sustentabilidade técnica, ou, a capacidade de perdurar e estar disponível no
futuro e a qualidade interna, ou, a forma como os softwares estão estruturados
e organizados internamente, nos ajudará a compreender o quanto tais
características impactam na reprodutibilidade de seus estudos.

Desta forma, definimos como objetivo geral deste trabalho avaliar o quanto a
sustentabilidade técnica e a qualidade interna dos softwares acadêmicos de
análise estática impactam na reprodutibilidade de seus estudos.

São objetivos específicos deste trabalho:

\begin{enumerate}
  \item Medir e avaliar a sustentabilidade técnica dos softwares acadêmicos de
        análise estática de software.
  \item Medir e avaliar a qualidade interna dos softwares acadêmicos de análise
        estática de software.
  \item Compreender o quanto a sustentabilidade técnica e a qualidade interna
        dos softwares acadêmicos de análise estática de software impactam na
        reprodutibilidade de seus estudos.
\end{enumerate}

\section{Metodologia de trabalho}

(pendente)

\section{Contribuições}

(pendente)

\section{Organização do texto}

(pendente)
