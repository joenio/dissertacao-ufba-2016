\xchapter{Caracterização das ferramentas}
{Este capítulo apresenta a caracterização inicial das ferramentas selecionadas a partir da revisão estruturada.}
\label{caracterizacao-ferramentas}


\section{Ferramentas excluídas da análise}

Segue ferramentas citadas nos artigos incluidos na primeita etapa da revisão
estruturada, os artigos incluidos são apenas aqueles que citavam criação de
alguma ferramenta, demais artigos incluidos na primeira seleção mas que não
criavam nenhuma ferramenta não estão listados aqui.

Os artigos que foram incluidos na primeira selecao e nao estao listados aqui de alguma forma
normalmente nao citavam no artigo o nome da ferramenta, nem onden encontrar, muitas vezes
apenas uma prova de conceito e os autores nao deram pistas sobre a ferramenta.

\section{Ferramentas acadêmicas incluídas da análise (SCAM)}

\subsection{AccessAnalysis}

AccessAnalysis é um plugin do Eclipse de análise estática 
para cálculo das métricas IGAT e IGAM
publicadas no artigo ``AccessAnalysis -- A Tool for Measuring the
Appropriateness of Access Modifiers in Java Systems'' do SCAM 2012,
disponível em \url{http://accessanalysis.sourceforge.net}. O código-fonte
utilizado em nosso estudo obtido no site da ferramenta foi o
\texttt{AccessAnalysis-1.2-src.zip}. Características da ferramenta:

\begin{description}

  \item {\it Lançamentos ({\it Releases}) - quantos lançamentos por ano:}
    \begin{table}[h!]
      \centering
      \begin{tabular}{| l | l |}
        \hline
        Ano  & Lançamentos    \\
        \hline
        2012 & 1.2, 1.1, 1.0  \\
        2010 & 0.17           \\
        \hline
      \end{tabular}
    \end{table}
    \begin{itemize}
      \item Obsoleta 0 vezes ao ano - intervalo entre novos lançamentos é maior que 1 ano
    \end{itemize}

  \item {\it Linguagem de programação - em qual linguagem a ferramenta é escrita:}
    \begin{itemize}
      \item Java
    \end{itemize}

\end{description}

\subsection{error-prone}

error-prone é uma ferramenta de localização de bugs construída em cima do
compilador {\it javac} publicada no artigo ``Building Useful Program Analysis
Tools Using an Extensible Java Compiler'' do SCAM 2012 disponível em
\url{http://code.google.com/p/error-prone}. O código-fonte utilizado em nosso
estudo obtido no site da ferramenta foi o \texttt{error-prone-2.0.9.tar.gz}.
Características da ferramenta:

\begin{description}

  \item {\it Lançamentos ({\it Releases}) - quantos lançamentos por ano:}
    \begin{table}[h!]
      \centering
      \begin{tabular}{| l | l |}
        \hline
        Ano  & Lançamentos                                          \\
        \hline
        2016 & 2.0.9, 2.0.8                                         \\
        2015 & 2.0.7, 2.0.6, 2.0.5, 2.0.4, 2.0.3, 2.0.2, 2.0.1, 2.0 \\
        \hline
      \end{tabular}
    \end{table}
    \begin{itemize}
      \item Frequentemente $>=$ 3 vezes ao ano - novas versões da ferramenta são lançadas 3 ou mais vezes por ano
    \end{itemize}

  \item {\it Linguagem de programação - em qual linguagem a ferramenta é escrita:}
    \begin{itemize}
      \item Java
    \end{itemize}

\end{description}

\subsection{EJB}

EJB é uma ferramenta de análise estática para criação de diagramas de sequência
publicada no artigo ``I2SD: Reverse Engineering Sequence Diagrams from
Enterprise Java Beans with Interceptors'' do SCAM 2011, disponível em
\url{https://www.dropbox.com/s/glhg8any43lccgm/EJB.zip}. Características da
ferramenta:

\begin{description}

  \item {\it Lançamentos ({\it Releases}) - quantos lançamentos por ano:}

    {\it A versão obtida foi disponibilizada no Dropbox pelos autores após solicitação
    por email, não há informações sobre o histórico de lançamentos.}

    \begin{itemize}
      \item Obsoleta $0$ vezes ao ano - intervalo entre novos lançamentos é maior que 1 ano
    \end{itemize}

  \item {\it Linguagem de programação - em qual linguagem a ferramenta é escrita:}
    \begin{itemize}
      \item Java
    \end{itemize}

\end{description}

\subsection{Indus}

Indus é uma biblioteca de {\it program
slicing}\footnote{http://en.wikipedia.org/wiki/Program\_slicing} publicada no
artigo ``An Overview of the Indus Framework for Analysis and Slicing of
Concurrent Java Software'' do SCAM 2006, disponível em
\url{http://indus.projects.cis.ksu.edu}.  O projeto está organizado em três
módulos, os seguintes arquivos, contendo o código-fonte dos três módulos,
foram copiados localmente para análise:
\texttt{indus.indus-src-20091220.zip},
\texttt{indus.javaslicer-src-20091220.zip} e
\texttt{indus.staticanalyses-src-20070305.zip}. Características da ferramenta:

\begin{description}

  \item {\it Lançamentos ({\it Releases}) - quantos lançamentos por ano:}
    \begin{table}[h!]
      \centering
      \begin{tabular}{| l | l |}
        \hline
        Ano  & Lançamentos                              \\
        \hline
        2007 & 0.8.1, 0.8.3.10, 0.8.3.11, 0.8.3.12, 0.8.3.15, 0.8.3.1, 0.8.3.6, 0.8.3.7, 0.8.3, 0.8 \\
        2006 & 0.7.0, 0.7.1, 0.7.2.2, 0.7.2             \\
        2005 & 0.6.1, 0.6.2, 0.6.3, 0.6.4.1, 0.6.4, 0.5 \\
        2004 & 0.3, 0.2, 0.1, 0.1a                      \\
        \hline
      \end{tabular}
    \end{table}
    \begin{itemize}
      \item Obsoleta $0$ vezes ao ano - intervalo entre novos lançamentos é maior que 1 ano
    \end{itemize}

  \item {\it Linguagem de programação - em qual linguagem a ferramenta é escrita:}
    \begin{itemize}
      \item Java
    \end{itemize}

\end{description}

\subsection{InputTracer}

InputTracer é uma ferramenta de análise dinâmica de binários x86 em Linux
publicado no artigo ``InputTracer: A Data-flow Analysis Tool for Manual
Program Comprehension of x86 Binaries'' do SCAM 2012 disponível em:
\url{http://www.ida.liu.se/divisions/adit/security/InputTracer}. O
código-fonte utilizado em nosso estudo obtido no site da ferramenta foi o
\texttt{valgrind-inputtracer.tar.gz}.  Características da ferramenta:

\begin{description}

  \item {\it Lançamentos ({\it Releases}) - quantos lançamentos por ano:}
    \begin{table}[h!]
      \centering
      \begin{tabular}{| l | l |}
        \hline
        Ano  & Lançamentos                       \\
        \hline
        2011 & 3.6.1                             \\
        2010 & 3.6.0                             \\
        2009 & 3.5.0, 3.4.1, 3.4.0               \\
        2008 & 3.3.1                             \\
        2007 & 3.3.0, 3.2.3, 3.2.2               \\
        2006 & 3.2.1, 3.2.0, 3.1.1               \\
        2005 & 3.1.0, 3.0.1, 3.0.0, 2.4.1, 2.4.0 \\
        2004 & 2.2.0, 2.2.0, 2.1.2, 2.1.1        \\
        2003 & 2.1.0, 2.0.0, 1.9.6, 1.9.5        \\
        \hline
      \end{tabular}
    \end{table}
    \begin{itemize}
      \item Obsoleta $0$ vezes ao ano - intervalo entre novos lançamentos é maior que 1 ano
    \end{itemize}

  \item {\it Linguagem de programação - em qual linguagem a ferramenta é escrita:}
    \begin{itemize}
      \item C
    \end{itemize}

\end{description}

\subsection{JastAdd}

JastAdd é um sistema para análise de código-fonte através da descrição de
atributos via gramática de atributos (AG) publicado no artigo ``Extending
Attribute Grammars with Collection Attributes -- Evaluation and Applications''
do SCAM 2007 disponível em \url{http://jastadd.cs.lth.se/web}. O código-fonte
utilizado em nosso estudo obtido no site da ferramenta foi o
\texttt{jastadd2-src.zip}. Características da ferramenta:

\begin{description}

  \item {\it Lançamentos ({\it Releases}) - quantos lançamentos por ano:}
    \begin{table}[h!]
      \centering
      \begin{tabular}{| l | l |}
        \hline
        Ano  & Lançamentos                              \\
        \hline
        2016 & 2.2.2, 2.2.1, 2.2.1, 2.2.0               \\
        2015 & 2.1.13, 2.1.12, 2.1.11                   \\
        2014 & 2.1.10, 2.1.9, 2.1.8, 2.1.7              \\
        2013 & 2.1.6, 2.1.5, 2.1.4, 2.1.3, 2.1.2, 2.1.1 \\
        \hline
      \end{tabular}
    \end{table}
    \begin{itemize}
      \item Frequentemente $>=$ 3 vezes ao ano - novas versões da ferramenta são lançadas 3 ou mais vezes por ano
    \end{itemize}

  \item {\it Linguagem de programação - em qual linguagem a ferramenta é escrita:}
    \begin{itemize}
      \item Java
    \end{itemize}

\end{description}

\subsection{Sonar Qube Plug-in}

Sonar Qube Plug-in é um plugin para o SourceMeter que extende a análise de
código Java com o uso do SonarQube publicado no artigo ``SourceMeter SonarQube
plug-in'' do SCAM 2014 disponível em:
\url{http://github.com/FrontEndART/SonarQube-plug-in}. O código-fonte
utilizado em nosso estudo obtido no site da ferramenta foi o
\texttt{SonarQube-plug-in-master.zip}. Características da ferramenta:

\begin{description}

  \item {\it Lançamentos ({\it Releases}) - quantos lançamentos por ano:}
    \begin{table}[h!]
      \centering
      \begin{tabular}{| l | l |}
        \hline
        Ano  & Lançamentos       \\
        \hline
        2016 & 8.0               \\
        2015 & 7.0.5, 7.0.4, 7.0 \\
        \hline
      \end{tabular}
    \end{table}
    \begin{itemize}
      \item Frequentemente $>=$ 3 vezes ao ano - novas versões da ferramenta são lançadas 3 ou mais vezes por ano
    \end{itemize}

  \item {\it Linguagem de programação - em qual linguagem a ferramenta é escrita:}
    \begin{itemize}
      \item Java
    \end{itemize}

\end{description}

\subsection{srcML}

srcML é um formato texto para representação de código-fonte e um conjunto de
ferramentas de transformação {\it source-to-source} publicada no artigo
``Lightweight Transformation and Fact Extraction with the srcML Toolkit'' do
SCAM 2011 disponível em
\url{http://www.sdml.info/projects/srcml/trunk}\footnote{este endereço
retornou "not found" em contato com os autores por email indicaram que o
projeto foi movido para http://www.srcML.org}. O código-fonte utilizado em
nosso estudo obtido no site da ferramenta foi o \texttt{srcML-src.tar.gz}.
Características da ferramenta:

\begin{description}

  \item {\it Lançamentos ({\it Releases}) - quantos lançamentos por ano:}
    \begin{table}[h!]
      \centering
      \begin{tabular}{| l | l |}
        \hline
        Ano  & Lançamentos                                                     \\
        \hline
        2015 & 0.9.5                                                           \\
        2014 & 0.8.0, Trunk 19109c, Trunk 19109b, Trunk 19109                  \\
        2013 & Trunk 17088                                                     \\
        2012 & Trunk 13990, Trunk 13953, Trunk 13925, Trunk 13528, Trunk 12359 \\
        2011 & Trunk 8007, Trunk 7990, Trunk 7481                              \\
        \hline
      \end{tabular}
    \end{table}
    \begin{itemize}
      \item Ocasionalmente $<$ 3 vezes ao ano - novas versões da ferramenta são lançadas menos que 3 vezes ao ano
    \end{itemize}

  \item {\it Linguagem de programação - em qual linguagem a ferramenta é escrita:}
    \begin{itemize}
      \item C++
    \end{itemize}

\end{description}

\subsection{TACLE}

TACLE é um plugin do Eclipse para análise de tipo ({\it Type Analysis}) e
construção de visualizaçao de grafos de chamada ({\it Call Graph}) publicado
no artigo ``Estimating the Run-Time Progress of a Call Graph Construction
Algorithm'' do SCAM 2006 disponível em
\url{http://presto.cse.ohio-state.edu/tacle}\footnote{este link está
indisponível, por email os autores indicaram o endereço
http://web.cse.ohio-state.edu/~rountev/presto/tacle/TACLE\_Download/tacle.html}.
O código-fonte utilizado em nosso estudo obtido no site da ferramenta foi o
\texttt{tacle\_1\_2\_1\_src.zip}. Características da ferramenta:

\begin{description}

  \item {\it Lançamentos ({\it Releases}) - quantos lançamentos por ano:}
    \begin{table}[h!]
      \centering
      \begin{tabular}{| l | l |}
        \hline
        Ano  & Lançamentos  \\
        \hline
        2006 & 1.2.1, 1.2.0 \\
        2005 & 1.1.0, 1.0.0 \\
        \hline
      \end{tabular}
    \end{table}
    \begin{itemize}
      \item Obsoleta $0$ vezes ao ano - intervalo entre novos lançamentos é maior que 1 ano
    \end{itemize}

  \item {\it Linguagem de programação - em qual linguagem a ferramenta é escrita:}
    \begin{itemize}
      \item Java
    \end{itemize}

\end{description}

\subsection{WALA}

WALA é uma ferramenta de análise estática para {\it bytecode} Java publicado
no artigo ``Effective Static Analysis to Find Concurrency Bugs In Java'' do
SCAM 2010 disponível em
\url{http://wala.sourceforge.net/wiki/index.php/Main_Page}. O código-fonte
utilizado em nosso estudo obtido no site da ferramenta foi o
\texttt{WALA-R\_1.3.8.tar.gz}. Características da ferramenta:

\begin{description}

  \item {\it Lançamentos ({\it Releases}) - quantos lançamentos por ano:}
    \begin{table}[h!]
      \centering
      \begin{tabular}{| l | l |}
        \hline
        Ano  & Lançamentos                        \\
        \hline
        2016 & 1.3.9                              \\
        2015 & 1.3.8, 1.3.7                       \\
        2013 & 1.3.6, 1.3.5                       \\
        2012 & 1.3.4, 1.3.3                       \\
        2011 & 1.3.2                              \\
        2010 & 1.3.1                              \\
        2009 & 1.3, 1.2.2                         \\
        2008 & 1.2.1, 1.2, 1.1.3, 1.1.2           \\
        2007 & 1.1.1, 1.1, 1.0.04, 1.0.03, 1.0.02 \\
        2006 & 1.0                                \\
        \hline
      \end{tabular}
    \end{table}
    \begin{itemize}
      \item Ocasionalmente $<$ 3 vezes ao ano - novas versões da ferramenta são lançadas menos que 3 vezes ao ano
    \end{itemize}

  \item {\it Linguagem de programação - em qual linguagem a ferramenta é escrita:}
    \begin{itemize}
      \item Java
    \end{itemize}

\end{description}

\section{Ferramentas acadêmicas incluídas da análise (ASE)}

\subsection{composite}

composite é uma ferramenta de verificação de modelo publicado no artigo
``Action Language Verifier'' do ASE 2001 disponível em
\url{http://www.cs.ucsb.edu/~bultan/composite/}. O código-fonte utilizado em
nosso estudo obtido no site da ferramenta foi o \texttt{composite-0.4.tar.gz}.
Características da ferramenta:

\begin{description}

  \item {\it Lançamentos ({\it Releases}) - quantos lançamentos por ano:}
    \begin{itemize}
      \item Sem informação histórica de lançamentos
    \end{itemize}

  \item {\it Linguagem de programação - em qual linguagem a ferramenta é escrita:}
    \begin{itemize}
      \item C
    \end{itemize}

\end{description}

\subsection{Kiasan/Bogor}

Kiasan/Bogor é uma ferramenta de verificação de modelo publicado no artigo
``Bogor/Kiasan: A k-bounded Symbolic Execution for Checking Strong Heap
Properties of Open Systems'' do ASE 2006 disponível em
\url{http://bogor.projects.cs.ksu.edu/manual}. O código-fonte utilizado em
nosso estudo obtido no site da ferramenta foi o
\texttt{bogor-src-1.2.20061023.1.zip}. Características da ferramenta:

\begin{description}

  \item {\it Lançamentos ({\it Releases}) - quantos lançamentos por ano:}
    \begin{table}[h!]
      \centering
      \begin{tabular}{| l | l |}
        \hline
        Ano  & Lançamentos                        \\
        \hline
        2006 & 20060419, 20060405, 20060316, 20060315, 20060314, 20060313  \\
             & 20060309, 20060212                                          \\
        2005 & 20051205, 20051202, 20051201, 20051123, 20051122, 20051116, \\
             & 20051104, 20051101, 20051031, 20051021, 20051020, 20051019, \\
             & 20051018, 20051013, 20051007, 20051006, 20051001, 20050930, \\
             & 20050927, 20050920, 20050825, 20050824, 20050822, 20050815, \\
             & 20050812, 20050811, 20050810, 20050809, 20050808, 20050805, \\
             & 20050804, 20050803, 20050727, 20050725, 20050722, 20050720, \\
             & 20050715, 20050714, 20050712, 20050711, 20050710, 20050708, \\
             & 20050707, 20050706, 20050705, 20050701, 20050630, 20050628, \\
             & 20050620, 20050613, 20050609, 20050608, 20050607, 20050606, \\
             & 20050603, 20050531, 20050530, 20050526, 20050525, 20050524, \\
             & 20050520, 20050519, 20050517, 20050516, 20050514, 20050509, \\
             & 20050508, 20050507, 20050506, 20050505, 20050504, 20050503, \\
             & 20050502, 20050501, 20050418, 20050413, 20050411, 20050316, \\
             & 20050314, 20050309, 20050303, 20050301, 20050225, 20050223, \\
             & 20050218, 20050217, 20050216, 20050212, 20050211, 20050210, \\
             & 20050209, 20050207, 20050206, 20050204, 20050203, 20050202, \\
             & 20050201, 20050130, 20050129, 20050128, 20050127, 20050126, \\
             & 20050125, 20050124, 20050121, 20050119, 20050117, 20050116, \\
             & 20050114, 20050113, 20050112, 20050111                      \\
        2004 & 20041230, 20041228, 20041222, 20041221, 20041220, 20041218, \\
             & 20041217, 20041215, 20041208, 20041207, 20041203, 20041202, \\
             & 20041201, 20041130, 20041124, 20041111, 20041108, 20041106, \\
             & 20041105, 20041104, 20041103, 20041102, 20041029, 20041025, \\
             & 20041009, 20041008, 20041007, 20041006, 20041005, 20041003, \\
             & 20040930, 20040929, 20040928, 20040927, 20040922, 20040920, \\
             & 20040919, 20040916, 20040915, 20040914, 20040913, 20040912, \\
             & 20040911, 20040908, 20040907, 20040904, 20040903, 20040902, \\
             & 20040901, 20040831, 20040830, 20040824, 20040823, 20040817, \\
             & 20040815, 20040806, 20040730, 20040726, 20040725, 20040723, \\
             & 20040722, 20040721, 20040713, 20040712, 20040711, 20040710, \\
             & 20040709, 20040708, 20040707, 20040703, 20040628, 20040625, \\
             & 20040601, 20040525, 20040513, 20040511, 20040510, 20040505, \\
             & 20040504, 20040503, 20040501, 20040430, 20040421, 20040420, \\
             & 20040411, 20040410, 20040408, 20040407, 20040327, 20040326, \\
             & 20040323, 20040301, 20040224, 20040223, 20040218, 20040211, \\
             & 20040209, 20040205, 20040130, 20040129, 20040127, 20040121, \\
             & 20040119, 20040118, 20040115, 20040112                      \\
        2003 & 20031222, 20031211, 20031209, 20031208, 20031205, 20031203, \\
             & 20031202, 20031108, 20031023, 20031022, 20031018, 20030926, \\
             & 20030924, 20030920, 20030909, 20030828, 20030827, 20030826, \\
             & 20030825, 20030823, 20030820, 20030819, 20030815, 20030811, \\
             & 20030809, 20030808, 20030807, 20030730, 20030729, 20030728, \\
             & 20030726, 20030725, 20030724, 20030707, 20030705, 20030704, \\
             & 20030703, 20030625, 20030624, 20030621, 20030620, 20030619, \\
             & 20030618, 20030616                                          \\
        \hline
      \end{tabular}
    \end{table}
    \begin{itemize}
      \item Obsoleta $0$ vezes ao ano - intervalo entre novos lançamentos é maior que 1 ano
    \end{itemize}

  \item {\it Linguagem de programação - em qual linguagem a ferramenta é escrita:}
    \begin{itemize}
      \item Java
    \end{itemize}

\end{description}

\subsection{PtYasm}

PtYasm é uma ferramenta de verificação de modelo publicado
no artigo ``PtYasm: Software Model Checking with Proof Templates'' do
ASE 2008 disponível em
\url{www.cs.toronto.edu/~tomhart/ptyasm}. O código-fonte
utilizado em nosso estudo obtido no site da ferramenta foi o
\texttt{ptyasm.april2008.tgz}. Características da ferramenta:

\begin{description}

  \item {\it Lançamentos ({\it Releases}) - quantos lançamentos por ano:}
    \begin{itemize}
      \item Sem informação histórica de lançamentos
    \end{itemize}

  \item {\it Linguagem de programação - em qual linguagem a ferramenta é escrita:}
    \begin{itemize}
      \item Java
    \end{itemize}

\end{description}

\subsection{ReAssert}

ReAssert é uma ferramenta de localização de falhas em testes e refatoração
desenvolvido como plugin Ecipse publicado no artigo ``ReAssert: Suggesting
Repairs for Broken Unit Tests'' do ASE 2009 disponível em
\url{http://mir.cs.illinois.edu/reassert}. O código-fonte utilizado em nosso
estudo obtido no site da ferramenta foi o \texttt{ReAssert\_0.4.1-src.zip}.
Características da ferramenta:

\begin{description}

  \item {\it Lançamentos ({\it Releases}) - quantos lançamentos por ano:}
    \begin{table}[h!]
      \centering
      \begin{tabular}{| l | l |}
        \hline
        Ano  & Lançamentos     \\
        \hline
        2009 & 0.1.0, 0.2.0    \\
        \hline
      \end{tabular}
    \end{table}
    \begin{itemize}
      \item Obsoleta $0$ vezes ao ano - intervalo entre novos lançamentos é maior que 1 ano
    \end{itemize}

  \item {\it Linguagem de programação - em qual linguagem a ferramenta é escrita:}
    \begin{itemize}
      \item Java
    \end{itemize}

\end{description}

\subsection{GUIZMO}

GUIZMO é uma ferramenta de inferência de layout publicado
no artigo ``Model-driven reverse engineering of legacy graphical user interfaces'' do
ASE 2010 disponível em
\url{http://modelum.es/trac/guizmo/}. O código-fonte
utilizado em nosso estudo obtido no site da ferramenta foi o
\texttt{guizmo-master.zip}. Características da ferramenta:

\begin{description}

  \item {\it Lançamentos ({\it Releases}) - quantos lançamentos por ano:}
    \begin{itemize}
      \item Sem informação histórica de lançamentos
    \end{itemize}

  \item {\it Linguagem de programação - em qual linguagem a ferramenta é escrita:}
    \begin{itemize}
      \item Java
    \end{itemize}

\end{description}


\subsection{JFlow}

JFlow é uma ferramenta de transformação {\it source-to-source} publicado
no artigo ``JFlow: Practical refactorings for flow-based parallelism'' do
ASE 2013 disponível em
\url{http://vazexqi.github.io/JFlow/}. O código-fonte
utilizado em nosso estudo obtido no site da ferramenta foi o
\texttt{vazexqi-JFlow-7cd7eaf.tar.gz}. Características da ferramenta:

\begin{description}

  \item {\it Lançamentos ({\it Releases}) - quantos lançamentos por ano:}
    \begin{table}[h!]
      \centering
      \begin{tabular}{| l | l |}
        \hline
        Ano  & Lançamentos                        \\
        \hline
        2012 & 8290c72, 99b5ecd, 32e0742, f62cb44, bfa7121 \\
        \hline
      \end{tabular}
    \end{table}
    \begin{itemize}
      \item Obsoleta $0$ vezes ao ano - intervalo entre novos lançamentos é maior que 1 ano
    \end{itemize}

  \item {\it Linguagem de programação - em qual linguagem a ferramenta é escrita:}
    \begin{itemize}
      \item Java
    \end{itemize}

\end{description}



\subsection{CSeq}

CSeq é uma ferramenta de transformação {\it source-to-source} para programas
concorrentes publicado no artigo ``CSeq: A concurrency pre-processor for
sequential C verification tools'' do ASE 2013 disponível em
\url{http://users.ecs.soton.ac.uk/gp4/cseq/files/cseq-0.5.zip}. O código-fonte
utilizado em nosso estudo obtido no site da ferramenta foi o
\texttt{cseq-0.5.zip}. Características da ferramenta:

\begin{description}

  \item {\it Lançamentos ({\it Releases}) - quantos lançamentos por ano:}
    \begin{itemize}
      \item Sem informação histórica de lançamentos
    \end{itemize}

  \item {\it Linguagem de programação - em qual linguagem a ferramenta é escrita:}
    \begin{itemize}
      \item C
    \end{itemize}

\end{description}

\subsection{Lotrack}

Lotrack é uma ferramenta de análise estática de configuração publicado no
artigo ``Tracking load-time configuration options'' do ASE 2014 disponível em
\url{https://github.com/MaxLillack/Lotrack}. O código-fonte utilizado em nosso
estudo obtido no site da ferramenta foi o \texttt{Lotrack-master.zip}.
Características da ferramenta:

\begin{description}

  \item {\it Lançamentos ({\it Releases}) - quantos lançamentos por ano:}
    \begin{itemize}
      \item Sem informação histórica de lançamentos
    \end{itemize}

  \item {\it Linguagem de programação - em qual linguagem a ferramenta é escrita:}
    \begin{itemize}
      \item Java
    \end{itemize}

\end{description}

\subsection{MPAnalyzer}

MPAnalyzer é uma ferramenta de análise de padrões publicado no artigo
``MPAnalyzer: a tool for finding unintended inconsistencies in program source
code'' do ASE 2014 disponível em
\url{https://github.com/YoshikiHigo/MPAnalyzer}. O código-fonte utilizado em
nosso estudo obtido no site da ferramenta foi o \texttt{MPAnalyzer-master.zip}.
Características da ferramenta:

\begin{description}

  \item {\it Lançamentos ({\it Releases}) - quantos lançamentos por ano:}
    \begin{itemize}
      \item Sem informação histórica de lançamentos
    \end{itemize}

  \item {\it Linguagem de programação - em qual linguagem a ferramenta é escrita:}
    \begin{itemize}
      \item Java
    \end{itemize}

\end{description}

\subsection{GumTree}

GumTree é uma ferramenta de análise de código-fonte e comparação de mudanças
publicado no artigo ``Fine-grained and Accurate Source Code Differencing'' do
ASE 2014 disponível em \url{https://github.com/jrfaller/gumtree}. O
código-fonte utilizado em nosso estudo obtido no site da ferramenta foi o
\texttt{gumtree-2.0.0.tar.gz}. Características da ferramenta:

\begin{description}

  \item {\it Lançamentos ({\it Releases}) - quantos lançamentos por ano:}
    \begin{table}[h!]
      \centering
      \begin{tabular}{| l | l |}
        \hline
        Ano  & Lançamentos                        \\
        \hline
        2015 & 2.1.0, 2.0.0                      \\
        2013 & 1.0.0                             \\
        \hline
      \end{tabular}
    \end{table}
    \begin{itemize}
      \item Ocasionalmente $<$ 3 vezes ao ano - novas versões da ferramenta são lançadas menos que 3 vezes ao ano
    \end{itemize}

  \item {\it Linguagem de programação - em qual linguagem a ferramenta é escrita:}
    \begin{itemize}
      \item Java
    \end{itemize}

\end{description}

\subsection{SPARTA}

SPARTA é uma ferramenta de análise estática de segurança pra detecção de {\it
malware} publicado no artigo ``Static Analysis of Implicit Control Flow:
Resolving Java Reflection and Android Intents (T)'' do ASE 2015 disponível em
\url{http://types.cs.washington.edu/sparta/}. O código-fonte utilizado em nosso
estudo obtido no site da ferramenta foi o \texttt{sparta-sparta-1.0.2.tar.gz}.
Características da ferramenta:

\begin{description}

  \item {\it Lançamentos ({\it Releases}) - quantos lançamentos por ano:}
    \begin{table}[h!]
      \centering
      \begin{tabular}{| l | l |}
        \hline
        Ano  & Lançamentos                        \\
        \hline
        2016 & 1.0.2                              \\
        2015 & 1.0.1                              \\
        2014 & 1.0.0, 0.9.9, 0.9.8, 0.9.7         \\
        2013 & 0.9.4, 0.9.3, 0.9.2, 0.9.1, 0.9.0, 0.8.1, 0.8.0 \\
        2012 & 0.7                                \\
        \hline
      \end{tabular}
    \end{table}
    \begin{itemize}
      \item Ocasionalmente $<$ 3 vezes ao ano - novas versões da ferramenta são lançadas menos que 3 vezes ao ano
    \end{itemize}

  \item {\it Linguagem de programação - em qual linguagem a ferramenta é escrita:}
    \begin{itemize}
      \item Java
    \end{itemize}

\end{description}

\section{Ferramentas da indústria incluídas da análise (NIST)}

\subsection{Clang Static Analyzer}

O {\it Clang Static Analyzer} é uma ferramenta de análise de código-fonte para
localização de bugs em códigos C, C++, e Objective-C disponível em
\url{http://clang-analyzer.llvm.org}. É distribuído junto ao código do próprio
projeto Clang\footnote{http://clang.llvm.org} e em nosso estudo utilizamos o
código em \texttt{cfe-3.7.1.src.tar.xz}. Características da ferramenta:

\begin{description}

  \item {\it Lançamentos ({\it Releases}) - quantos lançamentos por ano:}
    \begin{table}[h!]
      \centering
      \begin{tabular}{| l | l |}
        \hline
        Ano  & Lançamentos                              \\
        \hline
        2016 & 3.8.0, 3.7.1                             \\
        2015 & 3.7.0, 3.6.2, 3.6.1, 3.6.0, 3.5.2, 3.5.1 \\
        2014 & 3.5.0, 3.4.2, 3.4.1, 3.4                 \\
        2013 & 3.3                                      \\
        2012 & 3.2, 3.1                                 \\
        2011 & 3.0, 2.9                                 \\
        2010 & 2.8, 2.7                                 \\
        2009 & 2.6, 2.5                                 \\
        2008 & 2.4, 2.3, 2.2                            \\
        2007 & 2.1, 2.0                                 \\
        2006 & 1.9, 1.8, 1.7                            \\
        2005 & 1.6, 1.5                                 \\
        2004 & 1.4, 1.3, 1.2                            \\
        2003 & 1.1, 1.0                                 \\
        \hline
      \end{tabular}
    \end{table}
    \begin{itemize}
      \item Frequentemente $>=$ 3 vezes ao ano - novas versões da ferramenta são lançadas 3 ou mais vezes por ano
    \end{itemize}

  \item {\it Linguagem de programação - em qual linguagem a ferramenta é escrita:}
    \begin{itemize}
      \item C++
    \end{itemize}

\end{description}

\subsection{Closure Compiler}

{\it Closure Compiler} é um compilador que traduz código JavaScript em outro
JavaScript melhor e mais otimizado, está disponível em
\url{https://developers.google.com/closure/compiler}\footnote{O código fonte do
Closure Compiler pode ser obtido em:
http://github.com/google/closure-compiler} e foi utilizado em nosso estudo o
seguinte lançamento
\texttt{closure-compiler-closure-compiler-parent-v20160619.tar.gz}.
Características da ferramenta:

\begin{description}

  \item {\it Lançamentos ({\it Releases}) - quantos lançamentos por ano:}
    \begin{table}[h!]
      \centering
      \begin{tabular}{| l | l |}
        \hline
        Ano  & Lançamentos                              \\
        \hline
        2016 & v20160619, v20160517, v20160315, v20160208, v20160125 \\
        2015 & v20151216, v20151015, v20150920, v20150901, v20150729, v20150609, \\
             & v20150505, v20150315, v20150126 \\
        2014 & v20141215, v20141120, v20141023, v20140923, v20140814, v20140730, \\
             & v20140625, v20140508, v20140407, v20140303, v20140110 \\
        2013 & v20131118, v20131014, v20130823, v20130722, v20130603, v20130411, \\
             & v20130227, v20110811, v20110322, v20110405, v20110119, v20111003, \\
             & v20111114, v20112023, v20120305, v20120430, v20120711, v20121212, \\
             & v20120917 \\
        \hline
      \end{tabular}
    \end{table}
    \begin{itemize}
      \item Frequentemente $>=$ 3 vezes ao ano - novas versões da ferramenta são lançadas 3 ou mais vezes por ano
    \end{itemize}

  \item {\it Linguagem de programação - em qual linguagem a ferramenta é escrita:}
    \begin{itemize}
      \item Java
    \end{itemize}

\end{description}

\subsection{Cppcheck}

Ferramenta de análise estática de código C/C++ para checagem de vazamento de
memória, erros de alocação, entre outras falhas. Disponível em
\url{http://sourceforge.net/projects/cppcheck}. Em nosso estudo utilizamos o
código em \texttt{cppcheck-1.72.tar.bz2}. Características da ferramenta:

\begin{description}

  \item {\it Lançamentos ({\it Releases}) - quantos lançamentos por ano:}
    \begin{table}[h!]
      \centering
      \begin{tabular}{| l | l |}
        \hline
        Ano  & Lançamentos                                                \\
        \hline
        2016 & 1.74, 1.73, 1.72                                           \\
        2015 & 1.71, 1.70, 1.69 1.68                                      \\
        2014 & 1.67, 1.66, 1.65, 1.64, 1.63                               \\
        2013 & 1.62, 1.61, 1.60.1, 1.60, 1.59, 1.58                       \\
        2012 & 1.57, 1.56, 1.55, 1.54, 1.53                               \\
        2011 & 1.52, 1.51, 1.50, 1.49, 1.48, 1.47                         \\
        2010 & 1.46.1, 1.46, 1.45, 1.44, 1.43, 1.42, 1.41, 1.40           \\
        2009 & 1.39, 1.38, 1.37, 1.36, 1.35, 1.34, 1.33, 1.32, 1.31, 1.30 \\
        \hline
      \end{tabular}
    \end{table}
    \begin{itemize}
      \item Frequentemente $>=$ 3 vezes ao ano - novas versões da ferramenta são lançadas 3 ou mais vezes por ano
    \end{itemize}

  \item {\it Linguagem de programação - em qual linguagem a ferramenta é escrita:}
    \begin{itemize}
      \item C++
    \end{itemize}

\end{description}

\subsection{CQual}

Ferramenta de análise de typo ({\it type-based analysis}) que fornece um
mecanismo leve e prático para especificação e verificação de propriedades de
programas C. Disponível em \url{http://www.cs.umd.edu/~jfoster/cqual}. Em
nosso estudo utilizamos o código em \texttt{cqual-0.981.tar.gz}.
Características da ferramenta:

\begin{description}

  \item {\it Lançamentos ({\it Releases}) - quantos lançamentos por ano:}
    \begin{table}[h!]
      \centering
      \begin{tabular}{| l | l |}
        \hline
        Ano  & Lançamentos  \\
        \hline
        2004 & 0.981, 0.991 \\
        2003 & 0.98, 0.99   \\
        \hline
      \end{tabular}
    \end{table}
    \begin{itemize}
      \item Obsoleta $0$ vezes ao ano - intervalo entre novos lançamentos é maior que 1 ano
    \end{itemize}

  \item {\it Linguagem de programação - em qual linguagem a ferramenta é escrita:}
    \begin{itemize}
      \item C
    \end{itemize}

\end{description}

\subsection{FindBugs}

Uma ferramenta para localização de bugs em código Java disponível em
\url{http://findbugs.sourceforge.net}. Em nosso estudo utilizamos o código em
\texttt{findbugs-3.0.1-source.zip}. Características da ferramenta:

\begin{description}

  \item {\it Lançamentos ({\it Releases}) - quantos lançamentos por ano:}
    \begin{table}[h!]
      \centering
      \begin{tabular}{| l | l |}
        \hline
        Ano  & Lançamentos                \\
        \hline
        2015 & 3.0.1                      \\
        2014 & 3.0.0                      \\
        2013 & 2.0.3                      \\
        2012 & 2.0.2, 2.0.1               \\
        2011 & 2.0.0                      \\
        2009 & 1.3.9, 1.3.8               \\
        2008 & 1.3.7, 1.3.6, 1.3.5, 1.3.4 \\
        2007 & 1.2.1                      \\
        \hline
      \end{tabular}
    \end{table}
    \begin{itemize}
      \item Ocasionalmente $<$ 3 vezes ao ano - novas versões da ferramenta são lançadas menos que 3 vezes ao ano
    \end{itemize}

  \item {\it Linguagem de programação - em qual linguagem a ferramenta é escrita:}
    \begin{itemize}
      \item Java
    \end{itemize}

\end{description}

\subsection{FindSecurityBugs}

Plugin do FindBugs para auditoria de segurança em aplicações web Java,
disponível em \url{http://find-sec-bugs.github.io}.  O código-fonte utilizado
em nosso estudo obtido no site da ferramenta foi o
\texttt{findsecbugs-plugin-1.4.5-sources.jar}. Características da ferramenta:

\begin{description}

  \item {\it Lançamentos ({\it Releases}) - quantos lançamentos por ano:}
    \begin{table}[h!]
      \centering
      \begin{tabular}{| l | l |}
        \hline
        Ano  & Lançamentos                                     \\
        \hline
        2016 & 1.4.6, 1.4.5                                    \\
        2015 & 1.4.4, 1.4.3, 1.4.2, 1.4.1, 1.4.0, 1.3.1, 1.3.0 \\
        2014 & 1.2.1                                           \\
        2013 & 1.2.0, 1.1.0                                    \\
        2012 & 1.0.0                                           \\
        \hline
      \end{tabular}
    \end{table}
    \begin{itemize}
      \item Frequentemente $>=$ 3 vezes ao ano - novas versões da ferramenta são lançadas 3 ou mais vezes por ano
    \end{itemize}

  \item {\it Linguagem de programação - em qual linguagem a ferramenta é escrita:}
    \begin{itemize}
      \item Java
    \end{itemize}

\end{description}

\subsection{Jlint}

Uma ferramenta para verificaçao de código Java em busca de bugs,
inconsistências e problemas de sincronização disponível em
\url{http://sourceforge.net/projects/jlint}.  O código-fonte utilizado em
nosso estudo obtido no site da ferramenta foi o \texttt{jlint-3.1.2.zip}.
Características da ferramenta:

\begin{description}

  \item {\it Lançamentos ({\it Releases}) - quantos lançamentos por ano:}
    \begin{table}[h!]
      \centering
      \begin{tabular}{| l | l |}
        \hline
        Ano  & Lançamentos \\
        \hline
        2011 & 3.1.2       \\
        2010 & 3.1.1       \\
        2006 & 3.1         \\
        2004 & 3.0         \\
        \hline
      \end{tabular}
    \end{table}
    \begin{itemize}
      \item Obsoleta $0$ vezes ao ano - intervalo entre novos lançamentos é maior que 1 ano
    \end{itemize}

  \item {\it Linguagem de programação - em qual linguagem a ferramenta é escrita:}
    \begin{itemize}
      \item C++
    \end{itemize}

\end{description}

\subsection{Pixy}

Ferramenta de análise estática de código PHP para verificação de
vulnerabilidades de segurança. Disponível em
\url{http://github.com/oliverklee/pixy}. O código-fonte utilizado em nosso
estudo obtido no site da ferramenta foi o \texttt{pixy-master.zip}.
Características da ferramenta:

\begin{description}

  \item {\it Lançamentos ({\it Releases}) - quantos lançamentos por ano:}
    \begin{table}[h!]
      \centering
      \begin{tabular}{| l | l |}
        \hline
        Ano  & Lançamentos \\
        \hline
        2012 & 3.0.3       \\
        \hline
      \end{tabular}
    \end{table}
    \begin{itemize}
      \item Obsoleta $0$ vezes ao ano - intervalo entre novos lançamentos é maior que 1 ano
    \end{itemize}

  \item {\it Linguagem de programação - em qual linguagem a ferramenta é escrita:}
    \begin{itemize}
      \item Java
    \end{itemize}

\end{description}

\subsection{PMD}

Ferramenta de análise de código-fonte para localização falhas comuns de
programação com suporte a várias linguagens, disponível em
\url{http://pmd.github.io}.  O código-fonte utilizado em nosso estudo obtido
no site da ferramenta foi o \texttt{pmd-src-5.4.1.zip}. Características da
ferramenta:

\begin{description}

  \item {\it Lançamentos ({\it Releases}) - quantos lançamentos por ano:}
    \begin{table}[h!]
      \centering
      \begin{tabular}{| l | l |}
        \hline
        Ano  & Lançamentos                                                   \\
        \hline
        2016 & 5.5.0, 5.4.2, 5.3.7                                           \\
        2015 & 5.4.1, 5.4.0, 5.3.6, 5.3.5, 5.3.4, 5.3.3, 5.3.2, 5.3.1, 5.3.0 \\
        2014 & 5.2.3, 5.2.2, 5.2.1, 5.2.0, 5.1.3, 5.1.2, 5.1.1, 5.1.0        \\
        2013 & 5.0.5, 5.0.4, 5.0.3, 5.0.2                                    \\
        2012 & 5.0.1, 5.0.0                                                  \\
        2011 & 4.3.0, 4.2.6                                                  \\
        2009 & 4.2.5                                                         \\
        \hline
      \end{tabular}
    \end{table}
    \begin{itemize}
      \item Frequentemente $>=$ 3 vezes ao ano - novas versões da ferramenta são lançadas 3 ou mais vezes por ano
    \end{itemize}

  \item {\it Linguagem de programação - em qual linguagem a ferramenta é escrita:}
    \begin{itemize}
      \item Java
    \end{itemize}

\end{description}

\subsection{RATS}

Ferramenta de análise estática para auditoria de segurança disponível em
\url{http://code.google.com/archive/p/rough-auditing-tool-for-security}. O
código-fonte utilizado em nosso estudo obtido no site da ferramenta foi o
\texttt{rats-2.4.tgz}. Características da ferramenta:

\begin{description}

  \item {\it Lançamentos ({\it Releases}) - quantos lançamentos por ano:}
    \begin{table}[h!]
      \centering
      \begin{tabular}{| l | l |}
        \hline
        Ano  & Lançamentos \\
        \hline
        2013 & 2.4         \\
        2009 & 2.3         \\
        ??   & 1.5         \\
        \hline
      \end{tabular}
    \end{table}
    \begin{itemize}
      \item Obsoleta $0$ vezes ao ano - intervalo entre novos lançamentos é maior que 1 ano
    \end{itemize}

  \item {\it Linguagem de programação - em qual linguagem a ferramenta é escrita:}
    \begin{itemize}
      \item C
    \end{itemize}

\end{description}

\subsection{Smatch}

Ferramenta de análise estática para detecção de erros no Kernel disponível em
\url{http://smatch.sourceforge.net}. O código-fonte utilizado em nosso estudo
obtido no site da ferramenta foi o \texttt{smatch.git}. Características da
ferramenta:

\begin{description}

  \item {\it Lançamentos ({\it Releases}) - quantos lançamentos por ano:}
    \begin{table}[h!]
      \centering
      \begin{tabular}{| l | l |}
        \hline
        Ano  & Lançamentos      \\
        \hline
        2015 & 1.60             \\
        2013 & 1.59, 1.58, 1.57 \\
        2012 & 1.56             \\
        2010 & 1.55, 1.54       \\
        2009 & 1.53, 1.52, 1.51 \\
        \hline
      \end{tabular}
    \end{table}
    \begin{itemize}
      \item Ocasionalmente $<$ 3 vezes ao ano - novas versões da ferramenta são lançadas menos que 3 vezes ao ano
    \end{itemize}

  \item {\it Linguagem de programação - em qual linguagem a ferramenta é escrita:}
    \begin{itemize}
      \item C
    \end{itemize}

\end{description}

\subsection{Splint}

Splint is a tool for statically checking C programs for security
vulnerabilities and coding mistakes Ferramenta para verificação de programas
por vulnerabilidades de segurança e erros de código. Disponível em
\url{http://www.splint.org}. O código-fonte utilizado em nosso estudo obtido
no site da ferramenta foi o \texttt{splint-3.1.2.src.tgz}. Características da
ferramenta:

\begin{description}

  \item {\it Lançamentos ({\it Releases}) - quantos lançamentos por ano:}
    \begin{table}[h!]
      \centering
      \begin{tabular}{| l | l |}
        \hline
        Ano  & Lançamentos                        \\
        \hline
        2007 & 3.1.2                              \\
        2003 & 3.1.0                              \\
        ??   & 3.0.1.6                            \\
        2002 & 3.0.1.5                            \\
        ??   & 3.0.1.4, 3.0.1, 3.0.0.19, 3.0.0.18 \\
        ??   & 3.0.0.17, 3.0.0.15, 3.0.0.14       \\
        2001 & 3.0.0.13                           \\
        \hline
      \end{tabular}
    \end{table}
    \begin{itemize}
      \item Obsoleta $0$ vezes ao ano - intervalo entre novos lançamentos é maior que 1 ano
    \end{itemize}

  \item {\it Linguagem de programação - em qual linguagem a ferramenta é escrita:}
    \begin{itemize}
      \item C
    \end{itemize}

\end{description}

\subsection{UNO}

Uma ferramenta de análise de código-fonte para detecção de defeitos.
Disponível em \url{http://spinroot.com/uno}. O código-fonte utilizado em nosso
estudo obtido no site da ferramenta foi o \texttt{uno\_v213.tar.gz}.
Características da ferramenta:

\begin{description}

  \item {\it Lançamentos ({\it Releases}) - quantos lançamentos por ano:}
    \begin{table}[h!]
      \centering
      \begin{tabular}{| l | l |}
        \hline
        Ano  & Lançamentos                       \\
        \hline
        2007 & 2.13, 2.12, 2.11                  \\
        2006 & 2.9-2.10                          \\
        2005 & 2.8, 2.7, 2.6, 2.5, 2.4           \\
        2004 & 2.3, 2.2, 2.1, 2.0, 1.7           \\
        2003 & 1.8, 1.6, 1.5, 1.4, 1.3, 1.2, 1.1 \\
        \hline
      \end{tabular}
    \end{table}
    \begin{itemize}
      \item Obsoleta $0$ vezes ao ano - intervalo entre novos lançamentos é maior que 1 ano
    \end{itemize}

  \item {\it Linguagem de programação - em qual linguagem a ferramenta é escrita:}
    \begin{itemize}
      \item C
    \end{itemize}

\end{description}

\clearpage

\subsection{WAP}

Ferramenta para análise estática de código-fonte e mineraçao de dados para
detectar e corrigir vulnerabilidades em aplicações web. Disponível em
\url{http://awap.sourceforge.net}. O código-fonte utilizado em nosso estudo
obtido no site da ferramenta foi o \texttt{wap-2.1.tar.gz}. Características da
ferramenta:

\begin{description}

  \item {\it Lançamentos ({\it Releases}) - quantos lançamentos por ano:}
    \begin{table}[h!]
      \centering
      \begin{tabular}{| l | l |}
        \hline
        Ano  & Lançamentos                                 \\
        \hline
        2015 & 2.1, 2.0.5, 2.0.4, 2.0.3, 2.0.2, 2.0.1, 2.0 \\
        \hline
      \end{tabular}
    \end{table}
    \begin{itemize}
      \item Frequentemente $>=$ 3 vezes ao ano - novas versões da ferramenta são lançadas 3 ou mais vezes por ano
    \end{itemize}

  \item {\it Linguagem de programação - em qual linguagem a ferramenta é escrita:}
    \begin{itemize}
      \item Java
    \end{itemize}

\end{description}

