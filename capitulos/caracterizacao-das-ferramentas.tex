\xchapter{Caracterização das ferramentas}{}

bla bla ...

\subsection{Caracterização das ferramentas} \label{caracterizacao-das-ferramentas}

\citeonline{Novak2010} através de um estudo para construção de uma taxonomia
para ferramentas de análise estática propuseram uma classificação a partir de
uma série de categorias.

\begin{description}

  \item {\it Entrada - quais tipos de arquivos podem ser carregados na ferramenta:}
    \begin{itemize}
      \item Código-fonte - arquivos de código texto podem ser carregados
      \item Byte code - arquivos com Java Byte Code ou Microsoft
      \item Linguagem intermediária (MSIL) pode ser carregada
    \end{itemize}

  \item {\it Lançamentos ({\it Releases}) - quantos lançamentos por ano:}
    \begin{itemize}
      \item Frequentemente $>=$ 3 vezes ao ano - novas versões da ferramenta são lançadas 3 ou mais vezes por ano
      \item Ocasionalmente $<$ 3 vezes ao ano - novas versões da ferramenta são lançadas menos que 3 vezes ao ano
      \item Obsoleta 0 vezes ao ano - intervalo entre novos lançamentos é maior que 1 ano
    \end{itemize}

  \item {\it Linguagens suportadas - quais linguagens de programação a ferramenta suporta:}
    \begin{itemize}
      \item .NET - todas as linguagens compiladas em bibliotecas ou programas no framework .NET
      \item VB .NET - suporta VB.NET
      \item C\# - suporta C\#
      \item Java - suporta linguagem de programação Java
      \item C, C++ - suporta linguagem de programação C ou C++
    \end{itemize}

  \item {\it Tecnologia - quais tecnologias são usadas para procurar erros no código:}
    \begin{itemize}
      \item Dataflow - busca por erros com dataflow
      \item Sintaxe - busca por errors de sintaxe e correctness
      \item Prova de teoremas - procurar erros em provar diferentes teoremas
      \item Verificação de modelos - procurar erros com verificação de modelo
    \end{itemize}

  \item {\it Regras - conjunto de regras, quais são suportadas por diferentes código estático:}
    \begin{itemize}
      \item Estilo - inspeciona a aparência do código-fonte
      \item Naming - checa se as varáveis são nomeadas corretamente (ortografia, padroes de nomenclatura, ...)
      \item Geral - regras gerais de analise estática de código
      \item Concorrencia - erros com execução de código de concorrente
      \item Exceções - erros lançando ou não exceções
      \item Performance - erros de performance das aplicações
      \item Interoperabilidade - erros de comportamento comum
      \item Segurança - erros que podem impactar na segurança da aplicação
      \item SQL - procurar por "SQL injections" e outros erros de SQL
      \item Buffer overflow - erros de segurança, que explorar buffer overflow
      \item Manutenabilidade - regras para melhor manutenabilidade da aplicação
    \end{itemize}

  \item {\it Configurabilidade - habilidade de configurar a ferramenta:}
    \begin{itemize}
      \item Documento texto - comfiguração é feita via documento texto
      \item XML - configuração pe feita por documento XML
      \item GUI - configuraçãi é feita via interface gráfica
      \item Ruleset - ferramenta pode ligar/desligar conjunto de regras
    \end{itemize}

  \item {\it Extensibilidade - se a ferramenta pode ser extendida com regras próprias:}
    \begin{itemize}
      \item Possível - é possível extender
      \item Não possível - não é possível extender
    \end{itemize}

  \item {\it Disponibilidade - de que forma a ferramenta está disponível:}
    \begin{itemize}
      \item Código Aberto ({\it Open Source}) - a ferrament é livre e o código-fonte está disponível
      \item Grátis ({\it Free}) - a ferramenta é grátis mas o código-fonte não está disponível
      \item Comercial - a ferramenta está disponível mediante pagamento
    \end{itemize}

  \item {\it Experiência do usuário - de que forma a ferramenta pode ser usada, como é oferecida:}
    \begin{itemize}
      \item Integração com ambiente - como a ferramenta é integrada ao ambiente de trabalho
      \item Localização automática de erros no código - quando a ferramenta encontra um erro, ela leva ao local do erro
      \item Ajuda abrangente sobre falhas - se a ferramenta oferece ajuda na resolução de erros
      \item Interface de usuário - disponibilidade de uma interface de usuário
      \item Linha de comando - ela pode ser executada via linha de comando
      \item GUI - a ferramente pode ser executada em uma interface gráfica (GUI)
    \end{itemize}

  \item {\it Saída - representação dos resultados da ferramenta:}
    \begin{itemize}
      \item Arquivo texto - ferramenta pode apresentar resultados em arquivos texto
      \item Lista - ferramenta pode apresentar resultados numa interface de usuário customizada controlada em GUI
      \item Arquivo XML - ferramenta pode apresentar resultados em dados XML
      \item Arquivo HTML - ferramenta pode apresentar resultados em dados HTML
    \end{itemize}

\end{description}

Iremos utilizar algumas destas categorias na caracterização das ferramentas
selecionadas neste estudo, e vamos também caracterizar em relação à linguagem de
programação a qual foi escrita.

O Apêndice \ref{caracterizacao-ferramentas} traz uma caracterização inicial
das ferramentas segundo às categorias acima.
