\xchapter{Caracterização das ferramentas de análise estática}
{Este capítulo apresenta a caracterização das ferramentas selecionadas a partir
da revisão estruturada.}
\label{caracterizacao-ferramentas}

(pendente)

\section{Seleção das ferramentas}

\subsection{Revisão estruturada}

Selecionamos a conferência SCAM - Source Code Analysis and Manipulation Working
Conference\footnote{http://www.ieee-scam.org} e a conferência ASE - Automated
Software Engineering\footnote{http://ase-conferences.org}, por serem ambas
conferências com largo histórico de publicação sobre análise de programas.

Esta primeira atividade de busca passa então por todas as ediçoes destas duas
conferências até o ano de 2015, uma lista
completa e o endereço de cada edição onde os artigos foram obtidos está
documentado no Apêndice \ref{edicoes-conferencias}, lá está indicado os
endereços da conferência nos portais do IEEE Xplore ou do ACM Digital Library
onde os artigos em PDF estão disponíveis para download.

O resultado desta primeira atividade foi um total de 1879 artigos obtidos, 346
artigos do SCAM e 1533 artigos do ASE, os artigos foram documentados e estão disponíveis num arquivo externo
{\it artigos.ods}\footnote{http://github.com/joenio/dissertacao-ufba-2016/blob/master/revisao-estruturada/artigos.ods}.

A segunda atividade da revisão estruturada realizada em cima deste conjunto inicial
é um filtro realizado de forma automática com o auxílio do script
{\it
filter}\footnote{http://github.com/joenio/dissertacao-ufba-2016/blob/master/revisao-estruturada/filter}
escrito especialmente para este estudo. Este script busca em todo o conteúdo
dos artigos os seguintes termos:

\begin{verbatim}
  "tool" OU "framework"; E
  "download" OU "available"; E
  "http" OU "ftp"; E
  "static analysis" OU "parser".
\end{verbatim}


A segunda atividade da revisão estruturada realizada em cima deste conjunto de
1879 artigos realizada automaticamente pelo script filter
resultou num total de 436 contendo os termos acima, deste
total 155 artigos do SCAM e 281 do ASE. Os artigos selecionados nesta atividade
estão documentados no arquivo {\it artigos.ods} indicados na coluna ``Filtro(2)''
quando for o caso.

Estes 436 artigos foram submetidos à terceira e última atividade da revisão estruturada,
seleção de artigos, nesta seleção, ...

Ferramentas que sejam mais abrangentes do
que apenas análise estática mas que contenham esta função em seu conjunto
também serão selecionadas.

Ao final de todas as atividades da revisão estruturada encontramos um total de
21 ferramentas de análise estática, sendo 10 da conferência SCAM e 11 da
conferência ASE. As Tabelas \ref{artigos-do-scam} e \ref{artigos-do-ase}
apresentam um resumo desta revisão estruturada.

\begin{table}[H]
\caption{Total de artigos analisados em cada edição do SCAM}
\centering
\begin{tabular}{| l | c | c | c |}
\hline
Edição & (1) Busca & (2) Filtro & (3) Seleção \\
\hline
SCAM 2001 & 23    & 6         & -           \\
SCAM 2002 & 18    & 6         & -           \\
SCAM 2003 & 21    & 8         & -           \\
SCAM 2004 & 17    & 3         & -           \\
SCAM 2005 & 19    & 7         & -           \\
SCAM 2006 & 22    & 10        & 2           \\
SCAM 2007 & 23    & 7         & 1           \\
SCAM 2008 & 29    & 14        & -           \\
SCAM 2009 & 20    & 10        & -           \\
SCAM 2010 & 21    & 15        & 1           \\
SCAM 2011 & 21    & 10        & 2           \\
SCAM 2012 & 22    & 12        & 3           \\
SCAM 2013 & 24    & 13        & -           \\
SCAM 2014 & 36    & 16        & 1           \\
SCAM 2015 & 30    & 18        & -           \\
\hline
Total     & 346   & 155       & 10          \\
\hline
\end{tabular}
\label{artigos-do-scam}
\end{table}

\begin{table}[H]
\caption{Total de artigos analisados em cada edição do ASE}
\centering
\begin{tabular}{| l | c | c | c |}
\hline
Edição & (1) Busca & (2) Filtro & (3) Seleção \\
\hline
ASE 1991 & 28    & -         & -           \\
ASE 1992 & 25    & -         & -           \\
ASE 1993 & 21    & -         & -           \\
ASE 1994 & 23    & -         & -           \\
ASE 1995 & 23    & -         & -           \\
ASE 1996 & 15    & -         & -           \\
ASE 1997 & 47    & 1         & -           \\
ASE 1998 & 44    & 4         & -           \\
ASE 1999 & 50    & -         & -           \\
ASE 2000 & 44    & 2         & -           \\
ASE 2001 & 68    & 7         & 1           \\
ASE 2002 & 46    & 5         & -           \\
ASE 2003 & 54    & 5         & -           \\
ASE 2004 & 68    & 7         & -           \\
ASE 2005 & 79    & 9         & -           \\
ASE 2006 & 61    & 12        & 1           \\
ASE 2007 & 102   & 19        & -           \\
ASE 2008 & 90    & 18        & 1           \\
ASE 2009 & 89    & 19        & 1           \\
ASE 2010 & 87    & 20        & 1           \\
ASE 2011 & 112   & 28        & -           \\
ASE 2012 & 68    & 16        & -           \\
ASE 2013 & 90    & 28        & 2           \\
ASE 2014 & 100   & 39        & 3           \\
ASE 2015 & 99    & 42        & 1           \\
\hline
Total    & 1533  & 281       & 11          \\
\hline
\end{tabular}
\label{artigos-do-ase}
\end{table}

\subsection{Ferramentas da indústria}

Em paralelo à revisão estruturada para seleção de ferramentas da academia
foi realizada uma seleção manual no catálogo de ferramentas de análise estática do projeto
SAMATE\footnote{https://samate.nist.gov/index.php/Source\_Code\_Security\_Analyzers.html}
em busca de ferramentas da indústria.

O projeto SAMATE\footnote{http://samate.nist.gov} - {\em Software Assurance
Metrics and Tool Evaluation}, um projeto do NIST\footnote{http://nist.gov}
dedicado ao desenvolvimento de métodos que permitam avaliar e medir a
eficiência de ferramentas e técnicas sobre garantia de qualidade em software.
O site do projeto, disponível em \citeonline{SamateAnalysers}, mantém uma lista
de ferramentas de análise estática.

Nesta busca por ferramentas da indústria encontramos um total de 54 ferramentas
presentes no catálogo do projeto SAMATE, 19 tinham código-fonte disponível,
destas apenas 14 eram suportadas pelo Analizo (escritas em C, C++ ou Java).

Após download do código-fonte de cada ferramenta selecionada, em sua versão
mais recente, a ferramenta Analizo será utilizada para a coleta das métricas. 
A Tabela \ref{total-de-ferramentas} traz um resum com todas as ferramentas
selecionadas, tando da indústria quanto da academia.

\section{Caracterização das ferramentas de análise estática}

A seção à seguir traz uma caracterização inicial das ferramentas segundo às
categorias citadas anteriormente.

A ferramenta livre {\it sloccount}\footnote{http://www.dwheeler.com/sloccount}
foi utilizada para identificar a linguagem de programação em que cada
ferramenta é implementada. O tamanho em número de classes foi extraído utilizando a ferramenta
Analizo, uma das inúmeras métricas que ela extraí é a contagem do número total
de classes de um sistema. As demais informações foram obtidas de forma manual
lendo a documentação da ferramenta junto ao código-fonte ou no site da mesma.
Ao final desde capítulo a tabela \ref{total-de-ferramentas} apresenta um resumo
da caracterização feita.

\subsection{AccessAnalysis}

Plugin do Eclipse de análise estática de código-fonte Java
para cálculo das métricas IGAT e IGAM
disponível em \url{http://accessanalysis.sourceforge.net}. O código-fonte
utilizado em nosso estudo obtido no site da ferramenta foi o
\texttt{AccessAnalysis-1.2-src.zip}.

\subsection{Kiasan/Bogor}

Ferramenta de verificação de modelo disponível em
\url{http://bogor.projects.cs.ksu.edu/manual}. O código-fonte utilizado em
nosso estudo obtido no site da ferramenta foi o
\texttt{bogor-src-1.2.20061023.1.zip}.

Não possui número suficiente de releases para ser usado na análise evolutiva.

\subsection{composite}

Ferramenta de verificação de modelo disponível em
\url{http://www.cs.ucsb.edu/~bultan/composite/}. O código-fonte utilizado em
nosso estudo obtido no site da ferramenta foi o \texttt{composite-0.4.tar.gz}.

\subsection{CSeq}

Ferramenta de transformação {\it source-to-source} para programas C
concorrentes disponível em
\url{http://users.ecs.soton.ac.uk/gp4/cseq/files/cseq-0.5.zip}. O código-fonte
utilizado em nosso estudo obtido no site da ferramenta foi o
\texttt{cseq-0.5.zip}.

\subsection{EJB}

Ferramenta de análise estática para criação de diagramas de sequência
disponível em
\url{https://www.dropbox.com/s/glhg8any43lccgm/EJB.zip}.

\subsection{error-prone}

Ferramenta de localização de bugs em código Java construída em cima do
compilador {\it javac} disponível em
\url{http://code.google.com/p/error-prone}. O código-fonte utilizado em nosso
estudo obtido no site da ferramenta foi o \texttt{error-prone-2.0.9.tar.gz}.

\subsection{GUIZMO}

Ferramenta de inferência de layout disponível em
\url{http://modelum.es/trac/guizmo/}, plugin Eclipse. O código-fonte
utilizado em nosso estudo obtido no site da ferramenta foi o
\texttt{guizmo-master.zip}. Aceita como entrada um formato baseado em
XML\footnote{\url{http://wireframesketcher.com/help/xmlformat.html}} e gera
código GUI em Java Swing / ZK.

\subsection{GumTree}

Ferramenta de análise de código-fonte e comparação de mudanças
disponível em \url{https://github.com/jrfaller/gumtree}. O
código-fonte utilizado em nosso estudo obtido no site da ferramenta foi o
\texttt{gumtree-2.0.0.tar.gz}.

\subsection{Indus}

Biblioteca de {\it program
slicing}\footnote{http://en.wikipedia.org/wiki/Program\_slicing} Java disponível em
\url{http://indus.projects.cs.ksu.edu}. O projeto está organizado em três
módulos, os seguintes arquivos, contendo o código-fonte dos três módulos,
foram copiados localmente para análise:
\texttt{indus.indus-src-20091220.zip},
\texttt{indus.javaslicer-src-20091220.zip} e
\texttt{indus.staticanalyses-src-20070305.zip}.

Não possui número suficiente de releases para ser usado na análise evolutiva.

\subsection{JastAdd}

Sistema para análise de código-fonte através da descrição de
atributos via gramática de atributos (AG) disponível em \url{http://jastadd.cs.lth.se/web}. O código-fonte
utilizado em nosso estudo obtido no site da ferramenta foi o
\texttt{jastadd2-src.zip}.

\subsection{JFlow}

Ferramenta de transformação {\it source-to-source} disponível em
\url{http://vazexqi.github.io/JFlow/}. O código-fonte
utilizado em nosso estudo obtido no site da ferramenta foi o
\texttt{vazexqi-JFlow-7cd7eaf.tar.gz}.

\subsection{Lotrack}

Ferramenta de análise estática de configuração disponível em
\url{https://github.com/MaxLillack/Lotrack}. O código-fonte utilizado em nosso
estudo obtido no site da ferramenta foi o \texttt{Lotrack-master.zip}.

Não possui número suficiente de releases para ser usado na análise evolutiva.

\subsection{MPAnalyzer}

Ferramenta de análise de padrões disponível em
\url{https://github.com/YoshikiHigo/MPAnalyzer}. O código-fonte utilizado em
nosso estudo obtido no site da ferramenta foi o \texttt{MPAnalyzer-master.zip}.

\subsection{PtYasm}

Ferramenta de verificação de modelo disponível em
\url{www.cs.toronto.edu/~tomhart/ptyasm}. O código-fonte
utilizado em nosso estudo obtido no site da ferramenta foi o
\texttt{ptyasm.april2008.tgz}.

Não possui número suficiente de releases para ser usado na análise evolutiva.

\subsection{ReAssert}

Ferramenta de localização de falhas em testes e refatoração
desenvolvido como plugin Ecipse disponível em
\url{http://mir.cs.illinois.edu/reassert}. O código-fonte utilizado em nosso
estudo obtido no site da ferramenta foi o \texttt{ReAssert\_0.4.1-src.zip}.

\subsection{Sonar Qube Plug-in}

Plugin para o SourceMeter que extende a análise de
código Java com o uso do SonarQube disponível em:
\url{http://github.com/FrontEndART/SonarQube-plug-in}. O código-fonte
utilizado em nosso estudo obtido no site da ferramenta foi o
\texttt{SonarQube-plug-in-master.zip}.

\subsection{SPARTA}

Ferramenta de análise estática de segurança pra detecção de {\it
malware} disponível em
\url{http://types.cs.washington.edu/sparta/}. O código-fonte utilizado em nosso
estudo obtido no site da ferramenta foi o \texttt{sparta-sparta-1.0.2.tar.gz}.

\subsection{srcML}

Formato texto para representação de código-fonte e um conjunto de
ferramentas de transformação {\it source-to-source} disponível em
\url{http://www.sdml.info/projects/srcml/trunk}\footnote{este endereço
retornou "not found" em contato com os autores por email indicaram que o
projeto foi movido para http://www.srcML.org}. O código-fonte utilizado em
nosso estudo obtido no site da ferramenta foi o \texttt{srcML-src.tar.gz}.

Não possui número suficiente de releases para ser usado na análise evolutiva.

\subsection{TACLE}

Plugin do Eclipse para análise de tipo ({\it Type Analysis}) e
construção de visualizaçao de grafos de chamada ({\it Call Graph}) disponível em
\url{http://presto.cse.ohio-state.edu/tacle}\footnote{este link está
indisponível, por email os autores indicaram o endereço
http://web.cse.ohio-state.edu/~rountev/presto/tacle/TACLE\_Download/tacle.html}.
O código-fonte utilizado em nosso estudo obtido no site da ferramenta foi o
\texttt{tacle\_1\_2\_1\_src.zip}.

\subsection{WALA}

Ferramenta de análise estática para {\it bytecode} Java disponível em
\url{http://wala.sourceforge.net/wiki/index.php/Main_Page}. O código-fonte
utilizado em nosso estudo obtido no site da ferramenta foi o
\texttt{WALA-R\_1.3.8.tar.gz}.

Ferramenta selecionada para análise evolutiva, possui muitos releases e tem tamanho
em número de classes na média.

\subsection{Closure Compiler}

Compilador que traduz código JavaScript em outro
JavaScript melhor e mais otimizado, está disponível em
\url{https://developers.google.com/closure/compiler}\footnote{O código fonte do
Closure Compiler pode ser obtido em:
http://github.com/google/closure-compiler} e foi utilizado em nosso estudo o
seguinte lançamento
\texttt{closure-compiler-closure-compiler-parent-v20160619.tar.gz}.

Ferramenta selecionada para análise evolutiva, possui muitos releases e tem tamanho
em número de classes na média.

\subsection{Cppcheck}

Ferramenta de análise estática de código C/C++ para checagem de vazamento de
memória, erros de alocação, entre outras falhas. Disponível em
\url{http://sourceforge.net/projects/cppcheck}. Em nosso estudo utilizamos o
código em \texttt{cppcheck-1.72.tar.bz2}.

\subsection{CQual}

Ferramenta de análise de typo ({\it type-based analysis}) que fornece um
mecanismo leve e prático para especificação e verificação de propriedades de
programas C. Disponível em \url{http://www.cs.umd.edu/~jfoster/cqual}. Em
nosso estudo utilizamos o código em \texttt{cqual-0.981.tar.gz}.

\subsection{FindBugs}

Uma ferramenta para localização de bugs em código Java disponível em
\url{http://findbugs.sourceforge.net}. Em nosso estudo utilizamos o código em
\texttt{findbugs-3.0.1-source.zip}.

Ferramenta selecionada para análise evolutiva, possui muitos releases e tem tamanho
em número de classes na média.

\subsection{FindSecurityBugs}

Plugin do FindBugs para auditoria de segurança em aplicações web Java,
disponível em \url{http://find-sec-bugs.github.io}. O código-fonte utilizado
em nosso estudo obtido no site da ferramenta foi o
\texttt{findsecbugs-plugin-1.4.5-sources.jar}.

\subsection{Jlint}

Uma ferramenta para verificaçao de código Java em busca de bugs,
inconsistências e problemas de sincronização disponível em
\url{http://sourceforge.net/projects/jlint}.  O código-fonte utilizado em
nosso estudo obtido no site da ferramenta foi o \texttt{jlint-3.1.2.zip}.

\subsection{Pixy}

Ferramenta de análise estática de código PHP para verificação de
vulnerabilidades de segurança. Disponível em
\url{http://github.com/oliverklee/pixy}. O código-fonte utilizado em nosso
estudo obtido no site da ferramenta foi o \texttt{pixy-master.zip}.

\subsection{PMD}

Ferramenta de análise de código-fonte para localização falhas comuns de
programação com suporte a várias linguagens, disponível em
\url{http://pmd.github.io}. O código-fonte utilizado em nosso estudo obtido
no site da ferramenta foi o \texttt{pmd-src-5.4.1.zip}.

Ferramenta selecionada para análise evolutiva, possui muitos releases e tem tamanho
em número de classes na média.

\subsection{RATS}

Ferramenta de análise estática para auditoria de segurança 
de códigos C, C++, Perl, PHP e Python disponível em
\url{http://code.google.com/archive/p/rough-auditing-tool-for-security}. O
código-fonte utilizado em nosso estudo obtido no site da ferramenta foi o
\texttt{rats-2.4.tgz}.

\subsection{Smatch}

Ferramenta de análise estática para detecção de erros no Kernel disponível em
\url{http://smatch.sourceforge.net}. O código-fonte utilizado em nosso estudo
obtido no site da ferramenta foi o \texttt{smatch.git}.

\subsection{Splint}

Ferramenta para verificação de programas em C por vulnerabilidades de segurança e
erros de código. Disponível em \url{http://www.splint.org}. O código-fonte
utilizado em nosso estudo obtido no site da ferramenta foi o
\texttt{splint-3.1.2.src.tgz}.

Não possui número suficiente de releases para ser usado na análise evolutiva.

\subsection{UNO}

Uma ferramenta de análise de código-fonte C para detecção de defeitos.
Disponível em \url{http://spinroot.com/uno}. O código-fonte utilizado em nosso
estudo obtido no site da ferramenta foi o \texttt{uno\_v213.tar.gz}.

\subsection{WAP}

Ferramenta para análise estática de código-fonte PHP e mineraçao de dados para
detectar e corrigir vulnerabilidades em aplicações web. Disponível em
\url{http://awap.sourceforge.net}. O código-fonte utilizado em nosso estudo
obtido no site da ferramenta foi o \texttt{wap-2.1.tar.gz}.

\section{Resumo}

De forma que somando as ferramentas selecionadas na academia e na indústria
temos um total de 34 ferramentas, 14 da indústria e 20 da academia.  A Tabela
\ref{total-de-ferramentas} resume este total trazendo o nome de cada ferramenta
e algumas de suas características, a caracterização completa está documentado
no arquivo {\it
ferramentas-e-metricas.ods}\footnote{https://github.com/joenio/dissertacao-ufba-2016/blob/master/dataset/ferramentas-e-metricas.ods}
disponível no repositório desta dissertação.

\begin{table}[H]
  \caption{Resumo da caracterização das ferramentas}
  \centering
  \begin{tabular}{| c | l | l | c | l | l |}
    \hline
    \#  & Ferramentas da academia & Linguagem & Classes & Lançamentos \\
    \hline
    1  & AccessAnalysis          & Java   & 91    & Obsoleta       \\
    2  & Kiasan/Bogor            & Java   & 545   & Obsoleta       \\
    3  & composite               & C      & 17    & -              \\
    4  & CSeq                    & C      & 106   & -              \\
    5  & EJB                     & Java   & 244   & Obsoleta       \\
    6  & error-prone             & Java   & 1703  & Frequentemente \\
    7  & GUIZMO                  & Java   & 306   & -              \\
    8  & GumTree                 & Java   & 306   & Ocasionalmente \\
    9  & Indus                   & Java   & 524   & Obsoleta       \\
    10 & JastAdd                 & Java   & 59    & Frequentemente \\
    11 & JFlow                   & Java   & 177   & Obsoleta       \\
    12 & Lotrack                 & Java   & 4059  & -              \\
    13 & MPAnalyzer              & Java   & 240   & -              \\
    14 & PtYasm                  & Java   & 883   & -              \\
    15 & ReAssert                & Java   & 169   & Obsoleta       \\
    16 & Sonar Qube Plug-in      & Java   & 143   & Frequentemente \\
    17 & SPARTA                  & Java   & 268   & Ocasionalmente \\
    18 & srcML                   & C++    & 871   & Ocasionalmente \\
    19 & TACLE                   & Java   & 34    & Obsoleta       \\
    20 & WALA                    & Java   & 2626  & Ocasionalmente \\
    \hline
    \# & Ferramentas da indústria & Linguagem & Classes & Lançamentos \\
    \hline
    21 & Clang Static Analyzer    & C++   & ?     & Frequentemente \\
    22 & Closure Compiler         & Java  & 1842  & Frequentemente \\
    23 & Cppcheck                 & C++   & 338   & Frequentemente \\
    24 & CQual                    & C     & 78    & Obsoleta       \\
    25 & FindBugs                 & Java  & 1486  & Ocasionalmente \\
    26 & FindSecurityBugs         & Java  & 91    & Frequentemente \\
    27 & Jlint                    & C++   & 44    & Obsoleta       \\
    28 & Pixy                     & Java  & 229   & Obsoleta       \\
    29 & PMD                      & Java  & 1340  & Frequentemente \\
    30 & RATS                     & C     & 19    & Obsoleta       \\
    31 & Smatch                   & C     & 483   & Ocasionalmente \\
    32 & Splint                   & C     & 681   & Obsoleta       \\
    33 & UNO                      & C     & 19    & Obsoleta       \\
    34 & WAP                      & Java  & 338   & Frequentemente \\
    \hline
  \end{tabular}
  \label{total-de-ferramentas}
\end{table}

Dentre estas ferramentas as seguintes foram selecionadas para análise evolutiva:

\begin{itemize}
  \item Closure Compiler         
  \item FindBugs                 
  \item PMD                      
  \item WALA                    
  \item error-prone
  \item JastAdd
  \item SPARTA
  \item Cppcheck
  \item FindSecurityBugs
  \item Smatch
\end{itemize}


o clang foi removido pois a analise dele demorou muito, ficou rodando 1 semana q não
terminou, o analizo metrics que demorou tanto assim.

\begin{table}[H]
\caption{Métricas da ferramenta PMD}
  \centering
\begin{tabular}{|l|r|r|r|r|r|}
\hline
\multicolumn{1}{|c|}{\textbf{Release}} & \multicolumn{1}{c|}{\textbf{Classes}} & \multicolumn{1}{c|}{\textbf{CC}} & \multicolumn{1}{c|}{\textbf{SC 75}} & \multicolumn{1}{c|}{\textbf{SC 90}} & \multicolumn{1}{c|}{\textbf{SC 95}} \\ \hline
4.2.5 & 844 & 0,06 & 6 & 15 & 28 \\ \hline
4.3 & 852 & 0,06 & 6 & 15 & 28 \\ \hline
5.0.0 & 1043 & 0,03 & 6 & 14 & 25 \\ \hline
5.0.4 & 1052 & 0,02 & 6 & 12 & 24 \\ \hline
5.1.0 & 1238 & 0,02 & 6 & 12 & 24 \\ \hline
5.1.3 & 1254 & 0,02 & 6 & 12 & 25 \\ \hline
5.2.0 & 1295 & 0,02 & 6 & 12 & 25 \\ \hline
5.3.0 & 1341 & 0,01 & 6 & 12 & 25 \\ \hline
5.3.3 & 1342 & 0,02 & 6 & 12 & 25 \\ \hline
5.3.7 & 1374 & 0,01 & 6 & 12 & 25 \\ \hline
5.4.0 & 1332 & 0,02 & 6 & 12 & 26 \\ \hline
5.4.2 & 1366 & 0,02 & 6 & 12 & 26 \\ \hline
5.5.2 & 1530 & 0,01 & 6 & 12 & 25 \\ \hline
\multicolumn{6}{l}{\texttt{Notas:}} \\
\multicolumn{6}{l}{\texttt{CC = Custo de mudança}} \\
\multicolumn{6}{l}{\texttt{SC = Complexidade estrutural}} \\ \hline
\end{tabular}
\label{metricas-pmd}
\end{table}

\begin{table}[H]
\caption{Métricas da ferramenta WALA}
  \centering
\begin{tabular}{|l|r|r|r|r|r|}
\hline
\multicolumn{1}{|c|}{\textbf{Release}} & \multicolumn{1}{c|}{\textbf{Classes}} & \multicolumn{1}{c|}{\textbf{CC}} & \multicolumn{1}{c|}{\textbf{SC 75}} & \multicolumn{1}{c|}{\textbf{SC 90}} & \multicolumn{1}{c|}{\textbf{SC 95}} \\ \hline
1.0 & 1223 & 0,02 & 8 & 27 & 45 \\ \hline
1.0.02 & 1685 & 0,02 & 8 & 25 & 48 \\ \hline
1.1 & 1872 & 0,02 & 7 & 24 & 48 \\ \hline
1.1.2 & 1720 & 0,02 & 8 & 24 & 50 \\ \hline
1.2 & 1734 & 0,02 & 7 & 25 & 49 \\ \hline
1.2.1.1 & 1901 & 0,02 & 6 & 24 & 48 \\ \hline
1.2.2 & 1903 & 0,02 & 6 & 24 & 48 \\ \hline
1.3 & 1945 & 0,02 & 6 & 24 & 52 \\ \hline
1.3.3 & 2092 & 0,01 & 6 & 24 & 50 \\ \hline
1.3.5 & 2143 & 0,02 & 6 & 22 & 49 \\ \hline
1.3.6 & 2154 & 0,02 & 6 & 22 & 49 \\ \hline
1.3.8 & 2626 & 0,02 & 7 & 24 & 54 \\ \hline
1.3.9 & 2636 & 0,02 & 7 & 24 & 54 \\ \hline
\multicolumn{6}{l}{\texttt{Notas:}} \\
\multicolumn{6}{l}{\texttt{CC = Custo de mudança}} \\
\multicolumn{6}{l}{\texttt{SC = Complexidade estrutural}} \\ \hline
\end{tabular}
\label{metricas-wala}
\end{table}

\begin{table}[H]
\caption{Métricas da ferramenta FindBugs}
  \centering
\begin{tabular}{|l|r|r|r|r|r|}
\hline
\multicolumn{1}{|c|}{\textbf{Release}} & \multicolumn{1}{c|}{\textbf{Classes}} & \multicolumn{1}{c|}{\textbf{CC}} & \multicolumn{1}{c|}{\textbf{SC 75}} & \multicolumn{1}{c|}{\textbf{SC 90}} & \multicolumn{1}{c|}{\textbf{SC 95}} \\ \hline
1.2.1 & 1044 & 0,05 & 7 & 20 & 36 \\ \hline
1.3.4 & 1216 & 0,06 & 7 & 21 & 42 \\ \hline
1.3.5 & 1257 & 0,05 & 7 & 21 & 40 \\ \hline
1.3.6 & 1258 & 0,05 & 8 & 21 & 42 \\ \hline
1.3.7 & 1261 & 0,05 & 7 & 22 & 42 \\ \hline
1.3.8 & 1275 & 0,05 & 7 & 22 & 42 \\ \hline
1.3.9 & 1354 & 0,06 & 7 & 24 & 48 \\ \hline
2.0.0 & 1459 & 0,06 & 7 & 24 & 52 \\ \hline
2.0.1 & 1465 & 0,06 & 7 & 24 & 54 \\ \hline
2.0.2 & 1469 & 0,06 & 7 & 24 & 56 \\ \hline
2.0.3 & 1489 & 0,06 & 7 & 24 & 56 \\ \hline
3.0.0 & 1438 & 0,07 & 7 & 24 & 56 \\ \hline
3.0.1 & 1486 & 0,07 & 8 & 25 & 56 \\ \hline
\multicolumn{6}{l}{\texttt{Notas:}} \\
\multicolumn{6}{l}{\texttt{CC = Custo de mudança}} \\
\multicolumn{6}{l}{\texttt{SC = Complexidade estrutural}} \\ \hline
\end{tabular}
\label{metricas-findbugs}
\end{table}

\begin{table}[H]
\caption{Métricas da ferramenta Closure Compiler}
  \centering
\begin{tabular}{|l|r|r|r|r|r|}
\hline
\multicolumn{1}{|c|}{\textbf{Release}} & \multicolumn{1}{c|}{\textbf{Classes}} & \multicolumn{1}{c|}{\textbf{CC}} & \multicolumn{1}{c|}{\textbf{SC 75}} & \multicolumn{1}{c|}{\textbf{SC 90}} & \multicolumn{1}{c|}{\textbf{SC 95}} \\ \hline
20110119 & 1122 & 0,05 & 4 & 17 & 42 \\ \hline
20110811 & 1730 & 0,1 & 5 & 20 & 40 \\ \hline
20120305 & 1802 & 0,1 & 5 & 20 & 42 \\ \hline
20120917 & 1836 & 0,1 & 6 & 20 & 42 \\ \hline
20130227 & 1759 & 0,11 & 6 & 20 & 48 \\ \hline
20130722 & 1806 & 0,1 & 6 & 20 & 48 \\ \hline
20140110 & 2004 & 0,08 & 5 & 20 & 45 \\ \hline
20140730 & 1573 & 0,04 & 4 & 20 & 48 \\ \hline
20150126 & 1596 & 0,04 & 5 & 22 & 56 \\ \hline
20150729 & 1649 & 0,04 & 6 & 26 & 65 \\ \hline
20160125 & 1724 & 0,04 & 5 & 28 & 68 \\ \hline
20160713 & 1860 & 0,04 & 6 & 30 & 70 \\ \hline
\multicolumn{6}{l}{\texttt{Notas:}} \\
\multicolumn{6}{l}{\texttt{CC = Custo de mudança}} \\
\multicolumn{6}{l}{\texttt{SC = Complexidade estrutural}} \\ \hline
\end{tabular}
\label{metricas-closurecompiler}
\end{table}

As ferramentas com poucas linhas de código foram excluidas, estas
apreentam Change Cost alto, já é conhecido que a definição desta métrica
sofre deste problema, apresenta valores altos em projetos muito pequenos,
tambem removemos da analise aquelas ferramentas que nao tiveram valor
no percentil 75\%, pois a comparacao e analise se dará neste percentil
principalmente.

Com isso temos 11 projetos, destes iremos analisar longitudalmente
as releases e a evolucao dos valores de SC e CC (Change Cost), sao elas:

 Closure Compiler         13 releases analisados

 FindBugs                 13 releases

 Indus                    (poucos releases, deixando fora da analise longitudinal)

 Kiasan/Bogor             (mudou a forma de distribuir ao longo dos releases, dificil obter de forma consistente as versoes)

 Lotrack                  (sem releases, poucos commits no github, apenas 11)

 PMD                      13 releases

 PtYasm                   (nao tem releases disponivel, apenas a ultima versao)

 Splint                   (nao encontrado releases)

 srcML                    (releases nao encontrado)

 WALA                     13 releases

 error-prone              13 releases

 GumTree                  (tem apenas 2 releases do repositorio github)

 JastAdd                  13 releases

 Sonar Qube Plug-in       (apenas 4 releases no github)

 SPARTA                   13 releases

 Cppcheck                 13 releases

 FindSecurityBugs         13 releases
 
 Smatch                   13 releases

 WAP                      (apenas 7 releases no site, estou selecionando os que tenham ao menos 13 releases)

Comparacao entre ferramentas de tamanho similar:

nas 5 comparações de versões distintas com tamanhos similares entre pmd e findbugs,
apresentaram o mesmo resultado, pmd tem valores menos tanto para CC quanto para SC,
indicando que pmd tem um design mais modular que findbugs.

pmd 5.0.0 < findbugs 1.2.1
pmd 5.0.4 < findbugs 1.2.1
pmd 5.1.3 < findbugs 1.3.5
pmd 5.2.0 < findbugs 1.3.8
pmd 5.3.3 < findbugs 1.3.9

Ao comparar as imagens da matrix DSM dá para notar que isto reflete na matrix, 
pegando o findbugs 3.0.1 e o pmd 5.2.0, é possível notar na matrix que o findbugs
tem mais pontos nas duas diagonais da matrix, indicando dependencias ciclicar, e
design menos modular, enquanto o pmd concentra as dependencias na diagonal inferior
esquerda, indicando poucas dependencias ciclicar e um design mais modular.

findbugs	 findbugs-3.0.1	1486	0,07	8	25	56
pmd	 pmd-src-5.2.0	1295	0,02	6	12	25

/home/joenio/src/dissertacao-ufba-2016/dataset/static-analysis-tools/pmd/pmd-src-5.2.0.analizo.dsm.png
/home/joenio/src/dissertacao-ufba-2016/dataset/static-analysis-tools/findbugs/findbugs-3.0.1.analizo.dsm.png

accessanalysis < findsecuritybugs
indus < bogor
reassert > jflow
pmd-5.4.0 > pmd-5.3.0
pmd-5.4.2 > pmd-5.3.7
findbugs-3.0.1 > findbugs-2.0.3
pmd < closure-compiler
pixy > mpanalyzer
ejb > mpanalyzer
ejb > sparta

closure-compiler > wala
closure-compiler > wala
closure-compiler > wala
closure-compiler[Java] ? wala[Java]     (closure tem CC maior mas SC menor)

comparar linguagens diferentes não rola, sempre dá ruim, ver:

rats[C]        ? uno[C]                 (rats tem CC menor e SC maior)
cppcheck[C++]  ? wap[Java]              (cppcheck tem CC menor mas SC maior)
srcml[C++]     ? ptyasm[Java]           (srcml tem CC maior mas SC menor)
closure-compiler[Java] ? inputtracer[C] (closure tem CC menor mas SC diferentes nos percentis)
pmd[Java] ? srcml[C++]                  (pmd tem CC maior e SC diferentes nos percentis)
inputtracer[C] ? wala[Java]             (tem CC maior e SC diferentes mas no percentil 95 tem SC maior também)

fica claro que comparacao entre linguagens diferentes mesmo com tamanhos iguais não dá para chegar a conclusões nenhuma.

findbugs[Java] ? wala[Java]             (findbugs tem CC maior mas SC menor)
closure-compiler ? closure-compiler     (CC menor mas SC maior)

comparacao (v3) - ordenado por eloc - comparando apenas SC 95
===============

% findsecbugs-plugin-1.4.0-sources < sparta-code-0.6
% findsecbugs-plugin-1.4.1-sources < sparta-code-0.7
% findsecbugs-plugin-1.4.4-sources < sparta-code-0.8

% cseq-0.5 > find-sec-bugs-version-1.0.0

% sparta-code-0.9.2 < tacle_1_2_1_src
% sparta-code-0.9.4 < jastadd2-src-2.1.5
% MPAnalyzer-master > sparta-toolset-0.9.8
% ReAssert_0.4.1 > sparta-sparta-1.0.2
% sparta-sparta-1.0.2 < uno
% sparta-toolset-1.0.1-source < cppcheck-1.30
% SonarQube-plug-in-master > sparta-toolset-1.0.0-source

% findsecbugs-plugin-1.4.5-sources < rats-2.4
% jastadd2-src-2.1.2 > findsecbugs-plugin-1.4.6-sources
% find-sec-bugs-version-1.1.0 < jastadd2-src-2.1.4
% AccessAnalysis-1.2-src < jastadd2-src-2.1.9
% jastadd2-src-2.1.13 > jlint-3.1.2
% vazexqi-JFlow-7cd7eaf < gumtree-2.0.0
% composite-0.4 < smatch-1.0
% smatch-0.3 > EJB
% smatch-0.4 > cppcheck-1.35
% cppcheck-1.35 > guizmo-master
% smatch-1.51 < cqual-0.981
% pixy-master < smatch-1.52
% error-prone-2.0 < smatch-1.54
% smatch-1.54 > cppcheck-1.40
% smatch-1.55 > error-prone-2.0.2
% indus < smatch-1.56
% smatch-1.56 > error-prone-2.0.4
% smatch-1.59 > pmd-src-5.0.4
% error-prone-2.0.6 < pmd-src-4.2.5
% pmd-src-4.2.5 < smatch-1.60
% smatch-1.60 < cppcheck-1.45
% ptyasm > error-prone-2.0.8
% error-prone-2.0.8 < bogor-core
% pmd-src-5.1.0 > error-prone-2.0.9
% pmd-src-5.3.7 < wap-2.1
% error-prone-2.0.12 < pmd-src-5.5.2
% error-prone-2.0.13 < cppcheck-1.50
% cppcheck-1.50 > wala-code-4607-tags-R_1.0
% findbugs-1.2.1-source > error-prone-2.0.14
% cppcheck-1.55 > findbugs-1.3.4-source
% findbugs-1.3.8-source < cppcheck-1.60
% findbugs-1.3.9-source = wala-code-4607-tags-R_1.0.02
% wala-code-4607-tags-R_1.2 < cppcheck-1.62
% findbugs-2.0.2-source > wala-code-4607-tags-R_1.1
% cppcheck-1.65 > wala-code-4607-tags-R_1.2.2
% wala-code-4607-tags-R_1.3 < findbugs-3.0.0-source
% findbugs-3.0.1 > WALA-R_1.3.3
% WALA-R_1.3.3 < cppcheck-1.70
% cppcheck-1.75 > WALA-R_1.3.5
% WALA-R_1.3.6 > closure-compiler-20110119
% closure-compiler-20110119 < cppcheck-1.77
% cppcheck-1.77 < splint-3.1.2
% srcML-src > closure-compiler-20140730
% closure-compiler-20160713 > Lotrack-master

O percentil 75 tem muitos valores zero, os percentis 90 e 95 sao pracitamente iguais 
na comparacao, os maiores sao geralmente tb maior no outro, exceto uns 2 exemplos:
smatch-0.3/EJB e pmd-src-5.3.7/wap-2.1.




comparacao (v3) - ordenado por n modulos - comparando apenas SC 90 e 95
===============



rats-2.4 > uno
jlint-3.1.2 > findsecbugs-1.2.0
jastadd2-2.1.5 > findsecbugs-1.2.1
findsecbugs-1.3.0 < jastadd2-2.1.8
jastadd2-2.2.2 > findsecbugs-1.4.0
findsecbugs-1.4.2 < cqual-0.981
sparta-0.5 < findsecbugs-1.4.4
findsecbugs-1.4.5 > AccessAnalysis-1.2
AccessAnalysis-1.2 < cppcheck-1.30
cppcheck-1.30 < smatch-1.0
smatch-0.2 > findsecbugs-1.4.6
findsecbugs-1.4.6 > sparta-0.6
sparta-0.7 < smatch-0.3
cseq-0.5 > findsecbugs-1.5.0
smatch-0.4 > cppcheck-1.35
cppcheck-1.35 > sparta-0.8
cppcheck-1.40 > sparta-0.9.2
sparta-0.9.2 > findsecbugs-1.0.0
SonarQube-plug-in-master < smatch-1.51
smatch-1.52 > ReAssert\_0.4.1
smatch-1.53 > jfLow
gumtree-2.0.0 < cppcheck-1.45
smatch-1.54 > sparta-0.9.8
sparta-0.9.8 < pixy
cppcheck-1.50 > findsecbugs-1.1.0
findsecbugs-1.1.0 < MPAnalyzer
MPAnalyzer < EJB
sparta-1.0.1 < cppcheck-1.55
guizmo < cppcheck-1.60
smatch-1.56 < cppcheck-1.70
wap-2.1 < cppcheck-1.72
cppcheck-1.75 > smatch-1.58
pmd-4.3 < srcML
srcML < ptyasm
pmd-5.0.0 < findbugs-1.2.1
pmd-5.0.4 > error-prone-2.0
closure-compiler-20110119 > error-prone-2.0.2
error-prone-2.0.4 < findbugs-1.3.4
findbugs-1.3.4 < wala-4607-R1.0
wala-4607-R1.0 > pmd-5.1.0
pmd-5.1.3 < findbugs-1.3.5
findbugs-1.3.8 > pmd-5.2.0
pmd-5.3.3 < findbugs-1.3.9
findbugs-1.3.9 > pmd-5.4.2
pmd-5.3.7 < findbugs-3.0.0
findbugs-2.0.3 > error-prone-2.0.5
pmd-5.5.2 > error-prone-2.0.6
closure-compiler-20140730 > error-prone-2.0.7
error-prone-2.0.8 < closure-compiler-20150729
error-prone-2.0.9 < wala-4607-R1.1.2
wala-4607-R1.1.2 < closure-compiler-20160125
closure-compiler-20110811 < wala-4607-R1.2
error-prone-2.0.11 < closure-compiler-20160517
closure-compiler-20160713 > error-prone-2.0.12
error-prone-2.0.12 < wala-4607-R1.1
wala-4607-R1.2.2 > error-prone-2.0.13
error-prone-2.0.14 < wala-4607-R1.3
error-prone-2.0.15 < closure-compiler-20140110








\subsection{Analizo}

Numa primeira análise dos valores coletados pelo Analizo notamos uma anomalia
nos valores da métrica CBO, o que nos levou a investigar de perto os motivos,
esta anomalia se apresentava como valores extremamente altos para esta métrica,
bastante discrepante com as demais métricas calculadas.

Para entender se estes valores estavam corretor ou não, utilizamos uma outra
ferramenta para cálculo das métricas, em nossos estudos encontramos e
utilizamos uma versão de avaliação da ferramenta {\it SciTools
Understand}\footnote{http://scitools.com/trial-download-3} em sua versão
``4.0.853'' em Linux 64 bits. Os dados extraídos por esta ferramenta podem ser
encontrados em nosso
repositório\footnote{http://github.com/joenio/dissertacao-ufba-2016/tree/master/dataset/Understand
SciTools}. Eles demonstraram que os valores calculados pelo Analizo estavam
bastante alto em comparação com as demais métricas.

Assim, descobrimos que o Analizo tinha de fato um erro no cálculo da métrica
CBO, erro que foi corrigido durante este estudo e disponibilizado na versão
mais recente do Analizo, versão que está sendo utilizada aqui.
