\xchapter{Caracterização das ferramentas}
{Este capítulo apresenta a caracterização das ferramentas selecionadas a partir
da revisão estruturada.}
\label{caracterizacao-ferramentas}

As ferramentas de análise estática selecionadas serão caracterizadas
a partir de algumas dimensões que se mostrem necessárias ao nosso estudo, todas
as ferramentas foram obtidas em sua versão mais recente, apenas uma única versão
de cada ferramenta foi selecionada e será caracterizada a partir de tal versão,
esta cópia de cada ferramenta foi do seu código-fonte, algumas características
são obtidas do código-fonte, outras da documentação da ferramenta.

Esta caracterização será fundamental para compreender as métricas de código-fonte
extraídas destas ferramentas apresentadas no Apendice \ref{analise-metricas},
ferramentas que tenham interface gráfica de usuário por exemplo podem apresentar
valores diferentes daquelas que tenham apenas interface modo texto, suporte
a analisar múltiplas linguagens pode também trazer valores de métricas maiores ou
piores. Iremos levar estas características em consideração durante a análise
exploratória dos nosso dados.

Para compor as características e também evitar re-criar dimensões para características
de ferramentas deste domínio iremos nos basear no trabalho de
\citeonline{Novak2010}, onde um estudo para construção de uma taxonomia para
ferramentas de análise estática propuseram uma classificação a partir de uma
série de categorias.

Iremos utilizar algumas de suas características, apenas aquelas que nos forem necessárias
para interpretar e caracterizar a complexidade estrutural de tais ferramentas, segue abaixo todas
as características proposta no trabalho de \citeonline{Novak2010}.

\begin{description}

  \item {\it Entrada - quais tipos de arquivos podem ser carregados na ferramenta:}
    \begin{itemize}
      \item Código-fonte - arquivos de código texto podem ser carregados
      \item Byte code - arquivos com Java Byte Code ou Microsoft
      \item Linguagem intermediária (MSIL) pode ser carregada
    \end{itemize}

  \item {\it Lançamentos ({\it Releases}) - quantos lançamentos por ano:}
    \begin{itemize}
      \item Frequentemente $>=$ 3 vezes ao ano - novas versões da ferramenta são lançadas 3 ou mais vezes por ano
      \item Ocasionalmente $<$ 3 vezes ao ano - novas versões da ferramenta são lançadas menos que 3 vezes ao ano
      \item Obsoleta 0 vezes ao ano - intervalo entre novos lançamentos é maior que 1 ano
    \end{itemize}

  \item {\it Linguagens suportadas - quais linguagens de programação a ferramenta suporta:}
    \begin{itemize}
      \item .NET - todas as linguagens compiladas em bibliotecas ou programas no framework .NET
      \item VB .NET - suporta VB.NET
      \item C\# - suporta C\#
      \item Java - suporta linguagem de programação Java
      \item C, C++ - suporta linguagem de programação C ou C++
    \end{itemize}

  \item {\it Tecnologia - quais tecnologias são usadas para procurar erros no código:}
    \begin{itemize}
      \item Dataflow - busca por erros com dataflow
      \item Sintaxe - busca por errors de sintaxe e correctness
      \item Prova de teoremas - procurar erros em provar diferentes teoremas
      \item Verificação de modelos - procurar erros com verificação de modelo
    \end{itemize}

  \item {\it Regras - conjunto de regras, quais são suportadas por diferentes código estático:}
    \begin{itemize}
      \item Estilo - inspeciona a aparência do código-fonte
      \item Naming - checa se as varáveis são nomeadas corretamente (ortografia, padroes de nomenclatura, ...)
      \item Geral - regras gerais de analise estática de código
      \item Concorrencia - erros com execução de código de concorrente
      \item Exceções - erros lançando ou não exceções
      \item Performance - erros de performance das aplicações
      \item Interoperabilidade - erros de comportamento comum
      \item Segurança - erros que podem impactar na segurança da aplicação
      \item SQL - procurar por "SQL injections" e outros erros de SQL
      \item Buffer overflow - erros de segurança, que explorar buffer overflow
      \item Manutenabilidade - regras para melhor manutenabilidade da aplicação
    \end{itemize}

  \item {\it Configurabilidade - habilidade de configurar a ferramenta:}
    \begin{itemize}
      \item Documento texto - comfiguração é feita via documento texto
      \item XML - configuração pe feita por documento XML
      \item GUI - configuraçãi é feita via interface gráfica
      \item Ruleset - ferramenta pode ligar/desligar conjunto de regras
    \end{itemize}

  \item {\it Extensibilidade - se a ferramenta pode ser extendida com regras próprias:}
    \begin{itemize}
      \item Possível - é possível extender
      \item Não possível - não é possível extender
    \end{itemize}

  \item {\it Disponibilidade - de que forma a ferramenta está disponível:}
    \begin{itemize}
      \item Código Aberto ({\it Open Source}) - a ferrament é livre e o código-fonte está disponível
      \item Grátis ({\it Free}) - a ferramenta é grátis mas o código-fonte não está disponível
      \item Comercial - a ferramenta está disponível mediante pagamento
    \end{itemize}

  \item {\it Experiência do usuário - de que forma a ferramenta pode ser usada, como é oferecida:}
    \begin{itemize}
      \item Integração com ambiente - como a ferramenta é integrada ao ambiente de trabalho
      \item Localização automática de erros no código - quando a ferramenta encontra um erro, ela leva ao local do erro
      \item Ajuda abrangente sobre falhas - se a ferramenta oferece ajuda na resolução de erros
      \item Interface de usuário - disponibilidade de uma interface de usuário
      \item Linha de comando - ela pode ser executada via linha de comando
      \item GUI - a ferramente pode ser executada em uma interface gráfica (GUI)
    \end{itemize}

  \item {\it Saída - representação dos resultados da ferramenta:}
    \begin{itemize}
      \item Arquivo texto - ferramenta pode apresentar resultados em arquivos texto
      \item Lista - ferramenta pode apresentar resultados numa interface de usuário customizada controlada em GUI
      \item Arquivo XML - ferramenta pode apresentar resultados em dados XML
      \item Arquivo HTML - ferramenta pode apresentar resultados em dados HTML
    \end{itemize}

\end{description}

Iremos utilizar algumas destas categorias na caracterização das ferramentas
selecionadas neste estudo, e vamos também caracterizar em relação à algumas dimensões que nos
foram necessárias mas que não faziam parte das dimensões propostas por \citeonline{Novak2010},
estas dimensões são as seguintes.

\begin{description}

  \item {\it Linguagem de programação - em que linguagem de programação à ferramenta é escrita:}
    \begin{itemize}
      \item .NET
      \item VB .NET
      \item C\#
      \item Java
      \item C, C++
    \end{itemize}

  \item {\it Tamanho em número de classes - número de classes/módulos da ferramenta}

\end{description}

A seção à seguir traz uma caracterização inicial das ferramentas segundo às
categorias citadas anteriormente.

\section{Caracterização das ferramentas} \label{caracterizacao-das-ferramentas}

A ferramenta livre {\it sloccount}\footnote{http://www.dwheeler.com/sloccount}
foi utilizada para identificar a linguagem de programação em que cada
ferramenta é implementada. O tamanho em número de classes foi extraído utilizando a ferramenta
Analizo, uma das inúmeras métricas que ela extraí é a contagem do número total
de classes de um sistema. As demais informações foram obtidas de forma manual
lendo a documentação da ferramenta junto ao código-fonte ou no site da mesma.
Ao final desde capítulo a tabela \ref{total-de-ferramentas} apresenta um resumo
da caracterização feita.

\subsection{AccessAnalysis}

Plugin do Eclipse de análise estática 
para cálculo das métricas IGAT e IGAM
disponível em \url{http://accessanalysis.sourceforge.net}. O código-fonte
utilizado em nosso estudo obtido no site da ferramenta foi o
\texttt{AccessAnalysis-1.2-src.zip}.

\subsection{Kiasan/Bogor}

Ferramenta de verificação de modelo disponível em
\url{http://bogor.projects.cs.ksu.edu/manual}. O código-fonte utilizado em
nosso estudo obtido no site da ferramenta foi o
\texttt{bogor-src-1.2.20061023.1.zip}.

Não possui número suficiente de releases para ser usado na análise evolutiva.

\subsection{composite}

Ferramenta de verificação de modelo disponível em
\url{http://www.cs.ucsb.edu/~bultan/composite/}. O código-fonte utilizado em
nosso estudo obtido no site da ferramenta foi o \texttt{composite-0.4.tar.gz}.

\subsection{CSeq}

Ferramenta de transformação {\it source-to-source} para programas
concorrentes disponível em
\url{http://users.ecs.soton.ac.uk/gp4/cseq/files/cseq-0.5.zip}. O código-fonte
utilizado em nosso estudo obtido no site da ferramenta foi o
\texttt{cseq-0.5.zip}.

\subsection{EJB}

Ferramenta de análise estática para criação de diagramas de sequência
disponível em
\url{https://www.dropbox.com/s/glhg8any43lccgm/EJB.zip}.

\subsection{error-prone}

Ferramenta de localização de bugs construída em cima do
compilador {\it javac} disponível em
\url{http://code.google.com/p/error-prone}. O código-fonte utilizado em nosso
estudo obtido no site da ferramenta foi o \texttt{error-prone-2.0.9.tar.gz}.

\subsection{GUIZMO}

Ferramenta de inferência de layout disponível em
\url{http://modelum.es/trac/guizmo/}. O código-fonte
utilizado em nosso estudo obtido no site da ferramenta foi o
\texttt{guizmo-master.zip}.

\subsection{GumTree}

Ferramenta de análise de código-fonte e comparação de mudanças
disponível em \url{https://github.com/jrfaller/gumtree}. O
código-fonte utilizado em nosso estudo obtido no site da ferramenta foi o
\texttt{gumtree-2.0.0.tar.gz}.

\subsection{Indus}

Biblioteca de {\it program
slicing}\footnote{http://en.wikipedia.org/wiki/Program\_slicing} disponível em
\url{http://indus.projects.cis.ksu.edu}. O projeto está organizado em três
módulos, os seguintes arquivos, contendo o código-fonte dos três módulos,
foram copiados localmente para análise:
\texttt{indus.indus-src-20091220.zip},
\texttt{indus.javaslicer-src-20091220.zip} e
\texttt{indus.staticanalyses-src-20070305.zip}.

Não possui número suficiente de releases para ser usado na análise evolutiva.

\subsection{InputTracer}

Ferramenta de análise dinâmica de binários x86 em Linux
disponível em:
\url{http://www.ida.liu.se/divisions/adit/security/InputTracer}. O
código-fonte utilizado em nosso estudo obtido no site da ferramenta foi o
\texttt{valgrind-inputtracer.tar.gz}.

Não possui número suficiente de releases para ser usado na análise evolutiva.

\subsection{JastAdd}

Sistema para análise de código-fonte através da descrição de
atributos via gramática de atributos (AG) disponível em \url{http://jastadd.cs.lth.se/web}. O código-fonte
utilizado em nosso estudo obtido no site da ferramenta foi o
\texttt{jastadd2-src.zip}.

\subsection{JFlow}

Ferramenta de transformação {\it source-to-source} disponível em
\url{http://vazexqi.github.io/JFlow/}. O código-fonte
utilizado em nosso estudo obtido no site da ferramenta foi o
\texttt{vazexqi-JFlow-7cd7eaf.tar.gz}.

\subsection{Lotrack}

Ferramenta de análise estática de configuração disponível em
\url{https://github.com/MaxLillack/Lotrack}. O código-fonte utilizado em nosso
estudo obtido no site da ferramenta foi o \texttt{Lotrack-master.zip}.

Não possui número suficiente de releases para ser usado na análise evolutiva.

\subsection{MPAnalyzer}

Ferramenta de análise de padrões disponível em
\url{https://github.com/YoshikiHigo/MPAnalyzer}. O código-fonte utilizado em
nosso estudo obtido no site da ferramenta foi o \texttt{MPAnalyzer-master.zip}.

\subsection{PtYasm}

Ferramenta de verificação de modelo disponível em
\url{www.cs.toronto.edu/~tomhart/ptyasm}. O código-fonte
utilizado em nosso estudo obtido no site da ferramenta foi o
\texttt{ptyasm.april2008.tgz}.

Não possui número suficiente de releases para ser usado na análise evolutiva.

\subsection{ReAssert}

Ferramenta de localização de falhas em testes e refatoração
desenvolvido como plugin Ecipse disponível em
\url{http://mir.cs.illinois.edu/reassert}. O código-fonte utilizado em nosso
estudo obtido no site da ferramenta foi o \texttt{ReAssert\_0.4.1-src.zip}.

\subsection{Sonar Qube Plug-in}

Plugin para o SourceMeter que extende a análise de
código Java com o uso do SonarQube disponível em:
\url{http://github.com/FrontEndART/SonarQube-plug-in}. O código-fonte
utilizado em nosso estudo obtido no site da ferramenta foi o
\texttt{SonarQube-plug-in-master.zip}.

\subsection{SPARTA}

Ferramenta de análise estática de segurança pra detecção de {\it
malware} disponível em
\url{http://types.cs.washington.edu/sparta/}. O código-fonte utilizado em nosso
estudo obtido no site da ferramenta foi o \texttt{sparta-sparta-1.0.2.tar.gz}.

\subsection{srcML}

Formato texto para representação de código-fonte e um conjunto de
ferramentas de transformação {\it source-to-source} disponível em
\url{http://www.sdml.info/projects/srcml/trunk}\footnote{este endereço
retornou "not found" em contato com os autores por email indicaram que o
projeto foi movido para http://www.srcML.org}. O código-fonte utilizado em
nosso estudo obtido no site da ferramenta foi o \texttt{srcML-src.tar.gz}.

Não possui número suficiente de releases para ser usado na análise evolutiva.

\subsection{TACLE}

Plugin do Eclipse para análise de tipo ({\it Type Analysis}) e
construção de visualizaçao de grafos de chamada ({\it Call Graph}) disponível em
\url{http://presto.cse.ohio-state.edu/tacle}\footnote{este link está
indisponível, por email os autores indicaram o endereço
http://web.cse.ohio-state.edu/~rountev/presto/tacle/TACLE\_Download/tacle.html}.
O código-fonte utilizado em nosso estudo obtido no site da ferramenta foi o
\texttt{tacle\_1\_2\_1\_src.zip}.

\subsection{WALA}

Ferramenta de análise estática para {\it bytecode} Java disponível em
\url{http://wala.sourceforge.net/wiki/index.php/Main_Page}. O código-fonte
utilizado em nosso estudo obtido no site da ferramenta foi o
\texttt{WALA-R\_1.3.8.tar.gz}.

Ferramenta selecionada para análise evolutiva, possui muitos releases e tem tamanho
em número de classes na média.

\subsection{Clang Static Analyzer}

Ferramenta de análise de código-fonte para
localização de bugs em códigos C, C++, e Objective-C disponível em
\url{http://clang-analyzer.llvm.org}. É distribuído junto ao código do próprio
projeto Clang\footnote{http://clang.llvm.org} e em nosso estudo utilizamos o
código em \texttt{cfe-3.7.1.src.tar.xz}.

\subsection{Closure Compiler}

Compilador que traduz código JavaScript em outro
JavaScript melhor e mais otimizado, está disponível em
\url{https://developers.google.com/closure/compiler}\footnote{O código fonte do
Closure Compiler pode ser obtido em:
http://github.com/google/closure-compiler} e foi utilizado em nosso estudo o
seguinte lançamento
\texttt{closure-compiler-closure-compiler-parent-v20160619.tar.gz}.

Ferramenta selecionada para análise evolutiva, possui muitos releases e tem tamanho
em número de classes na média.

\subsection{Cppcheck}

Ferramenta de análise estática de código C/C++ para checagem de vazamento de
memória, erros de alocação, entre outras falhas. Disponível em
\url{http://sourceforge.net/projects/cppcheck}. Em nosso estudo utilizamos o
código em \texttt{cppcheck-1.72.tar.bz2}.

\subsection{CQual}

Ferramenta de análise de typo ({\it type-based analysis}) que fornece um
mecanismo leve e prático para especificação e verificação de propriedades de
programas C. Disponível em \url{http://www.cs.umd.edu/~jfoster/cqual}. Em
nosso estudo utilizamos o código em \texttt{cqual-0.981.tar.gz}.

\subsection{FindBugs}

Uma ferramenta para localização de bugs em código Java disponível em
\url{http://findbugs.sourceforge.net}. Em nosso estudo utilizamos o código em
\texttt{findbugs-3.0.1-source.zip}.

Ferramenta selecionada para análise evolutiva, possui muitos releases e tem tamanho
em número de classes na média.

\subsection{FindSecurityBugs}

Plugin do FindBugs para auditoria de segurança em aplicações web Java,
disponível em \url{http://find-sec-bugs.github.io}.  O código-fonte utilizado
em nosso estudo obtido no site da ferramenta foi o
\texttt{findsecbugs-plugin-1.4.5-sources.jar}.

\subsection{Jlint}

Uma ferramenta para verificaçao de código Java em busca de bugs,
inconsistências e problemas de sincronização disponível em
\url{http://sourceforge.net/projects/jlint}.  O código-fonte utilizado em
nosso estudo obtido no site da ferramenta foi o \texttt{jlint-3.1.2.zip}.

\subsection{Pixy}

Ferramenta de análise estática de código PHP para verificação de
vulnerabilidades de segurança. Disponível em
\url{http://github.com/oliverklee/pixy}. O código-fonte utilizado em nosso
estudo obtido no site da ferramenta foi o \texttt{pixy-master.zip}.

\subsection{PMD}

Ferramenta de análise de código-fonte para localização falhas comuns de
programação com suporte a várias linguagens, disponível em
\url{http://pmd.github.io}.  O código-fonte utilizado em nosso estudo obtido
no site da ferramenta foi o \texttt{pmd-src-5.4.1.zip}.

Ferramenta selecionada para análise evolutiva, possui muitos releases e tem tamanho
em número de classes na média.

\subsection{RATS}

Ferramenta de análise estática para auditoria de segurança disponível em
\url{http://code.google.com/archive/p/rough-auditing-tool-for-security}. O
código-fonte utilizado em nosso estudo obtido no site da ferramenta foi o
\texttt{rats-2.4.tgz}.

\subsection{Smatch}

Ferramenta de análise estática para detecção de erros no Kernel disponível em
\url{http://smatch.sourceforge.net}. O código-fonte utilizado em nosso estudo
obtido no site da ferramenta foi o \texttt{smatch.git}.

\subsection{Splint}

Ferramenta para verificação de programas por vulnerabilidades de segurança e
erros de código. Disponível em \url{http://www.splint.org}. O código-fonte
utilizado em nosso estudo obtido no site da ferramenta foi o
\texttt{splint-3.1.2.src.tgz}.

Não possui número suficiente de releases para ser usado na análise evolutiva.

\subsection{UNO}

Uma ferramenta de análise de código-fonte para detecção de defeitos.
Disponível em \url{http://spinroot.com/uno}. O código-fonte utilizado em nosso
estudo obtido no site da ferramenta foi o \texttt{uno\_v213.tar.gz}.

\subsection{WAP}

Ferramenta para análise estática de código-fonte e mineraçao de dados para
detectar e corrigir vulnerabilidades em aplicações web. Disponível em
\url{http://awap.sourceforge.net}. O código-fonte utilizado em nosso estudo
obtido no site da ferramenta foi o \texttt{wap-2.1.tar.gz}.

\section{Resumo}

De forma que somando as ferramentas selecionadas na academia e na indústria
temos um total de 35 ferramentas, 14 da indústria e 21 da academia.  A Tabela
\ref{total-de-ferramentas} resume este total trazendo o nome de cada ferramenta
e algumas de suas características, a caracterização completa está documentado
no arquivo {\it
ferramentas-e-metricas.ods}\footnote{https://github.com/joenio/dissertacao-ufba-2016/blob/master/dataset/ferramentas-e-metricas.ods}
disponível no repositório desta dissertação.

\begin{table}[H]
  \caption{Resumo da caracterização das ferramentas}
  \centering
  \begin{tabular}{| c | l | l | c | l | l |}
    \hline
    \#  & Ferramentas da academia & Linguagem & Classes & Lançamentos \\
    \hline
    1  & AccessAnalysis          & Java   & 91    & Obsoleta       \\
    2  & Kiasan/Bogor            & Java   & 545   & Obsoleta       \\
    3  & composite               & C      & 17    & -              \\
    4  & CSeq                    & C      & 106   & -              \\
    5  & EJB                     & Java   & 244   & Obsoleta       \\
    6  & error-prone             & Java   & 1703  & Frequentemente \\
    7  & GUIZMO                  & Java   & 306   & -              \\
    8  & GumTree                 & Java   & 306   & Ocasionalmente \\
    9  & Indus                   & Java   & 524   & Obsoleta       \\
    10 & InputTracer             & C      & 2045  & Obsoleta       \\
    11 & JastAdd                 & Java   & 59    & Frequentemente \\
    12 & JFlow                   & Java   & 177   & Obsoleta       \\
    13 & Lotrack                 & Java   & 4059  & -              \\
    14 & MPAnalyzer              & Java   & 240   & -              \\
    15 & PtYasm                  & Java   & 883   & -              \\
    16 & ReAssert                & Java   & 169   & Obsoleta       \\
    17 & Sonar Qube Plug-in      & Java   & 143   & Frequentemente \\
    18 & SPARTA                  & Java   & 268   & Ocasionalmente \\
    19 & srcML                   & C++    & 871   & Ocasionalmente \\
    20 & TACLE                   & Java   & 34    & Obsoleta       \\
    21 & WALA                    & Java   & 2626  & Ocasionalmente \\
    \hline
    \# & Ferramentas da indústria & Linguagem & Classes & Lançamentos \\
    \hline
    22 & Clang Static Analyzer    & C++   & ?     & Frequentemente \\
    23 & Closure Compiler         & Java  & 1842  & Frequentemente \\
    24 & Cppcheck                 & C++   & 338   & Frequentemente \\
    25 & CQual                    & C     & 78    & Obsoleta       \\
    26 & FindBugs                 & Java  & 1486  & Ocasionalmente \\
    27 & FindSecurityBugs         & Java  & 91    & Frequentemente \\
    28 & Jlint                    & C++   & 44    & Obsoleta       \\
    29 & Pixy                     & Java  & 229   & Obsoleta       \\
    30 & PMD                      & Java  & 1340  & Frequentemente \\
    31 & RATS                     & C     & 19    & Obsoleta       \\
    32 & Smatch                   & C     & 483   & Ocasionalmente \\
    33 & Splint                   & C     & 681   & Obsoleta       \\
    34 & UNO                      & C     & 19    & Obsoleta       \\
    35 & WAP                      & Java  & 338   & Frequentemente \\
    \hline
  \end{tabular}
  \label{total-de-ferramentas}
\end{table}

Dentre estas ferramentas as seguintes foram selecionadas para análise evolutiva:

\begin{itemize}
  \item Closure Compiler         
  \item FindBugs                 
  \item PMD                      
  \item WALA                    
\end{itemize}

o clang foi removido pois a analise dele demorou muito, ficou rodando 1 semana q não
terminou, o analizo metrics que demorou tanto assim.

\begin{table}[H]
\caption{Métricas da ferramenta PMD}
  \centering
\begin{tabular}{|l|r|r|r|r|r|}
\hline
\multicolumn{1}{|c|}{\textbf{Release}} & \multicolumn{1}{c|}{\textbf{Classes}} & \multicolumn{1}{c|}{\textbf{CC}} & \multicolumn{1}{c|}{\textbf{SC 75}} & \multicolumn{1}{c|}{\textbf{SC 90}} & \multicolumn{1}{c|}{\textbf{SC 95}} \\ \hline
4.2.5 & 844 & 0,06 & 6 & 15 & 28 \\ \hline
4.3 & 852 & 0,06 & 6 & 15 & 28 \\ \hline
5.0.0 & 1043 & 0,03 & 6 & 14 & 25 \\ \hline
5.0.4 & 1052 & 0,02 & 6 & 12 & 24 \\ \hline
5.1.0 & 1238 & 0,02 & 6 & 12 & 24 \\ \hline
5.1.3 & 1254 & 0,02 & 6 & 12 & 25 \\ \hline
5.2.0 & 1295 & 0,02 & 6 & 12 & 25 \\ \hline
5.3.0 & 1341 & 0,01 & 6 & 12 & 25 \\ \hline
5.3.3 & 1342 & 0,02 & 6 & 12 & 25 \\ \hline
5.3.7 & 1374 & 0,01 & 6 & 12 & 25 \\ \hline
5.4.0 & 1332 & 0,02 & 6 & 12 & 26 \\ \hline
5.4.2 & 1366 & 0,02 & 6 & 12 & 26 \\ \hline
5.5.2 & 1530 & 0,01 & 6 & 12 & 25 \\ \hline
\multicolumn{6}{l}{\texttt{Notas:}} \\
\multicolumn{6}{l}{\texttt{CC = Custo de mudança}} \\
\multicolumn{6}{l}{\texttt{SC = Complexidade estrutural}} \\ \hline
\end{tabular}
\label{metricas-pmd}
\end{table}

\begin{table}[H]
\caption{Métricas da ferramenta WALA}
  \centering
\begin{tabular}{|l|r|r|r|r|r|}
\hline
\multicolumn{1}{|c|}{\textbf{Release}} & \multicolumn{1}{c|}{\textbf{Classes}} & \multicolumn{1}{c|}{\textbf{CC}} & \multicolumn{1}{c|}{\textbf{SC 75}} & \multicolumn{1}{c|}{\textbf{SC 90}} & \multicolumn{1}{c|}{\textbf{SC 95}} \\ \hline
1.0 & 1223 & 0,02 & 8 & 27 & 45 \\ \hline
1.0.02 & 1685 & 0,02 & 8 & 25 & 48 \\ \hline
1.1 & 1872 & 0,02 & 7 & 24 & 48 \\ \hline
1.1.2 & 1720 & 0,02 & 8 & 24 & 50 \\ \hline
1.2 & 1734 & 0,02 & 7 & 25 & 49 \\ \hline
1.2.1.1 & 1901 & 0,02 & 6 & 24 & 48 \\ \hline
1.2.2 & 1903 & 0,02 & 6 & 24 & 48 \\ \hline
1.3 & 1945 & 0,02 & 6 & 24 & 52 \\ \hline
1.3.3 & 2092 & 0,01 & 6 & 24 & 50 \\ \hline
1.3.5 & 2143 & 0,02 & 6 & 22 & 49 \\ \hline
1.3.6 & 2154 & 0,02 & 6 & 22 & 49 \\ \hline
1.3.8 & 2626 & 0,02 & 7 & 24 & 54 \\ \hline
1.3.9 & 2636 & 0,02 & 7 & 24 & 54 \\ \hline
\multicolumn{6}{l}{\texttt{Notas:}} \\
\multicolumn{6}{l}{\texttt{CC = Custo de mudança}} \\
\multicolumn{6}{l}{\texttt{SC = Complexidade estrutural}} \\ \hline
\end{tabular}
\label{metricas-wala}
\end{table}

\begin{table}[H]
\caption{Métricas da ferramenta FindBugs}
  \centering
\begin{tabular}{|l|r|r|r|r|r|}
\hline
\multicolumn{1}{|c|}{\textbf{Release}} & \multicolumn{1}{c|}{\textbf{Classes}} & \multicolumn{1}{c|}{\textbf{CC}} & \multicolumn{1}{c|}{\textbf{SC 75}} & \multicolumn{1}{c|}{\textbf{SC 90}} & \multicolumn{1}{c|}{\textbf{SC 95}} \\ \hline
1.2.1 & 1044 & 0,05 & 7 & 20 & 36 \\ \hline
1.3.4 & 1216 & 0,06 & 7 & 21 & 42 \\ \hline
1.3.5 & 1257 & 0,05 & 7 & 21 & 40 \\ \hline
1.3.6 & 1258 & 0,05 & 8 & 21 & 42 \\ \hline
1.3.7 & 1261 & 0,05 & 7 & 22 & 42 \\ \hline
1.3.8 & 1275 & 0,05 & 7 & 22 & 42 \\ \hline
1.3.9 & 1354 & 0,06 & 7 & 24 & 48 \\ \hline
2.0.0 & 1459 & 0,06 & 7 & 24 & 52 \\ \hline
2.0.1 & 1465 & 0,06 & 7 & 24 & 54 \\ \hline
2.0.2 & 1469 & 0,06 & 7 & 24 & 56 \\ \hline
2.0.3 & 1489 & 0,06 & 7 & 24 & 56 \\ \hline
3.0.0 & 1438 & 0,07 & 7 & 24 & 56 \\ \hline
3.0.1 & 1486 & 0,07 & 8 & 25 & 56 \\ \hline
\multicolumn{6}{l}{\texttt{Notas:}} \\
\multicolumn{6}{l}{\texttt{CC = Custo de mudança}} \\
\multicolumn{6}{l}{\texttt{SC = Complexidade estrutural}} \\ \hline
\end{tabular}
\label{metricas-findbugs}
\end{table}

\begin{table}[H]
\caption{Métricas da ferramenta Closure Compiler}
  \centering
\begin{tabular}{|l|r|r|r|r|r|}
\hline
\multicolumn{1}{|c|}{\textbf{Release}} & \multicolumn{1}{c|}{\textbf{Classes}} & \multicolumn{1}{c|}{\textbf{CC}} & \multicolumn{1}{c|}{\textbf{SC 75}} & \multicolumn{1}{c|}{\textbf{SC 90}} & \multicolumn{1}{c|}{\textbf{SC 95}} \\ \hline
20110119 & 1122 & 0,05 & 4 & 17 & 42 \\ \hline
20110811 & 1730 & 0,1 & 5 & 20 & 40 \\ \hline
20120305 & 1802 & 0,1 & 5 & 20 & 42 \\ \hline
20120917 & 1836 & 0,1 & 6 & 20 & 42 \\ \hline
20130227 & 1759 & 0,11 & 6 & 20 & 48 \\ \hline
20130722 & 1806 & 0,1 & 6 & 20 & 48 \\ \hline
20140110 & 2004 & 0,08 & 5 & 20 & 45 \\ \hline
20140730 & 1573 & 0,04 & 4 & 20 & 48 \\ \hline
20150126 & 1596 & 0,04 & 5 & 22 & 56 \\ \hline
20150729 & 1649 & 0,04 & 6 & 26 & 65 \\ \hline
20160125 & 1724 & 0,04 & 5 & 28 & 68 \\ \hline
20160713 & 1860 & 0,04 & 6 & 30 & 70 \\ \hline
\multicolumn{6}{l}{\texttt{Notas:}} \\
\multicolumn{6}{l}{\texttt{CC = Custo de mudança}} \\
\multicolumn{6}{l}{\texttt{SC = Complexidade estrutural}} \\ \hline
\end{tabular}
\label{metricas-closurecompiler}
\end{table}

As ferramentas com poucas linhas de código foram excluidas, estas
apreentam Change Cost alto, já é conhecido que a definição desta métrica
sofre deste problema, apresenta valores altos em projetos pequenos,
tambem removemos da analise aquelas ferramentas que nao tiveram valor
no percentil 75\%, pois a comparacao e analise se dará neste percentil
principalmente.

Com isso temos 11 projetos, destes iremos analisar longitudalmente
as releases e a evolucao dos valores de SC e CC (Change Cost), sao elas:

 Closure Compiler         13 releases analisados

 FindBugs                 13 releases

 Indus                    (poucos releases, deixando fora da analise longitudinal)

 InputTracer              (nao tem releases disponiveis alem da ultima versao)

 Kiasan/Bogor             (mudou a forma de distribuir ao longo dos releases, dificil obter de forma consistente as versoes)

 Lotrack                  (sem releases, poucos commits no github, apenas 11)

 PMD                      13 releases

 PtYasm                   (nao tem releases disponivel, apenas a ultima versao)

 Splint                   (nao encontrado releases)

 srcML                    (releases nao encontrado)

 WALA                     13 releases


Comparacao entre ferramentas de tamanho similar:

nas 5 comparações de versões distintas com tamanhos similares entre pmd e findbugs,
apresentaram o mesmo resultado, pmd tem valores menos tanto para CC quanto para SC,
indicando que pmd tem um design mais modular que findbugs.

pmd < findbugs
pmd < findbugs
pmd < findbugs
pmd < findbugs
pmd < findbugs

Ao comparar as imagens da matrix DSM dá para notar que isto reflete na matrix, 
pegando o findbugs 3.0.1 e o pmd 5.2.0, é possível notar na matrix que o findbugs
tem mais pontos nas duas diagonais da matrix, indicando dependencias ciclicar, e
design menos modular, enquanto o pmd concentra as dependencias na diagonal inferior
esquerda, indicando poucas dependencias ciclicar e um design mais modular.


findbugs	 findbugs-3.0.1	1486	0,07	8	25	56
pmd	 pmd-src-5.2.0	1295	0,02	6	12	25
/home/joenio/src/dissertacao-ufba-2016/dataset/static-analysis-tools/pmd/pmd-src-5.2.0.analizo.dsm.png
/home/joenio/src/dissertacao-ufba-2016/dataset/static-analysis-tools/findbugs/findbugs-3.0.1.analizo.dsm.png


accessanalysis < findsecuritybugs
mpanalyzer < ejb
indus < bogor
reassert > jflow
pmd-5.4.0 > pmd-5.3.0
pmd-5.4.2 > pmd-5.3.7
findbugs-3.0.1 > findbugs-2.0.3
pixy > mpanalyzer
pmd < closure-compiler

closure-compiler > wala
closure-compiler > wala
wala[Java]     ? closure-compiler[Java] (wala tem CC maior mas SC diferentes)
closure-compiler[Java] ? wala[Java]     (closure tem CC maior mas SC menor)

comparar linguagens diferentes não rola, sempre dá ruim, ver:

rats[C]        ? uno[C]                 (rats tem CC menor e SC maior)
cppcheck[C++]  ? wap[Java]              (cppcheck tem CC menor mas SC maior)
srcml[C++]     ? ptyasm[Java]           (srcml tem CC maior mas SC menor)
closure-compiler[Java] ? inputtracer[C] (closure tem CC menor mas SC diferentes nos percentis)
pmd[Java] ? srcml[C++]                  (pmd tem CC maior e SC diferentes nos percentis)
inputtracer[C] ? wala[Java]             (tem CC maior e SC diferentes mas no percentil 95 tem SC maior também)

fica claro que comparacao entre linguagens diferentes mesmo com tamanhos iguais não dá para chegar a conclusões nenhuma.

findbugs[Java] ? wala[Java]             (findbugs tem CC maior mas SC menor)
closure-compiler ? closure-compiler     (CC menor mas SC maior)
