\xchapter{Caracterização das ferramentas}
{Este capítulo apresenta a caracterização inicial das ferramentas selecionadas a partir da revisão estruturada.}
\label{caracterizacao-ferramentas}

\section{Ferramentas excluídas da análise}

Segue ferramentas citadas nos artigos incluidos na primeita etapa da revisão
estruturada, os artigos incluidos são apenas aqueles que citavam criação de
alguma ferramenta, demais artigos incluidos na primeira seleção mas que não
criavam nenhuma ferramenta não estão listados aqui.

Os artigos que foram incluidos na primeira selecao e nao estao listados aqui de alguma forma
normalmente nao citavam no artigo o nome da ferramenta, nem onden encontrar, muitas vezes
apenas uma prova de conceito e os autores nao deram pistas sobre a ferramenta.

SCAM 2001/2002

\subsection{ReWeb}

00972676 | Restructuring Web Applications via Transformation Rules | inc

Nome: ReWeb (?)
Função: rewrite rules to Web applications
Url: ?
Escrito em: ?
Licença: ?
Autores do artigo: Filippo Ricca
                   Paolo Tonella
                   Ira D. Baxter

Diz que implementa uma prova de conceito que precisa ser evoluido (em trabalhos futuros)
mas nao diz nome ao certo, nem onde encontrar.

\subsection{DMS}

01134100 | Parallel Support for Source Code Analysis and Modification | inc

Nome: DMS Software Reengineering Toolkit
Função: 
Url: ?
Escrito em: PARLANSE
Licença: ?
Autores do artigo: Ira D. Baxter

Os autores também criaram a linguagem PARLANSE (We designed and implemented PARLANSE)

\subsection{EMBER}

01134102 | An Extensible Metrics Extraction Environment for Object-oriented Programming Languages | inc

Nome: EMBER (Extensible Metrics toolBench for Empirical Research)
Função: 
Url: ?
Escrito em: C/C++ (?)
Licença: ?
Autores do artigo: T.J.Harmer
                   F.G.Wilkie


\subsection{VADA}

01134105 | VADA: A Transformation-based System for Variable Dependence Analysis | inc

Nome: VADA
Função: variable dependence analysis
Url: ?
Escrito em: Prolog (?)
Licença: ?
Autores do artigo: Mark Harman
                   Chris Fox
                   Rob Hierons
                   Lin Hu
                   Joachim Wegener
                   Sebastian Danicic

\subsection{ADiMat}

01134106 | Combining source transformation and operator overloading techniques to compute derivatives for MATLAB programs | inc

Nome: ADiMat
Função: 
Url: ?
Escrito em: ?
Licença: ?
Autores do artigo: C. H. Bischof
                   H. M. Bücker
                   B. Lang
                   A. Rasch
                   A. Vehreschild

Foi implementado usando GNU tools Bison [11] and Flex

\subsection{AMOEBA}

01134116 | Semantics Guided Filtering of Combinatorial Graph Transformations in Declarative Equation-Based Languages | inc

Nome: AMOEBA (Automatic MOdelica Equation Based Analyzer)
Função: 
Url: ?
Escrito em: ?
Licença: ?
Autores do artigo: Peter Bunus
                   Peter Fritzson


\subsection{nome?}

01134117 | Visualization of Exception Propagation for Java using Static Analysis | inc

Nome: is implemented on top of Barat framework
Função: exception propagation analysis 
Url: ?
Escrito em: Java 
Licença: ?
Autores do artigo: Byeong-Mo Chang
                   Jang-Wu Jo

SCAM 2013/2014

\subsection{CCW}

06648181 | Code Clustering Workbench | excluido

O artigo disponibiliza um vídeo sobre a instalação e uso da ferramenta no link
abaixo mas o vídeo não está disponível, acesso em 12/nov/2016:

* http://youtu.be/vP552YRyz-s "Este vídeo não está disponível."

Ferramenta chamada CCW, não é uma ferramenta de análise estática e sim de
clusterização de código, utiliza por baixo uma outra ferramenta chamada
DepFinder.

Mas não informa nenhuma referência sobre o DepFinder, nenhum paper, nenhum
link para a ferramenta. Uma busca do Google retornou algumas ferramentas, as
primeiras são:

https://github.com/gapan/depfinder
A tool that finds dependencies of slackware packages

https://github.com/ericdill/depfinder
Find all the unique imports in your library

(não dá para ter certeza sobre quais destas 2 ferramentas o autor se refere
apenas lendo o artigo)

Resultado: este artigo será excluído da lista de artigos selecionados na
primeiro filtro.

\subsection{GASR}

06648184 | Aspectual Source Code Analysis with GASR | incluido

GASR: a source code analysis tool in the tradition of logic program querying
that reasons over A SPECT J source code.

GASR is a source code analysis tool that works, as-is, on ASPECTJ source code only.

Não diz a licença, aparentemente nao é livre, não informa onde encontrar a
ferramenta.

06648184 | Aspectual Source Code Analysis with GASR

Nome: GASR
Função: análise de código fonte
Url: ??
Escrito em: ??
Licença: ??
Autores do artigo: Johan Fabry
                   Coen De Roover
                   Viviane Jonckers



\subsection{SPALTER}

06648185 | Driving a Sound Static Software Analyzer with Branch-and-Bound | incluido

O artigo apresenta um plugin para o Frama-C chamado SPALTER

Informa que o código fonte pode ser encontrado em:
http://www.sts.tu-harburg.de/research/spalter.html

Referencia ao frama-c
http://blog.frama-c.com/index.php?post/2011/11/18/Analyzing-single-precision-floating-point-constants

Não explicita a licença do SPALTER no artigo

"não parece ser exatamente de análise estática, mas fica na lista por
enquanto, olhar mais de perto depois"

06648185 | Driving a Sound Static Software Analyzer with Branch-and-Bound

Nome: SPALTER
Função: plugin do Frama-C com objetivo de melhorar a precisão do Frama-C
Url: http://www.sts.tu-harburg.de/research/spalter.html
Escrito em: ??
Licença: ??
Autores do artigo: Sven Mattsen - sven.mattsen@tuhh.de
                   Pascal Cuoq - pascal.cuoq@cea.fr
                   Sibylle Schupp - schupp@tuhh.de

Informa que o código fonte pode ser encontrado em na URL acima mas não existe
link para download, uma busca no Google também não ajudou.

+ email enviado aos autores solicitando informações de onde encontrar o código fonte
  em 6/fev/2016
  + o autor respondeu em 08/02 enviando em anexo no email e informou que iria verificar
    com o webmaster a solução para por o link de volta online


\subsection{PtrTracker}

06648186 | PtrTracker: Pragmatic Pointer Analysis | incluido

In this work we present a new pointer analysis tool for C/C++ code...

To demonstrate the practicality of the solution we integrate the pointer
analyzer into the C/C++ bug checking tool Goanna...

Pointer Analysis. A pointer analysis (or alias analysis) de
termines a set of all possible (symbolic) memory locations a
pointer variable might point to during execution.

A ferramenta proposta no artigo chama PtrTracker e faz... a
tool that annotates all pointer dereferences in a program with a set of
possible pointees, which themselves are normal variables

Goanna é uma ferramenta de análise estática...

O artigo não informa licença nem dá pistas de onde fazer download da
ferramenta!

06648186 | PtrTracker: Pragmatic Pointer Analysis

Nome: PtrTracker
Função: análise de código C/C++, integra com a ferramenta Goanna
Url: não informado, não encontrado
Escrito em: ??
Licença: ??
Autores do artigo: Sebastian Biallas
                   Mads Chr. Olesen
                   Franck Cassez
                   Ralf Huuck

\subsection{JSNose}

06648192 | JSNose: Detecting JavaScript Code Smells | excluido

We present an automated
technique, called JSN OSE , to detect these code smells. Our
approach uses a metric-based algorithm, and combines static
with dynamic analysis to detect these smells in JavaScript
code.

We implement our approach in a tool called JSNose,
which is freely available;

extract JavaScript code from all .js and HTML
files, (4) parse the source code into an Abstract Syntax Tree
(AST) and analyze it by traversing the tree

We use the WebScarab proxy to intercept the JavaScrip-
t/HTML code. To parse the JavaScript code to an AST and
instrument the code, we use Mozilla Rhino. 2

Diz que está publicamente disponível mas não informa licença no paper:
http://salt.ece.ubc.ca/content/jsnose/

(JSNose não implementa análise estática de fato, e sim o Mozilla Rhino, este
artigo será excluído da seleção)

\subsection{Flowgen}

06975637 | Flowgen: Flowchart-Based Documentation Framework for C++ | inc 1

Apresenta o Flowgen, parseia arquivos C++ e gera diagramas em formato HTML, usa
anotações (comentários com sintaxe especial). Usa Clang como parser C++

Flowgen is complementary to the widely-used Doxy-
gen documentation tool.

A ferramenta foi avaliada com um estudo do software Vincia

O artigo não cita a licença da ferramenta. Escrito em Python.

\subsection{Pangea}

06975639 | Pangea: A Workbench for Statically Analyzing Multi-Language Software Corpora | excluido

Pangea,
an infrastructure allowing fast
development of static analyses on multi-language corpora.

Estudos que fazem análise de grande base de código-fonte (ex varios projetos)
seguem um fluxo comum de obter os fontes, encontrar forma de analisar código,
analisar o codigo, converter num modelo os dados extraidos, refinar os dados
da análise, finalmente publicar a análise.

Estes passos são comuns a todos estudos, 

corpora by providing a
repository of language-independent object model snapshots.

Este repositório de meta-models podem ser usados por pesquisadores para
facilmente comparar, medir, etc...

Pangea stores cross-language software corpora in a centralized repository

FAMIX-
compliant models using MSE as interchange format.
FAMIX is a well-known object-oriented meta-model and
MSE is the default interchange format between a series
of software analysis tools[2].

(a ideia parece com o que pensei para o codejuicer, mas faz muita coisa, ele
fornece ponta-a-ponta, tanto o repositorio e a infra quanto as ferramentas que
o pesquisados deve usar como cliente desta infra)

Mas é muito interessante, com um comando eu posso fazer download dos modelos
(dados já analisados de um certo projeto), posso também baixar o proprio
codigo fonte (n acho necessario).

O repositório atual (da data do artigo) conta com vários projetos Java e
SmallTalk

Estudos interessantes para eu ler no futuro sobre ideias de repositorios de
dados:

Daniel Rodriguez et al. [6] have recently published a survey
of all the existing software engineering repositories publicly
available. Some of the repositories they mention — The
Sourcerer project dataset [7], software-artifact infrastructure
repository [8] — could be taken into consideration to extend
Pangea.Data.
PROMISE [9] is a collection of intermediate results that
have been obtained from previous analyses. Its goal is to
provide raw materials for empirical studies. The provided data
is user contributed, often uncorrelated and usually focused on
very specific aspects and properties of software systems.
[6] D. Rodriguez, I. Herraiz, and R. Harrison, “On software engineering
repositories and their open problems,” in Realizing Artificial Intelligence
Synergies in Software Engineering (RAISE), 2012 First International
Workshop on, june 2012, pp. 52 –56.
[7] S. Bajracharya, J. Ossher, and C. Lopes, “Sourcerer: An internet-scale
software repository,” in Search-Driven Development-Users, Infrastruc-
ture, Tools and Evaluation, 2009. SUITE ’09. ICSE Workshop on, may
2009, pp. 1 –4.
[8] G. R. Group et al., “Software-artifact infrastructure repository (SIR),”
2009.
[9] T. Menzies, B. Caglayan, E. Kocaguneli, J. Krall, F. Peters, and
B. Turhan. (2012, June) The promise repository of empirical software
engineering data. [Online]. Available: http://promisedata.googlecode.com

(este é um projeto muito interessante mas não é uma ferramenta de análise
estática e sairá da coleção de papers do mapeamento)

\subsection{e-munity}

06975645 | Scalable Security Verification of Software at Compile Time | incluido

Implementa uma extensão do GCC chamada e-munity disponível em:

http://sourceforge.net/p/emunity/code/ci/master/tree/

Não cita a licença no artigo.

06975645 | Scalable Security Verification of Software at Compile Time

Nome: e-munity
Função: verificação de segurança em código fonte
Url: http://sourceforge.net/p/emunity/code/ci/master/tree/
Escrito em: C
Licença: não informa
Autores do artigo: Syrine Tlili
                   José M. Fernandez
                   Abdelfettah Belghith
                   Bilel Dridi
                   Soufien Hidouri

O e-munity é um patch em cima do GCC 4.8.2

* https://ftp.gnu.org/gnu/gcc/gcc-4.8.2/

Os resultados da análise desse projeto seriam basicamente os mesmos do GCC, portanto não
faz sentido considerar tal ferramenta para análise.


\subsection{TEBA}

06975663 | A Pattern Search Method for Unpreprocessed C Programs Based on Tokenized Syntax Trees | incluido

Implementar um parser de código C e também de QUERY, não dá um nome, chama
apenas de "search tool" (mentira o nome é TEBA). É um conjunto de ferramentas, disponível em:

http://tebasaki.jp/src

O estudo realiza aplicacao das ferramentas em projetos open source. Não
informa a licença das ferramentas. Escritas em Perl.

Nome: TEBA
Função: parser de C e QUERY
Url: http://tebasaki.jp/src
Escrito em: Perl
Licença: não informado
Autores to artigo: Atsushi Yoshida and Yoshinari Hachisu
                   Faculty of Science and Engineering, Nanzan University, Seto, Aichi, Japan
                   Email: {atsu, hachisu}@nanzan-u.ac.jp

Caracterização prática:

Link disponível: Sim
Código-fonte disponível: Sim
Último release:  Ver.1.04.2 2014/12/04
Licença: não conhecida, usa licença própria mas permissiva
Author: Atsuhi Yoshida

É um conjunto de ferramentas formado por: cparse.pl join-token.pl
rewrite\_token.pl rewrite.pl id\_unify.pl preg.pl rev\_macro.pl ext\_macro.pl
move\_ifdef.pl

Irei utilizar Analizo para avaliar as ferramentas e por hora Perl não é suportado, então
serão incluídos apenas ferramentas escritas em linguagens suportadas pelo Analizo.

SCAM 2015

\subsection{exceptionChecker.jar}

The use of C++ exception handling constructs: A comprehensive study

07335398.pdf: exceptionChecker.jar mas n consegui extrair o arquivo

\subsection{nome?}

Checking C++ codes for compatibility with operator overloading

07335405.pdf: desenvolve uma ferramenta baseada em clang mas n diz onde achar

\subsection{protopurity}

Detecting function purity in JavaScript

07335406.pdf: implementa https://github.com/jensnicolay/jipda/tree/scam2015/protopurity (mas eh JS)

\subsection{nome?}

07335408.pdf: implementa https://github.com/spreadsheetlab/XLParser parser para formulas excel (C\#)

A grammar for spreadsheet formulas evaluated on two large datasets

\subsection{????}

07335419.pdf: ferramenta para mineração em documentacao de apis https://sites.google.com/a/ncsu.edu/apisim/

Discovering likely mappings between APIs using text mining

(nao eh de analise estatica)

\subsection{nome?}

07335421.pdf: http://cse.iitkgp.ac.in/~chitta/pubs/EquivalenceChecker\_FSMDA.tar.gz

A translation validation framework for symbolic value propagation based equivalence checking of FSMDAs

\subsection{FaultBuster}

07335422.pdf: FaultBuster, http://www.sed.inf.u-szeged.hu/FaultBuster, mas não libera os fontes

FaultBuster: An automatic code smell refactoring toolset

\subsection{AVUS}

Improving prioritization of software weaknesses using security models with AVUS

07335423.pdf: https://github.com/Fraunhofer-AISEC/avus, nao eh de analise estatica


\section{Ferramentas incluídas da análise}

\subsection{AccessAnalysis}

AccessAnalysis é um plugin do Eclipse de análise estática 
para cálculo das métricas IGAT e IGAM
publicadas no artigo ``AccessAnalysis -- A Tool for Measuring the
Appropriateness of Access Modifiers in Java Systems'' do SCAM 2012,
disponível em \url{http://accessanalysis.sourceforge.net}. O código-fonte
utilizado em nosso estudo obtido no site da ferramenta foi o
\texttt{AccessAnalysis-1.2-src.zip}. Características da ferramenta:

\begin{description}

  \item {\it Lançamentos ({\it Releases}) - quantos lançamentos por ano:}
    \begin{table}[h!]
      \centering
      \begin{tabular}{| l | l |}
        \hline
        Ano  & Lançamentos    \\
        \hline
        2012 & 1.2, 1.1, 1.0  \\
        2010 & 0.17           \\
        \hline
      \end{tabular}
    \end{table}
    \begin{itemize}
      \item Obsoleta 0 vezes ao ano - intervalo entre novos lançamentos é maior que 1 ano
    \end{itemize}

  \item {\it Linguagem de programação - em qual linguagem a ferramenta é escrita:}
    \begin{itemize}
      \item Java
    \end{itemize}

\end{description}

\subsection{error-prone}

error-prone é uma ferramenta de localização de bugs construída em cima do
compilador {\it javac} publicada no artigo ``Building Useful Program Analysis
Tools Using an Extensible Java Compiler'' do SCAM 2012 disponível em
\url{http://code.google.com/p/error-prone}. O código-fonte utilizado em nosso
estudo obtido no site da ferramenta foi o \texttt{error-prone-2.0.9.tar.gz}.
Características da ferramenta:

\begin{description}

  \item {\it Lançamentos ({\it Releases}) - quantos lançamentos por ano:}
    \begin{table}[h!]
      \centering
      \begin{tabular}{| l | l |}
        \hline
        Ano  & Lançamentos                                          \\
        \hline
        2016 & 2.0.9, 2.0.8                                         \\
        2015 & 2.0.7, 2.0.6, 2.0.5, 2.0.4, 2.0.3, 2.0.2, 2.0.1, 2.0 \\
        \hline
      \end{tabular}
    \end{table}
    \begin{itemize}
      \item Frequentemente $>=$ 3 vezes ao ano - novas versões da ferramenta são lançadas 3 ou mais vezes por ano
    \end{itemize}

  \item {\it Linguagem de programação - em qual linguagem a ferramenta é escrita:}
    \begin{itemize}
      \item Java
    \end{itemize}

\end{description}

\subsection{EJB}

EJB é uma ferramenta de análise estática para criação de diagramas de sequência
publicada no artigo ``I2SD: Reverse Engineering Sequence Diagrams from
Enterprise Java Beans with Interceptors'' do SCAM 2011, disponível em
\url{https://www.dropbox.com/s/glhg8any43lccgm/EJB.zip}. Características da
ferramenta:

\begin{description}

  \item {\it Lançamentos ({\it Releases}) - quantos lançamentos por ano:}

    {\it A versão obtida foi disponibilizada no Dropbox pelos autores após solicitação
    por email, não há informações sobre o histórico de lançamentos.}

    \begin{itemize}
      \item Obsoleta $0$ vezes ao ano - intervalo entre novos lançamentos é maior que 1 ano
    \end{itemize}

  \item {\it Linguagem de programação - em qual linguagem a ferramenta é escrita:}
    \begin{itemize}
      \item Java
    \end{itemize}

\end{description}

\subsection{Indus}

Indus é uma biblioteca de {\it program
slicing}\footnote{http://en.wikipedia.org/wiki/Program\_slicing} publicada no
artigo ``An Overview of the Indus Framework for Analysis and Slicing of
Concurrent Java Software'' do SCAM 2006, disponível em
\url{http://indus.projects.cis.ksu.edu}.  O projeto está organizado em três
módulos, os seguintes arquivos, contendo o código-fonte dos três módulos,
foram copiados localmente para análise:
\texttt{indus.indus-src-20091220.zip},
\texttt{indus.javaslicer-src-20091220.zip} e
\texttt{indus.staticanalyses-src-20070305.zip}. Características da ferramenta:

\begin{description}

  \item {\it Lançamentos ({\it Releases}) - quantos lançamentos por ano:}
    \begin{table}[h!]
      \centering
      \begin{tabular}{| l | l |}
        \hline
        Ano  & Lançamentos                              \\
        \hline
        2007 & 0.8.1, 0.8.3.10, 0.8.3.11, 0.8.3.12, 0.8.3.15, 0.8.3.1, 0.8.3.6, 0.8.3.7, 0.8.3, 0.8 \\
        2006 & 0.7.0, 0.7.1, 0.7.2.2, 0.7.2             \\
        2005 & 0.6.1, 0.6.2, 0.6.3, 0.6.4.1, 0.6.4, 0.5 \\
        2004 & 0.3, 0.2, 0.1, 0.1a                      \\
        \hline
      \end{tabular}
    \end{table}
    \begin{itemize}
      \item Obsoleta $0$ vezes ao ano - intervalo entre novos lançamentos é maior que 1 ano
    \end{itemize}

  \item {\it Linguagem de programação - em qual linguagem a ferramenta é escrita:}
    \begin{itemize}
      \item Java
    \end{itemize}

\end{description}

\subsection{InputTracer}

InputTracer é uma ferramenta de análise dinâmica de binários x86 em Linux
publicado no artigo ``InputTracer: A Data-flow Analysis Tool for Manual
Program Comprehension of x86 Binaries'' do SCAM 2012 disponível em:
\url{http://www.ida.liu.se/divisions/adit/security/InputTracer}. O
código-fonte utilizado em nosso estudo obtido no site da ferramenta foi o
\texttt{valgrind-inputtracer.tar.gz}.  Características da ferramenta:

\begin{description}

  \item {\it Lançamentos ({\it Releases}) - quantos lançamentos por ano:}
    \begin{table}[h!]
      \centering
      \begin{tabular}{| l | l |}
        \hline
        Ano  & Lançamentos                       \\
        \hline
        2011 & 3.6.1                             \\
        2010 & 3.6.0                             \\
        2009 & 3.5.0, 3.4.1, 3.4.0               \\
        2008 & 3.3.1                             \\
        2007 & 3.3.0, 3.2.3, 3.2.2               \\
        2006 & 3.2.1, 3.2.0, 3.1.1               \\
        2005 & 3.1.0, 3.0.1, 3.0.0, 2.4.1, 2.4.0 \\
        2004 & 2.2.0, 2.2.0, 2.1.2, 2.1.1        \\
        2003 & 2.1.0, 2.0.0, 1.9.6, 1.9.5        \\
        \hline
      \end{tabular}
    \end{table}
    \begin{itemize}
      \item Obsoleta $0$ vezes ao ano - intervalo entre novos lançamentos é maior que 1 ano
    \end{itemize}

  \item {\it Linguagem de programação - em qual linguagem a ferramenta é escrita:}
    \begin{itemize}
      \item C
    \end{itemize}

\end{description}

\subsection{JastAdd}

JastAdd é um sistema para análise de código-fonte através da descrição de
atributos via gramática de atributos (AG) publicado no artigo ``Extending
Attribute Grammars with Collection Attributes -- Evaluation and Applications''
do SCAM 2007 disponível em \url{http://jastadd.cs.lth.se/web}. O código-fonte
utilizado em nosso estudo obtido no site da ferramenta foi o
\texttt{jastadd2-src.zip}. Características da ferramenta:

\begin{description}

  \item {\it Lançamentos ({\it Releases}) - quantos lançamentos por ano:}
    \begin{table}[h!]
      \centering
      \begin{tabular}{| l | l |}
        \hline
        Ano  & Lançamentos                              \\
        \hline
        2016 & 2.2.2, 2.2.1, 2.2.1, 2.2.0               \\
        2015 & 2.1.13, 2.1.12, 2.1.11                   \\
        2014 & 2.1.10, 2.1.9, 2.1.8, 2.1.7              \\
        2013 & 2.1.6, 2.1.5, 2.1.4, 2.1.3, 2.1.2, 2.1.1 \\
        \hline
      \end{tabular}
    \end{table}
    \begin{itemize}
      \item Frequentemente $>=$ 3 vezes ao ano - novas versões da ferramenta são lançadas 3 ou mais vezes por ano
    \end{itemize}

  \item {\it Linguagem de programação - em qual linguagem a ferramenta é escrita:}
    \begin{itemize}
      \item Java
    \end{itemize}

\end{description}

\subsection{Sonar Qube Plug-in}

Sonar Qube Plug-in é um plugin para o SourceMeter que extende a análise de
código Java com o uso do SonarQube publicado no artigo ``SourceMeter SonarQube
plug-in'' do SCAM 2014 disponível em:
\url{http://github.com/FrontEndART/SonarQube-plug-in}. O código-fonte
utilizado em nosso estudo obtido no site da ferramenta foi o
\texttt{SonarQube-plug-in-master.zip}. Características da ferramenta:

\begin{description}

  \item {\it Lançamentos ({\it Releases}) - quantos lançamentos por ano:}
    \begin{table}[h!]
      \centering
      \begin{tabular}{| l | l |}
        \hline
        Ano  & Lançamentos       \\
        \hline
        2016 & 8.0               \\
        2015 & 7.0.5, 7.0.4, 7.0 \\
        \hline
      \end{tabular}
    \end{table}
    \begin{itemize}
      \item Frequentemente $>=$ 3 vezes ao ano - novas versões da ferramenta são lançadas 3 ou mais vezes por ano
    \end{itemize}

  \item {\it Linguagem de programação - em qual linguagem a ferramenta é escrita:}
    \begin{itemize}
      \item Java
    \end{itemize}

\end{description}

\subsection{srcML}

srcML é um formato texto para representação de código-fonte e um conjunto de
ferramentas de transformação {\it source-to-source} publicada no artigo
``Lightweight Transformation and Fact Extraction with the srcML Toolkit'' do
SCAM 2011 disponível em
\url{http://www.sdml.info/projects/srcml/trunk}\footnote{este endereço
retornou "not found" em contato com os autores por email indicaram que o
projeto foi movido para http://www.srcML.org}. O código-fonte utilizado em
nosso estudo obtido no site da ferramenta foi o \texttt{srcML-src.tar.gz}.
Características da ferramenta:

\begin{description}

  \item {\it Lançamentos ({\it Releases}) - quantos lançamentos por ano:}
    \begin{table}[h!]
      \centering
      \begin{tabular}{| l | l |}
        \hline
        Ano  & Lançamentos                                                     \\
        \hline
        2015 & 0.9.5                                                           \\
        2014 & 0.8.0, Trunk 19109c, Trunk 19109b, Trunk 19109                  \\
        2013 & Trunk 17088                                                     \\
        2012 & Trunk 13990, Trunk 13953, Trunk 13925, Trunk 13528, Trunk 12359 \\
        2011 & Trunk 8007, Trunk 7990, Trunk 7481                              \\
        \hline
      \end{tabular}
    \end{table}
    \begin{itemize}
      \item Ocasionalmente $<$ 3 vezes ao ano - novas versões da ferramenta são lançadas menos que 3 vezes ao ano
    \end{itemize}

  \item {\it Linguagem de programação - em qual linguagem a ferramenta é escrita:}
    \begin{itemize}
      \item C++
    \end{itemize}

\end{description}

\subsection{TACLE}

TACLE é um plugin do Eclipse para análise de tipo ({\it Type Analysis}) e
construção de visualizaçao de grafos de chamada ({\it Call Graph}) publicado
no artigo ``Estimating the Run-Time Progress of a Call Graph Construction
Algorithm'' do SCAM 2006 disponível em
\url{http://presto.cse.ohio-state.edu/tacle}\footnote{este link está
indisponível, por email os autores indicaram o endereço
http://web.cse.ohio-state.edu/~rountev/presto/tacle/TACLE\_Download/tacle.html}.
O código-fonte utilizado em nosso estudo obtido no site da ferramenta foi o
\texttt{tacle\_1\_2\_1\_src.zip}. Características da ferramenta:

\begin{description}

  \item {\it Lançamentos ({\it Releases}) - quantos lançamentos por ano:}
    \begin{table}[h!]
      \centering
      \begin{tabular}{| l | l |}
        \hline
        Ano  & Lançamentos  \\
        \hline
        2006 & 1.2.1, 1.2.0 \\
        2005 & 1.1.0, 1.0.0 \\
        \hline
      \end{tabular}
    \end{table}
    \begin{itemize}
      \item Obsoleta $0$ vezes ao ano - intervalo entre novos lançamentos é maior que 1 ano
    \end{itemize}

  \item {\it Linguagem de programação - em qual linguagem a ferramenta é escrita:}
    \begin{itemize}
      \item Java
    \end{itemize}

\end{description}

\subsection{WALA}

WALA é uma ferramenta de análise estática para {\it bytecode} Java publicado
no artigo ``Effective Static Analysis to Find Concurrency Bugs In Java'' do
SCAM 2010 disponível em
\url{http://wala.sourceforge.net/wiki/index.php/Main_Page}. O código-fonte
utilizado em nosso estudo obtido no site da ferramenta foi o
\texttt{WALA-R\_1.3.8.tar.gz}. Características da ferramenta:

\begin{description}

  \item {\it Lançamentos ({\it Releases}) - quantos lançamentos por ano:}
    \begin{table}[h!]
      \centering
      \begin{tabular}{| l | l |}
        \hline
        Ano  & Lançamentos                        \\
        \hline
        2016 & 1.3.9                              \\
        2015 & 1.3.8, 1.3.7                       \\
        2013 & 1.3.6, 1.3.5                       \\
        2012 & 1.3.4, 1.3.3                       \\
        2011 & 1.3.2                              \\
        2010 & 1.3.1                              \\
        2009 & 1.3, 1.2.2                         \\
        2008 & 1.2.1, 1.2, 1.1.3, 1.1.2           \\
        2007 & 1.1.1, 1.1, 1.0.04, 1.0.03, 1.0.02 \\
        2006 & 1.0                                \\
        \hline
      \end{tabular}
    \end{table}
    \begin{itemize}
      \item Ocasionalmente $<$ 3 vezes ao ano - novas versões da ferramenta são lançadas menos que 3 vezes ao ano
    \end{itemize}

  \item {\it Linguagem de programação - em qual linguagem a ferramenta é escrita:}
    \begin{itemize}
      \item Java
    \end{itemize}

\end{description}

\subsection{Clang Static Analyzer}

O {\it Clang Static Analyzer} é uma ferramenta de análise de código-fonte para
localização de bugs em códigos C, C++, e Objective-C disponível em
\url{http://clang-analyzer.llvm.org}. É distribuído junto ao código do próprio
projeto Clang\footnote{http://clang.llvm.org} e em nosso estudo utilizamos o
código em \texttt{cfe-3.7.1.src.tar.xz}. Características da ferramenta:

\begin{description}

  \item {\it Lançamentos ({\it Releases}) - quantos lançamentos por ano:}
    \begin{table}[h!]
      \centering
      \begin{tabular}{| l | l |}
        \hline
        Ano  & Lançamentos                              \\
        \hline
        2016 & 3.8.0, 3.7.1                             \\
        2015 & 3.7.0, 3.6.2, 3.6.1, 3.6.0, 3.5.2, 3.5.1 \\
        2014 & 3.5.0, 3.4.2, 3.4.1, 3.4                 \\
        2013 & 3.3                                      \\
        2012 & 3.2, 3.1                                 \\
        2011 & 3.0, 2.9                                 \\
        2010 & 2.8, 2.7                                 \\
        2009 & 2.6, 2.5                                 \\
        2008 & 2.4, 2.3, 2.2                            \\
        2007 & 2.1, 2.0                                 \\
        2006 & 1.9, 1.8, 1.7                            \\
        2005 & 1.6, 1.5                                 \\
        2004 & 1.4, 1.3, 1.2                            \\
        2003 & 1.1, 1.0                                 \\
        \hline
      \end{tabular}
    \end{table}
    \begin{itemize}
      \item Frequentemente $>=$ 3 vezes ao ano - novas versões da ferramenta são lançadas 3 ou mais vezes por ano
    \end{itemize}

  \item {\it Linguagem de programação - em qual linguagem a ferramenta é escrita:}
    \begin{itemize}
      \item C++
    \end{itemize}

\end{description}

\subsection{Closure Compiler}

{\it Closure Compiler} é um compilador que traduz código JavaScript em outro
JavaScript melhor e mais otimizado, está disponível em
\url{https://developers.google.com/closure/compiler}\footnote{O código fonte do
Closure Compiler pode ser obtido em:
http://github.com/google/closure-compiler} e foi utilizado em nosso estudo o
seguinte lançamento
\texttt{closure-compiler-closure-compiler-parent-v20160619.tar.gz}.
Características da ferramenta:

\begin{description}

  \item {\it Lançamentos ({\it Releases}) - quantos lançamentos por ano:}
    \begin{table}[h!]
      \centering
      \begin{tabular}{| l | l |}
        \hline
        Ano  & Lançamentos                              \\
        \hline
        2016 & v20160619, v20160517, v20160315, v20160208, v20160125 \\
        2015 & v20151216, v20151015, v20150920, v20150901, v20150729, v20150609, \\
             & v20150505, v20150315, v20150126 \\
        2014 & v20141215, v20141120, v20141023, v20140923, v20140814, v20140730, \\
             & v20140625, v20140508, v20140407, v20140303, v20140110 \\
        2013 & v20131118, v20131014, v20130823, v20130722, v20130603, v20130411, \\
             & v20130227, v20110811, v20110322, v20110405, v20110119, v20111003, \\
             & v20111114, v20112023, v20120305, v20120430, v20120711, v20121212, \\
             & v20120917 \\
        \hline
      \end{tabular}
    \end{table}
    \begin{itemize}
      \item Frequentemente $>=$ 3 vezes ao ano - novas versões da ferramenta são lançadas 3 ou mais vezes por ano
    \end{itemize}

  \item {\it Linguagem de programação - em qual linguagem a ferramenta é escrita:}
    \begin{itemize}
      \item Java
    \end{itemize}

\end{description}

\subsection{Cppcheck}

Ferramenta de análise estática de código C/C++ para checagem de vazamento de
memória, erros de alocação, entre outras falhas. Disponível em
\url{http://sourceforge.net/projects/cppcheck}. Em nosso estudo utilizamos o
código em \texttt{cppcheck-1.72.tar.bz2}. Características da ferramenta:

\begin{description}

  \item {\it Lançamentos ({\it Releases}) - quantos lançamentos por ano:}
    \begin{table}[h!]
      \centering
      \begin{tabular}{| l | l |}
        \hline
        Ano  & Lançamentos                                                \\
        \hline
        2016 & 1.74, 1.73, 1.72                                           \\
        2015 & 1.71, 1.70, 1.69 1.68                                      \\
        2014 & 1.67, 1.66, 1.65, 1.64, 1.63                               \\
        2013 & 1.62, 1.61, 1.60.1, 1.60, 1.59, 1.58                       \\
        2012 & 1.57, 1.56, 1.55, 1.54, 1.53                               \\
        2011 & 1.52, 1.51, 1.50, 1.49, 1.48, 1.47                         \\
        2010 & 1.46.1, 1.46, 1.45, 1.44, 1.43, 1.42, 1.41, 1.40           \\
        2009 & 1.39, 1.38, 1.37, 1.36, 1.35, 1.34, 1.33, 1.32, 1.31, 1.30 \\
        \hline
      \end{tabular}
    \end{table}
    \begin{itemize}
      \item Frequentemente $>=$ 3 vezes ao ano - novas versões da ferramenta são lançadas 3 ou mais vezes por ano
    \end{itemize}

  \item {\it Linguagem de programação - em qual linguagem a ferramenta é escrita:}
    \begin{itemize}
      \item C++
    \end{itemize}

\end{description}

\subsection{CQual}

Ferramenta de análise de typo ({\it type-based analysis}) que fornece um
mecanismo leve e prático para especificação e verificação de propriedades de
programas C. Disponível em \url{http://www.cs.umd.edu/~jfoster/cqual}. Em
nosso estudo utilizamos o código em \texttt{cqual-0.981.tar.gz}.
Características da ferramenta:

\begin{description}

  \item {\it Lançamentos ({\it Releases}) - quantos lançamentos por ano:}
    \begin{table}[h!]
      \centering
      \begin{tabular}{| l | l |}
        \hline
        Ano  & Lançamentos  \\
        \hline
        2004 & 0.981, 0.991 \\
        2003 & 0.98, 0.99   \\
        \hline
      \end{tabular}
    \end{table}
    \begin{itemize}
      \item Obsoleta $0$ vezes ao ano - intervalo entre novos lançamentos é maior que 1 ano
    \end{itemize}

  \item {\it Linguagem de programação - em qual linguagem a ferramenta é escrita:}
    \begin{itemize}
      \item C
    \end{itemize}

\end{description}

\subsection{FindBugs}

Uma ferramenta para localização de bugs em código Java disponível em
\url{http://findbugs.sourceforge.net}. Em nosso estudo utilizamos o código em
\texttt{findbugs-3.0.1-source.zip}. Características da ferramenta:

\begin{description}

  \item {\it Lançamentos ({\it Releases}) - quantos lançamentos por ano:}
    \begin{table}[h!]
      \centering
      \begin{tabular}{| l | l |}
        \hline
        Ano  & Lançamentos                \\
        \hline
        2015 & 3.0.1                      \\
        2014 & 3.0.0                      \\
        2013 & 2.0.3                      \\
        2012 & 2.0.2, 2.0.1               \\
        2011 & 2.0.0                      \\
        2009 & 1.3.9, 1.3.8               \\
        2008 & 1.3.7, 1.3.6, 1.3.5, 1.3.4 \\
        2007 & 1.2.1                      \\
        \hline
      \end{tabular}
    \end{table}
    \begin{itemize}
      \item Ocasionalmente $<$ 3 vezes ao ano - novas versões da ferramenta são lançadas menos que 3 vezes ao ano
    \end{itemize}

  \item {\it Linguagem de programação - em qual linguagem a ferramenta é escrita:}
    \begin{itemize}
      \item Java
    \end{itemize}

\end{description}

\subsection{FindSecurityBugs}

Plugin do FindBugs para auditoria de segurança em aplicações web Java,
disponível em \url{http://find-sec-bugs.github.io}.  O código-fonte utilizado
em nosso estudo obtido no site da ferramenta foi o
\texttt{findsecbugs-plugin-1.4.5-sources.jar}. Características da ferramenta:

\begin{description}

  \item {\it Lançamentos ({\it Releases}) - quantos lançamentos por ano:}
    \begin{table}[h!]
      \centering
      \begin{tabular}{| l | l |}
        \hline
        Ano  & Lançamentos                                     \\
        \hline
        2016 & 1.4.6, 1.4.5                                    \\
        2015 & 1.4.4, 1.4.3, 1.4.2, 1.4.1, 1.4.0, 1.3.1, 1.3.0 \\
        2014 & 1.2.1                                           \\
        2013 & 1.2.0, 1.1.0                                    \\
        2012 & 1.0.0                                           \\
        \hline
      \end{tabular}
    \end{table}
    \begin{itemize}
      \item Frequentemente $>=$ 3 vezes ao ano - novas versões da ferramenta são lançadas 3 ou mais vezes por ano
    \end{itemize}

  \item {\it Linguagem de programação - em qual linguagem a ferramenta é escrita:}
    \begin{itemize}
      \item Java
    \end{itemize}

\end{description}

\subsection{Jlint}

Uma ferramenta para verificaçao de código Java em busca de bugs,
inconsistências e problemas de sincronização disponível em
\url{http://sourceforge.net/projects/jlint}.  O código-fonte utilizado em
nosso estudo obtido no site da ferramenta foi o \texttt{jlint-3.1.2.zip}.
Características da ferramenta:

\begin{description}

  \item {\it Lançamentos ({\it Releases}) - quantos lançamentos por ano:}
    \begin{table}[h!]
      \centering
      \begin{tabular}{| l | l |}
        \hline
        Ano  & Lançamentos \\
        \hline
        2011 & 3.1.2       \\
        2010 & 3.1.1       \\
        2006 & 3.1         \\
        2004 & 3.0         \\
        \hline
      \end{tabular}
    \end{table}
    \begin{itemize}
      \item Obsoleta $0$ vezes ao ano - intervalo entre novos lançamentos é maior que 1 ano
    \end{itemize}

  \item {\it Linguagem de programação - em qual linguagem a ferramenta é escrita:}
    \begin{itemize}
      \item C++
    \end{itemize}

\end{description}

\subsection{Pixy}

Ferramenta de análise estática de código PHP para verificação de
vulnerabilidades de segurança. Disponível em
\url{http://github.com/oliverklee/pixy}. O código-fonte utilizado em nosso
estudo obtido no site da ferramenta foi o \texttt{pixy-master.zip}.
Características da ferramenta:

\begin{description}

  \item {\it Lançamentos ({\it Releases}) - quantos lançamentos por ano:}
    \begin{table}[h!]
      \centering
      \begin{tabular}{| l | l |}
        \hline
        Ano  & Lançamentos \\
        \hline
        2012 & 3.0.3       \\
        \hline
      \end{tabular}
    \end{table}
    \begin{itemize}
      \item Obsoleta $0$ vezes ao ano - intervalo entre novos lançamentos é maior que 1 ano
    \end{itemize}

  \item {\it Linguagem de programação - em qual linguagem a ferramenta é escrita:}
    \begin{itemize}
      \item Java
    \end{itemize}

\end{description}

\subsection{PMD}

Ferramenta de análise de código-fonte para localização falhas comuns de
programação com suporte a várias linguagens, disponível em
\url{http://pmd.github.io}.  O código-fonte utilizado em nosso estudo obtido
no site da ferramenta foi o \texttt{pmd-src-5.4.1.zip}. Características da
ferramenta:

\begin{description}

  \item {\it Lançamentos ({\it Releases}) - quantos lançamentos por ano:}
    \begin{table}[h!]
      \centering
      \begin{tabular}{| l | l |}
        \hline
        Ano  & Lançamentos                                                   \\
        \hline
        2016 & 5.5.0, 5.4.2, 5.3.7                                           \\
        2015 & 5.4.1, 5.4.0, 5.3.6, 5.3.5, 5.3.4, 5.3.3, 5.3.2, 5.3.1, 5.3.0 \\
        2014 & 5.2.3, 5.2.2, 5.2.1, 5.2.0, 5.1.3, 5.1.2, 5.1.1, 5.1.0        \\
        2013 & 5.0.5, 5.0.4, 5.0.3, 5.0.2                                    \\
        2012 & 5.0.1, 5.0.0                                                  \\
        2011 & 4.3.0, 4.2.6                                                  \\
        2009 & 4.2.5                                                         \\
        \hline
      \end{tabular}
    \end{table}
    \begin{itemize}
      \item Frequentemente $>=$ 3 vezes ao ano - novas versões da ferramenta são lançadas 3 ou mais vezes por ano
    \end{itemize}

  \item {\it Linguagem de programação - em qual linguagem a ferramenta é escrita:}
    \begin{itemize}
      \item Java
    \end{itemize}

\end{description}

\subsection{RATS}

Ferramenta de análise estática para auditoria de segurança disponível em
\url{http://code.google.com/archive/p/rough-auditing-tool-for-security}. O
código-fonte utilizado em nosso estudo obtido no site da ferramenta foi o
\texttt{rats-2.4.tgz}. Características da ferramenta:

\begin{description}

  \item {\it Lançamentos ({\it Releases}) - quantos lançamentos por ano:}
    \begin{table}[h!]
      \centering
      \begin{tabular}{| l | l |}
        \hline
        Ano  & Lançamentos \\
        \hline
        2013 & 2.4         \\
        2009 & 2.3         \\
        ??   & 1.5         \\
        \hline
      \end{tabular}
    \end{table}
    \begin{itemize}
      \item Obsoleta $0$ vezes ao ano - intervalo entre novos lançamentos é maior que 1 ano
    \end{itemize}

  \item {\it Linguagem de programação - em qual linguagem a ferramenta é escrita:}
    \begin{itemize}
      \item C
    \end{itemize}

\end{description}

\subsection{Smatch}

Ferramenta de análise estática para detecção de erros no Kernel disponível em
\url{http://smatch.sourceforge.net}. O código-fonte utilizado em nosso estudo
obtido no site da ferramenta foi o \texttt{smatch.git}. Características da
ferramenta:

\begin{description}

  \item {\it Lançamentos ({\it Releases}) - quantos lançamentos por ano:}
    \begin{table}[h!]
      \centering
      \begin{tabular}{| l | l |}
        \hline
        Ano  & Lançamentos      \\
        \hline
        2015 & 1.60             \\
        2013 & 1.59, 1.58, 1.57 \\
        2012 & 1.56             \\
        2010 & 1.55, 1.54       \\
        2009 & 1.53, 1.52, 1.51 \\
        \hline
      \end{tabular}
    \end{table}
    \begin{itemize}
      \item Ocasionalmente $<$ 3 vezes ao ano - novas versões da ferramenta são lançadas menos que 3 vezes ao ano
    \end{itemize}

  \item {\it Linguagem de programação - em qual linguagem a ferramenta é escrita:}
    \begin{itemize}
      \item C
    \end{itemize}

\end{description}

\subsection{Splint}

Splint is a tool for statically checking C programs for security
vulnerabilities and coding mistakes Ferramenta para verificação de programas
por vulnerabilidades de segurança e erros de código. Disponível em
\url{http://www.splint.org}. O código-fonte utilizado em nosso estudo obtido
no site da ferramenta foi o \texttt{splint-3.1.2.src.tgz}. Características da
ferramenta:

\begin{description}

  \item {\it Lançamentos ({\it Releases}) - quantos lançamentos por ano:}
    \begin{table}[h!]
      \centering
      \begin{tabular}{| l | l |}
        \hline
        Ano  & Lançamentos                        \\
        \hline
        2007 & 3.1.2                              \\
        2003 & 3.1.0                              \\
        ??   & 3.0.1.6                            \\
        2002 & 3.0.1.5                            \\
        ??   & 3.0.1.4, 3.0.1, 3.0.0.19, 3.0.0.18 \\
        ??   & 3.0.0.17, 3.0.0.15, 3.0.0.14       \\
        2001 & 3.0.0.13                           \\
        \hline
      \end{tabular}
    \end{table}
    \begin{itemize}
      \item Obsoleta $0$ vezes ao ano - intervalo entre novos lançamentos é maior que 1 ano
    \end{itemize}

  \item {\it Linguagem de programação - em qual linguagem a ferramenta é escrita:}
    \begin{itemize}
      \item C
    \end{itemize}

\end{description}

\subsection{UNO}

Uma ferramenta de análise de código-fonte para detecção de defeitos.
Disponível em \url{http://spinroot.com/uno}. O código-fonte utilizado em nosso
estudo obtido no site da ferramenta foi o \texttt{uno\_v213.tar.gz}.
Características da ferramenta:

\begin{description}

  \item {\it Lançamentos ({\it Releases}) - quantos lançamentos por ano:}
    \begin{table}[h!]
      \centering
      \begin{tabular}{| l | l |}
        \hline
        Ano  & Lançamentos                       \\
        \hline
        2007 & 2.13, 2.12, 2.11                  \\
        2006 & 2.9-2.10                          \\
        2005 & 2.8, 2.7, 2.6, 2.5, 2.4           \\
        2004 & 2.3, 2.2, 2.1, 2.0, 1.7           \\
        2003 & 1.8, 1.6, 1.5, 1.4, 1.3, 1.2, 1.1 \\
        \hline
      \end{tabular}
    \end{table}
    \begin{itemize}
      \item Obsoleta $0$ vezes ao ano - intervalo entre novos lançamentos é maior que 1 ano
    \end{itemize}

  \item {\it Linguagem de programação - em qual linguagem a ferramenta é escrita:}
    \begin{itemize}
      \item C
    \end{itemize}

\end{description}

\subsection{WAP}

Ferramenta para análise estática de código-fonte e mineraçao de dados para
detectar e corrigir vulnerabilidades em aplicações web. Disponível em
\url{http://awap.sourceforge.net}. O código-fonte utilizado em nosso estudo
obtido no site da ferramenta foi o \texttt{wap-2.1.tar.gz}. Características da
ferramenta:

\begin{description}

  \item {\it Lançamentos ({\it Releases}) - quantos lançamentos por ano:}
    \begin{table}[h!]
      \centering
      \begin{tabular}{| l | l |}
        \hline
        Ano  & Lançamentos                                 \\
        \hline
        2015 & 2.1, 2.0.5, 2.0.4, 2.0.3, 2.0.2, 2.0.1, 2.0 \\
        \hline
      \end{tabular}
    \end{table}
    \begin{itemize}
      \item Frequentemente $>=$ 3 vezes ao ano - novas versões da ferramenta são lançadas 3 ou mais vezes por ano
    \end{itemize}

  \item {\it Linguagem de programação - em qual linguagem a ferramenta é escrita:}
    \begin{itemize}
      \item Java
    \end{itemize}

\end{description}
