\section{Software acadêmico}

Segundo \citeonline{allen2017engineering} software acadêmico ({\it academic
software}) é todo software usado para coletar, processar ou analisar resultados
de pesquisas com intenção de ser publicados na literatura academica (seja num
jornal, revista, conferência, monografia, livro ou tese), podem ser desde
protótipos escritos pelos próprios cientistas, até mesmo produtos completos
desenvolvidos profissionalmente.

Estes softwares são usados ativamente em diversos campos de pesquisa, como
matemática, biologia, física de partículas, astronomia, medicina e direito,
eles resolvem problemas comuns do cotidiano de pelo menos metade dos
pesquisadores de todas as áreas, desde grupos trabalhando exclusivamente com
problemas computacionais até grupos em laboratórios tradicionais ou em campo
\cite{wilson2014best}.

Além da aplicação, estes softwares variam também no papel que ocupam em suas
pesquisas, alguns fazem parte dos resultados da pesquisa, como por exemplo,
propostas de novos algoritmos ou técnicas de produção, outros são utilizados
como parte do método de pesquisa, como coleta ou análise de dados, sendo que
estes papeis não são excludentes.

%estes costumam ser citados pelos seus autores como uma das contribuições do
%estudo, seja principal ou secundária, 
%Esses softwares podem, de fato, ser um software de simulação complexo desenvolvido
%e executado em um computador de alto desempenho, mas também pode ser um
%software desenvolvido em um PC para incorporação em instrumentos; para
%manipular, analisar ou visualizar dados; ou para orquestrar fluxos de trabalho.

Diversos grupos acadêmicos tem aumentado a dependencia de softwares e à medida
que percebe-se que os softwares estão se tornando parte integrante dos
processos, ferramentas e produção científicas, torna-se necessário e urgente
discutir o seu desenvolvimento, visibilidade, qualidade e sustentabilidade.

%a uma discussão e sobre os softwares acadêmicos.
%tem
%percebido que os softwares precisam ser parte integral do prática científica
%\cite{momcheva2015software},

\section{Ecosistema de software acadêmico}

Unlike other technologies supporting science, software can
be copied and distributed at essentially no cost, potentially
opening the door to unprecedented levels of sharing and col-
laborative innovation. \cite{howison2011scientific}

While some of these seem relatively unproblematic, such as commercial
production in fields with immediately valuable applications, others appear
problematic. In particular we highlighted the potentially pernicious
implications of the academic credit production system for collaboration and
maintenance.

\subsection{Usuário final}

Cientistas engajado em domínio da ciência ocupam papel chave no ecosistema de
software acadêmico, estão dirigindo suas pesquisas e investigações, e no
processo juntaram artefatos de software para coletar, gerenciar, formatar,
analisar, modelar e visualizar dados, com o objetivo final de publicar seus
resultados na literatura acadêmica.

Estão preocupados com disponibilidade, qualidade e usabilidade dos softwares,
também com a capacidade de continuar úteis cientificamente, e poderão continuar
serem utilizados em conjunto com outros softwares.

Estes usuários estão interessados em saber o que os outros cientistas estão
usando, softwares com alta adoção no domínio mantém os pesquisadores mais
focados em suas contribuições científicas, uma vez que os revisores tem alta
chance de entenderem o método utilizado.

Uma segunda razão ...

Finalmente softwares muito utilizados e adotados são tanto sinal de sua qualidade
quanto a garantia que a equipe, estudantes e colaboradores conseguirão encontrar
e possivelmente encontrar ajuda entre outros peers para resolver questões sobre o
software, liberando assim para terem foco na pesquisa.

\subsection{Produção de software}

Um segundo papel é o que produtor de softwares acadêmicos, e distribuição. Varia
de indivíduos a times, ambos chamado de projetos para facilidade, produzindo
artefatos que são usados por pequenos grupos ou disciplinas inteiras.

Alguns softwares são desenvolvidos e muitas vezes ficam confinados em seus laboratórios
ou grupos, mas eventualmente são compartilhados e potencialmente amplamente adotados, tornando
o cientista parte do ecosistema do software (Van de Geijin 1997).

Muitas vezes são desenvolvidos em colaborações próximas entre cientistas de computação e
outras áreas, eles desenvolvem algoritmos que refletem as pesquisas dos pesquisadores
do domínio em questão. Um desafio é abstrair os problemas e implementar soluções que
podem ser adotadas por outros cientistas, especialmente em outros domínios.

Projetos se preocupam com o impacto cientifico tanto em termos de numero quando
tipo de usuários seu software atinge, e como seus softwares contribuem para a
ciencia que outros estão realizando.

Alguns projetos são gerenciados no estilo de código aberto, e tem
atraido com sucesso contribuições de muitos cientistas, incluindo
uma calda longa de contriuição que tem feito pequenas mas substanciais
contribuições.

\subsection{Distribuição}

Um terceiro papel é quem provê um cojunto de softwares que são
disponíveis aos usuários finais em seus trabalhos de pesquisa.
Este papel pode ser desempenhado como ciberinfraestrutura de software
em centros de supercomputação através de provedores de computação
em nuvem ou podem ser distribuidos, podem ser disponiveis para usuarios
fazer download em seus computadores pessoais.

Do ponto de vista de ecosistema, ambos os tipos de distribuiça
estão interessados nas mesmas questões: quem usa, ou não usa?
Qual versão usa? Em qual frequencia atualiza?

\subsection{"Pensadores" do ecosistema}

Um último papel é daqueles preocupados sobre o funcionamento do
próprio ecosistema e sua contribuição para a ciência.

Costuma ser agencias de financiamento mas pode ser também cientistas sênior do
domínio que refletem além de seu trabalho individual e pensam em seus campos
como um todo.

As preocupações rondam ao redor de questões sobre operação do ecosistema como
um sistema que toma recursos (tempo, dinheiro e atenção) e afeta a conduta da
ciência, tanto o geral como em campos individuais, complementado pelo interesse
de saber como o comportamento desse sistema pode ser influenciado.

A visão da ciberinfraestrutura, expressada no "Relatório Atkins" e instanciada
para o ecossistema de software científico na NSF chamada de Infraestrutura de
Software para Inovação Sustentada (NSF SI2),

Os softwares vem não apenas atuando no avanço da ciência, mas atuando com uma
eficiência crescente ao longo do tempo (Atkins 2003). A chave para isso é a
crença de que o software deve evoluir em direção a uma plataforma
compartilhada, com componentes que são reutilizados o mais amplamente possível,
já que os usuários finais e os produtores de componentes se agrupam em torno de
peças específicas de software.

a literatura sobre plataformas de software fora da ciencia tem chamado isso de
'coring' e 'tipping' (Gawer and Cusumano 2008),
onde uma comunidade descobre sua funcionalidade compartilhada e se agrupa
em pacotes que fornecem, levando ao uso eficiente de recursos através de
economias de escala.

coring também resulta em um aumento do uso sobreposto que facilita mais

Isto também resulta em um aumento do uso sobreposto que facilita mais
transparência na ciência, levando a uma maior qualidade e correctude (correctness), à medida
que mais olhos e esforços são direcionados para os mesmos códigos que são
sustentados e evoluem em longos períodos de utilidade científica.

coring em direção às plataformas pode ser contrastado com o seu oposto, muitas
vezes percebido por informantes: churn caótico disfuncional, com muitos
projetos com poucos usuários, cada um tendo vidas curtas que terminam com o
financiamento de concessão inicial, comunidades desconectadas e paralelas,
incompatibilidades teimosamente imutáveis e periódicas e tentativas
aparentemente não coordenadas de "reiniciar". Subjacente a isso é uma
preocupação que as oportunidades são perdidas e que o progresso da ciência é
abrandado (por exemplo, Stewart, Almes e Wheeler 2010).

Essas preocupações gerais sugerem um conjunto de questões específicas, com foco
em padrões globais e padrões emergentes dentro do ecossistema, incluindo: Quais
recursos foram destinados à produção de software? Quantos usuários ou
comunidades de usuários têm projetos? Quais são os impactos científicos desse
uso? Os números de usuários crescem? Os projetos possuem recursos e habilidades
suficientes para gerenciar seu crescimento? Quais projetos possuem
funcionalidades sobrepostas? Há quanto tempo os pedaços de software e projetos
persistem? Nós desconectamos as comunidades de usuários e desenvolvedores? São
componentes específicos, ou camadas de componentes, faltam? Que código
geralmente é usado em conjunto; são os projetos e as pessoas que produzem esses
componentes se comunicando adequadamente? Como podemos sustentar o software
crítico?

Junto com estas questões estão as questões de como influenciar o ecossistema,
incluindo questões de pontos de inflexão que levam ao uso coalescente, bem como
a intervenções políticas diretas incentivando o uso de componentes específicos.
Aqui há uma clara tensão entre um desejo de flexibilidade e liberdade, ligado
às expectativas de inovação científica e desejos de estruturas de autoridade e
controle de coordenação. As questões de influência incluem: como os programas
de financiamento e quais os requisitos em suas chamadas, resultaram em software
amplamente utilizado e impacto científico substancial? Quais são as
características dos campos que alcançaram maior coalescência? Quais jornais e
conferências têm políticas exemplares? Como o trabalho de software é visto
dentro das práticas de contratação e avaliação, como os casos de posse?

\cite{howison2015understanding}

\section{Métricas de ecosistema de software acadêmico}

Os interesses dos papeis envolvidos no ecosistema são muitos e variados,
eles se estendem através do ciclo de vida da criação, distribuição, uso e avaliação.

(figura aqui: a process model of software in science)

Recursos são devotados para a produção de software acadêmico. Usuários finais
cientistas (diretamente ou indiretamente) usam softwares acadêmicos para fazer
ciência, resultando em impacto científico. Impacto científico então justifica
investimentos e recursos, seja de forma antecipada para fins de planejamento,
seja como retrospectiva para avaliar investimentos realizados.

Muitas das métricas focam em apenas um único elemento isoladamente, perdendo
a importancia do ecosistema, ...

\subsection{Recursos para o software}



%%%%%%%%%%%%%%%%%%%%%%%%%%%%%%%%%%%%%%%%%%%%

\section{Problemas percebidos...}

\subsection{Desenvolvimento}

O desenvolvimento de software acadêmico, assim como de qualquer outro
software, exige conhecimento sobre o domínio de aplicação, o que torna o
seu desenvolvimento essencialmente diferente é que este conhecimento pode ser,
por exemplo, entender como o DNA genômico se transforma em cristais de
proteína, ou os meandros da dinâmica dos fluidos, ou como resolver 20 equações
diferenciais parciais simultâneas \cite{segal2008developing}.

Isto faz com que os próprios cientistas acabem desenvolvendo os seus softwares,
um estudo entre cientistas do reino unido mostrou, por exemplo, que 56\%
desenvolvem seus próprios softwares, pelo menos pacialmente
\cite{hettrick_2014_14809}, outro estudo similar mostrou que na astronomia este
número chega a 90\% \cite{momcheva2015software}.

A maior parte dos cientistas, no entanto, não possuem treinamento algum sobre
como escrever softwares de forma eficiente, faltam práticas básicas de
desenvolvimento, como escrever código legível, revisão de código, controle de
versão, testes unitários, entre outros \cite{wilson2017good}.

%ocasionando sérios
%erros computacionais em conclusões centrais da literatura acadêmica, gerando
%retrabalho para retratar tais erros nas mais diversas áreas da ciência
%\cite{Merali2010Computational}.
%
%como resultado, dados são perdidos,
%análises levam mais tempo que o necessário e os pesquisadores não conseguem a
%eficiência que poderiam ter ao trabalhar com softwares acadêmicos
%\cite{wilson2017good}.

Como consequência, muitos não testam ou documentam os seus softwares, causando
um impacto negativo na visibilidade dos softwares acadêmicos \cite{howison2013,
katz2014transitive} e na capacidade de serem encontrados e compartilhados.

%, e faz
%surgir questionamentos sobre sua qualidade, não apenas técnica, mas também a
%capacidade de ser encontrado, compartilhado e co-desenvolvido, qualidades
%importantes para a evolução do próprio software, mas também extremamente úteis
%para o uso eficiente dos limitados recursos da ciência \cite{howison2013,
%katz2014transitive}.

\subsection{Visibilidade}

Estudos tem mostrado que muitas pesquisas não mencionam o uso de softwares
mesmo tendo feito uso em seu método, mostram ainda que a maneira de citar
varia enormemente, prejudicando a visibilidade, ou ao menos, deixando de
contribuir, com a visibilidade destes softwares \cite{howison2016software}.

Não existe ainda amadurecimento suficiente sobre como citar softwares e
outros artefatos em pesquisas científicas, não temos um padrão de como fazê-lo,
cada autor cita à sua maneira, muitas vezes ao longo do texto, outras em seções
específicas sobre a implementação do software, nem semprem informam onde
encontrar uma cópia do software, ou ainda nem sobre o modelo em que o software
é distribuído, ou se é de alguma forma distribuído ao público.

%Softwares acadêmicos ainda não recebem o devido reconhecimento,
%muitas pesquisas nem ao menos mencionam sua utilização, um estudo recente com
%90 artigos de diversas áreas da biologia, selecionados aleatoriamente entre
%publicações usando softwares como método, mostrou que apenas 59 mencionavam o
%uso de softwares de alguma forma, os demais 31 artigos, apesar de usar software
%acadêmico, não mencionavam nada a respeito \cite{howison2016software},
%apenas entre 31\% e
%43\% das menções aos softwares acadêmicos envolvem citação formal.

Quando um software não é visível, ele é frequentemente excluido de {\it peer
reviews}, citações formais facilitam e promovem o avanço da ciência \cite{allen2014credit}, a carência
de um modelo de citação aos softwares aceito e em prática pelos pesquisadores
impacta negativo na visibilidade dos softwares acadêmicos e faz surgir uma
série de questionamentos sobre a sua qualidade e a capacidade de ser
encontrado, compartilhado e co-desenvolvido \cite{howison2013,
katz2014transitive} \cite{howison2016software}.

%historia da citacao na ciencia, como isso promove o avanço, problemas para
%citacao em artefatos digitais, solucao para identificador unico de autores de
%artigos, orcid.org resolve este problema, o mesmo para identificar artefatos
%digitais é o doi.org \cite{allen2014credit}.

%; um software em bom funcionamento devem atingir não apenas os
%objetivos de entendimento e transparencia, mas também os objetivos voltados
%para replicação \cite{Stodden2010}.

\subsection{Qualidade}

A qualidade dos softwares acadêmicos tem sido questionada,
a maioria também não sabe o quão confiável seu software é \cite{Merali2010Computational},
muitos estão em estado inicial de desenvolvimento \cite{marshall2013tools},
poucas foram testados fora do contexto onde foi desenvolvido \cite{Portillo12}.

%Cita um mapeamento sistemático com objetivo de encontrar ferramentas de
%comunicação e coordenação para suporte a times altamente distribuidos
%gograficamente, encontrou 132 ferramentas, para uso em projetos de software
%global. A maioria destas ferramentas foram desenvolvidas em centros de
%pesquisas, e apenas uma pequena porcentagem (18.9\%) foram testados fora do
%seu contexto onde foi desenvolvido \cite{Portillo12}.
%
%Cita um mapeamento feito sobre estudos que criam ferramentas para apoio a
%revisão sistemática no domínio de SE, 14 estudos foram selecionados, ao final
%apenas 8 tinham proposta de ferramentas, ao final conclui que as ferramentas
%encontradas estão em estado inicial de desenvolvimento \cite{marshall2013tools}.

A adoção e uso de softwares acadêmicos está relacionada, entre outros fatores,
também à sua qualidade, portanto é impoprtante medir e coletar sua qualidade de
alguma forma, qualidade é um vasto assunto, um dos problemas comuns enfrentado
pelos pesquisadores que desenvolvem tais softwares é a manutenabilidade
\cite{Prlic2012}.

Apesar de nem sempre ser possível, ou viável, ter tudo dentro de padrões
estritos, é preciso estar consciente das boas práticas ao produzir e utilizar
softwares acadêmicos, tanto para melhorar a própria abordagem quanto para
revisar outros trabalhos \cite{wilson2014best}.

Mas não só de qualidade interna vive o software, a capacidade de estar disponível, seja
de forma comercial, gratuita ou livre, documentação, instruções de uso, etc.

\subsection{Sustentabilidade} \label{sustentabilidade}

Se sustentabilidade não for levada em consideração em projetos de software, não
importa qual o domínio ou qual o propósito do software, perde-se a oportunidade
de causar mudanças positivas no planeta e na sociedade.

%Sustentabilidade técnica diz respeito a longevidade dos softwares, ou seja, a
%capacidade de continuar disponível no futuro.
%
%Essa definição de sustentabilidade de software é encontrada em mais detalhes no
%{\it Karlskrona Manifesto} \cite{becker2014karlskrona}, um documento que alerta
%sobre os impactos que os sistemas e a tecnologia da informação causam no futuro
%do planeta, convida praticantes e pesquisadores de software a refletir sobre
%o tema sustentabilidade na área da ciência da computação.

Sustentabilidade é um conceito guarda chuva composto de múltiplas dimensões, em
sua dimensão técnica, chamada sustentabilidade técnica, temos a preocupação com
a longevidade da informação, dos sistemas, e infraestrutura, e sua adequada
evolução frente as condições do ambiente em constante mudança. Software ocupa
um papel central nessa discussão, ele pode levar a crescentes consumo de
recurso, crescimento da desigualdade social, e influenciar no ganho ou perda de
auto-estima individual.

Science Code Manifesto \cite{barnes2013science}.
Foco em código fonte escrito especificamente para processar dados de
publicações, afirma que ``todo código fonte escrito especificamente para
processar dados de uma publicação deve estar disponível para os revisores e
leitores do paper''.

um caminho apontado como solução é acreditar que software deve evoluir para plataformas compartilhadas,
com componentes reusáveis tanto quanto possível, tanto para usuário final, quanto
para produtores de componentes (papel) agregando peças particulares de software,
crença de que o software deve evoluir em direção a uma plataforma
compartilhada, com componentes que são reutilizados o mais amplamente possível,
já que os usuários finais e os produtores de componentes se agrupam em torno de
peças específicas de software.
...
