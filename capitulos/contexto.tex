\xchapter{Contexto}{...}

%% The traditional “legs” (or “pillars”) of the scientific
%% method were theory and experimentation.
%% 
%% Computation has always been an
%% integral part of theory in science.
%% 
%% Computation has also always been
%% an integral part of experimentation.
%% 
%% So science is still carried out as an
%% ongoing interplay between theory and
%% experimentation.
%% 
%% \cite{Vardi_2010_SOT_1810891.1810892}

Software acadêmico ({\it research software}) é todo software usado para
coletar, produzir, processar ou analisar resultados de trabalhos publicados na
literatura acadêmica (seja em jornal, revista, conferência, monografia, livro,
dissertação ou tese), podem ser desde pequenos scripts ou protótipos até
produtos completos desenvolvidos profissionalmente, diferenciam-se entre
aqueles empregados como método de pesquisa para coleta e análise, daqueles
divulgados como resultado do estudo \citeonline{hettrick_2014_14809}.

%, como no desenvolvimento de novos algoritmos, métodos e técnicas, por exemplo 
%Softwares utilizados durante trabalhos de pesquisa mas que não se
%enquadram no descrito - como por exemplo editores de textos ou navegadores web
%- não são considerados softwares acadêmicos, segundo esta definição

Entre os estudos da ciência da computação, particularmente na engenharia de
software, nota-se um aumento constante no conjunto softwares acadêmicos
emergindo de pesquisas \cite{allen2017engineering}, outras áreas vivenciam
efeito semelhante, um estudo realizado entre 2014 e 2015 com dados de 1142
atrônomos de todos os níveis profissional mostrou que 100\% usam softwares em
suas pesquisas, a grande maioria, 90\%, escrevem parte do seu próprio software,
estes números independem da carreira profissional, área de pesquisa ou região
\cite{momcheva2015software}, outro estudo entre pesquisadores do reino unido
mostrou que 56\% escrevem parte dos seus softwares, independente da área de
conhecimento \cite{hettrick_2014_14809}.

Cientistas gastam hoje mais tempo com o uso e desenvolvimento do que no
passado, estudos mostram que gasta-se pelo menos 30\% do tempo desenvolvendo
softwares \cite{wilson2014best}, software é um aparato produzido e usado em
estudos experimentais assim como qualquer outro aparato físico, são tão
importantes para a ciência quanto são os telescópios ou tubos de ensaio.
Grupos de pesquisa trabalhando exclusivamente com problemas computacionais,
grupos em laboratórios tradicionais ou em campo, todos solucionam problemas
comuns por meio computacional, seja através de novos algoritmos ou atravpes de
softwares coleta e analisar grandes volumes de dados \cite{wilson2014best}.

Mas, apesar da importancia dos softwares acadêmicos, é comum não se dar o
devido reconhecimento, os softwares são frequentemente não citados, ou nem
mesmo considerados de ser citado, não raro, os desenvolvedores de softwares
acadêmicos (research software engineers) deixam de receber o devido crédito,
especialmente quando o software acadêmico carrega consigo uma grande
contribuição intelectual para a pesquisa.

Um estudo recente com 90 publicações da biologia mostou que apenas entre 31\% e
43\% das menções aos softwares acadêmicos envolvem citação formal; entre 15\% e
29\% dos softwares são inacessíveis, apenas entre 24\% e 40\% fornecem código
fonte, entre as 90 publicações do estudo apenas 59 mencionam o software de
alguma forma, as demais, apesar de usar algum software acadêmico, não mencionam
nada a respeito \cite{howison2016software}.

Enquanto muitos cientistas estão preocupados em validar seu
laboratório e equipamentos de campo, a maioria não sabe o quanto confiável é
seu software \cite{Vardi_2010_SOT_1810891.1810892} \cite{10.1109_32.328993}
\cite{Hatton1997}. Isto leva a sérios erros impactando conclusões centrais na
literatura publicada \cite{Merali2010Computational}. Recentes retrabalhos, comentários
técnicos, e correções causados por erros em métodos computacionais incluem
artigos publicados na Science [7,8], PNAS [9], no Journal of Molecular Biology
[10], Ecology Letters [11,12], o Journal of Mammalogy [13], Journal of the
American College of Cardiology [14], Hypertension [15], e The American Economic
Review [16].

Tal como acontece com outros experimentos, nem tudo deve ser feito
exatamente dentro dos padrões; entretanto, cientistas precisam estar
conscientes das boas práticas tanto para melhorar sua própria abordagem
tanto quanto par arevisar trabalhos dos outros \cite{wilson2014best}.

Entretando, a maior parte dos cientistas nunca tiveram algum treinamento
de como escrever software de forma eficiente e sustentável. Como resultado,
muitos não se preocupam quais ferramentas e práticas lhe permitem escrever código
confiável e manutenível com menos esforço.
\cite{wilson2014best}.

Como regra geral, pesquisadores não testam ou não documentam seus programas
rigorosamente, e eles raramente publicam seus códigos, fazendo quase impossível
reproduzir e verificar os resultados publicados gerados pelo software
acadêmico, dizem cientistas da computação \cite{Merali2010Computational}. Uma
dos argumentos comuns para não publicar o código associado com um artigo científico
para o mundo é que o código é ``ruim'' e o pesquisador será julgado negativamente baseado
na má qualidade do seu código \cite{allen2017engineering}.

Entretanto 90\% o mais são autodidatas e não tiveram
treinamento adequado para criar softwares de forma eficiente, falta
práticas básicas de desenvolvimento, como escrever código legível para
facilitar manutenção, uso de controle de versão, issue trackers, revisão de código,
testes unitários, e automação de tarefas \cite{wilson2014best}.

Computadores são agora essencial em todos os campos da ciência, mas a maior
parte dos pesquisadores nunca tiveram treinamento em práticas básicas de
laboratório com habilidades de pesquisa computação. Como resultado, dados são
perdidos, analises levam muito mais tempo que o necessário, e pesquisadores são
limitados em como efetivamente eles podem trabalhar com software e dados.
Fluxo de trabalho com computação necessita seguir as mesmas boas práticas que
projetos de laboratórios e notebooks, com dados organizados, passos
documentados, e o projeto estruturado para reprodutibilidade, mas pesquisadores
novos na computação frequentemente não sabem por onde começar. Este paper
apresenta um conjunto de boas práticas que todo pesquisador pode adotar,
independentemente do seu nível de habilidade em computação. Essas práticas
passam por gerenciamento de dados, programação, colaboração com colegas,
organização de projetos, tracking work, e escrita da manuscritos, sao
desenhados para uma grande variadade de fontes publicadas do noso dia a dia e
do nosso trabalho como voluntário organizando workshopts desde 2010
\cite{wilson2017good}.

A visibilidade do software no campo científico é uma questão, ligada
a conceitos que software sob a ciência é de qualidade questionável,
estes preocupação com qualidade não são apenas técnicas, mas se extendem para 
adequação do software para um largo compartilhamento, e sua abilidade de facilitar
o co-desenvolvimento que irá faer uso eficiente dos limitados recursos financeiros acadêmicos
(Howison \& Herbsleb, 2013; Katz et al., 2014).

O link tem dois pontos: primeiro, quando um software não é visível,
ele é frequentemente ecluido de peer review; segundo, essa falta de
visibilidade, ou uma forma particular de visibilidade, significa que
incentivos para produzir alta-qualidade, compartilhado amplamento, e
co-desenvolvido software pode estar em falta.

Um sistema em bom funcionamento devem atingir não apenas os objetivos
de entendimento e transparencia, mas também os objetivos voltados para replicação
(Stodden et al., 2010), complementando a disponibilidade das publicações como que
"um segundo pesquisador irá receber todos os benefícios do primeiro trabalho duro do pesquisador original"
(King, 1995, p. 445).

A situação com software é amplamente análoga (mas não identica) ao de dados
das publicações; de fato, todo dado é processado por softwares de alguma forma
(Borgman et al., 2012).

Não obstante, existe diferenças relevantes. Portanto, nosso inquiry into visibilidade
de software na comunicação acadêmica é complementar ao recente interesse
na citação de dados ...

\cite{howison2016software}

%%% Softwares acadêmicos são softwares desenvolvidos no decorrer de pesquisas
%%% científicas como parte de um estudo a ser publicado (seja num jornal, revista,
%%% conferência, monografia, livro ou tese), podem ser pequenos scripts contendo
%%% poucas linhas de código, protótipos, ou mesmo produtos de software completos
%%% que demonstram ou refletem os resultados de uma pesquisa, costumam ser
%%% utilizados para gerar, processar ou analisar resultados, mas podem ser para
%%% qualquer outro fim, basta que seja um dos artefatos gerado na pesquisa, será
%%% considerado software acadêmico.
%%% 
%%% Tais softwares podem ser encontrados na literatura pelo nome de {\it research
%%% tool} \cite{Portillo12}, {\it research-originated software} \cite{Kon2011},
%%% %{\it research software} \cite{hettrick_2014_14809} ou
%%% {\it academic software} \cite{allen2017engineering}, e costumam ser citados
%%% pelos seus autores como uma contribuição do estudo, seja principal ou
%%% secundária, alguns autores criam softwares acadêmicos como meio para atingir os
%%% resultados da pesquisa e não descrevem muito bem o software.
%%% 
%%% %, traduzido para software acadêmico para evitar o
%%% %termo {\it software de pesquisa}, a palavra {\it research} em português {\it
%%% %pesquisa} pode ser facilmente confundida com ferramentas ou sistemas de
%%% %pesquisa, como sites de pesquisa por exemplo,

Elevar o status do software não significa contratar um exército de programadores,
mas reconhecer as conquistas dos cientistas que criam ferramentas que usamos,
e remover barreiras para a produção, realização, contribuição, e compartilhamento de software astronomico.
Inevitavelmente alguns softwares irão continuar sendo úteis após o primeiro release,
alguns terão algums gerações de melhorias, outros serão usados na sua versão original sem atualização ou manutenção,
e alguns outros irão ser lançados e nunca utilizados. Isto é perfeitamente natural,
o ecosistema do software irá permitir que a comunidade decida qual o melhor uso e
colaboração em desenvolvimento atraés de um processo evolutivo, ao invés de agir como vigilantes
ou mantendo código proprietário para vantagens transitórias. O processo de testar, comparar,
provocar, desafiar, e prograsso incremental está no coração do processo científico 
\cite{weiner2009astronomical}.

Conhecimento novo na ciencia e engenharia depende de restados incrementais
produzidos por software acadêmico. Entretanto, saber como cientistas
desenvolvem e usam softwares em suas pesquisas é crítico para avaliar a
necessidade de melhorias no praticas atuais de desenvolvimento e para tomar
decisões sobre a futura alocação de recursos. Para este fim, uma pesquisa
conzidiza online entre outubro e dezembro de 2008 recebeu 2000 respostas.  A
maior conclusão foi que (1) o conhecimento necessário para desenvolver e usar
software acadêmico éadquirido em pareamentos e auto-estudo, mais do que da
educação formal e treinamento; (2) o número de cientistas usando
supercomputadores é pequeno comparado ao número usando desktops ou computadores
intermediários; (3) a maioria dos cientistas contam primariamente com softwares
com larga base de usuários; (4) enquanto muitos cientistas acreditam que testes
de software é importante, um pequeno número acredita que eles tem o suficiente
conhecmento sobre coneitos de testes; e (5) existe uma tendencia de cientistas
to rank standard software engineer- ing concepts higher if they work in large
software develop- ment projects and teams, but that there is no uniform trend
of association between rank of importance of software en- gineering concepts
and project/team size \cite{hannay2009scientists}.  Pesquisadores gastam mais
tempo hoje usando e desenvolvendo software do que no passado
\cite{hannay2009scientists}.

O Journal of Open Source Software (JOSS) é um livre a open-access jornal para
publicação de artigos descrevendo software acadêmico. Ele tem dois objetivos
principais, melhorar a qualidade dos softwares submetidos e prover mecanismos
para pesquisadores desenvolvedores de software acadêmico receber crédito pelos
seus softwares. Enquanto pensado para trabalhar dentro do atual sistema de
mérito da ciência, JOSS visa a escassez de recompensas para contribuições
importantes para a ciência realizadas em forma de software. JOSS publica
artigos que encapsulam sabedoria contida no software ele mesmo, e seu rigoroso
revisão em pares mirado nos componentes do software: funcionalidade,
documentação, testes, integração contínua, e a licença. Um artigo JOSS contém
um resumo descrevendo o objetivo e funcionalidades do software, referencias, e
um link para o software archive.  O artigo é um ponto de entrada para
submissçao que engloba o conjunto completo de artefatos de software. Artigos
aceitos no JOSS recebem um digital object identifier (DOI), te seus metadados
depositados no Crossref, e o artigo pode começar a colecionar citações e ser
indexados em serviços como Google Scholar e outros. No seu primeiro ano,
iniciado em Maio de 2016, JOSS publicou 111 artigos, com mais de 40 artigos
adicionais sob revisão \cite{smith2017journal}.

%%% 
%%% 
%%% Como não existe ainda amadurecimento suficiente sobre como citar softwares e
%%% outros artefatos em pesquisas científicas, não temos um padrão de como fazê-lo,
%%% cada autor cita à sua maneira, muitas vezes ao longo do texto, outras em seções
%%% específicas sobre a implementação do software, nem semprem informam onde
%%% encontrar uma cópia do software, ou ainda nem sobre o modelo em que o software
%%% é distribuído, ou se é de alguma forma distribuído ao público.
%%% 
%%% Os softwares acadêmicos neste estudo serão aquelas citados pelo autor como
%%% contribuição do estudo, então toda vez que citarmos software acadêmico estamos
%%% falando de artefatos de software citados pelo autor como contribuição do estudo.
%%% 
%%% 
%%% Os autores argumentam que a comunidade acadêmica de astronomia aumenta a
%%% dependencia de softwares e softwares precisam ser parte integral do prática
%%% científica \cite{momcheva2015software}.
