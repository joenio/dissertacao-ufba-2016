\xchapter{Contexto}{...}

Em diversas linhas de pesquisa da computação, em especial, a engenharia de
software, é muito comum que novos softwares sejam desenvolvidos durante
trabalhos de pesquisa, esses tem sido chamados na literatura de {\it
research-originated software} \cite{Kon2011}, {\it research tool}
\cite{Portillo12} ou {\it academic software} \cite{allen2017engineering}, e vêm
ganhando atenção da comunidade devido ao papel que ocupam na reprodutibilidade
de seus estudos \cite{Peng2011}.

A comunidade tem refletido sobre os problemas relacionados ao
desenvolvimento, promoção e sustentabilidade desses softwares, e o
impacto que tais problemas causam no meio científico \cite{allen2017engineering}. Esta
reflexão tem mostrado, por exemplo, que muitos estudos em engenharia de
software sofrem de dificuldades de repetição \cite{Tang2016}, e apontam
problemas específicos relacionados à manutenabilidade e a sustentabilidade
técnica dos softwares acadêmicos.

Manutenabilidade é uma característica de qualidade que indica o quão fácil é
realizar atividades de evolução e manutenção em softwares, um aspecto
importante aos pesquisadores interessados em adaptar softwares acadêmicos, algo
muitas vezes necessário ao reproduzir pesquisas anteriores \cite{Peng2011}.
Sustentabilidade técnica diz respeito a longevidade dos softwares, ou seja, a
capacidade de continuar disponível no futuro. Muitos pesquisadores não
disponibilizam os seus softwares \cite{robles2010replicating,
amann2015software} ou quanto o fazem enfrentam problemas com disponibilidade e
manutenabilidade \cite{Prlic2012}, isto leva a um corpo computacional
extramente difícil de reproduzir uma vez que mais da metade dos pesquisadores
desenvolvem seus próprios softwares \cite{hettrick_2014_14809}, além de ferir um dos
fundamentos da ciência de que novas descobertas sejam reproduzidas antes de
serem consideradas parte da base de conhecimento \cite{Stodden2009}.

Isto tem motivado a organização de conferências específicas sobre o tema, como
o RSE\footnote{Conference of Research Software Engineers
\url{http://rse.ac.uk/conf2017}}, WSSSPE\footnote{Workshop on Sustainable
Software for Science: Practice and Experiences
\url{http://wssspe.researchcomputing.org.uk}} e o RESER\footnote{Workshop on
Replication in Empirical Software Engineering Research
\url{http://sequoia.cs.byu.edu/reser}}, e tem contribuido para a compreensão
dos problemas relacionados aos softwares acadêmicos, abordando questões sobre
desenvolvimento, qualidade e sustentabiliade, sobre como citar softwares em
novas pesquisas, como promover e reconhecer o papel do pesquisador engenheiro
desenvolvedor de softwares acadêmicos, além de questões sobre infraestrutura,
ferramentas e práticas para o desenvolvimento de softwares acadêmicos de
forma sustentável.

Mas apesar desta crescente preocupação com os softwares acadêmicos ainda
sabe-se pouco sobre o quanto a sustentabilidade técnica e a manutenabilidade
impactam na reprodutibilidade de seus estudos, sobretudo em áreas específicas,
como a análise estática de software, uma área com uma longa e respeitável
tradição e que ainda sofre carência de estudos sobre avaliação e validação de
seus softwares \cite{Li2010, ilyas2016static}.

% (2) Tools to support systematic literature reviews in software engineering: A mapping study \cite{marshall2013tools}
%
% Cita um mapeamento feito sobre estudos que criam ferramentas para apoio a
% revisão sistemática no domínio de SE, 14 estudos foram selecionados, ao final
% apenas 8 tinham proposta de ferramentas, ao final conclui que as ferramentas
% encontradas estão em estado inicial de desenvolvimento. 
%
% (3) Tools used in Global Software Engineering: A systematic mapping review \cite{Portillo12}
%
% Cita um mapeamento sistemático com objetivo de encontrar ferramentas de
% comunicação e coordenação para suporte a times altamente distribuidos
% gograficamente, encontrou 132 ferramentas, para uso em projetos de software
% global. A maioria destas ferramentas foram desenvolvidas em centros de
% pesquisas, e apenas uma pequena porcentagem (18.9\%) foram testados fora do
% seu contexto onde foi desenvolvido.
%
% (5) Tools in mining software repositories \cite{chaturvedi2013tools}
%
% Faz uma revisão dos papers submetidos ao MSR desde 2007 até 2013 (?) e
% identifica data sets, ferramentas e técnicas utilizadas pelos autores, mais
% da metade dos papers usam ou criam ferramentas, categoriza as ferramentas em
% ferramentas novas, ferramentas tradicionais, protótipos e scripts para
% mineração de dados
%
% (6) A systematic literature review of software product line management tools \cite{pereira2015systematic}
%
% (???)
%
% (7) Software configuration management tools \cite{chan1997software}
%
% (???)
%
% (8) Comparison and evaluation of source code mining tools and techniques: A qualitative approach \cite{khatoon2013comparison}
%
% Lista ferramentas e técnicas para mineração de dados, estado da arte.
%
% (9) An overview of free software tools for general data mining \cite{jovic2014overview}
%
% Descreve característica dos 6 softwares livres mais usados para mineração de
% dados no geral.
%
% (10) Analyzing the State of Static Analysis: A Large-Scale Evaluation in Open Source Software \cite{beller2016analyzing}
%
% faz um estudo mostrando que analise estatica tem uma certa adocao em projetos livres
% e mostra onde pode-se melhorar nas ferramentas para aumentar a adoção
%
% Taming the Static Analysis Beast
% \cite{toman2017taming}
% Despite advances in tooling and mainstream success, static analysis development is still a
% painful process.

Dessa forma, definimos como objetivo geral deste trabalho avaliar o quanto a
sustentabilidade técnica e a manutenabilidade dos softwares acadêmicos de
análise estática impactam na reprodutibilidade de seus estudos.

Entre os objetivos da pesquisa, pretende-se:

\begin{enumerate}
  \item Encontrar e obter o código fonte dos softwares acadêmicos de análise
        estática.
  \item Medir e avaliar a sustentabilidade técnica e a manutenabilidade dos
        softwares acadêmicos de análise estática.
\end{enumerate}


%% The traditional “legs” (or “pillars”) of the scientific
%% method were theory and experimentation.
%% 
%% Computation has always been an
%% integral part of theory in science.
%% 
%% Computation has also always been
%% an integral part of experimentation.
%% 
%% So science is still carried out as an
%% ongoing interplay between theory and
%% experimentation.
%% 
%% \cite{Vardi_2010_SOT_1810891.1810892}

Software acadêmico\footnote{{\it research software}, {\it academic software},
{\it academic research software}} é todo software usado ou produzido para
coletar, processar ou analisar resultados em trabalhos publicados na literatura
acadêmica (seja em jornal, revista, conferência, monografia, livro, dissertação
ou tese), podem ser desde pequenos scripts ou protótipos até softwares
completos desenvolvidos profissionalmente.

Boa parte destes softwares são desenvolvidos na própria academia, um estudo
entre cientistas do reino unido mostrou, por exemplo, que 56\% desenvolvem seus próprios
softwares, pelo menos pacialmente \cite{hettrick_2014_14809}, em outras áreas,
como na astronomia, este número chega a 90\%
\cite{momcheva2015software}, em ciência da computação, particularmente em
engenharia de software, tem-se notado um aumento constante no número de novos
softwares acadêmicos \cite{allen2017engineering}.

Cientistas gastam mais tempo hoje utilizando e desenvolvendo
software do que gastavam no passado, softwares acadêmicos, assim como
qualquer outro aparato experimental, são tão importantes quanto
são os telescópios e tubos de ensaio, eles resolvem problemas comuns do
cotidiano de metade dos pesquisadores de todas as áreas do conhecimento, desde grupos
trabalhando exclusivamente com problemas computacionais até grupos em
laboratórios tradicionais ou em campo \cite{wilson2014best}.

% , pelo menos 30\% do tempo é gasto nesta atividade

Apesar disso, ainda há pouco reconhecimento aos softwares acadêmicos, muitas
pesquisas nem ao menos mencionam sua utilização, um estudo recente com 90
artigos de diversas áreas da biologia, selecionados aleatoriamente entre
publicações usando softwares acadêmicos, mostrou que apenas 59 mencionavam os
softwares de alguma forma, os demais 31 artigos, apesar de usar algum software
acadêmico, não mencionavam nada a respeito \cite{howison2016software}.

%, até mesmo aos pesquisadores\footnote{\it research
%software engineers} responsáveis pelo seu desenvolvimento,
%
% que apenas entre 31\% e
% 43\% das menções aos softwares acadêmicos envolvem citação formal; entre 15\% e
% 29\% dos softwares são inacessíveis, apenas entre 24\% e 40\% fornecem código
% fonte, entre as 90 publicações do estudo apenas 59 mencionam o software de
% alguma forma, as demais, apesar de usar algum software acadêmico, não mencionam
% nada a respeito \cite{howison2016software}.

Isto gera impacto negativo na visibilidade dos softwares acadêmicos, e faz
surgir questionamentos sobre sua qualidade, não apenas técnica, mas também a
capacidade de ser encontrado, compartilhado e co-desenvolvido, qualidades
importantes para a evolução do próprio software, mas também extremamente úteis
para o uso eficiente dos limitados recursos da ciência \cite{howison2013,
katz2014transitive}.

% historia da citacao na ciencia, como isso promove o avanço,
% problemas para citacao em artefatos digitais, solucao
% para identificador unico de autores de artigos, orcid.org
% resolve este problema, o mesmo para identificar artefatos
% digitais é o doi.org \cite{allen2014credit}
%
%Quando um software não é visível, ele é frequentemente excluido de {\it peer
%reviews}; um software em bom funcionamento devem atingir não apenas os
%objetivos de entendimento e transparencia, mas também os objetivos voltados
%para replicação \cite{Stodden2010}.
%
%Essa falta de visibilidade, ou uma forma
%particular de visibilidade, significa que incentivos para produzir
%alta-qualidade, compartilhado amplamento, e co-desenvolvido software pode estar
%em falta \cite{howison2016software}.
%
%Muitos cientistas estão preocupados em validar seus aparatos de
%laboratório ou equipamentos de campo, 

No entanto, parece ser regra geral não testar ou não documentar o próprio software,
pesquisadores geralmente não testam ou documentam seus softwares acadêmicos, a
maioria também não sabe o quão confiável seu software é, ocasionando sérios
erros computacionais em conclusões centrais da literatura acadêmica, gerando
retrabalho para retratar tais erros nas mais diversas áreas da ciência
\cite{Merali2010Computational}. Sabe-se que nem sempre é possível, ou viável,
ter tudo exatamente dentro dos padrões, mas é preciso estar consciente das boas
práticas ao produzir e utilizar softwares acadêmicos, tanto para melhorar a
própria abordagem quanto para revisar trabalhos de outros
\cite{wilson2014best}.

A maior parte dos cientistas (90\%) no entretanto nunca tiveram treinamento
algum de como escrever software de forma eficiente, faltam práticas básicas de
desenvolvimento, como escrever código legível, revisão de código, controle de
versão, testes unitários, entre outros, como resultado, dados são perdidos,
análises levam mais tempo que o necessário e os pesquisadores não conseguem a
eficiência que poderiam ter ao trabalhar com softwares acadêmicos
\cite{wilson2017good}.

%, de como escrever
%código legível, uso de controle de versão, issue trackers, revisão de código,
%testes unitários e automação de tarefas.

%O fluxo de trabalhos científicos usando computação precisam seguir as mesmas boas práticas qualquer projeto experimental, 
%
%e então melhorar os seus dados ou metodologia de alguma maneira.

Isto contradiz as boas práticas de qualquer projeto experimental ({\it
laboratory
notebooks}\footnote{\url{https://en.wikipedia.org/wiki/Lab_notebook}}, dados
organizados, passos documentados, projeto estruturado para reprodutibilidade) e
torna praticamente impossível utilizar o método mais comum e cientificamente
produtivo de produzir conhecimento novo a partir de pesquisas anteriores, a
replicação, ou seja, seguir os mesmos passos do autor original com
objetivo de validar, melhorar ou estender seus dados e sua metodologia
\cite{king1995replication, Stodden2010}.

%estas práticas permitem replicar descobertas anteriores seguindo
%o caminho do autor original,
%isto, segundo
%\citeonline{king1995replication}, é o método mais comum e cientificamente
%produtivo de produzir conhecimento novo para a ciência, tanto a ciência quanto
%a engenharia dependem de resultados incrementais para evoluir, no entando,
%
%Isto se torna ainda mais difícil visto que frequentemente os pesquisadores
%deixam de publicar o código dos seus softwares acadêmicos argumentando que o
%código é ``ruim'' e isto irá gerar julgamentos negativos ao pesquisador
%\cite{allen2017engineering}.
%
%o comum usado para não publicar o
%código associado com um artigo científico é que 
%
%isto, além de fazer os próprios pesquisadores enfrentar problemas ao replicar
%seus próprios trabalhos no futuro, torna quase impossível reproduzir e
%verificar os resultados de pesquisas anteriores, ferindo um dos princípios da
%ciência de que novas descobertas sejam reproduzidas antes de serem consideradas
%parte da base de conhecimento
%
% Few
% events in academic life are more
% frustrating than investing enormous
% amounts of time, effort, and pride
% in an article or book, only to have
% it ignored by the profession, not
% followed up by other researchers,
% not used to build upon for succeed-
% ing research, or not explored in
% other contexts. Moreover, being
% ignored is very damaging to a ca-
% reer, but being applauded, cited
% favorably, criticized, or even at-
% tacked are all equally strong evi-
% dence that you are being taken seri-
% ously for your contributions to the
% scholarly debate (see Feigenbaum
% and Levy 1993, citing Diamond
% 1988, and Leimer and Lesnoy
% 1982). Unfortunately, a recent
% study indicates that the modal num-
% ber of citations to articles in politi-
% cal science is zero: 90.1% of our
% articles are never cited (Hamilton
% 1991; Pendlebury 1994)! An even
% smaller fraction of articles stimu-
% lates active investigation by other
% researchers.
% This problem

%The pursuit of science and engineering research increas-
%ingly relies on activities that facilitate research but are not
%currently rewarded or recognized. This includes the sharing
%of data; development of common data resources, software
%and methodologies; and annotation of data and publica-
%tions.
%About half of the articles in many recent issues
%of Science describe research that depended on software,
%and a larger fraction analyze data.
%
%\cite{wilson2017good}
%\cite{Vardi_2010_SOT_1810891.1810892, 10.1109_32.328993, Hatton1997},
%
% Este paper
% apresenta um conjunto de boas práticas que todo pesquisador pode adotar,
% independentemente do seu nível de habilidade em computação. Essas práticas
% passam por gerenciamento de dados, programação, colaboração com colegas,
% organização de projetos, tracking work, e escrita da manuscritos, sao
% desenhados para uma grande variadade de fontes publicadas do noso dia a dia e
% do nosso trabalho como voluntário organizando workshopts desde 2010
% \cite{wilson2017good}.
%
% A situação com software é amplamente análoga (mas não identica) ao de dados
% das publicações; de fato, todo dado é processado por softwares de alguma forma
% (Borgman et al., 2012).
%
% Tais softwares podem ser encontrados na literatura pelo nome de {\it research
% tool} \cite{Portillo12}, {\it research-originated software} \cite{Kon2011},
% %{\it research software} \cite{hettrick_2014_14809} ou
% {\it academic software} \cite{allen2017engineering}, e costumam ser citados
% pelos seus autores como uma contribuição do estudo, seja principal ou
% secundária, alguns autores criam softwares acadêmicos como meio para atingir os
% resultados da pesquisa e não descrevem muito bem o software.
% 
% %, traduzido para software acadêmico para evitar o
% %termo {\it software de pesquisa}, a palavra {\it research} em português {\it
% %pesquisa} pode ser facilmente confundida com ferramentas ou sistemas de
% %pesquisa, como sites de pesquisa por exemplo,

Somado a isto temos ainda o fato de que pesquisadores raramente publicam seus
códigos, piorando ainda mais toda a situação, isto tem motivado a organização
de conferências específicas para discutir os problemas dos softwares
acadêmicos, como o RSE (Conference of Research Software Engineers)\footnote{
\url{http://rse.ac.uk/conf2017}}, WSSSPE (Workshop on Sustainable Software for
Science: Practice and Experiences)\footnote{
\url{http://wssspe.researchcomputing.org.uk}} e o RESER (Workshop on
Replication in Empirical Software Engineering Research)\footnote{
\url{http://sequoia.cs.byu.edu/reser}}, e tem agregado discussões das
comunidades de ciência aberta, reprodutibilidade e sustentabilidade de
software.

Apesar disso, poucos estudos tem focado sua atenção aos softwares acadêmicos de
origem científica\footnote{\it research-originated software \cite{Kon2011}},
especialmente os da engenharia de software, uma área que, ao menos em teoria,
não sofre com a falta de treinamento prévio em práticas de desenvolvimento.

Dessa forma, definimos como objetivo geral deste trabalho explorar como os
softwares acadêmicos de origem científica publicados na literatura acadêmica da
engenharia de software estão visíveis hoje e como são mencionados ao longo do
tempo.

%Entre os objetivos da pesquisa, pretende-se:
%
%\begin{enumerate}
%  \item Encontrar e caracterizar softwares acadêmicos de origem científica
%        publicados na engenharia de software.
%  \item Mensurar e avaliar a sustentabilidade dos softwares acadêmicos de
%        origem científica publicados na engenharia de software.
%\end{enumerate}
%
%Saber como cientistas desenvolvem e usam softwares em suas pesquisas é crítico
%para avaliar a necessidade de melhorias no praticas atuais de desenvolvimento e
%para tomar decisões sobre a futura alocação de recursos
%\cite{hannay2009scientists}.
%
%Inevitavelmente alguns softwares irão continuar sendo úteis após o primeiro
%release, alguns terão algums gerações de melhorias, outros serão usados na sua
%versão original sem atualização ou manutenção, e alguns outros irão ser
%lançados e nunca utilizados. Isto é perfeitamente natural, a comunidade ao
%redor do software irá decidir qual é o melhor caminho a se tomar num processo
%evolutivo \cite{weiner2009astronomical}.
%
%Diversas maneiras de inventivar citação formal entre artefatos digitais,
%software por exemplo, tem surgido, dentre elas uma iniciativa interessante
%é o Journal of Open Source Software (JOSS) é um livre a open-access jornal para
%publicação de artigos descrevendo software acadêmico. Ele tem dois objetivos
%principais, melhorar a qualidade dos softwares submetidos e prover mecanismos
%para pesquisadores desenvolvedores de software acadêmico receber crédito pelos
%seus softwares. Enquanto pensado para trabalhar dentro do atual sistema de
%mérito da ciência, JOSS visa a escassez de recompensas para contribuições
%importantes para a ciência realizadas em forma de software. JOSS publica
%artigos que encapsulam sabedoria contida no software ele mesmo, e seu rigoroso
%revisão em pares mirado nos componentes do software: funcionalidade,
%documentação, testes, integração contínua, e a licença. Um artigo JOSS contém
%um resumo descrevendo o objetivo e funcionalidades do software, referencias, e
%um link para o software archive.  O artigo é um ponto de entrada para
%submissçao que engloba o conjunto completo de artefatos de software. Artigos
%aceitos no JOSS recebem um digital object identifier (DOI), te seus metadados
%depositados no Crossref, e o artigo pode começar a colecionar citações e ser
%indexados em serviços como Google Scholar e outros. No seu primeiro ano,
%iniciado em Maio de 2016, JOSS publicou 111 artigos, com mais de 40 artigos
%adicionais sob revisão \cite{smith2017journal}.
%
%%% 
%%% 
%%% Como não existe ainda amadurecimento suficiente sobre como citar softwares e
%%% outros artefatos em pesquisas científicas, não temos um padrão de como fazê-lo,
%%% cada autor cita à sua maneira, muitas vezes ao longo do texto, outras em seções
%%% específicas sobre a implementação do software, nem semprem informam onde
%%% encontrar uma cópia do software, ou ainda nem sobre o modelo em que o software
%%% é distribuído, ou se é de alguma forma distribuído ao público.
%%% 
%%% Os softwares acadêmicos neste estudo serão aquelas citados pelo autor como
%%% contribuição do estudo, então toda vez que citarmos software acadêmico estamos
%%% falando de artefatos de software citados pelo autor como contribuição do estudo.
%%% 
%%% Os autores argumentam que a comunidade acadêmica de astronomia aumenta a
%%% dependencia de softwares e softwares precisam ser parte integral do prática
%%% científica \cite{momcheva2015software}.
