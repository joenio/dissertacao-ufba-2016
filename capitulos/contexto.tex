\xchapter{Contexto}{...}

%% The traditional “legs” (or “pillars”) of the scientific
%% method were theory and experimentation.
%% 
%% Computation has always been an
%% integral part of theory in science.
%% 
%% Computation has also always been
%% an integral part of experimentation.
%% 
%% So science is still carried out as an
%% ongoing interplay between theory and
%% experimentation.
%% 
%% \cite{Vardi_2010_SOT_1810891.1810892}

Software acadêmico\footnote{{\it research software}, {\it academic software},
{\it academic research software}} é todo software usado ou produzido para
coletar, processar ou analisar resultados de trabalhos publicados em literatura
acadêmica (seja em jornal, revista, conferência, monografia, livro, dissertação
ou tese), podem ser desde pequenos scripts ou protótipos até softwares
completos desenvolvidos profissionalmente.

Boa parte destes softwares são desenvolvidos na própria academia, entre
cientistas do reino unido, por exemplo, 56\% desenvolvem seu próprio software,
pelo menos pacialmente \cite{hettrick_2014_14809}, em áreas específicas, como
na astronomia, este número chega a 90\% \cite{momcheva2015software}, em ciência
da computação, particularmente em engenharia de software, tem-se notado um
aumento constante no número de softwares acadêmicos desenvolvidos pelos
próprios pesquisadores \cite{allen2017engineering}.

%,
%ou, softwares acadêmicos de origem científica\footnote{\it research-originated software \cite{Kon2011}},
%softwares acadêmicos científico.
%
%, ou seja, 
%originados na academia
%
% independente da área de conhecimento

Cientistas de todas as áreas gastam hoje mais tempo com o uso e desenvolvimento
de software do que gastavam no passado, softwares
acadêmicos, assim como qualquer outro aparato experimental, são tão importantes
para a ciência quanto os telescópios ou tubos de ensaio, eles resolvem problemas
comuns do cotidiano de pesquisadores trabalhando exclusivamente com problemas computacionais até grupos em
laboratórios tradicionais ou em campo \cite{wilson2014best}.

% , pelo menos 30\% do tempo é gasto nesta atividade

Apesar da evidente importância dos softwares acadêmicos há ainda pouco
reconhecimento aos mesmos, um estudo recente com publicações de diversas áreas
da biologia mostrou que em um conjunto de 90 artigos apenas 59 mencionavam
softwares acadêmicos de alguma forma, os demais 31 artigos, apesar de usarem
software acadêmico, não mencionam nada a respeito \cite{howison2016software}.

%, até mesmo aos pesquisadores\footnote{\it research
%software engineers} responsáveis pelo seu desenvolvimento,

% que apenas entre 31\% e
% 43\% das menções aos softwares acadêmicos envolvem citação formal; entre 15\% e
% 29\% dos softwares são inacessíveis, apenas entre 24\% e 40\% fornecem código
% fonte, entre as 90 publicações do estudo apenas 59 mencionam o software de
% alguma forma, as demais, apesar de usar algum software acadêmico, não mencionam
% nada a respeito \cite{howison2016software}.

Esta falta de reconhecimento gera um impacto negativo na visibilidade dos
softwares acadêmicos no meio científico, e faz surgir questionamentos sobre a
sua qualidade, não apenas qualidade técnica, mas também a capacidade de ser
encontrado, compartilhado e co-desenvolvido, qualidades que contribuem para o
uso eficiente dos limitados recursos da ciência \cite{howison2013,
katz2014transitive}.

% historia da citacao na ciencia, como isso promove o avanço,
% problemas para citacao em artefatos digitais, solucao
% para identificador unico de autores de artigos, orcid.org
% resolve este problema, o mesmo para identificar artefatos
% digitais é o doi.org \cite{allen2014credit}
%
%Quando um software não é visível, ele é frequentemente excluido de {\it peer
%reviews}; um software em bom funcionamento devem atingir não apenas os
%objetivos de entendimento e transparencia, mas também os objetivos voltados
%para replicação \cite{Stodden2010}. Essa falta de visibilidade, ou uma forma
%particular de visibilidade, significa que incentivos para produzir
%alta-qualidade, compartilhado amplamento, e co-desenvolvido software pode estar
%em falta \cite{howison2016software}.
%
%Muitos cientistas estão preocupados em validar seus aparatos de
%laboratório ou equipamentos de campo, 

É regra geral pesquisadores não testar e não documentar seus softwares, e
raramente publicam seus códigos, tornando quase impossível reproduzir e
verificar os resultados publicados, a maioria não sabe o quão confiável seu
software é, levando muitas vezes a sérios erros computacionais em conclusões
centrais na literatura acadêmica, causando retrabalhos em métodos
computacionais nas mais diversas áreas da ciência
\cite{Merali2010Computational}.

É certo que tal como acontece com outros
experimentos, nem tudo deve ser feito exatamente dentro dos padrões;
entretanto, cientistas precisam estar conscientes das boas práticas tanto para
melhorar sua própria abordagem tanto quanto par arevisar trabalhos dos outros
\cite{wilson2014best}.

%\cite{Vardi_2010_SOT_1810891.1810892, 10.1109_32.328993, Hatton1997},

um argumento comum usado para não publicar o
código associado com um artigo científico é que o código é ``ruim'' e o
pesquisador será julgado negativamente baseado na má qualidade do seu código
\cite{allen2017engineering}.

%The pursuit of science and engineering research increas-
%ingly relies on activities that facilitate research but are not
%currently rewarded or recognized. This includes the sharing
%of data; development of common data resources, software
%and methodologies; and annotation of data and publica-
%tions.
%About half of the articles in many recent issues
%of Science describe research that depended on software,
%and a larger fraction analyze data.
%digitais é o doi.org \cite{allen2014credit}

Este comportamento não é difícil de compreender quando sabe-se que a maior
parte dos cientistas nunca tiveram algum treinamento de como escrever software
de forma eficiente e sustentável, 90\% ou mais são autodidatas e nunca tiveram
treinamento adequado, faltam práticas básicas de desenvolvimento, como escrever
código legível, uso de controle de versão, issue trackers, revisão de código,
testes unitários, e automação de tarefas, como resultado, muitos não se
preocupam quais ferramentas e práticas lhe permitem escrever código confiável e
manutenível com menos esforço \cite{wilson2014best}.

Como resultado, dados são perdidos, analises levam muito mais tempo que o
necessário, e pesquisadores são limitados em como efetivamente eles podem
trabalhar com software e dados.  Fluxo de trabalho com computação necessita
seguir as mesmas boas práticas de projetos com seus {\it laboratory notebook},
dados organizados, passos documentados, e o projeto estruturado para
reprodutibilidade, mas pesquisadores novos na computação frequentemente não
sabem por onde começar \cite{wilson2017good}.

% Este paper
% apresenta um conjunto de boas práticas que todo pesquisador pode adotar,
% independentemente do seu nível de habilidade em computação. Essas práticas
% passam por gerenciamento de dados, programação, colaboração com colegas,
% organização de projetos, tracking work, e escrita da manuscritos, sao
% desenhados para uma grande variadade de fontes publicadas do noso dia a dia e
% do nosso trabalho como voluntário organizando workshopts desde 2010
% \cite{wilson2017good}.

% complementando a disponibilidade das publicações como que
% "um segundo pesquisador irá receber todos os benefícios do primeiro trabalho duro do pesquisador original"
% (King, 1995, p. 445).
% A situação com software é amplamente análoga (mas não identica) ao de dados
% das publicações; de fato, todo dado é processado por softwares de alguma forma
% (Borgman et al., 2012).


%%% Softwares acadêmicos são softwares desenvolvidos no decorrer de pesquisas
%%% científicas como parte de um estudo a ser publicado (seja num jornal, revista,
%%% conferência, monografia, livro ou tese), podem ser pequenos scripts contendo
%%% poucas linhas de código, protótipos, ou mesmo produtos de software completos
%%% que demonstram ou refletem os resultados de uma pesquisa, costumam ser
%%% utilizados para gerar, processar ou analisar resultados, mas podem ser para
%%% qualquer outro fim, basta que seja um dos artefatos gerado na pesquisa, será
%%% considerado software acadêmico.
%%% 
%%% Tais softwares podem ser encontrados na literatura pelo nome de {\it research
%%% tool} \cite{Portillo12}, {\it research-originated software} \cite{Kon2011},
%%% %{\it research software} \cite{hettrick_2014_14809} ou
%%% {\it academic software} \cite{allen2017engineering}, e costumam ser citados
%%% pelos seus autores como uma contribuição do estudo, seja principal ou
%%% secundária, alguns autores criam softwares acadêmicos como meio para atingir os
%%% resultados da pesquisa e não descrevem muito bem o software.
%%% 
%%% %, traduzido para software acadêmico para evitar o
%%% %termo {\it software de pesquisa}, a palavra {\it research} em português {\it
%%% %pesquisa} pode ser facilmente confundida com ferramentas ou sistemas de
%%% %pesquisa, como sites de pesquisa por exemplo,

Elevar o status do software não significa contratar um exército de programadores,
mas reconhecer as conquistas dos cientistas que criam ferramentas que usamos,
e remover barreiras para a produção, realização, contribuição, e compartilhamento de software astronomico.
Inevitavelmente alguns softwares irão continuar sendo úteis após o primeiro release,
alguns terão algums gerações de melhorias, outros serão usados na sua versão original sem atualização ou manutenção,
e alguns outros irão ser lançados e nunca utilizados. Isto é perfeitamente natural,
o ecosistema do software irá permitir que a comunidade decida qual o melhor uso e
colaboração em desenvolvimento atraés de um processo evolutivo, ao invés de agir como vigilantes
ou mantendo código proprietário para vantagens transitórias. O processo de testar, comparar,
provocar, desafiar, e prograsso incremental está no coração do processo científico 
\cite{weiner2009astronomical}.

Conhecimento novo na ciencia e engenharia depende de restados incrementais
produzidos por software acadêmico. Entretanto, saber como cientistas
desenvolvem e usam softwares em suas pesquisas é crítico para avaliar a
necessidade de melhorias no praticas atuais de desenvolvimento e para tomar
decisões sobre a futura alocação de recursos \cite{hannay2009scientists}.

% Pesquisadores gastam mais
% tempo hoje usando e desenvolvendo software do que no passado
% \cite{hannay2009scientists}.

Diversas maneiras de inventivar citação formal entre artefatos digitais,
software por exemplo, tem surgido, dentre elas uma iniciativa interessante
é o Journal of Open Source Software (JOSS) é um livre a open-access jornal para
publicação de artigos descrevendo software acadêmico. Ele tem dois objetivos
principais, melhorar a qualidade dos softwares submetidos e prover mecanismos
para pesquisadores desenvolvedores de software acadêmico receber crédito pelos
seus softwares. Enquanto pensado para trabalhar dentro do atual sistema de
mérito da ciência, JOSS visa a escassez de recompensas para contribuições
importantes para a ciência realizadas em forma de software. JOSS publica
artigos que encapsulam sabedoria contida no software ele mesmo, e seu rigoroso
revisão em pares mirado nos componentes do software: funcionalidade,
documentação, testes, integração contínua, e a licença. Um artigo JOSS contém
um resumo descrevendo o objetivo e funcionalidades do software, referencias, e
um link para o software archive.  O artigo é um ponto de entrada para
submissçao que engloba o conjunto completo de artefatos de software. Artigos
aceitos no JOSS recebem um digital object identifier (DOI), te seus metadados
depositados no Crossref, e o artigo pode começar a colecionar citações e ser
indexados em serviços como Google Scholar e outros. No seu primeiro ano,
iniciado em Maio de 2016, JOSS publicou 111 artigos, com mais de 40 artigos
adicionais sob revisão \cite{smith2017journal}.

%%% 
%%% 
%%% Como não existe ainda amadurecimento suficiente sobre como citar softwares e
%%% outros artefatos em pesquisas científicas, não temos um padrão de como fazê-lo,
%%% cada autor cita à sua maneira, muitas vezes ao longo do texto, outras em seções
%%% específicas sobre a implementação do software, nem semprem informam onde
%%% encontrar uma cópia do software, ou ainda nem sobre o modelo em que o software
%%% é distribuído, ou se é de alguma forma distribuído ao público.
%%% 
%%% Os softwares acadêmicos neste estudo serão aquelas citados pelo autor como
%%% contribuição do estudo, então toda vez que citarmos software acadêmico estamos
%%% falando de artefatos de software citados pelo autor como contribuição do estudo.
%%% 
%%% 
%%% Os autores argumentam que a comunidade acadêmica de astronomia aumenta a
%%% dependencia de softwares e softwares precisam ser parte integral do prática
%%% científica \cite{momcheva2015software}.
