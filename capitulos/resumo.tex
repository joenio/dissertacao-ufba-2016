\textbf{Contexto.} 
Software acadêmico é todo projeto de software utilizado para apoiar pesquisas
científicas, o uso crescente destes projetos tem feito a Ciência moderna
depender do software. O desenvolvimento sustentável destes projetos,
realizado na maior parte pelos próprios cientistas, contribui para o
crescimento do ecossistema de software acadêmico, resultando em mais projetos, com melhor
qualidade, com menor custo.
%
Entre as inúmeras dimensões da sustentabilidade, destacamos o seu aspecto
técnico, ou seja, a capacidade de continuar disponível, em novas plataformas,
atendendo continuamente às novas necessidades do ambiente através de uma
adequada evolução frente as condições em constante mudança.
%\cite{venters2014software, allen2017engineering}.
\textbf{Problema.} 
O desenvolvimento não sustentável de software acadêmico prejudica a capacidade
de reprodução de seus estudos e fere um dos fundamentos da Ciência de que novas
descobertas sejam reproduzidas antes de serem consideradas parte da base de
conhecimento.
%\cite{stodden2009enabling}.
Além disso, estes projetos sofrem de desordem caótica disfuncional ({\it ``dysfunctional
chaotic churn''}), ou seja, a existência de muitos projetos, com poucos
usuários, com ciclos de vida curtos, que terminam em paralelo ao financiamento
inicial, comunidades desconectadas e paralelas, incompatibilidades entre
projetos, e tentativas aparentemente não coordenadas de ``reiniciar'' tudo
({\it re-boots}).
%\cite{howison2015understanding}.
No entanto, não há evidências sobre estes problemas entre projetos de software
acadêmico desenvolvidos em pesquisas da Engenharia de Software, especialmente
em pesquisas de análise estática, uma área com uma longa tradição no
desenvolvimento de novos projetos.
\textbf{Objetivos.}
O objetivo geral desta pesquisa foi caracterizar projetos de software acadêmico
do domínio de análise estática em relação a sustentabilidade técnica, em termos
de suas características de publicização/disponibilidade, visibilidade
científica e manutenibilidade.
% Falar algo sobre disfunctional chaotic churn?
Com especial atenção ao problema de desordem caótica disfuncional
e o quanto a caracterização consegue explicar a este respeito.
\textbf{Métodos.}
Foi realizado um estudo exploratório sobre a sustentabilidade
técnica dos projetos de software acadêmico de análise estática desenvolvidos e
publicados nas conferências de Engenharia de Software ASE e SCAM até o ano de
2015.
%
Os projetos foram selecionados através de uma revisão de literatura, uma busca
nas bases da ACM e IEEE caracterizou estes projetos em relação ao número de
menções e contribuições em código fonte, uma pesquisa documental no site do projeto,
código fonte, manuais e repositórios trouxe informaçoes a respeito da
distribuição, licença e ciclo de vida de cada projeto.
%
%Por fim coletamos de cada projeto com código fonte disponível métricas de
%complexidade estrutural para caracterização da manutenabilidade dos projetos.
\textbf{Resultados.}
A revisão de literatura resultou na seleção de 60
projetos de software acadêmico. % entre 1873 artigos.
%
A caracterização destes projetos mostrou que 40\% não está disponível
publicamente, ou seja, não é possível obter o software na URL informada pelos
autores,
%. Isto dificulta a reprodutibilidade dos estudos que tenham feito
%uso destes projetos.
%
23\% não possui nenhuma menção nas bases
ACM e IEEE além da publicação inicial do projeto,
26\% dos projetos receberam contribuição em código fonte em outros estudos
além da publicação inicial.
%através de pesquisas além da publicação inicial do projeto de software.
%, 125 artigos distintos mencionam os projetos em
%estudos fazendo uso deles.
%
%16 projetos foram encontrados em mençoes em estudos contribuindo com tais
%projetos em nível de código fonte, apenas 
%
%Listar os resultados concretos, trazidos pelos dados analisados e interpretados.
%Apesar de não podermos generalizar, ...
% Conseguiu responder diretamente com seu trabalho? Se não conseguiu, não colocar em resultados. 
%... os atores deste ecossistema têm aproveitado oportunidades de colaboração para 
% - reduzir o retrabalho ... consegue responder isso?
% - aumentar a qualidade geral dos projetos? ... consegue responder?
% - proporcionar maior avanço das pesquisas na área de Engenharia de Software, especialmente dos estudos sobre análise estática.

% vim: filetype=tex
