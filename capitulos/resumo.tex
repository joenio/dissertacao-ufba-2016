%%%%\textbf{Contexto.}
O uso crescente de software acadêmico, isto é, 
software desenvolvido para apoiar pesquisas científicas,
tem reforçado os laços entre 
reprodutibilidade, um dos princípios fundamentais da ciência moderna,
e a sustentabilidade técnica do software,
aqui definida como a capacidade de se manter disponível e confiável,
atendendo continuamente às novas necessidades de um ambiente em constante mudança.
%
%%%%\textbf{Problema.} 
O desenvolvimento não sustentável de um software acadêmico pode 
afetar negativamente a reprodução de estudos científicos que utilizaram o software,
conduzida por pesquisadores independentes. 
Além disso, o desenvolvimento não sustentável de software acadêmico,
pode levar a um quadro de desordem caótica disfuncional (DCD) em um domínio,
caracterizado pela existência de muitos projetos similares, com poucos
usuários e ciclos de vida curtos, e que terminam em paralelo ao financiamento
inicial,  comunidades desconectadas e paralelas, incompatibilidade entre
projetos, e tentativas não coordenadas de ``reiniciar'' tudo
({\it re-boots}).
%
No entanto, não há evidências fundamentadas sobre a sustentabilidade técnica
e a DCD em software acadêmico da área de Engenharia de Software, especialmente
no domínio de análise estática, com uma longa tradição no
desenvolvimento de ferramentas para apoiar suas pesquisas.

%%%%\textbf{Objetivos.}
O objetivo geral deste trabalho foi 
caracterizar software acadêmico do domínio de análise estática 
em relação a sustentabilidade técnica, 
em termos de disponibilidade, visibilidade científica e manutenibilidade.
Tal caracterização pode ser útil para compreender como a DCD se manifesta
no domínio estudado.
%
%%%%\textbf{Métodos.}
Foi realizado um estudo exploratório sobre a sustentabilidade
técnica de software acadêmico de análise estática 
publicados em duas conferências importantes para a área: ASE e SCAM.
%
Com base nos projetos publicados,  %na ASE e SCAM, 
uma revisão da literatura foi realizada nas bases da ACM e IEEE, 
para a caracterização de cada software acadêmico 
em termos de tipos e número de menções feitas por outros artigos científicos,
e contribuições em seu código fonte.
Uma pesquisa documental realizada com base em 
código fonte, manuais e repositórios, trouxe informações a respeito da
distribuição, licença e ciclo de vida de cada software.
Para os projetos com código fonte disponível, métricas de
complexidade estrutural foram usadas para caracterização da manutenibilidade.
%
%%%%\textbf{Resultados.}
Foram encontrados 60 projetos de software acadêmico de análise estática
publicados em artigos da ASE e SCAM.
%
A caracterização desses projetos mostrou que: 
40\% não está disponível publicamente, ou seja, 
não é possível obter o software na URL informada pelos autores, 
inviabilizando a reprodução de estudos que utilizaram o software;
%
23\% não possui nenhuma menção nas bases ACM e IEEE, além de sua publicação inicial; e
26\% recebeu contribuição em código fonte de outros estudos.
No domínio de análise estática, observou-se a existência de muitos projetos similares, com poucos usuários, e ciclos de vida curtos.
%
