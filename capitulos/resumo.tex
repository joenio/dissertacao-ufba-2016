O uso crescente de software acadêmico, isto é, software desenvolvido para
apoiar pesquisas científicas nas mais diversas áreas do conhecimento, tem feito
a Ciência moderna depender da sustentabilidade técnica do software, aqui
definida como a capacidade de se manter disponível e confiável, atendendo
continuamente às novas necessidades de um ambiente em constante mudança.
%
O desenvolvimento não sustentável de software acadêmico pode ferir um dos
fundamentos da Ciência: a reprodutibilidade, ou a capacidade de reprodução de
estudos científicos por pesquisadores independentes.
%
Além disso, o desenvolvimento não sustentável de software acadêmico em um
domínio, pode levar a um quadro de desordem caótica disfuncional - DCD,
caracterizado pela existência de muitos projetos similares, com poucos usuários
e ciclos de vida curtos, e que terminam em paralelo ao financiamento inicial,
comunidades desconectadas e paralelas, incompatibilidade entre projetos, e
tentativas aparentemente não coordenadas de ``reiniciar'' tudo.
%
No entanto, não há evidências fundamentadas sobre DCD em software acadêmico da
área de Engenharia de Software, especialmente no domínio de análise estática,
com uma longa tradição no desenvolvimento de ferramentas para apoiar pesquisas
em diferentes áreas.
%
O objetivo geral desta pesquisa de mestrado foi caracterizar projetos de
software acadêmico do domínio de análise estática em relação a sustentabilidade
técnica, em termos de publicização, reconhecimento e ciclo de vida, e investigar sua relação com a DCD.
%
Foi realizado um estudo exploratório sobre a sustentabilidade técnica de
projetos de software acadêmico de análise estática publicados em duas
conferências importantes para a área: ASE e SCAM.
%
Com base nos projetos publicados, uma revisão da literatura foi realizada, nas
bases da ACM e IEEE, para a caracterização de cada projeto em termos
de tipos e número de menções feitas por outros artigos científicos, e
contribuições em seu código fonte.
%
Uma pesquisa documental realizada com base em código fonte, manuais e
repositórios, trouxe informações a respeito da distribuição, licença e ciclo de
vida do projeto.
%
Para os projetos com código fonte disponível, métrica de tamanho em número
de módulos foi utilizada para caracterizar o estágio de evolução no ciclo de
vida do software.
%
Foram encontrados 60 projetos de software acadêmico de análise estática
publicados em artigos da ASE e SCAM.
%
A caracterização desses projetos mostrou que: 40\% não está disponível
publicamente, ou seja, não é possível obter o software na URL informada pelos
autores, inviabilizando a reprodução de estudos que tenham usado tais projetos
de software;
%
23\% não possui nenhuma menção nas bases ACM e IEEE além da publicação inicial
do software; e 26\% recebeu contribuição em código fonte.
%
No domínio de análise estática, observou-se a existência de muitos projetos
similares, com poucos usuários, e ciclos de vida curtos.
% vim: filetype=tex
