O uso crescente de software acadêmico, isto é, software desenvolvido para
apoiar pesquisas científicas em diversas áreas do conhecimento, tem feito
a Ciência moderna depender da sustentabilidade técnica do software.
%, aqui definida como a capacidade de se manter disponível e confiável.
%
O desenvolvimento não sustentável de software acadêmico pode ferir um dos
fundamentos da Ciência: a reprodutibilidade, ou a capacidade de reprodução de
estudos científicos por pesquisadores independentes.
%
Além disso, o desenvolvimento não sustentável de software acadêmico em um
domínio, pode levar a um quadro de desordem caótica disfuncional (DCD),
caracterizado pela existência de muitos projetos similares, com poucos usuários
e ciclos de vida curtos, e que terminam em paralelo ao financiamento inicial,
comunidades desconectadas e paralelas, incompatibilidade entre projetos, e
tentativas aparentemente não coordenadas de ``reiniciar'' tudo.
%
No entanto, não há %evidências fundamentadas 
estudos sobre sustentabilidade técnica ou DCD em software acadêmico da
área de Engenharia de Software, especialmente no domínio de análise estática,
com uma longa tradição no desenvolvimento de ferramentas para apoiar pesquisas
em diferentes áreas.
%
O objetivo geral desta pesquisa de mestrado foi analisar projetos de
software acadêmico de análise estática 
com o propósito de caracterizar sua sustentabilidade técnica, 
com respeito a publicização, reconhecimento e ciclo de vida, 
na perspectiva do cientista -- desenvolvedor ou usuário -- de software
acadêmico no contexto das conferências 
ASE ({\it Automated Software Engineering}) e SCAM ({\it Working Conference on Source Code
Analysis \& Manipulation}).
%
O software acadêmico publicado nestas conferências 
foi objeto de uma pesquisa documental, realizada com base em código fonte, 
manuais e repositórios, para caracterização de sua publicização.
%a respeito da distribuição, licença e ciclo de vida do software.
%
Uma revisão da literatura foi realizada nas bases da ACM e IEEE
%para identificação de artigos científicos que mencionam essa amostra de software acadêmico
para a caracterização do reconhecimento do software acadêmico
em termos de tipos e número de menções feitas por outros artigos científicos, e
contribuições em seu código fonte.
%
Para software acadêmico com código fonte disponível, 
foi realizada a caracterização de seu ciclo de vida,
com base no número de módulos e no número de lançamentos.
%
Foram encontrados 60 projetos de software acadêmico de análise estática
publicados em artigos da ASE e SCAM.
%
A caracterização de sua sustentabilidade técnica mostrou que: 
40\% não está disponível publicamente, ou seja, não é possível obter 
o software na URL informada pelos autores, inviabilizando 
a reprodução de estudos que tenham usado tal software;
%
23\% não possui outra menção nas bases ACM e IEEE além das feitas na publicação original
do software; e 25\% recebeu contribuição em código fonte.
%
%A caracterização da sustentabilidade técnica de software acadêmico de análise estática
Pôde-se observar alguns indícios de DCD: 
existência de muitos projetos de software acadêmico de análise estática
com poucos usuários, e ciclos de vida curtos.
% vim: filetype=tex
