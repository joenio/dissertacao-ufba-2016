%%%%\textbf{Contexto.} 
%Software acadêmico é todo projeto de software utilizado para apoiar pesquisas científicas. 
O uso crescente de software acadêmico, isto é, software desenvolvido para apoiar pesquisas
científicas, % nas mais diversas áreas do conhecimento,
tem aumentado a dependência da Ciência moderna sobre a sua sustentabilidade técnica,
aqui definida como a capacidade de se manter disponível e confiável, % em novas plataformas,
atendendo continuamente às novas necessidades de um ambiente %através de uma
%adequada evolução frente às condições 
em constante mudança.
%
%%%%\textbf{Problema.} 
O desenvolvimento não sustentável de software acadêmico pode 
ferir um dos fundamentos da Ciência:
a reprodutibilidade, ou a qualidade que se atribui a um estudo científico 
quando ele pode ser reproduzido por pesquisadores independentes. 
Além disso, o desenvolvimento não sustentável de software acadêmico,
pode levar a um quadro de desordem caótica disfuncional (DCD) em um domínio,
caracterizado pela existência de muitos projetos similares, com poucos
usuários e ciclos de vida curtos, e que terminam em paralelo ao financiamento
inicial,  comunidades desconectadas e paralelas, incompatibilidade entre
projetos, e tentativas não coordenadas de ``reiniciar'' tudo
({\it re-boots}).
%\cite{howison2015understanding}.
No entanto, não há evidências fundamentadas sobre a sustentabilidade técnica
e a DCD em software acadêmico da área de Engenharia de Software, especialmente
no domínio de análise estática, com uma longa tradição no
desenvolvimento de ferramentas para apoiar pesquisas em diferentes áreas.
%%%%\textbf{Objetivos.}
O objetivo geral deste trabalho foi caracterizar projetos de software acadêmico
do domínio de análise estática em relação a sustentabilidade técnica, em termos
de publicização/disponibilidade, visibilidade
científica e manutenibilidade,
e investigar sua relação com a DCD no domínio de análise estática.
%%%%\textbf{Métodos.}
Foi realizado um estudo exploratório sobre a sustentabilidade
técnica de projetos de software acadêmico de análise estática 
publicados em duas conferências importantes para a área: ASE e SCAM.
%
Com base nos projetos publicados,  %na ASE e SCAM, 
uma revisão da literatura foi realizada,
nas bases da ACM e IEEE, 
para a caracterização de cada software acadêmico em termos de
tipos e número de menções feitas por outros artigos científicos,
e contribuições em seu código fonte.
Uma pesquisa documental realizada com base em 
código fonte, manuais e repositórios, trouxe informações a respeito da
distribuição, licença e ciclo de vida do projeto.
Para os projetos com código fonte disponível, métricas de
complexidade estrutural foram usadas para caracterização de sua manutenabilidade.
%%%%\textbf{Resultados.}
Foram encontrados 60 projetos de software acadêmico de análise estática % dentre 1873 artigos.
publicados em artigos da ASE e SCAM.
%
A caracterização desses projetos mostrou que: 
40\% não está disponível publicamente, ou seja, não é possível obter o software na URL informada pelos
autores, inviabilizando a reprodução de estudos que tenham usado tais projetos de software;
%
23\% não possui nenhuma menção nas bases
ACM e IEEE além da publicação inicial do software; 
e 26\% recebeu contribuição em código fonte em outros estudos
além da publicação inicial.
No domínio de análise estática, observou-se a existência de muitos projetos similares, com poucos
usuários, e ciclos de vida curtos.
%e que terminam em paralelo ao financiamento inicial; comunidades desconectadas e paralelas,
%incompatibilidades entre projetos, 
%e tentativas aparentemente não coordenadas de ``reiniciar'' tudo ({\it re-boots}).

%através de pesquisas além da publicação inicial do projeto de software.
%, 125 artigos distintos mencionam os projetos em
%estudos fazendo uso deles.
%
%16 projetos foram encontrados em mençoes em estudos contribuindo com tais
%projetos em nível de código fonte, apenas 
%
% vim: filetype=tex
