\xchapter{Conclusões}{}
\label{conclusoes}

%No domínio de análise estática, observou-se a existência de muitos encerrados
%e indisponíveis,
%com pouco reconhecimento, com poucos usuários , e
%ciclos de vida curtos ou em estágio inicial de desenvolvimento, 

Encontramos 60 projetos de software acadêmico de análise estática publicados
nas conferências ASE e SCAM numa revisão de literatura entre 1873 artigos,
estes projetos foram publicados pelos seus autores com identificação de nome e
URL para download, uma segunda revisão de literatura nas bases ACM e IEEE
encontrou 416 artigos mencionando estes projetos de software acadêmico.

Uma classificação das menções aos projetos entre estes 416 artigos encontrou
199 menções do tipo Cita, 124 do tipo Usa e 106 do tipo Contribui, este último
conjunto de menções inclui os artigos selecionados na primeira revisão de literatura
feito nas conferências ASE e SCAM.

Os projetos foram caracterizados em relação ao estágio de evolução e o código
fonte de 206 versões distintas de um conjunto de 24 projetos foram analisados
para coleta da métrica de tamanho em número de módulos, estes dados foram
utilizados para caracterizar os projetos em estágios {\it Initial development},
{\it Evolution}, {\it Servicing}, {\it Phaseout} e {\it Closedown}.

Observamos que 78\% dos projetos de software acadêmico de análise estática
encontram-se em estágio inicial de desenvolvimento ({\it Initial development})
sendo encerrados ({\it Phaseout}) ou já encerrados ({\it Closedown}). Indicando
que temos neste domínio muitos projetos sem atividade de evolução após a sua
criação e publicação.

9\% não foram caracterizados em termos de ciclo de vida por falta de informaçao
e apenas 13\% possuem características de projetos ativos com uso e evolução
constantes. Sendo considerados como bons candidados a projetos úteis para uso
em outras pesquisas.

Ou seja, 78\% do software acadêmico produzido no domínio de aplicação de
análise estática apresentam características de não terem sido desenvolvidos e
mantidos de forma tecnicamente sustentável, ou seja, não são disponíveis para
obtenção, estão em estágio inicial de desenvolviment ou foram encerrados.

Os projetos em estágio inicial apesar de não serem considerados projetos em
estágio de serem utilizados em outras pesquisa são bons candidados a serem
adotado em outras pesquisa como ponto de partida inicial para a adaptação e
atender novas demandas.

\section{Contribuições}

Esta pesquisa contribuiu caracterizando o software acadêmico do domínio
de análise estática publicados nas conferências ASE e SCAM.

Esta caracterização serve vários propósitos:

1 serve como ponto de partida para avaliar os problemas em relação a sustentabilidade
técnica do desenvolvimento de software neste domínio entre cientistas, de forma
que a partir daí possamos traçar estrtégias para solucionar e melhorar o campo tanto
em termos práticos quanto teóricos.

2 serve como mapa dos projetos existentes para interessados em utilizar ou mesmo
avaliar tais ferramentas, ou ainda, como base para iniciar novos desenvolvimentos,
especialmente entre os projetos em estágio inicial, que apresentam em alguns casos
indícios de estarem abandonados mas o código fonte deles possui grande potencial
de ser útil e reduzir o caminho tomado em novas pesquisas.

3 serve como auto-reflexão para aqueles que desenvolvem software acadêmico
compreender que existe uma grande quantidade de recurso investido no desenvolvimento
destes projetos sendo consumido de maneira ineficiente, e a partir daí, tomar
como exercício resolver estes problemas e reduzir duplicação de esforço e retrabalho.

4 contribui também para a discussão teórica e definição do que vem a ser sustentabilidade
de software, um tema que ainda carece de definição clara, além de contribui para
uma definição teórica e prática sobre ecossistema de software acadêmico de análise estática.

5 e finalmente e talvez mais importante  serve como uma alerta para os prejuízos
a Ciência que a indisponibilidade de código destes projetos causa uma vez que repetir
ou reproduir estudos do passado é uma excelente forma, tanto de validar ou refutar conclusões,
quanto de contribuir evoluindo as pesquisa e os dados, além de ser tb uma oportunidade
de evoluir os próprios códigos.

6 ...  A caracterização ... contribuir para a sustentabilidade do software acadêmico
de análise estática.

\section{Trabalhos futuros}

O trabalho mais imediato para extender esta pesquisa seria atualizar a revisão
de literatura inicial para extender o período de seleção de software acadêmico
de análise estática incluindo os anos de 2016 e 2017, uma vez que esta
dissertação limitou a seleção ao ano de 2015.

Outro trabalho importante é ampliar o escopo do estudo incluindo mais
conferências visando aumentar o realismo sobre o domínio estudado,
especialmente conferências tradicionais de Engenharia de Software, como por
exemplo, a conferência ICSE (International Conference on Software
Engineering)\footnote{\url{http://www.icse-conferences.org}} por ser uma das
mais tradicionais e importantes conferências da área de Engenharia de Software.
Consideramos também ser importante incluir o congresso CBSOFT (Congresso
Brasileiro de Software: Teoria e Prática)\footnote{\url{http://www.cbsoft.org}}
por ser uma importante conferência no contexto Brasileiro de Engenharia de
Software, com longo histórico, e trilhas de publicação de ferramentas.

Outro ponto em relação a conferências é coletar o fator de impacto da
conferência e utilizar esta informação na análise dos dados e discussão, é
possível que conferências com maior fator de impacto apresentem projetos com
maior reconhecimento e com características de publicização mais adequadas.

Além de ampliar o escopo de forma horizontal incluindo novas conferências, é
interessante também aumentar a qualidade da caracterização do software
incluindo novas dimensões de caracterização, especialmente dimensões na visão
de usuário, como por exemplo, documentação, facilidade de instalaçao, execução,
existência de manuais, avaliar o nível de descrição e apresentação do software
no site ou repositório, ou mesmo incluir aspectos de Engenharia de Software,
como por exemplo, testes, resolução de {\it issues}, métricas de qualidade como
por exemplo complexidade estrutural, número de contribuidores, número de
commits e uso de controle de versão.

%O autor do software provê informações sobre como citar o software apropriadmente? \cite{allen2017engineering}

Uma outra forma de enriquecer a caracterização de cada software é usar
caracterizações já realizadas em outros trabalhos científicos, muitos estudos
avaliam e comparam ferramentas de análise estática entre sí, os artigos
encontrados na seleção de software acadêmico são os primeiros candidatos a
serem utilizados como fonte de coleta de dados sobre cada software.

%apenas caracterizaçõas mas também avaliações, ou seja, entrar no conteúdo dos
%artigos encontrados sobre cada software, e usar as informações já
%caracterizadas para compor uma caracterização mais rica do conjunto de
%ferramentas.

%incluir outras bases na primeira revisão (estruturada) realizada para
%ver se outras ferramentas importantes não ficaram de for.

Outra possibilidade de trabalho futuro é revisar artigos publicados em jornais
com foco específico em publicação de software em busca mais projetos de
software acadêmico de análise estática, entre estes jornais podemos citar, por
exemplo, o JORS (Journal of Open Research
Software)\footnote{\url{https://openresearchsoftware.metajnl.com/articles/10.5334/jors.bt}},
JOSS (Journal of Open Source Software)\footnote{\url{http://joss.theoj.org}} e
o SoftwareX\footnote{\url{https://www.journals.elsevier.com/softwarex}} da
Elsevier. 

%Incluir e avaliar ferramentas publicadas e apresentadas no evento de
%ferramentas da UFBA.
%
%Medir a usabilidade do software acadêmico de análise estática e como podem ser
%mais úteis ao desenvolvedor de modo geral, especialmente para o cientista
%desenvolvedor de software.
%
%Investigar o impacto da sustentabilidade técnica do software acadêmico na
%reprodutibilidade dos estudos que fazem uso do software como apoio metodológico
%(coleta ou análise).
