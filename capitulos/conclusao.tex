\xchapter{Conclusões}{}
\label{conclusoes}

(pendente)

Observamos que 61\% do software acadêmico produzido no domínio de aplicação de
análise estática são sustentáveis, ou seja, continuam disponíveis ao longo do
tempo.

\section{Contribuições}

(pendente)

A caracterização ... contribuir para a sustentabilidade do software acadêmico
de análise estática.

\section{Trabalhos futuros}

O autor do software provê informações sobre como citar o software apropriadmente? \cite{allen2017engineering}

Incluir mais conferências na revisão de literatura para busca e seleção de
software acadêmico de análise estática, a princípio vemos como próximos passos
neste sentido incluir o CBSOFT por ser uma importante conferência em relação a
Brasil, com um bom histórico, e outra é o e ICSE por ser uma das mais
tradicionais e importantes conferências da área de Engenharia de Software.

incluir também a conferência SANER http://saner.unimol.it/index -
 o congresso é A2 = até 20 de out
 o journal é A2 = até 26 de jan.

Caracterizar outras informações a respeito de cada projeto de software, como,
possui manuais, descrição detalhadas de uso no site? Usa controle de versão?
Número de contribuidores, número de commits, etc.

olhar fator de impacto das conferencias ASE e SCAM, separar dois grupos de
projetos, e avaliar se tem mais mencoes com base no fator de impacto.

Incluir na caracterização de software acadêmico outros aspectos de Engenharia
de Ssoftware, como por exemplo, aspectos relacionados ao processo de
construção, exemplo, testes, resolução de issues, ou aspectos relacionados ao
produto, como métricas de qualidade como por exemplo complexidade estrutural.
Ou ainda, dimensões de uso ou na visão de usuário, como, documentação,
usabilidade, entre outros.

%estudo para o futuro, investigar proposicao P11 da teoria de terceiro, estudo
%relacionado a esta proposição cite:Maccormack2006, estudo mostrou evidencias
%de impacto na modularidade, Exploring the duality between product and
%organizational architectures: A test of the mirroring hypothesis Alan
%MacCormack, Carliss Baldwin, John Rusnak

usar mais dimensões na caracterização, ver issue \#18, 

usar caracterizacao ja realizadas em outros trabalhos, ver issue \#22,
compor a caracterização do software com caracterizações já realizadas,
não apenas caracterizaçõas mas avaliações, ou seja, entrar no conteúdo
dos artigos encontrados nas bases ACM e IEEE, e usar as informações
já caracterizadas para compor uma caracterização mais rica
do conjunto de ferramentas.

incluir outras bases na primeira revisão (estruturada) realizada para
ver se outras ferramentas importantes não ficaram de for.

atualizar revisões, incluindo ferramentas encontradas em 1, até o ano 2017. 

Revisar jornais com foco em software em busca de artigos sobre
software acadêmico de análise estática, 
Journal of Open Research Software (JORS) -
 https://openresearchsoftware.metajnl.com/articles/10.5334/jors.bt/.
Outro journal é o JOSS e o SoftwareX da Elsevier. https://www.journals.elsevier.com/softwarex/ joss.theoj.org.

ao falar de ecossistema seria bom citar o evento da ufba de tools onde o
analizo já foi apresentado, poderia também caracterizar as ferramentas
publicadas na ufba para ter uma visão de como elas são vistas, ciclo de vida,
etc.

Investigar o impacto da sustentabilidade técnica do software acadêmico na
reprodutibilidade dos estudos que fazem uso do software como apoio metodológico
(coleta ou análise).

%\subsection{Autoria das menções aos projetos de software de análise estática}

%das bibliotecas digitais, temos para cada um dos artigos todos os seus
%metadados,. Os autores de
%cada uma das menções ao software, por exemplo, serão utilizados na fase de
%análise para calcular o quanto de autores novos começaram a publicar sobre
%certo software acadêmico.

%Os dados coletados sobre as menções a cada software foram acrescentados com uma
%nova informação calculada a partir da autoria de cada artigo mencionando o
%software, estamos considerando que todos os autores de um certo artigo tem o
%mesmo peso em relação a menção ao software naquele estudo, mesmo sabendo que não
%é raro que cientistas trabalhando em conjunto apresentem níveis diferentes de
%domínio sobre cada parte da pesquisa.

%Os autores originais do primeiro artigo publicando sobre software, na maior
%parte dos casos é o mesmo paper selecionado no estudo anterior que deu origem
%ao conjunto de projetos, são considerados os autores originais do projeto, a
%partir desta primeira publicação cada artigo mencionando o software indica a
%entrada de novos atores no ecossistema daquele software.

%Foi coletado então para cada artigo em relação aos autores o quanto são
%autores novos, e classificamos em termos de todos os autores do estudo
%são novos, se todos já publicaram sobre o software anteriormente, ou se
%nenhum dos autores jamais publicou sobre aquele software anteriormente, A Tabela
%\ref{esquema-de-autoria} apresenta em detalhes este esquema.

%representado quantos novos atores foram incluídos
%no ecossistema daquele software, isto foi feito comparando o conjunto de autores
%da publicaçao com o conjunto acumulado de todos os autores anteriores, 
%podendo assumir um dos valores da Tabela \ref{coding-scheme-author}.

%\begin{table}[h]
%\caption{Esquema para classificação de autoria de menções aos projetos de software acadêmico.}
%\centering
%\begin{tabular}{ l c p{8cm} }
%  \hline
%  Novos atores no ecossistema & Peso & Explicação \\
%  \hline
%  Nenhum    & 0.5  & Nenhum dos autores jamais publicou sobre o software \\
%  Parte     & 0.25 & Uma parte dos autores já publicou sobre o software em anos anteriores \\
%  Todos     & 0.1  & Todos os autores já publicaram sobre o software em anos anteriores \\
%  Criadores & 0    & São os primeiros autores a publicar sobre o software \\
%  \hline
%\end{tabular}
%\label{esquema-de-autoria}
%\end{table}

%Estes dados foram calculados com base nos metadados já coletados anteriormente
%dispníveis nos arquivos BibTeX com as menções à cada projeto, tratamos os nomes
%dos autores para utilizar o mesmo formato, já que os artigos variam em relação
%ao padrão de nomes dos autores.

%Com os nomes dos autores normalizados comparamos e caso tenham a mesma string
%consideramos que trata-se de um mesmo pesquisador, os nomes iguals são
%considerados como sendo a mesma pessoa. Avaliamos cada artigo em ordem
%cronológica comparando os autores com relação a todos os autores anteriores.

%Este processo foi realizado automaticamente com um script escrito durante este
%trabalho de pesquisa, os dados resultantes são armazenados de volta nos
%arquivos BibTeX.

%\subsection{Autoria das menções aos projetos de software de análise estática}

% nao vou mais analisar nenhum dado de autoria, apenas numero de citacoes e tipo!!!
%
\begin{longtable}{ l c c }
\caption{Número de autores únicos mencionando os projetos de software.}
\label{authorship-table} \\
  \hline
  \hhline{ l c c |}
  \endfirsthead
  \hhline{ l c c |}
  \hline
  \textbf{Nome do software} & {\bf Autores únicos} & {\bf Menções} \\
  \hline
  \hhline{ l c c |}
  \endhead
  \hhline{---}
  \multicolumn{3}{c}{continua na próxima página} \\
  \hhline{---} \endfoot
  \hhline{---} \endlastfoot
  \textbf{Nome do software} & {\bf Autores únicos} & {\bf Menções} \\
  \hline
   2LS & 5 & 1 \\
   AccessAnalysis & 2 & 2 \\
   APIExample & 15 & 4 \\
   BEG & 11 & 9 \\
   ccJava & 6 & 5 \\
   CIVL & 24 & 6 \\
   CodeBoost & 36 & 15 \\
   CSL & 12 & 6 \\
   CPA+ & 11 & 5 \\
   CSeq & 13 & 5 \\
   DDVerify & 9 & 3 \\
   Derailer & 2 & 2 \\
   Diagnosys & 4 & 1 \\
   DOMPLETION & 3 & 2 \\
   DRC & 10 & 5 \\
   e-munity & 5 & 1 \\
   EJB & 8 & 3 \\
   Error Prone & 9 & 2 \\
   ESBMC & 119 & 42 \\
   ETXL & 2 & 1 \\
   FaultBuster & 5 & 1 \\
   Flowgen & 8 & 3 \\
   GRT & 25 & 10 \\
   GUIZMO & 3 & 1 \\
   GumTree & 68 & 19 \\
   HUSACCT & 10 & 7 \\
   Indus & 7 & 4 \\
   JastAdd & 107 & 45 \\
   JFlow & 21 & 7 \\
   JstereoCode & 24 & 8 \\
   Jtop & 5 & 2 \\
   Bogor/Kiasan & 43 & 16 \\
   Loopfrog & 14 & 5 \\
   Lotrack & 5 & 2 \\
   MPAnalyzer & 2 & 1 \\
   MSP & 5 & 2 \\
   mygcc & 21 & 7 \\
   PARSEWeb & 73 & 24 \\
   PAT & 7 & 3 \\
   PHP AiR & 8 & 9 \\
   protopurity & 4 & 1 \\
   Pseudogen & 8 & 1 \\
   PtYasm & 5 & 2 \\
   PuMoC & 2 & 2 \\
   PYTHIA & 3 & 2 \\
   ReAssert & 38 & 13 \\
   Rêve & 5 & 1 \\
   RRFinder & 12 & 3 \\
   Sapid/XML & 5 & 5 \\
   Sonar Qube Plug-in & 5 & 1 \\
   SPARTA & 15 & 4 \\
   srcML & 99 & 40 \\
   SWAT & 6 & 4 \\
   TACLE & 3 & 3 \\
   TEBA & 2 & 1 \\
   TestEra & 39 & 23 \\
   Vdiff & 15 & 5 \\
   WALA & 37 & 11 \\
   Wrangler & 61 & 33 \\
   XOgastan & 17 & 5 \\
  \hline
  {\bf Total} & 983 & 456 \\
\end{longtable}


