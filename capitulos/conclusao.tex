\xchapter{Conclusões}{}
\label{conclusoes}

Encontramos 60 projetos de software acadêmico de análise estática publicados
nas conferências ASE e SCAM numa revisão de literatura entre 1873 artigos,
estes projetos foram publicados pelos seus autores com identificação de nome e
URL para download, uma segunda revisão de literatura nas bases ACM e IEEE
encontrou 416 artigos mencionando estes projetos de software acadêmico.

Uma classificação das menções aos projetos entre estes 416 artigos encontrou
199 menções do tipo Cita, 124 do tipo Usa e 106 do tipo Contribui, este último
conjunto de menções inclui os artigos selecionados na primeira revisão de literatura
feito nas conferências ASE e SCAM.

Os projetos foram caracterizados em relação ao estágio de evolução e o código
fonte de 206 versões distintas de um conjunto de 24 projetos foram analisados
para coleta da métrica de tamanho em número de módulos, estes dados foram
utilizados para caracterizar os projetos em estágios {\it Initial development},
{\it Evolution}, {\it Servicing}, {\it Phaseout} e {\it Closedown}.

Observamos que 78\% dos projetos de software acadêmico de análise estática
encontram-se em estágio inicial de desenvolvimento ({\it Initial development})
sendo encerrados ({\it Phaseout}) ou já encerrados ({\it Closedown}). Indicando
que temos neste domínio muitos projetos sem atividade de evolução após a sua
criação e publicação.

9\% não foram caracterizados em termos de ciclo de vida por falta de informaçao
e apenas 13\% possuem características de projetos ativos com uso e evolução
constantes. Sendo considerados como bons candidados a projetos úteis para uso
em outras pesquisas.

Ou seja, 78\% do software acadêmico produzido no domínio de aplicação de
análise estática apresentam características de não terem sido desenvolvidos e
mantidos de forma tecnicamente sustentável, ou seja, não são disponíveis para
obtenção, estão em estágio inicial de desenvolviment ou foram encerrados.

Os projetos em estágio inicial apesar de não serem considerados projetos em
estágio de serem utilizados em outras pesquisa são bons candidados a serem
adotado em outras pesquisa como ponto de partida inicial para a adaptação e
atender novas demandas.

\section{Contribuições}

Esta pesquisa contribuiu caracterizando o software acadêmico do domínio
de análise estática publicados nas conferências ASE e SCAM.

Esta caracterização serve vários propósitos:

1 serve como ponto de partida para avaliar os problemas em relação a sustentabilidade
técnica do desenvolvimento de software neste domínio entre cientistas, de forma
que a partir daí possamos traçar estrtégias para solucionar e melhorar o campo tanto
em termos práticos quanto teóricos.

2 serve como mapa dos projetos existentes para interessados em utilizar ou mesmo
avaliar tais ferramentas, ou ainda, como base para iniciar novos desenvolvimentos,
especialmente entre os projetos em estágio inicial, que apresentam em alguns casos
indícios de estarem abandonados mas o código fonte deles possui grande potencial
de ser útil e reduzir o caminho tomado em novas pesquisas.

3 serve como auto-reflexão para aqueles que desenvolvem software acadêmico
compreender que existe uma grande quantidade de recurso investido no desenvolvimento
destes projetos sendo consumido de maneira ineficiente, e a partir daí, tomar
como exercício resolver estes problemas e reduzir duplicação de esforço e retrabalho.

4 contribui também para a discussão teórica e definição do que vem a ser sustentabilidade
de software, um tema que ainda carece de definição clara, além de contribui para
uma definição teórica e prática sobre ecossistema de software acadêmico de análise estática.

5 e finalmente e talvez mais importante  serve como uma alerta para os prejuízos
a Ciência que a indisponibilidade de código destes projetos causa uma vez que repetir
ou reproduir estudos do passado é uma excelente forma, tanto de validar ou refutar conclusões,
quanto de contribuir evoluindo as pesquisa e os dados, além de ser tb uma oportunidade
de evoluir os próprios códigos.

6 ...  A caracterização ... contribuir para a sustentabilidade do software acadêmico
de análise estática.

\section{Trabalhos futuros}

%O autor do software provê informações sobre como citar o software apropriadmente? \cite{allen2017engineering}

Incluir mais conferências na revisão de literatura para busca e seleção de
software acadêmico de análise estática, a princípio vemos como próximo passo
incluir o CBSOFT (Congresso Brasileiro de Software: Teoria e
Prática)\footnote{\url{http://www.cbsoft.org}} por ser uma importante
conferência no contexto Brasileiro de Engenharia de Software, com longo
histórico, e trilhas de publicação de ferramentas.

Além do CBSOFT vemos também como extremamente importante incluir a conferência
ICSE (International Conference on Software
Engineering)\footnote{\url{http://www.icse-conferences.org}} por ser uma das
mais tradicionais e importantes conferências da área de Engenharia de Software.

%incluir também a conferência SANER http://saner.unimol.it/index -
% o congresso é A2 = até 20 de out
% o journal é A2 = até 26 de jan.

Além de incluir outras conferências, pontuamos também como próximo passo
incluir outras dimensões para caracterização dos projetos, exemplo, se o
projeto possui manuais, avaliar o nível de detalhe e descrição do software no
seu site ou repositório, se usa controle de versão para gerenciar as mudanças
no código, o número de contribuidores, número de commits, e outras informações
relacionadas ao desenvolvimento dos projetos.
%Ou ainda,
%dimensões de uso ou na visão de usuário, como, documentação, usabilidade, entre
%outros.

Coletar o fator de impacto das conferências e utilizar esta informação na
análise dos dados e discussão, é possível que conferências com maior fator de
impacto apresentem projetos com maior reconhecimento e com características de
publicização mais adequadas.
%olhar fator de impacto das conferencias ASE e SCAM, separar dois grupos de
%projetos, e avaliar se tem mais mencoes com base no fator de impacto.

Incluir na caracterização de software acadêmico outros aspectos de Engenharia
de Software, como por exemplo, aspectos relacionados ao processo de construção,
testes, resolução de {\it issues}, ou aspectos relacionados ao produto, como
métricas de qualidade como por exemplo complexidade estrutural.

%estudo para o futuro, investigar proposicao P11 da teoria de terceiro, estudo
%relacionado a esta proposição cite:Maccormack2006, estudo mostrou evidencias
%de impacto na modularidade, Exploring the duality between product and
%organizational architectures: A test of the mirroring hypothesis Alan
%MacCormack, Carliss Baldwin, John Rusnak

%usar mais dimensões na caracterização, ver issue \#18, 

Usar caracterizacao já realizadas em outros trabalhos, e compor a
caracterização do software com dados já coletados em outros estudos, usar não
apenas caracterizaçõas mas também avaliações, ou seja, entrar no conteúdo dos
artigos encontrados sobre cada software, e usar as informações já
caracterizadas para compor uma caracterização mais rica do conjunto de
ferramentas.

%incluir outras bases na primeira revisão (estruturada) realizada para
%ver se outras ferramentas importantes não ficaram de for.

Atualizar a revisão de literatura inicial para extender o período incluindo
2016 e 2017, uma vez que nosso estudo limitou a seleção de software acadêmico
ao ano de 2015.

Revisar jornais com foco em publicação de software em busca de artigos sobre
software acadêmico de análise estática, por exemplo JORS (Journal of Open
Research
Software)\footnote{\url{https://openresearchsoftware.metajnl.com/articles/10.5334/jors.bt}},
JOSS (Journal of Open Source Software)\footnote{\url{http://joss.theoj.org}} e o
SoftwareX\footnote{\url{https://www.journals.elsevier.com/softwarex}} da
Elsevier. 

Incluir e avaliar ferramentas publicadas e apresentadas no evento de
ferramentas da UFBA.

Medir a usabilidade do software acadêmico de análise estática e como podem ser
mais úteis ao desenvolvedor de modo geral, especialmente para o cientista
desenvolvedor de software.

Investigar o impacto da sustentabilidade técnica do software acadêmico na
reprodutibilidade dos estudos que fazem uso do software como apoio metodológico
(coleta ou análise).
