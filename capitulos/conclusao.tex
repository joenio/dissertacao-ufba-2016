\xchapter{Conclusões}{}
\label{conclusoes}

%Mas independente de como seja calculado o impacto científico de uma determinada
%pesquisa o impacto causado se reverte potencialmente em mais recursos que
%poderão ser reinvestidos no próprio ecossistema onde o software está inserido.


O desenvolvimento de software acadêmico, de forma sustentável,
abre portas para elevar a qualidade geral do software e
da pesquisa científica, promovendo a reproducibilidade e
proporcionando um ambiente de compartilhamento e colaboração 
em oposição ao tradicional modelo de competição que permeia
o sistema de reputação e crédito científico.
%
Na área de engenharia de software, 
especialmente no domínio de análise estática,
com tradição no desenvolvimento de ferramentas para apoiar pesquisas
em diferentes áreas da ciência da computação,
a preocupação com a sustentabilidade técnica em software acadêmico
não pode ser desconsiderada. 

Esta pesquisa de mestrado caracterizou
projetos de software acadêmico de análise estática,
publicados até 2015 em artigos científicos das conferências ASE e SCAM,
em relação a sua sustentabilidade técnica, definida
em termos de publicização, reconhecimento e ciclo de vida.
%
O estudo sobre a publicização
%de software acadêmico de análise estática 
identificou 60 projetos de software acadêmico de análise estática
publicados originalmente nas conferências ASE e SCAM.
A caracterização da publicização desses projetos considerou
disponibilidade para download, acesso ao código fonte, forma de distribuição e licença.
Apenas 3\% dos 1873 artigos publicados nas conferências ASE e SCAM 
publicizaram software acadêmico de análise estática de forma adequada,
com indicação de URL para download.
%
O estudo sobre reconhecimento 
inspecionou \SearchUniqueCount \ artigos encontrados nas bases ACM
e IEEE através de busca avançada, 
usando características de \SoftwareCount \ projetos 
de software acadêmico de análise estática. 
A inspeção identificou  \ScreeningCount \ menções 
dos tipos \texttt{Cita}, \texttt{Usa} ou \texttt{Contribui}.
Houve um crescimento de 38\% ao ano no número de menções ao software acadêmico, 
e apenas 10\% do total de menções realizam contribuição de código fonte dos
projetos.
%
O estudo sobre ciclo de vida
caracterizou o estágio de evolução de software acadêmico de análise estática
com código fonte disponível,
considerando o número de lançamentos e de módulos no código fonte,
e revelou que a maior parte dos projetos encontra-se 
em estágio inicial de desenvolvimento ou encerrado.

Ao sintetizar os resultados (capítulo~\ref{discussao}) para responder a questão
geral de pesquisa ({\it \QuestaoGeralUm}) percebemos que a DCD é útil para
explicar a sustentabilidade técnica de um domínio de aplicação em níveis
distintos de profundidade, através da seguintes características:

\begin{description}
  \item [C1] Existência de muitos projetos com poucos usuários;
  \item [C2] Cada projeto tendo ciclos de vida curtos que se encerram junto ao financiamento inicial;
  \item [C3] Comunidades de usuários desconectadas e paralelas;
  \item [C4] Incompatibilidades entre os projetos de maneira persistente e imutável;
  \item [C5] Tentativas constantes e aparentemente não coordenadas de ``reiniciar'' tudo ({\it re-boots}).
\end{description}

Em nosso estudo utilizamos as características {\bf C1} e {\bf C2} da DCD para
explicar a sustentabilidade técnica dos projetos do ecossistema de análise
estática e percebemos que neste domínio há muitos projetos de software
acadêmico indisponíveis ou encerrados (78\%), com pouco reconhecimento e com
poucos usuários, com ciclos de vida curtos ou em estágio inicial de
desenvolvimento, revelando um ecossistema em que há pouca colaboração e
indícios de graves problemas de sustentabilidade.

%... a característica 
%(Existência de muitos projetos com poucos usuários) foi ... neste domínio; a
%característica 
%encerram junto ao financiamento inicial) foi parcialmente demonstrada onde
%projetos possuem ciclos de vida curtos, mas não coletados dados para realizar
%estudo sobre o financiamento dos projetos; as características {\bf C3}, {\bf
%C4} e {\bf C5} não foram avaliadas.

%sendo
%desordem caótica disfuncional ({\it ``dysfunctional chaotic churn''}),
%caracterizado por \cite{howison2015understanding}:

%A caracterização da sustentabilidade técnica de 
%software acadêmico de análise estática, no contexto de 
%60 projetos de análise estática estudados, mostrou que 
%há muitos projetos de software acadêmico indisponíveis ou encerrados (78\%), 
%com pouco reconhecimento, com poucos usuários, 
%e ciclos de vida curtos ou em estágio inicial de desenvolvimento,
%revelando um ecossistema em que há pouca colaboração
%e indícios de sintomas de desordem caótica disfuncional.

\section{Contribuições}

A caracterização da sustentabilidade técnica de software acadêmico de análise estática
publicado nas conferências de Engenharia de Software ASE e SCAM 
mapeou os projetos de software disponíveis e o grau de evolução que
se encontram no ciclo de vida.

Este mapeamento abre caminho para a compreensão de problemas relacionados a 
sustentabilidade de software acadêmico de análise estática 
e posterior definição de estratégias para
solucionar e melhorar o campo, tanto em termos práticos quanto teóricos.

O conhecimento a respeito dos projetos de software acadêmico de análise estática
existentes serve aos interessados em utilizar tais projetos em novas pesquisas,
seja como objeto de estudo, como apoio metodológico, ou ainda, como base para
iniciar novos desenvolvimentos.

%especialmente entre os projetos em estágio inicial, que apresentam em alguns casos
%indícios de estarem abandonados mas o código fonte deles possui grande potencial
%de ser útil e reduzir o caminho tomado em novas pesquisas.

Esta pesquisa contribui ainda com uma auto-reflexão sobre o campo de análise
estática e seus projetos de software acadêmico, especialmente em relação ao
esforço e recursos investidos no desenvolvimento de software neste domínio de
aplicação sendo consumidos de maneira ineficiente.

%, para que
%a partir daí, como exercício resolver estes problemas e reduzir duplicação de esforço e retrabalho.

No campo teórico, contribui para uma melhor compreensão do que vem a ser
sustentabilidade de software, um tema que ainda carece de definição clara,
especialmente sustentabilidade de software acadêmico de análise estática.

%além de contribui para
%uma definição teórica e prática sobre ecossistema de software acadêmico de análise estática.

Por fim, contribui alertando para os prejuízos que a indisponibilidade de
código destes projetos causam para a Ciência, uma vez que acesso ao código é
parte fundamental para validar ou refutar conclusões de pesquisas através da
reprodução e replicação.

%ou reproduir estudos do passado é uma excelente forma, tanto de validar ou refutar conclusões,
%quanto de contribuir evoluindo as pesquisa e os dados, além de ser tb uma oportunidade
%de evoluir os próprios códigos.

%6 ...  A caracterização ... contribuir para a sustentabilidade do software acadêmico
%de análise estática.

\section{Limitações}

%Escopo limitado.

O escopo do estudo limitou-se a selecionar software acadêmico em apenas um
domínio de aplicação, análise estática, e em apenas duas conferências de
Engenharia de Software, ASE e SCAM, e ainda, usando como data limite o ano de
2015. A busca por menções aos projetos também foi limitado a apenas duas
bibliotecas de indexação de publicações, ACM e IEEE.

%apenas projetos publicados com URL e indicados que estão disponíveis para download
%não levar em consideração o fator de impacto das conferências onde os artigos foram publicados

Além da limitação de escopo, este estudo realizou também uma série de
procedimentos manuais, a revisao de literatura para seleção de software
acadêmico foi realizada manualmente, a execução da busca nas bases ACM e IEEE
foi feita em atividades manuais.

%Procedimentos manuais.
%extração de informações de cada projeto também totalmente manual, incluindo as buscas nas bases ACM e IEEE
%coleta de dados ...

\section{Trabalhos futuros}

% Sobre escopo

O trabalho mais imediato para extender esta pesquisa seria atualizar a revisão
de literatura inicial para extender o período de seleção de software acadêmico
de análise estática incluindo os anos de 2016 e 2017, uma vez que esta
dissertação limitou a seleção ao ano de 2015.

Outro trabalho importante é ampliar o escopo do estudo incluindo mais
conferências visando aumentar o realismo sobre o domínio estudado,
especialmente conferências tradicionais de Engenharia de Software, como por
exemplo, a conferência ICSE (International Conference on Software
Engineering)\footnote{\url{http://www.icse-conferences.org}} por ser uma das
mais tradicionais e importantes conferências da área de Engenharia de Software.
Consideramos também ser importante incluir o congresso CBSOFT (Congresso
Brasileiro de Software: Teoria e Prática)\footnote{\url{http://www.cbsoft.org}}
por ser uma importante conferência no contexto Brasileiro de Engenharia de
Software, com longo histórico, e trilhas de publicação de ferramentas.

% Sobre procedimentos manuais
Usar técnicas de ... para identificar automaticamente menções e seus tipos ...
VER com Rodrigo.

% Sobre extensões: novas questões

Outro ponto em relação a conferências é coletar o fator de impacto da
conferência e utilizar esta informação na análise dos dados e discussão, é
possível que conferências com maior fator de impacto apresentem projetos com
maior reconhecimento e com características de publicização mais adequadas.

Além de ampliar o escopo de forma horizontal incluindo novas conferências, é
interessante também aumentar a qualidade da caracterização do software
incluindo novas dimensões de caracterização, especialmente dimensões na visão
de usuário, como por exemplo, documentação, facilidade de instalaçao, execução,
existência de manuais, avaliar o nível de descrição e apresentação do software
no site ou repositório, ou mesmo incluir aspectos de Engenharia de Software,
como por exemplo, testes, resolução de {\it issues}, métricas de qualidade como
por exemplo complexidade estrutural, número de contribuidores, número de
commits e uso de controle de versão.

%O autor do software provê informações sobre como citar o software apropriadmente? \cite{allen2017engineering}

Uma outra forma de enriquecer a caracterização de cada software é usar
caracterizações já realizadas em outros trabalhos científicos, muitos estudos
avaliam e comparam ferramentas de análise estática entre sí, os artigos
encontrados na seleção de software acadêmico são os primeiros candidatos a
serem utilizados como fonte de coleta de dados sobre cada software.

%apenas caracterizaçõas mas também avaliações, ou seja, entrar no conteúdo dos
%artigos encontrados sobre cada software, e usar as informações já
%caracterizadas para compor uma caracterização mais rica do conjunto de
%ferramentas.

%incluir outras bases na primeira revisão (estruturada) realizada para
%ver se outras ferramentas importantes não ficaram de for.

Outra possibilidade de trabalho futuro é revisar artigos publicados em jornais
com foco específico em publicação de software em busca mais projetos de
software acadêmico de análise estática, entre estes jornais podemos citar, por
exemplo, o JORS (Journal of Open Research
Software)\footnote{\url{https://openresearchsoftware.metajnl.com/articles/10.5334/jors.bt}},
JOSS (Journal of Open Source Software)\footnote{\url{http://joss.theoj.org}} \cite{smith2017journal} e
o SoftwareX\footnote{\url{https://www.journals.elsevier.com/softwarex}} da
Elsevier. 

%Incluir e avaliar ferramentas publicadas e apresentadas no evento de
%ferramentas da UFBA.
%
%Medir a usabilidade do software acadêmico de análise estática e como podem ser
%mais úteis ao desenvolvedor de modo geral, especialmente para o cientista
%desenvolvedor de software.
%
%Investigar o impacto da sustentabilidade técnica do software acadêmico na
%reprodutibilidade dos estudos que fazem uso do software como apoio metodológico
%(coleta ou análise).
