\xchapter{Conclusões}{}
\label{conclusoes}

Apenas 3\% dos artigos publicados nas conferências ASE e SCAM publicizam
software acadêmico de análise estática de forma adequada com indicação de URL
para download.
Neste domínio, observou-se a existência de muitos projetos de
software acadêmico encerrados e indisponíveis (78\%), com pouco reconhecimento,
com poucos usuários, e ciclos de vida curtos ou em estágio inicial de
desenvolvimento.
Há um crescimento no número de menções de 38\% ao ano a software acadêmico, e
apenas 10\% do total de menções realizam contribuição de código fonte aos
projetos.

Os problemas identificados neste estudo podem ser atribuídos em sua maioria a
questões culturais e falta de prática de algumas iniciativas bastante simples
mas muito efetivas para reverter o quadro .... algumas recomendações básicas
pode fazer uma grande diferença neste cenário.

Recomendações aos desenvolvedores de software acadêmico:

\begin{itemize}
  \item Sempre publicar o código fonte do software acadêmico!
  \item Utilizar licenças de software livre, especialmente licenças com mecanismo de copyleft, exemplo: GPL.
  \item Publicar o software em fontes com maior garantia de longevidade, como por exemplo, Github, Gitlab, Savannah ou Sourceforge
  \item Evitar publicar o software em infraestrutura particular ou própria, como por exemplo os servidores da universidade, tendem a mudar de endereço ou serem descontinuados.
  \item Fornecer instrução sobre como citar o software adequadamente, se possível incluir no repositório do projeto um arquvo BibTeX \texttt{paper.bib} com os metadados de como deve ser citado.
  \item Quando possível, dar preferência a colaborar com projetos existentes em detrimento de iniciar e desenvolver novos
  \item Publicar o software em jornais específicos para software e ferramentas, exemplos: JOSS, JORS e SoftwareX.
\end{itemize}

Recomendações aos usuários de software acadêmico:

\begin{itemize}
  \item Sempre indicar a versão do software utilizado na pesquisa.
  \item Quando possível fazer uso de citação formal ao mencionar software acadêmico, verique se o software fornece sugestão de como ser citado.
%  \item ao implementar provas de conceitos de novos algoritmos em estudos avaliando e comparando ferramentas enviar quando possível contribuição aos projetos utilizados
%* evitar manter guardado qualquer código implementado, mesmo que pareça inicialmente não útil para outros
\end{itemize}

Lições, a revisão de literatura realizada para encontrar menções a software
representa um alto grau de dificuldade e trabalho ``braçal'' uma vez que uma
revisão deste tipo seria simplificada muitissimo caso o software fosse citado
através de citação formal, uma vez que desta forma existem inúmeras soluções para
busca, metadados, ligações entre a literatura, etc... mas o fato de não haver
prática padronizada sobre como citar software, descobrir na literatura quando
umm software é citado é um trabalho de certa forma desencorajador, ...

%entre os projetos usando licenças de software livre há um maior número de menções (14\% mais menções que os demais projetos
%FALAR DE DCD aqui? ver últimas frases do RESUMO.

%O ecossistema de software, assim como nos sistemas naturais, necessita de
%fornecimento constante de energia para se manter estável e sustentável, notamos
%no entando que o ecossistema de software acadêmico de análise estática carece
%de investimento, especialmente em termos de contribuição de código fonte, onde


%O desenvolvimento de software sustentável tem sido identificado como um desafio chave no
%campo da Ciência e da Engenharia Computacional. Se sustentabilidade não for levada em
%consideração em projetos de software, não importa qual o domı́nio ou qual o propósito do
%software, perde-se a oportunidade de causar mudanças positivas no planeta e na sociedade
%(BECKER et al., 2014).

%, não sabemos no entanto qual a relação de causa e
%efeito em relação a este maior número de menções aos projetos usando licenças
%livres.

%Alguns projetos de software acadêmico evoluem de maneira independente
%de qualquer atividade acadêmica, com atualizações constantes e lançamentos de
%novas versões ao longo do tempo mas sem nenhuma menção recente na literatura
%acadêmica a respeito do software, 78\% dos projetos estão em estágio inicial de
%desenvolvimento ({\it Initial development}), encerrados ({\it Closedown}) ou
%sendo encerrados ({\it Phaseout}), apenas 13\% dos projetos estão em evolução
%({\it Evolution}) ou provendo serviços ({\it Servicing}).
%
%O tamanho médio do software acadêmico de análise estática em termos de número
%de módulos em estágio {\it Evolution} e {\it Servicing} é duas vezes maior que
%os projetos em {\it Initial development}, a evolução dos projetos mostra um crescimento
%constante no número de módulos no código fonte, confirmando a lei de Lehman de ``Crescimento Contínuo'' do software
%\cite{lehman1997metrics}.

%constantes. Sendo considerados como bons candidados a projetos úteis para uso
%em outras pesquisas.

%%%%%%%%%%%%%%%%%%%%%%%%%%%%%%%%%%%%%%%%%%%%%%%%%%%%

%... 19\%, ou seja, 80 artigos mencionando Uso ou Contribuição a software
%acadêmico de análise estática publicados nas conferências ASE e SCAM não podem
%ser replicados ou reproduzidos.

%apresentam características
%de terem sido desenvolvidos sem preocupação com a sustentabilidade técnica

%mantidos de forma tecnicamente sustentável, ou seja, não são disponíveis para
%obtenção, estão em estágio inicial de desenvolviment ou foram encerrados.

%Observamos que 78\% dos projetos de software acadêmico de análise estática
%encontram-se em estágio inicial de desenvolvimento ({\it Initial development})
%sendo encerrados ({\it Phaseout}) ou já encerrados ({\it Closedown}). Indicando
%que temos neste domínio muitos projetos sem atividade de evolução após a sua
%criação e publicação.

%9\% não foram caracterizados em termos de ciclo de vida por falta de informaçao e 

%Os projetos em estágio inicial apesar de não serem considerados projetos em
%estágio de serem utilizados em outras pesquisa são bons candidados a serem
%adotado em outras pesquisa como ponto de partida inicial para a adaptação e
%atender novas demandas.

%O código fonte de 206 versões foram analisados e os projetos foram
%caracterizados em estágios de evolução, entre {\it Initial development}, {\it
%Evolution}, {\it Servicing}, {\it Phaseout} ou {\it Closedown}.

%, este
%crescimento reflete o papel cada vez maior que o software possui na Ciência,
%mas apesar disso há ainda uma porção de software sem qualquer reconhecimento na
%literatura acadêmica, 21\% dos projetos não possuem qualquer menção além do
%artigo inicial onde foram publicados originalmente, 43\% dos projetos são
%utilizados em artigos além da publicação inicial e apenas 28\% recebem
%contribuição em código fonte.

%Uma avaliação sobre o nível de reconhecimento a estes projetos em publicações
%nas bases da ACM e IEEE encontrou 416 artigos mencionando os projetos de
%software acadêmico, entre 199 menções do tipo Cita, 124 do tipo Usa e 106 do
%tipo Contribui.

%Este uso é mencionado na literatura academica por meio de citação formal ou in-
%formal (SMITH; KATZ; NIEMEYER, 2016) e está estreitamente relacionado ao sistema
%econômico de reputação cientı́fica, uma vez que menções causam impacto cientı́fico direto
%tanto na publicação quanto no ecossistema de software acadêmico (KATZ, 2014).
%Este impacto direto geralmente justifica o investimentos de novos recursos no ecos-
%sistema, seja para fins de planejamento, por exemplo, uma retrospectiva para avaliar
%investimentos já realizados ou para fins de promoção e evolução do software acadêmico
%(HOWISON et al., 2015).

%com a identificação
%do nome do projeto e endereço URL para download do software, 60\% dos projetos
%de software acadêmico publicados nestes artigos encontram-se disponíveis para
%download, apenas 56\% possuem código fonte disponível publicamente, 35\%
%utilizam licenças de software livre.

%Esses artigos publicam 60 projetos de software acadêmico de
%análise estática, dentre os quais, 

%Entre 1873 artigos publicados nas conferências de Engenharia de Software ASE e
%SCAM até o ano de 2015 apenas 61 artigos publicam software acadêmico de análise
%estática publicizados minimamente com identificação do nome do projeto e URL
%para obtenção.


\section{Contribuições}

Esta pesquisa contribuiu caracterizando o software acadêmico do domínio de
análise estática publicado nas conferências de Engenharia de Software ASE e
SCAM trazendo um mapa dos projetos disponíveis e do grau de evolução que
se encontram no ciclo de vida.

Abrindo caminho para a compreensão dos problemas em relação a sustentabilidade de
software acadêmico de análise estática e a partir daí traçar estratégias para
solucionar e melhorar o campo tanto em termos práticos quanto teóricos.

O conhecimento a respeito dos projetos de software acadêmico de análise
existentes serve aos interessados em utilizar tais projetos em novas pesquisas,
seja como objeto de estudo, como apoio metodológico, ou ainda, como base para
iniciar novos desenvolvimentos.

%especialmente entre os projetos em estágio inicial, que apresentam em alguns casos
%indícios de estarem abandonados mas o código fonte deles possui grande potencial
%de ser útil e reduzir o caminho tomado em novas pesquisas.

Esta pesquisa contribui ainda com uma auto-reflexão sobre o campo de análise
estática e seus projetos de software acadêmico, especialmente em relação ao
esforço e recursos investidos no desenvolvimento de software neste domínio de
aplicação sendo consumidos de maneira ineficiente.

%, para que
%a partir daí, como exercício resolver estes problemas e reduzir duplicação de esforço e retrabalho.

No campo teórico, contribui para uma melhor compreensão do que vem a ser
sustentabilidade de software, um tema que ainda carece de definição clara,
especialmente sustentabilidade de software acadêmico de análise estática.

%além de contribui para
%uma definição teórica e prática sobre ecossistema de software acadêmico de análise estática.

Por fim, contribui alertando para os prejuízos que a indisponibilidade de
código destes projetos causam para a Ciência, uma vez que acesso ao código é
parte fundamental para validar ou refutar conclusões de pesquisas através da
reprodução e replicação.

%ou reproduir estudos do passado é uma excelente forma, tanto de validar ou refutar conclusões,
%quanto de contribuir evoluindo as pesquisa e os dados, além de ser tb uma oportunidade
%de evoluir os próprios códigos.

%6 ...  A caracterização ... contribuir para a sustentabilidade do software acadêmico
%de análise estática.

\section{Trabalhos futuros}

O trabalho mais imediato para extender esta pesquisa seria atualizar a revisão
de literatura inicial para extender o período de seleção de software acadêmico
de análise estática incluindo os anos de 2016 e 2017, uma vez que esta
dissertação limitou a seleção ao ano de 2015.

Outro trabalho importante é ampliar o escopo do estudo incluindo mais
conferências visando aumentar o realismo sobre o domínio estudado,
especialmente conferências tradicionais de Engenharia de Software, como por
exemplo, a conferência ICSE (International Conference on Software
Engineering)\footnote{\url{http://www.icse-conferences.org}} por ser uma das
mais tradicionais e importantes conferências da área de Engenharia de Software.
Consideramos também ser importante incluir o congresso CBSOFT (Congresso
Brasileiro de Software: Teoria e Prática)\footnote{\url{http://www.cbsoft.org}}
por ser uma importante conferência no contexto Brasileiro de Engenharia de
Software, com longo histórico, e trilhas de publicação de ferramentas.

Outro ponto em relação a conferências é coletar o fator de impacto da
conferência e utilizar esta informação na análise dos dados e discussão, é
possível que conferências com maior fator de impacto apresentem projetos com
maior reconhecimento e com características de publicização mais adequadas.

Além de ampliar o escopo de forma horizontal incluindo novas conferências, é
interessante também aumentar a qualidade da caracterização do software
incluindo novas dimensões de caracterização, especialmente dimensões na visão
de usuário, como por exemplo, documentação, facilidade de instalaçao, execução,
existência de manuais, avaliar o nível de descrição e apresentação do software
no site ou repositório, ou mesmo incluir aspectos de Engenharia de Software,
como por exemplo, testes, resolução de {\it issues}, métricas de qualidade como
por exemplo complexidade estrutural, número de contribuidores, número de
commits e uso de controle de versão.

%O autor do software provê informações sobre como citar o software apropriadmente? \cite{allen2017engineering}

Uma outra forma de enriquecer a caracterização de cada software é usar
caracterizações já realizadas em outros trabalhos científicos, muitos estudos
avaliam e comparam ferramentas de análise estática entre sí, os artigos
encontrados na seleção de software acadêmico são os primeiros candidatos a
serem utilizados como fonte de coleta de dados sobre cada software.

%apenas caracterizaçõas mas também avaliações, ou seja, entrar no conteúdo dos
%artigos encontrados sobre cada software, e usar as informações já
%caracterizadas para compor uma caracterização mais rica do conjunto de
%ferramentas.

%incluir outras bases na primeira revisão (estruturada) realizada para
%ver se outras ferramentas importantes não ficaram de for.

Outra possibilidade de trabalho futuro é revisar artigos publicados em jornais
com foco específico em publicação de software em busca mais projetos de
software acadêmico de análise estática, entre estes jornais podemos citar, por
exemplo, o JORS (Journal of Open Research
Software)\footnote{\url{https://openresearchsoftware.metajnl.com/articles/10.5334/jors.bt}},
JOSS (Journal of Open Source Software)\footnote{\url{http://joss.theoj.org}} e
o SoftwareX\footnote{\url{https://www.journals.elsevier.com/softwarex}} da
Elsevier. 

%Incluir e avaliar ferramentas publicadas e apresentadas no evento de
%ferramentas da UFBA.
%
%Medir a usabilidade do software acadêmico de análise estática e como podem ser
%mais úteis ao desenvolvedor de modo geral, especialmente para o cientista
%desenvolvedor de software.
%
%Investigar o impacto da sustentabilidade técnica do software acadêmico na
%reprodutibilidade dos estudos que fazem uso do software como apoio metodológico
%(coleta ou análise).
