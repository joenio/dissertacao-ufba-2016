\xchapter{Conclusões}
{}
\label{conclusoes}

(pendente)

%  Trabalhos futuros:
%  * usar mais dimensões na caracterização, ver issue #18
%  * usar caracterizacao ja realizadas em outros trabalhos, ver issue #22
% * 1. incluir outras bases na primeira revisão (estruturada) realizada para
%      ver se outras ferramentas importantes não ficaram de for.
% * 2. atualizar revisões, incluindo ferramentas encontradas em 1, até o ano 2017. 
% * incluir SANER http://saner.unimol.it/index
%    o congresso é A2 = até 20 de out
%    o journal é A2 = até 26 de jan

% Trabalhos relacionados:
% Esta definição pode ser encontrada, com algumas variações, pelos nomes de
% {\it research tool} \cite{Portillo12},
% {\it research-originated software} \cite{Kon2011},
% {\it research software} \cite{hettrick_2014_14809} ou
% {\it scientific software} \cite{segal2008developing},

%Este paper
%apresenta um conjunto de boas práticas que todo pesquisador pode adotar,
%independentemente do seu nível de habilidade em computação. Essas práticas
%passam por gerenciamento de dados, programação, colaboração com colegas,
%organização de projetos, tracking work, e escrita da manuscritos, sao
%desenhados para uma grande variadade de fontes publicadas do noso dia a dia e
%do nosso trabalho como voluntário organizando workshopts desde 2010
%\cite{wilson2017good}.

%\item FAIR principles \cite{wilkinson2016fair}\footnote{\url{https://www.nature.com/articles/sdata201618}}
%Foco em dados de pesquisa. O objetivo é fazer eles serem encontráveis,
%acessíveis, interoperável e reusável. Estes princípios podem ser
%generalizados para aplicar aos softwares.

%\item Open Science Peer Review Oath\footnote{\url{https://f1000research.com/articles/3-271/v2}}
%Concentra-se em potencializar os revisores para exigir acesso aberto aos
%softwares, práticas reprodutíveis e revisões transparentes.

%\item Open Access Pledge \cite{holcombe2011openaccess}\footnote{\url{http://www.openaccesspledge.com}}
%Concentra-se em publicar softwares e papers em locais de {\it open access}.

%Best Practices for Scientific Computing \cite{wilson2014best}
%resume as melhores práticas para melhorar a situação onde softwares
%academisoc sofrem de manutenabilidade, disponibiliade etc, boas praticas, etc

%complemento do artigo acima: 
%Good enough practices in scientific computing \cite{wilson2017good}

%Software Carpentry: lessons learned \cite{wilson2014software}
%(mais uma iniciativa preocupada com as habilidades dos pesquisadores
%com computacao, esta dificuldae gera pesquisas dificeis de reproduzir,
%repeticao de trabalho, etc.. licoes aprendidas ao longo de mais de 20 anos)

% ao falar de ecossistema seria bom citar o evento da ufba de tools onde o analizo já foi apresentado

% Academic Software Development Tools and Techniques
% resumo de um evento local para apresentacao de ferramentas academicas criadas com OO
% (esse é uma prática para incentivar colaboração e promover os projetos)
% uma das ferramentas que enontrei esta apresentado nesse paper, Rigi

%https://cos.io/about/news/center-open-science-receives-grant-james-s-mcdonnell-foundation-study-impact-registered-reports/
%The Center for Open Science (COS) is pleased to announce that it has received a
%$165,591 grant from the James S. McDonnell Foundation to undertake two studies
%evaluating the impact of Registered Reports (RRs) on research quality and
%outcomes. RRs were introduced in 2013 as an innovative method for improving
%reproducibility

%\subsection{Reproducibilidade do estudo}
