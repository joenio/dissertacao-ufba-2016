\xchapter{Introdução}{}

%\section{Apresentação}

<<<<<<< HEAD:capitulos/myintro.tex
% Em diversas linhas de pesquisa da computação, em especial, a engenharia de software, é muito comum que novos softwares (não creio ser recomendável colocar software no plural, pois é um termo que vem do inglês e representa coletivo) sejam desenvolvidos durante trabalhos de pesquisa.
%Esses tem sido chamados na literatura de {\it research-originated software} \cite{Kon2011}, {\it research tool} \cite{Portillo12} ou {\it academic software} \cite{allen2017engineering}, e vêm ganhando atenção da comunidade devido ao papel que ocupam na reprodutibilidade de seus estudos \cite{Peng2011}.

Em diversas linhas de pesquisa da Ciência da Computação, é prática comum que algum tipo de software seja desenvolvido para apoiar a pesquisa em andamento ou mesmo como principal resultado da pesquisa.
Falar MAIS sobre o contexto   ... Por exemplo, um pesquisador pode desenvolver ...
... exemplo,  ... Outros pesquisadores podem usar tal software em suas pesquisas ... outros podem usá-lo para comparar resultados com seus próprios resultados, etc ...

Software acadêmico pode ser definido como o software originado e associado ao desenvolvimento da pesquisa científica -- também referenciado na literatura como {\it research-originated software}~\cite{Kon2011}, {\it research tool}~\cite{Portillo12} ou {\it academic software}~\cite{allen2017engineering}. 

A comunidade tem refletido sobre os problemas relacionados ao
desenvolvimento, promoção e sustentabilidade desses softwares, e o
impacto que tais problemas causam no meio científico \cite{allen2017engineering}. Esta
reflexão tem mostrado, por exemplo, que muitos estudos em engenharia de
software sofrem de dificuldades de repetição \cite{Tang2016}, e apontam
problemas específicos relacionados à manutenabilidade e a sustentabilidade
técnica dos softwares acadêmicos.

Manutenabilidade é uma característica de qualidade que indica o quão fácil é
realizar atividades de evolução e manutenção em softwares, um aspecto
importante aos pesquisadores interessados em adaptar softwares acadêmicos, algo
muitas vezes necessário ao reproduzir pesquisas anteriores \cite{Peng2011}.
Sustentabilidade técnica diz respeito a longevidade dos softwares, ou seja, a
capacidade de continuar disponível no futuro. Muitos pesquisadores não
disponibilizam os seus softwares \cite{robles2010replicating,
amann2015software} ou quanto o fazem enfrentam problemas com disponibilidade e
manutenabilidade \cite{Prlic2012}, isto leva a um corpo computacional
extramente difícil de reproduzir uma vez que mais da metade dos pesquisadores
desenvolvem seus próprios softwares \cite{hettrick_2014_14809}, além de ferir um dos
fundamentos da ciência de que novas descobertas sejam reproduzidas antes de
serem consideradas parte da base de conhecimento \cite{Stodden2009}.

Isto tem motivado a organização de conferências específicas sobre o tema, como
o RSE\footnote{Conference of Research Software Engineers
\url{http://rse.ac.uk/conf2017}}, WSSSPE\footnote{Workshop on Sustainable
Software for Science: Practice and Experiences
\url{http://wssspe.researchcomputing.org.uk}} e o RESER\footnote{Workshop on
Replication in Empirical Software Engineering Research
\url{http://sequoia.cs.byu.edu/reser}}, e tem contribuido para a compreensão
dos problemas relacionados aos softwares acadêmicos, abordando questões sobre
desenvolvimento, qualidade e sustentabiliade, sobre como citar softwares em
novas pesquisas, como promover e reconhecer o papel do pesquisador engenheiro
desenvolvedor de softwares acadêmicos, além de questões sobre infraestrutura,
ferramentas e práticas para o desenvolvimento de softwares acadêmicos de
forma sustentável.

Mas apesar desta crescente preocupação com os softwares acadêmicos ainda
sabe-se pouco sobre o quanto a sustentabilidade técnica e a manutenabilidade
impactam na reprodutibilidade de seus estudos, sobretudo em áreas específicas,
como a análise estática de software, uma área com uma longa e respeitável
tradição e que ainda sofre carência de estudos sobre avaliação e validação de
seus softwares \cite{Li2010, ilyas2016static}.

% (2) Tools to support systematic literature reviews in software engineering: A mapping study \cite{marshall2013tools}
%
% Cita um mapeamento feito sobre estudos que criam ferramentas para apoio a
% revisão sistemática no domínio de SE, 14 estudos foram selecionados, ao final
% apenas 8 tinham proposta de ferramentas, ao final conclui que as ferramentas
% encontradas estão em estado inicial de desenvolvimento. 
%
% (3) Tools used in Global Software Engineering: A systematic mapping review \cite{Portillo12}
%
% Cita um mapeamento sistemático com objetivo de encontrar ferramentas de
% comunicação e coordenação para suporte a times altamente distribuidos
% gograficamente, encontrou 132 ferramentas, para uso em projetos de software
% global. A maioria destas ferramentas foram desenvolvidas em centros de
% pesquisas, e apenas uma pequena porcentagem (18.9\%) foram testados fora do
% seu contexto onde foi desenvolvido.
%
% (5) Tools in mining software repositories \cite{chaturvedi2013tools}
%
% Faz uma revisão dos papers submetidos ao MSR desde 2007 até 2013 (?) e
% identifica data sets, ferramentas e técnicas utilizadas pelos autores, mais
% da metade dos papers usam ou criam ferramentas, categoriza as ferramentas em
% ferramentas novas, ferramentas tradicionais, protótipos e scripts para
% mineração de dados
%
% (6) A systematic literature review of software product line management tools \cite{pereira2015systematic}
%
% (???)
%
% (7) Software configuration management tools \cite{chan1997software}
%
% (???)
%
% (8) Comparison and evaluation of source code mining tools and techniques: A qualitative approach \cite{khatoon2013comparison}
%
% Lista ferramentas e técnicas para mineração de dados, estado da arte.
%
% (9) An overview of free software tools for general data mining \cite{jovic2014overview}
%
% Descreve característica dos 6 softwares livres mais usados para mineração de
% dados no geral.
%
% (10) Analyzing the State of Static Analysis: A Large-Scale Evaluation in Open Source Software \cite{beller2016analyzing}
%
% faz um estudo mostrando que analise estatica tem uma certa adocao em projetos livres
% e mostra onde pode-se melhorar nas ferramentas para aumentar a adoção
%
% Taming the Static Analysis Beast
% \cite{toman2017taming}
% Despite advances in tooling and mainstream success, static analysis development is still a
% painful process.

Questão de Pesquisa: 
Quão sustentável é o software acadêmico de análise estática?
Qual a relação entre sustentabilidade e reproducibilidade?

\section{Objetivos}

O objetivo geral deste trabalho é caracterizar a relação entre sustentabilidade e reproducibilidade no contexto de software acadêmico e, mais especificamente, software de análise estática. 

%avaliar o quanto a sustentabilidade técnica e a manutenabilidade 
%de software acadêmico de análise estática impactam na reprodutibilidade de seus estudos.

%Entre os objetivos da pesquisa, pretende-se:
São objetivos específicos deste trabalho:
\begin{description}
  \item[O1] Caracterizar o software acadêmico de análise estática com respeito à sua sustentatibilidade técnica.
A caracterização será feita em um conjunto de software acadêmico de análise estática, com base em medidas para avaliar
sua sustentabilidade técnica e manutenabilidade.
%  \item[O2] Caracterizar o software acadêmico de análise estática com respeito à reproducibilidade de estudos que o utilizam.
  \item[O3] Caracterizar o software acadêmico de análise estática com respeito à sua manutenibilidade.
A caracterização será feita em um conjunto de software acadêmico de análise estática, com base em uma análise de trabalhos científicos que o utiliza ou adapta.
  \item[04] Avaliar a relação sustentabilidade e reproducibilidade para software acadêmico de análise estática.
% e avaliamos o quanto essas medidas impactam na reprodutibilidade das pesquisas onde os softwares foram criados
\end{description}
=======
Software acadêmico\footnote{{\it research software}, {\it academic software},
{\it academic research software}} é todo software usado ou produzido para
coletar, processar ou analisar resultados em trabalhos publicados na literatura
acadêmica (seja em jornal, revista, conferência, monografia, livro, dissertação
ou tese), podem ser desde pequenos scripts ou protótipos até softwares
completos desenvolvidos profissionalmente.

Boa parte destes softwares são desenvolvidos na própria academia, um estudo
entre cientistas do reino unido, por exemplo, mostrou que 56\% dos
pesquisadores desenvolvem seus próprios softwares, pelo menos pacialmente
\cite{hettrick_2014_14809}, em outras áreas, como na astronomia, este número
chega a 90\% \cite{momcheva2015software}, em ciência da computação,
particularmente em engenharia de software, tem-se notado um aumento constante
no número de novos softwares acadêmicos \cite{allen2017engineering}.

Cientistas gastam mais tempo hoje utilizando e desenvolvendo softwares do que
gastavam no passado, softwares acadêmicos, assim como qualquer outro aparato
experimental, são tão importantes para a ciência quanto são os telescópios ou
tubos de ensaio, eles resolvem problemas comuns do cotidiano de metade dos
pesquisadores de todas as áreas do conhecimento, desde grupos trabalhando
exclusivamente com problemas computacionais até grupos em laboratórios
tradicionais ou em campo \cite{wilson2014best}.

Apesar disso, softwares acadêmicos ainda não recebem o devido reconhecimento,
muitas pesquisas nem ao menos mencionam sua utilização, um estudo recente com
90 artigos de diversas áreas da biologia, selecionados aleatoriamente entre
publicações usando softwares como método, mostrou que apenas 59 mencionavam o
uso de softwares de alguma forma, os demais 31 artigos, apesar de usar software
acadêmico, não mencionavam nada a respeito \cite{howison2016software}.

Isto gera um impacto negativo na visibilidade dos softwares acadêmicos e faz
surgir questionamentos sobre a sua qualidade, não apenas técnica, mas também a
capacidade de ser encontrado, compartilhado e co-desenvolvido, qualidades
importantes para a evolução do próprio software, mas também extremamente útil
para um uso eficiente dos limitados recursos da ciência \cite{howison2013,
katz2014transitive}.

No entanto, parece ser regra geral não testar ou não documentar o próprio
software, pesquisadores geralmente não testam ou documentam seus softwares
acadêmicos, a maioria também não sabe o quão confiável seu software é,
ocasionando graves erros em conclusões centrais da literatura,
gerando retrabalho nas mais diversas áreas da ciência \cite{Merali2010Computational},
apesar de nem sempre ser possível, ou viável, ter tudo dentro de
padrões estritos, é preciso estar consciente das boas práticas ao
produzir e utilizar softwares acadêmicos, tanto para melhorar a própria
abordagem quanto para revisar outros trabalhos \cite{wilson2014best}.

A maior parte dos cientistas (90\%) no entretanto nunca tiveram treinamento
algum de como escrever software de forma eficiente, faltam práticas básicas de
desenvolvimento, como escrever código legível, revisão de código, controle de
versão, testes unitários, entre outros, como resultado, dados são perdidos,
análises levam mais tempo que o necessário e os pesquisadores não conseguem a
eficiência que poderiam ter ao trabalhar com softwares acadêmicos
\cite{wilson2017good}.

Isto contradiz as boas práticas de qualquer projeto experimental ({\it
laboratory
notebooks}\footnote{\url{https://en.wikipedia.org/wiki/Lab_notebook}}, dados
organizados, passos documentados, projeto estruturado para reprodutibilidade) e
torna praticamente impossível utilizar o método mais comum e cientificamente
produtivo de produzir conhecimento novo a partir de pesquisas anteriores, a
replicação, ou seja, seguir os mesmos passos do autor original com objetivo de
validar, melhorar ou estender seus dados e sua metodologia
\cite{king1995replication, Stodden2010}.

Somado a isto temos ainda o fato de que pesquisadores raramente publicam seus
códigos \cite{robles2010replicating, amann2015software}, piorando ainda mais toda a situação, isto tem motivado a organização
de conferências específicas para discutir os problemas dos softwares
acadêmicos, como o RSE (Conference of Research Software Engineers)\footnote{
\url{http://rse.ac.uk/conf2017}}, WSSSPE (Workshop on Sustainable Software for
Science: Practice and Experiences)\footnote{
\url{http://wssspe.researchcomputing.org.uk}} e o RESER (Workshop on
Replication in Empirical Software Engineering Research)\footnote{
\url{http://sequoia.cs.byu.edu/reser}}.

...(enumerar propostas de solução aqui, muitas fazem isso, outras fazem aquilo)...

%, e tem agregado discussões das
%comunidades de ciência aberta, reprodutibilidade e sustentabilidade de
%software.

No entando, ainda não se sabe ao certo como os engenheiros de software
desenvolvem e usam seus próprios softwares acadêmicos, um conhecimento necessário
para avaliar a necessidade de melhorias nas práticas atuais de desenvolvimento
e para tomar decisões sobre a futura alocação de recursos
\cite{hannay2009scientists}.

%e qual qualidade esses softwares apresentam ao longo do tempo.

Dessa forma, definimos como objetivo geral deste trabalho medir a qualidade
(técnica e não-técnica) dos softwares acadêmicos de origem científica da
engenharia de software e explorar como essas taxas mudam ao longo do tempo.

%* Como ocorre o co-desenvolvimento dos softwares
%* Como acontece colaboração na construção dos softwares
%* Como os softwares contribuem para a construcao de conhecimento novo em novas pesquisas derivadas
%
% * mais da metade desenvolvem seus próprios softwares
% * falta de visibilidade gera questionamentos sobre qualidade
% * falta de treinamento leva a produzir softwares sem qualidade
% * produtividade científica requer capacidade de replicação
% * capacidade de replicação depende de qualidade
%
%Ainda assim, poucos estudos tem focado sua atenção nos softwares acadêmicos de
%origem científica\footnote{\it research-originated software \cite{Kon2011}},
%especialmente na engenharia de software, uma área com um potencial inato para a
%criação de novos softwares, como se pode notar em áreas como a análise estática
%de programas, uma área com uma longa e respeitável tradição, e com um constante
%crescimento no número de ferramentas publicadas ao longo do tempo \cite{Li2010,
%ilyas2016static}.
%
%Estes softwares, fazem parte do método empregado em suas pesquisas, ao mesmo
%tempo, são também artefatos produzidos como resultado pelos seus autores, e em
%muitos casos são também a contribuição ou resultado principal de uma
%determinada pesquisa. Além da já citada importancia dos softwares acadêmicos na
%capacidade de reproduzir o passo a passo do autor original numa descoberta, os
%softwares acadêmicos de origem científica publicado como resultado pelo seus
%autores possuem uma dupla importância neste cenário.
%
%Eles passam a fazer parte do conjunto de softwares acadêmicos disponíveis
%para a comunidade acadêmica, sendo
%
%+ software acadêmico: software para coleta e análise, ou resultado
%    + de origem científica
%        + resultado
%        + método
%Software is a critical part of modern research and yet there is little support across the
%scholarly ecosystem for its acknowledgement and citation. Inspired by the activities
%of the FORCE11 working group focused on data citation, this document
%summarizes the recommendations of the FORCE11 Software Citation Working
%Group and its activities between June 2015 and April 2016. Based on a review of
%existing community practices, the goal of the working group was to produce a
%consolidated set of citation principles that may encourage broad adoption of a
%consistent policy for software citation across disciplines and venues. Our work is
%presented here as a set of software citation principles, a discussion of the motivations
%for developing the principles, reviews of existing community practice, and a
%discussion of the requirements these principles would place upon different
%stakeholders. Working examples and possible technical solutions for how these
%principles can be implemented will be discussed in a separate paper.
%
%What an identifier should resolve to
%While citing an identifier that points to, e.g., a GitHub repository can satisfy the principles
%of Unique Identification (3), Accessibility (5), and Specificity (6), such a repository
%cannot guarantee Persistence (4).
%
%Access to software
%The Accessibility principle (5) states that “software citations should permit and facilitate
%access to the software itself.” This does not mean that the software must be freely
%available. Rather, the metadata should provide enough information that the software
%can be accessed. If the software is free, the metadata will likely provide an identifier
%that can be resolved to a URL pointing to the specific version of the software being cited.
%For commercial software, the metadata should still provide information on how to
%access the specific software, but this may be a company’s product number or a link to a
%website that allows the software be purchased. As stated in the Persistence principle (4),
%we recognize that the software version may no longer be available, but it still should
%be cited along with information about how it was accessed.
%\cite{smith2016software}
>>>>>>> d884612799e758cfe829d0413b30ca915cceb4fa:capitulos/introducao.tex

\section{Metodologia de trabalho}

Nesta dissertação, foi investigado o quanto a sustentabilidade do software
acadêmico de análise estática impacta na reprodutibilidade dos seus estudos.
selecionamos softwares acadêmicos de análise estática, medimos a
sustentabilidade técnica e a manutenabilidade, e avaliamos o quanto essas medidas
impactam na reprodutibilidade das pesquisas onde os softwares foram criados.

\subsection{Seleção}

A seleção de softwares acadêmicos foi realizada através de um procedimento
inspirado na revisão e no mapeamento sistemático de literatura, chamado de
revisão estruturada, composto de atividades para seleção e coleta de
informações sobre softwares acadêmicos de análise estática, essa revisão
avaliou o histórico de publicações de 25 anos da conferência ASE e 15 anos da
conferência SCAM.

As informações coletadas sobre cada software inclui nome, descrição e o
endereço onde obter uma cópia, normalmente página web ou repositório de código
fonte, esses endereços foram verificados para confirmar se os softwares estão,
de fato, disponíveis.

\subsection{Caracterização}
Os softwares disponíveis foram avaliados em relação à disponibilidade de código
fonte e à licença utilizada, essas informações, e as demais coletadas até aqui,
foram distribuídas cronologicamente, e interpretadas numa perspectiva histórica
sobre a sustentabilidade técnica dos softwares acadêmicos de análise estática.

No segundo estudo, os softwares com código fonte disponível foram avaliados em
relação a sua manutenabilidade através da métrica de complexidade estrutural. A
coleta dessa métrica para cada software foi realizada pelo Analizo, uma suíte
de ferramentas para análise de código fonte, e está sendo considerado como um
indicador de manutenabilidade.

Um conjunto de softwares de análise estática da indústria foi incluído nesta
etapa, todos os dados coletados para os softwares acadêmicos foram também
coletados para este novo conjunto. Esses softwares foram então caracterizados em
relação à frequencia de lançamentos, linguagem de programação e o tipo de
entrada suportado.

Todas estas características foram comparadas entre sí, por exemplo, softwares
com maior frequencia de lançamentos, escritos na mesma linguagem de
programação, apresentam maior complexidade estrutural? Eles são da academia ou
da indústria? Softwares da indústria apresentam melhor manutenabilidade do que
os softwares acadêmicos?

Essas perguntas serão respondidas através de uma análise exploratória dos
dados, essa análise apresenta também uma perspectiva evolutiva de alguns
softwares, aqueles com maior frequencia de lançamentos foram selecionados para
esta avaliação.

(continua...)

\subsection{Síntese e Discussão}


\section{Contribuições}

(pendente)

\section{Organização do texto}

O capítulo \ref{fundamentacao} apresenta os fundamentos teóricos necessários
para a compreensão deste trabalho.

O capítulo \ref{sustentabilidade-tecnica} traz um estudo sobre a
sustentabilidade técnica e a disponibilidade dos softwares acadêmicos de
análise estática.

O capítulo \ref{complexidade-ferramentas} descreve um estudo sobre a
manutenabilidade dos softwares acadêmicos de análise estática.

O capítulo \ref{conclusoes} apresenta as considerações finais e discute os
resultados deste trabalho.
