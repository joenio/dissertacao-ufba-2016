\xchapter{Síntese dos Resultados}
{}
\label{sintese}


An so what?

O ecossistema de software acadêmico de análise estática sofre de disfunctional ...?

%Encarar o software acadêmico como a "plataforma" do ecossistema de software.
%Pensar no ecossistema de pesquisa e produção intelectual.

\section{Analizo como objeto de estudo}

Analizo \cite{Terceiro2010Analizo} é um conjunto de ferramentas para análise de código-fonte e
visualização com suporte a múltiplas linguagens de programação, software livre,
extensível e capaz de lidar com código-fonte não mais compilável. A capacidade
de lidar com código-fonte não mais compilável permite analisar código-fonte
com erros de sintaxe, com referências a bibliotecas não mais disponíveis, ou
que usem bibliotecas com mudanças de API.
%
Analizo tem sido utilizada em diversos estudos
desenvolvidos em nosso grupo de pesquisa (seção \ref{trabalhos-analizo}).

\subsection{Uso em trabalhos de pesquisa}
\label{trabalhos-analizo}

Analizo tem sido extensivamente usado por nosso grupo de pesquisa em diversos
estudos, com artigos publicados:

\begin{itemize}

% \item \cite{Amaral2009} usou o grafo de dependencia gerado pelo Analizo para 
%gerar uma matriz de evolução em um estudo de caso com o projeto VLC.

% \item \cite{Costa2009} fez uma comparação entre diferentes estratégias para
%  extração de informação de dependencias entre módulos do código-fonte,
%  resultando no desenvolvimento do Doxyparse - o extrator baseado no Doxygen do
%  Analizo.

  \item \cite{Terceiro2009} usou métricas em um estudo exploratório sobre a
  evolução da complexidade estrutural em projetos de software livre escritos em
  C.

  \item \cite{Morais2009} usou a ferramenta de métricas do Analizo como backend
  para o Kalibro, um software para avaliação e observação de métricas de código-fonte.
  
  \item \cite{Terceiro2010} usou o processamento de histórico de métricas para
  realizar um estudo exploratório sobre a evolução da complexidade estrutural em
  7 projetos de servidor web de diferentes tamanhos.

  \item \cite{Meirelles2010} usou o processamento em lote do Analizo para
  processas o código-fonte de mais de 6000 projetos de software livre do
  repositório Sourceforge.net.

  \item \cite{Meirelles2011} usou o Analizo em um estudo sobre impacto de
  métricas de código-fonte na atratividade de projetos de softwares livres.

  \item \cite{Terceiro2012Understanding} usou o Analizo para investigar fatores
  que influenciam na evolução da complexidade estrutural em projetos de software
  livres.

  \item \cite{Silva2012} usou o Analizo para minerar 16000 revisões de
  repositórios de projetos de software para investigar o potencial de uma nova
  métrica chamada Lack of Concern-based Cohesion.

  \item \cite{Ronaldo2015} utilizou o Analizo para extrair métricas de
  código-fonte de 14 versões do sistema Android e estudar a evoluçao da API e
  seus aplicativos.

\end{itemize}

A maioria destes trabalhos contribuíram com melhorias para o Analizo, fazendo
dele uma ferramenta bastante apropriada para pesquisas envolvendo análise de código-fonte,
sendo útil tanto para pesquisadores trabalhando com análise de código-fonte
quanto para profissionais que precisam analisar seus projetos em busca de
potenciais problemas ou melhorias.

Analizo é software livre, distribuído sob a licença GNU General Public License
versão 3. Seu código-fonte, bem como pacotes binários, manuais e tutoriais
podem ser obtidos em \url{http://www.analizo.org}. Todas as ferramentas são
auto-documentadas e podem ser consultadas como páginas de manual UNIX. Analizo
é escrito em Perl, sua última versão 1.19.1 lançada em 01 de Setembro de 2016
foi a versão utilizada neste estudo.


\section{Questões ...} 


\begin{itemize}

  \item Os projetos sustentáveis tem contribuidores além dos autores iniciais?
  \item Os projetos sustentáveis tem mais contribuidores?
  \item Os artigos de projetos sustentáveis são mais lidos, mais citados?
  \item Os autores de projetos sustentáveis tem mais publicações com o uso de software do que os autores?
  \item Os projetos são mais fáceis de manter?  de usar?
\end{itemize}


\section{Os autores de projetos de software acadêmico com maior sustentabilidade técnica têm mais publicações com o uso de software do que os autores de projetos com menor sustentabilidade?}



