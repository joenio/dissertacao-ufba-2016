
% o conteúdo deste arquivo é gerado automaticamente pelo script
% bin/softwares-summary, favor não editar manualmente

\xchapter{Softwares acadêmicos}{Este capítulo ...}
\label{softwares-summary}

\section{2LS - 2nd order Logic Solving}

Autores:
H. Y. Chen - C. David - D. Kroening - P. Schrammel - B. Wachter.

Análise de terminação para programas C usando resumo interprocedural baseado em modelos
publicado no artigo {\it Synthesising Interprocedural Bit-Precise Termination Proofs (T)}
por H. Y. Chen
no ASE 2015,
disponibilizado em \url{http://svn.cprover.org/wiki/doku.php?id=2ls for program analysis}
como software livre
sob uma licença BSD.

Software com lançamentos ocsaionais,
7 versões lançadas
entre 2015 e 2017,
escrito em C++,
uma busca por citações no {\bf IEEE Xplore} por
\texttt{(('2nd order Logic Solving') AND 2LS)}
e no {\bf ACM} com
\texttt{content.ftsec:(2LS) AND (order) AND (Logic) AND (Solving)}
retornou
37 resultados,
nenhum faz referência ao software.


\section{AccessAnalysis}

Autores:
C. Zoller - A. Schmolitzky.

Cálculo de métricas IGAT e IGAM
publicado no artigo {\it AccessAnalysis: A Tool for Measuring the Appropriateness of Access Modifiers in Java Systems}
por C. Zoller
no SCAM 2012,
disponibilizado em \url{http://accessanalysis.sourceforge.net}
como software livre
sob uma licença EPL.

Software considerado obsoleto,
4 versões lançadas
entre 2010 e 2012,
escrito em Java,
uma busca por citações no {\bf IEEE Xplore} por
\texttt{(AccessAnalysis)}
e no {\bf ACM} com
\texttt{content.ftsec:(+AccessAnalysis +Tool +Java +Modifiers)}
retornou
8 resultados,
apenas um faz referência ao software.

\begin{itemize}
\item 2012: C. Zoller - A. Schmolitzky
\end{itemize}

\section{APIExample}

Autores:
Lijie Wang - Lu Fang - Leye Wang - Ge Li - Bing Xie - Fuqing Yang.

Extração de informações de API Java e documentação automática com exemplos
publicado no artigo {\it APIExample: An effective web search based usage example recommendation system for java APIs}
por Lijie Wang
no ASE 2011,
disponibilizado em \url{http://www.apiexample.com}
mas inacessível em 09/08/2017.

Software sem informações sobre lançamentos ou releases,
uma busca por citações no {\bf IEEE Xplore} por
\texttt{((APIExample) AND Java)}
e no {\bf ACM} com
\texttt{content.ftsec:(+APIExample +Java)}
retornou
16 resultados,
{\bf 3} fazem referência ao software.

\begin{itemize}
\item 2013: Z. Zhu - Y. Zou - Y. Jin - B. Xie
\item 2013: J. E. Montandon - H. Borges - D. Felix - M. T. Valente
\item 2016: H. Yu - W. Song - T. Mine
\end{itemize}

\section{BEG - Bandera environment generator}

Autores:
O. Tkachuk - M. B. Dwyer - C. S. Pasareanu.

Criação automática de ambientes para verificação de modelos Java
publicado no artigo {\it Automated environment generation for software model checking}
por O. Tkachuk
no ASE 2003,
disponibilizado em \url{http://bandera.projects.cs.ksu.edu/}
mas inacessível em 09/08/2017.

Software sem informações sobre lançamentos ou releases,
uma busca por citações no {\bf IEEE Xplore} por
\texttt{(("Bandera environment generator") AND BEG)}
e no {\bf ACM} com
\texttt{content.ftsec:(+"Bandera environment generator" +BEG)}
retornou
10 resultados,
{\bf 8} fazem referência ao software.

\begin{itemize}
\item 2004: M. B. Dwyer - Robby - O. Tkachuk - W. Visser
\item 2006: O. Tkachuk - S. P. Rajan
\item 2006: V. Mutilin
\item 2007: O. Tkachuk - S. P. Rajan
\item 2007: P. Parizek - F. Plasil
\item 2010: P. Parizek - F. Plasil
\item 2010: O. Tkachuk - M. B. Dwyer
\item 2015: F. Siavashi - D. Truscan
\end{itemize}

\section{ccJava - Class-based Crosscutting Language for Java}

Autores:
Ubayashi, Naoyasu - Sakai, Akihiro - Tamai, Tetsuo.

Linguagem orientada a aspectos
publicado no artigo {\it An Aspect-oriented Weaving Mechanism Based on Component and Connector Architecture}
por Ubayashi, Naoyasu
no ASE 2007,
disponibilizado em \url{http://posl.minnie.ai.kyutech.ac.jp/}
mas inacessível em 09/08/2017.

Software sem informações sobre lançamentos ou releases,
uma busca por citações no {\bf IEEE Xplore} por
\texttt{(ccJava)}
e no {\bf ACM} com
\texttt{content.ftsec:(+ccJava)}
retornou
7 resultados,
{\bf 4} fazem referência ao software.

\begin{itemize}
\item 2007: N. Ubayashi - A. Sakai - T. Tamai
\item 2008: N. Ubayashi - Y. Sato - A. Sakai - T. Tamai
\item 2009: N. Ubayashi - H. Akatoki - J. Nomura
\item 2010: N. Ubayashi - J. Nomura - T. Tamai
\end{itemize}

\section{CIVL - Concurrency intermediate verification language}

Autores:
M. Zheng - M. S. Rogers - Z. Luo - M. B. Dwyer - S. F. Siegel.

Framework para verificação de programas concorrentes
publicado no artigo {\it CIVL: Formal Verification of Parallel Programs}
por M. Zheng
no ASE 2015,
disponibilizado em \url{http://vsl.cis.udel.edu/civl/}
como software livre
sob uma licença GPL.

Software com lançamentos frequentes,
36 versões lançadas
entre 2015 e 2017,
escrito em C,
uma busca por citações no {\bf IEEE Xplore} por
\texttt{(('concurrency intermediate verification') AND CIVL)}
e no {\bf ACM} com
\texttt{content.ftsec:(+civl +concurrency +intermediate +verification +language)}
retornou
8 resultados,
{\bf 5} fazem referência ao software.

\begin{itemize}
\item 2015: S. F. Siegel - M. Zheng - Z. Luo - T. K. Zirkel - A. V. Marianiello - J. G. Edenhofner - M. B. Dwyer - M. S. Rogers
\item 2015: G. Gopalakrishnan - J. Sawaya
\item 2015: H. A. L\'{o}pez - E. R. B. Marques - F. Martins - N. Ng - C. Santos - V. T. Vasconcelos - N. Yoshida
\item 2017: M. Jasper - M. Fecke - B. Steffen - M. Schordan - J. Meijer - J. van de Pol - F. Howar - S. F. Siegel
\item 2017: S. F. Siegel
\end{itemize}

\section{CodeBoost}

Autores:
O. S. Bagge - K. T. Kalleberg - M. Haveraaen - E. Visser.

Transformação source-to-source para otimização de programas C++
publicado no artigo {\it Design of the CodeBoost transformation system for domain-specific optimisation of C++ programs}
por O. S. Bagge
no SCAM 2003,
disponibilizado em \url{http://codeboost.org}
como software livre
sob uma licença GPL v2.

Software com lançamentos ocsaionais,
134 versões lançadas
entre 2000 e 2004,
escrito em C,
uma busca por citações no {\bf IEEE Xplore} por
\texttt{((CodeBoost) AND C++)}
e no {\bf ACM} com
\texttt{content.ftsec:(+CodeBoost +"C++" +Tool)}
retornou
25 resultados,
{\bf 14} fazem referência ao software.

\begin{itemize}
\item 2003: R. L\"{a}mmel - E. Visser - J. Visser
\item 2005: M. Schordan - D. Quinlan
\item 2005: E. Visser
\item 2005: M. Mernik - J. Heering - A. M. Sloane
\item 2006: D. Quinlan - M. Schordan - R. Vuduc - Q. Yi
\item 2006: M. Bravenboer - K. T. Kalleberg - R. Vermaas - E. Visser
\item 2008: P. Gottschling - A. Lumsdaine
\item 2009: J. J. Willcock - A. Lumsdaine - D. J. Quinlan
\item 2010: X. Tang - J. J\"{a}rvi
\item 2011: H. Chafi - A. K. Sujeeth - K. J. Brown - H. Lee - A. R. Atreya - K. Olukotun
\item 2011: K. J. Brown - A. K. Sujeeth - H. J. Lee - T. Rompf - H. Chafi - M. Odersky - K. Olukotun
\item 2012: S. Hong - H. Chafi - E. Sedlar - K. Olukotun
\item 2013: A. W. Keep - R. K. Dybvig
\item 2015: M. Scherr - S. Chiba
\end{itemize}

\section{CSL - Composite Symbolic Library}

Autores:
T. Bultan - T. Yavuz-Kahveci.

Verificação de modelos
publicado no artigo {\it Action Language Verifier}
por T. Bultan
no ASE 2001,
disponibilizado em \url{http://www.cs.ucsb.edu/~bultan/composite/}
gratuitamente
sem uma licença definida.

Software sem informações sobre lançamentos ou releases,
escrito em C,
uma busca por citações no {\bf IEEE Xplore} por
\texttt{((((composite) AND bultan) AND Action Language Verifier) OR ("Composite Symbolic Library"))}
e no {\bf ACM} com
\texttt{content.ftsec:(+composite +"Action Language Verifier")}
retornou
14 resultados,
{\bf 5} fazem referência ao software.

\begin{itemize}
\item 2004: X. Fu - T. Bultan - J. Su
\item 2005: D. Zhang - R. Cleaveland
\item 2006: Z. Yang - C. Wang - A. Gupta - F. Ivancic
\item 2006: T. Bultan - C. Heitmeyer
\item 2009: Z. Yang - C. Wang - A. Gupta - F. Ivan\v{c}i\'{c}
\end{itemize}

\section{CPA+ - Configurable program analysis with dynamic precision adjustment}

Autores:
D. Beyer - T. A. Henzinger - G. Theoduloz.

Análise configurável de programa com ajuste dinâmico de precisão
publicado no artigo {\it Program Analysis with Dynamic Precision Adjustment}
por D. Beyer
no ASE 2008,
disponibilizado em \url{http://www.cs.sfu.ca/~dbeyer/blast_cpaplus/}
mas inacessível em 09/08/2017.

Software sem informações sobre lançamentos ou releases,
uma busca por citações no {\bf IEEE Xplore} por
\texttt{(('program analysis') AND cpa+)}
e no {\bf ACM} com
\texttt{content.ftsec:(+cpa +dbeyer)}
retornou
8 resultados,
{\bf 4} fazem referência ao software.

\begin{itemize}
\item 2010: D. Beyer - M. E. Keremoglu - P. Wendler
\item 2012: D. Beyer - T. A. Henzinger - M. E. Keremoglu - P. Wendler
\item 2013: D. Beyer - S. L\"{o}we - E. Novikov - A. Stahlbauer - P. Wendler
\item 2015: D. Beyer - M. Dangl - D. Dietsch - M. Heizmann - A. Stahlbauer
\end{itemize}

\section{CSeq}

Autores:
B. Fischer - O. Inverso - G. Parlato.

Transformação source-to-source para programas C concorrentes
publicado no artigo {\it CSeq: A concurrency pre-processor for sequential C verification tools}
por B. Fischer
no ASE 2013,
disponibilizado em \url{http://users.ecs.soton.ac.uk/gp4/cseq/files/cseq-0.5.zip}
como software livre
sob uma licença BSD.

Software sem informações sobre lançamentos ou releases,
escrito em C,
uma busca por citações no {\bf IEEE Xplore} por
\texttt{((CSeq) AND soton)}
e no {\bf ACM} com
\texttt{content.ftsec:(+cseq +sequential +verification +tool)}
retornou
16 resultados,
{\bf 4} fazem referência ao software.

\begin{itemize}
\item 2015: A. Bouajjani - M. Emmi - C. Enea - J. Hamza
\item 2015: O. Inverso - T. L. Nguyen - B. Fischer - S. L. Torre - G. Parlato
\item 2016: M. Mayer - R. Madhavan
\item 2016: E. Tomasco - T. L. Nguyen - O. Inverso - B. Fischer - S. L. Torre - G. Parlato
\end{itemize}

\section{DDVerify}

Autores:
Witkowski, Thomas - Blanc, Nicolas - Kroening, Daniel - Weissenbacher, Georg.

Verificação de Linux drivers através de checagem de modelos
publicado no artigo {\it Model Checking Concurrent Linux Device Drivers}
por Witkowski, Thomas
no ASE 2007,
disponibilizado em \url{http://www.verify.ethz.ch/ddverify}
mas inacessível em 09/08/2017.

Software sem informações sobre lançamentos ou releases,
uma busca por citações no {\bf IEEE Xplore} por
\texttt{(DDVerify)}
e no {\bf ACM} com
\texttt{content.ftsec:(+DDVerify)}
retornou
4 resultados,
{\bf 2} fazem referência ao software.

\begin{itemize}
\item 2008: N. Blanc - D. Kroening
\item 2014: K. P. Dileep - A. Raghavendra - M. Suman - G. Devesh - S. V. Srikanth
\end{itemize}

\section{Derailer}

Autores:
Near, Joseph P. - Jackson, Daniel.

Localização de falhas de segurança em aplicações web
publicado no artigo {\it Derailer: Interactive Security Analysis for Web Applications}
por Near, Joseph P.
no ASE 2014,
disponibilizado em \url{http://people.csail.mit.edu/jnear/derailer}
como software livre
sob uma licença GPL v3.

Software considerado obsoleto,
2 versões lançadas
entre 2013 e 2014,
escrito em Ruby,
uma busca por citações no {\bf IEEE Xplore} por
\texttt{((Derailer) AND jnear)}
e no {\bf ACM} com
\texttt{content.ftsec:(+Derailer +analysis +security +web +tool +bugs +ruby)}
retornou
7 resultados,
apenas um faz referência ao software.

\begin{itemize}
\item 2016: J. P. Near - D. Jackson
\end{itemize}

\section{Diagnosys}

Autores:
T. F. Bissyandé - L. Réveillère - J. L. Lawall - G. Muller.

Construção de interfaces de debug para o kernel Linux
publicado no artigo {\it Diagnosys: automatic generation of a debugging interface to the Linux kernel}
por T. F. Bissyandé
no ASE 2012,
disponibilizado em \url{http://momentum.labri.fr/projects/diagnosys}
mas inacessível em 09/08/2017.

Software sem informações sobre lançamentos ou releases,
uma busca por citações no {\bf IEEE Xplore} por
\texttt{((Diagnosys) AND debugging)}
e no {\bf ACM} com
\texttt{content.ftsec:(+"Diagnosys tool" +Debugging +Linux +"device drivers")}
retornou
9 resultados,
nenhum faz referência ao software.


\section{DOMPLETION}

Autores:
Bajaj, Kartik - Pattabiraman, Karthik - Mesbah, Ali.

Sugestão de código javascript
publicado no artigo {\it Dompletion: DOM-aware JavaScript Code Completion}
por Bajaj, Kartik
no ASE 2014,
disponibilizado em \url{https://github.com/saltlab/dompletion}
gratuitamente
sem uma licença definida.

Software sem informações sobre lançamentos ou releases,
escrito em Javascript,
uma busca por citações no {\bf IEEE Xplore} por
\texttt{content.ftsec:(+dompletion +JavaScript)}
e no {\bf ACM} com
\texttt{((dompletion) AND JavaScript)}
retornou
2 resultados,
apenas um faz referência ao software.

\begin{itemize}
\item 2015: K. Bajaj - K. Pattabiraman - A. Mesbah
\end{itemize}

\section{DRC - Dangling Reference Checker}

Autores:
H. V. Nguyen - H. A. Nguyen - T. T. Nguyen - A. T. Nguyen - T. N. Nguyen.

Análise estática para detecção de referências inválidas em código dinâmico PHP
publicado no artigo {\it Dangling references in multi-configuration and dynamic PHP-based Web applications}
por H. V. Nguyen
no ASE 2013,
disponibilizado em \url{http://home.engineering.iastate.edu/~hungnv/Research/DRC}
mas inacessível em 09/08/2017.

Software sem informações sobre lançamentos ou releases,
uma busca por citações no {\bf IEEE Xplore} por
\texttt{(DRC Dangling Reference Checker)}
e no {\bf ACM} com
\texttt{content.ftsec:(+DRC +Dangling +Reference)}
retornou
19 resultados,
{\bf 4} fazem referência ao software.

\begin{itemize}
\item 2013: H. V. Nguyen - H. A. Nguyen - T. T. Nguyen - T. N. Nguyen
\item 2014: H. V. Nguyen - C. K\"{a}stner - T. N. Nguyen
\item 2014: L. Eshkevari - G. Antoniol - J. R. Cordy - M. Di Penta
\item 2015: H. V. Nguyen - C. K\"{a}stner - T. N. Nguyen
\end{itemize}

\section{e-munity}

Autores:
S. Tlili - J. M. Fernandez - A. Belghith - B. Dridi - S. Hidouri.

Verificação de segurança
publicado no artigo {\it Scalable Security Verification of Software at Compile Time}
por S. Tlili
no SCAM 2014,
disponibilizado em \url{http://sourceforge.net/p/emunity/code/ci/master/tree/}
gratuitamente
sem uma licença definida.

Software sem informações sobre lançamentos ou releases,
escrito em C,
uma busca por citações no {\bf IEEE Xplore} por
\texttt{(e-munity)}
e no {\bf ACM} com
\texttt{content.ftsec:(+"e-munity")}
retornou
1 resultados,
nenhum faz referência ao software.


\section{EJB Interceptor Analyzer}

Autores:
S. Roubtsov - A. Serebrenik - A. Mazoyer - M. v. d. Brand.

Criação de diagramas de sequência
publicado no artigo {\it I2SD: Reverse Engineering Sequence Diagrams from Enterprise Java Beans with Interceptors}
por S. Roubtsov
no SCAM 2011,
disponibilizado em \url{https://www.dropbox.com/s/glhg8any43lccgm/EJB.zip}
gratuitamente
sem uma licença definida.

Software sem informações sobre lançamentos ou releases,
escrito em Java,
uma busca por citações no {\bf IEEE Xplore} por
\texttt{(((I2SD) AND EJB) AND Java)}
e no {\bf ACM} com
\texttt{content.ftsec:(+I2SD +Java)}
retornou
5 resultados,
{\bf 2} fazem referência ao software.

\begin{itemize}
\item 2013: A. Sutii - S. Roubtsov - A. Serebrenik
\item 2017: Z. Mushtaq - G. Rasool - B. Shehzad
\end{itemize}

\section{Error Prone}

Autores:
E. Aftandilian - R. Sauciuc - S. Priya - S. Krishnan.

Localização de bugs em código Java construído em cima do compilador javac
publicado no artigo {\it Building Useful Program Analysis Tools Using an Extensible Java Compiler}
por E. Aftandilian
no SCAM 2012,
disponibilizado em \url{http://code.google.com/p/error-prone}
como software livre
sob uma licença Apache License v2.0.

Software com lançamentos frequentes,
22 versões lançadas
entre 2015 e 2017,
escrito em Java,
uma busca por citações no {\bf IEEE Xplore} por
\texttt{((((('error-prone tool') AND Analysis) AND 'java compiler') AND 'error checks') AND javac)}
e no {\bf ACM} com
\texttt{content.ftsec:(+"error-prone" +tool +javac +analysis +"java compiler")}
retornou
47 resultados,
apenas um faz referência ao software.

\begin{itemize}
\item 2015: C. Sadowski - J. v. Gogh - C. Jaspan - E. Söderberg - C. Winter
\end{itemize}

\section{ESBMC - Efficient SMT-Based Context-Bounded Model Checker}

Autores:
L. Cordeiro - B. Fischer - J. Marques-Silva.

Verificação de modelos
publicado no artigo {\it SMT-Based Bounded Model Checking for Embedded ANSI-C Software}
por L. Cordeiro
no ASE 2009,
disponibilizado em \url{http://users.ecs.soton.ac.uk/lcc08r/esbmc/}
mas inacessível em 09/08/2017.

Software sem informações sobre lançamentos ou releases,
uma busca por citações no {\bf IEEE Xplore} por
\texttt{(ESBMC)}
e no {\bf ACM} com
\texttt{content.ftsec:(+ESBMC)}
retornou
50 resultados,
{\bf 41} fazem referência ao software.

\begin{itemize}
\item 2010: L. Cordeiro - B. Fischer - J. Marques-Silva
\item 2011: R. Barreto - L. Cordeiro - B. Fischer
\item 2011: J. Behrend - D. Lettnin - P. Heckeler - J. Ruf - T. Kropf - W. Rosenstiel
\item 2011: L. Cordeiro - B. Fischer
\item 2012: P. A. Abdulla - M. F. Atig - O. Rezine - J. Stenman
\item 2012: A. M. Gharehbaghi - M. Fujita
\item 2013: B. Wachter - D. Kroening - J. Ouaknine
\item 2013: C. Y. Cho - V. D'Silva - D. Song
\item 2013: S. Falke - F. Merz - C. Sinz
\item 2013: K. Banerjee - M. S. Prabhu - P. Dasgupta
\item 2013: B. Fischer - O. Inverso - G. Parlato
\item 2013: M. Ramalho - M. Freitas - F. Sousa - H. Marques - L. Cordeiro - B. Fischer
\item 2014: M. Cordy - P. Heymans - A. Legay - P. Schobbens - B. Dawagne - M. Leucker
\item 2014: I. V. d. Bessa - H. I. Ismail - L. C. Cordeiro - J. E. C. Filho
\item 2014: F. A. P. Januario - L. C. Cordeiro - V. F. d. Lucena - E. B. d. L. Filho
\item 2014: P. Thomson - A. F. Donaldson - A. Betts
\item 2014: J. Behrend - A. Gruenhage - D. Schroeder - D. Lettnin - J. Ruf - T. Kropf - W. Rosenstiel
\item 2014: H. G\"{u}nther - G. Weissenbacher
\item 2015: E. H. d. S. Alves - L. C. Cordeiro - E. B. d. L. Filho
\item 2015: O. Inverso - T. L. Nguyen - B. Fischer - S. L. Torre - G. Parlato
\item 2015: A. Trindade - H. Ismail - L. Cordeiro
\item 2015: H. Rocha - H. Ismail - L. Cordeiro - R. Barreto
\item 2015: M. Chabot - K. Mazet - L. Pierre
\item 2015: F. R. M. Sousa - L. C. Cordeiro - E. B. de L. Filho
\item 2015: P. Darke - B. Chimdyalwar - R. Venkatesh - U. Shrotri - R. Metta
\item 2015: A. Gupta - T. A. Henzinger - A. Radhakrishna - R. Samanta - T. Tarrach
\item 2016: D. Beyer - M. Dangl - D. Dietsch - M. Heizmann
\item 2016: X. Xie - B. Chen - Y. Liu - W. Le - X. Li
\item 2016: P. Pereira - H. Albuquerque - H. Marques - I. Silva - C. Carvalho - L. Cordeiro - V. Santos - R. Ferreira
\item 2016: R. Paludo - D. Lettnin
\item 2016: R. Araújo - I. Bessa - L. C. Cordeiro - J. E. C. Filho
\item 2016: F. R. Monteiro
\item 2016: X. Zhang - Z. Yang - Q. Zheng - Y. Hao - P. Liu - L. Yu - M. Fan - T. Liu
\item 2016: D. Kroening - D. Poetzl - P. Schrammel - B. Wachter
\item 2016: L. Bentes - H. Rocha - E. Valentin - R. Barreto
\item 2016: L. C. Cordeiro - E. B. de Lima Filho
\item 2016: M. Gao - L. He - R. Majumdar - Z. Wang
\item 2017: L. Chaves - I. Bessa - L. Cordeiro - D. Kroening - E. Lima
\item 2017: X. Zhang
\item 2017: X. Zhang - Z. Yang - Q. Zheng - P. Liu - J. Chang - Y. Hao - T. Liu
\item 2017: X. Zheng - C. Julien - R. Podorozhny - F. Cassez - T. Rakotoarivelo
\end{itemize}

\section{ETXL}

Autores:
A. D. Thurston - J. R. Cordy.

Transformação de código
publicado no artigo {\it Evolving TXL}
por A. D. Thurston
no SCAM 2006,
disponibilizado em \url{http://www.cs.queensu.ca/home/thurston/etxl}
mas inacessível em 09/08/2017.

Software sem informações sobre lançamentos ou releases,
uma busca por citações no {\bf IEEE Xplore} por
\texttt{(((ETXL) AND source) AND transformation)}
e no {\bf ACM} com
\texttt{content.ftsec:(+ETXL)}
retornou
10 resultados,
nenhum faz referência ao software.


\section{FaultBuster}

Autores:
G. Szőke - C. Nagy - L. J. Fülöp - R. Ferenc - T. Gyimóthy.

Refatoração de code smells
publicado no artigo {\it FaultBuster: An automatic code smell refactoring toolset}
por G. Szőke
no SCAM 2015,
disponibilizado em \url{http://www.sed.inf.u-szeged.hu/FaultBuster}
gratuitamente
sob uma licença de demonstração.

Software sem informações sobre lançamentos ou releases,
uma busca por citações no {\bf IEEE Xplore} por
\texttt{(FaultBuster)}
e no {\bf ACM} com
\texttt{content.ftsec:(+FaultBuster)}
retornou
4 resultados,
nenhum faz referência ao software.


\section{Flowgen}

Autores:
D. A. Kosower - J. J. Lopez-Villarejo - S. Roubtsov.

Criação automática de grafos UML
publicado no artigo {\it Flowgen: Flowchart-Based Documentation Framework for C++}
por D. A. Kosower
no SCAM 2014,
disponibilizado em \url{https://github.com/jlopezvi/Flowgen}
como software livre
sob uma licença GPL v3.

Software sem informações sobre lançamentos ou releases,
escrito em Python,
uma busca por citações no {\bf IEEE Xplore} por
\texttt{((Flowgen) AND C++)}
e no {\bf ACM} com
\texttt{content.ftsec:(+Flowgen)}
retornou
8 resultados,
{\bf 2} fazem referência ao software.

\begin{itemize}
\item 2015: L. Georget - F. Tronel - V. V. T. Tong
\item 2016: M. Moser - J. Pichler
\end{itemize}

\section{GRT - Guided Random Testing}

Autores:
L. Ma - C. Artho - C. Zhang - H. Sato - J. Gmeiner - R. Ramler.

Geração automática de testes
publicado no artigo {\it GRT: Program-Analysis-Guided Random Testing (T)}
por L. Ma
no ASE 2015,
disponibilizado em \url{http://www.sites.google.com/site/grtprojectut/download}
mas inacessível em 09/08/2017.

Software sem informações sobre lançamentos ou releases,
uma busca por citações no {\bf IEEE Xplore} por
\texttt{((GRT) AND "Guided Random Testing")}
e no {\bf ACM} com
\texttt{content.ftsec:(+GRT) +"Guided Random Testing")}
retornou
13 resultados,
{\bf 7} fazem referência ao software.

\begin{itemize}
\item 2014: W. Chiang - G. Gopalakrishnan - Z. Rakamaric - A. Solovyev
\item 2015: L. Ma - C. Artho - C. Zhang - H. Sato - M. Hagiya - Y. Tanabe - M. Yamamoto
\item 2016: L. Ma - C. Zhang - B. Yu - J. Zhao
\item 2016: C. Artho - L. Ma
\item 2017: A. Arcuri - G. Fraser - R. Just
\item 2017: A. Panichella - F. Kifetew - P. Tonella
\item 2017: C. Artho - Q. Gros - G. Rousset - K. Banzai - L. Ma - T. Kitamura - M. Hagiya - Y. Tanabe - M. Yamamoto
\end{itemize}

\section{GUIZMO}

Autores:
S\'{a}nchez Ram\'{o}n, \'{O}scar - S\'{a}nchez Cuadrado, Jes\'{u}s - Garc\'{\i}a Molina, Jes\'{u}s.

Inferência de layout
publicado no artigo {\it Model-driven Reverse Engineering of Legacy Graphical User Interfaces}
por S\'{a}nchez Ram\'{o}n, \'{O}scar
no ASE 2010,
disponibilizado em \url{http://modelum.es/trac/guizmo/}
como software livre
sob uma licença Apache License v2.0.

Software sem informações sobre lançamentos ou releases,
escrito em Java,
uma busca por citações no {\bf IEEE Xplore} por
\texttt{(guizmo)}
e no {\bf ACM} com
\texttt{content.ftsec:(+guizmo)}
retornou
0 resultados,
nenhum faz referência ao software.


\section{GumTree}

Autores:
Falleri, Jean-R{\'e}my - Morandat, Flor{\'e}al - Blanc, Xavier - Martinez, Matias - Monperrus, Martin.

Comparação de mudanças
publicado no artigo {\it Fine-grained and Accurate Source Code Differencing}
por Falleri, Jean-R{\'e}my
no ASE 2014,
disponibilizado em \url{https://github.com/jrfaller/gumtree}
como software livre
sob uma licença LGPL v3.

Software com lançamentos ocsaionais,
3 versões lançadas
entre 2013 e 2015,
escrito em Java,
uma busca por citações no {\bf IEEE Xplore} por
\texttt{((GumTree) AND tool)}
e no {\bf ACM} com
\texttt{content.ftsec:(+GumTree +tool)}
retornou
37 resultados,
{\bf 17} fazem referência ao software.

\begin{itemize}
\item 2015: Q. Gao - H. Zhang - J. Wang - Y. Xiong - L. Zhang - H. Mei
\item 2015: K. Zimmerman - C. R. Rupakheti
\item 2016: P. Kreutzer - G. Dotzler - M. Ring - B. M. Eskofier - M. Philippsen
\item 2016: G. Dotzler - M. Philippsen
\item 2016: Q. Hanam - F. S. de M. Brito - A. Mesbah
\item 2016: I. Ahmed - R. Gopinath - C. Brindescu - A. Groce - C. Jensen
\item 2016: X. B. D. Le - D. Lo - C. L. Goues
\item 2016: A. T. Nguyen - M. Hilton - M. Codoban - H. A. Nguyen - L. Mast - E. Rademacher - T. N. Nguyen - D. Dig
\item 2017: A. Koyuncu - T. F. Bissyand{\'e} - D. Kim - J. Klein - M. Monperrus - Y. Le Traon
\item 2017: Q. D. Soetens - R. Robbes - S. Demeyer
\item 2017: p. rempel - P. Mader
\item 2017: S. H. Tan - J. Yi - Yulis - S. Mechtaev - A. Roychoudhury
\item 2017: M. A. Laverdière - E. Merlo
\item 2017: X. D. Le - D. Chu - D. Lo - C. Le Goues - W. Visser
\item 2017: J. Yi - U. Z. Ahmed - A. Karkare - S. H. Tan - A. Roychoudhury
\item 2017: R. Stevens - C. D. Roover
\item 2017: C. Macho - S. McIntosh - M. Pinzger
\end{itemize}

\section{HUSACCT - HU Software Architecture Compliance Checking Tool}

Autores:
Pruijt, Leo J. - K\"{o}ppe, Christian - van der Werf, Jan Martijn - Brinkkemper, Sjaak.

verificação de conformidade arquitetural
publicado no artigo {\it HUSACCT: Architecture Compliance Checking with Rich Sets of Module and Rule Types}
por Pruijt, Leo J.
no ASE 2014,
disponibilizado em \url{http://husacct.github.io/HUSACCT}
como software livre
sob uma licença AGPL.

Software com lançamentos frequentes,
22 versões lançadas
entre 2013 e 2017,
escrito em Java,
uma busca por citações no {\bf IEEE Xplore} por
\texttt{(HUSACCT)}
e no {\bf ACM} com
\texttt{content.ftsec:(+HUSACCT)}
retornou
7 resultados,
{\bf 6} fazem referência ao software.

\begin{itemize}
\item 2014: L. Pruijt - S. Brinkkemper
\item 2015: L. Pruijt - J. M. E. M. van der Werf
\item 2016: L. Pruijt - W. Wiersema
\item 2016: L. Pruijt - W. Wiersema - J. M. E. M. v. d. Werf - S. Brinkkemper
\item 2016: J. Peters - J. M. E. M. v. d. Werf - J. Hage
\item 2016: J. Peters - J. M. E. M. van der Werf
\end{itemize}

\section{Indus}

Autores:
V. P. Ranganath - J. Hatcliff.

Biblioteca de program slicing
publicado no artigo {\it An Overview of the Indus Framework for Analysis and Slicing of Concurrent Java Software (Keynote Talk - Extended Abstract)}
por V. P. Ranganath
no SCAM 2006,
disponibilizado em \url{http://indus.projects.cis.ksu.edu}
como software livre
sob uma licença EPL v1.0.

Software com lançamentos ?,
36 versões lançadas
entre 2005 e 2010,
escrito em Java,
uma busca por citações no {\bf IEEE Xplore} por
\texttt{("Indus Framework")}
e no {\bf ACM} com
\texttt{content.ftsec:(+"Indus Framework")}
retornou
6 resultados,
{\bf 3} fazem referência ao software.

\begin{itemize}
\item 2009: C. Zhang
\item 2012: J. Huang - C. Zhang
\item 2012: X. Wu - J. Wei - X. Wang
\end{itemize}

\section{JastAdd}

Autores:
Magnusson, Eva - Ekman, Torbjorn - Hedin, Gorel.

Análise de código-fonte através da descrição de atributos via gramática de atributos (AG)
publicado no artigo {\it Extending Attribute Grammars with Collection Attributes--Evaluation and Applications}
por Magnusson, Eva
no SCAM 2007,
disponibilizado em \url{http://jastadd.cs.lth.se/web}
como software livre
sob uma licença BSD License "modified".

Software com lançamentos frequentes,
24 versões lançadas
entre 2011 e 2017,
escrito em Java,
uma busca por citações no {\bf IEEE Xplore} por
\texttt{(JastAdd tool Attribute Grammars)}
e no {\bf ACM} com
\texttt{content.ftsec:(+JastAdd +tool +"Attribute Grammars" +"source code" +analysis)}
retornou
50 resultados,
{\bf 42} fazem referência ao software.

\begin{itemize}
\item 2003: M. G. J. van den Brand - P. Klint - J. J. Vinju
\item 2005: P. Andersson - K. Kuchcinski
\item 2005: H. Wu - J. Gray - S. Roychoudhury - M. Mernik
\item 2005: M. van den Brand - P. E. Moreau - J. Vinju
\item 2005: E. Visser
\item 2006: X. Wu - B. R. Bryant - J. Gray - S. Roychoudhury - M. Mernik
\item 2006: F. Gruian - P. Roop - Z. Salcic - I. Radojevic
\item 2006: F. Gortazar - M. Gallego - A. Duarte
\item 2007: T. Ekman - G. Hedin
\item 2007: J. Malec - A. Nilsson - K. Nilsson - S. Nowaczyk
\item 2007: D. Rebernak - M. Mernik
\item 2008: W. Lohmann - G. Riedewald - G. Wachsmuth
\item 2008: L. C. Kats - M. Bravenboer - E. Visser
\item 2008: T. Ekman - M. Sch\"{a}fer - M. Verbaere
\item 2008: M. Sch\"{a}fer - T. Ekman - O. de Moor
\item 2009: P. Fritzson - A. Pop - D. Broman - P. Aronsson
\item 2009: D. Rebernak - M. Mernik - H. Wu - J. Gray
\item 2010: P. Klint - T. van der Storm - J. Vinju
\item 2010: M. Pise
\item 2010: S. Markstrum - D. Marino - M. Esquivel - T. Millstein - C. Andreae - J. Noble
\item 2010: M. Schaefer - O. de Moor
\item 2010: L. Renggli - T. G\^{\i}rba - O. Nierstrasz
\item 2010: S. Sateanpattanakul - A. Walairacht
\item 2011: G. Hedin - J. Akesson - T. Ekman
\item 2011: D. Rodríguez-Cerezo - A. Sarasa-Cabezuelo - J. L. Sierra
\item 2011: S. Erdweg - T. Rendel - C. K\"{a}stner - K. Ostermann
\item 2012: J. Winther
\item 2012: D. Broman - P. Fritzson - G. Hedin - J. Åkesson
\item 2012: N. Fors - G. Hedin
\item 2012: M. de Jonge - E. Visser
\item 2013: G. Soares - R. Gheyi - T. Massoni
\item 2013: P. O. Larsson - F. Casella - F. Magnusson - J. Andersson - M. Diehl - J. Åkesson
\item 2013: J. \"{O}qvist - G. Hedin
\item 2014: K. Williams - M. Le - T. Kaminski - E. V. Wyk
\item 2014: N. Fors - G. Hedin
\item 2015: J. Bransen - A. Dijkstra - S. D. Swierstra
\item 2015: B. Basten - M. Hills - P. Klint - D. Landman - A. Shahi - M. J. Steindorfer - J. J. Vinju
\item 2016: T. Szabó - S. Erdweg - M. Voelter
\item 2016: S. Ryu
\item 2016: R. Muhammad - M. R. A. Setyautami
\item 2017: V. von Hof - K. F\"{o}gen - H. Kuchen
\item 2017: D. Petrashko - O. Lhot\'{a}k - M. Odersky
\end{itemize}

\section{JFlow}

Autores:
N. Chen - R. E. Johnson.

Transformação source-to-source
publicado no artigo {\it JFlow: Practical refactorings for flow-based parallelism}
por N. Chen
no ASE 2013,
disponibilizado em \url{http://vazexqi.github.io/JFlow/}
como software livre
sob uma licença Illinois/NCSA Open Source License.

Software considerado obsoleto,
5 versões lançadas
em 2012,
escrito em Java,
uma busca por citações no {\bf IEEE Xplore} por
\texttt{(JFlow tool Eclipse)}
e no {\bf ACM} com
\texttt{content.ftsec:(+JFlow +tool +Eclipse)}
retornou
16 resultados,
{\bf 6} fazem referência ao software.

\begin{itemize}
\item 2006: B. Hicks - K. Ahmadizadeh - P. McDaniel
\item 2006: C. Hammer - J. Krinke - F. Nodes
\item 2007: B. Hicks - D. King - P. McDaniel
\item 2011: K. Maruyama - T. Omori
\item 2011: J. Yu - S. Zhang - P. Liu - Z. Li
\item 2014: T. Choi - S. Kang - S. Yoon - S. Yang - S. Song - H. Park
\end{itemize}

\section{JstereoCode}

Autores:
L. Moreno - A. Marcus.

Detecção de esteriótipos Java
publicado no artigo {\it JStereoCode: automatically identifying method and class stereotypes in Java code}
por L. Moreno
no ASE 2012,
disponibilizado em \url{http://www.cs.wayne.edu/~severe/revenge/}
mas inacessível em 09/08/2017.

Software sem informações sobre lançamentos ou releases,
uma busca por citações no {\bf IEEE Xplore} por
\texttt{(JstereoCode)}
e no {\bf ACM} com
\texttt{content.ftsec:(+JstereoCode)}
retornou
13 resultados,
{\bf 7} fazem referência ao software.

\begin{itemize}
\item 2013: L. Moreno - J. Aponte - G. Sridhara - A. Marcus - L. Pollock - K. Vijay-Shanker
\item 2013: L. Moreno - A. Marcus - L. Pollock - K. Vijay-Shanker
\item 2013: P. Andras - A. Pakhira - L. Moreno - A. Marcus
\item 2015: B. Duffee - P. Andras
\item 2015: M. Linares-V\'{a}squez - L. F. Cort{\'e}s-Coy - J. Aponte - D. Poshyvanyk
\item 2016: J. Shen - X. Sun - B. Li - H. Yang - J. Hu
\item 2017: P. W. McBurney - S. Jiang - M. Kessentini - N. A. Kraft - A. Armaly - M. W. Mkaouer - C. McMillan
\end{itemize}

\section{Jtop}

Autores:
L. Zhang - J. Zhou - D. Hao - L. Zhang - H. Mei.

Gestão de casos de teste
publicado no artigo {\it Jtop: Managing JUnit Test Cases in Absence of Coverage Information}
por L. Zhang
no ASE 2009,
disponibilizado em \url{http://code.google.com/p/pku-jtop/}
mas inacessível em 09/08/2017.

Software sem informações sobre lançamentos ou releases,
uma busca por citações no {\bf IEEE Xplore} por
\texttt{(Jtop JUnit)}
e no {\bf ACM} com
\texttt{content.ftsec:(+Jtop +JUnit)}
retornou
4 resultados,
apenas um faz referência ao software.

\begin{itemize}
\item 2014: D. Hao - L. Zhang - L. Zhang - G. Rothermel - H. Mei
\end{itemize}

\section{Bogor/Kiasan}

Autores:
X. Deng - J. Lee - Robby.

Verificação de modelos
publicado no artigo {\it Bogor/Kiasan: A k-bounded Symbolic Execution for Checking Strong Heap Properties of Open Systems}
por X. Deng
no ASE 2006,
disponibilizado em \url{http://bogor.projects.cs.ksu.edu/manual/}
como software livre
sob uma licença SAnToS Laboratory Open Academic License.

Software sem informações sobre lançamentos ou releases,
escrito em Java,
uma busca por citações no {\bf IEEE Xplore} por
\texttt{((Kiasan/Bogor) OR Bogor/Kiasan)}
e no {\bf ACM} com
\texttt{content.ftsec:("Bogor/Kiasan")}
retornou
37 resultados,
{\bf 15} fazem referência ao software.

\begin{itemize}
\item 2006: M. B. Dwyer - J. Hatcliff
\item 2007: M. B. Dwyer - J. Hatcliff - R. Robby - C. S. Pasareanu - W. Visser
\item 2007: X. Deng - Robby - J. Hatcliff
\item 2007: X. Deng - J. H. Robby - J. Hatcliff
\item 2008: C. S. P\v{a}s\v{a}reanu - P. C. Mehlitz - D. H. Bushnell - K. Gundy-Burlet - M. Lowry - S. Person - M. Pape
\item 2008: C. Csallner - Y. Smaragdakis - T. Xie
\item 2008: D. Distefano - M. J. Parkinson J
\item 2009: Robby - P. Chalin
\item 2009: A. King - S. Procter - D. Andresen - J. Hatcliff - S. Warren - W. Spees - R. Jetley - P. Jones - S. Weininger
\item 2010: C. S. P\u{a}s\u{a}reanu - N. Rungta
\item 2010: M. Roberson - C. Boyapati
\item 2010: O. Tkachuk - M. B. Dwyer
\item 2013: A. Andrianova - V. Itsykson
\item 2014: B. Hillery - E. Mercer - N. Rungta - S. Person
\item 2015: G. Denaro - A. Margara - M. Pezz\`{e} - M. Vivanti
\end{itemize}

\section{Loopfrog}

Autores:
D. Kroening - N. Sharygina - S. Tonetta - A. Tsitovich - C. M. Wintersteiger.

Verificação de modelos
publicado no artigo {\it Loopfrog: A Static Analyzer for ANSI-C Programs}
por D. Kroening
no ASE 2009,
disponibilizado em \url{http://verify.inf.usi.ch/content/loopfrog}
gratuitamente
sem uma licença definida.

Software sem informações sobre lançamentos ou releases,
uma busca por citações no {\bf IEEE Xplore} por
\texttt{(Loopfrog)}
e no {\bf ACM} com
\texttt{content.ftsec:(+Loopfrog)}
retornou
6 resultados,
{\bf 4} fazem referência ao software.

\begin{itemize}
\item 2012: A. R. Choudhury - A. K. Bhattacharjee
\item 2013: A. V. Nori - R. Sharma
\item 2013: D. Larraz - A. Oliveras - E. Rodríguez-Carbonell - A. Rubio
\item 2013: E. Larson
\end{itemize}

\section{Lotrack}

Autores:
Lillack, Max - K\"{a}stner, Christian - Bodden, Eric.

Análise estática de configuração
publicado no artigo {\it Tracking Load-time Configuration Options}
por Lillack, Max
no ASE 2014,
disponibilizado em \url{https://github.com/MaxLillack/Lotrack}
gratuitamente
sem uma licença definida.

Software sem informações sobre lançamentos ou releases,
escrito em Java,
uma busca por citações no {\bf IEEE Xplore} por
\texttt{(Lotrack)}
e no {\bf ACM} com
\texttt{content.ftsec:(+Lotrack)}
retornou
4 resultados,
apenas um faz referência ao software.

\begin{itemize}
\item 2015: M. Sayagh - B. Adams
\end{itemize}

\section{MPAnalyzer}

Autores:
Higo, Yoshiki - Kusumoto, Shinji.

Análise de padrões disponível
publicado no artigo {\it MPAnalyzer: A Tool for Finding Unintended Inconsistencies in Program Source Code}
por Higo, Yoshiki
no ASE 2014,
disponibilizado em \url{https://github.com/YoshikiHigo/MPAnalyzer}
gratuitamente
sem uma licença definida.

Software sem informações sobre lançamentos ou releases,
escrito em Java,
uma busca por citações no {\bf IEEE Xplore} por
\texttt{(MPAnalyzer)}
e no {\bf ACM} com
\texttt{content.ftsec:(+MPAnalyzer)}
retornou
3 resultados,
nenhum faz referência ao software.


\section{MSP}

Autores:
Ketterlin, Alain - Clauss, Philippe.

Construção de modelo formal de acesso a memória
publicado no artigo {\it Recovering Memory Access Patterns of Executable Programs}
por Ketterlin, Alain
no SCAM 2010,
disponibilizado em \url{http://icps.u-strasbg.fr/software/msp}
mas inacessível em 09/08/2017.

Software sem informações sobre lançamentos ou releases,
uma busca por citações no {\bf IEEE Xplore} por
\texttt{((((MSP) AND tool) AND Binary) AND "Program Analysis")}
e no {\bf ACM} com
\texttt{content.ftsec:(+MSP +tool +Binary +"Program Analysis")}
retornou
37 resultados,
apenas um faz referência ao software.

\begin{itemize}
\item 2009: M. Gauger - P. J. Marrón - C. Niedermeier
\end{itemize}

\section{mygcc}

Autores:
N. Volanschi.

Verificação de programas C
publicado no artigo {\it A Portable Compiler-Integrated Approach to Permanent Checking}
por N. Volanschi
no ASE 2006,
disponibilizado em \url{http://mygcc.free.fr}
como software livre
sob uma licença GPL.

Software considerado obsoleto,
5 versões lançadas
,
escrito em C,
uma busca por citações no {\bf IEEE Xplore} por
\texttt{(mygcc)}
e no {\bf ACM} com
\texttt{content.ftsec:(+mygcc)}
retornou
7 resultados,
{\bf 6} fazem referência ao software.

\begin{itemize}
\item 2006: N. Volanschi
\item 2008: N. Volanschi - C. Rinderknecht
\item 2009: K. Yu - C. Wang - Y. l. Chen - M. x. Lin
\item 2009: V. Itsykson - M. Moiseev - V. Tsesko - A. Zakharov
\item 2010: L. Torri - G. Fachini - L. Steinfeld - V. Camara - L. Carro - É. Cota
\item 2014: S. Tlili - J. M. Fernandez - A. Belghith - B. Dridi - S. Hidouri
\end{itemize}

\section{PARSEWeb}

Autores:
Thummalapenta, Suresh - Xie, Tao.

Query para apoio e sugestão de reuso de bibliotecas
publicado no artigo {\it Parseweb: A Programmer Assistant for Reusing Open Source Code on the Web}
por Thummalapenta, Suresh
no ASE 2007,
disponibilizado em \url{http://ase.csc.ncsu.edu/parseweb}
mas inacessível em 09/08/2017.

Software sem informações sobre lançamentos ou releases,
uma busca por citações no {\bf IEEE Xplore} por
\texttt{(((PARSEWeb) AND tool) AND AST)}
e no {\bf ACM} com
\texttt{content.ftsec:(+PARSEWeb +tool +AST)}
retornou
49 resultados,
{\bf 23} fazem referência ao software.

\begin{itemize}
\item 2008: T. Ishio - H. Date - T. Miyake - K. Inoue
\item 2008: T. Xie - M. Acharya - S. Thummalapenta - K. Taneja
\item 2008: B. Dagenais - L. Hendren
\item 2010: J. Ossher - S. Bajracharya - C. Lopes
\item 2011: L. W. Mar - Y. C. Wu - H. C. Jiau
\item 2011: I. Keivanloo - C. Forbes - J. Rilling - P. Charland
\item 2011: L. Heinemann - B. Hummel
\item 2011: S. Khatoon - A. Mahmood - G. Li
\item 2011: K. Yessenov - Z. Xu - A. Solar-Lezama
\item 2012: D. Wightman - Z. Ye - J. Brandt - R. Vertegaal
\item 2012: I. Keivanloo - J. Rilling - P. Charland
\item 2012: L. Heinemann - V. Bauer - M. Herrmannsdoerfer - B. Hummel
\item 2012: A. T. Nguyen - T. T. Nguyen - H. A. Nguyen - A. Tamrawi - H. V. Nguyen - J. Al-Kofahi - T. N. Nguyen
\item 2012: A. Mishne - S. Shoham - E. Yahav
\item 2012: D. Perelman - S. Gulwani - T. Ball - D. Grossman
\item 2013: T. Gvero - V. Kuncak - I. Kuraj - R. Piskac
\item 2013: V. Raychev - M. Sch\"{a}fer - M. Sridharan - M. Vechev
\item 2013: F. Thung - D. Lo - J. Lawall
\item 2015: T. Gvero - V. Kuncak
\item 2015: V. Amintabar - A. Heydarnoori - M. Ghafari
\item 2016: B. Spasojevi\'{c} - M. Ghafari - O. Nierstrasz
\item 2016: H. Zhang - A. Jain - G. Khandelwal - C. Kaushik - S. Ge - W. Hu
\item 2016: D. Yang - A. Hussain - C. V. Lopes
\end{itemize}

\section{PAT - Puzzle-Based Automatic Testing}

Autores:
N. Chen - S. Kim.

Ambiente de teste automático
publicado no artigo {\it Puzzle-based automatic testing: bringing humans into the loop by solving puzzles}
por N. Chen
no ASE 2012,
disponibilizado em \url{http://pat.cse.ust.hk:8080}
mas inacessível em 09/08/2017.

Software sem informações sobre lançamentos ou releases,
uma busca por citações no {\bf IEEE Xplore} por
\texttt{("Puzzle-based automatic testing")}
e no {\bf ACM} com
\texttt{content.ftsec:(+"Puzzle-based automatic testing")}
retornou
13 resultados,
apenas um faz referência ao software.

\begin{itemize}
\item 2016: J. Wang - S. Wang - Q. Cui - Q. Wang
\end{itemize}

\section{PHP AiR}

Autores:
Hills, Mark - Klint, Paul - Vinju, Jurgen J..

Um framework para análise de código PHP escrito em Rascal
publicado no artigo {\it Static, Lightweight Includes Resolution for PHP}
por Hills, Mark
no ASE 2014,
disponibilizado em \url{https://github.com/cwi-swat/php-analysis}
gratuitamente
sem uma licença definida.

Software com lançamentos ?,
4 versões lançadas
entre 2011 e 2014,
escrito em Rascal,
uma busca por citações no {\bf IEEE Xplore} por
\texttt{("PHP AiR")}
e no {\bf ACM} com
\texttt{content.ftsec:(+"PHP AiR")}
retornou
11 resultados,
{\bf 8} fazem referência ao software.

\begin{itemize}
\item 2014: M. Hills - P. Klint
\item 2015: B. Basten - M. Hills - P. Klint - D. Landman - A. Shahi - M. J. Steindorfer - J. J. Vinju
\item 2015: M. J. Steindorfer - J. J. Vinju
\item 2015: M. Hills
\item 2015: M. Hills
\item 2015: M. Hills
\item 2016: M. Hills
\item 2017: D. Anderson - M. Hills
\end{itemize}

\section{protopurity}

Autores:
J. Nicolay - C. Noguera - C. De Roover - W. De Meuter.

Análise de impacto
publicado no artigo {\it Detecting function purity in JavaScript}
por J. Nicolay
no SCAM 2015,
disponibilizado em \url{https://github.com/jensnicolay/jipda/tree/scam2015/protopurity}
gratuitamente
sem uma licença definida.

Software sem informações sobre lançamentos ou releases,
escrito em Javascript,
uma busca por citações no {\bf IEEE Xplore} por
\texttt{content.ftsec:(+protopurity)}
e no {\bf ACM} com
\texttt{(protopurity)}
retornou
0 resultados,
nenhum faz referência ao software.


\section{Pseudogen}

Autores:
H. Fudaba - Y. Oda - K. Akabe - G. Neubig - H. Hata - S. Sakti - T. Toda - S. Nakamura.

Transformação de código-fonte em pseudo-código
publicado no artigo {\it Pseudogen: A Tool to Automatically Generate Pseudo-Code from Source Code}
por H. Fudaba
no ASE 2015,
disponibilizado em \url{http://ahclab.naist.jp/pseudogen}
gratuitamente
sem uma licença definida.

Software sem informações sobre lançamentos ou releases,
escrito em Python,
uma busca por citações no {\bf IEEE Xplore} por
\texttt{(Pseudogen)}
e no {\bf ACM} com
\texttt{content.ftsec:(+Pseudogen +"Pseudo-Code")}
retornou
5 resultados,
nenhum faz referência ao software.


\section{PtYasm}

Autores:
T. E. Hart - K. Ku - A. Gurfinkel - M. Chechik - D. Lie.

Verificação de modelos
publicado no artigo {\it Augmenting Counterexample-Guided Abstraction Refinement with Proof Templates}
por T. E. Hart
no ASE 2008,
disponibilizado em \url{http://www.cs.toronto.edu/~tomhart/ptyasm}
gratuitamente
sem uma licença definida.

Software sem informações sobre lançamentos ou releases,
escrito em Java,
uma busca por citações no {\bf IEEE Xplore} por
\texttt{(PtYasm)}
e no {\bf ACM} com
\texttt{content.ftsec:(+PtYasm)}
retornou
2 resultados,
nenhum faz referência ao software.


\section{PuMoC}

Autores:
F. Song - T. Touili.

Verificação de modelos
publicado no artigo {\it PuMoC: a CTL model-checker for sequential programs}
por F. Song
no ASE 2012,
disponibilizado em \url{http://www.liafa.jussieu.fr/~song/PuMoC}
mas inacessível em 09/08/2017.

Software sem informações sobre lançamentos ou releases,
uma busca por citações no {\bf IEEE Xplore} por
\texttt{(PuMoC)}
e no {\bf ACM} com
\texttt{content.ftsec:(+PuMoC)}
retornou
4 resultados,
apenas um faz referência ao software.

\begin{itemize}
\item 2013: F. Song - T. Touili
\end{itemize}

\section{PYTHIA}

Autores:
S. Mirshokraie - A. Mesbah - K. Pattabiraman.

Criação automática de casos de teste
publicado no artigo {\it PYTHIA: Generating test cases with oracles for JavaScript applications}
por S. Mirshokraie
no ASE 2013,
disponibilizado em \url{http://salt.ece.ubc.ca/software/pythia/}
mas inacessível em 09/08/2017.

Software sem informações sobre lançamentos ou releases,
uma busca por citações no {\bf IEEE Xplore} por
\texttt{((PYTHIA) AND JavaScript)}
e no {\bf ACM} com
\texttt{content.ftsec:(+PYTHIA +JavaScript)}
retornou
15 resultados,
apenas um faz referência ao software.

\begin{itemize}
\item 2014: S. Mirshokraie
\end{itemize}

\section{ReAssert}

Autores:
B. Daniel - V. Jagannath - D. Dig - D. Marinov.

Localização de falhas em testes e refatoração
publicado no artigo {\it ReAssert: Suggesting Repairs for Broken Unit Tests}
por B. Daniel
no ASE 2009,
disponibilizado em \url{http://mir.cs.illinois.edu/reassert}
como software livre
sob uma licença Illinois/NCSA Open Source License.

Software considerado obsoleto,
5 versões lançadas
entre 2009 e 2010,
escrito em Java,
uma busca por citações no {\bf IEEE Xplore} por
\texttt{(ReAssert tool "Unit Tests")}
e no {\bf ACM} com
\texttt{content.ftsec:(+ReAssert +tool +"Unit Tests")}
retornou
28 resultados,
{\bf 12} fazem referência ao software.

\begin{itemize}
\item 2010: B. Daniel - T. Gvero - D. Marinov
\item 2011: B. Daniel - D. Dig - T. Gvero - V. Jagannath - J. Jiaa - D. Mitchell - J. Nogiec - S. H. Tan - D. Marinov
\item 2011: S. R. Choudhary - D. Zhao - H. Versee - A. Orso
\item 2011: D. Mujumdar - M. Kallenbach - B. Liu - B. Hartmann
\item 2011: M. Mirzaaghaei
\item 2012: S. Negara - M. Vakilian - N. Chen - R. E. Johnson - D. Dig
\item 2012: L. S. Pinto - S. Sinha - A. Orso
\item 2012: W. Basit - F. Lodhi - M. U. Bhatti
\item 2013: F. Pastore - L. Mariani - G. Fraser
\item 2013: J. D. Strate - P. A. Laplante
\item 2016: D. Dig - R. Johnson - D. Marinov - B. Bailey - D. Batory
\item 2016: S. A. Carr - F. Logozzo - M. Payer
\end{itemize}

\section{Rêve}

Autores:
Felsing, Dennis - Grebing, Sarah - Klebanov, Vladimir - R\"{u}mmer, Philipp - Ulbrich, Mattias.

Verificação de regressão
publicado no artigo {\it Automating Regression Verification}
por Felsing, Dennis
no ASE 2014,
disponibilizado em \url{http://formal.iti.kit.edu/improve}
mas inacessível em 09/08/2017.

Software sem informações sobre lançamentos ou releases,
uma busca por citações no {\bf IEEE Xplore} por
\texttt{(Reve Automating regression verification)}
e no {\bf ACM} com
\texttt{content.ftsec:(+"Reve" +Automating +regression +verification)}
retornou
13 resultados,
nenhum faz referência ao software.


\section{RRFinder}

Autores:
Q. Wu - G. Liang - Q. Wang - T. Xie - H. Mei.

Mineração de especificação de liberação de recursos
publicado no artigo {\it Iterative mining of resource-releasing specifications}
por Q. Wu
no ASE 2011,
disponibilizado em \url{http://sa.seforge.org/RRFinder/}
mas inacessível em 09/08/2017.

Software sem informações sobre lançamentos ou releases,
uma busca por citações no {\bf IEEE Xplore} por
\texttt{(RRFinder)}
e no {\bf ACM} com
\texttt{content.ftsec:(+RRFinder)}
retornou
5 resultados,
{\bf 2} fazem referência ao software.

\begin{itemize}
\item 2014: S. Chaudhary - S. Fischmeister - L. Tan
\item 2015: J. J. Bai - Y. P. Wang - H. Q. Liu - S. M. Hu
\end{itemize}

\section{Sapid/XML}

Autores:
K. Maruyama - S. Yamamoto.

Representação intermediária de código Java usando XML ao invés de AST
publicado no artigo {\it A CASE tool platform using an XML representation of Java source code}
por K. Maruyama
no SCAM 2004,
disponibilizado em \url{http://www.jtool.org}
mas inacessível em 09/08/2017.

Software sem informações sobre lançamentos ou releases,
uma busca por citações no {\bf IEEE Xplore} por
\texttt{("Sapid/XML")}
e no {\bf ACM} com
\texttt{content.ftsec:(+"Sapid/XML")}
retornou
5 resultados,
{\bf 4} fazem referência ao software.

\begin{itemize}
\item 2005: T. Omori - K. Maruyama
\item 2005: K. Maruyama - S. Yamamoto
\item 2006: K. Maruyama
\item 2012: H. Li - S. Thompson
\end{itemize}

\section{Sonar Qube Plug-in}

Autores:
R. Ferenc - L. Langó - I. Siket - T. Gyimóthy - T. Bakota.

Extende o SourceMeter com análise de código Java com o uso do SonarQube
publicado no artigo {\it Source Meter Sonar Qube Plug-in}
por R. Ferenc
no SCAM 2014,
disponibilizado em \url{http://github.com/FrontEndART/SonarQube-plug-in}
gratuitamente
sob uma licença FrontEndART Software Ltd.

Software com lançamentos frequentes,
4 versões lançadas
entre 2015 e 2016,
escrito em Java,
uma busca por citações no {\bf IEEE Xplore} por
\texttt{(SonarQube-plug-in)}
e no {\bf ACM} com
\texttt{content.ftsec:(+"SonarQube-plug-in")}
retornou
2 resultados,
nenhum faz referência ao software.


\section{SPARTA - Static Program Analysis for Reliable Trusted Apps}

Autores:
P. Barros - R. Just - S. Millstein - P. Vines - W. Dietl - M. dAmorim - M. D. Ernst.

Segurança pra detecção de malware
publicado no artigo {\it Static Analysis of Implicit Control Flow: Resolving Java Reflection and Android Intents (T)}
por P. Barros
no ASE 2015,
disponibilizado em \url{http://types.cs.washington.edu/sparta}
gratuitamente
sem uma licença definida.

Software com lançamentos ocsaionais,
14 versões lançadas
entre 2012 e 2016,
escrito em Java,
uma busca por citações no {\bf IEEE Xplore} por
\texttt{(SPARTA "Program Analysis")}
e no {\bf ACM} com
\texttt{content.ftsec:(+SPARTA +"Program Analysis")}
retornou
7 resultados,
{\bf 3} fazem referência ao software.

\begin{itemize}
\item 2002: M. Pinzger - M. Fischer - H. Gall - M. Jazayeri
\item 2017: F. Roesner
\item 2017: D. Landman - A. Serebrenik - J. J. Vinju
\end{itemize}

\section{srcML}

Autores:
M. L. Collard - M. J. Decker - J. I. Maletic.

Transformação source-to-source
publicado no artigo {\it Lightweight Transformation and Fact Extraction with the srcML Toolkit}
por M. L. Collard
no SCAM 2011,
disponibilizado em \url{http://www.sdml.info/projects/srcml/trunk}
como software livre
sob uma licença GPL v3.

Software com lançamentos ocsaionais,
14 versões lançadas
entre 2011 e 2015,
escrito em C++,
uma busca por citações no {\bf IEEE Xplore} por
\texttt{(srcML Toolkit Java C)}
e no {\bf ACM} com
\texttt{content.ftsec:(+srcML +Toolkit +Java +"C++")}
retornou
43 resultados,
{\bf 39} fazem referência ao software.

\begin{itemize}
\item 2002: J. F. Power - B. A. Malloy
\item 2003: G. Antoniol - M. D. Penta - G. Masone - U. Villano
\item 2004: V. Wahler - D. Seipel - J. Wolff - G. Fischer
\item 2004: B. Cleary - C. Exton
\item 2004: K. Maruyama - S. Yamamoto
\item 2005: J. I. Maletic - M. L. Collard - B. Simoes
\item 2005: K. Maruyama - S. Yamamoto
\item 2005: M. Topolnik
\item 2007: G. CanforaHarman - M. D. Penta
\item 2008: G. Canfora - M. D. Penta
\item 2009: J. Nilsson - W. Lowe - J. Hall - J. Nivre
\item 2009: J. Nilsson - W. L\"{o}we - J. Hall - J. Nivre
\item 2009: J. Nödler - H. Neukirchen - J. Grabowski
\item 2010: M. L. Collard - J. I. Maletic - B. P. Robinson
\item 2011: G. Canfora - M. Di Penta - L. Cerulo
\item 2011: A. Corazza - S. D. Martino - V. Maggio - G. Scanniello
\item 2012: S. M. Alnaeli - A. Alali - J. I. Maletic
\item 2012: M. Abdelkader - M. Mimoun - S. M. Benslimane
\item 2012: J. L. Wilkerson - D. R. Tauritz - J. M. Bridges
\item 2013: K. Sakamoto
\item 2013: K. K. Chaturvedi - V. B. Sing - P. Singh
\item 2013: K. Sakamoto - K. Shimojo - R. Takasawa - H. Washizaki - Y. Fukazawa
\item 2013: M. L. Collard - M. J. Decker - J. I. Maletic
\item 2014: L. Moreno - G. Bavota - M. Di Penta - R. Oliveto - A. Marcus - G. Canfora
\item 2014: A. Potdar - E. Shihab
\item 2014: M. B. Zanjani - G. Swartzendruber - H. Kagdi
\item 2014: X. Zhang - M. Persson - M. Nyberg - B. Mokhtari - A. Einarson - H. Linder - J. Westman - D. Chen - M. Törngren
\item 2015: M. A. d. F. Farias - M. G. d. M. Neto - A. B. d. Silva - R. O. Spínola
\item 2015: N. J. Abid - N. Dragan - M. L. Collard - J. I. Maletic
\item 2015: J. I. Maletic - M. L. Collard
\item 2016: S. M. Alnaeli - M. Sarnowski - M. S. Aman - A. Abdelgawad - K. Yelamarthi
\item 2016: G. Bavota - B. Russo
\item 2016: M. J. Decker - K. Swartz - M. L. Collard - J. I. Maletic
\item 2016: M. Rahimi - W. Goss - J. Cleland-Huang
\item 2016: S. M. Alnaeli - M. Sarnowski - M. S. Aman - K. Yelamarthi - A. Abdelgawad - H. Jiang
\item 2016: C. D. Newman - T. Sage - M. L. Collard - H. W. Alomari - J. I. Maletic
\item 2017: B. Bartman - C. D. Newman - M. L. Collard - J. I. Maletic
\item 2017: H. Cai
\item 2017: F. Medeiros - M. Ribeiro - R. Gheyi - S. Apel - C. Kastner - B. Ferreira - L. Carvalho - B. Fonseca
\end{itemize}

\section{SWAT - Search based Web Application Tester}

Autores:
Alshahwan, Nadia - Harman, Mark.

Teste automático para aplicação web
publicado no artigo {\it Automated Web Application Testing Using Search Based Software Engineering}
por Alshahwan, Nadia
no ASE 2011,
disponibilizado em \url{http://www.cs.ucl.ac.uk/staff/nalshahw/swat}
mas inacessível em 09/08/2017.

Software sem informações sobre lançamentos ou releases,
uma busca por citações no {\bf IEEE Xplore} por
\texttt{(SWAT Search Web Application Tester)}
e no {\bf ACM} com
\texttt{content.ftsec:(+SWAT +Search +Web +Application +Tester)}
retornou
15 resultados,
{\bf 3} fazem referência ao software.

\begin{itemize}
\item 2012: N. Alshahwan - M. Harman
\item 2013: A. M. F. V. de Castro - G. A. Macedo - E. F. Collins - A. C. Dias-Neto
\item 2014: N. Alshahwan - M. Harman
\end{itemize}

\section{TACLE - Type Analysis and CalL graph construction for Eclipse}

Autores:
J. Sawin - A. Rountev.

Análise de tipo (Type Analysis) e construção e visualizaçao de grafos de chamada (Call Graph)
publicado no artigo {\it Estimating the Run-Time Progress of a Call Graph Construction Algorithm53-62}
por J. Sawin
no SCAM 2006,
disponibilizado em \url{http://presto.cse.ohio-state.edu/tacle}
gratuitamente
sem uma licença definida.

Software sem informações sobre lançamentos ou releases,
escrito em Java,
uma busca por citações no {\bf IEEE Xplore} por
\texttt{(TACLE "Type Analysis")}
e no {\bf ACM} com
\texttt{content.ftsec:(+TACLE +"Type Analysis")}
retornou
4 resultados,
{\bf 2} fazem referência ao software.

\begin{itemize}
\item 2005: M. Sharp - J. Sawin - A. Rountev
\item 2006: J. Sawin - M. Sharp - A. Rountev
\end{itemize}

\section{TEBA}

Autores:
A. Yoshida - Y. Hachisu.

Transformação source-to-source
publicado no artigo {\it A Pattern Search Method for Unpreprocessed C Programs Based on Tokenized Syntax Trees}
por A. Yoshida
no SCAM 2014,
disponibilizado em \url{http://tebasaki.jp/src}
gratuitamente
sem uma licença definida.

Software com lançamentos ocsaionais,
21 versões lançadas
entre 2010 e 2016,
escrito em Perl,
uma busca por citações no {\bf IEEE Xplore} por
\texttt{(((TEBA) AND tool) AND Programs)}
e no {\bf ACM} com
\texttt{content.ftsec:(+TEBA +tool)}
retornou
11 resultados,
nenhum faz referência ao software.


\section{TestEra}

Autores:
Marinov, Darko - Khurshid, Sarfraz.

Geração automática de testes
publicado no artigo {\it TestEra: A Novel Framework for Automated Testing of Java Programs}
por Marinov, Darko
no ASE 2001,
disponibilizado em \url{http://www.mit.edu/~sarfraz/testera}
mas inacessível em 09/08/2017.

Software sem informações sobre lançamentos ou releases,
uma busca por citações no {\bf IEEE Xplore} por
\texttt{(((((TestEra) AND framework) AND Java) AND testing) AND sarfraz)}
e no {\bf ACM} com
\texttt{content.ftsec:(+TestEra +framework +Java +testing +sarfraz)}
retornou
44 resultados,
{\bf 22} fazem referência ao software.

\begin{itemize}
\item 2002: C. Boyapati - S. Khurshid - D. Marinov
\item 2004: W. Visser - C. S. P\v{a}s\v{a}reanu - S. Khurshid
\item 2004: K. Sullivan - J. Yang - D. Coppit - S. Khurshid - D. Jackson
\item 2005: S. Khurshid - Y. L. Suen
\item 2006: S. Khurshid - M. Z. Malik - E. Uzuncaova
\item 2007: D. Shao - S. Khurshid - D. E. Perry
\item 2007: D. Shao - S. Khurshid - D. E. Perry
\item 2007: B. Elkarablieh - Y. Zayour - S. Khurshid
\item 2007: S. Misailovic - A. Milicevic - N. Petrovic - S. Khurshid - D. Marinov
\item 2008: S. A. Khalek - B. Elkarablieh - Y. O. Laleye - S. Khurshid
\item 2008: E. Uzuncaova - D. Garcia - S. Khurshid - D. Batory
\item 2009: J. H. Siddiqui - S. Khurshid
\item 2009: D. Rayside - A. Milicevic - K. Yessenov - G. Dennis - D. Jackson
\item 2010: D. Shao - D. Gopinath - S. Khurshid - D. E. Perry
\item 2010: M. Gligoric - T. Gvero - V. Jagannath - S. Khurshid - V. Kuncak - D. Marinov
\item 2011: S. A. Khalek - S. Khurshid
\item 2011: S. A. Khalek - G. Yang - L. Zhang - D. Marinov - S. Khurshid
\item 2011: S. A. Khalek - S. Khurshid
\item 2012: C. Boyapati - S. Khurshid - D. Marinov
\item 2012: R. Nokhbeh Zaeem - S. Khurshid
\item 2013: C. Cadar - F. Dadeau
\item 2014: G. Fraser - A. Arcuri
\end{itemize}

\section{Vdiff}

Autores:
Duley, Adam - Spandikow, Chris - Kim, Miryung.

Visualização de diferença de código-fonte
publicado no artigo {\it A Program Differencing Algorithm for Verilog HDL}
por Duley, Adam
no ASE 2010,
disponibilizado em \url{http://web.cs.ucla.edu/~miryung/software/vdiff/web/index.html}
mas inacessível em 09/08/2017.

Software sem informações sobre lançamentos ou releases,
uma busca por citações no {\bf IEEE Xplore} por
\texttt{(((Vdiff) AND verilog) AND HDL)}
e no {\bf ACM} com
\texttt{content.ftsec:(+Vdiff +verilog +HDL)}
retornou
10 resultados,
{\bf 4} fazem referência ao software.

\begin{itemize}
\item 2011: H. A. Nguyen - T. T. Nguyen - H. V. Nguyen - T. N. Nguyen
\item 2011: Y. Yu - T. T. Tun - B. Nuseibeh
\item 2015: N. Palix - J. R. Falleri - J. Lawall
\item 2016: G. Dotzler - M. Philippsen
\end{itemize}

\section{WALA}

Autores:
Z. D. Luo - L. Hillis - R. Das - Y. Qi.

Análise estática de bytecode Java
publicado no artigo {\it Effective Static Analysis to Find Concurrency Bugs in Java}
por Z. D. Luo
no SCAM 2010,
disponibilizado em \url{http://wala.sourceforge.net/wiki/index.php/Main_Page}
como software livre
sob uma licença Eclipse Public License v1.0.

Software com lançamentos ocsaionais,
37 versões lançadas
entre 2006 e 2017,
escrito em Java,
uma busca por citações no {\bf IEEE Xplore} por
\texttt{(((WALA) AND "Static Analysis") AND "Concurrency Bugs")}
e no {\bf ACM} com
\texttt{content.ftsec:(+WALA +"Static Analysis" +"Concurrency Bugs")}
retornou
12 resultados,
{\bf 10} fazem referência ao software.

\begin{itemize}
\item 2008: C. Hammer - J. Dolby - M. Vaziri - F. Tip
\item 2009: Y. Qi - R. Das - Z. D. Luo - M. Trotter
\item 2011: M. Vakilian - S. Negara - S. Tasharofi - R. E. Johnson
\item 2012: F. Thung - Lucia - D. Lo - L. Jiang - F. Rahman - P. T. Devanbu
\item 2012: S. Zhang - H. L\"{u} - M. D. Ernst
\item 2013: P. Liu - J. Dolby - C. Zhang
\item 2016: W. Wang - Y. Zheng - P. Liu - L. Xu - X. Zhang - P. Eugster
\item 2016: Q. Stiévenart - M. Vandercammen - W. D. Meuter - C. D. Roover
\item 2016: J. Huang - X. Zhang - L. Tan
\item 2017: M. Jakobs - H. Wehrheim
\end{itemize}

\section{Wrangler}

Autores:
H. Li - S. Thompson.

Refatoração de código Erlang
publicado no artigo {\it Refactoring Support for Modularity Maintenance in Erlang}
por H. Li
no SCAM 2010,
disponibilizado em \url{http://www.cs.kent.ac.uk/projects/wrangler/Home.html}
como software livre
sob uma licença BSD License "revised".

Software com lançamentos ocsaionais,
34 versões lançadas
até 2015,
escrito em Erlang,
uma busca por citações no {\bf IEEE Xplore} por
\texttt{((Wrangler) AND Erlang)}
e no {\bf ACM} com
\texttt{content.ftsec:(+Wrangler +Erlang)}
retornou
37 resultados,
{\bf 32} fazem referência ao software.

\begin{itemize}
\item 2006: H. Li - S. Thompson
\item 2008: H. Li - S. Thompson - G. Orosz - M. T\'{o}th
\item 2008: K. Sagonas - D. Luna
\item 2008: N. Sultana - S. Thompson
\item 2008: T. Nagy - A. Nagyn{\'e} V\'{\i}g
\item 2009: H. Li - S. Thompson
\item 2009: K. Sagonas - T. Avgerinos
\item 2009: T. Avgerinos - K. Sagonas
\item 2009: L. L\"{o}vei
\item 2010: R. Kitlei - I. Boz\'{o} - T. Kozsik - M. Tejfel - M. T\'{o}th
\item 2010: T. Arts - S. Thompson
\item 2010: D. Drienyovszky - D. Horp\'{a}csi - S. Thompson
\item 2010: J. Armstrong
\item 2011: H. Li - S. Thompson - T. Arts
\item 2012: H. Li - S. Thompson
\item 2012: J. Nicolay
\item 2012: C. Brown - K. Hammond - M. Danelutto - P. Kilpatrick
\item 2012: H. Li - S. Thompson
\item 2012: M. Hills - P. Klint - J. J. Vinju
\item 2013: H. Li - S. Thompson
\item 2013: G. Soares - R. Gheyi - T. Massoni
\item 2013: N. Walkinshaw - R. Taylor - J. Derrick
\item 2014: H. Li - S. Thompson - P. Lamela Seijas - M. A. Francisco
\item 2014: I. Boz\'{o} - V. Ford\'{o}s - Z. Horvath - M. T\'{o}th - D. Horp\'{a}csi - T. Kozsik - J. K\"{o}szegi - A. Barwell - C. Brown - K. Hammond
\item 2014: M. Mongiovi - G. Mendes - R. Gheyi - G. Soares - M. Ribeiro
\item 2014: M. Plasch - M. Hofmann - G. Ebenhofer - M. Rooker
\item 2015: R. Taylor - J. Derrick
\item 2015: J. Kim - D. Batory - D. Dig
\item 2016: E. Fernandes - J. Oliveira - G. Vale - T. Paiva - E. Figueiredo
\item 2016: A. P. Oliveira - P. S. L. Souza - S. R. S. Souza
\item 2016: A. D. Barwell - C. Brown - D. Castro - K. Hammond
\item 2017: M. Mongiovi - R. Gheyi - G. Soares - M. Ribeiro - P. Borba - L. Teixeira
\end{itemize}

\section{XOgastan}

Autores:
G. Antoniol - M. Di Penta - G. Masone - U. Villano.

Transformação source-to-source
publicado no artigo {\it XOgastan: XML-oriented gcc AST analysis and transformations}
por G. Antoniol
no SCAM 2003,
disponibilizado em \url{http://web.ing.unisannio.it/villano/students/masone}
mas inacessível em 09/08/2017.

Software sem informações sobre lançamentos ou releases,
uma busca por citações no {\bf IEEE Xplore} por
\texttt{(XOgastan)}
e no {\bf ACM} com
\texttt{content.ftsec:(+XOgastan)}
retornou
7 resultados,
{\bf 4} fazem referência ao software.

\begin{itemize}
\item 2004: T. Gschwind - M. Pinzger - H. Gall
\item 2006: C. Wagner - T. Margaria - H. G. Pagendarm
\item 2008: H. M. Kienle
\item 2011: Y. Higo - A. Saitoh - G. Yamada - T. Miyake - S. Kusumoto - K. Inoue
\end{itemize}


\section{Categorias}



\begin{itemize}
\item abre uma linha de pesquisa com objetivo de extender o software, o objetivo do artigo é adicionar suporte a SAR ao software, faz experimentos para avaliar a implementação: 1
\item apresenta 5 técnicas de otimização para exibir diferenças entre códigos, estes algoritmos são implementados em ferramentas distintas, GumTree incluido, avalia e compara os resultados, código disponível em https://github.com/FAU-Inf2/treedifferencing: 1
\item apresenta a plataforma tricorder, para análise de programas com objetivo de construir um ecosistema de dados de análise de programas, usa o software como um dos diversos analisadores estáticos integrados na plataforma: 1
\item apresenta o JastAdd Extensible Java Compiler (JastAddJ) implementado usando JastAdd: 1
\item apresenta o compilador archface compiler construído como uma extensão do ccJava, mesmos autores: 1
\item apresenta o software CSeq e cita o software como uma das opções de backend do CSeq: 1
\item apresenta o software em detalhes, usa como parte da solução Jrbx para gerenciar código fonte através de XML, disponibiliza tanto o software quando o Jrbx em http://www.jtool.org, cita o software nas referencias http://www.sapid.org: 1
\item apresenta o software, descreve como usar, tutorial: 1
\item apresenta o software, descreve em detalhe, indica site do projeto, usa a ferramenta para coleta/análise: 1
\item apresenta o software, integra o software como backend do JML4 (Java Modeling Language), cita nas referências que os programas e dados do experimento estão em http://bogor.projects.cis.ksu.edu/kiasan/kunit/laziersharp/ mas o site redireciona para o manuel do bogor, acessado em Fri Aug 18 22:22:09 UTC 2017: 1
\item apresenta o software, mostra os componentes core do software e sua arquitetura, enumera estudos avaliando o software: 1
\item apresenta um novo software Binary Guided Random Testing (BGRT): 1
\item apresenta um software que é evolução do ccJava mas mais genérico, projeto chamado Archface é voltado para programação e também design arquitetural, mesmos autores: 1
\item apresenta uma coleção de suítes de 'pthread benchmarks', e cita o ESBMC como objeto de etudo (avaliação): 1
\item autores do software, apresentam, publicam o software neste artigo: 1
\item autores do software, artigo também apresenta o software como contribuição, mesmo ano: 1
\item autores do software, testa o software: 1
\item avalia ferramentas livres de analise estatica, mygcc entre eles: 1
\item avalia ferramentas para bad smell disponível para download, whangler entre eles, faz uma revisão sistemática de literatura sobre ferramentas para detecção de 'bad smell': 1
\item avalia o software, o software é um trabalho anterior: 1
\item avalia o software, usa o software como objeto de estudo: 1
\item avalia uma séria de ferramentas, dentre eles o LoopFrog: 1
\item cita ECBMC mas faz referencia a um artigo com o nome do software: 1
\item cita como exemplo de software para regras de tradução para programas C: 1
\item cita em trabalhos futuros o artigo selecionado na revisão estruturada: 1
\item cita na fundamentação como exemplo de implementação sobre a geração de sequencias randomicas de chamada a métodos: 1
\item cita o algoritmo implementado no software como referencia e também avaliar sua implementação contra a nova abordagem desenvolvidda neste trabalho, cita mais de um artigo sobre o software na referencias: 1
\item cita o artigo selecionado na revisão estruturada, cita o software como resultado deste paper tb, implementou uma nova abordagem no software, mesmo autor principal: 1
\item cita o formato srcML como suportado pela ferramenta proposta CHIVE, usa o formato XML srcML: 1
\item cita o nome do software uma vez, mas o artigo está escrito em russo: 1
\item cita o software GRT como uma ferramenta para geração automática de casos de teste baseado no Randoop, usa o GRT para coletar dados para o estudo, compara os dados com seus proprios dados coletados com implementação própria: 1
\item cita o software ao falar de Action Language Verifier (ALV) que usa o software, o autor principal é o mesmo do artigo selecionado na revisão estruturada: 1
\item cita o software ao relatar um keynote na conferencia CSTVA: 1
\item cita o software com referencia de especificação de sintaxe, implementada na proposta de ferramenta JGroovy, avalia os resultados da implementação própria contra o software JastAdd: 1
\item cita o software como a primeira técnica baseada em especificação para 'bounded-exhaustive testing of object-oriented programs' publicado no ASE 2001: 1
\item cita o software como base para implementação da ferramenta Jifclipse, implementa um superconjunto da linguagem jflow: 1
\item cita o software como compatível com a ferramenta aDeryaft: 1
\item cita o software como exemplo: 9
\item cita o software como exemplo de 'automated generation of test input': 1
\item cita o software como exemplo de 'bounded model checking', cita o artigo onde foi proposto: 1
\item cita o software como exemplo de 'def-use analysis': 1
\item cita o software como exemplo de 'model checker': 1
\item cita o software como exemplo de 'security-typed languages': 1
\item cita o software como exemplo de abordagem para 'data race detection' em programas C/OpenMP': 1
\item cita o software como exemplo de extensão de uma linguagem existente: 1
\item cita o software como exemplo de ferramenta para 'efficient bounded model-checking': 1
\item cita o software como exemplo de ferramenta para gerar documentação a partir de análise de código fonte: 1
\item cita o software como exemplo de ferramenta para recomendação de API numa tabela comparativa: 1
\item cita o software como exemplo de ferramenta para refatoração de programas Erlang: 1
\item cita o software como exemplo de framework para geração de casos de teste baseados em execução simbólica: 1
\item cita o software como exemplo de técnica para reparo de caasos de teste: 1
\item cita o software como ferramenta para análise de código fonte: 1
\item cita o software como notável exemplo, muito adotado, de ferramenta para análise de código fonte: 1
\item cita o software como opção de parametro para a ferramenta Lazy-CSeq como backend: 1
\item cita o software como opção de uso numa proposta de implementação, proposta de metodologia: 1
\item cita o software como possibilidade de integração com o nova proposta apresentada no paper chamada TARA, é capaz de interpretar representações geradas pelo ESBMC e outros softwares: 1
\item cita o software como possibilidade de uso no estudo: 1
\item cita o software como possível incorporação à solução apresentada no artigo, cita o software como exemplo de ferramenta 'information flow checker': 1
\item cita o software como trabalho anterior sobre Rapid Type Analysis (RTA), construção de call graphs: 1
\item cita o software como trabalho relacionado: 118
\item cita o software como trabalho relacionado - artigo em ptbr: 1
\item cita o software como um exemplo de 'attribute grammar compiler': 1
\item cita o software como um exemplo de combinação entre técnicas de execução simbólica, checagem de modelos e 'automated deduction to modularly reason about deep semantic properties of open object-oriented systems': 1
\item cita o software como um recente esforço da área 'modeling multiple interacting concurrency models': 1
\item cita o software como uma das primeiras grandes aplicações do Stratego/XT: 1
\item cita o software como uma ferramenta open-source feita com suporte a SRMA, fezendo interessante para propósitos acadêmicos: 1
\item cita o software como uma referencia de sistemas para otimizações de programas C++ de domínios específicos implementado em Stratego: 1
\item cita o software como único projeto de pesquisa que tem um protótipo para 'type inference' que resolve ambiguidades sintáticas, mas infelizmente não é possível fazer uma avaliação completa do software pois não é publicamente disponível: 1
\item cita o software em trabalhos futuros: 3
\item cita o software em trabalhos futuros, cita nas referencias um artigo sobre o software, mas nao o mesmo selecionado na revisao estruturada: 1
\item cita o software entre outros como software de 'defect reporting', descreve a publicação e melhorias posteriores implementadas: 1
\item cita o software mas o artigo está escrito em alemão: 1
\item cita o software na introdução como exemplo de sistema de gramática de atributo: 1
\item cita o software na introdução fazendo referência ao artigo selecionado na revisão estruturada, o autor aqui é um dos autores: 1
\item cita o software na seção de ameaças a validade como um exemplo de implementação de algoritmo de diferenciação baseada em árvores: 1
\item cita o software na seção de limitações como exemplo de implementação de algoritmo de diferenciação em árvore a ser avaliado: 1
\item cita o software nas notas de agradecimentos finais: 1
\item cita o software nas referencias do trabalho: 1
\item cita o software numa referencia a outros autores utilizando como ferramenta para detectar comentários: 1
\item cita o software numa tabela com inumeros softwares encontrados na literatura de avaliação de configuração: 1
\item cita o software, cita algumas contribuições feita a partir de outros papers, com citação a eles: 1
\item cita o software, entre outros, como opção de analisador de software, apresenta uma tabela com descrição de cada software, qual a dispoibilidade (free, demo, livre), indica site do software: 1
\item cita o software, entre outros, num estudo sobre design e implementação de 'domain-specific aspect languages': 1
\item cita o software, entre outros, numa tabela resumindo componentes para construir 'static fact extractors', cita o artigo selecionado na revisão estruturada: 1
\item cita o srcML como exemplo de formato para representação de código fonte em XML: 1
\item cita que o datased do SECOLD pode ser acessado por outras aplicações, cita o software como exemplo de consulta via HTTP: 1
\item cita que o software pode ser utilizado como backend do toolbox DSVerifier: 1
\item cita um outro artigo sobre o software: 1
\item cita um tal MSP-GCC, n tenho certeza de ser o MSP: 1
\item compara o formato (XML) srcML com outros formatos: 1
\item compara o novo software LLBMC com o ESBMC e outro: 1
\item compara o software com o Crowd::Debug: 1
\item compara o software com o HaRe (Haskell Refactorer), o HaRe é um projeto dos autores, o software Wrangler também tem contribuição dos autores: 1
\item compara os resultados com outros softwares, faz contribuição ao software, não deixa claro mas aparantemente enviou as modificações ao projeto/repositorio oficial: 1
\item compara os resultados obtidos com a implementação própria contra o ESBMC e outro: 1
\item compara uma nova ferramenta com o software e mais duas: CBMC e LLBMC: 1
\item compara uma nova ferramenta com o software e outras duas: CBMC e CORRAL: 1
\item compara/avalia o software com mais outros dois, os autores são quase os mesmos, cita que o software foi um trabalho anteior: 1
\item contribui com o ESBMC++, que por sua vez é baseado no ESBMC e faz parte do mesmo projeto, aparentemente: 1
\item contribui com o software adicionando um algoritmo de 'operational model' para suporte a C++, chama a cntribuição de ESBMC++, apresenta o software, cita o software nas referencias: 1
\item contribui com um método de codificação que transforma código ccJava em modelo Alloy (alloy.mit.edu), o software é um trabalho anterior dos mesmos autores: 1
\item cria uma taxonomia para mineração de código fonte, ferramentas e técnicas, cita o software como exemplo de técnica de 'API Usage pattern': 1
\item demonstra o uso do software: 1
\item descreve o software, implementa uma nova abordagem usando JastAdd, usa o software: 1
\item descreve o software, indica site do software https://github.com/cwi-swat/php-analysis , usa o software: 1
\item descreve os últimos desenvolvimentos do software: 1
\item descreve um conjunto de softwares, descreve a ferramenta Xogastan, seu funcionamento, como é implementada: 1
\item desenvolve uma ferramenta com base na biblioteca srcML: 1
\item desenvolveram o software em trabalhos anteriores, demonstra como realizar build do software: 1
\item desenvolveram o software, autores também, para automatizar transformações: 1
\item dois dos autores incluindo o principal são os mesmos do artigo selecionado na revisão estruturada, mesmo ano de publicação, apresenta o software como também uma contribuição do estudo: 1
\item em trabalho anterior extendeu o ESBMC, neste artigo continuam e avaliam as abordagens utilizadas para verificação de modelos para softwares ANSI-C multi-thread: 1
\item extende o JastAdd extensible Java compiler (JastAddJ): 1
\item extende o software CPAchecker para suportar verificação de regreção com precisão de reuso, o CPAchecker é o mesmo software CPA+, mas os dados e endereço publicados originalmente no artigo selecionado na revisão estruturada estão obsoletos, o software mudou de nome e de endereço, o site do projeto agora é https://cpachecker.sosy-lab.org: 1
\item extende o software com um framework chamado KUnit para geração automáticaa de testes JUnit e visualização de objetos da pilha, são os mesmos autores do software original, apenas um autor diferente: 1
\item extende o software com uma série de refatorações, compara o software com outras alternativas: 1
\item extende o software para suportar análise dinâmica: 1
\item faz referencia ao formato srcML: 1
\item faz um mapeamento sistemático do estado da arte sobre mudanças em softwares, o GumTree é citado numa classificação de ferramentas: 1
\item faz uma comparação com Action Language Verifier (ALV) e cita o software apenas para apresentar o ALV que usa composite, cita o artigo relecionado na revisão estruturada: 1
\item faz uma revisão sistemática e encontra o software I2SD, descreve algumas de suas caracteristicas, cita o artigo selecionado na revisão estruturada: 1
\item faz uma revisão sistemática sobre análise estática e Java reflection, cita o software entre inúmeros outros de análise estática encontrados, caracteriza esses softwares: 1
\item faz uma revisão sobre ferramentas de desenvolvimento e teste para Erlang, cita o software dentre as ferramentas encontradas: 1
\item faz uma revisão sobre linguagens de script, cita o software pois possui capacidade de refatoracao e inspecao baseada em script: 1
\item ilustra o uso do software proposto SeCold, cita o software como exemplo de aplicaçoes de terceiro que podem consultar os dados via query HTTP: 1
\item implementa 'adjustable-block encoding' para o CPAchecker/CPA+, informam que as contribuições estão no site do projeto: 1
\item implementa melhorias no algoritmo de execução simbólica para programas orientado a objetos, mesmos autores: 1
\item implementa melhorias no software para detecção de clone: 1
\item implementa melhorias no software, chama de 'our PHP static analysis': 1
\item implementa o algoritmo do software numa outra ferramenta e faz uma comparação entre as duas implementações: 1
\item implementa um algoritmo de 'slicing' com base no software Indus: 1
\item implementa um estratégia para melhorar a escalabilidade do software, apresenta a teoria utilizada pelos autores originais de ferramenta, cita um artigo sobre o software nas referencias: 1
\item implementa um extensão do software: 1
\item implementa um novo software ESBMC-MC, baseado no próprio software, não deixa claro se enviou as contribuições de volta: 1
\item implementa um novo software Proactive-Debugger, utiliza o ESBMC no experimento e compara os resultados obtidos: 1
\item implementa um novo software chamado BugAID, usa um fork do software GumTree para análise de dados: 1
\item implementa um novo software em cima do srcML: 1
\item implementa um protótipo IMPARA http://www.cprover.org/concurrent-impact e compara a performance com outras ferramentas, dentre elas o ESBMC: 1
\item implementa um protótipo Proactive-Debugger, compara os resultados obtidos pela nova ferramenta e o ESBMC: 1
\item implementa um protótipo chamado LazySMA usando como base o código do software, não disponibiliza o código do protótipo, os autores do software selecionado na revisão estruturada estão todos incluídos aqui: 1
\item implementa uma estensão no JastAddJ disponível em http://jastadd.org/web/jastaddj/refactoring.php (not found em Tue Aug 22 06:10:19 UTC 2017): 1
\item implementa uma extensão do software chamada ESBMC-GPU para verificação de programas GPU escritos para o framework CUDA, disponibilizado em http://esbmc-gpu.org https://github.com/ssvlab/esbmc-gpu: 1
\item implementa uma extensão do software, indica o endereço onde está disponível http://users.ecs.soton.ac.uk/gp4/cseq/files/lazy-cseq-1.0.tar.gz: 1
\item implementa uma extensão no CPAchecker (mesmo CPA+) para 'conditional model checking', descreve onde as contribuições estão disponíveis indicando o site do próprio projeto original https://cpachecker.sosy-lab.org: 1
\item implementa uma extensão para o JastAdd Extensible Java Compiler (JastAddJ) e integra ao Eclipse: 1
\item implementa uma extensão para o software: 1
\item implementa uma ferramenta UNICOEN e compara ela com diversos softwares, entre eles o srcML: 1
\item implementa uma nova ferramenta SIFT, faz referencia a estudo anterior com mygcc: 1
\item implementa uma nova ferramenta VERIFYR, usa o ESBMC para o experimento juntamente com sua propria implementacao, compara os resultados: 1
\item implementa uma nova ferramenta XQuery-based Analysis Framework (XAF), usa o software: 1
\item implementa uma nova ferramenta usando o software Indus como base: 1
\item implementa uma solução BIDTEXT em cima do software WALA: 1
\item implementa uma solução em cima do software: 1
\item implementou melhoria a 'unparsed patterns' no software, cita o software nas referencias, informa site do projeto: 1
\item implementou novas técnicas de 'error-witness-driven program analysis' no CPAchecker, o mesmo software CPA+, mas os autores no artigo original não faziam ainda referencia ao nome CPAchecker e sim CPA+: 1
\item integra o software no framework PaRTE: 1
\item melhora o software para incorporar instruções extraidos de bibliotecas de testes, não deixa claro se envia as melhorias de volta ao software: 1
\item mesmo autor do software, cita como trabalho anterior: 1
\item mesmos autores, cita o software como trabalho anterior, o paper publica um software chamado SPACE, uma implementação open source disponibilizado em http://www.cs.berkeley.edu/~jnear/space , uma versão modificada do software (derailer), aparentemente é uma evolução do derailer com outro nome, mais abrangente: 1
\item mesmos autores, o software é um trabalho anterior, usa o software: 2
\item modifica o software e usa para coleta de dados, não informa se enviou de volta as contribuições: 1
\item mostra como funcionalidades implementadas no software pode ser refatorada para integrar na ferramenta QuickCheck: 1
\item o estudo é realizado em dados de estudos realizados anteriormente, benchmarks obtidos de aplicações bem conhecidas, ESBMC entre elas: 1
\item o software foi utilizado para coleta/análise de dados do estudo: 1
\item o software é citado como 'our tool', apresenta o software, as pesquisas que deram origem ao software, como ele participa de um ecossistema com outras ferramentas: 1
\item o software é citado numa tabela com características de 24 projetos: 1
\item o software é implementado usando o framework Bogor, o autor não deixa claro onde estão as contribuições ao Bogor, mas no repositório do Bogor existe uma pasta com o nome do software Kiasan, aparentemente são as contribuições feitas ao framework: 1
\item os dados e código são disponibilizados em http://esbmc.org/benchmarks, o software é usado no experimento como motor de verificação: 1
\item os mesmos autores, exceto por 1 autor a mais no artigo selecionado na revisão estruturada, mesmo ano, alguns meses antes do artigo da revisao estruturada, um artigo resumido mas apresentando o software basicamente como no artigo original: 1
\item pelo título parece citar, mas o artigo não foi encontrado para download: 1
\item propoe um framework Whispec construido em cima do TestEra, avalia os dois softwares: 1
\item propõe um sistema que usa o formato de XML definido pelo software, XSDML: 1
\item propõe uma abordagem para 'automatically introducing alternative data structures to support parallelism' integrada ao software: 1
\item publica um novo software que usa o WALA, cita o software nas referencias com site do projeto: 1
\item publica um software ChangeScribe que usa o software JStereoCode: 1
\item publica um software chamado BMCLua, transforma codigo Lua em C para entao ser utilizado como entrada para o ESBMC, apresenta o ESBMC: 1
\item publica uma ferramenta JESVisT usando Bandera e Indus, cita dois artigos nas referencias sobre o software Indus: 1
\item revisão sistemática sobre 'environment modeling' encontrou 2 estudos primários sobre o software, descreve em 1 parágrafo: 1
\item sao autores do software, desenvolveram, usa o formato srcML neste estudo: 1
\item são os mesmos autores do artigo selecionado na revisão estruturada, publicado no mesmo ano 1 mês depois: 1
\item teste e avalia o software, formaliza a definição da linguagem declarativa usada no software, mesmo autor: 1
\item testou o software como candidado para o estudo mas não gostou dos resultados e preferiu outra solução: 1
\item todos os autores do artigo selecionado na revisao estruturada estão na lista de autores aqui, publicado no mesmo mês e ano, apresenta a ferramenta com detalhes: 1
\item uma competição onde o software ficou em primeiro no ranking, apresenta o software: 1
\item usa e avalia o software contra outra ferramenta: 1
\item usa o GRT para coleta de dados do estudo, usa 1p projetos de software livre populares como objeto de estudo, cita 3 artigos sobre o software nas referencias: 1
\item usa o formato do software (formato XML-based srcML), indica site do projeto: 1
\item usa o software: 14
\item usa o software (o formato srcML) para implementar numa nova ferramenta srcSlice, disponibiliza a nova ferramenta em www.srcML.org: 1
\item usa o software como base para implementar parte da nova ferramenta JastAddJ, grammar, frontend, backend: 1
\item usa o software como motor de verificação, e cita o seguinte a respeito da escolha, é uma das ferramentas BMC mais eficientes segundo a competição de softwares de verificação em 2013 e 2014: 1
\item usa o software como objeto de estudo: 1
\item usa o software como objeto de estudo, analisado com implementação própria chamada Revealer: 1
\item usa o software como objeto de estudo, dentre outros projetos: 1
\item usa o software como objeto de estudo, dentre outros projetos, avalia e caracteriza estes projetos: 1
\item usa o software como parte de uma solução implementada: 1
\item usa o software contra uma versao modificada pelos autores deste artigo do mesmo software, disponibiliza em https://github.com/jyi/ITSP: 1
\item usa o software e avalia os resultados: 1
\item usa o software na implementação de um protótipo: 1
\item usa o software na implementação própria, enviou sugestão de melhorias para o software: 1
\item usa o software no estudo como parte de uma metodologia proposta no artigo, e cita que a ferramenta tem pouca adoção na indústria: 1
\item usa o software num estudo de caso, cita http://jastadd.org: 1
\item usa o software num experimento, compara os resultados com o mesmo experimento com outro software: 1
\item usa o software para análise (presto.cse.ohio-state.edu/tacle) e coleta de dados, cita planos de implementar melhorias no software: 1
\item usa o software para análise de dados: 2
\item usa o software para análise de dados, cita url do software: 1
\item usa o software para análise de dados, compara os resultados do software com o CBMC: 1
\item usa o software para análise de dados, compara resultados com outros softwares: 1
\item usa o software para análise dos dados com objetivo de demonstrar a efetividade da técnica de 'environment generation': 'The results show that environment generation implemented in BEG is an aggressive reduction technique.', os autores usaram o software em trabalhos anteriores, dedica uma sessão para descrever o software, cita o software nas referências com indicação de site, avalia o software: 1
\item usa o software para avaliar uma técnica proposta no estudo: 1
\item usa o software para coleta de dados: 6
\item usa o software para coleta de dados integrado sob o Smother: 1
\item usa o software para coleta de dados, apresenta o sofware com detalhes, usa para gerar ambientes para dois módulos de um grande aplicação Java cliente-servidor comercial: 1
\item usa o software para coleta e análise de dados: 3
\item usa o software para coleta e análise de dados, a escolha é justificada por ser muito eficiente e flexivel para construir analisadores estáticos: 1
\item usa o software para coleta/análise de dados: 8
\item usa o software para compor a solução apresentada, Jrbx (Java Refactoring Browser): 1
\item usa o software para fazer análise, usa para 'sequentialize a simple program which invokes a limited number of library methods: 1
\item usa o software para gerar entrada para o software proposto, aparentemente contribui de volta com o srcML, n deixa claro: 1
\item usa o software para implementar a semantica de uma linguagem: 1
\item usa o software para implementar o compilador do JModelica.org (JMC): 1
\item usa o software para implementar parte do SystemJ: 1
\item usa o software para implementar um parser C++ para o OCCF: 1
\item usa o software para implementar um protótipo: 1
\item usa o software para implementar uma ferramenta: 1
\item usa o software para implementar uma solução: 1
\item usa o software, cita um artigo sobre o software: 1
\item usa o software, entre outros, para coletar e analisar dados: 1
\item usa uma versão modificada do software por outros autores para coleta de dados: 1
\item usou o software em trabalho anterior: 1
\item usou o software num experimento para análise de dados, detalha bastante o software, usaram o software em trabalhos anteriores, extende o software para lidar com bibliotecas J2EE: 1
\item utiliza o software para realizar análise de dados, os autores utilizaram a ferramenta em trabalhos anteriores, descreve bem a ferramenta e indica site do projeto: 1
\item é basicamente o mesmo artigo selecionado na revisão sistemática, os autores apresentam basicamente o mesmo conteúdo, mesmos autores: 1
\end{itemize}

