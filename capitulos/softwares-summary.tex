
% o conteúdo deste arquivo é gerado automaticamente pelo script
% bin/softwares-summary, favor não editar manualmente

\xchapter{Softwares acadêmicos}{Este capítulo ...}
\label{softwares-summary}

\section{2LS - 2nd order Logic Solving}

Análise de terminação para programas C usando resumo interprocedural baseado em modelos
publicado no artigo {\it Synthesising Interprocedural Bit-Precise Termination Proofs (T)}
por H. Y. Chen
no ASE 2015,
disponibilizado em \url{http://svn.cprover.org/wiki/doku.php?id=2ls for program analysis}
como software livre
sob uma licença BSD.

Software com lançamentos ocsaionais,
7 versões lançadas
entre 2015 e 2017,
escrito em C++,
uma busca por citações no {\bf IEEE Xplore} por
\texttt{(('2nd order Logic Solving') AND 2LS)}
e no {\bf ACM} com
\texttt{content.ftsec:(2LS) AND (order) AND (Logic) AND (Solving)}
retornou
37 resultados,
nenhum faz referência ao software.


\section{AccessAnalysis}

Cálculo de métricas IGAT e IGAM
publicado no artigo {\it AccessAnalysis: A Tool for Measuring the Appropriateness of Access Modifiers in Java Systems}
por C. Zoller
no SCAM 2012,
disponibilizado em \url{http://accessanalysis.sourceforge.net}
como software livre
sob uma licença EPL.

Software considerado obsoleto,
4 versões lançadas
entre 2010 e 2012,
escrito em Java,
uma busca por citações no {\bf IEEE Xplore} por
\texttt{(AccessAnalysis)}
e no {\bf ACM} com
\texttt{content.ftsec:(+AccessAnalysis +Tool +Java +Modifiers)}
retornou
8 resultados,
apenas um faz referência ao software.

\begin{itemize}
\item Measuring Inappropriate Generosity with Access Modifiers in Java Systems (?)
\end{itemize}

\section{APIExample}

Extração de informações de API Java e documentação automática com exemplos
publicado no artigo {\it APIExample: An effective web search based usage example recommendation system for java APIs}
por Lijie Wang
no ASE 2011,
disponibilizado em \url{http://www.apiexample.com}
mas inacessível em 09/08/2017.

Software sem informações sobre lançamentos ou releases,
uma busca por citações no {\bf IEEE Xplore} por
\texttt{((APIExample) AND Java)}
e no {\bf ACM} com
\texttt{content.ftsec:(+APIExample +Java)}
retornou
16 resultados,
{\bf 3} fazem referência ao software.

\begin{itemize}
\item APIBook: An Effective Approach for Finding APIs (cita o software nos trabalhos relacionados)
\item Documenting APIs with examples: Lessons learned with the APIMiner platform (?)
\item Generating API-usage Example for Project Developers (cita o software nos trabalhos relacionados)
\end{itemize}

\section{BEG - Bandera environment generator}

Criação automática de ambientes para verificação de modelos Java
publicado no artigo {\it Automated environment generation for software model checking}
por O. Tkachuk
no ASE 2003,
disponibilizado em \url{http://bandera.projects.cs.ksu.edu/}
mas inacessível em 09/08/2017.

Software sem informações sobre lançamentos ou releases,
uma busca por citações no {\bf IEEE Xplore} por
\texttt{(("Bandera environment generator") AND BEG)}
e no {\bf ACM} com
\texttt{content.ftsec:(+"Bandera environment generator" +BEG)}
retornou
10 resultados,
{\bf 8} fazem referência ao software.

\begin{itemize}
\item Analyzing interaction orderings with model checking (?)
\item Application of Automated Environment Generation to Commercial Software (?)
\item Assume-guarantee verification of software components in SOFA 2 framework (cita o software como trabalho relacionado)
\item Combining Environment Generation and Slicing for Modular Software Model Checking (?)
\item Concurrent Testing of Java Components Using Java PathFinder (cita o software como trabalho relacionado)
\item Environment Modeling in Model-based Testing: Concepts, Prospects and Research Challenges: A Systematic Literature Review (?)
\item Environment generation for validating event-driven software using model checking (?)
\item Partial Verification of Software Components: Heuristics for Environment Construction (cita o software como trabalho relacionado)
\end{itemize}

\section{Kiasan/Bogor}

Verificação de modelos
publicado no artigo {\it Bogor/Kiasan: A k-bounded Symbolic Execution for Checking Strong Heap Properties of Open Systems}
por X. Deng
no ASE 2006,
disponibilizado em \url{http://bogor.projects.cs.ksu.edu/manual/}
como software livre
sob uma licença SAnToS Laboratory Open Academic License.

Software considerado obsoleto,
272 versões lançadas
entre 2003 e 2008,
escrito em Java,
uma busca por citações no {\bf IEEE Xplore} por
\texttt{((Kiasan/Bogor) OR Bogor/Kiasan)}
e no {\bf ACM} com
\texttt{content.ftsec:("Bogor/Kiasan")}
retornou
37 resultados,
{\bf 17} fazem referência ao software.

\begin{itemize}
\item A Publish-subscribe Architecture and Component-based Programming Model for Medical Device Interoperability (cita o software em trabalhos futuros)
\item Bogor: A Flexible Framework for Creating Software Model Checkers (?)
\item Combining Unit-level Symbolic Execution and System-level Concrete Execution for Testing Nasa Software (cita o software como trabalho relacionado)
\item DSD-Crasher: A Hybrid Analysis Tool for Bug Finding (cita o software como trabalho relacionado)
\item Domain-specific Model Checking Using The Bogor Framework (?)
\item Dynamic Data Flow Testing of Object Oriented Systems (cita o software como trabalho relacionado)
\item Efficient Modular Glass Box Software Model Checking (cita o software como trabalho relacionado)
\item Environment generation for validating event-driven software using model checking (?)
\item Formal Software Analysis Emerging Trends in Software Model Checking (?)
\item Generating unit tests using static analysis and contracts (cita o software como trabalho relacionado)
\item Kiasan/KUnit: Automatic Test Case Generation and Analysis Feedback for Open Object-oriented Systems (?)
\item Preliminary Design of a Unified JML Representation and Software Infrastructure (?)
\item Symbolic PathFinder: Symbolic Execution of Java Bytecode (cita o software como trabalho relacionado)
\item Symbolic execution for software testing in practice: preliminary assessment (?)
\item Towards A Case-Optimal Symbolic Execution Algorithm for Analyzing Strong Properties of Object-Oriented Programs (?)
\item Towards a Lazier Symbolic Pathfinder (?)
\item jStar: Towards Practical Verification for Java (cita o software como trabalho relacionado)
\end{itemize}

\section{ccJava}

Linguagem orientada a aspectos
publicado no artigo {\it An Aspect-oriented Weaving Mechanism Based on Component and Connector Architecture}
por Ubayashi, Naoyasu
no ASE 2007,
disponibilizado em \url{http://posl.minnie.ai.kyutech.ac.jp/}
mas inacessível em 09/08/2017.

Software sem informações sobre lançamentos ou releases,
uma busca por citações no {\bf IEEE Xplore} por
\texttt{(ccJava)}
e no {\bf ACM} com
\texttt{content.ftsec:(+ccJava)}
retornou
7 resultados,
{\bf 4} fazem referência ao software.

\begin{itemize}
\item Alloy-Based Lightweight Verification for Aspect-Oriented Architecture (?)
\item An Interface Mechanism for Encapsulating Weaving in Class-based AOP (?)
\item Archface: A Contract Place Where Architectural Design and Code Meet Together (?)
\item Pointcut-based Architectural Interface for Bridging a Gap Between Design and Implementation (?)
\end{itemize}

\section{CIVL - Concurrency intermediate verification language}

Framework para verificação de programas concorrentes
publicado no artigo {\it CIVL: Formal Verification of Parallel Programs}
por M. Zheng
no ASE 2015,
disponibilizado em \url{http://vsl.cis.udel.edu/civl/}
como software livre
sob uma licença GPL.

Software com lançamentos frequentes,
36 versões lançadas
entre 2015 e 2017,
escrito em C,
uma busca por citações no {\bf IEEE Xplore} por
\texttt{(('concurrency intermediate verification') AND CIVL)}
e no {\bf ACM} com
\texttt{content.ftsec:(+civl +concurrency +intermediate +verification +language)}
retornou
8 resultados,
{\bf 5} fazem referência ao software.

\begin{itemize}
\item Achieving Formal Parallel Program Debugging by Incentivizing CS/HPC Collaborative Tool Development (?)
\item CIVL Solutions to Verifythis 2016 Challenges (pelo título parece citar, mas o artigo não foi encontrado para download)
\item CIVL: the concurrency intermediate verification language (?)
\item Protocol-based Verification of Message-passing Parallel Programs (cita o software como trabalho relacionado)
\item The RERS 2017 Challenge and Workshop (Invited Paper) (?)
\end{itemize}

\section{CodeBoost}

Transformação source-to-source para otimização de programas C++
publicado no artigo {\it Design of the CodeBoost transformation system for domain-specific optimisation of C++ programs}
por O. S. Bagge
no SCAM 2003,
disponibilizado em \url{http://codeboost.org}
como software livre
sob uma licença GPL v2.

Software com lançamentos ocsaionais,
134 versões lançadas
entre 2000 e 2004,
escrito em C,
uma busca por citações no {\bf IEEE Xplore} por
\texttt{((CodeBoost) AND C++)}
e no {\bf ACM} com
\texttt{content.ftsec:(+CodeBoost +"C++" +Tool)}
retornou
25 resultados,
{\bf 13} fazem referência ao software.

\begin{itemize}
\item A Domain-specific Approach to Heterogeneous Parallelism (cita o software como trabalho relacionado)
\item A Heterogeneous Parallel Framework for Domain-Specific Languages (cita o software como trabalho relacionado)
\item A Nanopass Framework for Commercial Compiler Development (cita o software como trabalho relacionado)
\item Almost First-class Language Embedding: Taming Staged Embedded DSLs (cita o software como trabalho relacionado)
\item Annotating user-defined abstractions for optimization (cita o software como trabalho relacionado)
\item Generic Flow-sensitive Optimizing Transformations in C++ with Concepts (cita o software como trabalho relacionado)
\item Green-Marl: A DSL for Easy and Efficient Graph Analysis (cita o software como trabalho relacionado)
\item Integrating Semantics and Compilation: Using C++ Concepts to Develop Robust and Efficient Reusable Libraries (cita o software como trabalho relacionado)
\item Specifying transformation sequences as computation on program fragments with an abstract attribute grammar (cita o software como trabalho relacionado)
\item Strategic Programming Meets Adaptive Programming (?)
\item Stratego/XT 0.16: Components for Transformation Systems (?)
\item Transformations for abstractions (?)
\item When and How to Develop Domain-specific Languages (?)
\end{itemize}

\section{Composite Symbolic Library}

Verificação de modelos
publicado no artigo {\it Action Language Verifier}
por T. Bultan
no ASE 2001,
disponibilizado em \url{http://www.cs.ucsb.edu/~bultan/composite/}
gratuitamente
sem uma licença definida.

Software sem informações sobre lançamentos ou releases,
escrito em C,
uma busca por citações no {\bf IEEE Xplore} por
\texttt{((((composite) AND bultan) AND Action Language Verifier) OR ("Composite Symbolic Library"))}
e no {\bf ACM} com
\texttt{content.ftsec:(+composite +"Action Language Verifier")}
retornou
14 resultados,
{\bf 6} fazem referência ao software.

\begin{itemize}
\item Analyzing Tabular Requirements Specifications Using Infinite State Model Checking (?)
\item Automated model checking and testing for composite Web services (?)
\item Efficient Temporal-logic Query Checking for Presburger Systems (?)
\item Mixed Symbolic Representations for Model Checking Software Programs (?)
\item Model Checking Sequential Software Programs via Mixed Symbolic Analysis (?)
\item Realizability of conversation protocols with message contents (?)
\end{itemize}

\section{CPA+ - Configurable program analysis with dynamic precision adjustment}

Análise configurável de programa com ajuste dinâmico de precisão
publicado no artigo {\it Program Analysis with Dynamic Precision Adjustment}
por D. Beyer
no ASE 2008,
disponibilizado em \url{http://www.cs.sfu.ca/~dbeyer/blast_cpaplus/}
mas inacessível em 09/08/2017.

Software sem informações sobre lançamentos ou releases,
uma busca por citações no {\bf IEEE Xplore} por
\texttt{(('program analysis') AND cpa+)}
e no {\bf ACM} com
\texttt{content.ftsec:(+cpa +dbeyer)}
retornou
8 resultados,
{\bf 4} fazem referência ao software.

\begin{itemize}
\item Conditional Model Checking: A Technique to Pass Information Between Verifiers (?)
\item Precision Reuse for Efficient Regression Verification (?)
\item Predicate Abstraction with Adjustable-block Encoding (?)
\item Witness Validation and Stepwise Testification Across Software Verifiers (?)
\end{itemize}

\section{CSeq}

Transformação source-to-source para programas C concorrentes
publicado no artigo {\it CSeq: A concurrency pre-processor for sequential C verification tools}
por B. Fischer
no ASE 2013,
disponibilizado em \url{http://users.ecs.soton.ac.uk/gp4/cseq/files/cseq-0.5.zip}
como software livre
sob uma licença BSD.

Software sem informações sobre lançamentos ou releases,
escrito em C,
uma busca por citações no {\bf IEEE Xplore} por
\texttt{((CSeq) AND soton)}
e no {\bf ACM} com
\texttt{content.ftsec:(+cseq +sequential +verification +tool)}
retornou
16 resultados,
{\bf 4} fazem referência ao software.

\begin{itemize}
\item A Scala Library for Testing Student Assignments on Concurrent Programming (cita o software como trabalho relacionado)
\item Lazy Sequentialization for TSO and PSO via Shared Memory Abstractions (?)
\item Lazy-CSeq: A Context-Bounded Model Checking Tool for Multi-threaded C-Programs (?)
\item Tractable Refinement Checking for Concurrent Objects (?)
\end{itemize}

\section{DDVerify}

Verificação de Linux drivers através de checagem de modelos
publicado no artigo {\it Model Checking Concurrent Linux Device Drivers}
por Witkowski, Thomas
no ASE 2007,
disponibilizado em \url{http://www.verify.ethz.ch/ddverify}
mas inacessível em 09/08/2017.

Software sem informações sobre lançamentos ou releases,
uma busca por citações no {\bf IEEE Xplore} por
\texttt{(DDVerify)}
e no {\bf ACM} com
\texttt{content.ftsec:(+DDVerify)}
retornou
4 resultados,
{\bf 2} fazem referência ao software.

\begin{itemize}
\item Race Analysis for SystemC Using Model Checking (cita o software como trabalho relacionado)
\item Rules based automatic Linux Device Driver Verifier And Code Assistance (cita o software como trabalho relacionado)
\end{itemize}

\section{Derailer}

Localização de falhas de segurança em aplicações web
publicado no artigo {\it Derailer: Interactive Security Analysis for Web Applications}
por Near, Joseph P.
no ASE 2014,
disponibilizado em \url{http://people.csail.mit.edu/jnear/derailer}
como software livre
sob uma licença GPL v3.

Software considerado obsoleto,
2 versões lançadas
entre 2013 e 2014,
escrito em Ruby,
uma busca por citações no {\bf IEEE Xplore} por
\texttt{((Derailer) AND jnear)}
e no {\bf ACM} com
\texttt{content.ftsec:(+Derailer +analysis +security +web +tool +bugs +ruby)}
retornou
7 resultados,
apenas um faz referência ao software.

\begin{itemize}
\item Finding Security Bugs in Web Applications Using a Catalog of Access Control Patterns (?)
\end{itemize}

\section{Diagnosys}

Construção de interfaces de debug para o kernel Linux
publicado no artigo {\it Diagnosys: automatic generation of a debugging interface to the Linux kernel}
por T. F. Bissyandé
no ASE 2012,
disponibilizado em \url{http://momentum.labri.fr/projects/diagnosys}
mas inacessível em 09/08/2017.

Software sem informações sobre lançamentos ou releases,
uma busca por citações no {\bf IEEE Xplore} por
\texttt{((Diagnosys) AND debugging)}
e no {\bf ACM} com
\texttt{content.ftsec:(+"Diagnosys tool" +Debugging +Linux +"device drivers")}
retornou
9 resultados,
nenhum faz referência ao software.


\section{DOMPLETION}

Sugestão de código javascript
publicado no artigo {\it Dompletion: DOM-aware JavaScript Code Completion}
por Bajaj, Kartik
no ASE 2014,
disponibilizado em \url{https://github.com/saltlab/dompletion}
gratuitamente
sem uma licença definida.

Software sem informações sobre lançamentos ou releases,
escrito em Javascript,
uma busca por citações no {\bf IEEE Xplore} por
\texttt{content.ftsec:(+dompletion +JavaScript)}
e no {\bf ACM} com
\texttt{((dompletion) AND JavaScript)}
retornou
2 resultados,
apenas um faz referência ao software.

\begin{itemize}
\item LED: Tool for Synthesizing Web Element Locators (cita o software como trabalho relacionado)
\end{itemize}

\section{DRC - Dangling Reference Checker}

Análise estática para detecção de referências inválidas em código dinâmico PHP
publicado no artigo {\it Dangling references in multi-configuration and dynamic PHP-based Web applications}
por H. V. Nguyen
no ASE 2013,
disponibilizado em \url{http://home.engineering.iastate.edu/~hungnv/Research/DRC}
mas inacessível em 09/08/2017.

Software sem informações sobre lançamentos ou releases,
uma busca por citações no {\bf IEEE Xplore} por
\texttt{(DRC Dangling Reference Checker)}
e no {\bf ACM} com
\texttt{content.ftsec:(+DRC +Dangling +Reference)}
retornou
19 resultados,
{\bf 4} fazem referência ao software.

\begin{itemize}
\item Building Call Graphs for Embedded Client-side Code in Dynamic Web Applications (cita o software como trabalho relacionado)
\item Cross-language Program Slicing for Dynamic Web Applications (cita o software como trabalho relacionado)
\item DRC: A detection tool for dangling references in PHP-based web applications (?)
\item Identifying and Locating Interference Issues in PHP Applications: The Case of WordPress (cita o software como trabalho relacionado)
\end{itemize}

\section{e-munity}

Verificação de segurança
publicado no artigo {\it Scalable Security Verification of Software at Compile Time}
por S. Tlili
no SCAM 2014,
disponibilizado em \url{http://sourceforge.net/p/emunity/code/ci/master/tree/}
gratuitamente
sem uma licença definida.

Software sem informações sobre lançamentos ou releases,
escrito em C,
uma busca por citações no {\bf IEEE Xplore} por
\texttt{(e-munity)}
e no {\bf ACM} com
\texttt{content.ftsec:(+"e-munity")}
retornou
1 resultados,
nenhum faz referência ao software.


\section{EJB Interceptor Analyzer}

Criação de diagramas de sequência
publicado no artigo {\it I2SD: Reverse Engineering Sequence Diagrams from Enterprise Java Beans with Interceptors}
por S. Roubtsov
no SCAM 2011,
disponibilizado em \url{https://www.dropbox.com/s/glhg8any43lccgm/EJB.zip}
gratuitamente
sem uma licença definida.

Software sem informações sobre lançamentos ou releases,
escrito em Java,
uma busca por citações no {\bf IEEE Xplore} por
\texttt{(((I2SD) AND EJB) AND Java)}
e no {\bf ACM} com
\texttt{content.ftsec:(+I2SD +Java)}
retornou
5 resultados,
{\bf 2} fazem referência ao software.

\begin{itemize}
\item Detecting dependencies in Enterprise JavaBeans with SQuAVisiT (cita o software (I2SD) em trabalhos futuros)
\item Multilingual Source Code Analysis: A Systematic Literature Review (? (I2SD))
\end{itemize}

\section{Error Prone}

Localização de bugs em código Java construído em cima do compilador javac
publicado no artigo {\it Building Useful Program Analysis Tools Using an Extensible Java Compiler}
por E. Aftandilian
no SCAM 2012,
disponibilizado em \url{http://code.google.com/p/error-prone}
como software livre
sob uma licença Apache License v2.0.

Software com lançamentos frequentes,
22 versões lançadas
entre 2015 e 2017,
escrito em Java,
uma busca por citações no {\bf IEEE Xplore} por
\texttt{((((('error-prone tool') AND Analysis) AND 'java compiler') AND 'error checks') AND javac)}
e no {\bf ACM} com
\texttt{content.ftsec:(+"error-prone" +tool +javac +analysis +"java compiler")}
retornou
47 resultados,
apenas um faz referência ao software.

\begin{itemize}
\item Tricorder: Building a Program Analysis Ecosystem (?)
\end{itemize}

\section{ESBMC}

Verificação de modelos
publicado no artigo {\it SMT-Based Bounded Model Checking for Embedded ANSI-C Software}
por L. Cordeiro
no ASE 2009,
disponibilizado em \url{http://users.ecs.soton.ac.uk/lcc08r/esbmc/}
mas inacessível em 09/08/2017.

Software sem informações sobre lançamentos ou releases,
uma busca por citações no {\bf IEEE Xplore} por
\texttt{(ESBMC)}
e no {\bf ACM} com
\texttt{content.ftsec:(+ESBMC)}
retornou
50 resultados,
{\bf 41} fazem referência ao software.

\begin{itemize}
\item A methodology for early functional verification of embedded software combining virtual platforms and bounded model checking (?)
\item Applying Multi-core Model Checking to Hardware-Software Partitioning in Embedded Systems (?)
\item Automated Testing of Definition-Use Data Flow for Multithreaded Programs (?)
\item Automatic and configurable instrumentation of C programs with temporal assertion checkers (?)
\item BLITZ: Compositional Bounded Model Checking for Real-world Programs (?)
\item BMCLua: Verification of Lua programs in digital TV interactive applications (?)
\item Bounded Model Checking of State-space Digital Systems: The Impact of Finite Word-length Effects on the Implementation of Fixed-point Digital Controllers Based on State-space Modeling (?)
\item Bounded model checking of C++ programs based on the Qt framework (?)
\item CSeq: A Concurrency Pre-processor for Sequential C Verification Tools (?)
\item Concurrency Testing Using Schedule Bounding: An Empirical Study (?)
\item Continuous Verification of Large Embedded Software Using SMT-Based Bounded Model Checking (?)
\item Correctness Witnesses: Exchanging Verification Results Between Verifiers (?)
\item Counterexample Guided Abstraction Refinement of Product-line Behavioural Models (cita o software como trabalho relacionado)
\item Debugging Assertion Failures in Software Controllers Using a Reference Model (?)
\item Debugging Multithreaded Programs Using Symbolic Analysis (?)
\item Debugging Multithreaded Programs as if They Were Sequential (?)
\item Efficient and Scalable Runtime Monitoring for Cyber--Physical System (?)
\item Fault Localization in Multi-threaded C Programs Using Bounded Model Checking (?)
\item Incremental Bounded Software Model Checking (?)
\item JFORTES: Java Formal Unit TESt Generation (?)
\item LLSPLAT: Improving Concolic Testing by Bounded Model Checking (cita o software como trabalho relacionado)
\item Lazy-CSeq: A Context-Bounded Model Checking Tool for Multi-threaded C-Programs (?)
\item Model Checking Embedded C Software Using k-Induction and Invariants (?)
\item Multi-pushdown systems with budgets (?)
\item Optimized hybrid verification of embedded software (?)
\item Over-approximating Loops to Prove Properties Using Bounded Model Checking (?)
\item Proteus: Computing Disjunctive Loop Summary via Path Dependency Analysis (cita ECBMC mas faz referencia a um artigo com o nome do software)
\item SMT-Based Bounded Model Checking of C++ Programs (?)
\item SMT-Based Context-Bounded Model Checking for Embedded Systems: Challenges and Future Trends (?)
\item SMT-based Verification Applied to Non-convex Optimization Problems (?)
\item Scalable hybrid verification for embedded software (?)
\item Sound static deadlock analysis for C/Pthreads (cita o software como trabalho relacionado)
\item Succinct Representation of Concurrent Trace Sets (?)
\item The bounded model checker LLBMC (?)
\item Transaction-based post-silicon debug of many-core System-on-Chips (?)
\item Verification of Delta Form Realization in Fixed-Point Digital Controllers Using Bounded Model Checking (?)
\item Verifying CUDA Programs Using SMT-based Context-bounded Model Checking (?)
\item Verifying Digital Systems with MATLAB (?)
\item Verifying Embedded C Software with Timing Constraints Using an Untimed Bounded Model Checker (?)
\item Verifying Multi-threaded Software Using Smt-based Context-bounded Model Checking (?)
\item Verifying multi-threaded software with impact (?)
\end{itemize}

\section{ETXL}

Transformação de código
publicado no artigo {\it Evolving TXL}
por A. D. Thurston
no SCAM 2006,
disponibilizado em \url{http://www.cs.queensu.ca/home/thurston/etxl}
mas inacessível em 09/08/2017.

Software sem informações sobre lançamentos ou releases,
uma busca por citações no {\bf IEEE Xplore} por
\texttt{(((ETXL) AND source) AND transformation)}
e no {\bf ACM} com
\texttt{content.ftsec:(+ETXL)}
retornou
10 resultados,
apenas um faz referência ao software.

\begin{itemize}
\item Eating Our Own Dog Food: DSLs for Generative and Transformational Engineering (?)
\end{itemize}

\section{FaultBuster}

Refatoração de code smells
publicado no artigo {\it FaultBuster: An automatic code smell refactoring toolset}
por G. Szőke
no SCAM 2015,
disponibilizado em \url{http://www.sed.inf.u-szeged.hu/FaultBuster}
gratuitamente
sob uma licença de demonstração.

Software sem informações sobre lançamentos ou releases,
uma busca por citações no {\bf IEEE Xplore} por
\texttt{(FaultBuster)}
e no {\bf ACM} com
\texttt{content.ftsec:(+FaultBuster)}
retornou
4 resultados,
nenhum faz referência ao software.


\section{Flowgen}

Criação automática de grafos UML
publicado no artigo {\it Flowgen: Flowchart-Based Documentation Framework for C++}
por D. A. Kosower
no SCAM 2014,
disponibilizado em \url{https://github.com/jlopezvi/Flowgen}
como software livre
sob uma licença GPL v3.

Software sem informações sobre lançamentos ou releases,
escrito em Python,
uma busca por citações no {\bf IEEE Xplore} por
\texttt{((Flowgen) AND C++)}
e no {\bf ACM} com
\texttt{content.ftsec:(+Flowgen)}
retornou
8 resultados,
{\bf 3} fazem referência ao software.

\begin{itemize}
\item Documentation Generation from Annotated Source Code of Scientific Software: Position Paper (?)
\item Kayrebt: An activity diagram extraction and visualization toolset designed for the Linux codebase (cita o software como trabalho relacionado)
\item POSITION PAPER: Documentation Generation from Annotated Source Code of Scientific Software (?)
\end{itemize}

\section{GRT - Guided Random Testing}

Geração automática de testes
publicado no artigo {\it GRT: Program-Analysis-Guided Random Testing (T)}
por L. Ma
no ASE 2015,
disponibilizado em \url{http://www.sites.google.com/site/grtprojectut/download}
mas inacessível em 09/08/2017.

Software sem informações sobre lançamentos ou releases,
uma busca por citações no {\bf IEEE Xplore} por
\texttt{((GRT) AND "Guided Random Testing")}
e no {\bf ACM} com
\texttt{content.ftsec:(+GRT) +"Guided Random Testing")}
retornou
13 resultados,
{\bf 7} fazem referência ao software.

\begin{itemize}
\item Automated Test Case Generation as a Many-Objective Optimisation Problem with Dynamic Selection of the Targets (cita o software como trabalho relacionado)
\item Classification of Randomly Generated Test Cases (?)
\item Efficient Search for Inputs Causing High Floating-point Errors (?)
\item GRT at the SBST 2015 Tool Competition (?)
\item Model-Based API Testing of Apache ZooKeeper (?)
\item Private API Access and Functional Mocking in Automated Unit Test Generation (cita na fundamentação como exemplo de implementação sobre a geração de sequencias randomicas de chamada a métodos)
\item Retrofitting automatic testing through library tests reusing (?)
\end{itemize}

\section{GUIZMO}

Inferência de layout
publicado no artigo {\it Model-driven Reverse Engineering of Legacy Graphical User Interfaces}
por S\'{a}nchez Ram\'{o}n, \'{O}scar
no ASE 2010,
disponibilizado em \url{http://modelum.es/trac/guizmo/}
como software livre
sob uma licença Apache License v2.0.

Software sem informações sobre lançamentos ou releases,
escrito em Java,
uma busca por citações no {\bf IEEE Xplore} por
\texttt{(guizmo)}
e no {\bf ACM} com
\texttt{content.ftsec:(+guizmo)}
retornou
0 resultados,
nenhum faz referência ao software.


\section{GumTree}

Comparação de mudanças
publicado no artigo {\it Fine-grained and Accurate Source Code Differencing}
por Falleri, Jean-R{\'e}my
no ASE 2014,
disponibilizado em \url{https://github.com/jrfaller/gumtree}
como software livre
sob uma licença LGPL v3.

Software com lançamentos ocsaionais,
3 versões lançadas
entre 2013 e 2015,
escrito em Java,
uma busca por citações no {\bf IEEE Xplore} por
\texttt{((GumTree) AND tool)}
e no {\bf ACM} com
\texttt{content.ftsec:(+GumTree +tool)}
retornou
37 resultados,
{\bf 18} fazem referência ao software.

\begin{itemize}
\item A Feasibility Study of Using Automated Program Repair for Introductory Programming Assignments (?)
\item API Code Recommendation Using Statistical Learning from Fine-grained Changes (?)
\item An Automated Framework for Recommending Program Elements to Novices (N) (cita o software como trabalho relacionado)
\item Automatic Clustering of Code Changes (cita o software na seção de limitações como exemplo de implementação de algoritmo de diferenciação em árvore a ser avaliado)
\item Can Testedness Be Effectively Measured? (?)
\item Changes As First-Class Citizens: A Research Perspective on Modern Software Tooling (?)
\item Codeflaws: A Programming Competition Benchmark for Evaluating Automated Program Repair Tools (?)
\item Computing counter-examples for privilege protection losses using security models (cita o software na seção de ameaças a validade como um exemplo de implementação de algoritmo de diferenciação baseada em árvores)
\item Discovering Bug Patterns in JavaScript (?)
\item Extracting Build Changes with BuildDiff (?)
\item Extracting executable transformations from distilled code changes (cita o software como trabalho relacionado)
\item Fixing Recurring Crash Bugs via Analyzing Q amp;A Sites (T) (?)
\item History Driven Program Repair (?)
\item Impact of Tool Support in Patch Construction (?)
\item Improving pattern tracking with a language-aware tree differencing algorithm (?)
\item Move-optimized Source Code Tree Differencing (?)
\item Preventing Defects: The Impact of Requirements Traceability Completeness on Software Quality (?)
\item S3: Syntax- and Semantic-guided Repair Synthesis via Programming by Examples (?)
\end{itemize}

\section{HUSACCT - HU Software Architecture Compliance Checking Tool}

verificação de conformidade arquitetural
publicado no artigo {\it HUSACCT: Architecture Compliance Checking with Rich Sets of Module and Rule Types}
por Pruijt, Leo J.
no ASE 2014,
disponibilizado em \url{http://husacct.github.io/HUSACCT}
como software livre
sob uma licença AGPL.

Software com lançamentos frequentes,
22 versões lançadas
entre 2013 e 2017,
escrito em Java,
uma busca por citações no {\bf IEEE Xplore} por
\texttt{(HUSACCT)}
e no {\bf ACM} com
\texttt{content.ftsec:(+HUSACCT)}
retornou
7 resultados,
{\bf 6} fazem referência ao software.

\begin{itemize}
\item A Genetic Approach to Architectural Pattern Discovery (?)
\item A Metamodel for the Support of Semantically Rich Modular Architectures in the Context of Static Architecture Compliance Checking (?)
\item Architectural Pattern Definition for Semantically Rich Modular Architectures (?)
\item Dependency Related Parameters in the Reconstruction of a Layered Software Architecture (?)
\item Dependency Types and Subtypes in the Context of Architecture Reconstruction and Compliance Checking (?)
\item Rule Type Based Reasoning on Architecture Violations: A Case Study (?)
\end{itemize}

\section{Indus}

Biblioteca de program slicing
publicado no artigo {\it An Overview of the Indus Framework for Analysis and Slicing of Concurrent Java Software (Keynote Talk - Extended Abstract)}
por V. P. Ranganath
no SCAM 2006,
disponibilizado em \url{http://indus.projects.cis.ksu.edu}
como software livre
sob uma licença EPL v1.0.

Software com lançamentos ?,
36 versões lançadas
entre 2005 e 2010,
escrito em Java,
uma busca por citações no {\bf IEEE Xplore} por
\texttt{("Indus Framework")}
e no {\bf ACM} com
\texttt{content.ftsec:(+"Indus Framework")}
retornou
6 resultados,
{\bf 3} fazem referência ao software.

\begin{itemize}
\item Debug Concurrent Programs with Visualization and Inference of Event Structure (?)
\item FlexSync: An aspect-oriented approach to Java synchronization (?)
\item LEAN: Simplifying Concurrency Bug Reproduction via Replay-supported Execution Reduction (?)
\end{itemize}

\section{JastAdd}

Análise de código-fonte através da descrição de atributos via gramática de atributos (AG)
publicado no artigo {\it Extending Attribute Grammars with Collection Attributes--Evaluation and Applications}
por Magnusson, Eva
no SCAM 2007,
disponibilizado em \url{http://jastadd.cs.lth.se/web}
como software livre
sob uma licença BSD License "modified".

Software com lançamentos frequentes,
24 versões lançadas
entre 2011 e 2017,
escrito em Java,
uma busca por citações no {\bf IEEE Xplore} por
\texttt{(JastAdd tool Attribute Grammars)}
e no {\bf ACM} com
\texttt{content.ftsec:(+JastAdd +tool +"Attribute Grammars" +"source code" +analysis)}
retornou
50 resultados,
{\bf 42} fazem referência ao software.

\begin{itemize}
\item A Comparison of Two Metacompilation Approaches to Implementing a Complex Domain-specific Language (?)
\item A Compiler Extension for Parallel Matrix Programming (cita o software como trabalho relacionado)
\item A Language Generic Solution for Name Binding Preservation in Refactorings (?)
\item A framework for nonlinear model-predictive control using object-oriented modeling with a case study in power plant start-up (?)
\item A tool for compiler construction based on aspect-oriented specifications (?)
\item Aspect-oriented prolog in a language processing context (cita o software como trabalho relacionado)
\item Automated Behavioral Testing of Refactoring Engines (?)
\item Automatic model translation to UML from software product lines model using UML profile (?)
\item Detecting Spring Configurations Errors (cita o software mas o artigo está escrito em alemão)
\item Domain-specific aspect languages for modularising crosscutting concerns in grammars (?)
\item Embedding Languages Without Breaking Tools (cita o software como trabalho relacionado)
\item Extending Languages by Leveraging Compilers: From Modelica to Optimica (?)
\item Extending the JastAdd Extensible Java Compiler to Java 7 (?)
\item Formal Semantics Based Translator Generation and Tool Development in Practice (?)
\item Generator of efficient strongly typed abstract syntax trees in Java (cita o software como trabalho relacionado)
\item Handling of layout-sensitive semantics in a visual control language (?)
\item Implementing attribute grammars using conventional compiler construction tools (?)
\item Improving Precision of Generated ASTs (cita o software como trabalho relacionado)
\item IncA: A DSL for the definition of incremental program analyses (cita o software como trabalho relacionado)
\item Incremental Evaluation of Higher Order Attributes (?)
\item Intercepting dataflow connections in diagrams with inheritance (?)
\item JGroovy - an extensible Java Programming Language with Groovy (?)
\item Java to hardware compilation for non data flow applications (?)
\item JavaCOP: Declarative Pluggable Types for Java (cita o software como trabalho relacionado)
\item Knowledge-Based Reconfiguration of Automation Systems (?)
\item M3: A general model for code analytics in rascal (?)
\item MetaCET: An Object Oriented Tool for Language Design (cita o software como trabalho relacionado)
\item Miniphases: Compilation Using Modular and Efficient Tree Transformations (cita o software como trabalho relacionado)
\item Mixing Source and Bytecode: A Case for Compilation by Normalization (?)
\item On the Impact of DSL Tools on the Maintainability of Language Implementations (cita o software na introdução como exemplo de sistema de gramática de atributo)
\item Refactoring is Not (Yet) About Transformation (?)
\item Separation of Concerns in Compiler Development Using Aspect-orientation (?)
\item Sound and Extensible Renaming for Java (?)
\item Specifying and Implementing Refactorings (?)
\item SugarJ: Library-based Syntactic Language Extensibility (cita o software como trabalho relacionado)
\item Term Rewriting with Traversal Functions (cita o software como trabalho relacionado)
\item The Fika Parser Generator (?)
\item The Jastadd Extensible Java Compiler (?)
\item The SystemJ approach to system-level design (?)
\item ThisType for Object-Oriented Languages: From Theory to Practice (?)
\item Transformations for abstractions (?)
\item Weaving a Debugging Aspect into Domain-specific Language Grammars (cita o software como trabalho relacionado)
\end{itemize}

\section{JFlow}

Transformação source-to-source
publicado no artigo {\it JFlow: Practical refactorings for flow-based parallelism}
por N. Chen
no ASE 2013,
disponibilizado em \url{http://vazexqi.github.io/JFlow/}
como software livre
sob uma licença Illinois/NCSA Open Source License.

Software considerado obsoleto,
5 versões lançadas
em 2012,
escrito em Java,
uma busca por citações no {\bf IEEE Xplore} por
\texttt{(JFlow tool Eclipse)}
e no {\bf ACM} com
\texttt{content.ftsec:(+JFlow +tool +Eclipse)}
retornou
16 resultados,
{\bf 6} fazem referência ao software.

\begin{itemize}
\item A Security-aware Refactoring Tool for Java Programs (?)
\item From Languages to Systems: Understanding Practical Application Development in Security-typed Languages (cita o software como trabalho relacionado)
\item Intransitive Noninterference in Dependence Graphs (cita o software como trabalho relacionado)
\item Jifclipse: Development Tools for Security-typed Languages (cita o software como base para implementação da ferramenta Jifclipse, implementa um superconjunto da linguagem jflow)
\item LeakProber: A Framework for Profiling Sensitive Data Leakage Paths (cita o software como trabalho relacionado)
\item SuVMF: Software-defined Unified Virtual Monitoring Function for SDN-based Large-scale Networks (cita o software como trabalho relacionado)
\end{itemize}

\section{JstereoCode}

Detecção de esteriótipos Java
publicado no artigo {\it JStereoCode: automatically identifying method and class stereotypes in Java code}
por L. Moreno
no ASE 2012,
disponibilizado em \url{http://www.cs.wayne.edu/~severe/revenge/}
mas inacessível em 09/08/2017.

Software sem informações sobre lançamentos ou releases,
uma busca por citações no {\bf IEEE Xplore} por
\texttt{(JstereoCode)}
e no {\bf ACM} com
\texttt{content.ftsec:(+JstereoCode)}
retornou
13 resultados,
{\bf 8} fazem referência ao software.

\begin{itemize}
\item A measure to assess the behavior of method stereotypes in object-oriented software (?)
\item Automatic generation of natural language summaries for Java classes (?)
\item ChangeScribe: A Tool for Automatically Generating Commit Messages (?)
\item Detecting Communities of Methods Using Dynamic Analysis Data (?)
\item JSummarizer: An automatic generator of natural language summaries for Java classes (?)
\item On Automatic Summarization of What and Why Information in Source Code Changes (?)
\item RCLinker: Automated Linking of Issue Reports and Commits Leveraging Rich Contextual Information (?)
\item Towards Prioritizing Documentation Effort (cita em trabalhos futuros o artigo selecionado na revisão estruturada)
\end{itemize}

\section{Jtop}

Gestão de casos de teste
publicado no artigo {\it Jtop: Managing JUnit Test Cases in Absence of Coverage Information}
por L. Zhang
no ASE 2009,
disponibilizado em \url{http://code.google.com/p/pku-jtop/}
mas inacessível em 09/08/2017.

Software sem informações sobre lançamentos ou releases,
uma busca por citações no {\bf IEEE Xplore} por
\texttt{(Jtop JUnit)}
e no {\bf ACM} com
\texttt{content.ftsec:(+Jtop +JUnit)}
retornou
4 resultados,
apenas um faz referência ao software.

\begin{itemize}
\item A Unified Test Case Prioritization Approach (?)
\end{itemize}

\section{Loopfrog}

Verificação de modelos
publicado no artigo {\it Loopfrog: A Static Analyzer for ANSI-C Programs}
por D. Kroening
no ASE 2009,
disponibilizado em \url{http://verify.inf.usi.ch/content/loopfrog}
gratuitamente
sem uma licença definida.

Software sem informações sobre lançamentos ou releases,
uma busca por citações no {\bf IEEE Xplore} por
\texttt{(Loopfrog)}
e no {\bf ACM} com
\texttt{content.ftsec:(+Loopfrog)}
retornou
6 resultados,
{\bf 4} fazem referência ao software.

\begin{itemize}
\item Program analysis too loopy? Set the loops aside (cita o software como trabalho relacionado)
\item Proving termination of imperative programs using Max-SMT (?)
\item RED: A Tool for Runtime Error Detection in C Programs Using Abstract Interpretation (cita o software como trabalho relacionado)
\item Termination Proofs from Tests (?)
\end{itemize}

\section{Lotrack}

Análise estática de configuração
publicado no artigo {\it Tracking Load-time Configuration Options}
por Lillack, Max
no ASE 2014,
disponibilizado em \url{https://github.com/MaxLillack/Lotrack}
gratuitamente
sem uma licença definida.

Software sem informações sobre lançamentos ou releases,
escrito em Java,
uma busca por citações no {\bf IEEE Xplore} por
\texttt{(Lotrack)}
e no {\bf ACM} com
\texttt{content.ftsec:(+Lotrack)}
retornou
4 resultados,
apenas um faz referência ao software.

\begin{itemize}
\item Multi-layer software configuration: Empirical study on wordpress (?)
\end{itemize}

\section{MPAnalyzer}

Análise de padrões disponível
publicado no artigo {\it MPAnalyzer: A Tool for Finding Unintended Inconsistencies in Program Source Code}
por Higo, Yoshiki
no ASE 2014,
disponibilizado em \url{https://github.com/YoshikiHigo/MPAnalyzer}
gratuitamente
sem uma licença definida.

Software sem informações sobre lançamentos ou releases,
escrito em Java,
uma busca por citações no {\bf IEEE Xplore} por
\texttt{(MPAnalyzer)}
e no {\bf ACM} com
\texttt{content.ftsec:(+MPAnalyzer)}
retornou
3 resultados,
nenhum faz referência ao software.


\section{MSP}

Construção de modelo formal de acesso a memória
publicado no artigo {\it Recovering Memory Access Patterns of Executable Programs}
por Ketterlin, Alain
no SCAM 2010,
disponibilizado em \url{http://icps.u-strasbg.fr/software/msp}
mas inacessível em 09/08/2017.

Software sem informações sobre lançamentos ou releases,
uma busca por citações no {\bf IEEE Xplore} por
\texttt{((((MSP) AND tool) AND Binary) AND "Program Analysis")}
e no {\bf ACM} com
\texttt{content.ftsec:(+MSP +tool +Binary +"Program Analysis")}
retornou
37 resultados,
apenas um faz referência ao software.

\begin{itemize}
\item TinyModules: Code module exchange in TinyOS (cita um tal MSP-GCC, n tenho certeza de ser o MSP)
\end{itemize}

\section{mygcc}

Verificação de programas C
publicado no artigo {\it A Portable Compiler-Integrated Approach to Permanent Checking}
por N. Volanschi
no ASE 2006,
disponibilizado em \url{http://mygcc.free.fr}
como software livre
sob uma licença GPL.

Software considerado obsoleto,
5 versões lançadas
,
escrito em C,
uma busca por citações no {\bf IEEE Xplore} por
\texttt{(mygcc)}
e no {\bf ACM} com
\texttt{content.ftsec:(+mygcc)}
retornou
7 resultados,
{\bf 6} fazem referência ao software.

\begin{itemize}
\item An evaluation of free/open source static analysis tools applied to embedded software (?)
\item Automatic defects detection in industrial C/C++ software (cita o nome do software uma vez mas o artigo está escrito em russo)
\item Condate: A Proto-language at the Confluence Between Checking and Compiling (?)
\item Program Sifting: Select Property-Related Functions for Language-Based Static Analysis (?)
\item Scalable Security Verification of Software at Compile Time (cita o software como trabalho relacionado)
\item Unparsed Patterns: Easy User-extensibility of Program Manipulation Tools (?)
\end{itemize}

\section{PARSEWeb}

Query para apoio e sugestão de reuso de bibliotecas
publicado no artigo {\it Parseweb: A Programmer Assistant for Reusing Open Source Code on the Web}
por Thummalapenta, Suresh
no ASE 2007,
disponibilizado em \url{http://ase.csc.ncsu.edu/parseweb}
mas inacessível em 09/08/2017.

Software sem informações sobre lançamentos ou releases,
uma busca por citações no {\bf IEEE Xplore} por
\texttt{(((PARSEWeb) AND tool) AND AST)}
e no {\bf ACM} com
\texttt{content.ftsec:(+PARSEWeb +tool +AST)}
retornou
49 resultados,
{\bf 23} fazem referência ao software.

\begin{itemize}
\item An evaluation of source code mining techniques (?)
\item Automated dependency resolution for open source software (cita o software como trabalho relacionado)
\item Automated library recommendation (cita o software como trabalho relacionado)
\item Bing Developer Assistant: Improving Developer Productivity by Recommending Sample Code (cita o software como trabalho relacionado)
\item Complete Completion Using Types and Weights (cita o software como trabalho relacionado)
\item Data-driven Synthesis for Object-oriented Frameworks (cita o software como trabalho relacionado)
\item Enabling Static Analysis for Partial Java Programs (?)
\item ExceptionTracer: A Solution Recommender for Exceptions in an Integrated Development Environment (cita o software como trabalho relacionado)
\item From Query to Usable Code: An Analysis of Stack Overflow Code Snippets (cita o software como trabalho relacionado)
\item Graph-based pattern-oriented, context-sensitive source code completion (cita o software como trabalho relacionado)
\item Identifier-Based Context-Dependent API Method Recommendation (cita o software como trabalho relacionado)
\item Improving software reliability and productivity via mining program source code (?)
\item Mining Coding Patterns to Detect Crosscutting Concerns in Java Programs (cita o software como trabalho relacionado)
\item Recommending API Methods Based on Identifier Contexts (cita o software como trabalho relacionado)
\item Recommending Proper API Code Examples for Documentation Purpose (cita o software como trabalho relacionado)
\item Refactoring with Synthesis (cita o software como trabalho relacionado)
\item Semantic Web - The Missing Link in Global Source Code Analysis? (?)
\item SnipMatch: Using Source Code Context to Enhance Snippet Retrieval and Parameterization (cita o software como trabalho relacionado)
\item Synthesizing Java Expressions from Free-form Queries (cita o software como trabalho relacionado)
\item The Object Repository: Pulling Objects out of the Ecosystem (cita o software como trabalho relacionado)
\item Towards Sharing Source Code Facts Using Linked Data (?)
\item Type-directed Completion of Partial Expressions (cita o software como trabalho relacionado)
\item Typestate-based Semantic Code Search over Partial Programs (cita o software como trabalho relacionado)
\end{itemize}

\section{PAT - Puzzle-Based Automatic Testing}

Ambiente de teste automático
publicado no artigo {\it Puzzle-based automatic testing: bringing humans into the loop by solving puzzles}
por N. Chen
no ASE 2012,
disponibilizado em \url{http://pat.cse.ust.hk:8080}
mas inacessível em 09/08/2017.

Software sem informações sobre lançamentos ou releases,
uma busca por citações no {\bf IEEE Xplore} por
\texttt{("Puzzle-based automatic testing")}
e no {\bf ACM} com
\texttt{content.ftsec:(+"Puzzle-based automatic testing")}
retornou
13 resultados,
apenas um faz referência ao software.

\begin{itemize}
\item Local-based active classification of test report to assist crowdsourced testing (cita o software como trabalho relacionado)
\end{itemize}

\section{PHP AiR}

Um framework para análise de código PHP escrito em Rascal
publicado no artigo {\it Static, Lightweight Includes Resolution for PHP}
por Hills, Mark
no ASE 2014,
disponibilizado em \url{https://github.com/cwi-swat/php-analysis}
gratuitamente
sem uma licença definida.

Software com lançamentos ?,
4 versões lançadas
entre 2011 e 2014,
escrito em Rascal,
uma busca por citações no {\bf IEEE Xplore} por
\texttt{("PHP AiR")}
e no {\bf ACM} com
\texttt{content.ftsec:(+"PHP AiR")}
retornou
11 resultados,
{\bf 8} fazem referência ao software.

\begin{itemize}
\item Evolution of dynamic feature usage in PHP (?)
\item M3: A general model for code analytics in rascal (?)
\item Navigating the WordPress plugin landscape (?)
\item Optimizing Hash-array Mapped Tries for Fast and Lean Immutable JVM Collections (?)
\item PHP AiR: Analyzing PHP systems with Rascal (?)
\item Query Construction Patterns in PHP (?)
\item Supporting PHP Dynamic Analysis in PHP AiR (?)
\item Variable Feature Usage Patterns in PHP (T) (?)
\end{itemize}

\section{protopurity}

Análise de impacto
publicado no artigo {\it Detecting function purity in JavaScript}
por J. Nicolay
no SCAM 2015,
disponibilizado em \url{https://github.com/jensnicolay/jipda/tree/scam2015/protopurity}
gratuitamente
sem uma licença definida.

Software sem informações sobre lançamentos ou releases,
escrito em Javascript,
uma busca por citações no {\bf IEEE Xplore} por
\texttt{content.ftsec:(+protopurity)}
e no {\bf ACM} com
\texttt{(protopurity)}
retornou
0 resultados,
nenhum faz referência ao software.


\section{Pseudogen}

Transformação de código-fonte em pseudo-código
publicado no artigo {\it Pseudogen: A Tool to Automatically Generate Pseudo-Code from Source Code}
por H. Fudaba
no ASE 2015,
disponibilizado em \url{http://ahclab.naist.jp/pseudogen}
gratuitamente
sem uma licença definida.

Software sem informações sobre lançamentos ou releases,
escrito em Python,
uma busca por citações no {\bf IEEE Xplore} por
\texttt{(Pseudogen)}
e no {\bf ACM} com
\texttt{content.ftsec:(+Pseudogen +"Pseudo-Code")}
retornou
5 resultados,
nenhum faz referência ao software.


\section{PtYasm}

Verificação de modelos
publicado no artigo {\it Augmenting Counterexample-Guided Abstraction Refinement with Proof Templates}
por T. E. Hart
no ASE 2008,
disponibilizado em \url{http://www.cs.toronto.edu/~tomhart/ptyasm}
gratuitamente
sem uma licença definida.

Software sem informações sobre lançamentos ou releases,
escrito em Java,
uma busca por citações no {\bf IEEE Xplore} por
\texttt{(PtYasm)}
e no {\bf ACM} com
\texttt{content.ftsec:(+PtYasm)}
retornou
2 resultados,
nenhum faz referência ao software.


\section{PuMoC}

Verificação de modelos
publicado no artigo {\it PuMoC: a CTL model-checker for sequential programs}
por F. Song
no ASE 2012,
disponibilizado em \url{http://www.liafa.jussieu.fr/~song/PuMoC}
mas inacessível em 09/08/2017.

Software sem informações sobre lançamentos ou releases,
uma busca por citações no {\bf IEEE Xplore} por
\texttt{(PuMoC)}
e no {\bf ACM} com
\texttt{content.ftsec:(+PuMoC)}
retornou
4 resultados,
apenas um faz referência ao software.

\begin{itemize}
\item PoMMaDe: Pushdown Model-checking for Malware Detection (cita o software como trabalho relacionado)
\end{itemize}

\section{PYTHIA}

Criação automática de casos de teste
publicado no artigo {\it PYTHIA: Generating test cases with oracles for JavaScript applications}
por S. Mirshokraie
no ASE 2013,
disponibilizado em \url{http://salt.ece.ubc.ca/software/pythia/}
mas inacessível em 09/08/2017.

Software sem informações sobre lançamentos ou releases,
uma busca por citações no {\bf IEEE Xplore} por
\texttt{((PYTHIA) AND JavaScript)}
e no {\bf ACM} com
\texttt{content.ftsec:(+PYTHIA +JavaScript)}
retornou
15 resultados,
apenas um faz referência ao software.

\begin{itemize}
\item Effective Test Generation and Adequacy Assessment for JavaScript-based Web Applications (?)
\end{itemize}

\section{ReAssert}

Localização de falhas em testes e refatoração
publicado no artigo {\it ReAssert: Suggesting Repairs for Broken Unit Tests}
por B. Daniel
no ASE 2009,
disponibilizado em \url{http://mir.cs.illinois.edu/reassert}
como software livre
sob uma licença Illinois/NCSA Open Source License.

Software considerado obsoleto,
5 versões lançadas
entre 2009 e 2010,
escrito em Java,
uma busca por citações no {\bf IEEE Xplore} por
\texttt{(ReAssert tool "Unit Tests")}
e no {\bf ACM} com
\texttt{content.ftsec:(+ReAssert +tool +"Unit Tests")}
retornou
28 resultados,
{\bf 12} fazem referência ao software.

\begin{itemize}
\item A Literature Review of Research in Software Defect Reporting (?)
\item Automatic Contract Insertion with CCBot (cita o software como trabalho relacionado)
\item Automatic Test Suite Evolution (?)
\item COPE: Vision for a Change-Oriented Programming Environment (?)
\item CrowdOracles: Can the Crowd Solve the Oracle Problem? (cita o software como trabalho relacionado)
\item Crowdsourcing Suggestions to Programming Problems for Dynamic Web Development Languages (?)
\item Evaluating the Extended Refactoring Guidelines (?)
\item Is It Dangerous to Use Version Control Histories to Study Source Code Evolution? (?)
\item On Test Repair Using Symbolic Execution (?)
\item ReAssert: a tool for repairing broken unit tests (?)
\item Understanding Myths and Realities of Test-suite Evolution (cita o software como trabalho relacionado)
\item WATER: Web Application TEst Repair (cita o software como trabalho relacionado)
\end{itemize}

\section{Rêve}

Verificação de regressão
publicado no artigo {\it Automating Regression Verification}
por Felsing, Dennis
no ASE 2014,
disponibilizado em \url{http://formal.iti.kit.edu/improve}
mas inacessível em 09/08/2017.

Software sem informações sobre lançamentos ou releases,
uma busca por citações no {\bf IEEE Xplore} por
\texttt{(Reve Automating regression verification)}
e no {\bf ACM} com
\texttt{content.ftsec:(+"Reve" +Automating +regression +verification)}
retornou
13 resultados,
nenhum faz referência ao software.


\section{RRFinder}

Mineração de especificação de liberação de recursos
publicado no artigo {\it Iterative mining of resource-releasing specifications}
por Q. Wu
no ASE 2011,
disponibilizado em \url{http://sa.seforge.org/RRFinder/}
mas inacessível em 09/08/2017.

Software sem informações sobre lançamentos ou releases,
uma busca por citações no {\bf IEEE Xplore} por
\texttt{(RRFinder)}
e no {\bf ACM} com
\texttt{content.ftsec:(+RRFinder)}
retornou
5 resultados,
{\bf 2} fazem referência ao software.

\begin{itemize}
\item Automated resource release in device drivers (cita o software como trabalho relacionado)
\item em-SPADE: A Compiler Extension for Checking Rules Extracted from Processor Specifications (cita o software como trabalho relacionado)
\end{itemize}

\section{Sapid/XML}

Representação intermediária de código Java usando XML ao invés de AST
publicado no artigo {\it A CASE tool platform using an XML representation of Java source code}
por K. Maruyama
no SCAM 2004,
disponibilizado em \url{http://www.jtool.org}
mas inacessível em 09/08/2017.

Software sem informações sobre lançamentos ou releases,
uma busca por citações no {\bf IEEE Xplore} por
\texttt{("Sapid/XML")}
e no {\bf ACM} com
\texttt{content.ftsec:(+"Sapid/XML")}
retornou
5 resultados,
{\bf 4} fazem referência ao software.

\begin{itemize}
\item An Accurate and Convenient Undo Mechanism for Refactorings (?)
\item An easy-to-use extension mechanism using XML for an integrated development environment (?)
\item Design and implementation of an extensible and modifiable refactoring tool (?)
\item Let's Make Refactoring Tools User-extensible! (cita o software como trabalho relacionado)
\end{itemize}

\section{Sonar Qube Plug-in}

Extende o SourceMeter com análise de código Java com o uso do SonarQube
publicado no artigo {\it Source Meter Sonar Qube Plug-in}
por R. Ferenc
no SCAM 2014,
disponibilizado em \url{http://github.com/FrontEndART/SonarQube-plug-in}
gratuitamente
sob uma licença FrontEndART Software Ltd.

Software com lançamentos frequentes,
4 versões lançadas
entre 2015 e 2016,
escrito em Java,
uma busca por citações no {\bf IEEE Xplore} por
\texttt{(SonarQube-plug-in)}
e no {\bf ACM} com
\texttt{content.ftsec:(+"SonarQube-plug-in")}
retornou
2 resultados,
nenhum faz referência ao software.


\section{SPARTA - Static Program Analysis for Reliable Trusted Apps}

Segurança pra detecção de malware
publicado no artigo {\it Static Analysis of Implicit Control Flow: Resolving Java Reflection and Android Intents (T)}
por P. Barros
no ASE 2015,
disponibilizado em \url{http://types.cs.washington.edu/sparta}
gratuitamente
sem uma licença definida.

Software com lançamentos ocsaionais,
14 versões lançadas
entre 2012 e 2016,
escrito em Java,
uma busca por citações no {\bf IEEE Xplore} por
\texttt{(SPARTA "Program Analysis")}
e no {\bf ACM} com
\texttt{content.ftsec:(+SPARTA +"Program Analysis")}
retornou
7 resultados,
{\bf 3} fazem referência ao software.

\begin{itemize}
\item Challenges for Static Analysis of Java Reflection - Literature Review and Empirical Study (?)
\item Designing Application Permission Models that Meet User Expectations (?)
\item Revealer: a lexical pattern matcher for architecture recovery (?)
\end{itemize}

\section{srcML}

Transformação source-to-source
publicado no artigo {\it Lightweight Transformation and Fact Extraction with the srcML Toolkit}
por M. L. Collard
no SCAM 2011,
disponibilizado em \url{http://www.sdml.info/projects/srcml/trunk}
como software livre
sob uma licença GPL v3.

Software com lançamentos ocsaionais,
14 versões lançadas
entre 2011 e 2015,
escrito em C++,
uma busca por citações no {\bf IEEE Xplore} por
\texttt{(srcML Toolkit Java C)}
e no {\bf ACM} com
\texttt{content.ftsec:(+srcML +Toolkit +Java +"C++")}
retornou
43 resultados,
{\bf 39} fazem referência ao software.

\begin{itemize}
\item A CASE tool platform using an XML representation of Java source code (?)
\item A Contextualized Vocabulary Model for identifying technical debt on code comments (?)
\item A Flexible Framework for Quality Assurance of Software Artefacts with Applications to Java, UML, and TTCN-3 Test Specifications (?)
\item A Framework for Analyzing and Transforming Source Code Supporting Multiple Programming Languages (?)
\item A Large-Scale Empirical Study on Self-Admitted Technical Debt (?)
\item A Tool for Efficiently Reverse Engineering Accurate UML Class Diagrams (?)
\item A lightweight transformational approach to support large scale adaptive changes (?)
\item Achievements and Challenges in Software Reverse Engineering (?)
\item An Exploratory Study on Self-Admitted Technical Debt (?)
\item An XML Based Approach to Support the Evolution of Model-to-model Traceability Links (?)
\item An improved XML syntax for the java programming language (?)
\item Automatic Generation of Release Notes (?)
\item CHIVE - a program source visualisation framework (?)
\item Clone detection in source code by frequent itemset techniques (cita o software em trabalhos futuros)
\item Design and implementation of an extensible and modifiable refactoring tool (?)
\item Discipline Matters: Refactoring of Preprocessor Directives in the \#ifdef Hell (?)
\item Empirically Examining the Parallelizability of Open Source Software System (?)
\item Evolving Requirements-to-Code Trace Links across Versions of a Software System (?)
\item Experience on applying software architecture recovery to automotive embedded systems (?)
\item Exploration, Analysis, and Manipulation of Source Code Using srcML (?)
\item Frontiers of reverse engineering: A conceptual model (?)
\item Hybrid Program Dependence Approximation for Effective Dynamic Impact Prediction (?)
\item Impact Analysis of Change Requests on Source Code Based on Interaction and Commit Histories (?)
\item Investigating the use of lexical information for software system clustering (?)
\item Locating candidate web service in legacy software: A search based approach (?)
\item Multi-objective Coevolutionary Automated Software Correction (?)
\item Natural language parsing for fact extraction from source code (cita o software como trabalho relacionado)
\item New Frontiers of Reverse Engineering (?)
\item OCCF: A Framework for Developing Test Coverage Measurement Tools Supporting Multiple Programming Languages (?)
\item On the evolution of mobile computing software systems and C/C++ vulnerable code: Empirical investigation (?)
\item Parsing Formal Languages Using Natural Language Parsing Techniques (?)
\item Program annotation in XML: a parse-tree based approach (cita o software como trabalho relacionado)
\item Tools in Mining Software Repositories (?)
\item Using stereotypes in the automatic generation of natural language summaries for C++ methods (?)
\item Vulnerable C/C++ code usage in IoT software systems (?)
\item XOgastan: XML-oriented gcc AST analysis and transformations (?)
\item srcML: An Infrastructure for the Exploration, Analysis, and Manipulation of Source Code: A Tool Demonstration (?)
\item srcQL: A syntax-aware query language for source code (?)
\item srcSlice: A Tool for Efficient Static Forward Slicing (?)
\end{itemize}

\section{SWAT - Search based Web Application Tester}

Teste automático para aplicação web
publicado no artigo {\it Automated Web Application Testing Using Search Based Software Engineering}
por Alshahwan, Nadia
no ASE 2011,
disponibilizado em \url{http://www.cs.ucl.ac.uk/staff/nalshahw/swat}
mas inacessível em 09/08/2017.

Software sem informações sobre lançamentos ou releases,
uma busca por citações no {\bf IEEE Xplore} por
\texttt{(SWAT Search Web Application Tester)}
e no {\bf ACM} com
\texttt{content.ftsec:(+SWAT +Search +Web +Application +Tester)}
retornou
15 resultados,
{\bf 7} fazem referência ao software.

\begin{itemize}
\item A Synthetic Workload Generation Technique for Stress Testing Session-Based Systems (? - cita uma ferramenta com o mesmo nome mas aparentemente trata-se de outra ferramenta - Session-based Web Application Tester - SWAT)
\item Automatically Generating Bursty Benchmarks for Multitier Systems (? - cita uma ferramenta com o mesmo nome mas aparentemente trata-se de outra ferramenta - Session-based Web Application Tester - SWAT)
\item Coverage and Fault Detection of the Output-uniqueness Test Selection Criteria (? - mesmos autores)
\item Exploiting Nonstationarity for Performance Prediction (? - cita uma ferramenta com o mesmo nome mas aparentemente trata-se de outra ferramenta - Session-based Web Application Tester - SWAT)
\item Extension of Selenium RC tool to perform automated testing with databases in web applications (cita o software como trabalho relacionado)
\item Setting Realistic Think Times in Performance Testing: A Practitioner's Approach (? - cita uma ferramenta com o mesmo nome mas aparentemente trata-se de outra ferramenta - Session-based Web Application Tester - SWAT)
\item State Aware Test Case Regeneration for Improving Web Application Test Suite Coverage and Fault Detection (? - mesmos autores)
\end{itemize}

\section{TACLE - Type Analysis and CalL graph construction for Eclipse}

Análise de tipo (Type Analysis) e construção e visualizaçao de grafos de chamada (Call Graph)
publicado no artigo {\it Estimating the Run-Time Progress of a Call Graph Construction Algorithm53-62}
por J. Sawin
no SCAM 2006,
disponibilizado em \url{http://presto.cse.ohio-state.edu/tacle}
gratuitamente
sem uma licença definida.

Software sem informações sobre lançamentos ou releases,
escrito em Java,
uma busca por citações no {\bf IEEE Xplore} por
\texttt{(TACLE "Type Analysis")}
e no {\bf ACM} com
\texttt{content.ftsec:(+TACLE +"Type Analysis")}
retornou
4 resultados,
{\bf 2} fazem referência ao software.

\begin{itemize}
\item Building a Whole-program Type Analysis in Eclipse (?)
\item Generating Run-time Progress Reports for a Points-to Analysis in Eclipse (?)
\end{itemize}

\section{TEBA}

Transformação source-to-source
publicado no artigo {\it A Pattern Search Method for Unpreprocessed C Programs Based on Tokenized Syntax Trees}
por A. Yoshida
no SCAM 2014,
disponibilizado em \url{http://tebasaki.jp/src}
gratuitamente
sem uma licença definida.

Software com lançamentos ocsaionais,
21 versões lançadas
entre 2010 e 2016,
escrito em Perl,
uma busca por citações no {\bf IEEE Xplore} por
\texttt{(((TEBA) AND tool) AND Programs)}
e no {\bf ACM} com
\texttt{content.ftsec:(+TEBA +tool)}
retornou
11 resultados,
nenhum faz referência ao software.


\section{TestEra}

Geração automática de testes
publicado no artigo {\it TestEra: A Novel Framework for Automated Testing of Java Programs}
por Marinov, Darko
no ASE 2001,
disponibilizado em \url{http://www.mit.edu/~sarfraz/testera}
mas inacessível em 09/08/2017.

Software sem informações sobre lançamentos ou releases,
uma busca por citações no {\bf IEEE Xplore} por
\texttt{(((((TestEra) AND framework) AND Java) AND testing) AND sarfraz)}
e no {\bf ACM} com
\texttt{content.ftsec:(+TestEra +framework +Java +testing +sarfraz)}
retornou
44 resultados,
{\bf 22} fazem referência ao software.

\begin{itemize}
\item A Case for White-box Testing Using Declarative Specifications Poster Abstract (?)
\item A Large-Scale Evaluation of Automated Unit Test Generation Using EvoSuite (?)
\item ACM SIGSOFT Impact Paper Award 2012: Systematic Software Testing: The Korat Approach (?)
\item Agile Specifications (?)
\item An Automated Approach for Writing Alloy Specifications Using Instances (?)
\item Constraints in Software Testing, Verification and Analysis CSTVA'2013 (?)
\item Efficiently Generating Structurally Complex Inputs with Thousands of Objects (?)
\item Efficiently Running Test Suites Using Abstract Undo Operations (?)
\item Generalizing Symbolic Execution to Library Classes (?)
\item Korat: Automated Testing Based on Java Predicates (?)
\item Optimizing Incremental Scope-Bounded Checking with Data-Flow Analysis (cita o software como trabalho relacionado)
\item PKorat: Parallel Generation of Structurally Complex Test Inputs (cita o software como trabalho relacionado)
\item Parallel Test Generation and Execution with Korat (?)
\item Query-Aware Test Generation Using a Relational Constraint Solver (cita o software como trabalho relacionado)
\item Software assurance by bounded exhaustive testing (?)
\item Systematic Testing of Database Engines Using a Relational Constraint Solver (cita o software como trabalho relacionado)
\item Test Input Generation Using Dynamic Programming (cita o software como trabalho relacionado)
\item Test Input Generation with Java PathFinder (cita o software como trabalho relacionado)
\item Test generation through programming in UDITA (?)
\item TestEra: A Tool for Testing Java Programs Using Alloy Specifications (?)
\item Testing Software Product Lines Using Incremental Test Generation (?)
\item Whispec: White-box Testing of Libraries Using Declarative Specifications (?)
\end{itemize}

\section{Vdiff}

Visualização de diferença de código-fonte
publicado no artigo {\it A Program Differencing Algorithm for Verilog HDL}
por Duley, Adam
no ASE 2010,
disponibilizado em \url{http://web.cs.ucla.edu/~miryung/software/vdiff/web/index.html}
mas inacessível em 09/08/2017.

Software sem informações sobre lançamentos ou releases,
uma busca por citações no {\bf IEEE Xplore} por
\texttt{(((Vdiff) AND verilog) AND HDL)}
e no {\bf ACM} com
\texttt{content.ftsec:(+Vdiff +verilog +HDL)}
retornou
10 resultados,
{\bf 4} fazem referência ao software.

\begin{itemize}
\item Improving pattern tracking with a language-aware tree differencing algorithm (cita o software como trabalho relacionado)
\item Move-optimized Source Code Tree Differencing (cita o software como trabalho relacionado)
\item Specifying and detecting meaningful changes in programs (cita o software como trabalho relacionado)
\item iDiff: Interaction-based Program Differencing Tool (cita o software como trabalho relacionado)
\end{itemize}

\section{WALA}

Análise estática de bytecode Java
publicado no artigo {\it Effective Static Analysis to Find Concurrency Bugs in Java}
por Z. D. Luo
no SCAM 2010,
disponibilizado em \url{http://wala.sourceforge.net/wiki/index.php/Main_Page}
como software livre
sob uma licença Eclipse Public License v1.0.

Software com lançamentos ocsaionais,
37 versões lançadas
entre 2006 e 2017,
escrito em Java,
uma busca por citações no {\bf IEEE Xplore} por
\texttt{(((WALA) AND "Static Analysis") AND "Concurrency Bugs")}
e no {\bf ACM} com
\texttt{content.ftsec:(+WALA +"Static Analysis" +"Concurrency Bugs")}
retornou
12 resultados,
{\bf 10} fazem referência ao software.

\begin{itemize}
\item ARROW: Automated Repair of Races on Client-side Web Pages (?)
\item Detecting Sensitive Data Disclosure via Bi-directional Text Correlation Analysis (?)
\item Dynamic detection of atomic-set-serializability violations (?)
\item Finding Errors in Multithreaded GUI Applications (?)
\item Finding Incorrect Compositions of Atomicity (?)
\item Keshmesh: A Tool for Detecting and Fixing Java Concurrency Bug Patterns (?)
\item MulticoreSDK: A Practical and Efficient Data Race Detector for Real-world Applications (?)
\item Programs from Proofs: A Framework for the Safe Execution of Untrusted Software (cita o software como trabalho relacionado)
\item Scala-AM: A Modular Static Analysis Framework (cita o software como trabalho relacionado)
\item To what extent could we detect field defects? an empirical study of false negatives in static bug finding tools (cita o software nas referencias do trabalho)
\end{itemize}

\section{Wrangler}

Refatoração de código Erlang
publicado no artigo {\it Refactoring Support for Modularity Maintenance in Erlang}
por H. Li
no SCAM 2010,
disponibilizado em \url{http://www.cs.kent.ac.uk/projects/wrangler/Home.html}
como software livre
sob uma licença BSD License "revised".

Software com lançamentos ocsaionais,
34 versões lançadas
até 2015,
escrito em Erlang,
uma busca por citações no {\bf IEEE Xplore} por
\texttt{((Wrangler) AND Erlang)}
e no {\bf ACM} com
\texttt{content.ftsec:(+Wrangler +Erlang)}
retornou
37 resultados,
{\bf 32} fazem referência ao software.

\begin{itemize}
\item A Language-independent Parallel Refactoring Framework (cita o software como trabalho relacionado)
\item A Review-based Comparative Study of Bad Smell Detection Tools (?)
\item Analysis of Preprocessor Constructs in Erlang (cita o software como trabalho relacionado)
\item Automated API Migration in a User-extensible Refactoring Tool for Erlang Programs (?)
\item Automated Behavioral Testing of Refactoring Engines (cita o software como trabalho relacionado)
\item Automated Module Interface Upgrade (?)
\item Automatic Refactoring of Erlang Programs (?)
\item Automating Property-based Testing of Evolving Web Services (?)
\item Cleaning Up Erlang Code is a Dirty Job but Somebody's Gotta Do It (?)
\item Clone Detection and Removal for Erlang/OTP Within a Refactoring Environment (?)
\item Comparative Study of Refactoring Haskell and Erlang Programs (?)
\item Detecting overly strong preconditions in refactoring engines (cita o software como trabalho relacionado)
\item Discovering Parallel Pattern Candidates in Erlang (?)
\item Erlang (?)
\item Erlang Testing and Tools Survey (?)
\item Extracting Properties from Test Cases by Refactoring (?)
\item From Test Cases to FSMs: Augmented Test-driven Development and Property Inference (cita o software como trabalho relacionado)
\item Gradual Typing of Erlang Programs: A Wrangler Experience (?)
\item Inferring Extended Finite State Machine models from software executions (?)
\item Let's Make Refactoring Tools User-extensible! (?)
\item Mechanical Verification of Refactorings (cita o software nas notas de agradecimentos finais)
\item Multicore Profiling for Erlang Programs Using Percept2 (?)
\item Quickchecking Refactoring Tools (?)
\item Reduction of development time by using scriptable IEC 61499 function blocks in a dynamically loadable type library (?)
\item Refactoring with Wrangler, Updated: Data and Process Refactorings, and Integration with Eclipse (?)
\item Scaling Testing of Refactoring Engines (cita o software como trabalho relacionado)
\item Scripting a Refactoring with Rascal and Eclipse (?)
\item Scripting parametric refactorings in Java to retrofit design patterns (cita o software como trabalho relacionado)
\item Smother: An MC/DC Analysis Tool for Erlang (?)
\item Tearing Down the Multicore Barrier for Web Applications (cita o software como trabalho relacionado)
\item Towards Semi-automatic Data-type Translation for Parallelism in Erlang (?)
\item ValiErlang: A Structural Testing Tool for Erlang Programs (cita o software como trabalho relacionado - artigo em ptbr)
\end{itemize}

\section{XOgastan}

Transformação source-to-source
publicado no artigo {\it XOgastan: XML-oriented gcc AST analysis and transformations}
por G. Antoniol
no SCAM 2003,
disponibilizado em \url{http://web.ing.unisannio.it/villano/students/masone}
mas inacessível em 09/08/2017.

Software sem informações sobre lançamentos ou releases,
uma busca por citações no {\bf IEEE Xplore} por
\texttt{(XOgastan)}
e no {\bf ACM} com
\texttt{content.ftsec:(+XOgastan)}
retornou
7 resultados,
{\bf 4} fazem referência ao software.

\begin{itemize}
\item A Pluggable Tool for Measuring Software Metrics from Source Code (cita o software como trabalho relacionado)
\item Comparative Analysis of Tools for Automated Software Re-engineering Purposes (?)
\item Component-based tool development (?)
\item TUAnalyzer - analyzing templates in C++ code (cita o software como trabalho relacionado)
\end{itemize}


\section{Categorias}



\begin{itemize}
\item ?: 268
\item ? (I2SD): 1
\item ? - cita uma ferramenta com o mesmo nome mas aparentemente trata-se de outra ferramenta - Session-based Web Application Tester - SWAT: 4
\item ? - mesmos autores: 2
\item cita ECBMC mas faz referencia a um artigo com o nome do software: 1
\item cita em trabalhos futuros o artigo selecionado na revisão estruturada: 1
\item cita na fundamentação como exemplo de implementação sobre a geração de sequencias randomicas de chamada a métodos: 1
\item cita o nome do software uma vez mas o artigo está escrito em russo: 1
\item cita o software (I2SD) em trabalhos futuros: 1
\item cita o software como base para implementação da ferramenta Jifclipse, implementa um superconjunto da linguagem jflow: 1
\item cita o software como trabalho relacionado: 105
\item cita o software como trabalho relacionado - artigo em ptbr: 1
\item cita o software em trabalhos futuros: 2
\item cita o software mas o artigo está escrito em alemão: 1
\item cita o software na introdução como exemplo de sistema de gramática de atributo: 1
\item cita o software na seção de ameaças a validade como um exemplo de implementação de algoritmo de diferenciação baseada em árvores: 1
\item cita o software na seção de limitações como exemplo de implementação de algoritmo de diferenciação em árvore a ser avaliado: 1
\item cita o software nas notas de agradecimentos finais: 1
\item cita o software nas referencias do trabalho: 1
\item cita o software nos trabalhos relacionados: 2
\item cita um tal MSP-GCC, n tenho certeza de ser o MSP: 1
\item pelo título parece citar, mas o artigo não foi encontrado para download: 1
\end{itemize}

