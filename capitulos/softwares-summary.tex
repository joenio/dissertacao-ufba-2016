
% o conteúdo deste arquivo é gerado automaticamente pelo script
% bin/softwares-summary, favor não editar manualmente

\xchapter{Softwares acadêmicos}{Este capítulo ...}
\label{softwares-summary}

\section{2LS - 2nd order Logic Solving}

Análise de terminação para programas C usando resumo interprocedural baseado em modelos
publicado no artigo {\it Synthesising Interprocedural Bit-Precise Termination Proofs (T)}
por H. Y. Chen
no ASE 2015,
disponibilizado em \url{http://svn.cprover.org/wiki/doku.php?id=2ls for program analysis}
como software livre
sob uma licença BSD.

Software com lançamentos ocsaionais,
escrito em C++,
uma busca por citações no {\bf IEEE Xplore} por
\texttt{(('2nd order Logic Solving') AND 2LS)}
e no {\bf ACM} com
\texttt{content.ftsec:(2LS) AND (order) AND (Logic) AND (Solving)}
retornou
37 resultados, apenas
{\bf 0} faz referência ao software.



\section{AccessAnalysis}

Cálculo de métricas IGAT e IGAM
publicado no artigo {\it AccessAnalysis: A Tool for Measuring the Appropriateness of Access Modifiers in Java Systems}
por C. Zoller
no SCAM 2012,
disponibilizado em \url{http://accessanalysis.sourceforge.net}
como software livre
sob uma licença EPL.

Software considerado obsoleto,
escrito em Java,
uma busca por citações no {\bf IEEE Xplore} por
\texttt{(AccessAnalysis)}
e no {\bf ACM} com
\texttt{content.ftsec:(+AccessAnalysis +Tool +Java +Modifiers)}
retornou
8 resultados, apenas
{\bf 1} faz referência ao software.

\begin{enumerate}
\item Measuring Inappropriate Generosity with Access Modifiers in Java Systems
\end{enumerate}


\section{APIExample}

Extração de informações de API Java e documentação automática com exemplos
publicado no artigo {\it APIExample: An effective web search based usage example recommendation system for java APIs}
por Lijie Wang
no ASE 2011,
disponibilizado em \url{http://www.apiexample.com}
mas inacessível em 09/08/2017.

Software sem informações sobre lançamentos ou releases,
uma busca por citações no {\bf IEEE Xplore} por
\texttt{((APIExample) AND Java)}
e no {\bf ACM} com
\texttt{content.ftsec:(+APIExample +Java)}
retornou
16 resultados, apenas
{\bf 3} faz referência ao software.

\begin{enumerate}
\item Documenting APIs with examples: Lessons learned with the APIMiner platform
\item Generating API-usage Example for Project Developers
\item APIBook: An Effective Approach for Finding APIs
\end{enumerate}


\section{BEG - Bandera environment generator}

Criação automática de ambientes para verificação de modelos Java
publicado no artigo {\it Automated environment generation for software model checking}
por O. Tkachuk
no ASE 2003,
disponibilizado em \url{http://bandera.projects.cs.ksu.edu/}
mas inacessível em 09/08/2017.

Software sem informações sobre lançamentos ou releases,
uma busca por citações no {\bf IEEE Xplore} por
\texttt{(("Bandera environment generator") AND BEG)}
e no {\bf ACM} com
\texttt{content.ftsec:(+"Bandera environment generator" +BEG)}
retornou
10 resultados, apenas
{\bf 8} faz referência ao software.

\begin{enumerate}
\item Analyzing interaction orderings with model checking
\item Partial Verification of Software Components: Heuristics for Environment Construction
\item Concurrent Testing of Java Components Using Java PathFinder
\item Environment generation for validating event-driven software using model checking
\item Assume-guarantee verification of software components in SOFA 2 framework
\item Combining Environment Generation and Slicing for Modular Software Model Checking
\item Environment Modeling in Model-based Testing: Concepts, Prospects and Research Challenges: A Systematic Literature Review
\item Application of Automated Environment Generation to Commercial Software
\end{enumerate}


\section{Kiasan/Bogor}

Verificação de modelos
publicado no artigo {\it Bogor/Kiasan: A k-bounded Symbolic Execution for Checking Strong Heap Properties of Open Systems}
por X. Deng
no ASE 2006,
disponibilizado em \url{http://bogor.projects.cs.ksu.edu/manual/}
como software livre
sob uma licença SAnToS Laboratory Open Academic License.

Software considerado obsoleto,
escrito em Java,
uma busca por citações no {\bf IEEE Xplore} por
\texttt{((Kiasan/Bogor) OR Bogor/Kiasan)}
e no {\bf ACM} com
\texttt{content.ftsec:("Bogor/Kiasan")}
retornou
37 resultados, apenas
{\bf 19} faz referência ao software.

\begin{enumerate}
\item Bogor: A Flexible Framework for Creating Software Model Checkers
\item Domain-specific Model Checking Using The Bogor Framework
\item Formal Software Analysis Emerging Trends in Software Model Checking
\item Towards A Case-Optimal Symbolic Execution Algorithm for Analyzing Strong Properties of Object-Oriented Programs
\item Kiasan/KUnit: Automatic Test Case Generation and Analysis Feedback for Open Object-oriented Systems
\item Environment generation for validating event-driven software using model checking
\item Symbolic execution for software testing in practice: preliminary assessment
\item Generating unit tests using static analysis and contracts
\item Dynamic Data Flow Testing of Object Oriented Systems
\item Efficient Modular Glass Box Software Model Checking
\item Towards a Lazier Symbolic Pathfinder
\item Symbolic Execution for Software Testing in Practice: Preliminary Assessment
\item Dynamic Data Flow Testing of Object Oriented Systems
\item Symbolic PathFinder: Symbolic Execution of Java Bytecode
\item Combining Unit-level Symbolic Execution and System-level Concrete Execution for Testing Nasa Software
\item Preliminary Design of a Unified JML Representation and Software Infrastructure
\item DSD-Crasher: A Hybrid Analysis Tool for Bug Finding
\item jStar: Towards Practical Verification for Java
\item A Publish-subscribe Architecture and Component-based Programming Model for Medical Device Interoperability
\end{enumerate}


\section{ccJava}

Linguagem orientada a aspectos
publicado no artigo {\it An Aspect-oriented Weaving Mechanism Based on Component and Connector Architecture}
por Ubayashi, Naoyasu
no ASE 2007,
disponibilizado em \url{http://posl.minnie.ai.kyutech.ac.jp/}
mas inacessível em 09/08/2017.

Software sem informações sobre lançamentos ou releases,
uma busca por citações no {\bf IEEE Xplore} por
\texttt{(ccJava)}
e no {\bf ACM} com
\texttt{content.ftsec:(+ccJava)}
retornou
7 resultados, apenas
{\bf 5} faz referência ao software.

\begin{enumerate}
\item Alloy-Based Lightweight Verification for Aspect-Oriented Architecture
\item Archface: a contract place where architectural design and code meet together
\item Pointcut-based Architectural Interface for Bridging a Gap Between Design and Implementation
\item Archface: A Contract Place Where Architectural Design and Code Meet Together
\item An Interface Mechanism for Encapsulating Weaving in Class-based AOP
\end{enumerate}


\section{CIVL - Concurrency intermediate verification language}

Framework para verificação de programas concorrentes
publicado no artigo {\it CIVL: Formal Verification of Parallel Programs}
por M. Zheng
no ASE 2015,
disponibilizado em \url{http://vsl.cis.udel.edu/civl/}
como software livre
sob uma licença GPL.

Software com lançamentos frequentes,
escrito em C,
uma busca por citações no {\bf IEEE Xplore} por
\texttt{(('concurrency intermediate verification') AND CIVL)}
e no {\bf ACM} com
\texttt{content.ftsec:(+civl +concurrency +intermediate +verification +language)}
retornou
8 resultados, apenas
{\bf 6} faz referência ao software.

\begin{enumerate}
\item CIVL: the concurrency intermediate verification language
\item Achieving Formal Parallel Program Debugging by Incentivizing CS/HPC Collaborative Tool Development
\item CIVL: The Concurrency Intermediate Verification Language
\item Protocol-based Verification of Message-passing Parallel Programs
\item CIVL Solutions to Verifythis 2016 Challenges
\item The RERS 2017 Challenge and Workshop (Invited Paper)
\end{enumerate}


\section{CodeBoost}

Transformação source-to-source para otimização de programas C++
publicado no artigo {\it Design of the CodeBoost transformation system for domain-specific optimisation of C++ programs}
por O. S. Bagge
no SCAM 2003,
disponibilizado em \url{http://codeboost.org/}
como software livre
sob uma licença GPL v2.

Software com lançamentos ocsaionais,
escrito em C,
uma busca por citações no {\bf IEEE Xplore} por
\texttt{((CodeBoost) AND C++)}
e no {\bf ACM} com
\texttt{content.ftsec:(+CodeBoost +"C++" +Tool)}
retornou
25 resultados, apenas
{\bf 13} faz referência ao software.

\begin{enumerate}
\item Transformations for abstractions
\item Specifying transformation sequences as computation on program fragments with an abstract attribute grammar
\item Annotating user-defined abstractions for optimization
\item A Heterogeneous Parallel Framework for Domain-Specific Languages
\item Generic Flow-sensitive Optimizing Transformations in C++ with Concepts
\item When and How to Develop Domain-specific Languages
\item Green-Marl: A DSL for Easy and Efficient Graph Analysis
\item Integrating Semantics and Compilation: Using C++ Concepts to Develop Robust and Efficient Reusable Libraries
\item Almost First-class Language Embedding: Taming Staged Embedded DSLs
\item A Domain-specific Approach to Heterogeneous Parallelism
\item Stratego/XT 0.16: Components for Transformation Systems
\item Strategic Programming Meets Adaptive Programming
\item A Nanopass Framework for Commercial Compiler Development
\end{enumerate}


\section{Composite Symbolic Library}

Verificação de modelos
publicado no artigo {\it Action Language Verifier}
por T. Bultan
no ASE 2001,
disponibilizado em \url{http://www.cs.ucsb.edu/~bultan/composite/}
gratuitamente
sem uma licença definida.

Software sem informações sobre lançamentos ou releases,
escrito em C,
uma busca por citações no {\bf IEEE Xplore} por
\texttt{((((composite) AND bultan) AND Action Language Verifier) OR ("Composite Symbolic Library"))}
e no {\bf ACM} com
\texttt{content.ftsec:(+composite +"Action Language Verifier")}
retornou
14 resultados, apenas
{\bf 7} faz referência ao software.

\begin{enumerate}
\item Realizability of conversation protocols with message contents
\item Automated model checking and testing for composite Web services
\item Analyzing tabular requirements specifications using infinite state model checking
\item Analyzing Tabular Requirements Specifications Using Infinite State Model Checking
\item Mixed Symbolic Representations for Model Checking Software Programs
\item Efficient Temporal-logic Query Checking for Presburger Systems
\item Model Checking Sequential Software Programs via Mixed Symbolic Analysis
\end{enumerate}


\section{CPA+ - Configurable program analysis with dynamic precision adjustment}

Análise configurável de programa com ajuste dinâmico de precisão
publicado no artigo {\it Program Analysis with Dynamic Precision Adjustment}
por D. Beyer
no ASE 2008,
disponibilizado em \url{http://www.cs.sfu.ca/~dbeyer/blast_cpaplus/}
mas inacessível em 09/08/2017.

Software sem informações sobre lançamentos ou releases,
uma busca por citações no {\bf IEEE Xplore} por
\texttt{(('program analysis') AND cpa+)}
e no {\bf ACM} com
\texttt{content.ftsec:(+cpa +dbeyer)}
retornou
8 resultados, apenas
{\bf 4} faz referência ao software.

\begin{enumerate}
\item Conditional Model Checking: A Technique to Pass Information Between Verifiers
\item Precision Reuse for Efficient Regression Verification
\item Predicate Abstraction with Adjustable-block Encoding
\item Witness Validation and Stepwise Testification Across Software Verifiers
\end{enumerate}


\section{CSeq}

Transformação source-to-source para programas C concorrentes
publicado no artigo {\it CSeq: A concurrency pre-processor for sequential C verification tools}
por B. Fischer
no ASE 2013,
disponibilizado em \url{http://users.ecs.soton.ac.uk/gp4/cseq/files/cseq-0.5.zip}
como software livre
sob uma licença BSD.

Software sem informações sobre lançamentos ou releases,
escrito em C,
uma busca por citações no {\bf IEEE Xplore} por
\texttt{((CSeq) AND soton)}
e no {\bf ACM} com
\texttt{content.ftsec:(+cseq +sequential +verification +tool)}
retornou
16 resultados, apenas
{\bf 0} faz referência ao software.



\section{DDVerify}

Verificação de Linux drivers através de checagem de modelos
publicado no artigo {\it Model Checking Concurrent Linux Device Drivers}
por Witkowski, Thomas
no ASE 2007,
disponibilizado em \url{http://www.verify.ethz.ch/ddverify}
mas inacessível em 09/08/2017.

Software sem informações sobre lançamentos ou releases,
uma busca por citações no {\bf IEEE Xplore} por
\texttt{(DDVerify)}
e no {\bf ACM} com
\texttt{content.ftsec:(+DDVerify)}
retornou
4 resultados, apenas
{\bf 0} faz referência ao software.



\section{Derailer}

Localização de falhas de segurança em aplicações web
publicado no artigo {\it Derailer: Interactive Security Analysis for Web Applications}
por Near, Joseph P.
no ASE 2014,
disponibilizado em \url{http://people.csail.mit.edu/jnear/derailer}
como software livre
sob uma licença GPL v3.

Software considerado obsoleto,
escrito em Ruby,
uma busca por citações no {\bf IEEE Xplore} por
\texttt{((Derailer) AND jnear)}
e no {\bf ACM} com
\texttt{content.ftsec:(+Derailer +analysis +security +web +tool +bugs +ruby)}
retornou
7 resultados, apenas
{\bf 0} faz referência ao software.



\section{Diagnosys}

Construção de interfaces de debug para o kernel Linux
publicado no artigo {\it Diagnosys: automatic generation of a debugging interface to the Linux kernel}
por T. F. Bissyandé
no ASE 2012,
disponibilizado em \url{http://momentum.labri.fr/projects/diagnosys}
mas inacessível em 09/08/2017.

Software sem informações sobre lançamentos ou releases,
uma busca por citações no {\bf IEEE Xplore} por
\texttt{((Diagnosys) AND debugging)}
e no {\bf ACM} com
\texttt{content.ftsec:(+"Diagnosys tool" +Debugging +Linux +"device drivers")}
retornou
9 resultados, apenas
{\bf 0} faz referência ao software.



\section{DOMPLETION}

Sugestão de código javascript
publicado no artigo {\it Dompletion: DOM-aware JavaScript Code Completion}
por Bajaj, Kartik
no ASE 2014,
disponibilizado em \url{https://github.com/saltlab/dompletion}
gratuitamente
sem uma licença definida.

Software sem informações sobre lançamentos ou releases,
escrito em Javascript,
uma busca por citações no {\bf IEEE Xplore} por
\texttt{content.ftsec:(+dompletion +JavaScript)}
e no {\bf ACM} com
\texttt{((dompletion) AND JavaScript)}
retornou
2 resultados, apenas
{\bf 0} faz referência ao software.



\section{DRC - Dangling Reference Checker}

Análise estática para detecção de referências inválidas em código dinâmico PHP
publicado no artigo {\it Dangling references in multi-configuration and dynamic PHP-based Web applications}
por H. V. Nguyen
no ASE 2013,
disponibilizado em \url{http://home.engineering.iastate.edu/~hungnv/Research/DRC}
mas inacessível em 09/08/2017.

Software sem informações sobre lançamentos ou releases,
uma busca por citações no {\bf IEEE Xplore} por
\texttt{(DRC Dangling Reference Checker)}
e no {\bf ACM} com
\texttt{content.ftsec:(+DRC +Dangling +Reference)}
retornou
19 resultados, apenas
{\bf 0} faz referência ao software.



\section{e-munity}

Verificação de segurança
publicado no artigo {\it Scalable Security Verification of Software at Compile Time}
por S. Tlili
no SCAM 2014,
disponibilizado em \url{http://sourceforge.net/p/emunity/code/ci/master/tree/}
gratuitamente
sem uma licença definida.

Software sem informações sobre lançamentos ou releases,
escrito em C,
uma busca por citações no {\bf IEEE Xplore} por
\texttt{(e-munity)}
e no {\bf ACM} com
\texttt{content.ftsec:(+"e-munity")}
retornou
1 resultados, apenas
{\bf 0} faz referência ao software.



\section{EJB}

Criação de diagramas de sequência
publicado no artigo {\it I2SD: Reverse Engineering Sequence Diagrams from Enterprise Java Beans with Interceptors}
por S. Roubtsov
no SCAM 2011,
disponibilizado em \url{https://www.dropbox.com/s/glhg8any43lccgm/EJB.zip}
gratuitamente
sem uma licença definida.

Software considerado obsoleto,
escrito em Java,
uma busca por citações no {\bf IEEE Xplore} por
\texttt{(((I2SD) AND EJB) AND Java)}
e no {\bf ACM} com
\texttt{content.ftsec:(+I2SD +Java)}
retornou
5 resultados, apenas
{\bf 0} faz referência ao software.



\section{error-prone}

Localização de bugs em código Java construído em cima do compilador javac
publicado no artigo {\it Building Useful Program Analysis Tools Using an Extensible Java Compiler}
por E. Aftandilian
no SCAM 2012,
disponibilizado em \url{http://code.google.com/p/error-prone}
como software livre
sob uma licença Apache License v2.0.

Software com lançamentos frequentes,
escrito em Java,
uma busca por citações no {\bf IEEE Xplore} por
\texttt{((((('error-prone tool') AND Analysis) AND 'java compiler') AND 'error checks') AND javac)}
e no {\bf ACM} com
\texttt{content.ftsec:(+"error-prone" +tool +javac +analysis +"java compiler")}
retornou
47 resultados, apenas
{\bf 0} faz referência ao software.



\section{ESBMC}

Verificação de modelos
publicado no artigo {\it SMT-Based Bounded Model Checking for Embedded ANSI-C Software}
por L. Cordeiro
no ASE 2009,
disponibilizado em \url{http://users.ecs.soton.ac.uk/lcc08r/esbmc/}
mas inacessível em 09/08/2017.

Software sem informações sobre lançamentos ou releases,
uma busca por citações no {\bf IEEE Xplore} por
\texttt{(ESBMC)}
e no {\bf ACM} com
\texttt{content.ftsec:(+ESBMC)}
retornou
50 resultados, apenas
{\bf 0} faz referência ao software.



\section{ETXL}

Transformação de código
publicado no artigo {\it Evolving TXL}
por A. D. Thurston
no SCAM 2006,
disponibilizado em \url{http://www.cs.queensu.ca/home/thurston/etxl}
mas inacessível em 09/08/2017.

Software sem informações sobre lançamentos ou releases,
uma busca por citações no {\bf IEEE Xplore} por
\texttt{(((ETXL) AND source) AND transformation)}
e no {\bf ACM} com
\texttt{content.ftsec:(+ETXL)}
retornou
10 resultados, apenas
{\bf 0} faz referência ao software.



\section{FaultBuster}

Refatoração de code smells
publicado no artigo {\it FaultBuster: An automatic code smell refactoring toolset}
por G. Szőke
no SCAM 2015,
disponibilizado em \url{http://www.sed.inf.u-szeged.hu/FaultBuster}
gratuitamente
sob uma licença de demonstração.

Software sem informações sobre lançamentos ou releases,
uma busca por citações no {\bf IEEE Xplore} por
\texttt{(FaultBuster)}
e no {\bf ACM} com
\texttt{content.ftsec:(+FaultBuster)}
retornou
4 resultados, apenas
{\bf 0} faz referência ao software.



\section{Flowgen}

Criação automática de grafos UML
publicado no artigo {\it Flowgen: Flowchart-Based Documentation Framework for C++}
por D. A. Kosower
no SCAM 2014,
disponibilizado em \url{https://github.com/jlopezvi/Flowgen}
como software livre
sob uma licença GPL v3.

Software sem informações sobre lançamentos ou releases,
escrito em Python,
uma busca por citações no {\bf IEEE Xplore} por
\texttt{((Flowgen) AND C++)}
e no {\bf ACM} com
\texttt{content.ftsec:(+Flowgen)}
retornou
8 resultados, apenas
{\bf 0} faz referência ao software.



\section{GRT}

Geração automática de testes
publicado no artigo {\it GRT: Program-Analysis-Guided Random Testing (T)}
por L. Ma
no ASE 2015,
disponibilizado em \url{http://www.sites.google.com/site/grtprojectut/download}
mas inacessível em 09/08/2017.

Software sem informações sobre lançamentos ou releases,
uma busca por citações no {\bf IEEE Xplore} por
\texttt{((GRT) AND "Guided Random Testing")}
e no {\bf ACM} com
\texttt{content.ftsec:(+GRT) +"Guided Random Testing")}
retornou
13 resultados, apenas
{\bf 0} faz referência ao software.



\section{GUIZMO}

Inferência de layout
publicado no artigo {\it Model-driven Reverse Engineering of Legacy Graphical User Interfaces}
por S\'{a}nchez Ram\'{o}n, \'{O}scar
no ASE 2010,
disponibilizado em \url{http://modelum.es/trac/guizmo/}
como software livre
sob uma licença Apache License v2.0.

Software sem informações sobre lançamentos ou releases,
escrito em Java,
uma busca por citações no {\bf IEEE Xplore} por
\texttt{(guizmo)}
e no {\bf ACM} com
\texttt{content.ftsec:(+guizmo)}
retornou
0 resultados, apenas
{\bf 0} faz referência ao software.



\section{GumTree}

Comparação de mudanças
publicado no artigo {\it Fine-grained and Accurate Source Code Differencing}
por Falleri, Jean-R{\'e}my
no ASE 2014,
disponibilizado em \url{https://github.com/jrfaller/gumtree}
como software livre
sob uma licença LGPL v3.

Software com lançamentos ocsaionais,
escrito em Java,
uma busca por citações no {\bf IEEE Xplore} por
\texttt{((GumTree) AND tool)}
e no {\bf ACM} com
\texttt{content.ftsec:(+GumTree +tool)}
retornou
37 resultados, apenas
{\bf 0} faz referência ao software.



\section{HUSACCT}

?
publicado no artigo {\it HUSACCT: Architecture Compliance Checking with Rich Sets of Module and Rule Types}
por Pruijt, Leo J.
no ASE 2014,
disponibilizado em \url{http://husacct.github.io/HUSACCT/}
como software livre
sob uma licença AGPL.

Software com lançamentos ?,
escrito em Java,
uma busca por citações no {\bf IEEE Xplore} por
\texttt{(HUSACCT)}
e no {\bf ACM} com
\texttt{content.ftsec:(+HUSACCT)}
retornou
7 resultados, apenas
{\bf 0} faz referência ao software.



\section{Indus}

Biblioteca de program slicing
publicado no artigo {\it An Overview of the Indus Framework for Analysis and Slicing of Concurrent Java Software (Keynote Talk - Extended Abstract)}
por V. P. Ranganath
no SCAM 2006,
disponibilizado em \url{http://indus.projects.cis.ksu.edu}
como software livre
sob uma licença EPL v1.0.

Software com lançamentos ?,
escrito em Java,
uma busca por citações no {\bf IEEE Xplore} por
\texttt{("Indus Framework")}
e no {\bf ACM} com
\texttt{content.ftsec:(+"Indus Framework")}
retornou
6 resultados, apenas
{\bf 0} faz referência ao software.



\section{JastAdd}

Análise de código-fonte através da descrição de atributos via gramática de atributos (AG)
publicado no artigo {\it Extending Attribute Grammars with Collection Attributes--Evaluation and Applications}
por Magnusson, Eva
no SCAM 2007,
disponibilizado em \url{http://jastadd.cs.lth.se/web}
como software livre
sob uma licença BSD License "modified".

Software com lançamentos frequentes,
escrito em Java,
uma busca por citações no {\bf IEEE Xplore} por
\texttt{(JastAdd tool Attribute Grammars)}
e no {\bf ACM} com
\texttt{content.ftsec:(+JastAdd +tool +"Attribute Grammars" +"source code" +analysis)}
retornou
50 resultados, apenas
{\bf 0} faz referência ao software.



\section{JFlow}

Transformação source-to-source
publicado no artigo {\it JFlow: Practical refactorings for flow-based parallelism}
por N. Chen
no ASE 2013,
disponibilizado em \url{http://vazexqi.github.io/JFlow/}
como software livre
sob uma licença Illinois/NCSA Open Source License.

Software considerado obsoleto,
escrito em Java,
uma busca por citações no {\bf IEEE Xplore} por
\texttt{(JFlow tool Eclipse)}
e no {\bf ACM} com
\texttt{content.ftsec:(+JFlow +tool +Eclipse)}
retornou
16 resultados, apenas
{\bf 0} faz referência ao software.



\section{JstereoCode}

Detecção de esteriótipos Java
publicado no artigo {\it JStereoCode: automatically identifying method and class stereotypes in Java code}
por L. Moreno
no ASE 2012,
disponibilizado em \url{http://www.cs.wayne.edu/~severe/revenge/}
mas inacessível em 09/08/2017.

Software sem informações sobre lançamentos ou releases,
uma busca por citações no {\bf IEEE Xplore} por
\texttt{(JstereoCode)}
e no {\bf ACM} com
\texttt{content.ftsec:(+JstereoCode)}
retornou
13 resultados, apenas
{\bf 0} faz referência ao software.



\section{Jtop}

Gestão de casos de teste
publicado no artigo {\it Jtop: Managing JUnit Test Cases in Absence of Coverage Information}
por L. Zhang
no ASE 2009,
disponibilizado em \url{http://code.google.com/p/pku-jtop/}
mas inacessível em 09/08/2017.

Software sem informações sobre lançamentos ou releases,
uma busca por citações no {\bf IEEE Xplore} por
\texttt{(Jtop JUnit)}
e no {\bf ACM} com
\texttt{content.ftsec:(+Jtop +JUnit)}
retornou
4 resultados, apenas
{\bf 0} faz referência ao software.



\section{Loopfrog}

Verificação de modelos
publicado no artigo {\it Loopfrog: A Static Analyzer for ANSI-C Programs}
por D. Kroening
no ASE 2009,
disponibilizado em \url{http://verify.inf.usi.ch/content/loopfrog}
gratuitamente
sem uma licença definida.

Software sem informações sobre lançamentos ou releases,
uma busca por citações no {\bf IEEE Xplore} por
\texttt{(Loopfrog)}
e no {\bf ACM} com
\texttt{content.ftsec:(+Loopfrog)}
retornou
6 resultados, apenas
{\bf 0} faz referência ao software.



\section{Lotrack}

Análise estática de configuração
publicado no artigo {\it Tracking Load-time Configuration Options}
por Lillack, Max
no ASE 2014,
disponibilizado em \url{https://github.com/MaxLillack/Lotrack}
gratuitamente
sem uma licença definida.

Software sem informações sobre lançamentos ou releases,
escrito em Java,
uma busca por citações no {\bf IEEE Xplore} por
\texttt{(Lotrack)}
e no {\bf ACM} com
\texttt{content.ftsec:(+Lotrack)}
retornou
4 resultados, apenas
{\bf 0} faz referência ao software.



\section{MPAnalyzer}

Análise de padrões disponível
publicado no artigo {\it MPAnalyzer: A Tool for Finding Unintended Inconsistencies in Program Source Code}
por Higo, Yoshiki
no ASE 2014,
disponibilizado em \url{https://github.com/YoshikiHigo/MPAnalyzer}
gratuitamente
sem uma licença definida.

Software sem informações sobre lançamentos ou releases,
escrito em Java,
uma busca por citações no {\bf IEEE Xplore} por
\texttt{(MPAnalyzer)}
e no {\bf ACM} com
\texttt{content.ftsec:(+MPAnalyzer)}
retornou
3 resultados, apenas
{\bf 0} faz referência ao software.



\section{MSP}

Construção de modelo formal de acesso a memória
publicado no artigo {\it Recovering Memory Access Patterns of Executable Programs}
por Ketterlin, Alain
no SCAM 2010,
disponibilizado em \url{http://icps.u-strasbg.fr/software/msp}
mas inacessível em 09/08/2017.

Software sem informações sobre lançamentos ou releases,
uma busca por citações no {\bf IEEE Xplore} por
\texttt{((((MSP) AND tool) AND Binary) AND "Program Analysis")}
e no {\bf ACM} com
\texttt{content.ftsec:(+MSP +tool +Binary +"Program Analysis")}
retornou
37 resultados, apenas
{\bf 0} faz referência ao software.



\section{mygcc}

Verificação de programas C
publicado no artigo {\it A Portable Compiler-Integrated Approach to Permanent Checking}
por N. Volanschi
no ASE 2006,
disponibilizado em \url{http://mygcc.free.fr/}
como software livre
sob uma licença GPL.

Software considerado obsoleto,
escrito em C,
uma busca por citações no {\bf IEEE Xplore} por
\texttt{(mygcc)}
e no {\bf ACM} com
\texttt{content.ftsec:(+mygcc)}
retornou
7 resultados, apenas
{\bf 0} faz referência ao software.



\section{PARSEWeb}

Query para apoio e sugestão de reuso de bibliotecas
publicado no artigo {\it Parseweb: A Programmer Assistant for Reusing Open Source Code on the Web}
por Thummalapenta, Suresh
no ASE 2007,
disponibilizado em \url{http://ase.csc.ncsu.edu/parseweb/}
mas inacessível em 09/08/2017.

Software sem informações sobre lançamentos ou releases,
uma busca por citações no {\bf IEEE Xplore} por
\texttt{(((PARSEWeb) AND tool) AND AST)}
e no {\bf ACM} com
\texttt{content.ftsec:(+PARSEWeb +tool +AST)}
retornou
29 resultados, apenas
{\bf 0} faz referência ao software.



\section{PAT}

Ambiente de teste automático
publicado no artigo {\it Puzzle-based automatic testing: bringing humans into the loop by solving puzzles}
por N. Chen
no ASE 2012,
disponibilizado em \url{http://pat.cse.ust.hk:8080/}
mas inacessível em 09/08/2017.

Software sem informações sobre lançamentos ou releases,
uma busca por citações no {\bf IEEE Xplore} por
\texttt{("Puzzle-based automatic testing")}
e no {\bf ACM} com
\texttt{content.ftsec:(+"Puzzle-based automatic testing")}
retornou
13 resultados, apenas
{\bf 0} faz referência ao software.



\section{PHP AiR}

Um framework para análise de código PHP escrito em Rascal
publicado no artigo {\it Static, Lightweight Includes Resolution for PHP}
por Hills, Mark
no ASE 2014,
disponibilizado em \url{https://github.com/cwi-swat/php-analysis}
gratuitamente
sem uma licença definida.

Software com lançamentos ?,
escrito em Rascal,
uma busca por citações no {\bf IEEE Xplore} por
\texttt{("PHP AiR")}
e no {\bf ACM} com
\texttt{content.ftsec:(+"PHP AiR")}
retornou
11 resultados, apenas
{\bf 0} faz referência ao software.



\section{protopurity}

Análise de impacto
publicado no artigo {\it Detecting function purity in JavaScript}
por J. Nicolay
no SCAM 2015,
disponibilizado em \url{https://github.com/jensnicolay/jipda/tree/scam2015/protopurity}
gratuitamente
sem uma licença definida.

Software com lançamentos ?,
escrito em Javascript,
uma busca por citações no {\bf IEEE Xplore} por
\texttt{content.ftsec:(+protopurity)}
e no {\bf ACM} com
\texttt{(protopurity)}
retornou
0 resultados, apenas
{\bf 0} faz referência ao software.



\section{Pseudogen}

Transformação de código-fonte em pseudo-código
publicado no artigo {\it Pseudogen: A Tool to Automatically Generate Pseudo-Code from Source Code}
por H. Fudaba
no ASE 2015,
disponibilizado em \url{http://ahclab.naist.jp/pseudogen/}
gratuitamente
sem uma licença definida.

Software com lançamentos ?,
escrito em Python,
uma busca por citações no {\bf IEEE Xplore} por
\texttt{(Pseudogen)}
e no {\bf ACM} com
\texttt{content.ftsec:(+Pseudogen +"Pseudo-Code")}
retornou
5 resultados, apenas
{\bf 0} faz referência ao software.



\section{PtYasm}

Verificação de modelos
publicado no artigo {\it Augmenting Counterexample-Guided Abstraction Refinement with Proof Templates}
por T. E. Hart
no ASE 2008,
disponibilizado em \url{http://www.cs.toronto.edu/~tomhart/ptyasm}
gratuitamente
sem uma licença definida.

Software sem informações sobre lançamentos ou releases,
escrito em Java,
uma busca por citações no {\bf IEEE Xplore} por
\texttt{(PtYasm)}
e no {\bf ACM} com
\texttt{content.ftsec:(+PtYasm)}
retornou
2 resultados, apenas
{\bf 0} faz referência ao software.



\section{PuMoC}

Verificação de modelos
publicado no artigo {\it PuMoC: a CTL model-checker for sequential programs}
por F. Song
no ASE 2012,
disponibilizado em \url{http://www.liafa.jussieu.fr/~song/PuMoC}
mas inacessível em 09/08/2017.

Software sem informações sobre lançamentos ou releases,
uma busca por citações no {\bf IEEE Xplore} por
\texttt{(PuMoC)}
e no {\bf ACM} com
\texttt{content.ftsec:(+PuMoC)}
retornou
4 resultados, apenas
{\bf 0} faz referência ao software.



\section{PYTHIA}

Criação automática de casos de teste
publicado no artigo {\it PYTHIA: Generating test cases with oracles for JavaScript applications}
por S. Mirshokraie
no ASE 2013,
disponibilizado em \url{http://salt.ece.ubc.ca/software/pythia/}
mas inacessível em 09/08/2017.

Software sem informações sobre lançamentos ou releases,
uma busca por citações no {\bf IEEE Xplore} por
\texttt{((PYTHIA) AND JavaScript)}
e no {\bf ACM} com
\texttt{content.ftsec:(+PYTHIA +JavaScript)}
retornou
15 resultados, apenas
{\bf 0} faz referência ao software.



\section{ReAssert}

Localização de falhas em testes e refatoração
publicado no artigo {\it ReAssert: Suggesting Repairs for Broken Unit Tests}
por B. Daniel
no ASE 2009,
disponibilizado em \url{http://mir.cs.illinois.edu/reassert}
como software livre
sob uma licença Illinois/NCSA Open Source License.

Software considerado obsoleto,
escrito em Java,
uma busca por citações no {\bf IEEE Xplore} por
\texttt{(ReAssert tool "Unit Tests")}
e no {\bf ACM} com
\texttt{content.ftsec:(+ReAssert +tool +"Unit Tests")}
retornou
28 resultados, apenas
{\bf 0} faz referência ao software.



\section{Rêve}

Verificação de regressão
publicado no artigo {\it Automating Regression Verification}
por Felsing, Dennis
no ASE 2014,
disponibilizado em \url{http://formal.iti.kit.edu/improve/}
mas inacessível em 09/08/2017.

Software sem informações sobre lançamentos ou releases,
uma busca por citações no {\bf IEEE Xplore} por
\texttt{(Reve Automating regression verification)}
e no {\bf ACM} com
\texttt{content.ftsec:(+"Reve" +Automating +regression +verification)}
retornou
13 resultados, apenas
{\bf 0} faz referência ao software.



\section{RRFinder}

Mineração de especificação de liberação de recursos
publicado no artigo {\it Iterative mining of resource-releasing specifications}
por Q. Wu
no ASE 2011,
disponibilizado em \url{http://sa.seforge.org/RRFinder/}
mas inacessível em 09/08/2017.

Software sem informações sobre lançamentos ou releases,
uma busca por citações no {\bf IEEE Xplore} por
\texttt{(RRFinder)}
e no {\bf ACM} com
\texttt{content.ftsec:(+RRFinder)}
retornou
5 resultados, apenas
{\bf 0} faz referência ao software.



\section{Sapid/XML}

Representação intermediária de código Java usando XML ao invés de AST
publicado no artigo {\it A CASE tool platform using an XML representation of Java source code}
por K. Maruyama
no SCAM 2004,
disponibilizado em \url{http://www.jtool.org}
mas inacessível em 09/08/2017.

Software sem informações sobre lançamentos ou releases,
uma busca por citações no {\bf IEEE Xplore} por
\texttt{("Sapid/XML")}
e no {\bf ACM} com
\texttt{content.ftsec:(+"Sapid/XML")}
retornou
5 resultados, apenas
{\bf 0} faz referência ao software.



\section{Sonar Qube Plug-in}

Extende o SourceMeter com análise de código Java com o uso do SonarQube
publicado no artigo {\it Source Meter Sonar Qube Plug-in}
por R. Ferenc
no SCAM 2014,
disponibilizado em \url{http://github.com/FrontEndART/SonarQube-plug-in}
gratuitamente
sob uma licença FrontEndART Software Ltd.

Software com lançamentos frequentes,
escrito em Java,
uma busca por citações no {\bf IEEE Xplore} por
\texttt{(SonarQube-plug-in)}
e no {\bf ACM} com
\texttt{content.ftsec:(+"SonarQube-plug-in")}
retornou
2 resultados, apenas
{\bf 0} faz referência ao software.



\section{SPARTA - Static Program Analysis for Reliable Trusted Apps}

Segurança pra detecção de malware
publicado no artigo {\it Static Analysis of Implicit Control Flow: Resolving Java Reflection and Android Intents (T)}
por P. Barros
no ASE 2015,
disponibilizado em \url{http://types.cs.washington.edu/sparta/}
gratuitamente
sem uma licença definida.

Software com lançamentos ocsaionais,
escrito em Java,
uma busca por citações no {\bf IEEE Xplore} por
\texttt{(SPARTA "Program Analysis")}
e no {\bf ACM} com
\texttt{content.ftsec:(+SPARTA +"Program Analysis")}
retornou
7 resultados, apenas
{\bf 0} faz referência ao software.



\section{srcML}

Transformação source-to-source
publicado no artigo {\it Lightweight Transformation and Fact Extraction with the srcML Toolkit}
por M. L. Collard
no SCAM 2011,
disponibilizado em \url{http://www.sdml.info/projects/srcml/trunk}
como software livre
sob uma licença GPL v3.

Software com lançamentos ocsaionais,
escrito em C++,
uma busca por citações no {\bf IEEE Xplore} por
\texttt{(srcML Toolkit Java C)}
e no {\bf ACM} com
\texttt{content.ftsec:(+srcML +Toolkit +Java +"C++")}
retornou
43 resultados, apenas
{\bf 0} faz referência ao software.



\section{SWAT - Search based Web Application Tester}

Teste automático para aplicação web
publicado no artigo {\it Automated Web Application Testing Using Search Based Software Engineering}
por Alshahwan, Nadia
no ASE 2011,
disponibilizado em \url{http://www.cs.ucl.ac.uk/staff/nalshahw/swat}
mas inacessível em 09/08/2017.

Software sem informações sobre lançamentos ou releases,
uma busca por citações no {\bf IEEE Xplore} por
\texttt{(SWAT Search Web Application Tester)}
e no {\bf ACM} com
\texttt{content.ftsec:(+SWAT +Search +Web +Application +Tester)}
retornou
15 resultados, apenas
{\bf 0} faz referência ao software.



\section{TACLE - Type Analysis and CalL graph construction for Eclipse}

Análise de tipo (Type Analysis) e construção e visualizaçao de grafos de chamada (Call Graph)
publicado no artigo {\it Estimating the Run-Time Progress of a Call Graph Construction Algorithm53-62}
por J. Sawin
no SCAM 2006,
disponibilizado em \url{http://presto.cse.ohio-state.edu/tacle}
gratuitamente
sem uma licença definida.

Software considerado obsoleto,
escrito em Java,
uma busca por citações no {\bf IEEE Xplore} por
\texttt{(TACLE "Type Analysis")}
e no {\bf ACM} com
\texttt{content.ftsec:(+TACLE +"Type Analysis")}
retornou
4 resultados, apenas
{\bf 0} faz referência ao software.



\section{TEBA}

Transformação source-to-source
publicado no artigo {\it A Pattern Search Method for Unpreprocessed C Programs Based on Tokenized Syntax Trees}
por A. Yoshida
no SCAM 2014,
disponibilizado em \url{http://tebasaki.jp/src}
gratuitamente
sem uma licença definida.

Software com lançamentos ocsaionais,
escrito em Perl,
uma busca por citações no {\bf IEEE Xplore} por
\texttt{(((TEBA) AND tool) AND Programs)}
e no {\bf ACM} com
\texttt{content.ftsec:(+TEBA +tool)}
retornou
11 resultados, apenas
{\bf 0} faz referência ao software.



\section{TestEra}

Geração automática de testes
publicado no artigo {\it TestEra: A Novel Framework for Automated Testing of Java Programs}
por Marinov, Darko
no ASE 2001,
disponibilizado em \url{http://www.mit.edu/~sarfraz/testera/}
mas inacessível em 09/08/2017.

Software sem informações sobre lançamentos ou releases,
uma busca por citações no {\bf IEEE Xplore} por
\texttt{(((((TestEra) AND framework) AND Java) AND testing) AND sarfraz)}
e no {\bf ACM} com
\texttt{content.ftsec:(+TestEra +framework +Java +testing +sarfraz)}
retornou
44 resultados, apenas
{\bf 0} faz referência ao software.



\section{Vdiff}

Visualização de diferença de código-fonte
publicado no artigo {\it A Program Differencing Algorithm for Verilog HDL}
por Duley, Adam
no ASE 2010,
disponibilizado em \url{http://web.cs.ucla.edu/~miryung/software/vdiff/web/index.html}
mas inacessível em 09/08/2017.

Software sem informações sobre lançamentos ou releases,
uma busca por citações no {\bf IEEE Xplore} por
\texttt{(((Vdiff) AND verilog) AND HDL)}
e no {\bf ACM} com
\texttt{content.ftsec:(+Vdiff +verilog +HDL)}
retornou
10 resultados, apenas
{\bf 0} faz referência ao software.



\section{WALA}

Análise estática de bytecode Java
publicado no artigo {\it Effective Static Analysis to Find Concurrency Bugs in Java}
por Z. D. Luo
no SCAM 2010,
disponibilizado em \url{http://wala.sourceforge.net/wiki/index.php/Main_Page}
como software livre
sob uma licença Eclipse Public License v1.0.

Software com lançamentos ocsaionais,
escrito em Java,
uma busca por citações no {\bf IEEE Xplore} por
\texttt{(((WALA) AND "Static Analysis") AND "Concurrency Bugs")}
e no {\bf ACM} com
\texttt{content.ftsec:(+WALA +"Static Analysis" +"Concurrency Bugs")}
retornou
12 resultados, apenas
{\bf 0} faz referência ao software.



\section{Wrangler}

Refatoração de código Erlang
publicado no artigo {\it Refactoring Support for Modularity Maintenance in Erlang}
por H. Li
no SCAM 2010,
disponibilizado em \url{http://www.cs.kent.ac.uk/projects/wrangler/Home.html}
como software livre
sob uma licença BSD License "revised".

Software com lançamentos ocsaionais,
escrito em Erlang,
uma busca por citações no {\bf IEEE Xplore} por
\texttt{((Wrangler) AND Erlang)}
e no {\bf ACM} com
\texttt{content.ftsec:(+Wrangler +Erlang)}
retornou
37 resultados, apenas
{\bf 0} faz referência ao software.



\section{XOgastan}

Transformação source-to-source
publicado no artigo {\it XOgastan: XML-oriented gcc AST analysis and transformations}
por G. Antoniol
no SCAM 2003,
disponibilizado em \url{http://web.ing.unisannio.it/villano/students/masone}
mas inacessível em 09/08/2017.

Software sem informações sobre lançamentos ou releases,
uma busca por citações no {\bf IEEE Xplore} por
\texttt{(XOgastan)}
e no {\bf ACM} com
\texttt{content.ftsec:(+XOgastan)}
retornou
7 resultados, apenas
{\bf 0} faz referência ao software.




