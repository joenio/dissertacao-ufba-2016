
% o conteúdo deste capítulo é gerado automaticamente pelo
% script bin/softwares-summary

\xchapter{Softwares acadêmicos}{Este capítulo ...}
\label{softwares-summary}

\section{2LS - 2nd order Logic Solving}

Análise de terminação para programas C usando resumo interprocedural baseado em modelos.

Publicado em: Synthesising Interprocedural Bit-Precise Termination Proofs (T),
H. Y. Chen,
ASE
2015.

\section{AccessAnalysis}

Cálculo de métricas IGAT e IGAM.

Publicado em: AccessAnalysis: A Tool for Measuring the Appropriateness of Access Modifiers in Java Systems,
C. Zoller,
SCAM
2012.

\section{APIExample}

Extração de informações de API Java e documentação automática com exemplos.

Publicado em: APIExample: An effective web search based usage example recommendation system for java APIs,
Lijie Wang,
ASE
2011.

\section{BEG - Bandera environment generator}

Criação automática de ambientes para verificação de modelos Java.

Publicado em: Automated environment generation for software model checking,
O. Tkachuk,
ASE
2003.

\section{Kiasan/Bogor}

Verificação de modelos.

Publicado em: Bogor/Kiasan: A k-bounded Symbolic Execution for Checking Strong Heap Properties of Open Systems,
X. Deng,
ASE
2006.

\section{ccJava}

Linguagem orientada a aspectos.

Publicado em: An Aspect-oriented Weaving Mechanism Based on Component and Connector Architecture,
Ubayashi, Naoyasu,
ASE
2007.

\section{CIVL - Concurrency intermediate verification language}

Framework para verificação de programas concorrentes.

Publicado em: CIVL: Formal Verification of Parallel Programs,
M. Zheng,
ASE
2015.

\section{CodeBoost}

Transformação source-to-source para otimização de programas C++.

Publicado em: Design of the CodeBoost transformation system for domain-specific optimisation of C++ programs,
O. S. Bagge,
SCAM
2003.

\section{composite}

Verificação de modelos.

Publicado em: Action Language Verifier,
T. Bultan,
ASE
2001.

\section{CPA+ - Configurable program analysis with dynamic precision adjustment}

Análise configurável de programa com ajuste dinâmico de precisão.

Publicado em: Program Analysis with Dynamic Precision Adjustment,
D. Beyer,
ASE
2008.

\section{CSeq}

Transformação source-to-source para programas C concorrentes.

Publicado em: CSeq: A concurrency pre-processor for sequential C verification tools,
B. Fischer,
ASE
2013.

\section{DDVerify}

Verificação de Linux drivers através de checagem de modelos.

Publicado em: Model Checking Concurrent Linux Device Drivers,
Witkowski, Thomas,
ASE
2007.

\section{Derailer}

Localização de falhas de segurança em aplicações web.

Publicado em: Derailer: Interactive Security Analysis for Web Applications,
Near, Joseph P.,
ASE
2014.

\section{Diagnosys}

Construção de interfaces de debug para o kernel Linux.

Publicado em: Diagnosys: automatic generation of a debugging interface to the Linux kernel,
T. F. Bissyandé,
ASE
2012.

\section{DOMPLETION}

Sugestão de código javascript.

Publicado em: Dompletion: DOM-aware JavaScript Code Completion,
Bajaj, Kartik,
ASE
2014.

\section{DRC - Dangling Reference Checker}

Análise estática para detecção de referências inválidas em código dinâmico PHP.

Publicado em: Dangling references in multi-configuration and dynamic PHP-based Web applications,
H. V. Nguyen,
ASE
2013.

\section{e-munity}

Verificação de segurança.

Publicado em: Scalable Security Verification of Software at Compile Time,
S. Tlili,
SCAM
2014.

\section{EJB}

Criação de diagramas de sequência.

Publicado em: I2SD: Reverse Engineering Sequence Diagrams from Enterprise Java Beans with Interceptors,
S. Roubtsov,
SCAM
2011.

\section{error-prone}

Localização de bugs em código Java construído em cima do compilador javac.

Publicado em: Building Useful Program Analysis Tools Using an Extensible Java Compiler,
E. Aftandilian,
SCAM
2012.

\section{ESBMC}

Verificação de modelos.

Publicado em: SMT-Based Bounded Model Checking for Embedded ANSI-C Software,
L. Cordeiro,
ASE
2009.

\section{ETXL}

Transformação de código.

Publicado em: Evolving TXL,
A. D. Thurston,
SCAM
2006.

\section{FaultBuster}

Refatoração de code smells.

Publicado em: FaultBuster: An automatic code smell refactoring toolset,
G. Szőke,
SCAM
2015.

\section{Flowgen}

Criação automática de grafos UML.

Publicado em: Flowgen: Flowchart-Based Documentation Framework for C++,
D. A. Kosower,
SCAM
2014.

\section{GRT}

Geração automática de testes.

Publicado em: GRT: An Automated Test Generator Using Orchestrated Program Analysis,
L. Ma,
ASE
2015.

\section{GUIZMO}

Inferência de layout.

Publicado em: Model-driven Reverse Engineering of Legacy Graphical User Interfaces,
S\&#39;{a}nchez Ram\&#39;{o}n, \&#39;{O}scar,
ASE
2010.

\section{GumTree}

Comparação de mudanças.

Publicado em: Fine-grained and Accurate Source Code Differencing,
Falleri, Jean-R{\&#39;e}my,
ASE
2014.

\section{HUSACCT}

?.

Publicado em: HUSACCT: Architecture Compliance Checking with Rich Sets of Module and Rule Types,
Pruijt, Leo J.,
ASE
2014.

\section{Indus}

Biblioteca de program slicing.

Publicado em: An Overview of the Indus Framework for Analysis and Slicing of Concurrent Java Software (Keynote Talk - Extended Abstract),
V. P. Ranganath,
SCAM
2006.

\section{JastAdd}

Análise de código-fonte através da descrição de atributos via gramática de atributos (AG).

Publicado em: Extending Attribute Grammars with Collection Attributes--Evaluation and Applications,
Magnusson, Eva,
SCAM
2007.

\section{JFlow}

Transformação source-to-source.

Publicado em: JFlow: Practical refactorings for flow-based parallelism,
N. Chen,
ASE
2013.

\section{JstereoCode}

Detecção de esteriótipos Java.

Publicado em: JStereoCode: automatically identifying method and class stereotypes in Java code,
L. Moreno,
ASE
2012.

\section{Jtop}

Gestão de casos de teste.

Publicado em: Jtop: Managing JUnit Test Cases in Absence of Coverage Information,
L. Zhang,
ASE
2009.

\section{Loopfrog}

Verificação de modelos.

Publicado em: Loopfrog: A Static Analyzer for ANSI-C Programs,
D. Kroening,
ASE
2009.

\section{Lotrack}

Análise estática de configuração.

Publicado em: Tracking Load-time Configuration Options,
Lillack, Max,
ASE
2014.

\section{MPAnalyzer}

Análise de padrões disponível.

Publicado em: MPAnalyzer: A Tool for Finding Unintended Inconsistencies in Program Source Code,
Higo, Yoshiki,
ASE
2014.

\section{MSP}

Construção de modelo formal de acesso a memória.

Publicado em: Recovering Memory Access Patterns of Executable Programs,
Ketterlin, Alain,
SCAM
2010.

\section{mygcc}

Verificação de programas C.

Publicado em: A Portable Compiler-Integrated Approach to Permanent Checking,
N. Volanschi,
ASE
2006.

\section{PARSEWeb}

Query para apoio e sugestão de reuso de bibliotecas.

Publicado em: Parseweb: A Programmer Assistant for Reusing Open Source Code on the Web,
Thummalapenta, Suresh,
ASE
2007.

\section{PAT}

Ambiente de teste automático.

Publicado em: Puzzle-based automatic testing: bringing humans into the loop by solving puzzles,
N. Chen,
ASE
2012.

\section{PHP AiR}

?.

Publicado em: Static, Lightweight Includes Resolution for PHP,
Hills, Mark,
ASE
2014.

\section{protopurity}

Análise de impacto.

Publicado em: Detecting function purity in JavaScript,
J. Nicolay,
SCAM
2015.

\section{Pseudogen}

Transformação de código-fonte em pseudo-código.

Publicado em: Pseudogen: A Tool to Automatically Generate Pseudo-Code from Source Code,
H. Fudaba,
ASE
2015.

\section{PtYasm}

Verificação de modelos.

Publicado em: Augmenting Counterexample-Guided Abstraction Refinement with Proof Templates,
T. E. Hart,
ASE
2008.

\section{PuMoC}

Verificação de modelos.

Publicado em: PuMoC: a CTL model-checker for sequential programs,
F. Song,
ASE
2012.

\section{PYTHIA}

Criação automática de casos de teste.

Publicado em: PYTHIA: Generating test cases with oracles for JavaScript applications,
S. Mirshokraie,
ASE
2013.

\section{ReAssert}

Localização de falhas em testes e refatoração.

Publicado em: ReAssert: Suggesting Repairs for Broken Unit Tests,
B. Daniel,
ASE
2009.

\section{Rêve}

Verificação de regressão.

Publicado em: Automating Regression Verification,
Felsing, Dennis,
ASE
2014.

\section{RRFinder}

Mineração de especificação de liberação de recursos.

Publicado em: Iterative mining of resource-releasing specifications,
Q. Wu,
ASE
2011.

\section{Sapid/XML}

Representação intermediária de código Java usando XML ao invés de AST.

Publicado em: A CASE tool platform using an XML representation of Java source code,
K. Maruyama,
SCAM
2004.

\section{Sonar Qube Plug-in}

Extende o SourceMeter com análise de código Java com o uso do SonarQube.

Publicado em: Source Meter Sonar Qube Plug-in,
R. Ferenc,
SCAM
2014.

\section{SPARTA}

Segurança pra detecção de malware.

Publicado em: Static Analysis of Implicit Control Flow: Resolving Java Reflection and Android Intents (T),
P. Barros,
ASE
2015.

\section{srcML}

Transformação source-to-source.

Publicado em: Lightweight Transformation and Fact Extraction with the srcML Toolkit,
M. L. Collard,
SCAM
2011.

\section{SWAT}

Teste automático para aplicação web.

Publicado em: Automated Web Application Testing Using Search Based Software Engineering,
Alshahwan, Nadia,
ASE
2011.

\section{TACLE}

Análise de tipo (Type Analysis) e construção e visualizaçao de grafos de chamada (Call Graph).

Publicado em: Estimating the Run-Time Progress of a Call Graph Construction Algorithm53-62,
J. Sawin,
SCAM
2006.

\section{TEBA}

Transformação source-to-source.

Publicado em: A Pattern Search Method for Unpreprocessed C Programs Based on Tokenized Syntax Trees,
A. Yoshida,
SCAM
2014.

\section{TestEra}

Geração automática de testes.

Publicado em: TestEra: A Novel Framework for Automated Testing of Java Programs,
Marinov, Darko,
ASE
2001.

\section{Vdiff}

Visualização de diferença de código-fonte.

Publicado em: A Program Differencing Algorithm for Verilog HDL,
Duley, Adam,
ASE
2010.

\section{WALA}

Análise estática de bytecode Java.

Publicado em: Effective Static Analysis to Find Concurrency Bugs in Java,
Z. D. Luo,
SCAM
2010.

\section{Wrangler}

Refatoração de código Erlang.

Publicado em: Refactoring Support for Modularity Maintenance in Erlang,
H. Li,
SCAM
2010.

\section{XOgastan}

Transformação source-to-source.

Publicado em: XOgastan: XML-oriented gcc AST analysis and transformations,
G. Antoniol,
SCAM
2003.


