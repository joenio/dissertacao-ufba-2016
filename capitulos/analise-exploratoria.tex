\xchapter{Análise exploratória dos valores das métricas}
{Este capítulo apresenta uma análise exploratória e interpretação dos valores das métricas coletadas para cada ferramenta.}
\label{analise-metricas}

\subsection{Conexões aferentes de uma classe (ACC)}

ACC é um valor parcial de uma das métricas MOOD (Metrics for Object Oriented
Design) \cite{Brito1994} e mede o nível de acoplamento de uma classe. O
cálculo é feito através do número de classes que fazem referência a um outra
por meio de métodos ou atributos.

As Tabelas \ref{metrica-acc} e \ref{metrica-acc-industria} apresentam os
valores da métrica ACC para as ferramentas da academia e da indústria,
respectivamente.

%% begin.rcode metrica-acc, fig.align='center', results="asis"
% table = percentis_by_project("acc")
% total_modules = metric_by_project("total_modules")
% table = add_column(table, total_modules, colname = "classes")
% knitr_latex_table(table, "percentis da métrica ACC para as ferramentas da academia", "metrica-acc")
%% end.rcode

Duas ferramentas, accessanalysis e error-prone, tiveram valor 0 no percentil
75, é estranho que 75\% das classes destas 2 ferramentas, ambas escritas em
Java, não façam acesso através de métodos ou atributos a nenhuma outra classe
do mesmo sistema.

%% begin.rcode metrica-acc-industria, fig.align='center', results="asis"
% table = percentis_by_nist_project("acc")
% total_modules = metric_by_nist_project("total_modules")
% table = add_column(table, total_modules, colname = "classes")
% knitr_latex_table(table, "percentis da métrica ACC para as ferramentas da indústria", "metrica-acc-industria")
%% end.rcode

As ferramentas cqual e uno tiveram valores no percentil 75\% bem acima das
demais ferramentas, 24 e 34, respectivamente. Onde entre todas as outras o
maior valor foi 8.

\subsection{Média de complexidade ciclomática por método (ACCM)}

ACCM contabiliza o número de caminhos independentes que métodos de uma classe
pode seguir em sua execução. O cálculo é feito a partir do número de
estruturas condicionais encontrados nos métodos de um programa.

As Tabelas \ref{metrica-accm} e \ref{metrica-accm-industria} apresentam os
valores da métrica ACCM para as ferramentas da academia e da indústria,
respectivamente.

%% begin.rcode metrica-accm, fig.align='center', results="asis"
% table = percentis_by_project("accm")
% total_modules = metric_by_project("total_modules")
% table = add_column(table, total_modules, colname = "classes")
% knitr_latex_table(table, "percentis da métrica ACCM para as ferramentas da academia", "metrica-accm")
%% end.rcode

Esta métrica apresenta um comportamento sem muitas exceções dentro de cada
percentil, com intervalos entre 1.0 e 2.1 para percentil 75\%, 2.0 e 3.4 para
percentil 90\% e 2.9 e 5.0 para percentil 95\%.

%% begin.rcode metrica-accm-industria, fig.align='center', results="asis"
% table = percentis_by_nist_project("accm")
% total_modules = metric_by_nist_project("total_modules")
% table = add_column(table, total_modules, colname = "classes")
% knitr_latex_table(table, "percentis da métrica ACCM para as ferramentas da indústria", "metrica-accm-industria")
%% end.rcode

Também entre as ferramentas da indústria esta métrica não apresenta variações
muito grandes, com intervalos entre 1.0 e 6.9 para 75\%, 2.0 e 8.9 para
percentil 90\% e entre 4.0 e 15.6 para 95\%.

\subsection{Média do número de linhas de código por método (AMLOC)}

AMLOC é a média do número de linhas dos métodos de um módulo, apenas linhas
com código executável é calculada, comentários e linhas em branco são
desconsideradas do cálculo.

As Tabelas \ref{metrica-amloc} e \ref{metrica-amloc-industria} apresentam a
métrica AMLOC para as ferramentas da academia e da indústria, respectivamente.

%% begin.rcode metrica-amloc, fig.align='center', results="asis"
% table = percentis_by_project("amloc")
% total_modules = metric_by_project("total_modules")
% table = add_column(table, total_modules, colname = "classes")
% knitr_latex_table(table, "percentis da métrica AMLOC para as ferramentas da academia", "metrica-amloc")
%% end.rcode

Os valores para ferramentas da academia não nos chama atenção para nada
especial, com intervalos entre 5.4 e 14.5 para o percentil 75\%, entre 10.7 e
31.5 pra percentil 90\% e entre 15.4 e 51.4 para 95\%.

%% begin.rcode metrica-amloc-industria, fig.align='center', results="asis"
% table = percentis_by_nist_project("amloc")
% total_modules = metric_by_nist_project("total_modules")
% table = add_column(table, total_modules, colname = "classes")
% knitr_latex_table(table, "percentis da métrica AMLOC para as ferramentas da indústria", "metrica-amloc-industria")
%% end.rcode

Para a indústria os intervalos foram entre 7.0 e 36.9 para o percentil 75\%,
entre 16.0 e 62.1 para percentil 90\% e entre 22.0 e 119.7 para 95\%. A
ferramenta chama atenção por apresentr o menor valor no percentil 75\%, 7, e o
maior valor no prcentil 95\%, 119.7.

\subsection{Média do número de parâmetros por método (ANPM)}

ANPM é a média de parâmetros dos métodos de uma classe.

As Tabelas \ref{metrica-anpm} e \ref{metrica-anpm-industria} apresentam a
métrica ANPM para as ferramentas da academia e da indústria, respectivamente.

%% begin.rcode metrica-anpm, fig.align='center', results="asis"
% table = percentis_by_project("anpm")
% total_modules = metric_by_project("total_modules")
% table = add_column(table, total_modules, colname = "classes")
% knitr_latex_table(table, "percentis da métrica ANPM para as ferramentas da academia", "metrica-anpm")
%% end.rcode

%% begin.rcode metrica-anpm-industria, fig.align='center', results="asis"
% table = percentis_by_nist_project("anpm")
% total_modules = metric_by_nist_project("total_modules")
% table = add_column(table, total_modules, colname = "classes")
% knitr_latex_table(table, "percentis da métrica ANPM para as ferramentas da indústria", "metrica-anpm-industria")
%% end.rcode

\subsection{Acoplamento entre objetos (CBO)}

CBO é a recíproca da métrica ACC e mede quantas classes são utilizadas por uma
certa classe.

As Tabelas \ref{metrica-cbo} e \ref{metrica-cbo-industria} apresentam a
métrica CBO para as ferramentas da academia e da indústria, respectivamente.

%% begin.rcode metrica-cbo, fig.align='center', results="asis"
% table = percentis_by_project("cbo")
% total_modules = metric_by_project("total_modules")
% table = add_column(table, total_modules, colname = "classes")
% knitr_latex_table(table, "percentis da métrica CBO para as ferramentas da academia", "metrica-cbo")
%% end.rcode

%% begin.rcode metrica-cbo-industria, fig.align='center', results="asis"
% table = percentis_by_nist_project("cbo")
% total_modules = metric_by_nist_project("total_modules")
% table = add_column(table, total_modules, colname = "classes")
% knitr_latex_table(table, "percentis da métrica CBO para as ferramentas da indústria", "metrica-cbo-industria")
%% end.rcode

\subsection{Profundidade da árvore de herança (DIT)}

DIT mede a profundidade que uma classe se encontra na árvore de herança.

Os intervalos sugeridos são: até 2 (bom); entre 2 e 4 (regular); de 4 em
diante (ruim).

As Tabelas \ref{metrica-dit} e \ref{metrica-dit-industria} apresentam a
métrica DIT para as ferramentas da academia e da indústria, respectivamente.

%% begin.rcode metrica-dit, fig.align='center', results="asis"
% table = percentis_by_project("dit")
% total_modules = metric_by_project("total_modules")
% table = add_column(table, total_modules, colname = "classes")
% knitr_latex_table(table, "percentis da métrica DIT para as ferramentas da academia", "metrica-dit")
%% end.rcode

%% begin.rcode metrica-dit-industria, fig.align='center', results="asis"
% table = percentis_by_nist_project("dit")
% total_modules = metric_by_nist_project("total_modules")
% table = add_column(table, total_modules, colname = "classes")
% knitr_latex_table(table, "percentis da métrica DIT para as ferramentas da indústria", "metrica-dit-industria")
%% end.rcode

\subsection{Ausência de coesão em métodos (LCOM4)}

LCOM4 calcula quantos conjuntos de métodos relacionados existem dentro de uma
classe, isto é, métodos que compartilham utilização de algum atributo ou que
se referenciam.

Os intervalos sugeridos para código C++ e Java são: até 2 (bom); entre 2 e 5
(regular); de 5 em diante (ruim).

As Tabelas \ref{metrica-lcom4} e \ref{metrica-lcom4-industria} apresentam a
métrica LCOM4 para as ferramentas da academia e da indústria, respectivamente.

%% begin.rcode metrica-lcom4, fig.align='center', results="asis"
% table = percentis_by_project("lcom4")
% total_modules = metric_by_project("total_modules")
% table = add_column(table, total_modules, colname = "classes")
% knitr_latex_table(table, "percentis da métrica LCOM4 para as ferramentas da academia", "metrica-lcom4")
%% end.rcode

%% begin.rcode metrica-lcom4-industria, fig.align='center', results="asis"
% table = percentis_by_nist_project("lcom4")
% total_modules = metric_by_nist_project("total_modules")
% table = add_column(table, total_modules, colname = "classes")
% knitr_latex_table(table, "percentis da métrica LCOM4 para as ferramentas da indústria", "metrica-lcom4-industria")
%% end.rcode

\subsection{Número de linhas de código (LOC)}

LOC é a medida mais comum para o tamanho de um software, conta o número linhas
executáveis excluindo linhas em branco e comentários.

Os intervalos sugeridos para o LOC de uma classe (Java e C++) são: até 70
(bom); entre 70 e 130 (regular); de 130 em diante (ruim).

As Tabelas \ref{metrica-loc} e \ref{metrica-loc-industria} apresentam a
métrica LOC para as ferramentas da academia e da indústria, respectivamente.

%% begin.rcode metrica-loc, fig.align='center', results="asis"
% table = percentis_by_project("loc")
% total_modules = metric_by_project("total_modules")
% table = add_column(table, total_modules, colname = "classes")
% knitr_latex_table(table, "percentis da métrica LOC para as ferramentas da academia", "metrica-loc")
%% end.rcode

%% begin.rcode metrica-loc-industria, fig.align='center', results="asis"
% table = percentis_by_nist_project("loc")
% total_modules = metric_by_nist_project("total_modules")
% table = add_column(table, total_modules, colname = "classes")
% knitr_latex_table(table, "percentis da métrica LOC para as ferramentas da indústria", "metrica-loc-industria")
%% end.rcode

\subsection{Número de atributos (NOA)}

NOA contabiliza o número de atributos de uma classe.

As Tabelas \ref{metrica-noa} e \ref{metrica-noa-industria} apresentam a
métrica NOA para as ferramentas da academia e da indústria, respectivamente.

%% begin.rcode metrica-noa, fig.align='center', results="asis"
% table = percentis_by_project("noa")
% total_modules = metric_by_project("total_modules")
% table = add_column(table, total_modules, colname = "classes")
% knitr_latex_table(table, "percentis da métrica NOA para as ferramentas da academia", "metrica-noa")
%% end.rcode

%% begin.rcode metrica-noa-industria, fig.align='center', results="asis"
% table = percentis_by_nist_project("noa")
% total_modules = metric_by_nist_project("total_modules")
% table = add_column(table, total_modules, colname = "classes")
% knitr_latex_table(table, "percentis da métrica NOA para as ferramentas da indústria", "metrica-noa-industria")
%% end.rcode

\subsection{Número de filhos (NOC)}

NOC é o número total de flhos de uma classe.

As Tabelas \ref{metrica-noc} e \ref{metrica-noc-industria} apresentam a
métrica NOC para as ferramentas da academia e da indústria, respectivamente.

%% begin.rcode metrica-noc, fig.align='center', results="asis"
% table = percentis_by_project("noc")
% total_modules = metric_by_project("total_modules")
% table = add_column(table, total_modules, colname = "classes")
% knitr_latex_table(table, "percentis da métrica NOC para as ferramentas da academia", "metrica-noc")
%% end.rcode

%% begin.rcode metrica-noc-industria, fig.align='center', results="asis"
% table = percentis_by_nist_project("noc")
% total_modules = metric_by_nist_project("total_modules")
% table = add_column(table, total_modules, colname = "classes")
% knitr_latex_table(table, "percentis da métrica NOC para as ferramentas da indústria", "metrica-noc-industria")
%% end.rcode

\subsection{Número de métodos (NOM)}

NOM indica o tamanho das classes em termos das suas operações implementadas.

As Tabelas \ref{metrica-nom} e \ref{metrica-nom-industria} apresentam a
métrica NOM para as ferramentas da academia e da indústria, respectivamente.

%% begin.rcode metrica-nom, fig.align='center', results="asis"
% table = percentis_by_project("nom")
% total_modules = metric_by_project("total_modules")
% table = add_column(table, total_modules, colname = "classes")
% knitr_latex_table(table, "percentis da métrica NOM para as ferramentas da academia", "metrica-nom")
%% end.rcode

%% begin.rcode metrica-nom-industria, fig.align='center', results="asis"
% table = percentis_by_nist_project("nom")
% total_modules = metric_by_nist_project("total_modules")
% table = add_column(table, total_modules, colname = "classes")
% knitr_latex_table(table, "percentis da métrica NOM para as ferramentas da indústria", "metrica-nom-industria")
%% end.rcode

\subsection{Número de atributos públicos (NPA)}

NPA mede o encapsulamento entre classes.

Os intervalos sugeridos para Java e C++ são: até 1 (bom); entre 1 e 9
(regular); de 9 em diante (ruim).

As Tabelas \ref{metrica-npa} e \ref{metrica-npa-industria} apresentam a
métrica NPA para as ferramentas da academia e da indústria, respectivamente.

%% begin.rcode metrica-npa, fig.align='center', results="asis"
% table = percentis_by_project("npa")
% total_modules = metric_by_project("total_modules")
% table = add_column(table, total_modules, colname = "classes")
% knitr_latex_table(table, "percentis da métrica NPA para as ferramentas da academia", "metrica-npa")
%% end.rcode

%% begin.rcode metrica-npa-industria, fig.align='center', results="asis"
% table = percentis_by_nist_project("npa")
% total_modules = metric_by_nist_project("total_modules")
% table = add_column(table, total_modules, colname = "classes")
% knitr_latex_table(table, "percentis da métrica NPA para as ferramentas da indústria", "metrica-npa-industria")
%% end.rcode

\subsection{Número de métodos públicos (NPM)}

NPM indica o tamanho da ``interface'' da classe.

Os intervalos sugeridos para Java e C++ são: até 10 (bom); entre 10 e 40
(regular); de 40 em diante (ruim).

As Tabelas \ref{metrica-npm} e \ref{metrica-npm-industria} apresentam a
métrica NPA para as ferramentas da academia e da indústria, respectivamente.

%% begin.rcode metrica-npm, fig.align='center', results="asis"
% table = percentis_by_project("npm")
% total_modules = metric_by_project("total_modules")
% table = add_column(table, total_modules, colname = "classes")
% knitr_latex_table(table, "percentis da métrica NPM para as ferramentas da academia", "metrica-npm")
%% end.rcode

%% begin.rcode metrica-npm-industria, fig.align='center', results="asis"
% table = percentis_by_nist_project("npm")
% total_modules = metric_by_nist_project("total_modules")
% table = add_column(table, total_modules, colname = "classes")
% knitr_latex_table(table, "percentis da métrica NPM para as ferramentas da indústria", "metrica-npm-industria")
%% end.rcode

\subsection{Resposta para uma classe (RFC)}

RFC conta o número de métodos que podem ser executados a partir de uma
mensagem enviada a um objeto dessa classe.

As Tabelas \ref{metrica-rfc} e \ref{metrica-rfc-industria} apresentam a
métrica RFC para as ferramentas da academia e da indústria, respectivamente.

%% begin.rcode metrica-rfc, fig.align='center', results="asis"
% table = percentis_by_project("rfc")
% total_modules = metric_by_project("total_modules")
% table = add_column(table, total_modules, colname = "classes")
% knitr_latex_table(table, "percentis da métrica RFC para as ferramentas da academia", "metrica-rfc")
%% end.rcode

%% begin.rcode metrica-rfc-industria, fig.align='center', results="asis"
% table = percentis_by_nist_project("rfc")
% total_modules = metric_by_nist_project("total_modules")
% table = add_column(table, total_modules, colname = "classes")
% knitr_latex_table(table, "percentis da métrica RFC para as ferramentas da indústria", "metrica-rfc-industria")
%% end.rcode

\subsection{Complexidade estrutural (SC)}

SC é medida através da combinação das métricas de acoplamento (CBO) e coesão
(LCOM4).

As Tabelas \ref{metrica-sc} e \ref{metrica-sc-industria} apresentam a
métrica SC para as ferramentas da academia e da indústria, respectivamente.

%% begin.rcode metrica-sc, fig.align='center', results="asis"
% table = percentis_by_project("sc")
% total_modules = metric_by_project("total_modules")
% table = add_column(table, total_modules, colname = "classes")
% knitr_latex_table(table, "percentis da métrica SC para as ferramentas da academia", "metrica-sc")
%% end.rcode

%% begin.rcode sumario-sc, fig.align='center', results="asis"
% table = percentis_by_project("sc")
% table = table[-1:-5,]
% table = table[-4,]
% xt = xtable(summary(t(table)), caption="resumo da métrica SC nos percentis 75, 90 e 95 para as ferramentas da academia")
% print(xt, table.placement="H", caption.placement="top")
%% end.rcode

%% begin.rcode metrica-sc-industria, fig.align='center', results="asis"
% table = percentis_by_nist_project("sc")
% total_modules = metric_by_nist_project("total_modules")
% table = add_column(table, total_modules, colname = "classes")
% knitr_latex_table(table, "percentis da métrica SC para as ferramentas da indústria", "metrica-sc-industria")
%% end.rcode

%% begin.rcode sumario-sc-industria, fig.align='center', results="asis"
% table = percentis_by_nist_project("sc")
% table = table[-1:-5,]
% table = table[-4,]
% xt = xtable(summary(t(table)), caption="resumo da métrica SC nos percentis 75, 90 e 95 para as ferramentas da indústria")
% print(xt, table.placement="H", caption.placement="top")
%% end.rcode

\xchapter{Histogramas}{}

%% begin.rcode histograma-acc, fig.align='center', results="asis"
% histograma('acc', 'histograma da métrica ACC para todas as ferramentas')
%% end.rcode

%% begin.rcode histograma-accm, fig.align='center', results="asis"
% histograma('accm', 'histograma da métrica ACCM para todas as ferramentas')
%% end.rcode

%% begin.rcode histograma-cbo, fig.align='center', results="asis"
% histograma('cbo', 'histograma da métrica CBO para todas as ferramentas')
%% end.rcode

%% begin.rcode histograma-loc, fig.align='center', results="asis"
% histograma('loc', 'histograma da métrica LOC para todas as ferramentas')
%% end.rcode

%% begin.rcode histograma-amloc, fig.align='center', results="asis"
% histograma('amloc', 'histograma da métrica AMLOC para todas as ferramentas')
%% end.rcode

%% begin.rcode histograma-anpm, fig.align='center', results="asis"
% histograma('anpm', 'histograma da métrica ANPM para todas as ferramentas')
%% end.rcode

%% begin.rcode histograma-dit, fig.align='center', results="asis"
% histograma('dit', 'histograma da métrica DIR para todas as ferramentas')
%% end.rcode

%% begin.rcode histograma-rfc, fig.align='center', results="asis"
% histograma('rfc', 'histograma da métrica RFC para todas as ferramentas')
%% end.rcode

%% begin.rcode histograma-noa, fig.align='center', results="asis"
% histograma('noa', 'histograma da métrica NOA para todas as ferramentas')
%% end.rcode

%% begin.rcode histograma-noc, fig.align='center', results="asis"
% histograma('noc', 'histograma da métrica NOC para todas as ferramentas')
%% end.rcode

%% begin.rcode histograma-npm, fig.align='center', results="asis"
% histograma('npm', 'histograma da métrica NPM para todas as ferramentas')
%% end.rcode

%% begin.rcode histograma-nom, fig.align='center', results="asis"
% histograma('nom', 'histograma da métrica NOM para todas as ferramentas')
%% end.rcode

%% begin.rcode histograma-sc, fig.align='center', results="asis"
% histograma('sc', 'histograma da métrica SC para todas as ferramentas')
%% end.rcode

