\xchapter{Análise da complexidade estrutural dos softwares científicos}
{...}
\label{manutenabilidade-ferramentas}

(pendente)

usar a dimensao abaixo para definir quais usar na avaliacao longitudinal:

%, e qual a frequência de lançamentos indicando se são
%atualizadas frequentemente ou estão obsoletas.

\begin{description}

  \item {\it Lançamentos ({\it Releases}) - quantos lançamentos por ano:}
    \begin{itemize}
      \item Frequentemente $>=$ 3 vezes ao ano\\
        {\it \small novas versões da ferramenta são lançadas 3 ou mais vezes por ano}
      \item Ocasionalmente $<$ 3 vezes ao ano\\
        {\it \small novas versões da ferramenta são lançadas menos que 3 vezes ao ano}
      \item Obsoleta 0 vezes ao ano\\
        {\it \small intervalo entre novos lançamentos é maior que 1 ano}
    \end{itemize}

\end{description}

O autor não deixa claro como categorizar softwares sem lançamentos nos últimos
anos mas com histórico de lançamento frequente em anos anteriores. Assim, será
considerado todo o histórico de lançamentos e só serão considerados obsoletos
por exemplo softwares que nunca tenha tido mais de 1 lançamento ao ano
considerando todo o histórico dele. Da mesma forma, será considerado ocasional
apenas aqueles que sempre tiveram no máximo 2 lançamentos ao ano. Esta dimensão
irá nos dizer o grau de evolução de cada ferramenta, considerando que softwares
com lançamentos frequentes estão evoluindo.

======================

\begin{description}

  \item {\it Entrada - quais tipos de arquivos podem ser carregados na ferramenta:}
    \begin{itemize}
      \item Código-fonte - arquivos de código texto podem ser carregados
      \item Byte code - arquivos com Java Byte Code ou Microsoft
      \item Linguagem intermediária (MSIL) pode ser carregada
    \end{itemize}

  \item {\it Linguagens suportadas - quais linguagens de programação a ferramenta suporta:}
    \begin{itemize}
      \item .NET - todas as linguagens compiladas em bibliotecas ou programas no framework .NET
      \item VB .NET - suporta VB.NET
      \item C\# - suporta C\#
      \item Java - suporta linguagem de programação Java
      \item C, C++ - suporta linguagem de programação C ou C++
    \end{itemize}

\end{description}

As dimensões apresentada por \citeonline{Novak2010} não cobrem alguns aspectos
importantes percebidos ao longo deste estudo, assim novas dimensões serão utilizadas
em complemento às dimensões citadas acima.

\begin{description}

  \item {\it Linguagem de programação - em que linguagem de programação à ferramenta é escrita:}
    \begin{itemize}
      \item .NET
      \item VB .NET
      \item C\#
      \item Java
      \item C, C++
    \end{itemize}

\end{description}


================


A ferramenta livre {\it sloccount}\footnote{http://www.dwheeler.com/sloccount}
foi utilizada para identificar a linguagem de programação em que cada
ferramenta é implementada. O tamanho em número de classes foi extraído utilizando a ferramenta
Analizo, uma das inúmeras métricas que ela extraí é a contagem do número total
de classes de um sistema. 

===============

\section{Data collection procedure}

\begin{itemize}
  \item Pesquisa livre em fontes na internet para busca e seleção de ferramentas da indústria
  \item Obtenção do código-fonte de cada ferramenta
  \item Cálculo e coleta de métricas de código-fonte
  \item Obtenção de código-fonte de mais versões de ferramentas com {\it Lançamentos} frequentes ou ocasionais
  \item Cálculo e coleta de métricas de código-fonte das novas versões
\end{itemize}

==============

\subsection{Analizo}

Numa primeira análise dos valores coletados pelo Analizo notamos uma anomalia
nos valores da métrica CBO, o que nos levou a investigar de perto os motivos,
esta anomalia se apresentava como valores extremamente altos para esta métrica,
bastante discrepante com as demais métricas calculadas.

Para entender se estes valores estavam corretor ou não, utilizamos uma outra
ferramenta para cálculo das métricas, em nossos estudos encontramos e
utilizamos uma versão de avaliação da ferramenta {\it SciTools
Understand}\footnote{http://scitools.com/trial-download-3} em sua versão
``4.0.853'' em Linux 64 bits. Os dados extraídos por esta ferramenta podem ser
encontrados em nosso
repositório\footnote{http://github.com/joenio/dissertacao-ufba-2016/tree/master/dataset/Understand
SciTools}. Eles demonstraram que os valores calculados pelo Analizo estavam
bastante alto em comparação com as demais métricas.

Assim, descobrimos que o Analizo tinha de fato um erro no cálculo da métrica
CBO, erro que foi corrigido durante este estudo e disponibilizado na versão
mais recente do Analizo, versão que está sendo utilizada aqui.


=============

\subsection{Ferramentas da indústria}

Em paralelo à revisão estruturada para seleção de ferramentas da academia
foi realizada uma seleção manual no catálogo de ferramentas de análise estática do projeto
SAMATE\footnote{https://samate.nist.gov/index.php/Source\_Code\_Security\_Analyzers.html}
em busca de ferramentas da indústria.

O projeto SAMATE\footnote{http://samate.nist.gov} - {\em Software Assurance
Metrics and Tool Evaluation}, um projeto do NIST\footnote{http://nist.gov}
dedicado ao desenvolvimento de métodos que permitam avaliar e medir a
eficiência de ferramentas e técnicas sobre garantia de qualidade em software.
O site do projeto, disponível em \citeonline{SamateAnalysers}, mantém uma lista
de ferramentas de análise estática.

Nesta busca por ferramentas da indústria encontramos um total de 54 ferramentas
presentes no catálogo do projeto SAMATE, 19 tinham código-fonte disponível,
destas apenas 14 eram suportadas pelo Analizo (escritas em C, C++ ou Java).

Após download do código-fonte de cada ferramenta selecionada, em sua versão
mais recente, a ferramenta Analizo será utilizada para a coleta das métricas. 
A Tabela \ref{total-de-ferramentas} traz um resum com todas as ferramentas
selecionadas, tando da indústria quanto da academia.

=====

Deste total de 35, 4 tem lançamentos frequentes, 9 são obsoletas, 8 tem
lançamentos ocasional e 14 não possui informação sobre lançamentos.

=====

Uma vez identificados os artigos que publicaram ferramentas do domínio
desejado, procuramos no próprio artigo por referências de onde encontrar o
código-fonte da ferramenta. Neste momento, pode-se enfrentar algumas situações.

\begin{itemize}

  \item Os autores afirmam que a ferramenta está disponível mas o artigo
    não contém referências de onde encontrar o código-fonte, estes
    autores serão contactados, por email, solicitando informações de onde
    obter o código-fonte da ferramenta.

  \item O artigo indica onde obter o código-fonte da ferramenta, mas o acesso ao local
    indicado não está disponível, ou está disponível mas o software não se
    encontra lá, os autores serão contactados, solicitando informações
    atualizadas de onde obter uma cópia do código-fonte da ferramenta.

  \item Artigos que indicam onde obter o código-fonte da ferramenta e a referência
    está correta. Será feito o download do código-fonte da última versão
    disponível.

\end{itemize}

Uma vez que os autores sejam contactados por email e respondam com informações
sobre onde obter o software, a ferramenta é adicionada ao conjunto de
ferramentas a serem analisadas.

=================


\begin{description}

  \item {\it Contexto - onde a ferramenta surgiu:}
    \begin{itemize}
      \item Academia - foi desenvolvida inicialmente em contexto acadêmico
      \item Indústria - foi desenvolvido fora da academia
    \end{itemize}

  \item {\it Tamanho em número de classes - número de classes/módulos da ferramenta}

  \item {\it Nível de manutenabilidade - interpretação das seguintes métricas de código-fonte:}
    \begin{itemize}
      \item Complexidade Estrutural
      \item Custo de Mudança
    \end{itemize}

\end{description}


\subsection{Ferramentas da indústria}

Em paralelo à revisão estruturada para seleção de ferramentas da academia será
realizada uma seleção manual não estruturada para busca de ferramentas da
indústria. O objetivo é aumentar o conjunto de objetos de estudo bem como ter
ferramentas de outros contextos além da academia, isto irá proporcionar uma
nova dimensão na caracterzação das ferramentas e permitirá realizar comparação
entre ferramentas de contextos distintos.


==============================

Dentre estas ferramentas as seguintes foram selecionadas para análise evolutiva:

\begin{itemize}
  \item Closure Compiler         
  \item FindBugs                 
  \item PMD                      
  \item WALA                    
  \item error-prone
  \item JastAdd
  \item SPARTA
  \item Cppcheck
  \item FindSecurityBugs
  \item Smatch
\end{itemize}

o clang foi removido pois a analise dele demorou muito, ficou rodando 1 semana q não
terminou, o analizo metrics que demorou tanto assim.

\begin{table}[H]
\caption{Métricas da ferramenta PMD}
  \centering
\begin{tabular}{|l|r|r|r|r|r|}
\hline
\multicolumn{1}{|c|}{\textbf{Release}} & \multicolumn{1}{c|}{\textbf{Classes}} & \multicolumn{1}{c|}{\textbf{CC}} & \multicolumn{1}{c|}{\textbf{SC 75}} & \multicolumn{1}{c|}{\textbf{SC 90}} & \multicolumn{1}{c|}{\textbf{SC 95}} \\ \hline
4.2.5 & 844 & 0,06 & 6 & 15 & 28 \\ \hline
4.3 & 852 & 0,06 & 6 & 15 & 28 \\ \hline
5.0.0 & 1043 & 0,03 & 6 & 14 & 25 \\ \hline
5.0.4 & 1052 & 0,02 & 6 & 12 & 24 \\ \hline
5.1.0 & 1238 & 0,02 & 6 & 12 & 24 \\ \hline
5.1.3 & 1254 & 0,02 & 6 & 12 & 25 \\ \hline
5.2.0 & 1295 & 0,02 & 6 & 12 & 25 \\ \hline
5.3.0 & 1341 & 0,01 & 6 & 12 & 25 \\ \hline
5.3.3 & 1342 & 0,02 & 6 & 12 & 25 \\ \hline
5.3.7 & 1374 & 0,01 & 6 & 12 & 25 \\ \hline
5.4.0 & 1332 & 0,02 & 6 & 12 & 26 \\ \hline
5.4.2 & 1366 & 0,02 & 6 & 12 & 26 \\ \hline
5.5.2 & 1530 & 0,01 & 6 & 12 & 25 \\ \hline
\multicolumn{6}{l}{\texttt{Notas:}} \\
\multicolumn{6}{l}{\texttt{CC = Custo de mudança}} \\
\multicolumn{6}{l}{\texttt{SC = Complexidade estrutural}} \\ \hline
\end{tabular}
\label{metricas-pmd}
\end{table}

\begin{table}[H]
\caption{Métricas da ferramenta WALA}
  \centering
\begin{tabular}{|l|r|r|r|r|r|}
\hline
\multicolumn{1}{|c|}{\textbf{Release}} & \multicolumn{1}{c|}{\textbf{Classes}} & \multicolumn{1}{c|}{\textbf{CC}} & \multicolumn{1}{c|}{\textbf{SC 75}} & \multicolumn{1}{c|}{\textbf{SC 90}} & \multicolumn{1}{c|}{\textbf{SC 95}} \\ \hline
1.0 & 1223 & 0,02 & 8 & 27 & 45 \\ \hline
1.0.02 & 1685 & 0,02 & 8 & 25 & 48 \\ \hline
1.1 & 1872 & 0,02 & 7 & 24 & 48 \\ \hline
1.1.2 & 1720 & 0,02 & 8 & 24 & 50 \\ \hline
1.2 & 1734 & 0,02 & 7 & 25 & 49 \\ \hline
1.2.1.1 & 1901 & 0,02 & 6 & 24 & 48 \\ \hline
1.2.2 & 1903 & 0,02 & 6 & 24 & 48 \\ \hline
1.3 & 1945 & 0,02 & 6 & 24 & 52 \\ \hline
1.3.3 & 2092 & 0,01 & 6 & 24 & 50 \\ \hline
1.3.5 & 2143 & 0,02 & 6 & 22 & 49 \\ \hline
1.3.6 & 2154 & 0,02 & 6 & 22 & 49 \\ \hline
1.3.8 & 2626 & 0,02 & 7 & 24 & 54 \\ \hline
1.3.9 & 2636 & 0,02 & 7 & 24 & 54 \\ \hline
\multicolumn{6}{l}{\texttt{Notas:}} \\
\multicolumn{6}{l}{\texttt{CC = Custo de mudança}} \\
\multicolumn{6}{l}{\texttt{SC = Complexidade estrutural}} \\ \hline
\end{tabular}
\label{metricas-wala}
\end{table}

\begin{table}[H]
\caption{Métricas da ferramenta FindBugs}
  \centering
\begin{tabular}{|l|r|r|r|r|r|}
\hline
\multicolumn{1}{|c|}{\textbf{Release}} & \multicolumn{1}{c|}{\textbf{Classes}} & \multicolumn{1}{c|}{\textbf{CC}} & \multicolumn{1}{c|}{\textbf{SC 75}} & \multicolumn{1}{c|}{\textbf{SC 90}} & \multicolumn{1}{c|}{\textbf{SC 95}} \\ \hline
1.2.1 & 1044 & 0,05 & 7 & 20 & 36 \\ \hline
1.3.4 & 1216 & 0,06 & 7 & 21 & 42 \\ \hline
1.3.5 & 1257 & 0,05 & 7 & 21 & 40 \\ \hline
1.3.6 & 1258 & 0,05 & 8 & 21 & 42 \\ \hline
1.3.7 & 1261 & 0,05 & 7 & 22 & 42 \\ \hline
1.3.8 & 1275 & 0,05 & 7 & 22 & 42 \\ \hline
1.3.9 & 1354 & 0,06 & 7 & 24 & 48 \\ \hline
2.0.0 & 1459 & 0,06 & 7 & 24 & 52 \\ \hline
2.0.1 & 1465 & 0,06 & 7 & 24 & 54 \\ \hline
2.0.2 & 1469 & 0,06 & 7 & 24 & 56 \\ \hline
2.0.3 & 1489 & 0,06 & 7 & 24 & 56 \\ \hline
3.0.0 & 1438 & 0,07 & 7 & 24 & 56 \\ \hline
3.0.1 & 1486 & 0,07 & 8 & 25 & 56 \\ \hline
\multicolumn{6}{l}{\texttt{Notas:}} \\
\multicolumn{6}{l}{\texttt{CC = Custo de mudança}} \\
\multicolumn{6}{l}{\texttt{SC = Complexidade estrutural}} \\ \hline
\end{tabular}
\label{metricas-findbugs}
\end{table}

\begin{table}[H]
\caption{Métricas da ferramenta Closure Compiler}
  \centering
\begin{tabular}{|l|r|r|r|r|r|}
\hline
\multicolumn{1}{|c|}{\textbf{Release}} & \multicolumn{1}{c|}{\textbf{Classes}} & \multicolumn{1}{c|}{\textbf{CC}} & \multicolumn{1}{c|}{\textbf{SC 75}} & \multicolumn{1}{c|}{\textbf{SC 90}} & \multicolumn{1}{c|}{\textbf{SC 95}} \\ \hline
20110119 & 1122 & 0,05 & 4 & 17 & 42 \\ \hline
20110811 & 1730 & 0,1 & 5 & 20 & 40 \\ \hline
20120305 & 1802 & 0,1 & 5 & 20 & 42 \\ \hline
20120917 & 1836 & 0,1 & 6 & 20 & 42 \\ \hline
20130227 & 1759 & 0,11 & 6 & 20 & 48 \\ \hline
20130722 & 1806 & 0,1 & 6 & 20 & 48 \\ \hline
20140110 & 2004 & 0,08 & 5 & 20 & 45 \\ \hline
20140730 & 1573 & 0,04 & 4 & 20 & 48 \\ \hline
20150126 & 1596 & 0,04 & 5 & 22 & 56 \\ \hline
20150729 & 1649 & 0,04 & 6 & 26 & 65 \\ \hline
20160125 & 1724 & 0,04 & 5 & 28 & 68 \\ \hline
20160713 & 1860 & 0,04 & 6 & 30 & 70 \\ \hline
\multicolumn{6}{l}{\texttt{Notas:}} \\
\multicolumn{6}{l}{\texttt{CC = Custo de mudança}} \\
\multicolumn{6}{l}{\texttt{SC = Complexidade estrutural}} \\ \hline
\end{tabular}
\label{metricas-closurecompiler}
\end{table}

As ferramentas com poucas linhas de código foram excluidas, estas
apreentam Change Cost alto, já é conhecido que a definição desta métrica
sofre deste problema, apresenta valores altos em projetos muito pequenos,
tambem removemos da analise aquelas ferramentas que nao tiveram valor
no percentil 75\%, pois a comparacao e analise se dará neste percentil
principalmente.

Com isso temos 11 projetos, destes iremos analisar longitudalmente
as releases e a evolucao dos valores de SC e CC (Change Cost), sao elas:

 Closure Compiler         13 releases analisados

 FindBugs                 13 releases

 Indus                    (poucos releases, deixando fora da analise longitudinal)

 Kiasan/Bogor             (mudou a forma de distribuir ao longo dos releases, dificil obter de forma consistente as versoes)

 Lotrack                  (sem releases, poucos commits no github, apenas 11)

 PMD                      13 releases

 PtYasm                   (nao tem releases disponivel, apenas a ultima versao)

 Splint                   (nao encontrado releases)

 srcML                    (releases nao encontrado)

 WALA                     13 releases

 error-prone              13 releases

 GumTree                  (tem apenas 2 releases do repositorio github)

 JastAdd                  13 releases

 Sonar Qube Plug-in       (apenas 4 releases no github)

 SPARTA                   13 releases

 Cppcheck                 13 releases

 FindSecurityBugs         13 releases
 
 Smatch                   13 releases

 WAP                      (apenas 7 releases no site, estou selecionando os que tenham ao menos 13 releases)

Comparacao entre ferramentas de tamanho similar:

nas 5 comparações de versões distintas com tamanhos similares entre pmd e findbugs,
apresentaram o mesmo resultado, pmd tem valores menos tanto para CC quanto para SC,
indicando que pmd tem um design mais modular que findbugs.

pmd 5.0.0 < findbugs 1.2.1
pmd 5.0.4 < findbugs 1.2.1
pmd 5.1.3 < findbugs 1.3.5
pmd 5.2.0 < findbugs 1.3.8
pmd 5.3.3 < findbugs 1.3.9

Ao comparar as imagens da matrix DSM dá para notar que isto reflete na matrix, 
pegando o findbugs 3.0.1 e o pmd 5.2.0, é possível notar na matrix que o findbugs
tem mais pontos nas duas diagonais da matrix, indicando dependencias ciclicar, e
design menos modular, enquanto o pmd concentra as dependencias na diagonal inferior
esquerda, indicando poucas dependencias ciclicar e um design mais modular.

findbugs	 findbugs-3.0.1	1486	0,07	8	25	56
pmd	 pmd-src-5.2.0	1295	0,02	6	12	25

/home/joenio/src/dissertacao-ufba-2016/dataset/static-analysis-tools/pmd/pmd-src-5.2.0.analizo.dsm.png
/home/joenio/src/dissertacao-ufba-2016/dataset/static-analysis-tools/findbugs/findbugs-3.0.1.analizo.dsm.png

accessanalysis < findsecuritybugs
indus < bogor
reassert > jflow
pmd-5.4.0 > pmd-5.3.0
pmd-5.4.2 > pmd-5.3.7
findbugs-3.0.1 > findbugs-2.0.3
pmd < closure-compiler
pixy > mpanalyzer
ejb > mpanalyzer
ejb > sparta

closure-compiler > wala
closure-compiler > wala
closure-compiler > wala
closure-compiler[Java] ? wala[Java]     (closure tem CC maior mas SC menor)

comparar linguagens diferentes não rola, sempre dá ruim, ver:

rats[C]        ? uno[C]                 (rats tem CC menor e SC maior)
cppcheck[C++]  ? wap[Java]              (cppcheck tem CC menor mas SC maior)
srcml[C++]     ? ptyasm[Java]           (srcml tem CC maior mas SC menor)
closure-compiler[Java] ? inputtracer[C] (closure tem CC menor mas SC diferentes nos percentis)
pmd[Java] ? srcml[C++]                  (pmd tem CC maior e SC diferentes nos percentis)
inputtracer[C] ? wala[Java]             (tem CC maior e SC diferentes mas no percentil 95 tem SC maior também)

fica claro que comparacao entre linguagens diferentes mesmo com tamanhos iguais não dá para chegar a conclusões nenhuma.

findbugs[Java] ? wala[Java]             (findbugs tem CC maior mas SC menor)
closure-compiler ? closure-compiler     (CC menor mas SC maior)

comparacao (v3) - ordenado por eloc - comparando apenas SC 95
===============

% findsecbugs-plugin-1.4.0-sources < sparta-code-0.6
% findsecbugs-plugin-1.4.1-sources < sparta-code-0.7
% findsecbugs-plugin-1.4.4-sources < sparta-code-0.8

% cseq-0.5 > find-sec-bugs-version-1.0.0

% sparta-code-0.9.2 < tacle_1_2_1_src
% sparta-code-0.9.4 < jastadd2-src-2.1.5
% MPAnalyzer-master > sparta-toolset-0.9.8
% ReAssert_0.4.1 > sparta-sparta-1.0.2
% sparta-sparta-1.0.2 < uno
% sparta-toolset-1.0.1-source < cppcheck-1.30
% SonarQube-plug-in-master > sparta-toolset-1.0.0-source

% findsecbugs-plugin-1.4.5-sources < rats-2.4
% jastadd2-src-2.1.2 > findsecbugs-plugin-1.4.6-sources
% find-sec-bugs-version-1.1.0 < jastadd2-src-2.1.4
% AccessAnalysis-1.2-src < jastadd2-src-2.1.9
% jastadd2-src-2.1.13 > jlint-3.1.2
% vazexqi-JFlow-7cd7eaf < gumtree-2.0.0
% composite-0.4 < smatch-1.0
% smatch-0.3 > EJB
% smatch-0.4 > cppcheck-1.35
% cppcheck-1.35 > guizmo-master
% smatch-1.51 < cqual-0.981
% pixy-master < smatch-1.52
% error-prone-2.0 < smatch-1.54
% smatch-1.54 > cppcheck-1.40
% smatch-1.55 > error-prone-2.0.2
% indus < smatch-1.56
% smatch-1.56 > error-prone-2.0.4
% smatch-1.59 > pmd-src-5.0.4
% error-prone-2.0.6 < pmd-src-4.2.5
% pmd-src-4.2.5 < smatch-1.60
% smatch-1.60 < cppcheck-1.45
% ptyasm > error-prone-2.0.8
% error-prone-2.0.8 < bogor-core
% pmd-src-5.1.0 > error-prone-2.0.9
% pmd-src-5.3.7 < wap-2.1
% error-prone-2.0.12 < pmd-src-5.5.2
% error-prone-2.0.13 < cppcheck-1.50
% cppcheck-1.50 > wala-code-4607-tags-R_1.0
% findbugs-1.2.1-source > error-prone-2.0.14
% cppcheck-1.55 > findbugs-1.3.4-source
% findbugs-1.3.8-source < cppcheck-1.60
% findbugs-1.3.9-source = wala-code-4607-tags-R_1.0.02
% wala-code-4607-tags-R_1.2 < cppcheck-1.62
% findbugs-2.0.2-source > wala-code-4607-tags-R_1.1
% cppcheck-1.65 > wala-code-4607-tags-R_1.2.2
% wala-code-4607-tags-R_1.3 < findbugs-3.0.0-source
% findbugs-3.0.1 > WALA-R_1.3.3
% WALA-R_1.3.3 < cppcheck-1.70
% cppcheck-1.75 > WALA-R_1.3.5
% WALA-R_1.3.6 > closure-compiler-20110119
% closure-compiler-20110119 < cppcheck-1.77
% cppcheck-1.77 < splint-3.1.2
% srcML-src > closure-compiler-20140730
% closure-compiler-20160713 > Lotrack-master

O percentil 75 tem muitos valores zero, os percentis 90 e 95 sao pracitamente iguais 
na comparacao, os maiores sao geralmente tb maior no outro, exceto uns 2 exemplos:
smatch-0.3/EJB e pmd-src-5.3.7/wap-2.1.




comparacao (v3) - ordenado por n modulos - comparando apenas SC 90 e 95
===============



rats-2.4 > uno
jlint-3.1.2 > findsecbugs-1.2.0
jastadd2-2.1.5 > findsecbugs-1.2.1
findsecbugs-1.3.0 < jastadd2-2.1.8
jastadd2-2.2.2 > findsecbugs-1.4.0
findsecbugs-1.4.2 < cqual-0.981
sparta-0.5 < findsecbugs-1.4.4
findsecbugs-1.4.5 > AccessAnalysis-1.2
AccessAnalysis-1.2 < cppcheck-1.30
cppcheck-1.30 < smatch-1.0
smatch-0.2 > findsecbugs-1.4.6
findsecbugs-1.4.6 > sparta-0.6
sparta-0.7 < smatch-0.3
cseq-0.5 > findsecbugs-1.5.0
smatch-0.4 > cppcheck-1.35
cppcheck-1.35 > sparta-0.8
cppcheck-1.40 > sparta-0.9.2
sparta-0.9.2 > findsecbugs-1.0.0
SonarQube-plug-in-master < smatch-1.51
smatch-1.52 > ReAssert\_0.4.1
smatch-1.53 > jfLow
gumtree-2.0.0 < cppcheck-1.45
smatch-1.54 > sparta-0.9.8
sparta-0.9.8 < pixy
cppcheck-1.50 > findsecbugs-1.1.0
findsecbugs-1.1.0 < MPAnalyzer
MPAnalyzer < EJB
sparta-1.0.1 < cppcheck-1.55
guizmo < cppcheck-1.60
smatch-1.56 < cppcheck-1.70
wap-2.1 < cppcheck-1.72
cppcheck-1.75 > smatch-1.58
pmd-4.3 < srcML
srcML < ptyasm
pmd-5.0.0 < findbugs-1.2.1
pmd-5.0.4 > error-prone-2.0
closure-compiler-20110119 > error-prone-2.0.2
error-prone-2.0.4 < findbugs-1.3.4
findbugs-1.3.4 < wala-4607-R1.0
wala-4607-R1.0 > pmd-5.1.0
pmd-5.1.3 < findbugs-1.3.5
findbugs-1.3.8 > pmd-5.2.0
pmd-5.3.3 < findbugs-1.3.9
findbugs-1.3.9 > pmd-5.4.2
pmd-5.3.7 < findbugs-3.0.0
findbugs-2.0.3 > error-prone-2.0.5
pmd-5.5.2 > error-prone-2.0.6
closure-compiler-20140730 > error-prone-2.0.7
error-prone-2.0.8 < closure-compiler-20150729
error-prone-2.0.9 < wala-4607-R1.1.2
wala-4607-R1.1.2 < closure-compiler-20160125
closure-compiler-20110811 < wala-4607-R1.2
error-prone-2.0.11 < closure-compiler-20160517
closure-compiler-20160713 > error-prone-2.0.12
error-prone-2.0.12 < wala-4607-R1.1
wala-4607-R1.2.2 > error-prone-2.0.13
error-prone-2.0.14 < wala-4607-R1.3
error-prone-2.0.15 < closure-compiler-20140110

%% De forma que somando as ferramentas selecionadas na academia e na indústria
%% temos um total de 34 ferramentas, 14 da indústria e 20 da academia.  
%% 
%% \begin{table}[H]
%%   \caption{Resumo da caracterização das ferramentas}
%%   \centering
%%   \begin{tabular}{| c | l | l | c | l | l |}
%%     \hline
%%     \# & Ferramentas da indústria & Linguagem & Classes & Lançamentos \\
%%     \hline
%%     22 & Closure Compiler         & Java  & 1842  & Frequentemente \\
%%     23 & Cppcheck                 & C++   & 338   & Frequentemente \\
%%     24 & CQual                    & C     & 78    & Obsoleta       \\
%%     25 & FindBugs                 & Java  & 1486  & Ocasionalmente \\
%%     26 & FindSecurityBugs         & Java  & 91    & Frequentemente \\
%%     27 & Jlint                    & C++   & 44    & Obsoleta       \\
%%     28 & Pixy                     & Java  & 229   & Obsoleta       \\
%%     29 & PMD                      & Java  & 1340  & Frequentemente \\
%%     30 & RATS                     & C     & 19    & Obsoleta       \\
%%     31 & Smatch                   & C     & 483   & Ocasionalmente \\
%%     32 & Splint                   & C     & 681   & Obsoleta       \\
%%     33 & UNO                      & C     & 19    & Obsoleta       \\
%%     34 & WAP                      & Java  & 338   & Frequentemente \\
%%     \hline
%%   \end{tabular}
%%   \label{total-de-ferramentas}
%% \end{table}

% \subsection{AccessAnalysis}
% 
% O código-fonte
% utilizado em nosso estudo obtido no site da ferramenta foi o
% \texttt{AccessAnalysis-1.2-src.zip}.
% 
% \subsection{Kiasan/Bogor}
% 
% O código-fonte utilizado em
% nosso estudo obtido no site da ferramenta foi o
% \texttt{bogor-src-1.2.20061023.1.zip}.
% 
% Não possui número suficiente de releases para ser usado na análise evolutiva.
% 
% \subsection{composite}
% 
% O código-fonte utilizado em
% nosso estudo obtido no site da ferramenta foi o \texttt{composite-0.4.tar.gz}.
% 
% \subsection{CSeq}
% 
% \O código-fonte
% utilizado em nosso estudo obtido no site da ferramenta foi o
% \texttt{cseq-0.5.zip}.
% 
% \subsection{EJB}
% 
% \subsection{error-prone}
% 
% O código-fonte utilizado em nosso
% estudo obtido no site da ferramenta foi o \texttt{error-prone-2.0.9.tar.gz}.
% 
% \subsection{GUIZMO}
% 
% O código-fonte
% utilizado em nosso estudo obtido no site da ferramenta foi o
% \texttt{guizmo-master.zip}. Aceita como entrada um formato baseado em
% XML\footnote{\url{http://wireframesketcher.com/help/xmlformat.html}} e gera
% código GUI em Java Swing / ZK.
% 
% \subsection{GumTree}
% 
% código-fonte utilizado em nosso estudo obtido no site da ferramenta foi o
% \texttt{gumtree-2.0.0.tar.gz}.
% 
% \subsection{Indus}
% 
% O projeto está organizado em três
% módulos, os seguintes arquivos, contendo o código-fonte dos três módulos,
% foram copiados localmente para análise:
% \texttt{indus.indus-src-20091220.zip},
% \texttt{indus.javaslicer-src-20091220.zip} e
% \texttt{indus.staticanalyses-src-20070305.zip}.
% 
% Não possui número suficiente de releases para ser usado na análise evolutiva.
% 
% \subsection{JastAdd}
% 
% O código-fonte
% utilizado em nosso estudo obtido no site da ferramenta foi o
% \texttt{jastadd2-src.zip}.
% 
% \subsection{JFlow}
% 
% O código-fonte
% utilizado em nosso estudo obtido no site da ferramenta foi o
% \texttt{vazexqi-JFlow-7cd7eaf.tar.gz}.
% 
% \subsection{Lotrack}
% 
% O código-fonte utilizado em nosso
% estudo obtido no site da ferramenta foi o \texttt{Lotrack-master.zip}.
% 
% Não possui número suficiente de releases para ser usado na análise evolutiva.
% 
% \subsection{MPAnalyzer}
% 
% O código-fonte utilizado em
% nosso estudo obtido no site da ferramenta foi o \texttt{MPAnalyzer-master.zip}.
% 
% \subsection{PtYasm}
% 
% O código-fonte
% utilizado em nosso estudo obtido no site da ferramenta foi o
% \texttt{ptyasm.april2008.tgz}.
% 
% Não possui número suficiente de releases para ser usado na análise evolutiva.
% 
% \subsection{ReAssert}
% 
% O código-fonte utilizado em nosso
% estudo obtido no site da ferramenta foi o \texttt{ReAssert\_0.4.1-src.zip}.
% 
% \subsection{Sonar Qube Plug-in}
% 
% O código-fonte
% utilizado em nosso estudo obtido no site da ferramenta foi o
% \texttt{SonarQube-plug-in-master.zip}.
% 
% \subsection{SPARTA}
% 
% O código-fonte utilizado em nosso
% estudo obtido no site da ferramenta foi o \texttt{sparta-sparta-1.0.2.tar.gz}.
% 
% \subsection{srcML}
% 
% \url{http://www.sdml.info/projects/srcml/trunk}\footnote{este endereço
% retornou "not found" em contato com os autores por email indicaram que o
% projeto foi movido para http://www.srcML.org}. O código-fonte utilizado em
% nosso estudo obtido no site da ferramenta foi o \texttt{srcML-src.tar.gz}.
% 
% Não possui número suficiente de releases para ser usado na análise evolutiva.
% 
% \subsection{TACLE}
% 
% disponível em
% \url{http://presto.cse.ohio-state.edu/tacle}\footnote{este link está
% indisponível, por email os autores indicaram o endereço
% http://web.cse.ohio-state.edu/~rountev/presto/tacle/TACLE\_Download/tacle.html}.
% O código-fonte utilizado em nosso estudo obtido no site da ferramenta foi o
% \texttt{tacle\_1\_2\_1\_src.zip}.
% 
% \subsection{WALA}
% 
% O código-fonte
% utilizado em nosso estudo obtido no site da ferramenta foi o
% \texttt{WALA-R\_1.3.8.tar.gz}.
% 
% Ferramenta selecionada para análise evolutiva, possui muitos releases e tem tamanho
% em número de classes na média.


% \subsection{Closure Compiler}
% 
% Compilador que traduz código JavaScript em outro
% JavaScript melhor e mais otimizado, está disponível em
% \url{https://developers.google.com/closure/compiler}\footnote{O código fonte do
% Closure Compiler pode ser obtido em:
% http://github.com/google/closure-compiler} e foi utilizado em nosso estudo o
% seguinte lançamento
% \texttt{closure-compiler-closure-compiler-parent-v20160619.tar.gz}.
% 
% Ferramenta selecionada para análise evolutiva, possui muitos releases e tem tamanho
% em número de classes na média.
% 
% \subsection{Cppcheck}
% 
% Ferramenta de análise estática de código C/C++ para checagem de vazamento de
% memória, erros de alocação, entre outras falhas. Disponível em
% \url{http://sourceforge.net/projects/cppcheck}. Em nosso estudo utilizamos o
% código em \texttt{cppcheck-1.72.tar.bz2}.
% 
% \subsection{CQual}
% 
% Ferramenta de análise de typo ({\it type-based analysis}) que fornece um
% mecanismo leve e prático para especificação e verificação de propriedades de
% programas C. Disponível em \url{http://www.cs.umd.edu/~jfoster/cqual}. Em
% nosso estudo utilizamos o código em \texttt{cqual-0.981.tar.gz}.
% 
% \subsection{FindBugs}
% 
% Uma ferramenta para localização de bugs em código Java disponível em
% \url{http://findbugs.sourceforge.net}. Em nosso estudo utilizamos o código em
% \texttt{findbugs-3.0.1-source.zip}.
% 
% Ferramenta selecionada para análise evolutiva, possui muitos releases e tem tamanho
% em número de classes na média.
% 
% \subsection{FindSecurityBugs}
% 
% Plugin do FindBugs para auditoria de segurança em aplicações web Java,
% disponível em \url{http://find-sec-bugs.github.io}. O código-fonte utilizado
% em nosso estudo obtido no site da ferramenta foi o
% \texttt{findsecbugs-plugin-1.4.5-sources.jar}.
% 
% \subsection{Jlint}
% 
% Uma ferramenta para verificaçao de código Java em busca de bugs,
% inconsistências e problemas de sincronização disponível em
% \url{http://sourceforge.net/projects/jlint}.  O código-fonte utilizado em
% nosso estudo obtido no site da ferramenta foi o \texttt{jlint-3.1.2.zip}.
% 
% \subsection{Pixy}
% 
% Ferramenta de análise estática de código PHP para verificação de
% vulnerabilidades de segurança. Disponível em
% \url{http://github.com/oliverklee/pixy}. O código-fonte utilizado em nosso
% estudo obtido no site da ferramenta foi o \texttt{pixy-master.zip}.
% 
% \subsection{PMD}
% 
% Ferramenta de análise de código-fonte para localização falhas comuns de
% programação com suporte a várias linguagens, disponível em
% \url{http://pmd.github.io}. O código-fonte utilizado em nosso estudo obtido
% no site da ferramenta foi o \texttt{pmd-src-5.4.1.zip}.
% 
% Ferramenta selecionada para análise evolutiva, possui muitos releases e tem tamanho
% em número de classes na média.
% 
% \subsection{RATS}
% 
% Ferramenta de análise estática para auditoria de segurança 
% de códigos C, C++, Perl, PHP e Python disponível em
% \url{http://code.google.com/archive/p/rough-auditing-tool-for-security}. O
% código-fonte utilizado em nosso estudo obtido no site da ferramenta foi o
% \texttt{rats-2.4.tgz}.
% 
% \subsection{Smatch}
% 
% Ferramenta de análise estática para detecção de erros no Kernel disponível em
% \url{http://smatch.sourceforge.net}. O código-fonte utilizado em nosso estudo
% obtido no site da ferramenta foi o \texttt{smatch.git}.
% 
% \subsection{Splint}
% 
% Ferramenta para verificação de programas em C por vulnerabilidades de segurança e
% erros de código. Disponível em \url{http://www.splint.org}. O código-fonte
% utilizado em nosso estudo obtido no site da ferramenta foi o
% \texttt{splint-3.1.2.src.tgz}.
% 
% Não possui número suficiente de releases para ser usado na análise evolutiva.
% 
% \subsection{UNO}
% 
% Uma ferramenta de análise de código-fonte C para detecção de defeitos.
% Disponível em \url{http://spinroot.com/uno}. O código-fonte utilizado em nosso
% estudo obtido no site da ferramenta foi o \texttt{uno\_v213.tar.gz}.
% 
% \subsection{WAP}
% 
% Ferramenta para análise estática de código-fonte PHP e mineraçao de dados para
% detectar e corrigir vulnerabilidades em aplicações web. Disponível em
% \url{http://awap.sourceforge.net}. O código-fonte utilizado em nosso estudo
% obtido no site da ferramenta foi o \texttt{wap-2.1.tar.gz}.
