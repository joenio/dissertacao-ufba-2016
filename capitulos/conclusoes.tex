\xchapter{Conclusões}{} \label{conclusoes}

\section{Resultados preliminares}

\subsection{Revisão estruturada}

Realizamos uma busca por ferramentas na conferência SCAM, onde 315 artigos
foram avaliados em duas etapas referentes as duas atividades da revisão
estruturada: (1) busca automatizada a partir do script de busca e (2) seleção
a partir da leitura do artigo. A primeira etapa resultou em 133 artigos,
destes apenas 9 foram selecionados na segunda etapa, resultando daí 9
ferramentas de análise estática com disponibilidade de código-fonte.
A Tabela \ref{artigos-do-scam} apresenta um resumo desta revisão estruturada.

\begin{table}[H]
\caption{Total de artigos analisados em cada edição do SCAM}
\centering
\begin{tabular}{| l | c | c | c |}
\hline
Edição    & Total de artigos & (1) Busca & (2) Seleção \\
\hline
SCAM 2001 & 23    & 6         & -           \\
SCAM 2002 & 18    & 6         & -           \\
SCAM 2003 & 21    & 8         & -           \\
SCAM 2004 & 17    & 3         & -           \\
SCAM 2005 & 19    & 7         & -           \\
SCAM 2006 & 22    & 10        & 2           \\
SCAM 2007 & 23    & 7         & 1           \\
SCAM 2008 & 29    & 14        & -           \\
SCAM 2009 & 20    & 10        & -           \\
SCAM 2010 & 21    & 15        & 1           \\
SCAM 2011 & 21    & 10        & 2           \\
SCAM 2012 & 22    & 12        & 3           \\
SCAM 2013 & 24    & 13        & -           \\
SCAM 2014 & 36    & 16        & 1           \\
SCAM 2015 & 30    & 18        & 0           \\
\hline
Total     & 346   & 155       & 10          \\
\hline
\end{tabular}
\label{artigos-do-scam}
\end{table}

\subsection{Seleção e caracterização das ferramentas}

Junto à revisão estruturada para seleção de ferramentas da academia, fizemos
uma seleção manual no catálogo de ferramentas de análise estática do projeto
SAMATE\footnote{https://samate.nist.gov/index.php/Source\_Code\_Security\_Analyzers.html}
para encontrar ferramentas da indústria, nesta seleção os critérios foram,
ferramentas com código-fonte disponível, implementadas nas linguagens de
programação suportadas pelo Analizo. Nesta seleção, de um total de 54 ferramentas
presentes no catálogo do projeto SAMATE, 19 tinham código-fonte
disponível, destas apenas 14 eram suportadas pelo Analizo.

De forma que somando as ferramentas selecionadas na academia e na indústria
temos um total de 23 ferramentas, 14 da indústria e 9 da academia.  A
Tabela \ref{total-de-ferramentas} resume este total trazendo o nome de cada
ferramenta e a linguagem de programação em que é escrita.

\begin{table}[H]
  \caption{Ferramentas selecionadas e em qual linguagem é escrita}
  \centering
  \begin{tabular}{| c | l | l | c | l | l |}
    \hline
    \# & Ferramentas da indústria & Linguagem & n  & Ferramentas da academia & Linguagem  \\
    \hline
    1  & Clang Static Analyzer    & C++       & 15 & AccessAnalysis          & Java       \\
    2  & Closure Compiler         & Java      & 16 & EJB                     & Java       \\
    3  & Cppcheck                 & C++       & 17 & error-prone             & Java       \\
    4  & CQual                    & C         & 18 & Indus                   & Java       \\
    5  & FindBugs                 & Java      & 19 & InputTracer             & C          \\
    6  & FindSecurityBugs         & Java      & 20 & JastAdd                 & Java       \\
    7  & Jlint                    & C++       & 21 & Sonar Qube Plug-in      & Java       \\
    8  & Pixy                     & Java      & 22 & srcML                   & C++        \\
    9  & PMD                      & Java      & 23 & TACLE                   & Java       \\
    10 & RATS                     & C         & 24 & WALA                    & Java       \\
    11 & Smatch                   & C         &    &                         &            \\
    12 & Splint                   & C         &    &                         &            \\
    13 & UNO                      & C         &    &                         &            \\
    14 & WAP                      & Java      &    &                         &            \\
    \hline
  \end{tabular}
  \label{total-de-ferramentas}
\end{table}

Para cada uma dessas 24 ferramentas selecionadas foi realizado uma caracterização
inicial em relaçao à categoria {\it Lançamentos ({\it Releases}) - quantos lançamentos
por ano}, identificando se é:

\begin{itemize}
  \item Frequentemente $>=$ 3 vezes ao ano - novas versões da ferramenta são lançadas 3 ou mais vezes por ano
  \item Ocasionalmente $<$ 3 vezes ao ano - novas versões da ferramenta são lançadas menos que 3 vezes ao ano
  \item Obsoleta 0 vezes ao ano - intervalo entre novos lançamentos é maior que 1 ano
\end{itemize}

Os detalhes desta caracterização são apresentados no Apêndice \ref{caracterizacao-ferramentas}.

\subsubsection{Comparacao Analizo vs SciTools Understand}

Os valores de métricas foram também calculados com a ferramenta Understand
versão ``4.0.853'' em Linux 64 bits.

\subsection{Análise exploratória dos valores das métricas}

Numa análise preliminar dos valores das métricas nos percentis 75, 90 e 95,
apresentados no Apêndice \ref{analise-metricas}, percebemos uma forte
similaridade com o trabalho em que estamos nos apoiando \cite{Meirelles2013}
em relação aos intervalos dos valores de métricas.

A Tabela \ref{metricas-90} faz um resumo da distribuição das métricas para todas
as ferramentas estudadas neste trabalho no
percentil 90, cada linha
representa uma ferramenta, os nomes foram suprimidos para ganhar
espaço.

%% begin.rcode metricas-90, fig.align='center', results="asis"
% percentil = '90%'
% table = data.frame(t(percentil_all_projects(percentil)), t(percentil_all_nist_projects(percentil)))
% xt = xtable(t(table), caption='valores das métricas no percentil 90 para as ferramentas da academia e indústria', digits=0, label='metricas-90')
% print(xt, table.placement="H", include.rownames=FALSE)
%% end.rcode

Podemos perceber, por exemplo, que para a métrica \texttt{accm}, no percentil 90,
as ferramentas se encontram entre os valores 2 à 9, o que coloca a maioria
delas nos intervalos sugeridos por \cite{Meirelles2013} para
códigos C, C++ e Java, ou seja, 3,1 a 5,3, 2,1 a 4,0 e 2,9 a 4,4,
respectivamente.

A mesma similaridade é percebida para a métrica \texttt{anpm}, onde as
ferramentas se encontram entre os valores de 1 a 3 e os intervalos sugeridos por
\cite{Meirelles2013} para códigos C, C++ e Java estão entre 3,1 a 4,0, 2,1 a
3,0 e 1,6 a 2,0. De forma que a maioria das ferramentas se encontram dentro
dos intervalos sugeridos por \cite{Meirelles2013}.

Já para a métrica de acomplamento \texttt{cbo} esta similaridade não se repete
, nossas ferramentas variam de 7 a 986, e os intervalos propostos por
\cite{Meirelles2013} estão entre 6 a 9, 4 a 5 e 4 a 6, para C, C++ e Java
respectivamente. Percebemos claramente que \texttt{cbo} tem um comportamento
distinto, o que nos leva a considerar que iremos precisar esta métrica com
mais cuidado, e possivelmente realizar análises mais aprofundadas com ela.

Isto claramente faz um certo gancho com o objetivo geral deste trabalho,
caractarizar a complexidade estrutural de ferramentas de análise estática,
pois uma das formas de calcular complexidade estrutural, e que possivelmente
iremos utilizar aqui, é feita em relação à \texttt{cbo} e \texttt{lcom4}. Na
Figura \ref{fig:grafico-comparativo} é fácil perceber o quanto \texttt{cbo} destoa
das outras métricas.

%% begin.rcode grafico-comparativo, fig.show='hold', fig.cap="distribuição dos percentis para as ferramentas da academia e da indústria"
% par(mfrow=c(3,2), oma=c(0,0,0,0))
%
% table = data.frame(percentis_by_project("accm"), percentis_by_nist_project("accm"))
% matplot(table, type="l", pch=1, xlab="percentis", ylab="valor", xaxt="n", cex.lab=0.6, cex.axis=0.6, cex.sub=0.6, cex.main=0.6)
% axis(1, at=1:length(rownames(table)), labels=rownames(table), cex.axis=0.6)
% title(main="accm")
%
% table = data.frame(percentis_by_project("nom"), percentis_by_nist_project("nom"))
% matplot(table, type="l", pch=1, xlab="percentis", ylab="valor", xaxt="n", cex.lab=0.6, cex.axis=0.6, cex.sub=0.6, cex.main=0.6)
% axis(1, at=1:length(rownames(table)), labels=rownames(table), cex.axis=0.6)
% title(main="nom")
%
% table = data.frame(percentis_by_project("lcom4"), percentis_by_nist_project("lcom4"))
% matplot(table, type="l", pch=1, xlab="percentis", ylab="valor", xaxt="n", cex.lab=0.6, cex.axis=0.6, cex.sub=0.6, cex.main=0.6)
% axis(1, at=1:length(rownames(table)), labels=rownames(table), cex.axis=0.6)
% title(main="lcom4")
%
% table = data.frame(percentis_by_project("cbo"), percentis_by_nist_project("cbo"))
% matplot(table, type="l", pch=1, xlab="percentis", ylab="valor", xaxt="n", cex.lab=0.6, cex.axis=0.6, cex.sub=0.6, cex.main=0.6)
% axis(1, at=1:length(rownames(table)), labels=rownames(table), cex.axis=0.6)
% title(main="cbo")
%
% table = data.frame(percentis_by_project("acc"), percentis_by_nist_project("acc"))
% matplot(table, type="l", pch=1, xlab="percentis", ylab="valor", xaxt="n", cex.lab=0.6, cex.axis=0.6, cex.sub=0.6, cex.main=0.6)
% axis(1, at=1:length(rownames(table)), labels=rownames(table), cex.axis=0.6)
% title(main="acc")
%
% table = data.frame(percentis_by_project("sc"), percentis_by_nist_project("sc"))
% matplot(table, type="l", pch=1, xlab="percentis", ylab="valor", xaxt="n", cex.lab=0.6, cex.axis=0.6, cex.sub=0.6, cex.main=0.6)
% axis(1, at=1:length(rownames(table)), labels=rownames(table), cex.axis=0.6)
% title(main="sc")
%% end.rcode

\section{Próximos passos}

Os próximos passos estão concentrados no estudo dos valores de distribuição
das métricas nos percentis 75, 90 e 95, comparar os valores e comportamentos
encontrados em nossos dados com os trabalhos de referência sobre valores de
métricas de código-fonte e sua discussão sobre a qualidade interna.

Após esta análise inicial iremos adicionar mais ferramentas a partir de uma
nova revisão estruturada, e então atualizar os valores e discussões de forma a
refletir a inclusão das novas ferramentas selecionadas neste ciclo de revisão.

A partir daí teremos definição dos valores de referência para ferramentas de
análise estática de código-fonte, com isto iremos calcular quanto cada
ferramenta está distante dos valores de referência, isto dará indícios sobre a
qualidade interna de cada ferramenta.
