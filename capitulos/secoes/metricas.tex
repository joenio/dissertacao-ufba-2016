\section{Métricas}

\subsection{Métricas de reconhecimento e visibilidade}

... (voltar aqui após escrever metodologia) ...

%o artigo abaixo faz um estudo e usa metrica para calcular o impacto
%das citacoes e mencoes ao software, usa um calculo baseado no numero
%de ocorrencias que o nome do software aparece
%Disciplinary differences of software use and impact in scientific literature
% eu fiz a mesma coisa mas numa avaliação qualitativa, meu peso é, 0 ou 1, menciona o software ou não menciona'
%se menciona, qual tipo, usa, contribui, etc... isto eh que vai dar o valor/peso/metrica

\subsection{Métricas de sustentabilidade}

... (voltar aqui após escrever metodologia) ...

\subsection{Métricas de manutenabilidade}

\subsection{Complexidade de software}


%% Métricas de software podem ser classificadas em métricas de processo, métricas
%% de projeto e métricas de produto.
%% 
%% Métricas de processo medem atributos relacionados ao ciclo de desenvolvimento
%% e manutenção de software. Métricas de projeto indicam se a execução do
%% processo está progredindo conforme planejado (por exemplo, relação entre o
%% tamanho do software entregue e o esforço total dispendido em seu
%% desenvolvimento).
%% 
%% Métricas de produto medem atributos de produtos e artefatos, como documentos,
%% diagramas, código fonte e arquivos binários. Neste trabalho,
%% apenas métricas de produto serão utilizadas.
%% 
%% Métricas de produto podem ser classificadas em internas (medem propriedades
%% visíveis apenas aos desenvolvedores) ou externas (medem propriedades visíveis
%% aos usuários) \cite{Mohamed1994}.
%% 
%% Neste trabalho, são utilizadas métricas de produto e, especificamente,
%% métricas de código fonte, que cobrem aspectos de tamanho, complexidade e
%% qualidade que podem ser medidos a partir do código fonte de um software.
%% 
%% Métricas de software tem um escopo bastante abrangente, e o termo está
%% associado com muitas atividades da engenharia de software: Medidade e modelos
%% para estimativa de custo e esforço, Coleção de dados, Modelos e medidas de
%% qualidade, Modelos de confiabilidade, Métricas de segurança, Métricas
%% estruturais e de complexidade, Avaliação de maturidade de capacidade,
%% Gerenciamento através de métricas, Avaliação de métodos e ferramentas.
%% 
\subsection{Métricas de código fonte} \label{metricas-de-codigo}

As propriedades visíveis aos desenvolvedores podem ser medidas através de
métricas de código fonte. A observação e o monitoramento de seus valores podem
indicar aspectos relevantes à manutenibilidade de um programa. Dentre as
inúmeras métricas de código fonte nosso interesse está em métricas que indicam
características relevantes à modularidade de um produto de software,
complexidade estrutural e custo de mudança.

%Structural and Complexity Metrics
%Desirable quality attributes like reliability and maintainability cannot be
%measured until some operational version of the code is available. Yet, we
%wish to be able to predict which parts of the software system are likely to be
%less reliable, more difficult to test, or require more maintenance than oth-
%ers, even before the system is complete. As a result, we measure structural
%attributes of representations of the software that are available in advance
%of (or without the need for) execution; then, we try to establish empiri-
%cally predictive theories to support quality assurance, quality control, and
%quality prediction. These representations include control flow graphs that
%usually model code and various unified modeling language (UML) dia-
%grams that model software designs and requirements. Structural metrics
%can involve the arrangement of program modules, for example, the use
%and properties of design patterns. These models and related metrics are
%described in Chapter 9.

Do ponto de vista de métricas, neste trabalho, estamos interessados, de fato, na métrica
de complexidade estrutural SC ({\it Structural Complexity}), uma medida da
complexidade de projetos de sistema de software proposta por
\citeonline{Darcy2005} como indicador da complexidade dos sistemas de software
em relação à sua estrutura interna e ao relacionamento entre os seus
componentes.

Complexidade é um tema bastante amplo, e para compreender onde os
sistemas de softwares se encaixam neste contexto precisamos definir brevemente
o que vem a ser sistemas complexos.

Sistemas complexos são sistemas no qual grandes redes de componentes sem
controle central dão origem a um comportamento
coletivo complexo, com processamento sofisticado de informação, e adaptação
através de aprendizado ou evolução \cite{Mitchell2009}. As seguintes
características são comuns a todos os sistemas complexos:

\begin{description}

  \item[Comportamento coletivo complexo.] Apesar de serem compostos por
  elementos bastante simples individualmente, sistemas complexos podem exibir
  comportamentos coletivos bastante sofisticados.

  \item[Troca de sinais e processamento de informação.] Em cada um destes
  sistemas, seus componentes individuais consomem e produzem informação entre
  si. Parte do comportamento do sistema envolve transformação dessa informação.

  \item[Adaptação.] Sistemas complexos adaptam-se a novas situações de forma a
  aumentar suas chances de sobrevivência diante de novas condições em seu
  ambiente.

\end{description}

Os sistemas complexos podem ser classificados como naturais ou artificiais, os
sistemas naturais são aqueles cuja constituição não tem participação humana, ao
contrário dos sistemas artificiais que são projetados por humanos, com
objetivos e funções previamente definidos \cite{Simon1996}. Neste sentido,
sistemas de software podem ser caracterizados como sistemas complexos
artificiais, pois exibem comportamento coletivo complexo, realizam trocas de
sinais e processamento de informação e passam por adaptação para se adequar a
mudanças em seu ambiente.

Sistemas de software são compostos por componentes, em geral, chamados de
módulos, que possuem tanto estado quanto comportamento próprios,
módulos individuais de um sistema de software são simples quando comparados com
o sistema como um todo. Módulos produzem informação para outros módulos
através de parâmetros em chamadas de sub-rotinas e consomem informação através
dos valores de retornos destas chamadas. O fluxo contínuo de novos requisitos e
de mudanças no ambiente operacional de sistemas de software força-os a se
manter em constante evolução em busca de “sobrevivência”.

Esta medida é, possivelmente, um indicativo de problemas na manutenibilidade de
sistemas de software, em especial sobre o esforço necessário para atividades de
manutenção \cite{Terceiro2012}. Ela está relacionada a como os módulos de um
programa estão organizados bem como à estrutura interna de cada módulo. Esta
métrica pode dar indícios importantes sobre características arquiteturais de um
programa de software e pode explicar seus atributos de qualidade interna.

%Modularity
%Modularity describes the logical partitioning of software into several parts, components, and modules.
%Software will be easy to understand and change when composed of independent modules.
%“A Software Maintainability Evaluation Methodology”, 1981
%\cite{kumar2012survey}

%Faz um experimento usando CBO LCOM e outras metricas como preditor de manutenabilidade...
%\cite{Dagpinar2003}

%A complexidade de um sistema de software pode ser de três tipos: a complexidade
%do problema, a complexidade procedural, e a complexidade do projeto do sistema.
%Esta última é o foco deste trabalho.
%A complexidade do problema está relacionada ao domínio do problema.
%A complexidade procedural está relacionada à estrutura lógica da programa, em es-
%pecial do seu comprimento, em termos de número de tokens, linhas de código fonte, ou
%estruturas de controle. Este último tipo é o que iremos estudar neste trabalho.
%
%Os sistemas complexos podem naturais ou artificiais, uma colônia de formigas
%por exemplo pode ser caracterizado como um sistema complexo natural, onde
%individualmente cada formiga se apresenta como criatura relativamente simples,
%com instintos básicos como procurar alimento, responder a estímulos químicos
%vindos de outras formigas, combater intrusos, etc. No entanto, quando
%observadas coletivamente em uma colônia, as formigas aparentam ser muito mais
%sofisticadas. Elas são capazes de se organizar em diferentes atividades, criar
%estruturas complexas dentro de seu formigueiro, e de encontrar o caminho mais
%curto para uma fonte de alimento.
%
%Os sistemas naturais são aqueles cuja constituição não tem participação humana.
%Sistemas artificiais são projetados por humanos, com objetivos e funções
%definidos.  Sistemas artificiais podem ou não serem projetados à imagem de um
%sistema natural, e durante a sua concepção eles são discutidos em termos tanto
%de suas características (o que eles são) como de necessidades que eles devem
%satisfazer (o que eles deveriam ser) \cite{Simon1996}.
%
%É importante ressaltar que esta caracterização de sistemas de software como
%sistemas complexos diz respeito à estrutura interna dos sistemas, ou seja, aos
%componentes que o constituem e ao relacionamento entre estes componentes. Não
%foram considerados outros aspectos importantes de sistemas complexos, como por
%exemplo o seu relacionamento com o ambiente externo.
%
%como uma combinação das métricas de acoplamento (CBO) e coesão (LCOM4), 
%
%Sistemas de software, no entanto, se
%diferenciam dos sistemas complexos naturais pelo fato de serem projetados;
%consequentemente, o seu processo evolucionário não é intrinsecamente parte do
%seu comportamento, mas fruto da ação consciente de seus desenvolvedores.
%
%\cite{Tegarden1995}
%
%"The implication of this result is that, when
%designing, implementing, and maintaining software to control complexity, both coupling and cohesion should be considered jointly,
%instead of independently" Darcy 2005
%
%Many studies have demonstrated a significant correlation between
%LOC and the cyclomatic number. The researchers usually suggest that
%this correlation proves that cyclomatic number increases with size; that
%is, larger code is more complex code. However, careful interpretation of
%the measures and their association reveals only that the number of deci-
%sions increases with code length, a far less profound conclusion. The cyclo-
%matic number may be just another size measure. Chapter 9 contains more
%detailed discussion of validation for the McCabe measures.
%
%{\bf SC} {\it Structural Complexity (Complexidade estrutural)}: mede a
%complexidade do software \cite{Darcy2005} combinando os valores de CBO e LCOM4.

\citeonline{Darcy2005} definem complexidade estrutural como uma combinação de
acoplamento e coesão. Estes são dois conceitos complementares: enquanto o
acoplamento reflete o relacionamento entre módulos, a coesão nos fornece uma
visão da organização dos componentes internos de um módulo e seus
relacionamentos.

Uma formalização da métrica proposta por \citeonline{Darcy2005} pode ser
expressa da seguinte maneira, para um projeto $p$ e seu conjunto de módulos
$M(p)$, a complexidade estrutural $SC(p)$ de $p$ é:

\begin{equation}
SC(p) = \frac
{ \displaystyle \sum_{m \in M(p)} CBO(m) \times LCOM4(m) }
{ |M(p)| }
\end{equation}

Esta medida de complexidade estrutural é portanto a complexidade
estrutural média entre todos os módulos do sistema.

\begin{itemize}

  \item {\bf CBO} {\it Coupling Between Objects (Acoplamento entre objetos)}:
    mede o acoplamento entre objetos do software \cite{Chidamber1994}
    calculando em nível de classe o número total de acessos à outras classes do
    mesmo sistema, é comum ser também chamada de Fan-out da classe. CBO é então
    definida pela seguinte fórmula:

\begin{equation}
\label{formula-cbo}
CBO(C) = \sum_{i=1}^{n} cliente(C, Ci)
\end{equation}

Onde:

\begin{equation}
cliente(Ci, Cj) =
  \begin{cases}
    1 \text{ se } Ci \Rightarrow Cj \wedge Ci \neq Cj \\
    0 \text{ caso contrario} \\
  \end{cases}
\end{equation}

A notação $ Ci \Rightarrow Cj $ indica acesso à atributos, variáveis, métodos ou funções
entre módulos ou classes.

Apesar de ser possível formalizar a métrica CBO através da fórmula acima, sua descriçao original é
bastante complexa, levando à implementações variadas do seu cálculo
\cite{Lincke2008}. A definição original, por exemplo, inclui explicitamente
acoplamento via herança \cite{Harrison1998}, no entando não deixa claro como
deve ser tratado métodos herdados \cite{Briand1999}. A definição original
afirma também que apenas chamadas explícitas (e não chamadas implicitas) de
construtores são contabilizadas. Algumas definições de CBO incluem não apenas $
cliente(Ci, Cj) $ mas também a recíproca $ cliente(Cj, Ci) $ de forma que o valor
final inclui classes que ela acessa somado ao número de classes do sistema que
a acessam \cite{Sant2008}.

Quanto mais as classes forem independentes, mais fácil é reutilizá-las e menos
arriscado é modificá-las. Classes mais acopladas precisam de mais rigor em
testes, pois mais partes do sistema dependem delas.

  \item {\bf LCOM4} {\it Lack of Cohesion in Methods (Ausência de coesão em
    métodos)}: mede o grau de falta de coesão em métodos \cite{Hitz1995}.

O cálculo de LCOM4 é dado através de grafo não-orientado em que os nós ou
vértices são os métodos e atributos de uma classe e as arestas são acessos à
métodos e atributos. O cálculo desta métrica pode ser formalizado como a
seguir \cite{Silva2012}.

Seja $ X $ uma classe qualquer e $ M_x $ o conjunto de métodos desta classe,
considere um grafo simples não-orientado $ G_x(V, E) $, sendo:

\begin{equation}
V = M_x
\text{ e }
E = \{ \langle m, n \rangle \in V \times V \}
\end{equation}

Onde:
\begin{equation}
(\exists i \in Ix : (m \text{ accessos } i) \land (n \text{ accessos } i)) \lor (m \text{ chamadas } n) \lor (n \text{ chamadas } m)
\end{equation}

O valor da métrica LCOM4 para uma classe $ X $ é então definido como o número
de componentes conectados do grafo $ G_x (1 \leq LCOM(x) \geq | M_x |)$.

Coesão entre os métodos de uma classe é uma propriedade desejável, portanto o
valor ideal para esta métrica é 1. Se uma classe tem diferentes conjuntos de
métodos não relacionados entre si, é um indício de que a classe deveria ser
refatorada em classes menores e mais coesas.

\end{itemize}
