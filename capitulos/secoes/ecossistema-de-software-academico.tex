\section{Ecossistema de software}

Ecossistema de software é definido, segundo \citeonline{manikas2013software},
como a interação entre diversos atores numa plataforma tecnológica comum,
resultando em novas soluções de software ou novos serviços. Atores do
ecossistema, motivados por um conjunto de interesses, conectam-se entre si e ao
próprio sistema numa relação simbiótica, fazendo a plataforma tecnológica
evoluir enquanto permite o envolvimento e contribuição de novos e diferentes
atores \cite{manikas2013software}.

Nesta relação, os atores são beneficiados de formas diferentes a depender da
natureza do ecossistema. Num ambiente comercial, por exemplo, os atores são
beneficiados diretamente através de receita financeira (salário, prêmios, etc),
enquanto num sistema não-comercial os atores estão motivados por questões
não-monetárias, como fama, reconhecimento, ideologia, etc
\cite{manikas2013software}.

De modo geral, em ecossistemas de software, os benefícios recebidos pelos
atores e proporcionados pelo ecossistema, aumentam com o passar do tempo por
meio de uma relação de benefício mútuo.  Este modelo geral de funcionamento, no
entanto, pode variar a depender do contexto em que se insere o ecossistema,
especialmente no relacionamento entre os atores que pode variar entre
mutualismo, parasitismo, antagonismo e competição, etc
\cite{manikas2013software}.

\section{Ecossistema de software acadêmico}

O ecossistema de software acadêmico possui a particularidade de se relacionar
com o sistema econômico de reputação científica, especialmente com o seu modelo
de publicações, influenciando e sendo influenciado diretamente pelo impacto de
suas publicações \cite{howison2015understanding}.

Neste cenário, interessado em compreender as relações neste ecossistema,
\citeonline{howison2015understanding} criou um framework para pensar e refletir
sobre o processo de produção de software no meio científico, e identificou quatro
papéis básicos envolvidos no ecossistema de software acadêmico: 
(1) cientistas usuários finais, (2) produtores e distribuidores de software, (3)
administradores de infraestrutura e (4) pesquisadores preocupados com o
funcionamento do ecossistema como um todo.

\subsection{Cientistas usuários finais}

Cientistas ocupam papel chave no ecossistema de software acadêmico.  Em seus
processos de investigação e experimentação, fazem uso crescente de software
para coleta, gerenciamento, transformação, análise, modelagem e visualização de
dados. Além da preocupação com qualidade e usabilidade, estes cientistas estão
também preocupados com a disponibilidade e com a capacidade do software em
continuar sendo útil \cite{howison2015understanding}.

Finalmente, cientistas estão interessados também em saber o que outros
cientistas estão usando em suas pesquisas. A alta adoção de um software, além
de ser um bom indicador de qualidade, mantém os cientistas mais livres e com
maior foco em suas próprias pesquisas, uma vez que podem encontrar ajuda entre
os seus pares para resolver questões sobre o uso do software
\cite{howison2015understanding}.

%e de podendo ser utilizado em conjunto com outras
%soluções de software.

%garantir que o grupo de pesquisa, estudantes e colaboradores consigam 
% simplifica também o trabalho dos
%revisores pois encontrarão as mesmas facilidades no uso.

\subsection{Produtor e distribuidor de software acadêmico}

O papel de produtor e distribuidor de software costuma ser desempenhado por
equipes colaborando entre sí, geralmente compostas por cientistas da computação
e cientistas do domínio onde a pesquisa se insere.
Geralmente, o cientista da computação é o responsável por implementar os
algoritmos, métodos ou resultados gerados pelo estudo
\cite{howison2015understanding}.

Um desafio comum enfrentado neste papel é conseguir abstrair os problemas e
implementar soluções abrangentes em software.
Muitas vezes, o software criado fica confinado em seu laboratório ou grupo, mas
eventualmente é compartilhado e amplamente adotado, tornando o cientista autor
do software e da pesquisa parte do ecossistema \cite{howison2015understanding}.

Entre as inúmeras preocupações do produtor e distribuidor de software, podemos
destacar a preocupação acerca de como o software contribui para as
investigações científicas que outros pesquisadores estão realizando.
Alguns projetos são gerenciados no estilo de código aberto, e têm atraído com
sucesso contribuições de muitos cientistas, incluindo contribuidores que tem
fazem pequenas, porém substanciais contribuições
\cite{howison2015understanding}.

\subsection{Provedor de infraestrutura}

O provedor de infraestrutura é aquele que provê cojuntos de software aos
cientistas usuários finais. Este conjunto pode estar disponível em forma de
download para que seja utilizado em computadores pessoais ou pode estar
disponível em forma de serviços, como por exemplo, ciberinfraestrutura
\cite{council2007cyberinfrastructure, stewart2010cyberinfrastructure} de
software.
%ciberinfraestrutura de software hospedados em centros de supercomputação
%usando provedores de computação em nuvem.

Do ponto de vista do ecossistema, os dois tipos de distribuição implicam nas
mesmas preocupações e questões: quem usa o software, qual versão é utilizada, 
qual a frequencia de atualização, entre outras \cite{howison2015understanding}.
% preocupações relacionadas à distribuição e uso.

\subsection{Pesquisador}

Este último papel, chamado de pesquisador num sentido amplo, refere-se a
qualquer pessoa preocupada com o funcionamento do ecossistema e com a sua
contribuição para a Ciência. O papel de pesquisador costuma ser desempenhado
por agências de fomento ou por cientistas preocupados com o seu trabalho
individual e com o impacto em seu campo de pesquisa
\cite{howison2015understanding}.

As preocupações incluem questões sobre a operação do ecossistema
como um sistema que consome recursos (tempo, dinheiro e atenção) e afeta a
conduta da ciência, tanto no geral como em campos específicos, 
e sobre a compreensão do comportamento desse sistema e de 
como pode ser influenciado \cite{howison2015understanding}.

\section{Modelo de desenvolvimento de software acadêmico}

% recurso, uso e impacto

Cada ator desempenha um papel importante na estabilidade e sustentabilidade do
ecossistema \cite{dhungana2010software}. Assim como nos ecossistemas naturais,
o ecossistema de software necessita de fornecimento constante de energia, seja
na forma de novos desenvolvimentos ou na forma de ações de manutenção
\cite{dhungana2010software}.

Os atores participam dentro de seus próprios interesses, mas sempre causando
impacto no sistema como um todo \cite{manikas2013software}.
Cientistas usuários finais usam software acadêmico (direta ou indiretamente)
para fazer Ciência, resultando em impacto científico. Tal impacto científico
justifica novos investimentos, fazendo o ecossistema crescer
\cite{howison2015understanding}.

\begin{figure}[h]
  \center
  \includegraphics[scale=0.5]{imagens/process-model-scientific-software-dia.png}
  \caption{Um modelo de process de software na Ciência~\cite{howison2015understanding}}
  \label{process-model-scientific-software}
\end{figure}

A Figura \ref{process-model-scientific-software} apresenta um modelo de
desenvolvimento de software acadêmico, explicitando as relações entre os
elementos \textit{software}, \textit{recursos}, \textit{uso} e
\textit{impactos}, detalhados a seguir.

\subsection{Software acadêmico}

Software acadêmico ({\it academic software}) é todo software usado para
coletar, processar ou analisar resultados de pesquisas com intenção de
publicação na literatura acadêmica (periódicos, revistas, conferências,
monografias, livros ou teses), incluindo desde protótipos escritos pelos
próprios cientistas, a produtos completos desenvolvidos profissionalmente
\cite{allen2017engineering}.

Podem ser projetos de software desenvolvidos num modelo de {\bf Software como
serviço de suporte} ({\it Software as Supporting Service}), sobrevivendo
totalmente à parte do sistema de reputação acadêmica, ou no modelo de {\bf
Software para crédito acadêmico} ({\it Software for academic credit}), estando
seu desenvolvimento intrinsecamente associado ao sistema de reputação acadêmica
\cite{howison2011scientific}.

No segundo caso, Software para crédito acadêmico, a relação com o sistema de
reputação acadêmica pode ainda variar entre motivações distintas resultando em
(1) Software Incidental ({\it Incidental software})
feito puramente para apoiar e facilitar pesquisas,
(2) Prática de software paralela ({\it A parallel software practice})
feito com objetivo de ser utilizado por outros pesquisadores, ou
(3) Um subcampo de software ({\it A Software Subfield}),
onde o próprio software é considerado uma contribuição primária para a Ciência
\cite{howison2011scientific}.

%reputação direta pelo trabalho com o software.
%geralmente publicado em artigo de ferramenta em paralelo aos artigos sobre 
%a pesquisa principal sobre o domínio sendo estudado,

%Garante sua sobreviência independente do sistema de reputação acadêmica,
%geralmente software comercial,
%feito por desenvolvedores de software contratados,
%não costuma receber menção na literatura acadêmica.


  %Software for academic credit
  %\item [Software para crédito acadêmico]

%Está intrinsecamente associado ao sistema de reputação acadêmico,
%tendo como principal incentivo o crédito acadêmico, sendo desenvolvido
%como um Software Incidental ({\it Incidental software})
%feito puramente para apoiar e facilitar a pesquisa,
%como Uma Prática de Software Paralela ({\it A parallel software practice})
%feito com objetivo de ser utilizado por outros pesquisadores,
%geralmente publicado em artigo de ferramenta em paralelo aos artigos sobre 
%a pesquisa principal sobre o domínio sendo estudado,
%como Um Subcampo de Software ({\it A Software Subfield}),
%onde o software é considerado uma contribuição primária para a ciência,
%reputação direta pelo trabalho com o software.

%Incidental software
%typically written by individuals and not made available for
%others to use, at least not in any formal or on-going way.
%
%  %
%  [Uma prática de software paralela]
%        (scientific needed enhanced by publishing 'software papers' alongside domain research)
%
%  %
%  [Um subcampo de software]
%        (reputação direta pelo trabalho do software);
%o segundo sistema de produção que está associado ao crédito acadêmico
%publicações sobre software costuma ser visto como contribuição primária

  %Hybrids
%  \item [Híbridos]
%        licença-dual e 'software work' dentro de grandes colaborações (software como uma contribuição científica direta)
%
%\end{description}

%Software acadêmico também é referenciado na literatura acadêmica como
%{\it research tool} \cite{Portillo12},
%{\it research-originated software} \cite{Kon2011},
%{\it research software} \cite{hettrick2014uk} ou
%{\it scientific software} \cite{segal2008developing}.
%%esses artefatos tem sido estudados dos mais variados pontos
%%de vista, desde de sua qualidade interna, até o impacto que
%%causam no meio científico.

% Proprietary versus public domain licensing of software and research products
% este paper mostra que usar GPL em software traz vantagens e fala dos problemas em nao disponibilizar software academico

%Entre as inúmeras funções desempenhadas pelo software acadêmico em suas
%pesquisas, identificam-se motivações distintas para a sua criação,
%seis modelos de produção de software na ciência que abstraem uma série de
%características sobre como são criados, mantidos e compartilhados,
%estes modelos são caracterizados especialmente pelos incentivos e
%recompensas disponíveis aos seus atores e estão associados a um conjunto de
%práticas de engenharia de software \cite{howison2011scientific}.

\subsection{Recursos}

Os recursos investidos na produção de software acadêmico vêm de diversas
fontes, incluindo ganhos monetários diretos, recursos alocados em projetos, e
colaboração entre laboratórios de pesquisa \cite{howison2015understanding}.
Grande parte dos recursos vem do ``tempo livre'' dos pesquisadores em busca de
soluções para suas pesquisas e perpassa por financiamentos diversos obtidos na
carreira individual do cientista, prêmios recebidos, etc
\cite{howison2015understanding}.

Independente da origem dos recursos, grande parte do desenvolvimento de
software acadêmico é realizado pelos próprios cientistas \cite{hettrick2014uk,
momcheva2015software}.
Esta tendência tem sido interpretada como um reflexo do conhecimento sobre
domínio da pesquisa muitas vezes necessário ao desenvolvedor deste software
\cite{segal2008developing}.

No caso da pesquisa em Engenharia de Software, este conhecimento teórico sobre
o domínio se confunde, muitas vezes, com a própria prática de desenvolvimento.

\subsection{Uso}

% ... software é distribuido, utilizado e dá suporte à ciencia, gerando impacto ...

Metade dos pesquisadores de todas as áreas da Ciência fazem uso intenso de
software acadêmico, desde grupos trabalhando exclusivamente com problemas
computacionais até grupos em laboratórios tradicionais ou em campo
\cite{wilson2014best}.

Este uso é mencionado em suas pesquisas, por meio de citação formal ou informal
\cite{smith2016software}.
Estas menções são parte do sistema econômico de reputação científica e causam
impacto científico direto tanto na publicação quanto no ecossistema de software
acadêmico\cite{katz2014transitive}.

Este impacto direto geralmente justifica o investimentos de novos recursos no
ecossistema, seja para fins de planejamento, como retrospectiva para avaliar
investimentos já realizados ou para promover evolução do software acadêmico
\cite{howison2015understanding}.

\subsection{Impacto científico}

Ao longo da história, a citação formal tem sido utilizada para garantir
autenticidade e autoridade, ao invés de crédito e reconhecimento
\cite{katz2014transitive}.
Na história ocidental, a citação surge no final do século XVI e, no início do
século XVIII surgem o sistema legal por trás do sistema de citações e a lei de
``copyright'' para garantir os direitos dos autores \cite{katz2014transitive}.

Apesar do grande uso para garantir autenticidade e autoridade, o sistema de
citações e a informação sobre a autoria das publicações tem sido realmente
utilizado para avaliações importantes dentro do corpo científico
\cite{katz2014transitive}.
Por exemplo, ``backward citing'' tem sido utilizado para se certificar quem de
fato contribuiiu para um certo avanço ou descoberta, e ``forward citing'' tem
sido usada em casos onde se quer entender como uma idéia foi usada após o seu
surgimento ou publicação \cite{katz2014transitive}.

%Tradicionalmente, um autor cita um artigo anterior adicionando uma referência 
%ao autor, título, local de publicação, etc.

Conhecimento novo é claramente construído a partir do conhecimento passado e o
sistema de citações formais tem promovido avanços significativos
\cite{katz2014transitive}.
No entanto, esse conceito não tem funcionado tão bem para produtos digitais
como o software, que muitas vezes depende de outro software, fragmentos de
código, e algoritmos \cite{katz2014transitive}.

Este debate ocorre há bastante tempo entre as diversas áreas da bibliometria,
cienciometria, altmetria e áreas similares \cite{gouveia2013altmetria}.
Por exemplo, o fator de impacto, proposto na década de 90
\cite{history_citation_indexing}, apesar de contribuir para a Ciência, por
vezes é utilizado da forma errada e mostra as deficiências de lidar bem com
produtos digitais gerados durante pesquisas \cite{katz2014transitive}.

%%% PAREI AQUI %%%

%%%%%%%%%%%%%%%%%%%%%%%%%%%%%%%%%%%%%%%%%%%%%%%%%%%%%%%%%%%%%%%%%%%%%

%Cita um mapeamento feito sobre estudos que criam ferramentas para apoio a
%revisão sistemática no domínio de SE, 14 estudos foram selecionados, ao final
%apenas 8 tinham proposta de ferramentas, ao final conclui que as ferramentas
%encontradas estão em estado inicial de desenvolvimento \cite{marshall2013tools}.

%Cita um mapeamento sistemático com objetivo de encontrar ferramentas de
%comunicação e coordenação para suporte a times altamente distribuidos
%gograficamente, encontrou 132 ferramentas, para uso em projetos de software
%global. A maioria destas ferramentas foram desenvolvidas em centros de
%pesquisas, e apenas uma pequena porcentagem (18.9\%) foram testados fora do
%seu contexto onde foi desenvolvido \cite{Portillo12}.

%Computer systems research spans sub-disciplines that in-
%clude embedded and real-time systems, compilers, network-
%ing, and operating systems. Our contention is that a number
%of structural factors inhibit quality research. We highlight
%some of the factors we have encountered in our work and ob-
%served in published papers and propose solutions that could
%both increase the productivity of researchers and the quality
%of their output \cite{Vitek2011}.

%Além da aplicação, estes softwares variam também no papel que ocupam em suas
%pesquisas, alguns fazem parte dos resultados da pesquisa, como por exemplo,
%propostas de novos algoritmos ou técnicas de produção, outros são utilizados
%como parte do método de pesquisa, como coleta ou análise de dados, sendo que
%estes papeis não são excludentes.
%
%estes costumam ser citados pelos seus autores como uma das contribuições do
%estudo, seja principal ou secundária, 
%Esses softwares podem, de fato, ser um software de simulação complexo desenvolvido
%e executado em um computador de alto desempenho, mas também pode ser um
%software desenvolvido em um PC para incorporação em instrumentos; para
%manipular, analisar ou visualizar dados; ou para orquestrar fluxos de trabalho.

%e à medida
%que percebe-se que os softwares estão se tornando parte integrante dos
%processos, ferramentas e produção científicas, torna-se necessário e urgente
%discutir o seu desenvolvimento, visibilidade, qualidade e sustentabilidade.

% mostrar os beneficios da ciencia aberta, ciberinfraestrutura, etc
% * (favorecendo a ciencia e tornando a vida mais feliz para todos)

% mostrar os problemas para a ciência como um todo
% * causando problemas para o progresso de ciência, dados perdidos, etc, retrabalho
%   dificuldade de reprodução, etc...

%, não apenas técnica, mas também a
%capacidade de ser encontrado, compartilhado e co-desenvolvido, qualidades
%importantes para a evolução do próprio software, mas também extremamente útil
%para um uso eficiente dos limitados recursos da ciência \cite{howison2013,
%katz2014transitive}.

%contradizendo as boas
%práticas de qualquer projeto experimental, de ter {\it laboratory
%notebooks}\footnote{\url{https://en.wikipedia.org/wiki/Lab_notebook}}, dados
%organizados, passos documentados, e projeto estruturado para reprodutibilidade.

%softwares acadêmicos, assim
%como qualquer outro aparato experimental, são tão importantes para a ciência
%quanto são os telescópios ou tubos de ensaio \cite{wilson2014best}.

%Cientistas gastam mais tempo hoje utilizando e desenvolvendo softwares do que
%gastavam no passado.

%Software is a critical part of modern research and yet there is little support across the
%scholarly ecosystem for its acknowledgement and citation. Inspired by the activities
%of the FORCE11 working group focused on data citation, this document
%summarizes the recommendations of the FORCE11 Software Citation Working
%Group and its activities between June 2015 and April 2016. Based on a review of
%existing community practices, the goal of the working group was to produce a
%consolidated set of citation principles that may encourage broad adoption of a
%consistent policy for software citation across disciplines and venues. Our work is
%presented here as a set of software citation principles, a discussion of the motivations
%for developing the principles, reviews of existing community practice, and a
%discussion of the requirements these principles would place upon different
%stakeholders. Working examples and possible technical solutions for how these
%principles can be implemented will be discussed in a separate paper.
%\cite{smith2016software}

%Improving academic software engineering projects: A comparative study of academic and industry projects
%(compara as praticas de desenvolvimento da industria e academia e sugere melhorias, 1998!)
%https://link.springer.com/article/10.1023%2FA%3A1018925902814?LI=true

% papel pesquisador no ecossistema de soft academico
%
%Essas preocupações gerais sugerem um conjunto de questões específicas, com foco
%em padrões globais e padrões emergentes dentro do ecossistema, incluindo: Quais
%recursos foram destinados à produção de software? Quantos usuários ou
%comunidades de usuários têm projetos? Quais são os impactos científicos desse
%uso? Os números de usuários crescem? Os projetos possuem recursos e habilidades
%suficientes para gerenciar seu crescimento? Quais projetos possuem
%funcionalidades sobrepostas? Há quanto tempo os pedaços de software e projetos
%persistem? Nós desconectamos as comunidades de usuários e desenvolvedores? São
%componentes específicos, ou camadas de componentes, faltam? Que código
%geralmente é usado em conjunto; são os projetos e as pessoas que produzem esses
%componentes se comunicando adequadamente? Como podemos sustentar o software
%crítico?
%
%Aqui há uma clara tensão entre um desejo de flexibilidade e liberdade, ligado
%às expectativas de inovação científica e desejos de estruturas de autoridade e
%controle de coordenação. As questões de influência incluem: como os programas
%de financiamento e quais os requisitos em suas chamadas, resultaram em software
%amplamente utilizado e impacto científico substancial? Quais são as
%características dos campos que alcançaram maior coalescência? Quais jornais e
%conferências têm políticas exemplares? Como o trabalho de software é visto
%dentro das práticas de contratação e avaliação, como os casos de posse?
%
%\cite{howison2015understanding}

%Ao longo da história, a citação formal foi para autenticação e autoridade, em
%vez de de crédito e reconhecimento ou atribuição. A  científico citação na
%história ocidental aparece no final dos anos 1500. No início dos anos 1700, a
%citação também aparece no sistema legal como método de compreensão dos
%precedentes \cite{katz2014transitive}.

%A ideia de direitos autorais como reconhecendo aos direitos dos seus autores
%também surge nesse tempo, 1710, talvez devido a uma lenta tendência social
%societária de reconhecer a propriedade intelectual, uma idéia que parece ter se
%desenvolvido ao lado da imprensa]. Observe que a autoria de papers é realmente
%usado para notar os autores reais do artigo quanto para notar os contribuidores
%do projeto.
%Para muitos desses, o
%identificador que deve ser citado - um "nome" que se refere a um produto único
%não é claro.

%Additionally, if a cited library depends
%on another library, the contribution of this second library
%is not captured. Citation of a dataset should perhaps give
%credit to the people who gathered the data, as well as
%those who curated it, but the paper author may not know
%or be able to find these details.

%Mas independente de como seja calculado o impacto científico de uma determinada
%pesquisa o impacto causado se reverte potencialmente em mais recursos que
%poderão ser reinvestidos no próprio ecossistema onde o software está inserido.

%Science Code Manifesto \cite{barnes2013science}.
%Foco em código fonte escrito especificamente para processar dados de
%publicações, afirma que ``todo código fonte escrito especificamente para
%processar dados de uma publicação deve estar disponível para os revisores e
%leitores do paper''.

%Sustentabilidade é um conceito guarda chuva composto de múltiplas dimensões, em
%sua dimensão técnica, chamada sustentabilidade técnica, temos a preocupação com
%a longevidade da informação, dos sistemas, e infraestrutura, e sua adequada
%evolução frente as condições do ambiente em constante mudança.

%citações formais facilitam e promovem o avanço
%da ciência, mesmo diante da falta de um padrão para citar artefatos digitais
%\cite{allen2014credit}.

%Um estudo recente com 90 artigos de diversas áreas da biologia, selecionados
%aleatoriamente entre publicações usando softwares como método, mostrou que
%apenas 59 mencionavam o uso de softwares de alguma forma, os demais 31 artigos,
%apesar de usar software acadêmico, não mencionavam nada a respeito
%\cite{howison2016software}, apenas entre 31\% e 43\% das menções aos softwares
%acadêmicos envolvem citação formal.

%Não existe ainda amadurecimento suficiente sobre como citar softwares e
%outros artefatos digitais em pesquisas científicas, não temos um padrão de como fazê-lo,
%cada autor cita à sua maneira, muitas vezes ao longo do texto, outras em seções
%específicas sobre a implementação do software, nem semprem informam onde
%encontrar uma cópia do software, ou ainda nem sobre o modelo em que o software
%é distribuído, ou se é de alguma forma distribuído ao público.

%Entre os softwares acadêmicos desenvolvidos por cientistas como apoio em suas
%pesquisas, não é raro que pesquisadores deixem de disponibilizar estes artefatos,
%assim como outros desdobramentos da pesquisa, como dados e outros. Ou ainda,
%mesmo disponibilizando tais artefatos em locais de público acesso, com o tempo,
%tais locais se tornam indisponíveis inviabilizando a obtenção de tais
%artefatos.

%A comunidade tem refletido sobre os problemas relacionados ao
%desenvolvimento, promoção e sustentabilidade desses softwares, e o
%impacto que tais problemas causam no meio científico \cite{allen2017engineering}.

%, e faz
%surgir questionamentos sobre sua qualidade, não apenas técnica, mas também a
%capacidade de ser encontrado, compartilhado e co-desenvolvido, qualidades
%importantes para a evolução do próprio software, mas também extremamente úteis
%para o uso eficiente dos limitados recursos da ciência \cite{howison2013,
%katz2014transitive}.

%, o ecossistema de software acadêmico por
%exemplo possui a particularidade de estar inserido no sistema de reputação
%científica de alguma forma.

%os atores recebem mais (ou melhores) benefícios com o crescimento
%do ecossistema, o ecossistema oferece cada vez mais (ou melhores) benefícios
%com as atividades dos seus atores, resultando numa relação de benefício
%mútuo.

%e pelo seu sistema de crédito acadêmico.

% (5) Tools in mining software repositories \cite{chaturvedi2013tools}
% Faz uma revisão dos papers submetidos ao MSR desde 2007 até 2013 (?) e
% identifica data sets, ferramentas e técnicas utilizadas pelos autores, mais
% da metade dos papers usam ou criam ferramentas, categoriza as ferramentas em
% ferramentas novas, ferramentas tradicionais, protótipos e scripts para
% mineração de dados

%que podem ser adotadas por outros cientistas,
%especialmente em outros domínios. 

%preocupam-se com o impacto cientifico tanto em termos de numero
%quando de tipos de usuários que seu software atinge, 

%a partir da evolução deste software ou a partir da produção de novo software.

