\section{Sustentabilidade de software acadêmico}

%\subsection{Sustentabilidade}

O desenvolvimento de software sustentável tem sido identificado como um desafio
chave no campo da ciência e da engenharia computacional, se sustentabilidade
não for levada em consideração em projetos de software, não importa qual o
domínio ou qual o propósito do software, perde-se a oportunidade de causar
mudanças positivas no planeta e na sociedade \cite{becker2014karlskrona}.

Sustentabilidade apesar de ser um conceito complexo e com mútiplas dimensões,
levando a debates profundos, possui um conceito geral bastante simples: refere-se à
capacidade de perdurar e de continuar sendo suportado ao longo do tempo, o que 
implica nas qualidades de longevidade e manunetabilidade do software
\cite{venters2014software}.

Software sustentável é aquele que continua a estar disponível no futuro, em
novas plataformas, atendendo continuamente às novas necessidades do ambiente
através de uma adequada evolução frente as condições em constante mudança
\cite{allen2017engineering}.

No entando, estudos mostram o decaimento de URLs ao longo do tempo, em
publicações com produção de artefatos digitais disponibilizados nestes
endereços tem uma tendência a tornarem-se indisponíveis ao longo dos anos
\cite{wren2017use}.

Isto tem motivado iniciativas de tornar estes artefatos duráveis e disponíveis,
visando especialmente garantir a longevidade dos artefatos e proporcionar que
um segundo pesquisador receba todos os benefícios do trabalho duro do primeiro
pesquisador \cite{king1995replication},
o {\it Journal of the American Statistical Association (JASA)}, por
exemplo, tem insistido na necessiade de estar disponíveis código e dados ao
menos durante a revisão dos manuscritos \cite{baker2016scientists}, agências de
financiamento, como o {\it US National Science Foundation}, estão começando a
reconhecer produtos de pesquisa, como software, assim como fazem com as
publicações, tornando o software produzido em pesquisas cidadão de primeira
classe na ciência \cite{allen2017engineering}.

Isto garante longevidade mas não implica em boa manutenabilidade, a qualidade
dos softwares acadêmicos tem sido questionada, a maioria dos cientistas autores
de software não sabe o quão confiável seu software é
\cite{merali2010computational}, muitos dos projetos de software acadêmico estão
em estado inicial de desenvolvimento \cite{marshall2013tools}, poucos foram
testados fora do contexto onde foram desenvolvidos \cite{portillo2012tools}.

\subsection{Problemas}

% problemas identificados no ecossistema de software acadêmico

O ecossistema de software acadêmico sofre de um fenômeno chamado de  
desordem caótica disfuncional ({\it ``dysfunctional chaotic
churn''}), caracterizado por:

\begin{quote}
Existência de muitos projetos, com poucos usuários, com
ciclos de vida curtos, que terminam em paralelo ao financiamento inicial,
comunidades desconectadas e paralelas, incompatibilidades entre projetos, e
tentativas aparentemente não coordenadas de ``reiniciar'' tudo ({\it re-boots})
\cite{howison2015understanding}.
\end{quote}

Este problema, apesar de ser apenas uma percepção, coincide com inúmeras
evidências a respeito de problemas com o desenvolvimento, reconhecimento e
sustentabilidade do software acadêmico \cite{allen2017engineering}.

%sabe-se que parte dos problemas são realmente fato, por exemplo,
%o {\it Dagstuhl Perspective Workshop}, evento organizado por um grupo de
%pesquisadores sêniores de renome internacional, realizado anualmente na
%universidade de Dagstuhl\footnote{\url{http://www.dagstuhl.de}} com o objetivo
%refletir sobre o estado da ciência da computação explorando tópicos novos e
%emergentes, em sua mais recente edição o workshop debateu sobre software

\subsubsection{Desenvolvimento}

O desenvolvimento de software acadêmico exige, muitas vezes, conhecimento
específico sobre o domínio do estudo sendo realizado,
por exemplo, entender como o DNA genômico
se transforma em cristais de proteína, ou estar familiarizado com os meandros
da dinâmica dos fluidos, ou saber como resolver 20 equações diferenciais
parciais simultâneas \cite{segal2008developing}.

Isto explica a grande participação dos cientistas no desenvolvimento de
software acadêmico, estudos tem mostrado que no reino unido entre todas as
áreas da ciência 56\% dos cientistas estão envolvidos no desenvolvimento de
software acadêmico \cite{hettrick2014uk}, outros estudos em grupos específicos mostram números ainda
maiores, na astronomia, por exemplo, 90\% dos cientistas desenvolvem software
acadêmico \cite{momcheva2015software}.

No entanto, a maior parte dos cientistas nunca tiveram treinamento algum sobre como escrever
softwares de forma eficiente, muitos não testam ou documentam os seus
softwares, faltam práticas básicas de desenvolvimento, como escrever código
legível, revisão de código, controle de versão, testes unitários, entre outros
\cite{wilson2017good}.

Isto tem ocasionado sérios erros computacionais em conclusões centrais da
literatura acadêmica, gerando retrabalho para retratar tais erros nas mais
diversas áreas da ciência \cite{merali2010computational}.
Dados são perdidos, análises levam mais tempo que o necessário e os
pesquisadores não conseguem a eficiência que poderiam ter ao trabalhar com
softwares acadêmicos \cite{wilson2017good}.
Causando um impacto negativo na visibilidade dos softwares acadêmicos e na
capacidade de serem encontrados e compartilhados \cite{howison2013incentives,
katz2014transitive}.

\subsubsection{Reconhecimento}

% visibilidade

Apesar do crescimento no uso de software e na consequente dependência entre
cientistas de todos os campos, tornando o software acadêmico parte integral da
prática científica, apesar do apelo da comunidade científica para que o
software acadêmico seja tratado como cidadão de primeira classe, estudos tem
mostrado que muitas pesquisas não mencionam sequer o uso de software acadêmico
em suas publicações mesmo tendo feito uso de tais artefatos
\cite{momcheva2015software} \cite{howison2016software}.

Isto tem prejudicado a visibilidade do software acadêmico causando impacto
negativo em seu ecossistema, um software invisível é frequentemente excluído de
revisões por pares, uma atividade que costuma contribuir para a qualidade geral
do trabalho publicado, além disso, o
impacto negativo na visibilidade do software acadêmico faz surgir uma
série de questionamentos sobre a sua qualidade e também sobre a
capacidade de ser encontrado, compartilhado e co-desenvolvido
\cite{howison2013incentives, katz2014transitive} \cite{howison2016software}.

Apesar de nem sempre ser possível, ou viável, ter tudo dentro de padrões
estritos, é preciso estar consciente das boas práticas ao produzir e utilizar
softwares acadêmicos, tanto para melhorar a própria abordagem quanto para
revisar outros trabalhos \cite{wilson2014best}. Um software acadêmico em bom
funcionamento devem atingir não apenas os objetivos de entendimento e
transparencia, mas também os objetivos voltados para replicação
\cite{stodden2010reproducible}, seja logo após sua publicação, seja daqui a 10 ou 50 anos.

\subsubsection{Manutenabilidade}

% falar de manutenabilidade como um eixo dentro de sustentabilidade técnica

Manutenabilidade é uma característica de qualidade que indica o quão fácil é
realizar atividades de evolução e manutenção em software
\cite{kumar2012survey}, um aspecto importante aos pesquisadores interessados em
adaptar software acadêmico, algo muitas vezes necessário ao reproduzir
pesquisas anteriores \cite{peng2011reproducible}.

Manutenabilidade é uma característica de qualidade externa que indica o quão
fácil é realizar atividades de evolução e manutenção em componentes de
software, ela pode ser medida através de características de qualidade interna
\cite{hashim1996software, dagpinar2003predicting}, uma vez que grande parte dos
engenheiros de software assumem que uma boa estrutura interna resulta em boa
qualidade externa \cite{fenton2014software}.
A estrutura interna de um software pode ser avaliada através da sua
complexidade, uma característica bastante referenciada na literatura como um
importante indicador de qualidade, estudos mostram que quanto maior a
complexidade, maior é o esforço de manutenção \cite{hashim1996software,
darcy2005structural}, em especial a complexidade estrutural, uma medida definida em
termos de acoplamento e coesão \cite{terceiro2012caracterizacao}.

A adoção e uso de software acadêmico está relacionado também à sua qualidade,
portanto é importante medir e coletar sua qualidade de alguma forma, qualidade
é um vasto assunto, um dos problemas comuns enfrentado pelos pesquisadores que
desenvolvem tais softwares é a manutenabilidade \cite{prlic2012ten}.

Estudos tem mostrado que grande parte das ferramentas de software criadas na
academia estão em estado inicial de desenvolvimento \cite{marshall2013tools} e
que apenas uma pequena porcentagem são testados fora do contexto onde foi
desenvolvido \cite{portillo2012tools}.


%Iniciativas desta natureza resolvem o problema de disponibilidade destes
%artefatos mas ainda não garantem adequada evolução frente a contínua mudança
%do ambiente, apesar de sustentabilidade não implicar diretamente em qualidade,
%
%Tanto a ciência quanto a
%engenharia dependem de resultados incrementais para sua evolução. No terceiro
%compromisso, relacionado ao conceito {\it desenvolvimento}, o Dagstuhl
%Manifesto enfatiza a necessidade de medir a qualidade e a sustentabilidade dos
%softwares científicos, tanto a priori quanto a posteriori.

%Um estudo sobre ecossistema de software acadêmico percebeu através dos relatos
%de grande parte dos colaboradores participantes do estudo que os projetos de software
%acadêmicos desenvolvidos na própria academia 


%Olhar melhor os atributos de qualidade no artigo de Venters:
% https://openresearchsoftware.metajnl.com/articles/10.5334/jors.ao/
% We propose that software sustainability should be considered in a similar manner to the concept of dependability[16]; 
% a measure of a system’s availability, integrity, maintainability, reliability, and safety 
% where the attributes of dependability are defined as:
% Availability: readiness for correct service;
% Integrity: the absence of improper system alteration;
% Maintainability: undergo modifications and repairs;
% Reliability: continuity of correct service;
% Safety: the absence of catastrophic consequences on the user(s) and the environment.

\begin{comment}

We propose that software sustainability can be defined as ‘a measure of a systems extensibility, interoperability, maintainability, portability, reusability, scalability, and usability’ where the attributes are defined as:
Extensibility: a measure of the software’s ability to be extended and the level of effort required to implement the extension;
Interoperability: the effort required to couple software systems together.
Maintainability: the effort required to locate and fix an error in operational software;
Portability: the effort required to port software from one hardware platform or software environment to another;
Reusability: the extent to which software can be reused in other applications;
Scalability: the extent to which software can accommodate horizontal or vertical growth.
Usability: the extent to which a product can be used by specified users to achieve specified goals with effectiveness, efficiency, and satisfaction in a specified context of use.
If we accept that the concept of sustainability goes beyond the software artifact itself then other quality attributes such as efficiency may be appropriate candidates:
Efficiency: the amount of computing resources and code required to execute a function.

\end{comment}

