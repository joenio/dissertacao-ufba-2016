\section{Sustentabilidade de software acadêmico}

O desenvolvimento de software sustentável tem sido identificado como um desafio
chave no campo da Ciência e da Engenharia Computacional.
Se sustentabilidade
não for levada em consideração em projetos de software, não importa qual o
domínio ou qual o propósito do software, perde-se a oportunidade de causar
mudanças positivas no planeta e na sociedade \cite{becker2014karlskrona}.

Sustentabilidade de software apesar de ser um conceito complexo e com mútiplas dimensões,
levando a debates profundos, possui um conceito geral bastante simples: refere-se à
capacidade de perdurar e de continuar sendo suportado ao longo do tempo, o que 
implica nas qualidades de longevidade e manunenabilidade do software
\cite{venters2014software}.

%Sustentabilidade de software tem sido um tema de intenso debate e inúmeras definições,
%frameworks, modelos, propostas de tornar sustentabilidade como um requerimento
%não-funcional de software, no entando em todos os em resumo uma conclusão tem sido
%geral e central, entre os frameworks, o contexto onde o software está inserido influencia na medição
%de sustentaabilidade \cite{venters2014software}. No entando, uma definição
%comum e compartilhada ainda não é realidade, não existe ainda uma definição
%clara do que significa sustentabilidade de software, e isto não pode ser
%subestimado. Entretando, esta falta de concenso, junto ao fato de que uma
%visão comum entre as propostas de frameworks para sustentabilidade é de que
%a definição e medição está intimamente relacionada ao contexto, ou domínio do
%problema \cite{venters2014software}, dessa forma uma boa estratégia de colaborar com esta grande figura é
%definir sustentabilidade em recortes distintos, como, sustentabildiade de
%software acadêmico de análise estática, por exemplo.

Software sustentável é aquele que continua a estar disponível no futuro, em
novas plataformas, atendendo continuamente às novas necessidades do ambiente
através de uma adequada evolução frente as condições em constante mudança
\cite{allen2017engineering}. Este conceito, no entando, ao ser aplicado ao
contexto de software acadêmico e artefatos digitais, mostra-nos um cenário
bastante preocupante, uma vez que, estudos recentes mostram que há uma tendência
ao decaimento de URLs ao longo do tempo em publicações com produção de
artefatos digitais \cite{wren2017use}.
%disponibilizados nestes endereços tem uma tendência a
%tornarem-se indisponíveis ao longo dos anos

Isto tem motivado iniciativas de tornar estes artefatos duráveis e disponíveis,
visando especialmente garantir a longevidade dos artefatos e proporcionar que
um segundo pesquisador receba todos os benefícios do trabalho duro do primeiro
pesquisador \cite{king1995replication},
o JASA\footnote{\url{http://www.amstat.org/PUBLICATIONS/jasa}} ({\it Journal of the American Statistical Association}), por
exemplo, tem insistido na necessiade de estar disponíveis código e dados ao
menos durante a revisão dos manuscritos \cite{baker2016scientists}, agências de
financiamento, como o {\it US National Science Foundation}, estão começando a
reconhecer produtos de pesquisa, como software, assim como fazem com as
publicações, tornando o software produzido durante pesquisas em cidadão de primeira
classe na Ciência \cite{allen2017engineering}.

Mas, além das garantias de longevidade ao software, a sua qualidade também deve
ser avaliada,
uma vez que
a maioria dos cientistas autores
de software não sabe o quão confiável seu software é
\cite{merali2010computational}, muitos dos projetos de software acadêmico estão
em estado inicial de desenvolvimento \cite{marshall2013tools}, e poucos foram
testados fora do contexto onde foram desenvolvidos \cite{portillo2012tools}.

%mas não implica em boa manutenibilidade, uma qualidade especialmente
%importante aos projetos de software acadêmico, visto que sua qualidade tem sido questionada, onde 

%\subsection{Problemas}

Dessa forma, não é difícil perceber que o ecossistema de software acadêmico sofre graves
problemas, percepção bastante similar ao fenômeno chamado de  desordem
caótica disfuncional ({\it ``dysfunctional chaotic churn''}), caracterizado
por:

\begin{quote}
Existência de muitos projetos, com poucos usuários, com
ciclos de vida curtos, que terminam em paralelo ao financiamento inicial,
comunidades desconectadas e paralelas, incompatibilidades entre projetos, e
tentativas aparentemente não coordenadas de ``reiniciar'' tudo ({\it re-boots})
\cite{howison2015understanding}.
\end{quote}

Este problema, apesar de ser apenas uma percepção, coincide com inúmeras
evidências a respeito de problemas com o desenvolvimento, reconhecimento e
sustentabilidade de software acadêmico \cite{allen2017engineering}.

%sabe-se que parte dos problemas são realmente fato, por exemplo,
%o {\it Dagstuhl Perspective Workshop}, evento organizado por um grupo de
%pesquisadores sêniores de renome internacional, realizado anualmente na
%universidade de Dagstuhl\footnote{\url{http://www.dagstuhl.de}} com o objetivo
%refletir sobre o estado da ciência da computação explorando tópicos novos e
%emergentes, em sua mais recente edição o workshop debateu sobre software

\begin{description}
\item \textbf{Desenvolvimento.}
O desenvolvimento de software acadêmico exige, muitas vezes, conhecimento
específico sobre o domínio do estudo sendo realizado,
por exemplo, entender como o DNA genômico
se transforma em cristais de proteína, ou estar familiarizado com os meandros
da dinâmica dos fluidos, ou saber como resolver 20 equações diferenciais
parciais simultâneas \cite{segal2008developing}.

Isto explica a grande participação dos cientistas no desenvolvimento de
software acadêmico, estudos tem mostrado, por exemplo, que no reino unido entre todas as
áreas da Ciência 56\% dos cientistas estão envolvidos no desenvolvimento de
software acadêmico \cite{hettrick2014uk}, outros estudos em grupos específicos mostram números ainda
maiores, na astronomia, por exemplo, 90\% dos cientistas desenvolvem software
acadêmico \cite{momcheva2015software}.

No entanto, a maior parte dos cientistas nunca tiveram treinamento algum sobre como escrever
software de forma eficiente, muitos não testam ou documentam os seus projetos de
software, faltam práticas básicas de desenvolvimento, como escrever código
legível, revisão de código, controle de versão, testes unitários, entre outros
\cite{wilson2017good}.

Isto tem ocasionado sérios erros computacionais em conclusões centrais da
literatura acadêmica, gerando retrabalho para retratar tais erros nas mais
diversas áreas da Ciência \cite{merali2010computational}.
Dados são perdidos, análises levam mais tempo que o necessário e os
pesquisadores não conseguem a eficiência que poderiam ter ao trabalhar com
software acadêmico \cite{wilson2017good}.
Causando um impacto negativo na visibilidade do software acadêmico e na
capacidade de ser encontrado e compartilhado \cite{howison2013incentives,
katz2014transitive}.

\item \textbf{Reconhecimento.}
% visibilidade
Apesar do crescimento no uso de software e na consequente dependência entre
cientistas de todos os campos, tornando o software acadêmico parte integral da
prática científica, apesar do apelo da comunidade científica para que o
software acadêmico seja tratado como cidadão de primeira classe, estudos tem
mostrado que muitas pesquisas não mencionam sequer o uso de software acadêmico
em suas publicações mesmo tendo feito uso de tais artefatos
\cite{momcheva2015software, howison2016software}.

Isto tem prejudicado a visibilidade do software acadêmico causando impacto
negativo em seu ecossistema, um software invisível é frequentemente excluído de
revisões por pares, uma atividade que costuma contribuir para a qualidade geral
do trabalho publicado, além disso, o
impacto negativo na visibilidade do software acadêmico faz surgir uma
série de questionamentos sobre a sua qualidade e também sobre a
capacidade de ser encontrado, compartilhado e co-desenvolvido
\cite{howison2013incentives, katz2014transitive, howison2016software}.

Apesar de nem sempre ser possível, ou viável, ter tudo dentro de padrões
estritos, é preciso estar consciente das boas práticas ao produzir e utilizar
software acadêmico, tanto para melhorar a própria abordagem quanto para
revisar outros trabalhos \cite{wilson2014best}. Um software acadêmico em bom
funcionamento deve atingir não apenas os objetivos de entendimento e
transparencia, mas também os objetivos voltados para replicação,
seja logo após sua publicação, seja daqui a 10 ou 50 anos \cite{stodden2010reproducible}.

\item \textbf{Manutenibilidade.}
% falar de manutenibilidade como um eixo dentro de sustentabilidade técnica
%
Manutenibilidade é uma característica de qualidade que indica o quão fácil é
realizar atividades de evolução e manutenção em software
\cite{kumar2012survey}, um aspecto importante aos pesquisadores interessados em
adaptar software acadêmico, algo muitas vezes necessário ao reproduzir
pesquisas anteriores \cite{peng2011reproducible}.

Estudos tem mostrado que grande parte das ferramentas de software criadas na
academia estão em estado inicial de desenvolvimento \cite{marshall2013tools} e
que apenas uma pequena porcentagem são testados fora do contexto onde foi
desenvolvido \cite{portillo2012tools}.

%A adoção e uso de software acadêmico está relacionado também à sua qualidade,
%portanto é importante medir e coletar sua qualidade de alguma forma, qualidade
%é um vasto assunto, um dos problemas comuns enfrentado pelos pesquisadores que
%desenvolvem tais projetos de software é a manutenibilidade \cite{prlic2012ten}.

A adoção e uso de software acadêmico está relacionado também à sua qualidade,
portanto é importante medir e coletar sua qualidade de alguma forma,
manutenibilidade apesar de ser um atributo de qualidade de software
inerentemente externa, pode ser medida através de
características de qualidade interna \cite{hashim1996software,
dagpinar2003predicting}, uma vez que grande parte dos engenheiros de software
assumem que uma boa estrutura interna resulta em boa qualidade externa
\cite{fenton2014software}.

No entanto é importante compreender os projetos de software acadêmico também em
termos de evolução e ciclo de vida, algo que pode influenciar enormemente as
medidas de qualidade medidas através de atributos internos do software, como
métricas de código fonte, por exemplo. 

\end{description}

%Iniciativas desta natureza resolvem o problema de disponibilidade destes
%artefatos mas ainda não garantem adequada evolução frente a contínua mudança
%do ambiente, apesar de sustentabilidade não implicar diretamente em qualidade,

%Tanto a ciência quanto a
%engenharia dependem de resultados incrementais para sua evolução. No terceiro
%compromisso, relacionado ao conceito {\it desenvolvimento}, o Dagstuhl
%Manifesto enfatiza a necessidade de medir a qualidade e a sustentabilidade dos
%softwares científicos, tanto a priori quanto a posteriori.

%Um estudo sobre ecossistema de software acadêmico percebeu através dos relatos
%de grande parte dos colaboradores participantes do estudo que os projetos de software
%acadêmicos desenvolvidos na própria academia 
