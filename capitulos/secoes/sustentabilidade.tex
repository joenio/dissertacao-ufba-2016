\section{Sustentabilidade de software acadêmico}
\label{sec:sustentabilidade}

O desenvolvimento de software sustentável tem sido identificado como um desafio
chave no campo da Ciência e da Engenharia Computacional.
Se sustentabilidade
não for levada em consideração em projetos de software, não importa qual o
domínio ou qual o propósito do software, perde-se a oportunidade de causar
mudanças positivas no planeta e na sociedade \cite{becker2014karlskrona}.

Sustentabilidade de software apesar de ser um conceito complexo e com mútiplas dimensões,
levando a debates profundos, possui um conceito geral bastante simples: refere-se à
capacidade de perdurar e de continuar sendo suportado ao longo do tempo, o que 
implica nas qualidades de longevidade e manutenibilidade do software
\cite{venters2014software}.

Software sustentável é aquele que continua a estar disponível no futuro, em
novas plataformas, atendendo continuamente às novas necessidades do ambiente
através de uma adequada evolução frente a condições em constante mudança
\cite{allen2017engineering}. Este conceito, no entanto, ao ser aplicado ao
contexto de software acadêmico e artefatos digitais, mostra-nos um cenário
bastante preocupante, uma vez que, estudos recentes mostram que há uma tendência
ao decaimento de URLs ao longo do tempo em publicações com produção de
artefatos digitais \cite{wren2017use}.

Isto tem motivado iniciativas de tornar estes artefatos duráveis e disponíveis,
visando especialmente garantir a longevidade dos artefatos e proporcionar que
um segundo pesquisador receba todos os benefícios do trabalho do primeiro
pesquisador \cite{king1995replication}.
O {\it JASA (Journal of the American Statistical Association})\footnote{\url{http://www.amstat.org/PUBLICATIONS/jasa}}, por
exemplo, tem insistido na necessidade de disponibilizar código e dados ao
menos durante a revisão dos manuscritos \cite{baker2016scientists}.
Agências de financiamento, como o {\it US National Science Foundation}, estão começando a
reconhecer produtos de pesquisa, como software, assim como fazem com as
publicações, tornando o software produzido durante pesquisas em cidadão de primeira
classe na Ciência \cite{allen2017engineering}.

Mas, além das garantias de longevidade ao software, a sua qualidade também deve
ser avaliada, uma vez que
a maioria dos cientistas autores
de software não sabe o quão confiável seu software é
\cite{merali2010computational}.
Muitos dos projetos de software acadêmico estão
em estado inicial de desenvolvimento \cite{marshall2013tools}, e poucos foram
testados fora do contexto onde foram desenvolvidos \cite{portillo2012tools}.

Dessa forma, não é difícil perceber que o ecossistema de software acadêmico sofre graves
problemas, percepção bastante similar ao fenômeno chamado de  desordem
caótica disfuncional ({\it ``dysfunctional chaotic churn''}), caracterizado
por:

\begin{quote}
Existência de muitos projetos com poucos usuários,
com ciclos de vida curtos que se encerram junto ao financiamento inicial,
comunidades de usuários desconectadas e paralelas,
incompatibilidades entre os projetos de maneira persistente e imutável, e
tentativas constantes e aparentemente não coordenadas de ``reiniciar'' tudo ({\it re-boots})
\cite{howison2015understanding}.
\end{quote}

Este problema, apesar de ser apenas uma percepção, coincide com inúmeras
evidências a respeito de problemas com o desenvolvimento, reconhecimento e
sustentabilidade de software acadêmico \cite{allen2017engineering}.

\begin{description}
\item \textbf{Desenvolvimento.}
O desenvolvimento de software acadêmico exige, muitas vezes, conhecimento
específico sobre o domínio do estudo sendo realizado,
por exemplo, entender como o DNA genômico
se transforma em cristais de proteína, ou estar familiarizado com os meandros
da dinâmica dos fluidos, ou saber como resolver 20 equações diferenciais
parciais simultâneas \cite{segal2008developing}.

Isto explica a grande participação dos cientistas no desenvolvimento de
software acadêmico. Estudos têm mostrado, por exemplo, que no Reino Unido entre todas as
áreas da Ciência 56\% dos cientistas estão envolvidos no desenvolvimento de
software acadêmico \cite{hettrick2014uk}, outros estudos em grupos específicos mostram números ainda
maiores, na astronomia, por exemplo, 90\% dos cientistas desenvolvem software
acadêmico \cite{momcheva2015software}.

No entanto, a maior parte dos cientistas nunca tiveram treinamento algum sobre como escrever
software de forma eficiente, muitos não testam ou documentam os seus projetos de
software, faltam práticas básicas de desenvolvimento, como escrever código
legível, revisão de código, controle de versão, testes unitários, entre outros
\cite{wilson2017good}.

Isto tem ocasionado sérios erros computacionais em conclusões centrais da
literatura acadêmica, gerando retrabalho para retratar tais erros nas mais
diversas áreas da Ciência \cite{merali2010computational}.
Dados são perdidos, análises levam mais tempo que o necessário e os
pesquisadores não conseguem a eficiência que poderiam ter ao trabalhar com
software acadêmico \cite{wilson2017good},
causando um impacto negativo na visibilidade do software acadêmico e na
capacidade de ser encontrado e compartilhado \cite{howison2013incentives,
katz2014transitive}.

\item \textbf{Reconhecimento.}

Apesar do crescimento no uso de software e na consequente dependência entre
cientistas de todos os campos, tornando o software acadêmico parte integral da
prática científica, apesar do apelo da comunidade científica para que o
software acadêmico seja tratado como cidadão de primeira classe, estudos têm
mostrado que muitas pesquisas não mencionam sequer o uso de software acadêmico
em suas publicações mesmo tendo feito uso de tais artefatos
\cite{momcheva2015software, howison2016software}.

Isto tem prejudicado a visibilidade do software acadêmico causando impacto
negativo em seu ecossistema, um software invisível é frequentemente excluído de
revisões por pares, uma atividade que costuma contribuir para a qualidade geral
do trabalho publicado. Além disso, o
impacto negativo na visibilidade do software acadêmico faz surgir uma
série de questionamentos sobre a sua qualidade e também sobre a
capacidade de ser encontrado, compartilhado e co-desenvolvido
\cite{howison2013incentives, katz2014transitive, howison2016software}.

Apesar de nem sempre ser possível ou viável, ter tudo dentro de padrões
estritos, é preciso estar consciente das boas práticas ao produzir e utilizar
software acadêmico, tanto para melhorar a própria abordagem quanto para
revisar outros trabalhos \cite{wilson2014best}. Um software acadêmico em bom
funcionamento deve atingir não apenas os objetivos de entendimento e
transparência, mas também os objetivos voltados para replicação,
seja logo após sua publicação, seja daqui a 10 ou 50 anos \cite{stodden2010reproducible}.

\item \textbf{Manutenibilidade.}

Manutenibilidade é uma característica de qualidade que indica o quão fácil é
realizar atividades de evolução e manutenção em software
\cite{kumar2012survey}, um aspecto importante aos pesquisadores interessados em
adaptar software acadêmico, algo muitas vezes necessário ao reproduzir
pesquisas anteriores \cite{peng2011reproducible}.

Estudos têm mostrado que grande parte das ferramentas de software criadas na
academia estão em estado inicial de desenvolvimento \cite{marshall2013tools} e
que apenas uma pequena porcentagem são testados fora do contexto onde foi
desenvolvido \cite{portillo2012tools}.

A adoção e uso de software acadêmico está relacionado também à sua qualidade,
portanto é importante medir e coletar sua qualidade de alguma forma.
Manutenibilidade, por exemplo, 
apesar de ser um atributo de qualidade de software inerentemente externa, 
pode ser medida através de características de qualidade interna \cite{hashim1996software,
dagpinar2003predicting}, uma vez que grande parte dos engenheiros de software
assumem que uma boa estrutura interna resulta em boa qualidade externa
\cite{fenton2014software}.

No entanto é importante compreender os projetos de software acadêmico também em
termos de evolução e ciclo de vida, algo que pode influenciar enormemente as
medidas de qualidade coletadas através de atributos internos do software, como
métricas de código-fonte, por exemplo. 

\end{description}
