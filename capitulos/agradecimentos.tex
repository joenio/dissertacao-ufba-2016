Agradeço aos amigos
Daniela Feitosa, Marcos Ronaldo, Fabiana Goa, Lucas Kanashiro e Hilmer Neri
pelas valiosas dicas durante os momentos em que precisei de ajuda.

À Mariel Zasso pelas inúmeras ótimas tardes de estudo e trabalho em colaboração,
contribuindo para o avanço e foco no trabalho, por vezes difícil de manter.

Aos professores do PGCOMP Claudio Sant'ana e Ivan Machado pelo ótimo feedback e
sugestões de melhorias gerais a todo a pesquisa.

Ao todos os amigos do Programa de Pós-Graduação em Ciência da Computação da UFBA
pela troca de conhecimento, em especial ao amigo Crescêncio Lima.

A minha irmã Talita Gentil pelo apoio e descontração em todos os momentos,
exceto na defesa por não estar presente.

Aos meus grandes amigos e parceiros de lar José Flávio e Paulo Alexandre
por me servirem café, cerveja e outros petiscos sempre que me tranquei ao quarto
para conclusão de mais um capítulo deste texto.

Ao amigo e ex-sócio Antonio Terceiro por de algums forma me incentivar a
entrar no programa de mestrado PGCOMP da Universidade Federal da Bahia.

À minha eterna amiga, ex-esposa, Janaína Santos Valente por participar pacientemente
com todo apoio desde o início desta pesquisa, e pelo grande apoio, recepção e
carinho, especialmente nas últimas semanas de conclusão deste trabalho.

Agradeço também à banca de defesa Rodrigo Rocha e Sandro
Andrade pela atenção dada ao trabalho e pelas dicas valiosas para melhoria da
qualidade final do estudo.

À professora e orientadora Christina von Flach pela liberdade
dada nos rumos da pesquisa e pela enorme capacidade de orientação.

Ao meu amigo e co-orientador Paulo Meirelles pela energia em toda a trajetória
me fazendo continuar e chegar té aqui.

Um agradecimento também a Universidade Federal da Bahia (UFBA) por me aceitar no
ótimo Programa de Pós-Graduação em Ciência da Computação e oferecer ...
