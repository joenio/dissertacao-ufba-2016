Agradeço as amigas
Daniela Soares Feitosa,
Fabiana Goa,
Hilmer Rodrigues Neri,
Lucas Kanashiro Duarte e
Marcos Ronaldo Pereira Júnior
pelas correções, idéias e dicas valiosas ao texto.

À
Mariel Rosauro Zasso
pelas inúmeras tardes de estudo e trabalho em colaboração, contribuindo para o
avanço do trabalho e para a manutenção do meu foco no estudo, por vezes difícil
de manter.

Aos professores do Programa de Pós-Graduação em Ciência da Computação (PGCOMP-UFBA)
Cláudio Nogueira Sant'Anna e
Ivan do Carmo Machado
pelo ótimo {\it feedback} e sugestões de melhorias a toda a pesquisa.

A todos os amigos, mestrandos e doutorandos, do PGCOMP-UFBA pela troca de
conhecimento e apoio, em especial ao amigo Crescencio Rodrigues Lima Neto,
pelas ótimas reflexões a respeito do tema desta pesquisa.

A minha irmã
Talita Fernanda Gentil
pelo apoio e descontração em todos os momentos, exceto na defesa por não estar
presente ¬¬.

Aos meus grandes amigos e parceiros de lar
José Flávio Fernandino Maciel e
Paulo Alexandre Elias Passos
por me cuidarem em diversos momentos de isolamento e concentração necessários para
a conclusão desta pesquisa.

Ao amigo e ex-sócio de Colivre
Antonio Soares de Azevedo Terceiro
por ter me incentivado a entrar no programa de mestrado do PGCOMP-UFBA.

À minha grande amiga,
Janaína Santos Valente
por participar pacientemente com todo apoio desde o início desta pesquisa, e
pelo grande apoio, recepção e carinho, especialmente nos últimos meses de
conclusão deste trabalho.

Agradeço também à banca de defesa
Sandro Santos Andrade e
Rodrigo Rocha Gomes e Souza
pela atenção dada ao trabalho e pelas dicas valiosas para melhoria da qualidade
final do estudo.

À minha professora e orientadora
Christina von Flach Garcia Chavez
pelo cuidado, pela orientação e pela liberdade dada nos rumos desta pesquisa.

Ao meu amigo e co-orientador
Paulo Roberto Miranda Meirelles
pela energia investida, me fazendo continuar e seguir em frente, chegando até
aqui.

Obrigado a todos!
