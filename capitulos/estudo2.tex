\xchapter{Visibilidade científica dos projetos de software acadêmico de análise estática}
{Este capítulo apresenta uma revisão de literatura nas bases da ACM e IEEE em
busca de menções aos projetos de software acadêmico de análise estática
publicados nas conferências ASE e SCAM até o ano de 2015.}

Este estudo caracterizou como os projetos de software acadêmico de análise
estática publicados nas conferências de Engenharia de Software ASE e SCAM
selecionados no Capítulo \ref{estudo1} são mencionados em publicações
encontradas nas bases ACM e IEEE.

A seção ? apresenta ...

\section{Introdução e Motivação}

(pendente)

\section{Fundamentação} \label{estudo2:fundamentacao} % {{{

\subsection{Visibilidade científica}

Visibilidade científica é ...

\subsection{Menções}

Menções neste estudo refere-se a qualquer referência aos projetos de software
estudados, incluindo citações formais e informais, mas identificando o projeto
de software minimamente com o nome.

% }}}

\section{Definição} \label{estudo2:definicao} % {{{

Como os projetos de software de análise estática desenvolvidos e publicados nas
conferências de Engenharia de Software ASE e SCAM são mencionados em
publicações nas bases da ACM e IEEE?

\subsection{Definição do Objetivo}

\begin{description}
\item{\bf Objeto de estudo.} 
O objeto de estudo são projetos de software publicados nas conferências ASE e SCAM.

\item{\bf Propósito.} 
O propósito deste estudo é caracterizar menções aos projetos.

\item{\bf Perspectiva.} 
A perspectiva considerada é a de cientistas.

\item{\bf Foco de qualidade.} 
O principal aspecto de qualidade estudado é a visibilidade científica aos projetos.

\item{\bf Contexto.} 
O estudo foi conduzido com publicações das bases ACM e IEEE.
\end{description}

\subsection{Sumário da Definição}

Analisar os \textit{projetos de software acadêmico de análise estática publicados nas conferências ASE e SCAM}
com o propósito de \textit{caracterizar}
com respeito a \textit{visibilidade científica}
na perspectiva de \textit{cientistas}
no contexto de \textit{publicações nas bases ACM e IEEE}.

\subsection{Questões de Pesquisa}

Neste estudo as seguintes questões de pesquisa, a respeito dos projetos de
software acadêmico de análise estática, serão investigadas:

\newcommand{\EstudoDoisQuestaoUm}{Como os projetos de software acadêmico de
análise estática publicados nas conferências ASE e SCAM são mencionados em
publicações encontradas nas bases ACM e IEEE ao longo dos anos?}

\newcommand{\EstudoDoisQuestaoDois}{Os projetos de software acadêmico de
análise estática publicados nas conferências ASE e SCAM são utilizados em
estudos encontrados nas bases ACM e IEEE?}

\newcommand{\EstudoDoisQuestaoTres}{Os projetos de software acadêmico de
análise estática publicados nas conferências ASE e SCAM recebem contribuições
de estudos encontrados nas bases ACM e IEEE?}

\begin{description}
  \item [Q1:] \EstudoDoisQuestaoUm
  \item [Q2:] \EstudoDoisQuestaoDois
  \item [Q3:] \EstudoDoisQuestaoTres
\end{description}

\subsection{Métricas}

Para responder às questões de pesquisas, as seguintes métricas serão usadas:

\begin{enumerate}
  \item Número de publicações nas bases ACM e IEEE mencionando os projetos de software
  \item Número de publicações nas bases ACM e IEEE mencionando uso aos projetos de software
  \item Número de publicações nas bases ACM e IEEE mencionando contribuição aos projetos de software
%  \item Número de menções aos projetos de software nas bases ACM e IEEE
%  \item Número de autores publicando sobre os projetos nas bases da ACM e IEEE
\end{enumerate}

% }}}

\section{Planejamento do Estudo} \label{estudo2:planejamento} % {{{

O estudo foi realizado a partir de uma revisão de literatura nas bases da ACM e
IEEE em busca de menções aos projetos de software acadêmico de análise
estática, as menções encontradas foram codificadas num esquema utilizado para
caracterização dos estudos selecionados na revisão de literatura.

\subsection{Menções aos projetos de software de análise estática}

As menções aos projetos de software acadêmico de análise estática foram pesquisados nas
bases ACM e IEEE através de um procedimento organizado em 4 passos: Busca,
Triagem, Keywording e Extração.

\subsubsection{Passo 1: Busca}

% strings de busca > conjunto de metadados bibtex

Neste primeiro passo elaboramos as strings de busca para cada projeto, são duas
strings para cada um, uma para a base da ACM, outra para a base da IEEE, os
resultados da busca de cada string, uma para cada base, são armazenados
localmente em arquivos no formato BibTeX, ambas as bases exportam os resultados
neste formato.

As strings de busca são elaboradas a partir dos dados coletados no estudo
anterior, como, nome do projeto, descrição, nome do autor da publicação
original, etc. A elaboração das strings ocorre num processo incremental e
iterativo, iniciando por uma busca utilizando apenas o nome do projeto, se os
resultados, inspecionados através do título, trazem muitos estudos sem relação
alguma com o projeto especializamos um pouco mais a string de busca adicionando
outros dados além do nome do projeto.

%buscamos inicialmente usando apenas o nome do software, analisamos o número total
%de resultados e os títulos destes resultados, quando o número de resultados for
%muito grande, acima de XXX, e os títulos não aparentavam ser estudos com relação
%aos projetos, incluímos mais características, como por exemplo, parte da descrição
%do software, ou parte da URL, autores, etc.

%% Cuidado que ACM cita IEEE.
Os resultados trazidos pela string final são armazenados localmente em formato
BibTeX e pdf, estes resultados devem incluir ao menos os artigos selecionados
no estudo anterior onde os projetos foram encontrados, já que tais artigos são
o caso mínimo de resultado desejado nesta fase de busca. Serão armezados 3
arquivos BibTeX para cada software, um com o resultado da busca na ACM, outro
no IEEE, e um com um merge destes dois mantendo resultados únicos sem
duplicidade.

%Para cada software, os resultados foram agrupados
%num arquivo único, sem duplicidade entre os resultados trazidos por cada base
%bibliográfica. O arquivo de metadados de cada software contém informações sobre
%o artigo, autores, ano de publicação, conferência, jornal, etc. Os artigos
%também foram armazenados localmente, no formato pdf para serem analisados na
%triagem.

\subsubsection{Passo 2: Triagem}

% leitura dos pdf > apenas os papers relevantes que mencionam o nome do software

Os artigos encontrados no passo anterior serão inspecionados manualmente em
busca de confirmar se fazem referência de fato aos projetos de software,
os dados gerados neste passo são armazenados no próprio arquivo BibTeX criado
no passo anterior contendo os resultados únicos encontrados.

A inspeção tem como objetivo selecionar os artigos relevantes para este estudo,
mantendo apenas aqueles com menção ao nome do software, seja a menção em
formato de citação formal ou informal, cada resultado do arquivo BibTeX deve
ser atualizado com o resultado da triagem, indicando se faz referência ao
nome do projeto ou não.

%de cada artigo é realizada com o auxílio da funcionalidade de busca
%do leitor de pdf utilizado neste estudo\footnote{Utilizamos software Evince v3.22.1}
%utilizado para leitura dos artigos, com o auxílio da busca encontramos cada
%ocorrência ao nome do software tomando nota a confirmação sobre a menção
%encontrada.

%Esta informação foi armazenada no próprio arquivo BibTeX num campo adicional
%aos demais campos dos metadados do artigo; ao final, temos os metadados do
%artigo como, título, autores, ano de publicação, etc, e também uma indicação se
%o artigo cita realmente o software.

\subsubsection{Passo 3: Keywording}

% papers relevantes > esquema de tipos de menção

Nesta fase da revisão de literatura criamos o esquema de codificação para
classificação das menções, o esquema é criado a partir da identificação do
contexto em que os projetos são mencionados em cada artigo, para cada menção
toma-se nota sobre como o software é mencionado naquele artigo.

As notas são comentários livres resumindo as menções encontradas no artigo
sobre um determinado software, um artigo científico pode mencionar um software
diversas vezes, de diversas formas, desde uma simples menção nos trabalhos
relacionados até uma grande contribuição ao software.

O conjunto final de todas as notas são agrupadas e analisadas individualmente
com o objetivo de criar a partir dessas notas um conjunto mínimo de códigos
para classificação de menções a software acadêmico. Este esquema deve ser construído
como uma escala de pesos, onde o último tipo de menção inclui implicitamente os
tipos de menor peso.

%Esta leitura irá gerar
%uma escala de tipos de menção ao software. Cada artigo assume, em relação ao
%software, um valor nesta escala de tipos de menção.

%Foi lido em busca de encontrar menções ao nome do software, em qual contexto o
%software é mencionado e de que forma é mencionado, resume cada tipo de menção com explicação dos casos em
%que se enquadram, o método utilizado para.

\subsubsection{Passo 4: Extração}

A partir do esquema para classificação das menções extraímos para cada artigo
qual o tipo de menção ele faz ao software, quando o artigo menciona o software
diversas vezes adotamos o tipo de menção com maior peso, indicando que os pesos
menores estão incluídos.

%A partir da classificação , Tabela \ref{esquema-de-mencao}, extraímos de cada
%artigo mencionando os projetos de acordo com o esquema de codificação criado no

A extração desses dados devem ser armazenadas no próprio arquivo BibTeX, sendo
atualizado com a inclusão de um novo campo indicando o tipo de menção do artigo
ao software.

Ao final da revisão de literatura teremos para cada projeto de software um
conjunto de artigos com menções classificadas por tipo e contexto, estando
incluído nestes conjuntos ao menos os próprios artigos que deram origem ao
conjunto de projetos utilizados como objeto de estudo neste trabalho.

%menções a eles nas bases do ACM e IEEE, temos o tipo de cada menção, se
%contribuiu com o projeto, se foi avaliado, ou apenas citado como referência, os
%artigos selecionados no estudo anterior que deram origem ao conjunto de
%projetos de software estão incluídos nestes resultados.

% }}}

\section{Preparação} \label{estudo2:preparacao} % {{{

Criamos as strings de busca.

Implementamos scripts para análise dos dados.

Implementamos scripts para plotar dados.

% }}}

\section{Coleta de dados}

Seguindo o planejamento e preparação descritos nas seções
\ref{estudo2:planejamento} e \ref{estudo2:preparacao} iniciamos a coleta dos
dados através da revisão de literatura nas bases ACM e IEEE.

\subsection{Menções aos projetos de software de análise estática}

\newcommand{\SearchACMCount}{438}
\newcommand{\SearchIEEECount}{459}
\newcommand{\SearchCount}{897}
\newcommand{\SearchUniqueCount}{684
}
\newcommand{\ScreeningCount}{456}
\newcommand{\ScreeningUniqueCount}{360
}


\subsubsection{Passo 1: Busca}

Uma string de busca foi definida para cada software acadêmico selecionado.
Além do nome do software pesquisado, as strings de busca incluíram outras
características do software sempre que necessário.

Encontramos um total de \SearchCount \ resultados usando as strings de busca
para cada projeto nas bases ACM e IEEE, deste total \SearchACMCount \
resultados foram encontrados na base ACM e \SearchIEEECount \ foram encontrados
no IEEE.  Dentre o resultado total, \SearchUniqueCount \ são únicos.

Alguns artigos fazem menção a mais de 1 projeto, fizemos o download de cada
artigo ...

\subsubsection{Passo 2: Triagem}

Inspecionamos cada um dos \SearchUniqueCount \ artigos em busca de menções ao
nome do projeto de software associado com a busca que retornou estes artigos,
nesta busca utilizamos como critério qualquer ocorrencia ao nome do projeto de
software mas que seja realmente referente ao software em sí, alguns projetos de
software do nosso conjunto possui nomes comuns que foram encontrados nos
artigos mas não faziam referencia ao software, em alguns casos os nomes
apareciam em fórmulas matemáticas ou como parte de um outra palavra.

Ao final selecionadmos para cada projeto um conjunto de artigos que de fato
mencionam o software, ainda sem caracterizar o tipo de menção, apenas se
realmente menciona, ao total selecionamos \ScreeningCount \ menções.
Este total de menções foram encontrados ao meio a \ScreeningUniqueCount \ artigos.

%334 artigos fazem menção aos 60 projetos
%350 não faz menção aos projetos

\subsubsection{Passo 3: Keywording}

A avaliação das \ScreeningCount \ menções gerou uma escala de tipos de menção
ao software, detalhado na Tabela \ref{esquema-de-mencao}, com quatro valores
distintos, onde o último tipo de menção com maior valor inclui todos os demais.
Um tipo de menção com maior peso inclui implicitamente o tipo de menção de
menor peso, e assim sucessivamente.

\begin{table}[h]
\caption{Esquema para classificação de menções aos projetos software acadêmico.}
\centering
\begin{tabular}{ l p{10cm} }
  \hline
  Tipo de menção           & Explicação \\
  \hline
  Cita      & Apenas cita o software ou é o mesmo artigo onde o software selecionado; É um artigo com ``mesmo'' conteúdo publicado na ``mesma'' época; O artigo apenas descreve o software; Menciona o software numa tabela com outros, classifica; Menciona o software como exemplo; Menciona o software como trabalho relacionado; Menciona o software em trabalhos futuros. \\
  Usa       & Avalia ou caracteriza o software; Usa para coleta ou análise de dados; Usa como objeto de estudo; Usa o software como parte de uma solução, implementação, etc; Cria um software derivado mas não disponibiliza as contribuições. \\
  Contribui & Contribuição pequena ou moderada; Extende o software; Integra o software a outros sistemas, formatos de entrada/saída, APIs, etc (seja implementando suporte no software ou do outro lado); Refatora parte do software; Implementa parte do software em outro projeto e compara resultados. \\
  Cria      & Cria; Contribuição inicial criando o projeto; Faz uma grande contribuição; Refatora todo o software; Abre o código de um software que antes era de código fechado. \\
  \hline
\end{tabular}
\label{esquema-de-mencao}
\end{table}

%também o tipo de menção que é feita ao software: se apenas cita como exemplo,
%se contribui com novos algoritmos e técnicas, se avalia, ou apenas descreve o
%software comparando-o com outro software.

\subsubsection{Passo 4: Extração}

Coletamos as informações relacionadas a cada menção encontrada, cada um dos
\ScreeningUniqueCount \ artigos encontrados mencionando os projetos, coletamos
como o projeto é mencionado, encontramos \ScreeningCount \ menções aos
projetos neste conjunto de artigos.

Os dados coletados na revisão de literatura durante os quatro passos são
apresentados em resumo na Tabela \ref{literature-review-table} indicando o
número de resultados encontrados para cada projeto de software.


\begin{longtable}{ l c c c c c }
\caption{Resumo dos resultados encontrados na revisão de literatura.}
\label{literature-review-table} \\
  \hline
  \hhline{ l c c c c c |}
  \endfirsthead
  \hhline{ l c c c c c |}
  \hline
   \multirow{2}{*}{\textbf{Nome do software}} & \multicolumn{5}{c}{{\bf Menções}} \\
   & \textbf{Cita} & \textbf{Usa} & \textbf{Contribui} & \textbf{Cria} & \textbf{Total} \\
  \hline
  \hhline{ l c c c c c |}
  \endhead
  \hhline{------}
  \multicolumn{6}{c}{continua na próxima página} \\
  \hhline{------} \endfoot
  \hhline{------} \endlastfoot
   \multirow{2}{*}{\textbf{Nome do software}} & \multicolumn{5}{c}{{\bf Menções}} \\
   & \textbf{Cita} & \textbf{Usa} & \textbf{Contribui} & \textbf{Cria} & \textbf{Total} \\
  \hline
   2LS & 0 & 0 & 0 & 1 & 1 \\
   AccessAnalysis & 1 & 0 & 0 & 1 & 2 \\
   APIExample & 3 & 0 & 0 & 1 & 4 \\
   BEG & 4 & 3 & 1 & 1 & 9 \\
   ccJava & 1 & 0 & 3 & 1 & 5 \\
   CIVL & 5 & 0 & 0 & 1 & 6 \\
   CodeBoost & 14 & 0 & 0 & 1 & 15 \\
   CSL & 2 & 2 & 1 & 1 & 6 \\
   CPA+ & 0 & 0 & 4 & 1 & 5 \\
   CSeq & 1 & 2 & 1 & 1 & 5 \\
   DDVerify & 2 & 0 & 0 & 1 & 3 \\
   Derailer & 1 & 0 & 0 & 1 & 2 \\
   Diagnosys & 0 & 0 & 0 & 1 & 1 \\
   DOMPLETION & 1 & 0 & 0 & 1 & 2 \\
   DRC & 4 & 0 & 0 & 1 & 5 \\
   e-munity & 0 & 0 & 0 & 1 & 1 \\
   EJB & 1 & 1 & 0 & 1 & 3 \\
   Error Prone & 0 & 1 & 0 & 1 & 2 \\
   ESBMC & 16 & 18 & 7 & 1 & 42 \\
   ETXL & 0 & 0 & 0 & 1 & 1 \\
   FaultBuster & 0 & 0 & 0 & 1 & 1 \\
   Flowgen & 2 & 0 & 0 & 1 & 3 \\
   GRT & 5 & 2 & 1 & 2 & 10 \\
   GUIZMO & 0 & 0 & 0 & 1 & 1 \\
   GumTree & 6 & 11 & 1 & 1 & 19 \\
   HUSACCT & 3 & 2 & 0 & 2 & 7 \\
   Indus & 0 & 3 & 0 & 1 & 4 \\
   JastAdd & 25 & 15 & 4 & 1 & 45 \\
   JFlow & 5 & 1 & 0 & 1 & 7 \\
   JstereoCode & 2 & 5 & 0 & 1 & 8 \\
   Jtop & 0 & 1 & 0 & 1 & 2 \\
   Bogor/Kiasan & 10 & 0 & 5 & 1 & 16 \\
   Loopfrog & 2 & 2 & 0 & 1 & 5 \\
   Lotrack & 1 & 0 & 0 & 1 & 2 \\
   MPAnalyzer & 0 & 0 & 0 & 1 & 1 \\
   MSP & 1 & 0 & 0 & 1 & 2 \\
   mygcc & 3 & 1 & 1 & 2 & 7 \\
   PARSEWeb & 22 & 1 & 0 & 1 & 24 \\
   PAT & 2 & 0 & 0 & 1 & 3 \\
   PHP AiR & 1 & 5 & 2 & 1 & 9 \\
   protopurity & 0 & 0 & 0 & 1 & 1 \\
   Pseudogen & 0 & 0 & 0 & 1 & 1 \\
   PtYasm & 0 & 0 & 0 & 2 & 2 \\
   PuMoC & 1 & 0 & 0 & 1 & 2 \\
   PYTHIA & 0 & 0 & 1 & 1 & 2 \\
   ReAssert & 9 & 3 & 0 & 1 & 13 \\
   Rêve & 0 & 0 & 0 & 1 & 1 \\
   RRFinder & 2 & 0 & 0 & 1 & 3 \\
   Sapid/XML & 1 & 2 & 1 & 1 & 5 \\
   Sonar Qube Plug-in & 0 & 0 & 0 & 1 & 1 \\
   SPARTA & 1 & 2 & 0 & 1 & 4 \\
   srcML & 14 & 24 & 1 & 1 & 40 \\
   SWAT & 1 & 2 & 0 & 1 & 4 \\
   TACLE & 1 & 1 & 0 & 1 & 3 \\
   TEBA & 0 & 0 & 0 & 1 & 1 \\
   TestEra & 17 & 5 & 0 & 1 & 23 \\
   Vdiff & 4 & 0 & 0 & 1 & 5 \\
   WALA & 5 & 5 & 0 & 1 & 11 \\
   Wrangler & 16 & 10 & 6 & 1 & 33 \\
   XOgastan & 4 & 0 & 0 & 1 & 5 \\
  \hline
  {\bf Total} & 222 & 130 & 40 & 64 & 456 \\
\end{longtable}



\section{Análise dos Dados} % Raw results from the analysis

Dados de 60 projetos de software acadêmico de análise estática desenvolvidos e
publicados na literatura acadêmica de engenharia de software, informações sobre
diversas formas de menção ao nome destes projetos na literatura acadêmica,
o número de autores mencionando o software ao longo do tempo.

\section{Interpretação dos Resultados} % Hypothesis rejection

... a Tabela \ref{visibility-table} ...



\begin{longtable}{ l *{17}{c} }
\caption{Número de menções por ano (de 2001 até 2017).}
\label{authorship-table} \\
  \hline
  \hhline{ l *{17}{c} |}
  \endfirsthead
  \hhline{ l *{17}{c} |}
  \hline
  \textbf{Projeto} & 1 & 2 & 3 & 4 & 5 & 6 & 7 & 8 & 9 & 10 & 11 & 12 & 13 & 14 & 15 & 16 & 17 \\
  \hline
  \hhline{ l *{17}{c} |}
  \endhead
  \hhline{------------------}
  \multicolumn{17}{c}{continua na próxima página} \\
  \hhline{------------------} \endfoot
  \hhline{------------------} \endlastfoot
  \textbf{Projeto} & 1 & 2 & 3 & 4 & 5 & 6 & 7 & 8 & 9 & 10 & 11 & 12 & 13 & 14 & 15 & 16 & 17 \\
  \hline
    2LS & - & - & - & - & - & - & - & - & - & - & - & - & - & - & HASH(0x55bf7861a9e8) & - & - \\
    AccessAnalysis & - & - & - & - & - & - & - & - & - & - & - & HASH(0x55bf785f3850) & - & - & - & - & - \\
    APIExample & - & - & - & - & - & - & - & - & - & - & HASH(0x55bf785f7f60) & - & HASH(0x55bf785f8140) & - & - & HASH(0x55bf784c87c0) & - \\
    BEG & - & - & HASH(0x55bf78604500) & HASH(0x55bf78604698) & - & HASH(0x55bf784c58d0) & HASH(0x55bf786047a0) & - & - & HASH(0x55bf78604668) & - & - & - & - & HASH(0x55bf784c5960) & - & - \\
    ccJava & - & - & - & - & - & - & HASH(0x55bf78618b28) & HASH(0x55bf78619b18) & HASH(0x55bf784cc760) & HASH(0x55bf78619c68) & - & - & - & - & - & - & - \\
    CIVL & - & - & - & - & - & - & - & - & - & - & - & - & - & - & HASH(0x55bf7858a380) & - & HASH(0x55bf7858a578) \\
    CodeBoost & - & - & HASH(0x55bf78606f70) & - & HASH(0x55bf78607108) & HASH(0x55bf784c5e40) & - & HASH(0x55bf78607258) & HASH(0x55bf786072b8) & HASH(0x55bf78607318) & HASH(0x55bf78607378) & HASH(0x55bf786073d8) & HASH(0x55bf78607438) & - & HASH(0x55bf78607498) & - & - \\
    CSL & HASH(0x55bf786279e8) & - & - & HASH(0x55bf78627bb0) & HASH(0x55bf784dd4a8) & HASH(0x55bf78627cb8) & - & - & HASH(0x55bf78627d18) & - & - & - & - & - & - & - & - \\
    CPA+ & - & - & - & - & - & - & - & HASH(0x55bf785892a0) & - & HASH(0x55bf785894c8) & - & HASH(0x55bf784bc0b0) & HASH(0x55bf785895d0) & - & HASH(0x55bf78589630) & - & - \\
    CSeq & - & - & - & - & - & - & - & - & - & - & - & - & HASH(0x55bf7861c668) & - & HASH(0x55bf7861c8b0) & HASH(0x55bf784d9470) & - \\
    DDVerify & - & - & - & - & - & - & HASH(0x55bf785f8908) & HASH(0x55bf785f9910) & - & - & - & - & - & HASH(0x55bf784c8a30) & - & - & - \\
    Derailer & - & - & - & - & - & - & - & - & - & - & - & - & - & HASH(0x55bf7857d2f8) & - & HASH(0x55bf7857d508) & - \\
    Diagnosys & - & - & - & - & - & - & - & - & - & - & - & HASH(0x55bf78622ac0) & - & - & - & - & - \\
    DOMPLETION & - & - & - & - & - & - & - & - & - & - & - & - & - & HASH(0x55bf785f11d0) & HASH(0x55bf785f13b0) & - & - \\
    DRC & - & - & - & - & - & - & - & - & - & - & - & - & HASH(0x55bf78591368) & HASH(0x55bf78591530) & HASH(0x55bf784ba840) & - & - \\
    e-munity & - & - & - & - & - & - & - & - & - & - & - & - & - & HASH(0x55bf7862cfd8) & - & - & - \\
    EJB & - & - & - & - & - & - & - & - & - & - & HASH(0x55bf7860f870) & - & HASH(0x55bf7860fa08) & - & - & - & HASH(0x55bf784d1538) \\
    Error Prone & - & - & - & - & - & - & - & - & - & - & - & HASH(0x55bf78602428) & - & - & HASH(0x55bf78602680) & - & - \\
    ESBMC & - & - & - & - & - & - & - & - & HASH(0x55bf785e3e30) & HASH(0x55bf785e4028) & HASH(0x55bf784c17c8) & HASH(0x55bf785e4130) & HASH(0x55bf785e3ff8) & HASH(0x55bf785e4238) & HASH(0x55bf784c1858) & HASH(0x55bf785e3e90) & HASH(0x55bf784c18d0) \\
    ETXL & - & - & - & - & - & HASH(0x55bf785f53e8) & - & - & - & - & - & - & - & - & - & - & - \\
    FaultBuster & - & - & - & - & - & - & - & - & - & - & - & - & - & - & HASH(0x55bf78609230) & - & - \\
    Flowgen & - & - & - & - & - & - & - & - & - & - & - & - & - & HASH(0x55bf7861d0f0) & HASH(0x55bf7861d288) & HASH(0x55bf784d9710) & - \\
    GRT & - & - & - & - & - & - & - & - & - & - & - & - & - & HASH(0x55bf78625708) & HASH(0x55bf78625630) & HASH(0x55bf784dcf80) & HASH(0x55bf78625780) \\
    GUIZMO & - & - & - & - & - & - & - & - & - & HASH(0x55bf7862b2f0) & - & - & - & - & - & - & - \\
    GumTree & - & - & - & - & - & - & - & - & - & - & - & - & - & HASH(0x55bf785ff928) & HASH(0x55bf785ffb50) & HASH(0x55bf784cb058) & HASH(0x55bf785ff9b8) \\
    HUSACCT & - & - & - & - & - & - & - & - & - & - & - & - & - & HASH(0x55bf7857eff8) & HASH(0x55bf7857f190) & HASH(0x55bf7847a220) & - \\
    Indus & - & - & - & - & - & HASH(0x55bf78635c48) & - & - & HASH(0x55bf78635db0) & - & - & HASH(0x55bf784d8740) & - & - & - & - & - \\
    JastAdd & - & - & HASH(0x55bf78616598) & - & HASH(0x55bf78616748) & HASH(0x55bf784cc0e8) & HASH(0x55bf786165f8) & HASH(0x55bf784cc148) & HASH(0x55bf784cc0d0) & HASH(0x55bf78616718) & HASH(0x55bf78616b80) & HASH(0x55bf784cc250) & HASH(0x55bf784cc2e0) & HASH(0x55bf784cc328) & HASH(0x55bf784cc1d8) & HASH(0x55bf784cc3d0) & HASH(0x55bf78617dc0) \\
    JFlow & - & - & - & - & - & HASH(0x55bf78590558) & HASH(0x55bf78590768) & - & - & - & HASH(0x55bf784ba510) & - & HASH(0x55bf78590870) & HASH(0x55bf785908d0) & - & - & - \\
    JstereoCode & - & - & - & - & - & - & - & - & - & - & - & HASH(0x55bf78600d80) & HASH(0x55bf78600b28) & - & HASH(0x55bf784cb478) & HASH(0x55bf78600eb8) & HASH(0x55bf78601098) \\
    Jtop & - & - & - & - & - & - & - & - & HASH(0x55bf785f4e60) & - & - & - & - & HASH(0x55bf785f50d0) & - & - & - \\
    Bogor/Kiasan & - & - & - & - & - & HASH(0x55bf78620978) & HASH(0x55bf78620af8) & HASH(0x55bf784d9bd8) & HASH(0x55bf784d9c20) & HASH(0x55bf78620cf0) & - & - & HASH(0x55bf786209c0) & HASH(0x55bf78620df8) & HASH(0x55bf78620e58) & - & - \\
    Loopfrog & - & - & - & - & - & - & - & - & HASH(0x55bf7862ad98) & - & - & HASH(0x55bf7862af30) & HASH(0x55bf784dd9b8) & - & - & - & - \\
    Lotrack & - & - & - & - & - & - & - & - & - & - & - & - & - & HASH(0x55bf785f3160) & HASH(0x55bf785f3328) & - & - \\
    MPAnalyzer & - & - & - & - & - & - & - & - & - & - & - & - & - & HASH(0x55bf784dd058) & - & - & - \\
    MSP & - & - & - & - & - & - & - & - & HASH(0x55bf7858bd30) & HASH(0x55bf7858bee0) & - & - & - & - & - & - & - \\
    mygcc & - & - & - & - & - & HASH(0x55bf785f7258) & - & HASH(0x55bf785f73d8) & HASH(0x55bf784c84a8) & HASH(0x55bf785f74e0) & - & - & - & HASH(0x55bf785f7540) & - & - & - \\
    PARSEWeb & - & - & - & - & - & - & HASH(0x55bf7860de80) & HASH(0x55bf7860e078) & - & HASH(0x55bf784d11c0) & HASH(0x55bf7860dee0) & HASH(0x55bf7860e228) & HASH(0x55bf7860e288) & - & HASH(0x55bf7860e2e8) & HASH(0x55bf7860e348) & - \\
    PAT & - & - & - & - & - & - & - & - & - & - & - & HASH(0x55bf78623530) & - & - & - & HASH(0x55bf78623698) & HASH(0x55bf784dcba8) \\
    PHP AiR & - & - & - & - & - & - & - & - & - & - & - & - & - & HASH(0x55bf78633bc0) & HASH(0x55bf78633d40) & HASH(0x55bf784d8248) & HASH(0x55bf784d8290) \\
    protopurity & - & - & - & - & - & - & - & - & - & - & - & - & - & - & HASH(0x55bf785e79a8) & - & - \\
    Pseudogen & - & - & - & - & - & - & - & - & - & - & - & - & - & - & HASH(0x55bf78631c38) & - & - \\
    PtYasm & - & - & - & - & - & - & - & HASH(0x55bf785e4b08) & - & - & - & - & - & - & - & - & - \\
    PuMoC & - & - & - & - & - & - & - & - & - & - & - & HASH(0x55bf78580838) & HASH(0x55bf78580a30) & - & - & - & - \\
    PYTHIA & - & - & - & - & - & - & - & - & - & - & - & - & HASH(0x55bf785f9eb0) & HASH(0x55bf785fa0a8) & - & - & - \\
    ReAssert & - & - & - & - & - & - & - & - & HASH(0x55bf785f0740) & HASH(0x55bf785f0938) & HASH(0x55bf784bec70) & HASH(0x55bf785f0a40) & HASH(0x55bf785f0908) & - & - & HASH(0x55bf784bed00) & - \\
    Rêve & - & - & - & - & - & - & - & - & - & - & - & - & - & HASH(0x55bf78609860) & - & - & - \\
    RRFinder & - & - & - & - & - & - & - & - & - & - & HASH(0x55bf785f1b90) & - & - & HASH(0x55bf785f1d40) & HASH(0x55bf784bf1e0) & - & - \\
    Sapid/XML & - & - & - & HASH(0x55bf78622538) & HASH(0x55bf786226a0) & HASH(0x55bf784d9ff8) & - & - & - & - & - & HASH(0x55bf78622568) & - & - & - & - & - \\
    Sonar Qube Plug-in & - & - & - & - & - & - & - & - & - & - & - & - & - & HASH(0x55bf78634238) & - & - & - \\
    SPARTA & - & HASH(0x55bf785e64f8) & - & - & - & - & - & - & - & - & - & - & - & - & HASH(0x55bf785e6660) & - & HASH(0x55bf784c1e58) \\
    srcML & - & HASH(0x55bf78587020) & HASH(0x55bf785871e8) & HASH(0x55bf7847ae20) & HASH(0x55bf785872f0) & - & HASH(0x55bf785871b8) & HASH(0x55bf7847aeb0) & HASH(0x55bf785874a0) & HASH(0x55bf78587500) & HASH(0x55bf78587080) & HASH(0x55bf78587608) & HASH(0x55bf7847aef8) & HASH(0x55bf78587710) & HASH(0x55bf7847ae08) & HASH(0x55bf78587818) & HASH(0x55bf784bbc48) \\
    SWAT & - & - & - & - & - & - & - & - & - & - & HASH(0x55bf7861a460) & HASH(0x55bf7861a610) & HASH(0x55bf784cca60) & HASH(0x55bf7861a718) & - & - & - \\
    TACLE & - & - & - & - & HASH(0x55bf7862bc68) & HASH(0x55bf7862cc78) & - & - & - & - & - & - & - & - & - & - & - \\
    TEBA & - & - & - & - & - & - & - & - & - & - & - & - & - & HASH(0x55bf78629e48) & - & - & - \\
    TestEra & HASH(0x55bf78631100) & HASH(0x55bf786312c8) & - & HASH(0x55bf784d4b68) & HASH(0x55bf786313d0) & HASH(0x55bf786312b0) & HASH(0x55bf786314d8) & HASH(0x55bf78631538) & HASH(0x55bf78631190) & HASH(0x55bf784d4ca0) & HASH(0x55bf786316e8) & HASH(0x55bf78631770) & HASH(0x55bf786317d0) & HASH(0x55bf78631830) & - & - & - \\
    Vdiff & - & - & - & - & - & - & - & - & - & HASH(0x55bf785fb940) & HASH(0x55bf785fbb20) & - & - & - & HASH(0x55bf784caab8) & HASH(0x55bf785fbc28) & - \\
    WALA & - & - & - & - & - & - & - & HASH(0x55bf7858e130) & HASH(0x55bf7858e340) & HASH(0x55bf784bca28) & HASH(0x55bf7858e448) & HASH(0x55bf7858e4a8) & HASH(0x55bf7858e508) & - & - & HASH(0x55bf7858e1d8) & HASH(0x55bf7858e610) \\
    Wrangler & - & - & - & - & - & HASH(0x55bf785ec928) & - & HASH(0x55bf785ecac0) & HASH(0x55bf784be580) & HASH(0x55bf785ec970) & HASH(0x55bf784be568) & HASH(0x55bf785eca90) & HASH(0x55bf785ece50) & HASH(0x55bf784be688) & HASH(0x55bf784be718) & HASH(0x55bf784be760) & HASH(0x55bf784be7a8) \\
    XOgastan & - & - & HASH(0x55bf784addd0) & HASH(0x55bf7857ccb0) & - & HASH(0x55bf78479bf8) & - & HASH(0x55bf7857cdb8) & - & - & HASH(0x55bf7857ce18) & - & - & - & - & - & - \\
  \hline
\end{longtable}



\subsection{Q1 - \EstudoDoisQuestaoUm}

% Demográfica.


Entre os 60 projetos de software estudados 14 não foram encontrados na revisão
de literatura em publicações após a publicação inicial, tendo apenas uma
publicação cada um, alguns tendo duas publicações na mesma época sendo uma
publicação específica numa trilha de ferramenta, ou senndo 2 artigos
semelhantes publicados na mesma época em conferências distintas, mas sendo os
mesmos autores.

\begin{enumerate}
  \item 2LS
  \item AccessAnalysis
  \item Diagnosys
  \item e-munity
  \item ETXL
  \item FaultBuster
  \item GUIZMO
  \item MPAnalyzer
  \item protopurity
  \item Pseudogen
  \item PtYasm
  \item Rêve
  \item SonarQube Plugin
  \item TEBA
\end{enumerate}

Nenhum destes projetos tiveram menção após a publicação inicial do projeto,
entre estes 14 projetos, apenas três (Diagnosys, ETXL e Rêve) estão
indisponíveis para download, os outros 11 estão disponíveis, todos com código
fonte, exceto o FaultBuster que está disponível apenas em formato binário.

46 projetos são mencionados após a publicação inicial encontrada na literatura.

Ao todo foram encontradas 465 menções aos 60 projetos de software, estas
menções estão num conjunto de 334 artigos selecionados na revisão de literatura
nas bases ACM e IEEE.



Entre todas as menções a maior parte (222) apenas cita os projetos de software
como exemplo de implementação, apenas cita o software ou é o mesmo artigo onde
o software selecionado; É um artigo com “mesmo” conteúdo publicado na “mesma”
época; O artigo apenas descreve o software; Menciona o software numa tabela com
outros, classifica; Menciona o software como exemplo; Menciona o software como
trabalho relacionado; Menciona o software em trabalhos futuros.

% ./bin/mentions dataset/software/*/citations.bib contribution_weight=0.1 | grep " title = " | wc -l

AS 222 menções foram encontradas em 204 artigos distintos.

\subsection{Q2 - \EstudoDoisQuestaoDois}

Entre as 465 menções encontradas, 130 são menções que avalia ou caracteriza o
software; Usa para coleta ou análise de dados; Usa como objeto de estudo; Usa o
software como parte de uma solução, implementação, etc; Cria um software
derivado mas não disponibiliza as contribuições.

Entre os 60 projetos de software, 27 são mencionados entre estas 130 menções
encontradas.

% ./bin/mentions dataset/software/*/citations.bib contribution_weight=0.25

Entre as 130 menções, foram encontradas entre 125 artigos distintos.

\subsection{Q3 - \EstudoDoisQuestaoTres}

Número de menções a cada software acadêmico e qual o contexto e tipo de menção
é feita, quem e quantos são os autores de cada menção.

% ./bin/mentions dataset/software/*/citations.bib contribution_weight=0.5

40 menções contribuindo com os projetos foram encontrados, cada menção
encontrado num artigo distinto. Estas 40 menções são em relação a apenas
16 projetos entre os 60 do conjunto total, apenas 16 projetos receberam
contribuções advindas de pesquisas publicadas nas bases ACM e IEEE.

\subsection{Q4 - ???}

Entre os 60 projetos de software selecionados no estudo anterior, a grande maioria
foi encontrado em artigos 

O GRT recebe uma grande contribuição além da publicação inicial criando o projeto,
é o único, todos os outros recebem contribuições de peso apenas na criação, ou seja,
no paper original publicando o projeto pela primeira vez.

\subsection{Q5 - ???}

Os projetos tiveram contribuições em código em publicações posteriores aos
artigos inicias publicando o software?

Quantos autores novos além dos autores originais/iniciais do projeto
foram encontrados contribuindo com os projetos?



\begin{longtable}{ l *{17}{c} }
\caption{Número de publicações mencionando contribuição por ano (de 2001 até 2017).}
\label{contributors-table} \\
  \hline
  \hhline{ l *{17}{c} |}
  \endfirsthead
  \hhline{ l *{17}{c} |}
  \hline
  \textbf{Projeto} & 01 & 02 & 03 & 04 & 05 & 06 & 07 & 08 & 09 & 10 & 11 & 12 & 13 & 14 & 15 & 16 & 17 \\
  \hline
  \hhline{ l *{17}{c} |}
  \endhead
  \hhline{------------------}
  \multicolumn{17}{c}{continua na próxima página} \\
  \hhline{------------------} \endfoot
  \hhline{------------------} \endlastfoot
  \textbf{Projeto} & 01 & 02 & 03 & 04 & 05 & 06 & 07 & 08 & 09 & 10 & 11 & 12 & 13 & 14 & 15 & 16 & 17 \\
  \hline
    2LS &
      - &
      - &
      - &
      - &
      - &
      - &
      - &
      - &
      - &
      - &
      - &
      - &
      - &
      - &
      c &
      - &
      - \\
    AccessAnalysis &
      - &
      - &
      - &
      - &
      - &
      - &
      - &
      - &
      - &
      - &
      - &
      c &
      - &
      - &
      - &
      - &
      - \\
    APIExample &
      - &
      - &
      - &
      - &
      - &
      - &
      - &
      - &
      - &
      - &
      c &
      - &
      m &
      - &
      - &
      m &
      - \\
    BEG &
      - &
      - &
      c &
      u &
      - &
      u &
      u &
      - &
      - &
      c &
      - &
      - &
      - &
      - &
      m &
      - &
      - \\
    ccJava &
      - &
      - &
      - &
      - &
      - &
      - &
      c &
      c &
      c &
      c &
      - &
      - &
      - &
      - &
      - &
      - &
      - \\
    CIVL &
      - &
      - &
      - &
      - &
      - &
      - &
      - &
      - &
      - &
      - &
      - &
      - &
      - &
      - &
      c &
      - &
      m \\
    CodeBoost &
      - &
      - &
      c &
      - &
      m &
      m &
      - &
      m &
      m &
      m &
      m &
      m &
      m &
      - &
      m &
      - &
      - \\
    CSL &
      c &
      - &
      - &
      m &
      m &
      c &
      - &
      - &
      u &
      - &
      - &
      - &
      - &
      - &
      - &
      - &
      - \\
    CPA+ &
      - &
      - &
      - &
      - &
      - &
      - &
      - &
      c &
      - &
      c &
      - &
      c &
      c &
      - &
      c &
      - &
      - \\
    CSeq &
      - &
      - &
      - &
      - &
      - &
      - &
      - &
      - &
      - &
      - &
      - &
      - &
      c &
      - &
      c &
      u &
      - \\
    DDVerify &
      - &
      - &
      - &
      - &
      - &
      - &
      c &
      m &
      - &
      - &
      - &
      - &
      - &
      m &
      - &
      - &
      - \\
    Derailer &
      - &
      - &
      - &
      - &
      - &
      - &
      - &
      - &
      - &
      - &
      - &
      - &
      - &
      c &
      - &
      m &
      - \\
    Diagnosys &
      - &
      - &
      - &
      - &
      - &
      - &
      - &
      - &
      - &
      - &
      - &
      c &
      - &
      - &
      - &
      - &
      - \\
    DOMPLETION &
      - &
      - &
      - &
      - &
      - &
      - &
      - &
      - &
      - &
      - &
      - &
      - &
      - &
      c &
      m &
      - &
      - \\
    DRC &
      - &
      - &
      - &
      - &
      - &
      - &
      - &
      - &
      - &
      - &
      - &
      - &
      c &
      m &
      m &
      - &
      - \\
    e-munity &
      - &
      - &
      - &
      - &
      - &
      - &
      - &
      - &
      - &
      - &
      - &
      - &
      - &
      c &
      - &
      - &
      - \\
    EJB &
      - &
      - &
      - &
      - &
      - &
      - &
      - &
      - &
      - &
      - &
      c &
      - &
      m &
      - &
      - &
      - &
      u \\
    Error Prone &
      - &
      - &
      - &
      - &
      - &
      - &
      - &
      - &
      - &
      - &
      - &
      c &
      - &
      - &
      u &
      - &
      - \\
    ESBMC &
      - &
      - &
      - &
      - &
      - &
      - &
      - &
      - &
      c &
      u &
      c &
      u &
      c &
      c &
      c &
      c &
      u \\
    ETXL &
      - &
      - &
      - &
      - &
      - &
      c &
      - &
      - &
      - &
      - &
      - &
      - &
      - &
      - &
      - &
      - &
      - \\
    FaultBuster &
      - &
      - &
      - &
      - &
      - &
      - &
      - &
      - &
      - &
      - &
      - &
      - &
      - &
      - &
      c &
      - &
      - \\
    Flowgen &
      - &
      - &
      - &
      - &
      - &
      - &
      - &
      - &
      - &
      - &
      - &
      - &
      - &
      c &
      m &
      m &
      - \\
    GRT &
      - &
      - &
      - &
      - &
      - &
      - &
      - &
      - &
      - &
      - &
      - &
      - &
      - &
      m &
      c &
      c &
      u \\
    GUIZMO &
      - &
      - &
      - &
      - &
      - &
      - &
      - &
      - &
      - &
      c &
      - &
      - &
      - &
      - &
      - &
      - &
      - \\
    GumTree &
      - &
      - &
      - &
      - &
      - &
      - &
      - &
      - &
      - &
      - &
      - &
      - &
      - &
      c &
      u &
      u &
      c \\
    HUSACCT &
      - &
      - &
      - &
      - &
      - &
      - &
      - &
      - &
      - &
      - &
      - &
      - &
      - &
      c &
      u &
      c &
      - \\
    Indus &
      - &
      - &
      - &
      - &
      - &
      c &
      - &
      - &
      u &
      - &
      - &
      u &
      - &
      - &
      - &
      - &
      - \\
    JastAdd &
      - &
      - &
      m &
      - &
      u &
      u &
      c &
      c &
      m &
      c &
      u &
      u &
      c &
      u &
      m &
      u &
      m \\
    JFlow &
      - &
      - &
      - &
      - &
      - &
      m &
      u &
      - &
      - &
      - &
      m &
      - &
      c &
      m &
      - &
      - &
      - \\
    JstereoCode &
      - &
      - &
      - &
      - &
      - &
      - &
      - &
      - &
      - &
      - &
      - &
      c &
      u &
      - &
      u &
      u &
      m \\
    Jtop &
      - &
      - &
      - &
      - &
      - &
      - &
      - &
      - &
      c &
      - &
      - &
      - &
      - &
      u &
      - &
      - &
      - \\
    Bogor/Kiasan &
      - &
      - &
      - &
      - &
      - &
      c &
      c &
      m &
      c &
      m &
      - &
      - &
      m &
      c &
      m &
      - &
      - \\
    Loopfrog &
      - &
      - &
      - &
      - &
      - &
      - &
      - &
      - &
      c &
      - &
      - &
      m &
      u &
      - &
      - &
      - &
      - \\
    Lotrack &
      - &
      - &
      - &
      - &
      - &
      - &
      - &
      - &
      - &
      - &
      - &
      - &
      - &
      c &
      m &
      - &
      - \\
    MPAnalyzer &
      - &
      - &
      - &
      - &
      - &
      - &
      - &
      - &
      - &
      - &
      - &
      - &
      - &
      c &
      - &
      - &
      - \\
    MSP &
      - &
      - &
      - &
      - &
      - &
      - &
      - &
      - &
      m &
      c &
      - &
      - &
      - &
      - &
      - &
      - &
      - \\
    mygcc &
      - &
      - &
      - &
      - &
      - &
      c &
      - &
      c &
      m &
      u &
      - &
      - &
      - &
      m &
      - &
      - &
      - \\
    PARSEWeb &
      - &
      - &
      - &
      - &
      - &
      - &
      c &
      u &
      - &
      m &
      m &
      m &
      m &
      - &
      m &
      m &
      - \\
    PAT &
      - &
      - &
      - &
      - &
      - &
      - &
      - &
      - &
      - &
      - &
      - &
      c &
      - &
      - &
      - &
      m &
      m \\
    PHP AiR &
      - &
      - &
      - &
      - &
      - &
      - &
      - &
      - &
      - &
      - &
      - &
      - &
      - &
      c &
      c &
      u &
      u \\
    protopurity &
      - &
      - &
      - &
      - &
      - &
      - &
      - &
      - &
      - &
      - &
      - &
      - &
      - &
      - &
      c &
      - &
      - \\
    Pseudogen &
      - &
      - &
      - &
      - &
      - &
      - &
      - &
      - &
      - &
      - &
      - &
      - &
      - &
      - &
      c &
      - &
      - \\
    PtYasm &
      - &
      - &
      - &
      - &
      - &
      - &
      - &
      c &
      - &
      - &
      - &
      - &
      - &
      - &
      - &
      - &
      - \\
    PuMoC &
      - &
      - &
      - &
      - &
      - &
      - &
      - &
      - &
      - &
      - &
      - &
      c &
      m &
      - &
      - &
      - &
      - \\
    PYTHIA &
      - &
      - &
      - &
      - &
      - &
      - &
      - &
      - &
      - &
      - &
      - &
      - &
      c &
      c &
      - &
      - &
      - \\
    ReAssert &
      - &
      - &
      - &
      - &
      - &
      - &
      - &
      - &
      c &
      u &
      u &
      u &
      m &
      - &
      - &
      m &
      - \\
    Rêve &
      - &
      - &
      - &
      - &
      - &
      - &
      - &
      - &
      - &
      - &
      - &
      - &
      - &
      c &
      - &
      - &
      - \\
    RRFinder &
      - &
      - &
      - &
      - &
      - &
      - &
      - &
      - &
      - &
      - &
      c &
      - &
      - &
      m &
      m &
      - &
      - \\
    Sapid/XML &
      - &
      - &
      - &
      c &
      c &
      u &
      - &
      - &
      - &
      - &
      - &
      m &
      - &
      - &
      - &
      - &
      - \\
    Sonar Qube Plug-in &
      - &
      - &
      - &
      - &
      - &
      - &
      - &
      - &
      - &
      - &
      - &
      - &
      - &
      c &
      - &
      - &
      - \\
    SPARTA &
      - &
      u &
      - &
      - &
      - &
      - &
      - &
      - &
      - &
      - &
      - &
      - &
      - &
      - &
      c &
      - &
      u \\
    srcML &
      - &
      m &
      m &
      u &
      u &
      - &
      m &
      m &
      u &
      u &
      c &
      u &
      u &
      u &
      u &
      c &
      u \\
    SWAT &
      - &
      - &
      - &
      - &
      - &
      - &
      - &
      - &
      - &
      - &
      c &
      u &
      m &
      u &
      - &
      - &
      - \\
    TACLE &
      - &
      - &
      - &
      - &
      u &
      c &
      - &
      - &
      - &
      - &
      - &
      - &
      - &
      - &
      - &
      - &
      - \\
    TEBA &
      - &
      - &
      - &
      - &
      - &
      - &
      - &
      - &
      - &
      - &
      - &
      - &
      - &
      c &
      - &
      - &
      - \\
    TestEra &
      c &
      m &
      - &
      u &
      m &
      m &
      u &
      u &
      m &
      m &
      m &
      m &
      m &
      m &
      - &
      - &
      - \\
    Vdiff &
      - &
      - &
      - &
      - &
      - &
      - &
      - &
      - &
      - &
      c &
      m &
      - &
      - &
      - &
      m &
      m &
      - \\
    WALA &
      - &
      - &
      - &
      - &
      - &
      - &
      - &
      u &
      u &
      c &
      u &
      u &
      m &
      - &
      - &
      u &
      m \\
    Wrangler &
      - &
      - &
      - &
      - &
      - &
      u &
      - &
      u &
      c &
      c &
      c &
      c &
      u &
      c &
      u &
      u &
      m \\
    XOgastan &
      - &
      - &
      c &
      m &
      - &
      m &
      - &
      m &
      - &
      - &
      m &
      - &
      - &
      - &
      - &
      - &
      - \\
  \hline
\end{longtable}




A comparação entre os nomes do autores passou antes pela normalização
no formato de representação, visto que cada artigo utiliza um formato
de nome diferente, transformamos, por exemplo, ``Bajaj, Kon'' em ``Bajaj K.'',
``Costa, Kim A.'' em ``Costa K. A.'', ``Rajan, Sreeranga P.'' em ``Rajan S. P.'',
``Pol, Jaco van de'' em ``Pol J. van de'', ``Zijiang Yang'' em ``Yang Z.'',
e assim comparamos dois nomes considerando que cada um destes nomes representa,
de forma única, um autor.

\subsection{Ameaças à validade}

\section{Conclusões}

FALTA uma síntese aqui. 
Este estudo ...
Resultados mostram que ...
Algumas tendências emergiram a partir da leitura ...

Planejamos fazer outro estudo ... 


% o artigo com resumo do RESER 2011 diz \cite{knutson2010report}:
% 4) Re-
% search tools are either not available or not usable, so precise
% replication is impractical [1, 2, 8, 18, 19].

%Acesso ao software
%
%5º Princípio da citação ao softwares, Acessibilidade:
%
%``citações aos softwares devem permitir e facilitar acesso ao software,
%metadados, documentação, dados e outros materiais necessários tanto
%para humanos quanto para máquinas se informar do referido software''
%
%Não significa que o software deva estar disponível gratuitamente, mas que
%os metadados devem prover informação suficiente para que o software seja
%acessado. Se o software é livre, os metadados devem prover um identificador
%que pode ser resolvido para uma URL apontando para a versão específica
%do software sendo citado.
%
%Pra softwares comerciais, os metadados devem ainda prover informações sobre
%como acessa o software, mas pode ser um número de telefone da empresa que
%vende o software ou o link para um site que venda o software
%
%\cite{smith2016software}
%
%5. Accessibility: Software citations should facilitate access to the software itself and to its


%\subsection{Autoria das menções aos projetos de software de análise estática}

%das bibliotecas digitais, temos para cada um dos artigos todos os seus
%metadados,. Os autores de
%cada uma das menções ao software, por exemplo, serão utilizados na fase de
%análise para calcular o quanto de autores novos começaram a publicar sobre
%certo software acadêmico.

%Os dados coletados sobre as menções a cada software foram acrescentados com uma
%nova informação calculada a partir da autoria de cada artigo mencionando o
%software, estamos considerando que todos os autores de um certo artigo tem o
%mesmo peso em relação a menção ao software naquele estudo, mesmo sabendo que não
%é raro que cientistas trabalhando em conjunto apresentem níveis diferentes de
%domínio sobre cada parte da pesquisa.

%Os autores originais do primeiro artigo publicando sobre software, na maior
%parte dos casos é o mesmo paper selecionado no estudo anterior que deu origem
%ao conjunto de projetos, são considerados os autores originais do projeto, a
%partir desta primeira publicação cada artigo mencionando o software indica a
%entrada de novos atores no ecossistema daquele software.

%Foi coletado então para cada artigo em relação aos autores o quanto são
%autores novos, e classificamos em termos de todos os autores do estudo
%são novos, se todos já publicaram sobre o software anteriormente, ou se
%nenhum dos autores jamais publicou sobre aquele software anteriormente, A Tabela
%\ref{esquema-de-autoria} apresenta em detalhes este esquema.

%representado quantos novos atores foram incluídos
%no ecossistema daquele software, isto foi feito comparando o conjunto de autores
%da publicaçao com o conjunto acumulado de todos os autores anteriores, 
%podendo assumir um dos valores da Tabela \ref{coding-scheme-author}.

%\begin{table}[h]
%\caption{Esquema para classificação de autoria de menções aos projetos de software acadêmico.}
%\centering
%\begin{tabular}{ l c p{8cm} }
%  \hline
%  Novos atores no ecossistema & Peso & Explicação \\
%  \hline
%  Nenhum    & 0.5  & Nenhum dos autores jamais publicou sobre o software \\
%  Parte     & 0.25 & Uma parte dos autores já publicou sobre o software em anos anteriores \\
%  Todos     & 0.1  & Todos os autores já publicaram sobre o software em anos anteriores \\
%  Criadores & 0    & São os primeiros autores a publicar sobre o software \\
%  \hline
%\end{tabular}
%\label{esquema-de-autoria}
%\end{table}

%Estes dados foram calculados com base nos metadados já coletados anteriormente
%dispníveis nos arquivos BibTeX com as menções à cada projeto, tratamos os nomes
%dos autores para utilizar o mesmo formato, já que os artigos variam em relação
%ao padrão de nomes dos autores.

%Com os nomes dos autores normalizados comparamos e caso tenham a mesma string
%consideramos que trata-se de um mesmo pesquisador, os nomes iguals são
%considerados como sendo a mesma pessoa. Avaliamos cada artigo em ordem
%cronológica comparando os autores com relação a todos os autores anteriores.

%Este processo foi realizado automaticamente com um script escrito durante este
%trabalho de pesquisa, os dados resultantes são armazenados de volta nos
%arquivos BibTeX.

%\subsection{Autoria das menções aos projetos de software de análise estática}
%

% nao vou mais analisar nenhum dado de autoria, apenas numero de citacoes e tipo!!!
%
\begin{longtable}{ l c c }
\caption{Número de autores únicos mencionando os projetos de software.}
\label{authorship-table} \\
  \hline
  \hhline{ l c c |}
  \endfirsthead
  \hhline{ l c c |}
  \hline
  \textbf{Nome do software} & {\bf Autores únicos} & {\bf Menções} \\
  \hline
  \hhline{ l c c |}
  \endhead
  \hhline{---}
  \multicolumn{3}{c}{continua na próxima página} \\
  \hhline{---} \endfoot
  \hhline{---} \endlastfoot
  \textbf{Nome do software} & {\bf Autores únicos} & {\bf Menções} \\
  \hline
   2LS & 5 & 1 \\
   AccessAnalysis & 2 & 2 \\
   APIExample & 15 & 4 \\
   BEG & 11 & 9 \\
   ccJava & 6 & 5 \\
   CIVL & 24 & 6 \\
   CodeBoost & 36 & 15 \\
   CSL & 12 & 6 \\
   CPA+ & 11 & 5 \\
   CSeq & 13 & 5 \\
   DDVerify & 9 & 3 \\
   Derailer & 2 & 2 \\
   Diagnosys & 4 & 1 \\
   DOMPLETION & 3 & 2 \\
   DRC & 10 & 5 \\
   e-munity & 5 & 1 \\
   EJB Interceptor Analyzer & 8 & 3 \\
   Error Prone & 9 & 2 \\
   ESBMC & 119 & 42 \\
   ETXL & 2 & 1 \\
   FaultBuster & 5 & 1 \\
   Flowgen & 8 & 3 \\
   GRT & 25 & 10 \\
   GUIZMO & 3 & 1 \\
   GumTree & 68 & 19 \\
   HUSACCT & 10 & 7 \\
   Indus & 7 & 4 \\
   JastAdd & 107 & 45 \\
   JFlow & 21 & 7 \\
   JstereoCode & 24 & 9 \\
   Jtop & 5 & 3 \\
   Bogor/Kiasan & 43 & 16 \\
   Loopfrog & 14 & 6 \\
   Lotrack & 5 & 2 \\
   MPAnalyzer & 2 & 1 \\
   MSP & 5 & 2 \\
   mygcc & 21 & 7 \\
   PARSEWeb & 73 & 24 \\
   PAT & 7 & 4 \\
   PHP AiR & 8 & 9 \\
   protopurity & 4 & 1 \\
   Pseudogen & 8 & 1 \\
   PtYasm & 5 & 2 \\
   PuMoC & 2 & 3 \\
   PYTHIA & 3 & 3 \\
   ReAssert & 38 & 14 \\
   Rêve & 5 & 1 \\
   RRFinder & 12 & 4 \\
   Sapid/XML & 5 & 5 \\
   Sonar Qube Plug-in & 5 & 1 \\
   SPARTA & 15 & 4 \\
   srcML & 99 & 40 \\
   SWAT & 6 & 4 \\
   TACLE & 3 & 3 \\
   TEBA & 2 & 1 \\
   TestEra & 39 & 24 \\
   Vdiff & 15 & 5 \\
   WALA & 37 & 11 \\
   Wrangler & 61 & 33 \\
   XOgastan & 17 & 5 \\
  \hline
  {\bf Total} & 983 & 465 \\
\end{longtable}



