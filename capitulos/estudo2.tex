\xchapter{Visibilidade dos projetos de software acadêmico de análise estática}
{Este capítulo apresenta uma revisão de literatura nas bases da ACM e IEEE em
busca de menções aos projetos de software acadêmico de análise estática
publicados nas conferências ASE e SCAM até o ano de 2015.}

Este estudo avaliou como os projetos de software acadêmico de análise estática
publicados nas conferências de Engenharia de Software ASE e SCAM são
mencionados na literatura acadêmica através de uma revisão de literatura nas
bases da ACM e IEEE.

A seção ? apresenta ...

\section{Introdução e Motivação}

(pendente)

\section{Fundamentação} \label{estudo2:fundamentacao} % {{{

(pendente)

%Acesso ao software
%
%5º Princípio da citação ao softwares, Acessibilidade:
%
%``citações aos softwares devem permitir e facilitar acesso ao software,
%metadados, documentação, dados e outros materiais necessários tanto
%para humanos quanto para máquinas se informar do referido software''
%
%Não significa que o software deva estar disponível gratuitamente, mas que
%os metadados devem prover informação suficiente para que o software seja
%acessado. Se o software é livre, os metadados devem prover um identificador
%que pode ser resolvido para uma URL apontando para a versão específica
%do software sendo citado.
%
%Pra softwares comerciais, os metadados devem ainda prover informações sobre
%como acessa o software, mas pode ser um número de telefone da empresa que
%vende o software ou o link para um site que venda o software
%
%\cite{smith2016software}
%
%5. Accessibility: Software citations should facilitate access to the software itself and to its

% }}}

\section{Definição} \label{estudo2:definicao} % {{{

Como os projetos de software de análise estática desenvolvidos e publicados nas
conferências de Engenharia de Software ASE e SCAM são mencionados em
publicações nas bases da ACM e IEEE?

\subsection{Definição do Objetivo}

\begin{description}
\item{\bf Objeto de estudo.} 
O objeto de estudo são projetos de software publicados nas conferências ASE e SCAM.

\item{\bf Propósito.} 
O propósito deste estudo é caracterizar a visibilidade acadêmica dos projetos.

\item{\bf Perspectiva.} 
A perspectiva considerada é a de cientistas ...

\item{\bf Foco de qualidade.} 
O principal aspecto de qualidade estudado é o reconhecimento dos projetos.

\item{\bf Contexto.} 
O estudo foi conduzido com publicações das bases ACM e IEEE.
\end{description}

\subsection{Sumário da Definição}

Analisar os \textit{projetos de software acadêmico de análise estática}
com o propósito de \textit{caracterizar}
com respeito ao \textit{reconhecimento por artigos publicados nas bases da ACM e IEEE}
na perspectiva do \textit{cientistas...}
no contexto de \textit{software acadêmico de análise estática publicado nas conferências ASE e SCAM}.

\subsection{Questões de Pesquisa}

Neste estudo as seguintes questões de pesquisa, a respeito dos projetos de
software acadêmico de análise estática, serão investigadas:

\newcommand{\EstudoDoisQuestaoUm}{Como os projetos de software acadêmico de
análise estática publicados nas conferências ASE e SCAM são mencionados em
publicações das bases ACM e IEEE?}

\newcommand{\EstudoDoisQuestaoDois}{Os projetos de software acadêmico de
análise estática publicados nas conferências ASE e SCAM são avaliados em
estudos encontrados nas bases ACM e IEEE?}

\newcommand{\EstudoDoisQuestaoTres}{Os projetos de software acadêmico de
análise estática publicados nas conferências ASE e SCAM recebem contribuições
advindas de outros pesquisadores além dos autores originais?}

\newcommand{\EstudoDoisQuestaoQuatro}{
...
}

\begin{description}
  \item [Q1:] \EstudoDoisQuestaoUm
  \item [Q2:] \EstudoDoisQuestaoDois
  \item [Q3:] \EstudoDoisQuestaoTres
  \item [Q4:] \EstudoDoisQuestaoQuatro
\end{description}

\subsection{Métricas}

Para responder às questões de pesquisas, as seguintes métricas serão usadas:

\begin{enumerate}
  \item Número de menções aos projetos de software nas bases ACM e IEEE
  \item Número de autores publicando sobre os projetos nas bases da ACM e IEEE
\end{enumerate}

% }}}

\section{Planejamento do Estudo}

O estudo foi realizado a partir de uma revisão de literatura nas bases da ACM e
IEEE por ocorrências de estudos mencionando os projetos de software acadêmico
de análise estática selecionados no estudo anterior, Capítulo \ref{estudo1}, os
resultados dessa busca foram avaliados em relação ao tipo de menção,
classificando-os, e a autoria de cada mençao.

\subsection{Menções aos projetos de software de análise estática}

O estudo anterior selecionou e caracterizou 60 projetos de software acadêmico
de análise estática publicados nas conferências de Engenharia de Software ASE e
SCAM, a partir das características destes projetos realizamos uma busca
nas bases da ACM e IEEE por ocorrências de estudos mencionando estes projetos.

\subsubsection{Passo 1: Busca}

% strings de busca > conjunto de metadados bibtex

A busca nas bases da ACM e IEEE utilizou as características dos projetos trazidos
pelo estudo anterior, para cada projeto de software, criamos uma {\it string} de
busca para cada base, estas {\it strings} foram elaboradas utilizando o nome dos
projetos de software, e em alguns casos utilizamos outras características além do
nome.

A decisão sobre quais características utilizar para compor as {\it strings} de busca
foi tomada com base em testes realizados na própria busca, em cada base, ACM e IEEE,
buscamos inicialmente usando apenas o nome do software, analisamos o número total
de resultados e os títulos destes resultados, quando o número de resultados for
muito grande, acima de XXX, e os títulos não aparentavam ser estudos com relação
aos projetos, incluímos mais características, como por exemplo, parte da descrição
do software, ou parte da URL, autores, etc.

%% Cuidado que ACM cita IEEE.
Os resultados trazidos pela string final foram armazenados localmente em
arquivos no formato BibTeX, estes resultados incluem os artigos selecionados no
estudo anterior onde os projetos foram encontrados. Para cada software, os resultados foram agrupados
num arquivo único, sem duplicidade entre os resultados trazidos por cada base
bibliográfica. O arquivo de metadados de cada software contém informações sobre
o artigo, autores, ano de publicação, conferência, jornal, etc. Os artigos
também foram armazenados localmente, no formato pdf para serem analisados na
triagem.

\subsubsection{Passo 2: Triagem}

% leitura dos pdf > apenas os papers relevantes que mencionam o nome do software

Com base nos resultados das buscas, iniciamos a leitura de cada artigo em busca
de confirmar se os autores, de fato, mencionam o software acadêmico em alguma
seção do artigo.

A busca por ocorrências ao nome do software foi realizado com o auxílio da
funcionalidade de busca do leitor de pdf utilizado para ...  mecanismo de busca
do leitor de pdf\footnote{Utilizamos o software livre Evince versão 3.22.1}
utilizado para leitura dos artigos, com o auxílio da busca encontramos cada
ocorrência ao nome do software tomando nota a confirmação sobre a menção
encontrada.

Esta informação foi armazenada no próprio arquivo BibTeX num campo adicional
aos demais campos dos metadados do artigo; ao final, temos os metadados do
artigo como, título, autores, ano de publicação, etc, e também uma indicação se
o artigo cita realmente o software.

\subsubsection{Passo 3: Keywording}

% papers relevantes > esquema de tipos de menção

Para cada artigo selecionado na triagem inspecionamos o artigo para identificar
em qual contexto o software acadêmico é mencionado, se os autores mencionam
alguma contribuição ao software, ou se o software foi utilizado apenas para
coleta de dados, ou se nem mesmo mencionam o nome do software.

Um artigo científico pode fazer uma ``menção'' ao software acadêmico diversas
vezes e de diversas formas -- desde uma simples menção nos trabalhos
relacionados até uma grande contribuição ao software. Esta leitura irá gerar
uma escala de tipos de menção ao software. Cada artigo assume, em relação ao
software, um valor nesta escala de tipos de menção. Tabela
\ref{esquema-de-mencao}.

\begin{table}[h]
\caption{Esquema para classificação de menções aos projetos software acadêmico.}
\centering
\begin{tabular}{ l p{10cm} }
  \hline
  Tipo de menção           & Explicação \\
  \hline
  Não menciona o software  & Não menciona o nome do software em nenhum contexto \\
  Menciona o software      & Apenas cita o software ou é o mesmo artigo onde o software selecionado; É um artigo com ``mesmo'' conteúdo publicado na ``mesma'' época; O artigo apenas descreve o software; Menciona o software numa tabela com outros, classifica; Menciona o software como exemplo; Menciona o software como trabalho relacionado; Menciona o software em trabalhos futuros \\
  Usa o software           & Avalia ou caracteriza o software; Usa para coleta ou análise de dados; Usa como objeto de estudo; Usa o software como parte de uma solução, implementação, etc; Cria um software derivado mas não disponibiliza as contribuições \\
  Contribui ou integra     & Contribuição pequena ou moderada; Extende o software; Integra o software a outros sistemas, formatos de entrada/saída, APIs, etc (seja implementando suporte no software ou do outro lado); Refatora parte do software; Implementa parte do software em outro projeto e compara resultados \\
  Contribui ou cria        & Cria; Contribuição inicial criando o projeto; Faz uma grande contribuição; Refatora todo o software; Abre o código de um software que antes era de código fechado \\
  \hline
\end{tabular}
\label{esquema-de-mencao}
\end{table}

Foi lido em busca de encontrar menções ao nome do software, em qual contexto o
software é mencionado e de que forma é mencionado, resume cada tipo de menção com explicação dos casos em
que se enquadram, o método utilizado para.

\subsubsection{Passo 4: Extração}

A partir da classificação , Tabela \ref{esquema-de-mencao}, extraímos de cada
artigo mencionando os projetos de acordo com o esquema de codificação criado no
passo anterior, os dados foram armazenados nos próprios arquivos BibTex,
criamos X novos campos em cada entrada de artigo nestes metadados com o tipo de
menção que é feito naquele artigo.

Ao final dessa revisão temos para cada projeto de software dados sobre as
menções a eles nas bases do ACM e IEEE, temos o tipo de cada menção, se
contribuiu com o projeto, se foi avaliado, ou apenas citado como referência, os
artigos selecionados no estudo anterior que deram origem ao conjunto de
projetos de software estão incluídos nestes resultados.

\subsection{Autoria das menções aos projetos de software de análise estática}

%das bibliotecas digitais, temos para cada um dos artigos todos os seus
%metadados,. Os autores de
%cada uma das menções ao software, por exemplo, serão utilizados na fase de
%análise para calcular o quanto de autores novos começaram a publicar sobre
%certo software acadêmico.

Os dados coletados sobre as menções a cada software foram acrescentados com uma
nova informação calculada a partir da autoria de cada artigo mencionando o
software, estamos considerando que todos os autores de um certo artigo tem o
mesmo peso em relação a menção ao software naquele estudo, mesmo sabendo que não
é raro que cientistas trabalhando em conjunto apresentem níveis diferentes de
domínio sobre cada parte da pesquisa.

Os autores originais do primeiro artigo publicando sobre software, na maior
parte dos casos é o mesmo paper selecionado no estudo anterior que deu origem
ao conjunto de projetos, são considerados os autores originais do projeto, a
partir desta primeira publicação cada artigo mencionando o software indica a
entrada de novos atores no ecossistema daquele software.

Foi coletado então para cada artigo em relação aos autores o quanto são
autores novos, e classificamos em termos de todos os autores do estudo
são novos, se todos já publicaram sobre o software anteriormente, ou se
nenhum dos autores jamais publicou sobre aquele software anteriormente, A Tabela
\ref{esquema-de-autoria} apresenta em detalhes este esquema.

%representado quantos novos atores foram incluídos
%no ecossistema daquele software, isto foi feito comparando o conjunto de autores
%da publicaçao com o conjunto acumulado de todos os autores anteriores, 
%podendo assumir um dos valores da Tabela \ref{coding-scheme-author}.

\begin{table}[h]
\caption{Esquema para classificação de autoria de menções aos projetos de software acadêmico.}
\centering
\begin{tabular}{ l c p{8cm} }
  \hline
  Novos atores no ecossistema & Peso & Explicação \\
  \hline
  Nenhum    & 0.5  & Nenhum dos autores jamais publicou sobre o software \\
  Parte     & 0.25 & Uma parte dos autores já publicou sobre o software em anos anteriores \\
  Todos     & 0.1  & Todos os autores já publicaram sobre o software em anos anteriores \\
  Criadores & 0    & São os primeiros autores a publicar sobre o software \\
  \hline
\end{tabular}
\label{esquema-de-autoria}
\end{table}

Estes dados foram calculados com base nos metadados já coletados anteriormente
dispníveis nos arquivos BibTeX com as menções à cada projeto, tratamos os nomes
dos autores para utilizar o mesmo formato, já que os artigos variam em relação
ao padrão de nomes dos autores.

Com os nomes dos autores normalizados comparamos e caso tenham a mesma string
consideramos que trata-se de um mesmo pesquisador, os nomes iguals são
considerados como sendo a mesma pessoa. Avaliamos cada artigo em ordem
cronológica comparando os autores com relação a todos os autores anteriores.

Este processo foi realizado automaticamente com um script escrito durante este
trabalho de pesquisa, os dados resultantes são armazenados de volta nos
arquivos BibTeX.

\section{Preparação}

(pendente)

\section{Coleta de dados}

...

\subsection{Menções aos projetos de software de análise estática}

\subsubsection{Passo 1: Busca}

Uma string de busca foi definida para cada software acadêmico selecionado.
Além do nome do software pesquisado, as strings de busca incluíram outras
características do software sempre que necessário.
A Tabela \ref{search-strings-table} apresenta o número de resultados encontrados para cada
projeto de software.


\begin{longtable}{ l c c c c }
\caption{Número total de resultados da busca nas bases da ACM e IEEE.}
\label{search-strings-table} \\
  \hline
  \hhline{ l c c c c |}
  \endfirsthead
  \hhline{ l c c c c |}
  \hline
   \multirow{2}{*}{\textbf{Nome do software}} & \multicolumn{3}{c}{{\bf Resultados}} & \multirow{2}{*}{\textbf{Menções}} \\
   & \textbf{ACM} & \textbf{IEEE} & \textbf{Total} & \\
  \hline
  \hhline{ l c c c c |}
  \endhead
  \hhline{-----}
  \multicolumn{5}{c}{continua na próxima página} \\
  \hhline{-----} \endfoot
  \hhline{-----} \endlastfoot
   \multirow{2}{*}{\textbf{Nome do software}} & \multicolumn{3}{c}{{\bf Resultados}} & \multirow{2}{*}{\textbf{Menções}} \\
   & \textbf{ACM} & \textbf{IEEE} & \textbf{Total} & \\
  \hline
   2LS & 11 & 26 & 37 & 1 \\
   AccessAnalysis & 5 & 3 & 8 & 2 \\
   APIExample & 10 & 6 & 16 & 4 \\
   BEG & 4 & 6 & 10 & 9 \\
   ccJava & 4 & 3 & 7 & 5 \\
   CIVL & 6 & 2 & 8 & 6 \\
   CodeBoost & 16 & 9 & 25 & 15 \\
   CSL & 7 & 7 & 14 & 6 \\
   CPA+ & 4 & 4 & 8 & 5 \\
   CSeq & 12 & 4 & 16 & 5 \\
   DDVerify & 2 & 2 & 4 & 3 \\
   Derailer & 6 & 1 & 7 & 2 \\
   Diagnosys & 6 & 3 & 9 & 1 \\
   DOMPLETION & 1 & 1 & 2 & 2 \\
   DRC & 15 & 4 & 19 & 5 \\
   e-munity & 0 & 1 & 1 & 1 \\
   EJB Interceptor Analyzer & 1 & 4 & 5 & 3 \\
   Error Prone & 23 & 24 & 47 & 2 \\
   ESBMC & 20 & 30 & 50 & 42 \\
   ETXL & 7 & 3 & 10 & 1 \\
   FaultBuster & 1 & 3 & 4 & 1 \\
   Flowgen & 4 & 4 & 8 & 3 \\
   GRT & 2 & 11 & 13 & 10 \\
   GUIZMO & 0 & 0 & 0 & 1 \\
   GumTree & 20 & 17 & 37 & 19 \\
   HUSACCT & 5 & 2 & 7 & 7 \\
   Indus & 2 & 4 & 6 & 4 \\
   JastAdd & 26 & 24 & 50 & 45 \\
   JFlow & 11 & 5 & 16 & 7 \\
   JstereoCode & 4 & 9 & 13 & 9 \\
   Jtop & 2 & 2 & 4 & 3 \\
   Bogor/Kiasan & 19 & 18 & 37 & 16 \\
   Loopfrog & 3 & 3 & 6 & 6 \\
   Lotrack & 1 & 3 & 4 & 2 \\
   MPAnalyzer & 2 & 1 & 3 & 1 \\
   MSP & 17 & 20 & 37 & 2 \\
   mygcc & 2 & 5 & 7 & 7 \\
   PARSEWeb & 29 & 20 & 49 & 24 \\
   PAT & 3 & 10 & 13 & 4 \\
   PHP AiR & 3 & 8 & 11 & 9 \\
   protopurity & 0 & 0 & 0 & 1 \\
   Pseudogen & 1 & 4 & 5 & 1 \\
   PtYasm & 0 & 2 & 2 & 2 \\
   PuMoC & 3 & 1 & 4 & 3 \\
   PYTHIA & 5 & 10 & 15 & 3 \\
   ReAssert & 16 & 12 & 28 & 14 \\
   Rêve & 6 & 7 & 13 & 1 \\
   RRFinder & 3 & 2 & 5 & 4 \\
   Sapid/XML & 1 & 4 & 5 & 5 \\
   Sonar Qube Plug-in & 1 & 1 & 2 & 1 \\
   SPARTA & 2 & 5 & 7 & 4 \\
   srcML & 9 & 34 & 43 & 40 \\
   SWAT & 10 & 5 & 15 & 4 \\
   TACLE & 2 & 2 & 4 & 3 \\
   TEBA & 3 & 8 & 11 & 1 \\
   TestEra & 23 & 21 & 44 & 24 \\
   Vdiff & 4 & 6 & 10 & 5 \\
   WALA & 8 & 4 & 12 & 11 \\
   Wrangler & 25 & 12 & 37 & 33 \\
   XOgastan & 0 & 7 & 7 & 5 \\
  \hline
  {\bf Total} & 438 & 459 & 897 & 465 \\
\end{longtable}



Encontramos ao total somando os resultados de todos os projetos 897 artigos
com os termos de busca nas bases ACM e IEEE, é possível que exista artigos
que apareçam repetidos em mais de um projeto, os este número total não
são resultados únicos, mas a triagem no passo seguinte foi realizado por
projeto, portanto não nos importamos com artigos que se repetem e citam
mais de um dos projetos do nosso conjunto.

% unico
% cat cache/citations.bib | grep " title = " | wc -l

Foi feito o download de cada artigo armazenando-os junto aos metadados
de cada projeto. Entre os 897 resultados da busca, 684 são únicos, ou seja,
213 artigos foram encontrados em mais de um projeto.

\subsubsection{Passo 2: Triagem}

Inspecionamos cada artigo em busca de menções ao nome do projeto de software
associado com a busca que retornou estes artigos, nesta busca utilizamos como
critério qualquer ocorrencia ao nome do projeto de software mas que seja realmente
referente ao software em sí, alguns projetos de software do nosso conjunto possui
nomes comuns que foram encontrados nos artigos mas não faziam referencia
ao software, em alguns casos os nomes apareciam em fórmulas matemáticas ou
como parte de um outra palavra.

Ao final selecionadmos para cada projeto um conjunto de artigos que de fato
mencionam o software, ainda sem caracterizar o tipo de menção, apenas se
realmente menciona.

334 artigos fazem menção aos 60 projetos

350 não faz menção aos projetos

\subsubsection{Passo 3: Keywording}

A avaliação dos 334 artigos diversos artigos nos gerou uma escala de tipos de menção ao
software, detalhado na Tabela \ref{esquema-de-mencao}, com 5 valores
distintos, onde o último tipo de menção com maior valor inclui todos os demais.
Um tipo de menção com maior peso inclui implicitamente o tipo de
menção de menor peso, e assim sucessivamente.

também o tipo de menção que é feita ao software: se apenas cita como exemplo,
se contribui com novos algoritmos e técnicas, se avalia, ou apenas descreve o
software comparando-o com outro software.

\subsubsection{Passo 4: Extração}

Coletamos as informações relacionadas a cada menção encontrada, cada um dos 334
artigos encontrados mencionando os projetos, coletamos como o projeto é
mencionado, encontramos 451 menções aos 60 projetos nestes 334 artigos
encontrados.

% dos 334 artigos:
% 
% $contribution_weight=0.1$  173 artigos citam apenas
% 
% $contribution_weight=0.25$ 110 artigos usam ou avaliam ou caracterizam
% 
% $contribution_weight=0.5$ 37 contribuição pequena ou moderada
% 
% $contribution_weight=1$ 14 contribuicao grande ou cria

(pendente)


\begin{longtable}{ l c c c c c }
\caption{Número total de menções por tipo aos projetos de software.}
\label{search-strings-table} \\
  \hline
  \hhline{ l c c c c c |}
  \endfirsthead
  \hhline{ l c c c c c |}
  \hline
   \multirow{2}{*}{\textbf{Nome do software}} & \multicolumn{5}{c}{{\bf Menções}} \\
   & \textbf{Cita} & \textbf{Usa} & \textbf{Contribui} & \textbf{Cria} & \textbf{Total} \\
  \hline
  \hhline{ l c c c c c |}
  \endhead
  \hhline{------}
  \multicolumn{6}{c}{continua na próxima página} \\
  \hhline{------} \endfoot
  \hhline{------} \endlastfoot
   \multirow{2}{*}{\textbf{Nome do software}} & \multicolumn{5}{c}{{\bf Menções}} \\
   & \textbf{Cita} & \textbf{Usa} & \textbf{Contribui} & \textbf{Cria} & \textbf{Total} \\
  \hline
   2LS & 0 & 0 & 0 & 1 & 1 \\
   AccessAnalysis & 1 & 0 & 0 & 1 & 2 \\
   APIExample & 3 & 0 & 0 & 1 & 4 \\
   BEG & 4 & 3 & 1 & 1 & 9 \\
   ccJava & 1 & 0 & 3 & 1 & 5 \\
   CIVL & 5 & 0 & 0 & 1 & 6 \\
   CodeBoost & 14 & 0 & 0 & 1 & 15 \\
   CSL & 2 & 2 & 1 & 1 & 6 \\
   CPA+ & 0 & 0 & 4 & 1 & 5 \\
   CSeq & 1 & 2 & 1 & 1 & 5 \\
   DDVerify & 2 & 0 & 0 & 1 & 3 \\
   Derailer & 1 & 0 & 0 & 1 & 2 \\
   Diagnosys & 0 & 0 & 0 & 1 & 1 \\
   DOMPLETION & 1 & 0 & 0 & 1 & 2 \\
   DRC & 4 & 0 & 0 & 1 & 5 \\
   e-munity & 0 & 0 & 0 & 1 & 1 \\
   EJB Interceptor Analyzer & 1 & 1 & 0 & 1 & 3 \\
   Error Prone & 0 & 1 & 0 & 1 & 2 \\
   ESBMC & 16 & 18 & 7 & 1 & 42 \\
   ETXL & 0 & 0 & 0 & 1 & 1 \\
   FaultBuster & 0 & 0 & 0 & 1 & 1 \\
   Flowgen & 2 & 0 & 0 & 1 & 3 \\
   GRT & 5 & 2 & 1 & 2 & 10 \\
   GUIZMO & 0 & 0 & 0 & 1 & 1 \\
   GumTree & 6 & 11 & 1 & 1 & 19 \\
   HUSACCT & 3 & 2 & 0 & 2 & 7 \\
   Indus & 0 & 3 & 0 & 1 & 4 \\
   JastAdd & 25 & 15 & 4 & 1 & 45 \\
   JFlow & 5 & 1 & 0 & 1 & 7 \\
   JstereoCode & 2 & 5 & 0 & 2 & 9 \\
   Jtop & 0 & 1 & 0 & 2 & 3 \\
   Bogor/Kiasan & 10 & 0 & 5 & 1 & 16 \\
   Loopfrog & 2 & 2 & 0 & 2 & 6 \\
   Lotrack & 1 & 0 & 0 & 1 & 2 \\
   MPAnalyzer & 0 & 0 & 0 & 1 & 1 \\
   MSP & 1 & 0 & 0 & 1 & 2 \\
   mygcc & 3 & 1 & 1 & 2 & 7 \\
   PARSEWeb & 22 & 1 & 0 & 1 & 24 \\
   PAT & 2 & 0 & 0 & 2 & 4 \\
   PHP AiR & 1 & 5 & 2 & 1 & 9 \\
   protopurity & 0 & 0 & 0 & 1 & 1 \\
   Pseudogen & 0 & 0 & 0 & 1 & 1 \\
   PtYasm & 0 & 0 & 0 & 2 & 2 \\
   PuMoC & 1 & 0 & 0 & 2 & 3 \\
   PYTHIA & 0 & 0 & 1 & 2 & 3 \\
   ReAssert & 9 & 3 & 0 & 2 & 14 \\
   Rêve & 0 & 0 & 0 & 1 & 1 \\
   RRFinder & 2 & 0 & 0 & 2 & 4 \\
   Sapid/XML & 1 & 2 & 1 & 1 & 5 \\
   Sonar Qube Plug-in & 0 & 0 & 0 & 1 & 1 \\
   SPARTA & 1 & 2 & 0 & 1 & 4 \\
   srcML & 14 & 24 & 1 & 1 & 40 \\
   SWAT & 1 & 2 & 0 & 1 & 4 \\
   TACLE & 1 & 1 & 0 & 1 & 3 \\
   TEBA & 0 & 0 & 0 & 1 & 1 \\
   TestEra & 17 & 5 & 0 & 2 & 24 \\
   Vdiff & 4 & 0 & 0 & 1 & 5 \\
   WALA & 5 & 5 & 0 & 1 & 11 \\
   Wrangler & 16 & 10 & 6 & 1 & 33 \\
   XOgastan & 4 & 0 & 0 & 1 & 5 \\
  \hline
  {\bf Total} & 222 & 130 & 40 & 73 & 465 \\
\end{longtable}



\subsection{Autoria das menções aos projetos de software de análise estática}

\section{Análise dos Dados} % Raw results from the analysis

Dados de 60 projetos de software acadêmico de análise estática desenvolvidos e
publicados na literatura acadêmica de engenharia de software, informações sobre
diversas formas de menção ao nome destes projetos na literatura acadêmica,
o número de autores mencionando o software ao longo do tempo.

\section{Interpretação dos Resultados} % Hypothesis rejection

\subsection{Q1 - \EstudoDoisQuestaoUm}

% Demográfica.

\subsection{Q2 - \EstudoDoisQuestaoDois}


\subsection{Q3 - \EstudoDoisQuestaoTres}

Número de menções a cada software acadêmico e qual o contexto e tipo de menção
é feita, quem e quantos são os autores de cada menção.

\subsection{Q4 - \EstudoDoisQuestaoQuatro}

A comparação entre os nomes do autores passou antes pela normalização
no formato de representação, visto que cada artigo utiliza um formato
de nome diferente, transformamos, por exemplo, ``Bajaj, Kon'' em ``Bajaj K.'',
``Costa, Kim A.'' em ``Costa K. A.'', ``Rajan, Sreeranga P.'' em ``Rajan S. P.'',
``Pol, Jaco van de'' em ``Pol J. van de'', ``Zijiang Yang'' em ``Yang Z.'',
e assim comparamos dois nomes considerando que cada um destes nomes representa,
de forma única, um autor.

\subsection{Ameaças à validade}

\section{Conclusões}

FALTA uma síntese aqui. 
Este estudo ...
Resultados mostram que ...
Algumas tendências emergiram a partir da leitura ...

Planejamos fazer outro estudo ... 


% o artigo com resumo do RESER 2011 diz \cite{knutson2010report}:
% 4) Re-
% search tools are either not available or not usable, so precise
% replication is impractical [1, 2, 8, 18, 19].
