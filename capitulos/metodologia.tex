cada citação pontua no máximo 1 ponto para o peso final do paper ao quanto
contribui para a sustentabilidade técnica do software, esta pontuação será
calculada com base nos pesos (em porcentagem) 'contribution_weight' e
'authorship_weight', este último valor é aplicado à contribution weight,
ou seja contribution_weight é acrescido a partir do valor de authorship_weight.
o valor final se ultrapassar 1 será cortado no limite 1 (máximo), a ideia não é muito
os números, não queremos saber se são numeros altos, queremos constancia, queremos
medir se existe um nível de contribuição mínimo aos softwares, isto está
sendo proposto como algo que mantém o software vivo e útil para a comunidade
acadêmmica. (por hora o valor mínimo "ideal" por ano é "0.5", ou seja, um
valor bem modesto, este valor indica que houve ao menos uma contribuição
ao software, ou que teve citações suficientes equivalente a uma contribuição,
o software ao ser muito citado ganha mais visibilidade, impacta na possibilidade
de maior adoção e maior contribuição por terceiros.

* revisão estruturada
   paper{step} = 'structured-review';
* citações ao software
   citations{key}{step} = 'review-citations';

escala de peso da contribuição:
===============================

cita
  contribution_weight=0.1
avalia ou usa
  contribution_weight=0.25
contribui
  contribution_weight=0.5
cria ou CONTRIBUI
  contribution_weight=1

escala de peso da autoria:
==========================

são os primeiros autores a publicar sobre o software
  authorship_weight=0
todos os autores já publicaram sobre o software em anos anteriores
  authorship_weight=0.1
uma parte dos autores já publicou sobre o software em anos anteriores
  authorship_weight=0.25
nenhum dos autores jamais publicou sobre o software
  authorship_weight=0.5

contribuições sem peso:
=======================

o software mudou de nome, qual o nome antigo, qual o novo nome
um novo software foi criado a partir daqui, qual o nome do novo
um novo software foi criado com base neste

  weightless_contributions={},

####

cita
  - apenas cita o software ou é o mesmo artigo selecionado na revisão estruturada
  - ou é um artigo com "mesmo" conteúdo publicado na "mesma" época ^
  - descreve o software
  - cita numa tabela com outros, classifica
  - cita como exemplo
  - cita trabalho relacionado
  - cita em trabalhos futuros
avalia ou usa
  - avalia ou caracteriza o software
  - usa para coleta ou análise de dados
  - usa como objeto de estudo
  - usa o software como parte de uma solução, implementação, etc
  - cria um software derivado mas não disponibiliza as contribuições
contribui
  - contribuição pequena ou moderada
  - extende o software
  - integra o software a outros sistemas, formatos de entrada/saída, APIs, etc
    (seja implementando suporte no software ou do outro lado)
  - refatora parte do software
  - implementa parte do software em outro projeto e compara resultados
cria ou CONTRIBUI
  - contribuição inicial criando o projeto
  - faz uma grande contribuição
  - refatora todo o software
  - abre o código


#### artigos nao encontrados para download:

  url = {http://doi.acm.org/10.1145/3090064.3090070},
