\xchapter{Método de pesquisa}
{Este capítulo apresenta a metodologia utilizada para coleta e análise dos
dados do ecossistema de software acadêmico de análise estática}
\label{metodologia}

%Lembrar de destacar o contexto (ASE, SCAM, etc.)
Este trabalho apresenta um estudo de caso exploratório ({\it exploratory case
study}) \cite{stol2015holistic} sobre a sustentabilidade do ecossistema de
software acadêmico de análise estática.
% exploratória ou descritiva? exploratória

% Decidir se será HIPOTESE geral ou QUESTAO DE PESQUISA principal.
Hipótese geral: o atual modelo de desenvolvimento de software do ecossistema de
software acadêmico de análise estática é insustentável.

% Q1: O ecossistema de software acadêmico de análise estática sofre %sérios problemas de sustentabilidade?
% Q2: Quais os tipos de problema?
% Q3: ...

% compartilhamento é um requisito para colaboração
% disponibilidade é um requisito para colaboração
% licenças expressas previamente são requisitos para colaboração
% visibilidade do software é requisito para colaboração
% preocupação com sustentabilidade e qualidade dos produtos é requisito para colaboração

%%%%%%%%%%%%%%%%%%%%%%%%%%%%%%%%%%%%%%%%%%%%%%%%%%%

No segundo estudo, os softwares com código fonte disponível foram avaliados em
relação a sua manutenabilidade através da métrica de complexidade estrutural. A
coleta dessa métrica para cada software foi realizada pelo Analizo, uma suíte
de ferramentas para análise de código fonte, e está sendo considerado como um
indicador de manutenabilidade.

%Um conjunto de softwares de análise estática da indústria foi incluído nesta
%etapa, todos os dados coletados para os softwares acadêmicos foram também
%coletados para este novo conjunto. Esses softwares foram então caracterizados em
%relação à frequencia de lançamentos, linguagem de programação e o tipo de
%entrada suportado.

%Questão de pesquisa:

%* Como ocorre o co-desenvolvimento dos softwares
%* Como acontece colaboração na construção dos softwares
%* Como os softwares contribuem para a construcao de conhecimento novo em novas pesquisas derivadas

% * mais da metade desenvolvem seus próprios softwares
% * falta de visibilidade gera questionamentos sobre qualidade
% * falta de treinamento leva a produzir softwares sem qualidade
% * produtividade científica requer capacidade de replicação
% * capacidade de replicação depende de qualidade

%(mover os coding schema para anexos ou (nao) e manter todos os campos incluindo os de howison
%e os meus, em cada etapa vou preenchendo mais dados, na seleção estruturada pego o mínimo,
%na próxima coleta preencho com mais questões, criador, etc)

%será
%aplicado automaticamente com auxílio de um script desenvolvido durante este
%trabalho de pesquisa, detalhes deste script, outros artefatos produzidos
%durante esta pesquisa, e onde obtê-los pode ser encontrado no Apêndice
%\ref{reproducibilidade-do-estudo}.

%\begin{verbatim}
%  "tool" OU "framework"; E
%  "download" OU "available"; E
%  "http" OU "ftp"; E
%  "static analysis" OU "parser".
%\end{verbatim}

%As informações coletadas sobre cada software inclui nome, descrição e o
%endereço onde obter uma cópia, normalmente página web ou repositório de código
%fonte, esses endereços foram verificados para confirmar se os softwares estão,
%de fato, disponíveis.

%segundo as definições de software {\it
%livre} e {\it open} da Free Software
%Foundation\footnote{\url{https://www.gnu.org/philosophy/free-sw.html}} e Open
%Source Initiative\footnote{\url{https://opensource.org/osd}}, respectivamente,

%#### artigos nao encontrados para download:
%
%  url = {http://doi.acm.org/10.1145/3090064.3090070},

% * revisão estruturada
%    paper{step} = 'structured-review';
% * citações ao software
%    citations{key}{step} = 'review-citations';

% contribuições sem peso:
% =======================
% o software mudou de nome, qual o nome antigo, qual o novo nome
% um novo software foi criado a partir daqui, qual o nome do novo
% um novo software foi criado com base neste

%O antipositivismo, por outro lado, postula que toda a verdade é construída
%socialmente, o que significa que os seres humanos criam sua própria verdade
%sobre as questões de relevância para elas e essas verdades socialmente
%construídas são válidas e valiosas.

%as próprias pessoas, introduzem aspectos que são especialmente difíceis de capturar.
%Entretando, estudos tentando capturar comportamento
%humano como isto se relaciona ao engenharia de software tem aumentado e, não
%surpreendentemente, estão aumentando o uso empregando de métodos qualitativos

%O pesquisador positivista vê a verdade objetiva quanto possível, ou seja,
%existe alguma verdade absoluta sobre as questões de relevância, mesmo que essa
%verdade seja evasiva e que o papel da pesquisa seja cada vez mais próximo
%disso.

%O estudo de engenharia de software tem sido complexo e difícil. A complexidade
%surge de questões técnicas, do desastrada intersecção entre máquina as
%capacidades (ou qualidades?) humanas e de máquina, e do papel central que
%pessoas tem na realização de tarefas da engenharia de software.

%Os primeitos dois aspectos oferecem mais do que apenas problemas complexos para
%manter os pesquisadores de engenharia de software empírica ocupados. Mas o
%último fator, as próprias pessoas, introduzem aspectos que são especial,ente
%difíceis de capturar. Entretando, estudos tentando capturar comportamento
%humano como isto se relaciona ao engenharia de software tem aumentado e, não
%surpreendentemente, estão aumentando o uso empregando de métodos qualitativos

%Veja Creswell (1998) para uma explicação
%mais completa do positivismo, do interpretivismo, de outros quadros filosóficos
%relacionados e do papel dos métodos de pesquisa qualitativa neles.
%   weightless\_contributions={},

%Os métodos qualitativos são apropriados para (mesmo, implicitamente) pesquisa
%positivista em engenharia de software, e um pesquisador não precisa se
%inscrever de todo o coração para a visão de mundo interpretativa
%(antipositivista) para aplicá-los.

%Os dados qualitativos são dados representados como texto e imagens, não números (Gilgun, 1992).

%O foco deste capítulo é bastante estreito, na medida em que se concentra em
%apenas algumas técnicas, e apenas alguns dos possíveis projetos de pesquisa que estão bem
%adequado para tópicos comuns de pesquisa em engenharia de software. Veja Judd et al. (1991),
%Lincoln e Guba (1985), Miles e Huberman (1994) e Taylor e Bogdan
%(1984) para descrições de outros métodos qualitativos.

%A apresentação deste capítulo divide métodos qualitativos para aqueles para
%coletando dados e aqueles para análise de dados. Exemplos de vários métodos são fornecidos
%para cada um, e os métodos podem ser combinados entre si, bem como com
%métodos quantitativos.

%Ao longo deste capítulo, serão extraídos exemplos de
%vários estudos de engenharia de software, incluindo (von Mayrhauser e Vans 1996;
%Guindon et al., 1987; Lethbridge et al., 2005; Perry et al. 1994; Lutters and Seaman,
%2007; Singer, 1998; Orlikowski 1993).

%Exemplos mais detalhados também serão usados
%de estudos descritos em Parra et al. (1997) e Seaman e Basili (1998) porque
%eles representam a experiência do autor (tanto positiva como negativa).

%A Computação muitas vezes é vista como uma disciplina de
%engenharia. Existe a engenharia de software, a engenharia de
%computação e a engenharia de computadores, cada qual com
%um objetivo diferenciado, mas sendo que todas têm em
%comum a produção de conhecimento para aplicação em
%processos de produção de software, sistemas ou hardware.

%A ciência aplicada muitas vezes é confundida com a
%tecnologia. Mas, como será visto adiante, são coisas distintas.

%A perspectiva crítica entende o mundo como a construção histórica e social
%de relações de poder e dominação. Nesta visão sistemas de informação pro-
%vavelmente herdam da sociedade relações de poder, alienação e dominação,
%e revelar essas heranças é o objetivo central da pesquisa qualitativa de fundo
%crítico. [Myers and Young. 1997] é um bom exemplo de pesquisa qualitativa de
%fundo crítico em CC.

%Para muitos pesquisadores de ciências sociais, os métodos qualitativos são
%reservados exclusivamente para uso de pesquisadores antipositivistas e não devem
%ser misturados com métodos quantitativos ou pontos de vista positivistas.

%Os métodos de pesquisa qualitativa foram projetados,
%principalmente por pesquisadores educacionais e outros cientistas sociais
%(Taylor e Bogdan, 1984), para estudar as complexidades de humanos (por exemplo,
%motivação, comunicação, compreensão).


%Numa primeira definição, métodos qualitativos diferem de métodos quanti-
%tativos porque se ocupam de variáveis que não podem ser medidas, apenas
%observadas. Essa é uma dicotomia muito simplista. Métodos qualitativos vêm
%das ciências sociais, em oposição aos métodos quantitativos que derivam das
%ciências naturais.

%Essa diferença na origem já é suficiente para que visões
%diferentes sobre o que é ciência, e como se faz ciência, tornem definições sus-
%cintas sobre o que é um ou outro método muito difícil.


%surgiram de um 
%Historicamente, os métodos de pesquisa qualitativos surgiram da tradição
%interpretativa, ou antipositivista, na pesquisa em ciências sociais.
%O antipositivismo, por sua vez, surgiu como uma reação ao positivismo, que foi
%e continua a ser o fundamento filosófico prevalecente (implícito) da pesquisa
%em ciências naturais e físicas, incluindo a ciência da computação e a
%engenharia de software.

%se presta a combinar recursos qualitativos e métodos quantitativos, a fim de
%aproveitar os pontos fortes de ambos.

%pelo encontro de qualidades
%humanas e de máquinas, e o papel central que as pessoas desempenham nas tarefas
%da engenharia de software. Os aspectos humanos são especialmente difíceis de
%capturar e tem chamado atenção e atraído métodos de pesquisa e coletada de
%dados tradicionais de outros campos, especialmente, das ciências sociais.
%
%ciencias formais: logica e matematica, na computação: teoria dos algoritmos, linguagens formais, autômatos
%ciencias empiricas:
%  ciencias naturais: astronomia, fisica, quimica, na computação: eletronica, circuito logicos
%  ciencias sociais: historia, psicologia, sociologia, na computação: informatica na educação, comércio eletrônico, games, IHC
%
%ciencias puras: formal (lógica) ou empírica (cosmologia), na computação: pouca atuação e presença da computação ainda
%ciencias aplicadas: engenharias, na computação: engenharia de software, informatica na educação, etc
%
%ciencias exatas: matemática, física, química, na computação: a computação é uma ciência exata, a princípio
%ciencias inexatas: metereologia, economia, maioria ciencias sociais, na computação: algoritmos geneticos, casos de redes neurais
%
%ciencias duras: rigor cientifico em observações, experimentos, etc
%  ciencias duras formais: muito uso de logica e matematica
%  ciencias duras naturais: costumam depender de estatistica, exige rigor na comprovação de resultados empiricos, medicina
%
%ciencias moles: aceitam evidencias via estudos de caso por exemplo
%
%ciencias duras X ciencias moles, computação: normalmente entende-se computação como ciência dura, mas ainda
%                                             existe dificuldade de providenciar dados ....
%
%


%Contrário a fontes como [Myers 1997], que classifica a pesquisa qualitativa
%em 4 grupos, eu acho a divisão em apenas dois grupos mais produtiva: a
%pesquisa observacional e a pesquisa-ação (action research).

%A pesquisa
%observacional tem como objetivo observar o ambiente, mas não modificá-lo; já
%o objetivo central da pesquisa-ação é modificar o ambiente.

%É claro que só a
%presença do pesquisador causa alguma modificação no ambiente, mas essa
%modificação não é o objetivo da pesquisa observacional, e algumas variantes
%da pesquisa observacional tentam eliminar esse efeito.

%Segundo vários autores (por exemplo [Orlikowski and Baroudi 1991]) a pes-
%quisa qualitativa onservacional pode ser dividida segundo a perspectiva filosó-
%fica ou epistemológica que a embasa em:
%
%positivista
%interpretativista
%crítica
%
%Eles também são usados para responder o "porquê" às perguntas já
%abordadas pela pesquisa quantitativa.

%Normalmente entende-se a Computação como uma ciência
%dura, mas a realidade ainda, em muitos casos é que os
%pesquisadores têm dificuldade em providenciar dados em
%quantidade suficiente para dar suporte empírico a suas
%conclusões. Assim é que se vêem ainda muitos artigos em
%Computação que utilizam um ou alguns poucos estudos de
%caso para tentar “validar” uma técnica, modelo ou teoria.
%Como visto adiante, o estudo de caso é uma excelente fonte de
%dados para uma pesquisa exploratória, mas, a não ser no caso
%de contradição de uma teoria comumente aceita, o estudo de
%caso não valida a hipótese em estudo.

%Embora a posição filosófica implícita predominante dessas áreas
%de pesquisa permaneça positivista.
%Historicamente os métodos de pesquisa da ciência
%da computação e engenharia de software se fundamentam em filosofias opostas
%ao que fizeram surgir os métodos qualitativos, assim como as ciências naturais
%e físicas.

%Os métodos são descritos aqui em termos de como eles poderiam ser usados em um estudo que
%mistura métodos qualitativos e quantitativos, como muitas vezes são em estudos de software
%Engenharia.

%Eles ajudam a responder perguntas que envolvem variáveis difíceis
%de quantificar (particularmente a característica humana como
%motivação, percepção e experiência).

%outros campos, especialmente, das ciências sociais.
%Métodos qualitativos, então, foram necessários para capturar e descrever essas
%realidades socialmente construídas.

%Há desvantagens no
%entanto.  A análise qualitativa é geralmente mais intensiva em mão-de-obra em
%comparação com análise quantitativa. Os resultados qualitativos geralmente são
%considerados "mais suaves/leves" ou "Mais confusos/fuzzier" do que resultados
%quantitativos, especialmente em comunidades técnicas como a nossa.  Eles são
%também mais difíceis de resumir ou simplificar.

%A principal vantagem da pesquisa bibliográfica reside no fato de permitir ao
%investigador a cobertura de uma gama de fenômenos muito mais ampla do que
%aquela que poderia pesquisar diretamente

%A pesquisa bibliográfica também é indispensável nos estudos históricos

%Enquanto a pesquisa bibliográfica se utiliza fundamentalmente das contribuições
%dos diversos autores sobre determinado assunto, a pesquis"ã documental yale-se
%de materiais que não recebem ainda um tratamento analítico, ou que ainda podem
%ser reelaborados de acordo com os objetos da pesquisa

%A pesquisa bibliográfica é desenvolvida com base em material já elaborado,
%constituído principalmente de livros e artigos científicos

%Uma vantagem da pesquisa documental é que os documentos constituem fonte rica e
%estável de dados

%Outra vantagem da pesquisa documental está em seu custo

%Outra vantagem da pesquisa documental é não exigir contato com os sujeitos da
%pesquisa

%\cite{wazlawick2015metodologia}

%por exemplo, em pesquisadores em sistemas de
%informação, interação homem-computador e engenharia de software.
%emprestado a experiência acumulada dos cientistas sociais aplicadas no contexto
%da computação, 

%De um modo geral, métodos qualitativos em ciência da computaçao são métodos que
%se caracterizam por ser um estudo aprofundado de um sistema no ambiente onde
%ele está sendo usado, ou, em alguns casos, onde se espera que o sistema seja
%usado. Métodos qualitativos sempre envolvem pessoas, e na maioria das vezes
%sistemas.

%A metodologia adotada neste estudo para a coleta de dados é do tipo: A técnica
%de coleta de dados utilizada neste trabalho é a Consulta Documental a materiais
%escritos de diferentes tipos dos registros ....  ; manuais, relatórios e outros

%3.3. Independent Techniques
%       3.3.1. Analysis of Electronic Databases of Work Performed
%       3.3.2. Analysis of Tool Logs
%       3.3.3. Documentation Analysis
%       3.3.4. Static and Dynamic Analysis of a System
%4.2. Coding and Analyzing the Data
%\cite{singer2008software}

%Estudos em engenharia de software descrevemos uma série de técnicas de coleta
%de dados para esses estudos, organizados em torno de uma taxonomia com base no
%grau em que a interação com engenheiros de software é necessária.

%mas pouco se sabe sobre como os engenheiros de software realizam seu
%trabalho. Para melhorar as ferramentas e a prática de engenharia de software,

%A principal vantagem de usar métodos qualitativos é que eles forçam a
%pesquisador para investigar a complexidade do problema em vez de abstrai-lo.
%Assim, os resultados são mais ricos e mais informativos. Técnicas
%independentes, ou seja, são técnicas caracterizadas pela ausencia da
%necessidade de interação entre pesquisador e com os atores sendo estudados.

%, sendo marcada por
%aspectos comportamentais humanos, 
%bastante

%essencialmente por
%questões humanas e de máquinas, e o papel que as pessoas desempenham nas
%tarefas da prática em engenharia de software, os aspectos humanos são
%especialmente difíceis de capturar e tem atraído a atenção dos cientistas para
%métodos de pesquisa pouco usuais, tanto em estudos da ciência da computação,
%quanto da engenharia de software.

%se presta a
%combinar métodos qualitativos e quantitativos, a fim de aproveitar os pontos
%fortes de ambos. A pesquisa de engenharia de software tem sido bastante
%atrasada para reconhecer o valor dos estudos qualitativos. Esta atenção tem
%sido notada nas últimas décadas, por exemplo, através da adoção de métodos
%qualitativos, especialmente em estudos de campo voltados para estudar
%profissionais reais à medida que resolver problemas reais.

%estudo de campo, com coleta de dados através de pesquisa documental

%\cite{brooks2008replication}
%\cite{wainer2007metodos}
%\cite{singer2008software}
%\cite{wazlawick2015metodologia}

%A engenharia de software, apesar de ser uma atividade marcada fortemente por
%aspectos técnicos e humanos, tem sido lenta em reconhecer o valor do método de
%pesquisa qualitativo e seu alto potencial para investigar a complexidade de
%problemas reais sem necessidade de abstrai-los, fonecendo resultados ricos e
%informativos, especialmente sobre questões relacionadas a crenças,
%experiências, atitudes e opiniões de indivíduos ou grupos
%\cite{seaman1999qualitative}.

%, com as seguintes características
%principais:

%estratégia de pesquisa: trabalho de campo
%método de pesquisa: exploratory case study,

%Adotamos uma estratégia de pesquisa de trabalho de campo (Field Studies),
%segundo o framework apresentado em \citeonline{stol2015holistic}, de
%configuração natural (Natural Settings),

%novos conhecimentos e uma compreensão mais profunda dos fenômenos investigados
%* without any intervention by the researchers.
%* focuses on a particular phenomenon, organization or system
%* level of generalizability to a large population is much lower (due to the specific context)
%* to study software professionals or software systems
%* do not include any deliberate modification of the environment in which the research is conducted
%* having a low level of precision of measurement or control as would be found in laboratory experiments
% \item conduzida numa configuração de mundo real
% \item máximo no realismo do contexto
%Maximizes realism of context
%Low on precision of measurement
%Low on generalizability of results

%\begin{itemize}
%  \item Com o foco num fenômeno, organização ou sistema em particupar
%  \item Com um baixo nível de generalização e alto realismo do contexto
%  \item Sem intervenção do pesquisador no ambiente
%\end{itemize}

%STRATEGIES:
%
% * I    Field Studies +‘maximum’ in realism of context
% * I    Field Experiments
%
% * II   Experimental Simulations
% * II   Laboratory Experiments
%
% * III  Judgment Tasks
% * III  Sample Surveys
%
% * IV   Formal Theory
% * IV   Computer Simulations

%duas dimensões principais: obtrusiveness e generality
%                           ‘intrusion’

%As configurações em que as estratégias de pesquisa são adotadas variam,
%four different types of research settings; I, II, III, IV

%I     Natural Settings
%      * the researcher has no goal to make any changes or control any variables of interest
%      * intrusao em Field Experiments é maior que em Field Studies

%II    Contrived Settings
%      * artificially created by a researcher for the sole purpose of the study
%      * top of the ‘obtrusiveness’, maior controle que Field Experiment, Laboratory Experiments tem o maximo de controle

%III   Setting-Independent
%      * aim to gather observations of behavior (people, systems) that is independent from the setting
%      * Judgment Tasks are used to elicit responses from a set experts, or judges about a certain topic
%      * Sample Surveys aim to gather data from a larger group of respondents and as such usually have a better generalizability

%IV    No Empirical Setting
%      * are not empirical strategies but rather are theoretical.

%The choice of research strategy, ‘big picture’

%(( Runkel and McGrath derived a framework to position different research strategies [29] \cite{runkel1972research} ))

%A ``estratégia de pesquisa'' tem um impacto significativo sobre o que pode e
%não pode ser alcançado em um estudo em termos de aquisição de novos
%conhecimentos e uma compreensão mais profunda dos fenômenos investigados
%\cite{stol2015holistic}.

%Utilizamos como estratégia de 
%\cite{seaman1999qualitative}.

%Exemplos de
%seus resultados são o código-fonte, a documentação e os relatórios. Os
%subprodutos são criados no processo de trabalho, por exemplo, solicitações de
%trabalho, trocas de logs e saída do gerenciamento de configuração e ferramentas
%de compilação. Esses repositórios ou arquivos podem servir como fonte primária
%de informações.

%Este trabalho apresenta uma pesquisa exploratória, do tipo documental, com o
%objetivo principal de aprimorar o conhecimento a respeito da sustentabilidade
%do ecossistema de software acadêmico de análise estática.

% Third degree: Independent analysis of work artifacts where already available and sometimes compiled data is used. This is for example the case when documents such as requirements specifications and failure reports from an organization are analyzed or when data from organizational databases such as time accounting is analyzed.
