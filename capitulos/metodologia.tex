\xchapter{Método de pesquisa}
{Este capítulo apresenta a metodologia utilizada para coleta e análise dos
dados do ecossistema de software acadêmico de análise estática}
\label{metodologia}

Este trabalho apresenta uma pesquisa exploratória, do tipo documental, com o
objetivo principal de aprimorar o conhecimento a respeito da sustentabilidade
do ecossistema de software acadêmico de análise estática.

Hipótese geral: O ecossistema de software acadêmico de análise
estática sofre sérios problemas de sustentabilidade, causando retrabalho,
impedindo a colaboração e desacelerando o avanço geral do campo de pesquisa de
análise estática.



%, uma suíte de ferramentas
%para análise de código fonte.

%O Analizo, apesar de ter sido utilizado especialmente para coleta de dados,
%será apresentado em detalhes no Capítulo \ref{analizo} pois ele faz parte
%também do contexto e motivação deste trabalho.


%%%%%%%%%%%%%%%%%%%%%%%%%%%%%%%%%%%%%%%%%%%%%%%%%%%

%No segundo estudo, os softwares com código fonte disponível foram avaliados em
%relação a sua manutenabilidade através da métrica de complexidade estrutural. A
%coleta dessa métrica para cada software foi realizada pelo Analizo, uma suíte
%de ferramentas para análise de código fonte, e está sendo considerado como um
%indicador de manutenabilidade.

%Um conjunto de softwares de análise estática da indústria foi incluído nesta
%etapa, todos os dados coletados para os softwares acadêmicos foram também
%coletados para este novo conjunto. Esses softwares foram então caracterizados em
%relação à frequencia de lançamentos, linguagem de programação e o tipo de
%entrada suportado.

%Questão de pesquisa:

%* Como ocorre o co-desenvolvimento dos softwares
%* Como acontece colaboração na construção dos softwares
%* Como os softwares contribuem para a construcao de conhecimento novo em novas pesquisas derivadas

% * mais da metade desenvolvem seus próprios softwares
% * falta de visibilidade gera questionamentos sobre qualidade
% * falta de treinamento leva a produzir softwares sem qualidade
% * produtividade científica requer capacidade de replicação
% * capacidade de replicação depende de qualidade

%%%%%%%%%%%%%%%%%%%%%%%%%%%%%%%%%%%%%

% pesquisa empírica (qualitativa?) 

%Segundo vários autores (por exemplo [Orlikowski and Baroudi 1991]) a pes-
%quisa qualitativa onservacional pode ser dividida segundo a perspectiva filosó-
%fica ou epistemológica que a embasa em:
%
%positivista
%interpretativista
%crítica
%
%A perspectiva crítica entende o mundo como a construção histórica e social
%de relações de poder e dominação. Nesta visão sistemas de informação pro-
%vavelmente herdam da sociedade relações de poder, alienação e dominação,
%e revelar essas heranças é o objetivo central da pesquisa qualitativa de fundo
%crítico. [Myers and Young. 1997] é um bom exemplo de pesquisa qualitativa de
%fundo crítico em CC.

% pesquisa documental
% 
% Boa parte dos estudos exploratórios pode ser definida como pesquisas
% bibliográficas
% 
% A principal vantagem da pesquisa bibliográfica reside no fato de permitir ao
% investigador a cobertura de uma gama de fenômenos muito mais ampla do que
% aquela que poderia pesquisar diretamente
% 
% A pesquisa bibliográfica também é indispensável nos estudos históricos
% 
% Enquanto a pesquisa bibliográfica se utiliza fundamentalmente das contribuições
% dos diversos autores sobre determinado assunto, a pesquis"ã documental yale-se
% de materiais que não recebem ainda um tratamento analítico, ou que ainda podem
% ser reelaborados de acordo com os objetos da pesquisa
% 
% A pesquisa bibliográfica é desenvolvida com base em material já elaborado,
% constituído principalmente de livros e artigos científicos
% 
% Uma vantagem da pesquisa documental é que os documentos constituem fonte rica e
% estável de dados
% 
% Outra vantagem da pesquisa documental está em seu custo
% 
% Outra vantagem da pesquisa documental é não exigir contato com os sujeitos da
% pesquisa

%(mover os coding schema para anexos ou (nao) e manter todos os campos incluindo os de howison
%e os meus, em cada etapa vou preenchendo mais dados, na seleção estruturada pego o mínimo,
%na próxima coleta preencho com mais questões, criador, etc)

%será
%aplicado automaticamente com auxílio de um script desenvolvido durante este
%trabalho de pesquisa, detalhes deste script, outros artefatos produzidos
%durante esta pesquisa, e onde obtê-los pode ser encontrado no Apêndice
%\ref{reproducibilidade-do-estudo}.

%\begin{verbatim}
%  "tool" OU "framework"; E
%  "download" OU "available"; E
%  "http" OU "ftp"; E
%  "static analysis" OU "parser".
%\end{verbatim}

%As informações coletadas sobre cada software inclui nome, descrição e o
%endereço onde obter uma cópia, normalmente página web ou repositório de código
%fonte, esses endereços foram verificados para confirmar se os softwares estão,
%de fato, disponíveis.

%segundo as definições de software {\it
%livre} e {\it open} da Free Software
%Foundation\footnote{\url{https://www.gnu.org/philosophy/free-sw.html}} e Open
%Source Initiative\footnote{\url{https://opensource.org/osd}}, respectivamente,

%#### artigos nao encontrados para download:
%
%  url = {http://doi.acm.org/10.1145/3090064.3090070},

% * revisão estruturada
%    paper{step} = 'structured-review';
% * citações ao software
%    citations{key}{step} = 'review-citations';

% contribuições sem peso:
% =======================
% o software mudou de nome, qual o nome antigo, qual o novo nome
% um novo software foi criado a partir daqui, qual o nome do novo
% um novo software foi criado com base neste
%   weightless\_contributions={},
