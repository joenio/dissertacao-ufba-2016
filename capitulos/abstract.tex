The increasing adoption of academic software, the software designed to
support scientific research in various areas of knowledge, has made the
modern Science depends on the technical sustainability of software.
%
Unsustainable development of academic software makes it dificult one of the
Science foundations: the reproducibility, or the reproduction's capacity of
scientific studies by independent researchers.
%
In addition, non-sustainable development of academic software can lead to a
``dysfunctional chaotic churn'' - DCC, characterized by the existence of many
similar projects, with few users and short life cycles, ending in parallel with
the initial funding, disconnected and parallel communities, incompatibility
between projects, and seemingly uncoordinated attempts to ``reboot''
everything.
%
However, there are no studies on technical sustainability or DCC in academic
software of the Software Engineering field, especially in the field of static
analysis, with a long tradition in the development of tools to support research
in different areas.
%
The overall objective of this master's research was to analyze the static
analysis software with the purpose of characterizing its technical
sustainability, with respect to publicity, recognition and life cycle, from the
perspective of the scientist -- developer or user -- of academic software in
the context of ASE ({\it Automated Software Engineering}) and SCAM ({\it
Working Conference on Source Code Analysis \& Manipulation}).
%
The academic software published at these conferences was the object of a
documentary research, carried out based on source code, manuals and
repositories.
%, to characterize its publicity.
%
A literature review was carried out at ACM and IEEE for the characterization of
academic software recognition in terms of the types and number of mentions made
by other scientific articles, including contributions in its source code.
%
For academic software with source code available, we carried out the
characterization of its life cycle, based on the number of modules and the
number of releases.
%
We found 60 projects published in ASE and SCAM conferences.
%
The characterization of its technical sustainability showed that: 40\% is not
publicly available, it is not possible to obtain the software in the URL
informed by the authors, making it hard to reproduce of studies that have used
such software;
%
23\% has no mentions in the ACM and IEEE besides those made in the original
publication of the software; and 30\% received contribution in source code.
%
We could observe some evidence of DCC: existence of many academic software
projects of static analysis with few users, and short life cycles.
%
78\% of the static analysis academic software projects are in the initial state
of development, discontinued or terminated.
% vim: filetype=tex
