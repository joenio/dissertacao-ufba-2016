\xchapter{Discussão}{}
\label{discussao}

Respostas para perguntas gerais. Fazer uma seção para cada.
Cuidado com a questão das funcionalidades de cada tool -- pode ser o motivo de ser mais usada,
mesmo que não seja sustentável.

Uma questão interessante é a de ``integridade''. 
O que acontece com os resultados anteriores (publicados)
se houver um erro ou uma mudança substancial da forma de calcular algo?
Na prática, só deveríamos permitir refactorings no código 
de um software acadêmico ou adição de novas funcionalidades?
(Chris)

An so what?

O ecossistema de software acadêmico de análise estática sofre de disfunctional ...?

%Encarar o software acadêmico como a "plataforma" do ecossistema de software.
%Pensar no ecossistema de pesquisa e produção intelectual.

\section{Questões ...} 

(pendente)

\section{Os autores de projetos de software acadêmico com maior sustentabilidade técnica têm mais publicações com o uso de software do que os autores de projetos com menor sustentabilidade?}

\section{Trabalhos relacionados}

\citeonline{segal2008developing}
investigam como o desenvolvimento de software acadêmico pode ser melhorado e
enfatiza as diferenças do software acadêmico em relação aos demais tipos de
software, onde o conhecimento sobre o domínio pode muitas vezes incluir temas
avançados e pouco comuns fora do meio científico.

\citeonline{knutson2010report}
ao resumir as conclusões do evento RESER (Workshop on Replication in Empirical
Software Engineering Research) de 2011 cita que ferramentas de software
acadêmico estão indisponíveis ou não são usáveis em estágios não útil, tornando
replicação precisa impraticável.

\citeonline{robles2010replicating} num revisão de 171 artigos do MSR entre 2004 e 2009
em busca de conjunto de dados, artefatos e ferramentas utilizadas nos estudos
necessárias para replicação mostrou que a maioria dos artigos não conseguiram encontrar
as ferramentas mesmo quando o autor explicitamente afirma que fizeram uma.

%\citeonline{holcombe2011openaccess} no projeto {\it Open Access Pledge}
%\footnote{\url{http://www.openaccesspledge.com}} concentra-se em publicar
%softwares e papers em locais de {\it open access}.

\citeonline{portillo2012tools}
através de um mapeamento sistemático mostra que grande parte das ferramentas de
software criadas na academia estão em estado inicial de desenvolvimento que
apenas uma pequena porcentagem são testados fora do contexto onde foi
desenvolvido. 

\citeonline{chaturvedi2013tools}
faz uma revisão de literatura entre artigos submetidos ao MSR de 2007 até 2013,
identifica conjunto de dados, ferramentas e técnicas utilizadas pelos autores,
mais da metade dos artigos usam ou criam ferramentas, categoriza as ferramentas
em ferramentas novas, ferramentas tradicionais, protótipos e scripts para
mineração de dados.

\citeonline{barnes2013science}
cria o manifesto {\it Science Code Manifesto} e enfatiza que todo código fonte
escrito especificamente para processar dados de publicações devem estar
disponíveis aos revisores e leitores do paper.

\citeonline{marshall2013tools} num mapeamento sistemático sobre artigos criando
ferramentas de apoio a revisão sistemática no domínio de SE conclui que as
ferramentas encontradas estão em estado inicial de desenvolvimento.

\citeonline{hettrick2014uk} mostra que no reino unido entre todas as áreas da
ciência 56\% dos cientistas estão envolvidos no desenvolvimento de software
acadêmico.

\citeonline{wilson2014best} resume as melhores práticas para melhoria da
manutenibilidade e disponibilidade do software acadêmico desenvolvido por
cientistas.

\citeonline{wilson2014software} num resumo sobre as lições aprendidas em 20
anos da iniciativa {\it Software Carpentry} sobre atividades de melhoria das
habilidades dos pesquisadores com computacao.

\citeonline{amann2015software}
investigam através de uma revisão sistemática de literatura uma década de
publicações e encontram que muito poucos estudos são replicáveis visto que
faltam informações incluindo dados e ferramentas, apenas 20\% dos estudos
possuem ferramentas disponíveis.

\citeonline{momcheva2015software}
num survey com 1142 participantes sobre o uso de software em pesquisas da
astronomia mostrou que 90\% dos cientistas escrevem software e 100\% usam
software em suas pesquisas.

\citeonline{beller2016analyzing} avalia e sugere caminhos para melhorar o
desenvolvimento de ferramentas de análise estática com o objetivo de aumentar a
adoção.

\citeonline{smith2016software} resume recomendações sobre como citar software
na literatura acadêmica com objetivo de encorajar uma ampla adoção e uma
política consistente para citação de software entre as múltiplas disciplinas.

\citeonline{smith2016software} afirma que ``citações aos softwares devem
permitir e facilitar acesso ao software, metadados, documentação, dados e
outros materiais necessários tanto para humanos quanto para máquinas''.

%\citeonline{wilkinson2016fair} através do {\it FAIR
%principles}\footnote{\url{https://www.nature.com/articles/sdata201618}} com
%foco em dados de pesquisa, o objetivo é fazer eles serem encontráveis,
%acessíveis, interoperável e reusável. Estes princípios podem ser generalizados
%para aplicar aos softwares.

\citeonline{howison2016software}
numa revisão de literatura mostrou que publicações usando softwares como
método, mostrou que apenas 59 mencionavam o uso de softwares de alguma forma,
os demais 31 artigos, apesar de usar software acadêmico, não mencionavam nada a
respeito.

\citeonline{wilson2017good} apresenta um conjunto de boas práticas que todo
pesquisador pode adotar, independentemente do seu nível de habilidade em
computação. Essas práticas passam por gerenciamento de dados, programação,
colaboração com colegas, organização de projetos, tracking work, e escrita da
manuscritos.
%, sao desenhados para uma grande variadade de fontes publicadas do
%noso dia a dia e do nosso trabalho como voluntário organizando workshopts desde
%2010.

%Open Science Peer Review Oath\footnote{\url{https://f1000research.com/articles/3-271/v2}}
%Concentra-se em potencializar os revisores para exigir acesso aberto aos
%softwares, práticas reprodutíveis e revisões transparentes.

%Muitos pesquisadores não
%disponibilizam os seus softwares 
 %ou quanto o fazem enfrentam problemas com disponibilidade e
%manutenibilidade \cite{prlic2012ten}
