\xchapter{Síntese de Resultados}{}
\label{discussao}

Este capítulo apresenta 
a síntese dos resultados desta pesquisa através de uma
discussão guiada pelas questões apresentadas na Seção \ref{discussao:questoes}.

A Seção \ref{sec:discussao} apresenta 
uma discussão geral a respeito da
sustentabilidade técnica do software acadêmico de análise estática,
a partir dos resultados dos estudos apresentados nos Capítulos \ref{estudo1}, \ref{estudo2} e
\ref{estudo3} sobre a publicização, reconhecimento e ciclo de vida do software
acadêmico de análise estática.
A Seção \ref{sec:recomendacoes} apresenta recomendações para desenvolvedores e usuários de software acadêmico no sentido de dar corpo a práticas já conhecidas mas que trazem uma grande contribuição a este cenário.
Finalmente, a Seção \ref{sec:trabalhosrelacionados} apresenta trabalhos relacionados ao software acadêmico em termos de melhorias no processo de desenvolvimento, reconhecimento e sustentabilidade.

\section{Questões} \label{discussao:questoes}

\newcommand{\QuestaoUm}{
  Como evolui o reconhecimento ao software acadêmico de análise estática
  publicado nas conferências de Engenharia de Software ASE e SCAM?
}
\newcommand{\QuestaoDois}{
  A publicização do software acadêmico de análise estática publicado nas
  conferências de Engenharia de Software ASE e SCAM influencia o seu
  reconhecimento?
}
\newcommand{\QuestaoTres}{
  O ciclo de vida do software acadêmico de análise estática publicado nas
  conferências de Engenharia de Software ASE e SCAM influencia o seu
  reconhecimento?
}
\newcommand{\QuestaoQuatro}{
  Qual o tamanho do software acadêmico de análise estática publicado nas
  conferências de Engenharia de Software ASE e SCAM?
}
\newcommand{\QuestaoCinco}{
  Como evolui o tamanho do software acadêmico de análise estática publicado nas
  conferências de Engenharia de Software ASE e SCAM?
}
\newcommand{\QuestaoSeis}{
  É possível replicar ou reproduzir pesquisas que mencionam software
  acadêmico de análise estática publicado nas conferências de Engenharia de
  Software ASE e SCAM?
}
\newcommand{\QuestaoSete}{
  O software acadêmico de análise estática publicado nas conferências de
  Engenharia de Software ASE e SCAM é sustentável tecnicamente?
}
\newcommand{\QuestaoOito}{
  O software acadêmico de análise estática publicado nas conferências de
  Engenharia de Software ASE e SCAM é útil e maduro suficiente para ser
  utilizado em outras pesquisas?
}

\begin{description}
  \item [Q1] \QuestaoUm
  \item [Q2] \QuestaoDois
  \item [Q3] \QuestaoTres
  \item [Q4] \QuestaoQuatro
  \item [Q5] \QuestaoCinco
  \item [Q6] \QuestaoSeis
\end{description}

\section{Discussão}
\label{sec:discussao}

Considerando que esta pesquisa adotou uma estratégia de 
estudo de campo em ambiente natural,
com alto realismo no contexto estudado e,
consequentemente, com baixa generalização,
seus resultados e conclusões são específicos 
ao domínio de análise estática e, em especial,
ao conjunto de projetos observados.

\subsection{Q1 - \QuestaoUm} % taxa crescimento menções

O software acadêmico tem recebido maior atenção na literatura acadêmica com o
passar do tempo; há uma clara evolução no número total de menções a software
acadêmico de análise estática em publicações das bases ACM e IEEE, conforme
Figura \ref{mentions-by-year}.

\begin{figure}[ht]
  \centering
  \includegraphics[scale=0.6]{imagens/mentions-projects-by-year.png}
  \includegraphics[scale=0.6]{imagens/mentions-type-by-year.png}
  \caption{Número de projetos e menções por tipo ao ano.}
  \label{mentions-by-year}
\end{figure}

Este crescimento, no entanto, pode estar sofrendo influência do aumento de
projetos, uma vez que há também crescimento no número de projetos a cada ano.
Entretanto, ao isolar os dados, mantendo o número de projetos constante ao longo
do tempo, há um crescimento de 38\% ao ano, conforme Figura
\ref{mentions-trend}.

\begin{figure}[ht]
  \center
  \includegraphics[scale=0.6]{imagens/mentions-trend.png}
  \caption{Crescimento no número de menções ao ano (crescimento médio = 38\%).}
  \label{mentions-trend}
\end{figure}

Entre os 60 projetos de software acadêmico de análise estática estudados, 
13 não possuem reconhecimento acadêmico, ou seja, são
mencionados unicamente no artigo original publicado na ASE ou na SCAM,
que apresentou o software pela primeira vez. 
Os outros 47 projetos possuem maior reconhecimento acadêmico, 
considerando que foram encontradas menções em um ou mais artigos
diferentes da primeira publicação.

\subsection{Q2 - \QuestaoDois} % reconhecimento

Há suspeitas de correlação entre o número de menções e o uso de licenças de
software livre. 
Há 201 menções aos 38 projetos software acadêmico de análise estática sem licença definida, 
enquanto há 228 menções aos 22 projetos com licença, ou seja, 
mesmo com um número menor de projetos, o número de menções encontradas foi 14\% maior, 
conforme Figura \ref{license-vs-mentions}.

\begin{figure}[ht]
  \center
  \includegraphics[scale=0.6]{imagens/license-vs-mentions.png}
  \caption{Relação entre o uso de licença de software livre e o número de menções.}
  \label{license-vs-mentions}
\end{figure}

Não foi encontrada relação entre o número de menções e 
outras características do software acadêmico, por exemplo,
disponibilidade para download, linguagem de programação, acesso ao site ou número
de lançamentos.

\subsection{Q3 - \QuestaoTres} % idade média

Os projetos em estágio de {\it Desenvolvimento inicial} possuem menos menções em
comparação com o tempo de vida do que os
projetos em estágio de {\it Evolução} ou de {\it Manutenção} que
possuem mais lançamentos do que o tempo de vida do projeto.

Foram encontradas 160 menções para projetos em estágio {\it Encerrado},
possivelmente indicando artigos publicados antes do projeto entrar nesta fase ou indicando que o projeto está acessível apenas para os autores destes artigos mas não para o público em geral,
71 menções para projetos em estágio de {\it Evolução} ou {\it Manutenção} e 
131 menções para projetos em estágio de {\it Desenvolvimento inicial}.

Os projetos \texttt{s19}, \texttt{s38} e \texttt{s56} foram encontrados em um
grande número de menções, incluindo publicações recentes entre os anos de 2016
e 2017, no entanto o fato de serem projetos em estágio {\it Encerrado} chama
atenção, uma vez que não estão disponíveis publicamente mas continuam recebendo
atenção da academia, este fato não foi investigado, mas estes projetos são
objetos de estudo interessantes para compreender este fenômeno.

Não encontramos relação entre o ciclo de vida do software acadêmico e o reconhecimento
medido através de menções na literatura acadêmica nas bases ACM e IEEE.

\subsection{Q4 - \QuestaoQuatro} % tamanho médio

O tamanho médio do software acadêmico de análise estática publicado nas
conferências ASE e SCAM é de 820 módulos, neste contexto módulo refere-se às
unidades que compõem um sistema de software.  Paradigmas e linguagens de
programação possuem uma ou mais construções que fazem o papel de módulo --
``classes'', ``aspectos'', ``tipos abstratos de dados'', ou ``arquivos-fonte''.
Esta média considera a última versão disponível em código-fonte de cada projeto.

O tamanho médio dos projetos em estágio de {\it Desenvolvimento inicial} é de 595
módulos. Estes projetos são muito menores que os projetos em estágio de {\it Evolução} 
ou  {\it Manutenção} que possuem 1261 módulos, em média. 
Os projetos em estágio de {\it Descontinuado} ou {\it Encerrado} não possuem código disponível e,
portanto, não sabemos seu tamanho.
A Figura \ref{modules-average} apresenta estes números.

\begin{figure}[ht]
  \center
  \includegraphics[scale=0.6]{imagens/modules-total.png}
  \includegraphics[scale=0.6]{imagens/modules-average.png}
  \caption{Número de módulos por projeto e por estágio de evolução no ciclo de vida.}
  \label{modules-average}
\end{figure}

De 20 projetos em estágio de {\it Desenvolvimento inicial}, 
12 projetos tiveram o código-fonte analisado. 
Todos os 8 projetos em estágio de {\it Evolução} ou {\it Manutenção}
tiveram seu código-fonte analisado.
Vale destacar que estes dois estágios foram agregados num único conjunto 
pois possuem características próximas: 
os 2 projetos em estágio de {\it Evolução} estão
muito mais próximos dos projetos em estágio de {\it Manutenção} do que 
do projetos em {\it Desenvolvimento inicial}. 

\subsection{Q5 - \QuestaoCinco} % evolução no tamanho

Ao analisar a evolução no tamanho (em número de módulos) 
dos projetos em estágio de {\it Manutenção}, apresentados na Figura \ref{modules-evolution-servicing},
nota-se um crescimento no número total e médio de módulos
em todos os projetos ao longo dos anos.

\begin{figure}[ht]
  \center
  \includegraphics[scale=0.8]{imagens/modules-evolution-servicing.png}

  \includegraphics[scale=0.8]{imagens/modules-evolution-average.png}
  \caption{Evolução no número de módulos dos projetos em \textit{Manutenção}.
           Os picos de queda do projeto \texttt{s58} estão associados a lançamentos
           de versões intermediárias ou de pré-lançamento identificadas como:
           1.0M, dilla\_0.0.2-RC2, X10\_2.1.1 e X10\_2.1.2}
  \label{modules-evolution-servicing}
\end{figure}

Ao analisar os projetos individualmente, em busca de relação entre
menções e a evolução no tamanho do código-fonte (em número de módulos), 
observamos que, para alguns projetos, existe tal relação.
Especulamos que outros projetos podem ter surgido na academia mas 
tiveram seu desenvolvimento posterior externo ao meio científico, 
ao menos para os que foram publicados e indexados nas bases da ACM e do IEEE.

\begin{description}

  \item[Software \texttt{s6} (CIVL).]
    Este projeto teve 5 menções, sendo que em 2015 houve menção do tipo \texttt{Contribui} 
    e em 2017 apenas menções do tipo \texttt{Cita}. 
    Apesar de poucas menções do tipo \texttt{Contribui},
    o projeto apresenta características de estar evoluindo juntamente a atividade
    acadêmica.

  \item[Software \texttt{s18} (Error Prone).]
    Este projeto foi mencionado em apenas 1 artigo em 2012, sendo que o
    primeiro lançamento da versão 2.0 foi em 2015. Este projeto aparenta ser um
    projeto que ganhou vida própria e seguiu fora da academia.

  \item[Software \texttt{s26} (HUSACCT).]
    Este projeto foi mencionado em 7 publicações distintas; em 2014 houve
    menção Contribui, nos anos seguintes possui menções de uso e contribuição.
    A princípio é um projeto com ligação estreita com a atividade acadêmica.

  \item[Software \texttt{s28} (JastAdd).]
    Este projeto está entre os que mais receberam menções, entre 2003 até 2013.
    O projeto tem muitas menções, incluindo contribuições; após 2013, possui apenas
    citações e uso sem novas contribuições. Observa-se uma clara ligação
    entre a evolução do projeto e a atividade acadêmica.

  \item[Software \texttt{s51} (SPARTA).]
    Este projeto teve 4 menções, sendo apenas 1 menção Contribui em 2015,
    e menção Cita e Usa em 2017. O histórico de lançamentos inicia em 2012
    sendo que as primeiras menções aparecem em 2002, o que nos leva a
    questionar: onde estão os lançamentos anteriores a 2012?

  \item[Software \texttt{s58} (WALA).]
    Este projeto foi mencionado 11 vezes, sendo apenas 1 menção Contribui em 2010.
    As demais menções são, em grande parte, de uso do software. 
    Entre todos os projetos estudados, este é o que possui maior janela de tempo entre lançamentos, 
    com um histórico de lançamentos de 11 anos. A média dos demais  projetos neste estágio de evolução
    é de 4 anos.

\end{description}

Em resumo, há um crescimento constante no tamanho em número de módulos dos
projetos de software acadêmico de análise estática.
Entretanto, alguns projeto apresentam características de evolução 
totalmente independentes da atividade acadêmica, 
indicando que os seus desenvolvedores, 
apesar de continuar evoluindo os projetos de software,
não tem publicado artigos científicos que os mencionam.

\subsection{Q6 - \QuestaoSeis}

Projetos de software acadêmico de análise estática em estágio de {\it Encerrado} 
foram encontrados em menções do tipo Usa em 38 artigos 
e em menções do tipo Contribui em 42 artigos.
Estes projetos não estão disponíveis para download, seja binário ou código-fonte, 
e a URL indicada pelos autores não permanece acessível, 
indicando que os resultados de tais artigos não podem ser reproduzidos.

Em resumo, as pesquisas reportadas em 80 artigos com menção de Uso ou Contribuição em 
software acadêmico de análise estática publicados nas conferências ASE e SCAM 
não podem ser replicadas ou reproduzidas por conta da ausência do software,
caso o software seja disponibilizado outros fatores podem levar a dificuldades
de replicação, como o acesso aos dados, instruções e detalhes sobre coleta e
análise de dados, entre outros fatores.

\section{Recomendações}
\label{sec:recomendacoes}

Os problemas identificados neste estudo podem ser atribuídos, em grande parte,
a baixos orçamentos, limitação de tempo e alta rotatividade entre os
grupos de pesquisa. Outros são, possivelmente, ocasionados por questões
culturais \cite{niemeyer2017open}, como, por exemplo, a tímida adoção de
práticas da Ciência Aberta entre pesquisadores.

Ciência Aberta é um movimento que tem por objetivo tornar a pesquisa
científica, seus dados e sua disseminação acessíveis a todos os interessados,
sejam amadores ou profissionais \cite{WikipediaOpenScience}. Sua principal
motivação está em possibilitar a reprodução dos resultados de pesquisas e em
garantir transparência das metodologias utilizadas. Isto aumenta o impacto
social das pesquisas e gera economia de tempo e dinheiro para os pesquisadores
e para as instituições \cite{nesta2010open}.

Assim, surge um conjunto de ações que podem ser tomadas pelos diferentes atores
em direção a garantir sustentabilidade nos projetos de software, ações para
praticantes de software, pesquisadores, associações profissionais, educadores,
clientes e usuários. No que tange aos resultados deste estudo, as recomendações
são bastante simples mas muito efetivas.

Recomendações aos desenvolvedores de software acadêmico \cite{jimenez_four_2017}:

\begin{description}
  \item [Torne o código-fonte do software público o mais cedo possível.]

    Desenvolva o \\ código-fonte de maneira pública e acessível; utilizar
    repositórios de controle de versão (exemplos, GitHub, Gitlab ou Savannah)
    desde o início do projeto. Quanto mais tempo o projeto seguir um modelo
    fechado, mais difícil se torna abrir. Evite publicar o software em
    infraestrutura particular ou própria, como por exemplo, servidores da
    universidade, que tendem a mudar de endereço ou serem descontinuados.

  \item [Faça o software fácil de ser encontrado e forneça metadados.]

    Forneça metadados a respeito do software através de registros comumente
    adotados pela sua comunidade. Facilitar a descoberta do projeto de software
    e seu código-fonte registrando os metadados a respeito do software em
    locais de registro populares e conhecidos pela comunidade. Metadados devem
    incluir informações de localização do código-fonte, contribuidores,
    licença, versão, identificados, referências e como citar o software.

    Boas opções são periódicos específicos para software e ferramentas, exemplos:
    JOSS, JORS e SoftwareX. Fornecer instrução sobre como citar o software
    adequadamente, se possível incluir no repositório do projeto um arquivo
    BibTeX com os metadados de como deve ser citado.

  \item [Adote uma licença e respeite as licenças de outros pacotes de software.]

    Adote uma licença de software livre adequada para deixar claro como usar,
    modificar e redistribuir o código-fonte em termos e condições claros e bem
    definidos. Defina a licença de maneira pública e acessível no repositório
    de código-fonte e garanta que o software está em conformidade com as
    licenças de todas as dependências do software.

    Sugestão, preferir especialmente licenças com mecanismo de copyleft, como
    por exemplo: GPL.

  \item [Defina processos claros de contribuição, governança e comunicação.]

    Abrir o \\ código do software não significa que o software será desenvolvido
    de maneira pública e colaborativa. Apesar de ser algo desejável,
    recomendações por licenças e práticas comuns de software livre não
    determinam a estratégia para colaborar com a comunidade de desenvolvimento.
    Entretanto, projetos devem ser claros sobre como contribuições devem ser
    feitas e incorporadas tendo modelos de governança transparentes e canais de
    comunicação claros.

\end{description}

Inevitavelmente alguns projetos de software acadêmico irão continuar sendo
úteis após o primeiro lançamento, alguns terão algumas gerações de melhorias,
outros serão usados na sua versão original sem atualização ou manutenção e
alguns outros serão lançados e nunca utilizados. Isto é natural e 
a comunidade ao redor do software irá decidir qual é o melhor caminho a se
tomar num processo evolutivo \cite{weiner2009astronomical}.

No entanto, sendo o software acadêmico uma parte primordial da produção
científica, ele deve estar disponível e continuar disponível para futuras
gerações. Dessa forma, realizar estudos desta natureza, refletindo sobre
o papel do software e evidenciando o quanto ele é tratado em termos
de publicização e reconhecimento é primordial.

No entanto, algumas lições aprendidas na realização deste estudo nos dizem que
atividades de revisão de literatura em busca de menções a software representam
um alto grau de dificuldade e volume de trabalho ``braçal''. Este tipo de trabalho seria
simplificado enormemente caso fosse dado ao software um maior reconhecimento e
incentivos a citação formal destes artefatos digitais, ou ao menos, que haja
alguma padronização sobre como citar software. Tal padronização permitiria automatizar e
recuperar, de forma segura, a relação entre os projetos, entre artigos e o
software, entre pesquisadores e software, assim como é possível hoje para toda
a produção de literatura.

\section{Trabalhos relacionados}
\label{sec:trabalhosrelacionados}

Não encontramos artigos científicos que tratam de sustentabilidade técnica
de software acadêmico segundo publicização, reconhecimento e ciclo de vida,
conforme proposto e aplicado em nossa pesquisa sobre sustentabilidade
técnica de software acadêmico de análise estática.

Entretanto, a preocupação com software acadêmico e sua sustentabilidade
extrapola a área de engenharia de software.
A pesquisa sobre software acadêmico abrange aspectos 
direta ou indiretamente relacionados a sua sustentabilidade,
incluindo o envolvimento de cientistas de diversas áreas sem conhecimento
sobre boas práticas de desenvolvimento até 
questões relacionadas a reprodutibilidade científica.

Em 2013, \citeonline{barnes2013science}
criaram o manifesto {\it Science Code Manifesto} que enfatiza que todo código-fonte
escrito especificamente para processar dados de publicações deve estar
disponível aos revisores e leitores da publicação.

\citeonline{hettrick2014uk} mostraram que, no Reino Unido, em todas as áreas da
ciência, 56\% dos cientistas estão envolvidos no desenvolvimento de software
acadêmico. 
Em 2015, \citeonline{amann2015software}
investigaram, por meio de uma revisão sistemática de literatura, uma década de
publicações e concluíram que poucos estudos são replicáveis:
faltam informações incluindo dados e ferramentas, e apenas 20\% dos estudos
possuem ferramentas disponíveis.
Ainda em 2015, \citeonline{momcheva2015software} realizaram
um survey com 1142 participantes sobre o uso de software em pesquisas da
astronomia e mostraram que 90\% dos cientistas escreve software e 100\% usa
software em suas pesquisas.

\citeonline{smith2016software} apresentaram recomendações sobre como citar software
na literatura acadêmica com objetivo de encorajar uma ampla adoção e uma
política consistente para citação de software entre as múltiplas disciplinas.
%
\citeonline{smith2016software} afirmaram que ``citações ao software devem
permitir e facilitar acesso ao software, metadados, documentação, dados e
outros materiais necessários tanto para humanos quanto para máquinas''.

\citeonline{howison2016software},
por meio de uma  revisão de literatura, mostraram que,
em  publicações usando software como
método, apenas 59 de um total de 90 artigos mencionavam o uso de software de alguma forma;
os demais 31 artigos, apesar de usar software acadêmico, não mencionavam nada a
respeito.

\citeonline{wilson2017good} apresentaram um conjunto de boas práticas que todo
pesquisador pode adotar, independente do seu nível de habilidade em
computação. Essas práticas passam por gerenciamento de dados, programação,
colaboração com colegas, organização de projetos, {\it tracking work}, e escrita da
manuscritos.

%%--------------------------------------%%

Na Engenharia de Software, 
\citeonline{segal2008developing}
investigaram como o desenvolvimento de software acadêmico poderia ser melhorado e
enfatizaram as diferenças do software acadêmico em relação aos demais tipos de
software, onde o conhecimento sobre o domínio pode muitas vezes incluir temas
avançados e pouco comuns fora do meio científico.

\citeonline{knutson2010report},
ao resumir as conclusões do evento RESER ({\it Workshop on Replication in Empirical
Software Engineering Research}) de 2011, pontuou que ferramentas de software
acadêmico estão indisponíveis ou não são usáveis, tornando
replicação precisa impraticável.

\citeonline{robles2010replicating}, em revisão de 171 artigos do MSR entre 2004 e 2009,
em busca de conjunto de dados, artefatos e ferramentas utilizadas nos estudos
necessárias para replicação, mostrou que a maioria dos artigos não conseguiu encontrar
as ferramentas mesmo quando o autor explicitamente afirma que uma ferramenta foi desenvolvida.

\citeonline{kon2011free}, numa revisão de literatura da conferência SBES entre
os anos 2000 e 2010 mostrou que há um número crescente de pesquisadores
Brasileiros disponibilizando suas ferramentas como software livre, no entanto,
no período pesquisado, encontraram 14 autores com publicação de software acadêmico
que apesar de estarem dispostos a disponibilizar os seus projetos como software
livre não obtiveram sucesso nesta tarefa.

\citeonline{portillo2012tools}, por meio de
um mapeamento sistemático, mostraram que grande parte das ferramentas de
software criadas na academia estão em estado inicial de desenvolvimento, e que
apenas uma pequena porcentagem é testada fora do contexto onde foi
desenvolvida. 

\citeonline{chaturvedi2013tools}
fizeram uma revisão de literatura de artigos submetidos ao MSR de 2007 até 2013,
e identificaram conjuntos de dados, ferramentas e técnicas utilizados pelos autores.
A revisão identificou que mais da metade dos artigos do MSR usa ou cria ferramentas,
e categorizou as ferramentas em:
ferramentas novas, ferramentas tradicionais, protótipos e scripts para
mineração de dados.

\citeonline{marshall2013tools} realizaram um mapeamento sistemático sobre 
artigos que apresentam ferramentas de apoio a revisão sistemática no domínio da
Engenharia de Software
e concluiu que as
ferramentas encontradas estão em estado inicial de desenvolvimento.

\citeonline{wilson2014best} resumiram as melhores práticas para melhoria da
manutenibilidade e disponibilidade do software acadêmico desenvolvido por
cientistas.
\citeonline{wilson2014software} reportou  
as lições aprendidas em 20 anos da iniciativa {\it Software Carpentry} 
para atividades de melhoria das
habilidades dos pesquisadores com computação.

Em artigo recente no contexto de adoção de ferramentas de análise estática,
\citeonline{beller2016analyzing} avaliaram e sugeriram caminhos 
para melhorar o desenvolvimento de ferramentas de análise estática 
com o objetivo de aumentar sua adoção.
