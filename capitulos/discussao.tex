\xchapter{Discussão}{}
\label{discussao}

Respostas para perguntas gerais. Fazer uma seção para cada.
Cuidado com a questão das funcionalidades de cada tool -- pode ser o motivo de ser mais usada,
mesmo que não seja sustentável.

Uma questão interessante é a de ``integridade''. 
O que acontece com os resultados anteriores (publicados)
se houver um erro ou uma mudança substancial da forma de calcular algo?
Na prática, só deveríamos permitir refactorings no código 
de um software acadêmico ou adição de novas funcionalidades?
(Chris)

An so what?

O ecossistema de software acadêmico de análise estática sofre de disfunctional ...?

%Encarar o software acadêmico como a "plataforma" do ecossistema de software.
%Pensar no ecossistema de pesquisa e produção intelectual.

\section{Questões ...} 

(pendente)

\section{Os autores de projetos de software acadêmico com maior sustentabilidade técnica têm mais publicações com o uso de software do que os autores de projetos com menor sustentabilidade?}

\section{Trabalhos relacionados}

software acadêmico - Esta definição pode ser encontrada, com algumas variações, pelos nomes de
{\it research tool} %\cite{Portillo12},
{\it research-originated software} %\cite{Kon2011},
{\it research software} %\cite{hettrick2014uk} ou
{\it scientific software} %\cite{segal2008developing}.

%Olhar melhor os atributos de qualidade no artigo de Venters:
% https://openresearchsoftware.metajnl.com/articles/10.5334/jors.ao/
% We propose that software sustainability should be considered in a similar manner to the concept of dependability[16]; 
% a measure of a system’s availability, integrity, maintainability, reliability, and safety 
% where the attributes of dependability are defined as:
% Availability: readiness for correct service;
% Integrity: the absence of improper system alteration;
% Maintainability: undergo modifications and repairs;
% Reliability: continuity of correct service;
% Safety: the absence of catastrophic consequences on the user(s) and the environment.

%\begin{comment}
%
%We propose that software sustainability can be defined as ‘a measure of a systems extensibility, interoperability, maintainability, portability, reusability, scalability, and usability’ where the attributes are defined as:
%Extensibility: a measure of the software’s ability to be extended and the level of effort required to implement the extension;
%Interoperability: the effort required to couple software systems together.
%Maintainability: the effort required to locate and fix an error in operational software;
%Portability: the effort required to port software from one hardware platform or software environment to another;
%Reusability: the extent to which software can be reused in other applications;
%Scalability: the extent to which software can accommodate horizontal or vertical growth.
%Usability: the extent to which a product can be used by specified users to achieve specified goals with effectiveness, efficiency, and satisfaction in a specified context of use.
%If we accept that the concept of sustainability goes beyond the software artifact itself then other quality attributes such as efficiency may be appropriate candidates:
%Efficiency: the amount of computing resources and code required to execute a function.
%
%\end{comment}

%e
%mostra onde pode-se melhorar no desenvolvimento dessas ferramentas para
%aumentar a adoção \cite{beller2016analyzing}.

%, informações
%limitadas a respeito da qualidade da das ferramentas de análise estática
%metodologia para avaliação sistemática de ferramentas de análise estática,
%avalia 6 ferramentas comerciais, apresenta o Scanstud, um framework para
%avaliação sistemática de ferramentas de análise estática para segurança
%\cite{Johns2011}.

%, faz um estudo
%propondo uma taxonomia e um conjunto de dimensões para caracterização de
%ferramentas de análise estática \cite{Novak2010}.

% (5) Tools in mining software repositories \cite{chaturvedi2013tools}
% Faz uma revisão dos papers submetidos ao MSR desde 2007 até 2013 (?) e
% identifica data sets, ferramentas e técnicas utilizadas pelos autores, mais
% da metade dos papers usam ou criam ferramentas, categoriza as ferramentas em
% ferramentas novas, ferramentas tradicionais, protótipos e scripts para
% mineração de dados

%Science Code Manifesto \cite{barnes2013science}.
%Foco em código fonte escrito especificamente para processar dados de
%publicações, afirma que ``todo código fonte escrito especificamente para
%processar dados de uma publicação deve estar disponível para os revisores e
%leitores do paper''.


%Improving academic software engineering projects: A comparative study of academic and industry projects
%(compara as praticas de desenvolvimento da industria e academia e sugere melhorias, 1998!)
%https://link.springer.com/article/10.1023%2FA%3A1018925902814?LI=true


%Software is a critical part of modern research and yet there is little support across the
%scholarly ecosystem for its acknowledgement and citation. Inspired by the activities
%of the FORCE11 working group focused on data citation, this document
%summarizes the recommendations of the FORCE11 Software Citation Working
%Group and its activities between June 2015 and April 2016. Based on a review of
%existing community practices, the goal of the working group was to produce a
%consolidated set of citation principles that may encourage broad adoption of a
%consistent policy for software citation across disciplines and venues. Our work is
%presented here as a set of software citation principles, a discussion of the motivations
%for developing the principles, reviews of existing community practice, and a
%discussion of the requirements these principles would place upon different
%stakeholders. Working examples and possible technical solutions for how these
%principles can be implemented will be discussed in a separate paper.
%\cite{smith2016software}

%Cita um mapeamento feito sobre estudos que criam ferramentas para apoio a
%revisão sistemática no domínio de SE, 14 estudos foram selecionados, ao final
%apenas 8 tinham proposta de ferramentas, ao final conclui que as ferramentas
%encontradas estão em estado inicial de desenvolvimento \cite{marshall2013tools}.

%Cita um mapeamento sistemático com objetivo de encontrar ferramentas de
%comunicação e coordenação para suporte a times altamente distribuidos
%gograficamente, encontrou 132 ferramentas, para uso em projetos de software
%global. A maioria destas ferramentas foram desenvolvidas em centros de
%pesquisas, e apenas uma pequena porcentagem (18.9\%) foram testados fora do
%seu contexto onde foi desenvolvido \cite{Portillo12}.


Este paper
apresenta um conjunto de boas práticas que todo pesquisador pode adotar,
independentemente do seu nível de habilidade em computação. Essas práticas
passam por gerenciamento de dados, programação, colaboração com colegas,
organização de projetos, tracking work, e escrita da manuscritos, sao
desenhados para uma grande variadade de fontes publicadas do noso dia a dia e
do nosso trabalho como voluntário organizando workshopts desde 2010
\cite{wilson2017good}.

FAIR principles \cite{wilkinson2016fair}\footnote{\url{https://www.nature.com/articles/sdata201618}}
Foco em dados de pesquisa. O objetivo é fazer eles serem encontráveis,
acessíveis, interoperável e reusável. Estes princípios podem ser
generalizados para aplicar aos softwares.

Open Science Peer Review Oath\footnote{\url{https://f1000research.com/articles/3-271/v2}}
Concentra-se em potencializar os revisores para exigir acesso aberto aos
softwares, práticas reprodutíveis e revisões transparentes.

Open Access Pledge \cite{holcombe2011openaccess}\footnote{\url{http://www.openaccesspledge.com}}
Concentra-se em publicar softwares e papers em locais de {\it open access}.

Best Practices for Scientific Computing \cite{wilson2014best}
resume as melhores práticas para melhorar a situação onde softwares
academisoc sofrem de manutenabilidade, disponibiliade etc, boas praticas, etc

complemento do artigo acima: 
Good enough practices in scientific computing \cite{wilson2017good}

Software Carpentry: lessons learned \cite{wilson2014software}
(mais uma iniciativa preocupada com as habilidades dos pesquisadores
com computacao, esta dificuldae gera pesquisas dificeis de reproduzir,
repeticao de trabalho, etc.. licoes aprendidas ao longo de mais de 20 anos)

Academic Software Development Tools and Techniques
resumo de um evento local para apresentacao de ferramentas academicas criadas com OO
(esse é uma prática para incentivar colaboração e promover os projetos)
uma das ferramentas que enontrei esta apresentado nesse paper, Rigi

https://cos.io/about/news/center-open-science-receives-grant-james-s-mcdonnell-foundation-study-impact-registered-reports/
The Center for Open Science (COS) is pleased to announce that it has received a
\$165,591 grant from the James S. McDonnell Foundation to undertake two studies
evaluating the impact of Registered Reports (RRs) on research quality and
outcomes. RRs were introduced in 2013 as an innovative method for improving
reproducibility

o artigo com resumo do RESER 2011 diz \cite{knutson2010report}:
4) Re-
search tools are either not available or not usable, so precise
replication is impractical [1, 2, 8, 18, 19].

%Acesso ao software
%
%5º Princípio da citação ao softwares, Acessibilidade:
%
%``citações aos softwares devem permitir e facilitar acesso ao software,
%metadados, documentação, dados e outros materiais necessários tanto
%para humanos quanto para máquinas se informar do referido software''
%
%Não significa que o software deva estar disponível gratuitamente, mas que
%os metadados devem prover informação suficiente para que o software seja
%acessado. Se o software é livre, os metadados devem prover um identificador
%que pode ser resolvido para uma URL apontando para a versão específica
%do software sendo citado.
%
%Pra softwares comerciais, os metadados devem ainda prover informações sobre
%como acessa o software, mas pode ser um número de telefone da empresa que
%vende o software ou o link para um site que venda o software
%
%\cite{smith2016software}
%
%5. Accessibility: Software citations should facilitate access to the software itself and to its
