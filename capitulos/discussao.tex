\xchapter{Discussão}{}
\label{discussao}

Respostas para perguntas gerais. Fazer uma seção para cada.
Cuidado com a questão das funcionalidades de cada tool -- pode ser o motivo de ser mais usada,
mesmo que não seja sustentável.

\begin{itemize}
  \item Os artigos de projetos sustentáveis são mais lidos, mais citados?
  \item Os autores de projetos sustentáveis tem mais publicações 
com o uso de software do que os autores de projetos não sustentáveis?
  \item Os projetos sustentáveis são usados em outras pesquisas?
  \item Os projetos sustentáveis tem contribuidores além dos autores iniciais?
  \item Os projetos sustentáveis tem mais contribuidores?
\end{itemize}

Uma questão interessante é a de ``integridade''. 
O que acontece com os resultados anteriores (publicados)
se houver um erro ou uma mudança substancial da forma de calcular algo?
Na prática, só deveríamos permitir refactorings no código 
de um software acadêmico ou adição de novas funcionalidades?
(Chris)

An so what?

O ecossistema de software acadêmico de análise estática sofre de disfunctional ...?

%Encarar o software acadêmico como a "plataforma" do ecossistema de software.
%Pensar no ecossistema de pesquisa e produção intelectual.

\section{Questões ...} 

\begin{itemize}

  \item Os projetos sustentáveis tem contribuidores além dos autores iniciais?
  \item Os projetos sustentáveis tem mais contribuidores?
  \item Os artigos de projetos sustentáveis são mais lidos, mais citados?
  \item Os autores de projetos sustentáveis tem mais publicações com o uso de software do que os autores?
  \item Os projetos são mais fáceis de manter?  de usar?
\end{itemize}


\section{Os autores de projetos de software acadêmico com maior sustentabilidade técnica têm mais publicações com o uso de software do que os autores de projetos com menor sustentabilidade?}




\section{Trabalhos relacionados}

software acadêmico - Esta definição pode ser encontrada, com algumas variações, pelos nomes de
{\it research tool} \cite{Portillo12},
{\it research-originated software} \cite{Kon2011},
{\it research software} \cite{hettrick2014uk} ou
{\it scientific software} \cite{segal2008developing}.

Este paper
apresenta um conjunto de boas práticas que todo pesquisador pode adotar,
independentemente do seu nível de habilidade em computação. Essas práticas
passam por gerenciamento de dados, programação, colaboração com colegas,
organização de projetos, tracking work, e escrita da manuscritos, sao
desenhados para uma grande variadade de fontes publicadas do noso dia a dia e
do nosso trabalho como voluntário organizando workshopts desde 2010
\cite{wilson2017good}.

FAIR principles \cite{wilkinson2016fair}\footnote{\url{https://www.nature.com/articles/sdata201618}}
Foco em dados de pesquisa. O objetivo é fazer eles serem encontráveis,
acessíveis, interoperável e reusável. Estes princípios podem ser
generalizados para aplicar aos softwares.

Open Science Peer Review Oath\footnote{\url{https://f1000research.com/articles/3-271/v2}}
Concentra-se em potencializar os revisores para exigir acesso aberto aos
softwares, práticas reprodutíveis e revisões transparentes.

Open Access Pledge \cite{holcombe2011openaccess}\footnote{\url{http://www.openaccesspledge.com}}
Concentra-se em publicar softwares e papers em locais de {\it open access}.

Best Practices for Scientific Computing \cite{wilson2014best}
resume as melhores práticas para melhorar a situação onde softwares
academisoc sofrem de manutenabilidade, disponibiliade etc, boas praticas, etc

complemento do artigo acima: 
Good enough practices in scientific computing \cite{wilson2017good}

Software Carpentry: lessons learned \cite{wilson2014software}
(mais uma iniciativa preocupada com as habilidades dos pesquisadores
com computacao, esta dificuldae gera pesquisas dificeis de reproduzir,
repeticao de trabalho, etc.. licoes aprendidas ao longo de mais de 20 anos)

Academic Software Development Tools and Techniques
resumo de um evento local para apresentacao de ferramentas academicas criadas com OO
(esse é uma prática para incentivar colaboração e promover os projetos)
uma das ferramentas que enontrei esta apresentado nesse paper, Rigi

https://cos.io/about/news/center-open-science-receives-grant-james-s-mcdonnell-foundation-study-impact-registered-reports/
The Center for Open Science (COS) is pleased to announce that it has received a
\$165,591 grant from the James S. McDonnell Foundation to undertake two studies
evaluating the impact of Registered Reports (RRs) on research quality and
outcomes. RRs were introduced in 2013 as an innovative method for improving
reproducibility

o artigo com resumo do RESER 2011 diz \cite{knutson2010report}:
4) Re-
search tools are either not available or not usable, so precise
replication is impractical [1, 2, 8, 18, 19].

%Acesso ao software
%
%5º Princípio da citação ao softwares, Acessibilidade:
%
%``citações aos softwares devem permitir e facilitar acesso ao software,
%metadados, documentação, dados e outros materiais necessários tanto
%para humanos quanto para máquinas se informar do referido software''
%
%Não significa que o software deva estar disponível gratuitamente, mas que
%os metadados devem prover informação suficiente para que o software seja
%acessado. Se o software é livre, os metadados devem prover um identificador
%que pode ser resolvido para uma URL apontando para a versão específica
%do software sendo citado.
%
%Pra softwares comerciais, os metadados devem ainda prover informações sobre
%como acessa o software, mas pode ser um número de telefone da empresa que
%vende o software ou o link para um site que venda o software
%
%\cite{smith2016software}
%
%5. Accessibility: Software citations should facilitate access to the software itself and to its
