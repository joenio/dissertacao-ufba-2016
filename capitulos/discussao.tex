\xchapter{Discussão}{}
\label{discussao}

Respostas para perguntas gerais. Fazer uma seção para cada.
Cuidado com a questão das funcionalidades de cada tool -- pode ser o motivo de ser mais usada,
mesmo que não seja sustentável.

Uma questão interessante é a de ``integridade''. 
O que acontece com os resultados anteriores (publicados)
se houver um erro ou uma mudança substancial da forma de calcular algo?
Na prática, só deveríamos permitir refactorings no código 
de um software acadêmico ou adição de novas funcionalidades?
(Chris)

An so what?

O ecossistema de software acadêmico de análise estática sofre de disfunctional ...?

%Encarar o software acadêmico como a "plataforma" do ecossistema de software.
%Pensar no ecossistema de pesquisa e produção intelectual.

\section{Questões ...} 

(pendente)

\section{Os autores de projetos de software acadêmico com maior sustentabilidade técnica têm mais publicações com o uso de software do que os autores de projetos com menor sustentabilidade?}

\section{Trabalhos relacionados}

\citeonline{segal2008developing} investigam como o desenvolvimento de software
acadêmico pode ser melhorado e enfatiza sobre o conhecimento sobre o domínio da
pesquisa muitas vezes necessário ao ator desenvolvedor do software.

\citeonline{knutson2010report} ao discutir as conclusões do evento RESER de
2011 diz que software acadêmico quando não está disponível, não está em estágio
útil, não é uspavel, tornando replicação precisa impraticável.

\citeonline{holcombe2011openaccess} no projeto {\it Open Access Pledge}
\footnote{\url{http://www.openaccesspledge.com}} concentra-se em publicar
softwares e papers em locais de {\it open access}.

\citeonline{portillo2012tools} mostra que grande parte das ferramentas de
software criadas na academia estão em estado inicial de desenvolvimento que
apenas uma pequena porcentagem são testados fora do contexto onde foi
desenvolvido. Cita um mapeamento sistemático com objetivo de encontrar
ferramentas de comunicação e coordenação para suporte a times altamente
distribuidos gograficamente, encontrou 132 ferramentas, para uso em projetos de
software global. A maioria destas ferramentas foram desenvolvidas em centros de
pesquisas, e apenas uma pequena porcentagem (18.9\%) foram testados fora do seu
contexto onde foi desenvolvido \cite{portillo2012tools}.

\citeonline{chaturvedi2013tools}
faz uma revisão dos papers submetidos ao MSR desde 2007 até 2013 e
identifica data sets, ferramentas e técnicas utilizadas pelos autores, mais
da metade dos papers usam ou criam ferramentas, categoriza as ferramentas em
ferramentas novas, ferramentas tradicionais, protótipos e scripts para
mineração de dados.

\citeonline{barnes2013science} no manifesto {\it Science Code Manifesto} com
discurso com foco em código fonte escrito especificamente para processar dados
de publicações, afirma que ``todo código fonte escrito especificamente para
processar dados de uma publicação deve estar disponível para os revisores e
leitores do paper''.

\citeonline{marshall2013tools} num mapeamento feito sobre estudos que criam
ferramentas para apoio a revisão sistemática no domínio de SE, 14 estudos foram
selecionados, ao final apenas 8 tinham proposta de ferramentas, ao final
conclui que as ferramentas encontradas estão em estado inicial de
desenvolvimento.

\citeonline{hettrick2014uk} num estudo com entrevistas mostra que no reino
unido entre todas as áreas da ciência 56\% dos cientistas estão envolvidos no
desenvolvimento de software acadêmico.

\citeonline{wilson2014best} resume as melhores práticas para melhorar a
situação onde softwares academisoc sofrem de manutenabilidade, disponibiliade
etc, boas praticas, etc.

\citeonline{wilson2014software} num resumo sobre as lições aprendidas na iniciativa
{\it Software Carpentry} ...  (mais uma iniciativa preocupada com as
habilidades dos pesquisadores com computacao, esta dificuldae gera pesquisas
dificeis de reproduzir, repeticao de trabalho, etc.. licoes aprendidas ao longo
de mais de 20 anos).

\citeonline{beller2016analyzing}.  avalia e sugere caminhos para melhorar o
desenvolvimento de ferramentas de análise estática para aumentar a adoção.

\citeonline{smith2016software} resume recomendações sobre como citar software
na literatura acadêmica com objetivo de encorajar uma ampla adoção e uma
política consistente para citação entre múltiplas disciplinas.

\citeonline{smith2016software} afirma que ``citações aos softwares devem
permitir e facilitar acesso ao software, metadados, documentação, dados e
outros materiais necessários tanto para humanos quanto para máquinas se
informar do referido software''.

\citeonline{wilkinson2016fair} através do {\it FAIR
principles}\footnote{\url{https://www.nature.com/articles/sdata201618}} com
foco em dados de pesquisa, o objetivo é fazer eles serem encontráveis,
acessíveis, interoperável e reusável. Estes princípios podem ser generalizados
para aplicar aos softwares.

\citeonline{wilson2017good} apresenta um conjunto de boas práticas que todo
pesquisador pode adotar, independentemente do seu nível de habilidade em
computação. Essas práticas passam por gerenciamento de dados, programação,
colaboração com colegas, organização de projetos, tracking work, e escrita da
manuscritos, sao desenhados para uma grande variadade de fontes publicadas do
noso dia a dia e do nosso trabalho como voluntário organizando workshopts desde
2010.

Open Science Peer Review Oath\footnote{\url{https://f1000research.com/articles/3-271/v2}}
Concentra-se em potencializar os revisores para exigir acesso aberto aos
softwares, práticas reprodutíveis e revisões transparentes.
