\xchapter{Planejamento do estudo experimental}
{...}
\label{planejamento}

(pendente)

\section{Hipóteses} \label{hipoteses}

Para responder a questão colocada acima, definimos as seguintes hipóteses:

\begin{enumerate}
  \item[{\bf H0:}] {\em Não existe correlação entre características das
  ferramentas de análise estática e sua manutenabilidade.}
  \item[{\bf H1:}] {\em Existe correlação entre características das ferramentas
  de análise estática e sua manutenabilidade.}
\end{enumerate}

\section{Variáveis}

Variáveis independentes: Características das ferramentas de análise estática.

Variáveis dependentes: Nível de manutenabilidade das ferramentas de análise estática.

\section{Design}

Comparar a manutenabilidade das ferramentas agrupando-as por caractarísticas diferentes.

Avaliar quais características influenciam nas métricas que representam manutenabilidade.

Para garantir o princípio de ``randomization'' irei comparar com o maior número
de características das ferramentas possíveis.

Para garantir o princípio de ``blocking'' irei isolar efeitos colaterais dos
fatores que influenciam nos valores das métricas, como: domínio de aplicação, à
linguagem de programação e ao tamanho do software em número de classes.

Para garantir o princípio de ``balancing'' selecionei o mesmo número de
releases das ferramentas que serão analisadas longitudemente.

Tipo to experimento "One factor with more than two treatments".

Factor: manutenabilidade das ferramentas de análise estática.

Treatment: comparação entre grupos com características distintas.

\section{Instrumentation}

Selecionar as ferramentas de análise estática.

Obter o código-fonte das ferramentas selecionadas.

Definir como calcular a manutenabilidade dessas ferramentas.

Preparar a coleta das métricas para cálculo da manutenabilidade.

Como agrupar os valores das métricas para testes.
