\xchapter{Manutenibilidade dos projetos de software acadêmico de análise estática}
{Este capítulo apresenta ...}

% Introduction
% Background
% Experimental Setup (hipoteses / design)
% Results (data analysis)
% Discussion
% Threats to validity
% Conclusions

Estudar o software acadêmico trazido pelo estudo 1 (mais citados? outro?)

SERIA UM ESTUDO na linha dos estudos de  ATTRACTIVENESS.
Nesse caso, o aspecto de adoção para uso e adoção para contribuição.

\section{Motivação}

%\section{Análise da complexidade estrutural dos softwares científicos}
%\label{complexidade-ferramentas}

Análise da complexidade estrutural dos softwares científicos de análise
estática de código-fonte como forma de avaliar a qualidade interna dos mesmos.

% O manifesto de Dagstuhl ao falar sobre sustentabilidade de software levanta também preocupações em relação à sua qualidade:
%
% \begin{quote}
%   Como podemos garantir a qualidade do software acadêmico? Como podemos
%   monitorar, orientar, divulgar e avaliar a qualidade do software acadêmico?
%   Como podemos gerenciar e garantir confiança entre as equipes de pesquisa
%   acadêmica considerando software desenvolvido para métodos de presquisa e/ou
%   de produção?
% \end{quote}

Poucos estudos sobre avaliação de qualidade interna de softwares fazem isso com
{\it softwares científicos}, algo de extrema importância para compreender como
tais artefatos estão contribuindo para a divulgação do conhecimento e para
replicação dos resultados das pesquisas em engenharia de software.

Avaliar os {\it softwares científicos} do ponto de vista de sua qualidade pode
ajudar a compreender quanta atenção é dada ao seu desenvolvimento, uma vez que
tradicionalmente os seus autores enfrentam problemas com manutenabilidade
\cite{Prlic2012}.

% referencia sobre complexidade:
% Livro: Software Metrics, A Rigorous and Practical Approach [3rd 2015]
% 9.1.1 Structural Complexity Properties
%
%
%We will also
%investigate the maintainability measures taxonomy by
%Oman et al where 92 measures are listed and classified [24].
%[24] Oman, P., Hagemeister, J., and Ash, D., A Definition
%and Taxonomy for Software Maintainability, report
%SETL Report 91-08-TR, University of Idaho, 1991.
%
%\cite{Measurements of Software Maintainability}
%
%Taxonomia, metricas, etc ... sobre manutenabilidade de software.
%
%\cite{Metrics for Assessing a Software System’s Maintainability}



\section{Definição} \label{sec:study2:definition}

% Por que o estudo será realizado?
% explore how the amount of structural complexity and available documentation may influence
% the different levels of participation in an academic software ecosystem.

\subsection{Definição do Objetivo}

\begin{description}
\item{\bf Objeto de estudo.}
O objeto de estudo são projetos de software acadêmico de análise estática e sua sustentabilidade técnica.

\item{\bf Propósito.}
O propósito deste estudo é caracterizar 

\item{\bf Perspectiva.}
A perspectiva considerada é a de cientistas e pesquisadores, isto é,
o cientista ou pesquisador gostaria de conhecer ecossistemas de software acadêmico de análise estática
em termos de ...

\item{\bf Foco de qualidade.}
O principal aspecto de qualidade estudado é a manutenibilidade e complexidade estrutural.

\item{\bf Contexto.}
O estudo foi conduzido com um subconjunto dos projetos de software acadêmico de análise estática
publicados nas conferências ASE e SCAM, e avaliados como sustentáveis.
\end{description}

\subsection{Sumário da Definição}

%% GQM template

Analisar a \textit{software acadêmico de análise estática} % object of study
com o propósito de \textit{caracterizar}  % purpose  % era medir e avaliar - coloquei 'caracterizar'
com respeito a \textit{complexidade estrutural (manutenibilidade)}  % quality focus
na perspectiva do \textit{pesquisador} e do \textit{cientista}% perspective
no contexto de \textit{software acadêmico de análise estática publicado nas conferências ASE e SCAM
e referenciado por artigos publicados nas bases da ACM e IEEE}. % context

Os mais citados / referenciados?


\subsection{Questões de Pesquisa}

Neste estudo as seguintes questões de pesquisa, a respeito do ecossistema de
software acadêmico de análise estática, serão investigadas:


\newcommand{\QuestaoOne}{Existe relação entre o número de menções/usuários ao software e sua complexidade estrutural?}
\newcommand{\QuestaoTwo}{Existe relação entre o número de contribuidores do software e sua complexidade estrutural?}
\newcommand{\QuestaoThree}{Existe relação entre o número de contribuidores-pesquisadores do software e sua complexidade estrutural?}

\begin{description}
  \item [Q1:] \QuestaoOne
  \item [Q2:] \QuestaoTwo
  \item [Q3:] \QuestaoThree
\end{description}

\subsection{Métricas}

• Dependent variables (attractiveness)
– Number of Downloads: a proxy for the number of users of the project;
– Number of Members: a proxy for the number of developers in the project.
ETC.

\section{Planejamento do estudo}

\subsection{Seleção de Contexto}

\subsection{Formulação de Hipóteses}

Hipótese: Relacionar número de menções/usuários à complexidade estrutural.

Hipótese: Relacionar o número de contribuidores-pesquisadores à complexidade estrutural.

Hipótese: Relacionar número de contribuidores (geral) à complexidade estrutural.

\subsection{Design}

Um fator que pode influenciar os resultados é o domínio de aplicação [CITAR TERCEIRO], 
porém, como estudamos apenas ferramentas de análise estática, consideramos este fator
isolado e sem influência sobre o estudo.

Uma vez que tenhamos o conjunto de pares de ferramentas de tamanhos similares
iremos analisar os valores das métricas de complexidade estrutural e custo de
mudança para cada par e interpretar qual das ferramentas possui melhor
manutenabilidade, partimos do princípio que valores menores para cada uma
destas métricas implicam em melhor manutenabilidade.

Esta análise dos pares irá apenas nos dizer quais ferramentas tem melhor
manutenabilidade em comparação com as demais, não é uma informação que
pode ser interpretada de forma geral, ela não diz que a ferramenta tem
ou não uma boa manutenabilidade, indica apenas que uma certa ferramenta
tem melhor manutenabilidade que uma outra.

Com o conhecimento de quais ferramentas tem melhor manutenabilidade iremos
utilizar as dimensões caracterizadas para cada ferramenta para verificar se
tais características influenciam na manutenabilidade, não apenas notar se
influenciam, mas poderemos notar quais características influenciam. Por
exemplo, será que as ferramentas que aceitam como entrada análise da linguagem
Java possuem melhor manutenabilidade do que aquelas que aceitam C++?

Assim, iremos analisar e interpretar cada par de ferramenta de tamanho
similar, sempre à luz de suas características a fim de compreender quais
características influenciam em sua manutenabilidade.


\section{Operação} 

\subsection{Métricas de software}

Ao final, coletamos também algumas métricas do software em relação ao seu
ecossistema de software e em relação a sua qualidade interna: número
total de lançamentos, data e número de versão de cada lançamento, número de
commits e a complexidade estrutural do código fonte.

As informações sobre lançamentos foram coletadas manualmente em arquivos de
changelog, no site do projeto, ou em tags no próprio repositório de código
fonter. O número de commits foi coletado com o uso do git via linha de comando,
o cálculo e coleta da métrica de complexidade estrutural do código fonte foi
coletada com software de análise estática Analizo.
Os dados sobre complexidade estrutural do código fonte não foram utilizados
neste estudo.
% Coloquei a FRASE acima para discutirmos depois.

\subsection{Execução}

Para cada ferramenta de análise estática selecionada, coletamos  suas métricas de código-fonte
com suporte da ferramenta {\it analizo metrics}. 
Esta coleta foi automatizada pelo script {\it
analyze-all-projects}\footnote{http://github.com/joenio/dissertacao-ufba-2016/blob/master/dataset/analyze-all-projects}
escrito durante este estudo e  disponível no
repositório\footnote{http://github.com/joenio/dissertacao-ufba-2016} desta
pesquisa.

Analizo é software livre, distribuído sob a licença GNU General Public License
versão 3. Seu código-fonte, bem como pacotes binários, manuais e tutoriais
podem ser obtidos em \url{http://www.analizo.org}. Todas as ferramentas são
auto-documentadas e podem ser consultadas como páginas de manual UNIX. 
Analizo é escrito em Perl, sua última versão 1.19.1 lançada em 01 de Setembro de 2016
foi a versão utilizada neste estudo.

\subsection{Coleta de dados} \label{analise}

Os dados coletados incluem a caracterização das ferramentas de análise
estática, bem como, os valores das métricas de código-fonte de complexidade
estrutural para cada ferramenta. A coleta das métricas de
código-fonte foi realizada pelo Analizo com o auxílio do script {\em
analyze-all-projects}\footnote{http://github.com/joenio/dissertacao-ufba-2016/blob/master/dataset/analyze-all-projects}. 

Todos os dados foram agregados numa planilha.

%% as releases e a evolucao dos valores de SC e CC (Change Cost), sao elas:
%% 
%% WALA                     13 releases
%% error-prone              13 releases
%% JastAdd                  13 releases
%% SPARTA                   13 releases
%% Indus                    (poucos releases, deixando fora da analise longitudinal)
%% Kiasan/Bogor             (mudou a forma de distribuir ao longo dos releases, dificil obter de forma consistente as versoes)
%% Lotrack                  (sem releases, poucos commits no github, apenas 11)
%% PtYasm                   (nao tem releases disponivel, apenas a ultima versao)
%% srcML                    (releases nao encontrado)
%% GumTree                  (tem apenas 2 releases do repositorio github)
%% Sonar Qube Plug-in       (apenas 4 releases no github)
%% CIVL                     13 releases
%% CodeBoost                (poucos releases, apenas 6 versoes para download no site)
%% 2LS                      (poucos releases, apenas 7)
%% 
% aproveitar perte destas referencias ao justificar o uso de percentis ao inves de média
%
%Observar métricas de código-fonte em nível de projetos de software leva
%ao seguinte desafio: como obter valores de métricas que representem todo o projeto sendo
%que métricas de código-fonte usualmente são calculadas para cada elemento do sistema, como arquivos ou classes?
%
%Este desafio tem sido amplamente discutido em estudos sobre definição de
%intervalos de referência ({\it thresholds}) para métricas de
%código-fonte \cite{Shatnawi2010, Kaur2013, Herbold2011}. Intervalos de
%referência são valores conhecidos para uma dada medida
%\cite[Chapter~2.1]{Lanza2007} com algum valor semantico, por exemplo, se
%medirmos a altura das pessoas e definirmos até 2 metros como alto, então
%pessoas acima de 2 metros serão classificadas como muito altas.
%
%Intervalos de referência podem ser definidos de diversas formas, desde
%abordagens baseadas em modelos estatísticos \cite{Shatnawi2010, Kaur2013}
%até aprendizado de máquina \cite{Herbold2011} e inteligência artificial.
%Entre as inúmeras abordagens, muitas partem de estudos empíricos
%usando softwares da indústria como objeto de estudo, geralmente com
%softwares de domínios específicos, parte-se da coleta de dados de
%métricas de código-fonte e com uso de uma abordagem, ou uma combinação entre
%elas, chega-se aos intervalos.
%
%Estes intervalos são também continuamente avaliados a fim de saber se são
%válidos ou não, as abordagens utilizadas para calcular os intervalos levam em
%consideração inúmeros aspectos na tentativa de validar os valores encontrados,
%como por exemplo a natureza dos dados, se seguem a lei de distribuição de
%potência
%\cite{Wheeldon2003,Potanin2005,Concas2007,Ferreira2009,Yao2009,Clauset2009} ou
%seguem uma distribuição normal
%\cite{Baxter2006,Lanza2007,Herraiz2011,Herraiz2012}, avaliam ainda se possuem
%cauda longa, se são livre de escala, entre outros aspectos.


\section{Análise e Interpretação de Resultados}

(pendente)

\section{Ameaças à validade}

(pendente)

\section{Conclusões}

(pendente)

%% As ferramentas com poucas linhas de código foram excluidas, estas
%% apreentam Change Cost alto, já é conhecido que a definição desta métrica
%% sofre deste problema, apresenta valores altos em projetos muito pequenos,
%% tambem removemos da analise aquelas ferramentas que nao tiveram valor
%% no percentil 75\%, pois a comparacao e analise se dará neste percentil
%% principalmente.

%% Com isso temos 11 projetos, destes iremos analisar longitudalmente
%% as releases e a evolucao dos valores de SC e CC (Change Cost), sao elas:
%% Industria:
%%  FindBugs                 13 releases
%%  Closure Compiler         13 releases analisados
%%  PMD                      13 releases
%%  Smatch                   13 releases
%%  FindSecurityBugs         13 releases
%%  Cppcheck                 13 releases
%%  Splint                   (nao encontrado releases)
%%  WAP                      (apenas 7 releases no site, estou selecionando os que tenham ao menos 13 releases)

%% 
%% Comparacao entre ferramentas de tamanho similar:
%% 
%% nas 5 comparações de versões distintas com tamanhos similares entre pmd e findbugs,
%% apresentaram o mesmo resultado, pmd tem valores menos tanto para CC quanto para SC,
%% indicando que pmd tem um design mais modular que findbugs.
%% 
%% pmd 5.0.0 < findbugs 1.2.1
%% pmd 5.0.4 < findbugs 1.2.1
%% pmd 5.1.3 < findbugs 1.3.5
%% pmd 5.2.0 < findbugs 1.3.8
%% pmd 5.3.3 < findbugs 1.3.9
%% 
%% Ao comparar as imagens da matrix DSM dá para notar que isto reflete na matrix, 
%% pegando o findbugs 3.0.1 e o pmd 5.2.0, é possível notar na matrix que o findbugs
%% tem mais pontos nas duas diagonais da matrix, indicando dependencias ciclicar, e
%% design menos modular, enquanto o pmd concentra as dependencias na diagonal inferior
%% esquerda, indicando poucas dependencias ciclicar e um design mais modular.
%% 
%% findbugs	 findbugs-3.0.1	1486	0,07	8	25	56
%% pmd	 pmd-src-5.2.0	1295	0,02	6	12	25
%% 
%% /home/joenio/src/dissertacao-ufba-2016/dataset/static-analysis-tools/pmd/pmd-src-5.2.0.analizo.dsm.png
%% /home/joenio/src/dissertacao-ufba-2016/dataset/static-analysis-tools/findbugs/findbugs-3.0.1.analizo.dsm.png
%% 
%% accessanalysis < findsecuritybugs
%% indus < bogor
%% reassert > jflow
%% pmd-5.4.0 > pmd-5.3.0
%% pmd-5.4.2 > pmd-5.3.7
%% findbugs-3.0.1 > findbugs-2.0.3
%% pmd < closure-compiler
%% pixy > mpanalyzer
%% ejb > mpanalyzer
%% ejb > sparta
%% 
%% closure-compiler > wala
%% closure-compiler > wala
%% closure-compiler > wala
%% closure-compiler[Java] ? wala[Java]     (closure tem CC maior mas SC menor)
%% 
%% comparar linguagens diferentes não rola, sempre dá ruim, ver:
%% 
%% rats[C]        ? uno[C]                 (rats tem CC menor e SC maior)
%% cppcheck[C++]  ? wap[Java]              (cppcheck tem CC menor mas SC maior)
%% srcml[C++]     ? ptyasm[Java]           (srcml tem CC maior mas SC menor)
%% closure-compiler[Java] ? inputtracer[C] (closure tem CC menor mas SC diferentes nos percentis)
%% pmd[Java] ? srcml[C++]                  (pmd tem CC maior e SC diferentes nos percentis)
%% inputtracer[C] ? wala[Java]             (tem CC maior e SC diferentes mas no percentil 95 tem SC maior também)
%% 
%% fica claro que comparacao entre linguagens diferentes mesmo com tamanhos iguais não dá para chegar a conclusões nenhuma.
%% 
%% findbugs[Java] ? wala[Java]             (findbugs tem CC maior mas SC menor)
%% closure-compiler ? closure-compiler     (CC menor mas SC maior)
%% 
%% comparacao (v3) - ordenado por eloc - comparando apenas SC 95
%% ===============
%% 
%% % findsecbugs-plugin-1.4.0-sources < sparta-code-0.6
%% % findsecbugs-plugin-1.4.1-sources < sparta-code-0.7
%% % findsecbugs-plugin-1.4.4-sources < sparta-code-0.8
%% 
%% % cseq-0.5 > find-sec-bugs-version-1.0.0
%% 
%% % sparta-code-0.9.2 < tacle_1_2_1_src
%% % sparta-code-0.9.4 < jastadd2-src-2.1.5
%% % MPAnalyzer-master > sparta-toolset-0.9.8
%% % ReAssert_0.4.1 > sparta-sparta-1.0.2
%% % sparta-sparta-1.0.2 < uno
%% % sparta-toolset-1.0.1-source < cppcheck-1.30
%% % SonarQube-plug-in-master > sparta-toolset-1.0.0-source
%% 
%% % findsecbugs-plugin-1.4.5-sources < rats-2.4
%% % jastadd2-src-2.1.2 > findsecbugs-plugin-1.4.6-sources
%% % find-sec-bugs-version-1.1.0 < jastadd2-src-2.1.4
%% % AccessAnalysis-1.2-src < jastadd2-src-2.1.9
%% % jastadd2-src-2.1.13 > jlint-3.1.2
%% % vazexqi-JFlow-7cd7eaf < gumtree-2.0.0
%% % composite-0.4 < smatch-1.0
%% % smatch-0.3 > EJB
%% % smatch-0.4 > cppcheck-1.35
%% % cppcheck-1.35 > guizmo-master
%% % smatch-1.51 < cqual-0.981
%% % pixy-master < smatch-1.52
%% % error-prone-2.0 < smatch-1.54
%% % smatch-1.54 > cppcheck-1.40
%% % smatch-1.55 > error-prone-2.0.2
%% % indus < smatch-1.56
%% % smatch-1.56 > error-prone-2.0.4
%% % smatch-1.59 > pmd-src-5.0.4
%% % error-prone-2.0.6 < pmd-src-4.2.5
%% % pmd-src-4.2.5 < smatch-1.60
%% % smatch-1.60 < cppcheck-1.45
%% % ptyasm > error-prone-2.0.8
%% % error-prone-2.0.8 < bogor-core
%% % pmd-src-5.1.0 > error-prone-2.0.9
%% % pmd-src-5.3.7 < wap-2.1
%% % error-prone-2.0.12 < pmd-src-5.5.2
%% % error-prone-2.0.13 < cppcheck-1.50
%% % cppcheck-1.50 > wala-code-4607-tags-R_1.0
%% % findbugs-1.2.1-source > error-prone-2.0.14
%% % cppcheck-1.55 > findbugs-1.3.4-source
%% % findbugs-1.3.8-source < cppcheck-1.60
%% % findbugs-1.3.9-source = wala-code-4607-tags-R_1.0.02
%% % wala-code-4607-tags-R_1.2 < cppcheck-1.62
%% % findbugs-2.0.2-source > wala-code-4607-tags-R_1.1
%% % cppcheck-1.65 > wala-code-4607-tags-R_1.2.2
%% % wala-code-4607-tags-R_1.3 < findbugs-3.0.0-source
%% % findbugs-3.0.1 > WALA-R_1.3.3
%% % WALA-R_1.3.3 < cppcheck-1.70
%% % cppcheck-1.75 > WALA-R_1.3.5
%% % WALA-R_1.3.6 > closure-compiler-20110119
%% % closure-compiler-20110119 < cppcheck-1.77
%% % cppcheck-1.77 < splint-3.1.2
%% % srcML-src > closure-compiler-20140730
%% % closure-compiler-20160713 > Lotrack-master
%% 
%% O percentil 75 tem muitos valores zero, os percentis 90 e 95 sao pracitamente iguais 
%% na comparacao, os maiores sao geralmente tb maior no outro, exceto uns 2 exemplos:
%% smatch-0.3/EJB e pmd-src-5.3.7/wap-2.1.
%% 
%% comparacao (v3) - ordenado por n modulos - comparando apenas SC 90 e 95
%% ===============
%% 
%% rats-2.4 > uno
%% jlint-3.1.2 > findsecbugs-1.2.0
%% jastadd2-2.1.5 > findsecbugs-1.2.1
%% findsecbugs-1.3.0 < jastadd2-2.1.8
%% jastadd2-2.2.2 > findsecbugs-1.4.0
%% findsecbugs-1.4.2 < cqual-0.981
%% sparta-0.5 < findsecbugs-1.4.4
%% findsecbugs-1.4.5 > AccessAnalysis-1.2
%% AccessAnalysis-1.2 < cppcheck-1.30
%% cppcheck-1.30 < smatch-1.0
%% smatch-0.2 > findsecbugs-1.4.6
%% findsecbugs-1.4.6 > sparta-0.6
%% sparta-0.7 < smatch-0.3
%% cseq-0.5 > findsecbugs-1.5.0
%% smatch-0.4 > cppcheck-1.35
%% cppcheck-1.35 > sparta-0.8
%% cppcheck-1.40 > sparta-0.9.2
%% sparta-0.9.2 > findsecbugs-1.0.0
%% SonarQube-plug-in-master < smatch-1.51
%% smatch-1.52 > ReAssert\_0.4.1
%% smatch-1.53 > jfLow
%% gumtree-2.0.0 < cppcheck-1.45
%% smatch-1.54 > sparta-0.9.8
%% sparta-0.9.8 < pixy
%% cppcheck-1.50 > findsecbugs-1.1.0
%% findsecbugs-1.1.0 < MPAnalyzer
%% MPAnalyzer < EJB
%% sparta-1.0.1 < cppcheck-1.55
%% guizmo < cppcheck-1.60
%% smatch-1.56 < cppcheck-1.70
%% wap-2.1 < cppcheck-1.72
%% cppcheck-1.75 > smatch-1.58
%% pmd-4.3 < srcML
%% srcML < ptyasm
%% pmd-5.0.0 < findbugs-1.2.1
%% pmd-5.0.4 > error-prone-2.0
%% closure-compiler-20110119 > error-prone-2.0.2
%% error-prone-2.0.4 < findbugs-1.3.4
%% findbugs-1.3.4 < wala-4607-R1.0
%% wala-4607-R1.0 > pmd-5.1.0
%% pmd-5.1.3 < findbugs-1.3.5
%% findbugs-1.3.8 > pmd-5.2.0
%% pmd-5.3.3 < findbugs-1.3.9
%% findbugs-1.3.9 > pmd-5.4.2
%% pmd-5.3.7 < findbugs-3.0.0
%% findbugs-2.0.3 > error-prone-2.0.5
%% pmd-5.5.2 > error-prone-2.0.6
%% closure-compiler-20140730 > error-prone-2.0.7
%% error-prone-2.0.8 < closure-compiler-20150729
%% error-prone-2.0.9 < wala-4607-R1.1.2
%% wala-4607-R1.1.2 < closure-compiler-20160125
%% closure-compiler-20110811 < wala-4607-R1.2
%% error-prone-2.0.11 < closure-compiler-20160517
%% closure-compiler-20160713 > error-prone-2.0.12
%% error-prone-2.0.12 < wala-4607-R1.1
%% wala-4607-R1.2.2 > error-prone-2.0.13
%% error-prone-2.0.14 < wala-4607-R1.3
%% error-prone-2.0.15 < closure-compiler-20140110
%% 
%% %% De forma que somando as ferramentas selecionadas na academia e na indústria
%% %% temos um total de 34 ferramentas, 14 da indústria e 20 da academia.  
%% %% 
%% %% \begin{table}[H]
%% %%   \caption{Resumo da caracterização das ferramentas}
%% %%   \centering
%% %%   \begin{tabular}{| c | l | l | c | l | l |}
%% %%     \hline
%% %%     \# & Ferramentas da indústria & Linguagem & Classes & Lançamentos \\
%% %%     \hline
%% %%     22 & Closure Compiler         & Java  & 1842  & Frequentemente \\
%% %%     23 & Cppcheck                 & C++   & 338   & Frequentemente \\
%% %%     24 & CQual                    & C     & 78    & Obsoleta       \\
%% %%     25 & FindBugs                 & Java  & 1486  & Ocasionalmente \\
%% %%     26 & FindSecurityBugs         & Java  & 91    & Frequentemente \\
%% %%     27 & Jlint                    & C++   & 44    & Obsoleta       \\
%% %%     28 & Pixy                     & Java  & 229   & Obsoleta       \\
%% %%     29 & PMD                      & Java  & 1340  & Frequentemente \\
%% %%     30 & RATS                     & C     & 19    & Obsoleta       \\
%% %%     31 & Smatch                   & C     & 483   & Ocasionalmente \\
%% %%     32 & Splint                   & C     & 681   & Obsoleta       \\
%% %%     33 & UNO                      & C     & 19    & Obsoleta       \\
%% %%     34 & WAP                      & Java  & 338   & Frequentemente \\
%% %%     \hline
%% %%   \end{tabular}
%% %%   \label{total-de-ferramentas}
%% %% \end{table}
%% 

%Os valores encontrados serão avaliados sempre tendo em vista os intervalos
%sugeridos na Tabela \ref{valores-frequentes}, esta tabela traz os valores encontrados
%no estudo que estamos replicando em parte \cite{Meirelles2013}.
%
%\begin{table}[H]
%  \caption{Valores frequentes\cite{Meirelles2013}}
%  \centering
%  \begin{tabular}{| c | l | l | l | l | l |}
%    \hline
%    Métrica           & Linguagem & Muito frequente & Frequente & Pouco frequente & Não frequente \\
%    \hline
%\multirow{3}{*}{CBO}   & C         & 0 -- 5,0   & 6,0 -- 9,0   & 9,0 -- 12,0  & $>$ 12,0  \\
%                       & C++       & 0 -- 3,0   & 4,0 -- 5,0   & 6,0 -- 7,0   & $>$ 7,0   \\
%                       & Java      & 0 -- 3,0   & 4,0 -- 6,0   & 7,0 -- 9,0   & $>$ 9,0   \\
%    \hline
%\multirow{3}{*}{LCOM4} & C         & 0 -- 5,0   & 6,0 -- 12,0  & 12,0 -- 20,0 & $>$ 20,0  \\
%                       & C++       & 0 -- 5,0   & 6,0 -- 10,0  & 10,0 -- 14,0 & $>$ 14,0  \\
%                       & Java      & 0 -- 3,0   & 4,0 -- 7,0   & 8,0 -- 12,0  & $>$ 12,0  \\
%    \hline
%\multirow{3}{*}{SC}    & C         & 0 -- 18,0  & 19,0 -- 77,0 & 78,0 -- 168,0 & $>$ 168,0 \\
%                       & C++       & 0 -- 12,0  & 13,0 -- 28,0 & 29,0 -- 51,0  & $>$ 51,0  \\
%                       & Java      & 0 -- 6,0   & 7,0 -- 21,0  & 22,0 -- 45,0  & $>$ 45,0  \\
%    \hline
%  \end{tabular}
%  \label{valores-frequentes}
%\end{table}

%\section{Design}

%No entando é conhecido que alguns fatores inflenciam o valor de algumas métricas,
%para evitar tais influências iremos isolar estes fatores realizando comparações
%entre ferramentas com os mesmos fatores, por exemplo, comparação entre linguagens diferentes,
%domínio de aplicação diferentes, tamanho em número de classes.

%O estudo é um experimento com um fator e mais de dois tratamentos, o fator
%neste estudo é a manutenabilidade das ferramentas de análise estática e o
%tratamento será uma série de comparações entre grupos distintos de ferramentas
%com características comuns.

%Para garantir o princípio de ``randomization'' irei comparar com o maior número
%de características das ferramentas possíveis.
%Para garantir o princípio de ``balancing'' selecionei o mesmo número de
%releases das ferramentas que serão analisadas longitudemente.

%A investigação será realizada a partir de uma busca e seleção de ferramentas de
%análise estática, em seguida para cada ferramenta selecionada iremos obter
%o código-fonte da ferramenta, com código-fonte em mão iremos calcular métricas
%de complexidade estrutural e custo de mudança, em paralelo as características
%dessas ferramentas serão documentadas, neste ponto a análise e interpretação
%dos dados se iniciará, o objetivo será compreender quais características
%implicam na manutenabilidade.

% * fugir de valores de referencia
% * avaliar evolucao de grupos de ferramentas, ex: plugins eclipse academia VS plugins eclipse industria
% * agregar um pouco dos protocolos da revisao estruturada na revisao estruturada
% * mostrar graficos com evolucao de SC de cada grupo/ferramenta
% * discutir essa evolucao em cada grupo/ferramenta, mostrando que aqui esta evoluindo pra melhor, aqui pra pior, etc
% * essas ferramentas fezem da fato o que o autor informa que faz? é válida? é possível validar isso de que forma?
%   verificar quais estudos tem validacao cientifica, ou seja, fazem estudo de caso? experimento? etc...?

%outro fator de peso para definir o valor final do peso é se houve lençamentos
%de novas versões do software naquele ano, se houve ao menos 1 versão lançada,
%isso leva o final\_weight para o valor máximo 1.0 (que representa 100\%)

