\xchapter{Reproducibilidade do estudo}
{}
\label{reproducibilidade-do-estudo}

Este trabalho de pesquisa, incluindo texto, imagens, dados e códigos estão
todos disponíveis e públicos sob a licença CC BY-SA 4.0 no seguinte endereço:

\begin{itemize}
  \item \url{https://github.com/joenio/dissertacao-ufba-2016}
\end{itemize}

A maior parte das atividades de comunicação e reuniões de orientação realizadas
neste estudo estão documentadas em {\it issues} neste repositório ou na wiki do
grupo de pesquisa aSide, em:

\begin{itemize}
  \item \url{http://wiki.dcc.ufba.br/Aside/Orientacao2014JoenioCosta}
\end{itemize}

O texto foi escrito em \LaTeX \ e a maior parte dos códigos desenvolvidos foram
feitos em Perl, a organização do repositório e localização de cada arquivo
fonte está descrito na Tabela \ref{arquivos-repositorio}, alguns elementos
serão detalhados à seguir.

% versão de cada software utilizado, ambiente, Debian, Perl, Analizo
% libre office, Evince, YAML, BibTeX, navegador, etc

\begin{table}[h]
\caption{Organização de arquivos e pastas do repositório.}
\centering
\begin{tabular}{ l p{10cm} }
  \hline
  Arquivo ou pasta            & Descrição           \\
  \hline
  \texttt{bibliografia.bib}   & Arquivo BibTeX com as referências bibliográficas utilizadas no texto. \\
  \texttt{bin/}               & Pasta contendo os scripts desenvolvidos para coleta, transformação e análise de dados. \\
  \texttt{capitulos/}         & Arquivos \LaTeX \ com o conteúdo/texto de cada capítulo da dissertação. \\
  \texttt{dataset/}           & Dados coletados durante os estudos apresentados nos Capítulos \ref{estudo1}, \ref{estudo2} e \ref{estudo3}. \\
  \texttt{dissertacao.tex}    & Arquivo \LaTeX \ principal com inclusão dos capítulos, título e resumo do texto. \\
  \texttt{documents/}         & Documentos gerados pelos templates com apresentação dos dados coletados. \\
  \texttt{imagens/}           & Imagens, gráficos e demais elementos visuais utilizados no texto. \\
  \texttt{lib/}               & ...                 \\
  \texttt{Makefile}           & Conujunto de regras para execução dos scripts e compilação do código fonte \LaTeX \ em PDF.                 \\
  \texttt{README.md}          & ...                 \\
  \texttt{templates/}         & Templates (modelos) para leitura dos dados coletados e transformação em documentos. \\
  \texttt{ufbathesis.cls}     & Arquivo de estilo do \LaTeX \ com definições e padrão de formatação deacordo as regras de publicação da UFBA. \\
  \hline
\end{tabular}
\label{arquivos-repositorio}
\end{table}

\section{Organização do repositório}

\subsection{\texttt{bin/}}

detalhes de implementação e descrição de cada script

instruções de uso

\subsection{\texttt{dataset/}}

falar da planilha de revisao de literatura do estudo1

apresentar a estrutura de diretórios e arquivos para cada software

\subsection{\texttt{templates/}}

% TODO
% documentar instalação das dependencias do script para filtro
% documentar a instaação do sloccount
% descrever o formato YAML utilizado para caracterização dos projetos

%produz um resultado indicando todas as linguagens e
%quanto do código total é escrito em cada uma delas.

%detalhes sobre o formato YAML, estrutura utilizada para armazenamento e
%instalação estão documentados no Apêndice \ref{reproducibilidade-do-estudo}.

%Detalhes sobre o funcionamento do script de fitro, código fonte e instruções
%de uso podem ser consultados no Apêndice \ref{reproducibilidade-do-estudo}.

%Instruções de uso do sloccount e como ele foi utilizado neste estudo está
%documentado no Apêndice \ref{reproducibilidade-do-estudo}.

%com estas palavras num script, instalamos todas as suas
%dependências de execução, os detalhes de instalação e forma de uso deste script
%e suas dependências são documentados no Apêndice
%\ref{reproducibilidade-do-estudo}.

%O formato desta planilha , sobre a estrutura de pastas utilizadas
%para armazenar os artigos, e onde encontrar o arquivo utilizado neste estudo
%para a coleta destes dados pode ser consultado em detalhes no Apêndice
%\ref{reproducibilidade-do-estudo}.

%Todos estes scripts estão disponíveis no repositório desta dissertação e foram desenvolvidos
%utilizando a a linguagem de
%programação Perl\footnote{\url{http://perl.org}} com o auxílio dos
%módulos Modern::Perl\footnote{\url{http://metacpan.org/pod/Modern::Perl}},
%YAML\footnote{\url{http://metacpan.org/pod/YAML}},
%Mojo::Template\footnote{\url{http://metacpan.org/pod/Mojo::Template}},
%Text::BibTeX\footnote{\url{http://metacpan.org/pod/Text::BibTeX}} e
%List::Util\footnote{\url{http://metacpan.org/pod/List::Util}}.
%A maior parte da lógica foi implementada no arquivo
%\texttt{lib/Dissertacao.pm} com objetivo de reduzir repetição de código e
%proporcionar reuso.

%Um documento
%CSV com todas as strings de busca pode ser encontrado no repositório desta
%dissertaçao no arquivo \texttt{documents/search-strings.csv}.

%Para auxiliar a definição da identificação para cada artigo implementamos o script \texttt{bin/ids}, este script


Apresentar detalhes de implementação, execução e uso do script bin/run-analizo, bem como detalhes para instalação do Analizo.

Apresentar e detalhar os arquivos scam-links.md e ase-link.md (citado no estudo1:preparacao).

Os artigos analisados na revisão estruturada estão todos documentados arquivo
{\it
dataset/dataset.ods}\footnote{\url{http://github.com/joenio/dissertacao-ufba-2016/blob/master/dataset/dataset.ods}},
uma planilha no formato aberto {\it Open Document Format for Office
Applications}\footnote{\url{http://www.oasis-open.org/committees/office}}.

Nesta planilha está documentada cada etapa da revisão estruturada, indicando em
cada artigo analisado qual o estado do mesmo, se foi ou não incluído na
execução da atividade.  Nesta planilha é possível encontrar também o nome de
cada ferramenta e uma caracterização completa.

O script utilizado na segunda atividade da revisão estruturada -- {\it (2)
Filtro} -- também está neste mesmo repositório no arquivo {\it
dataset/revisao-estruturada/filter}\footnote{\url{http://github.com/joenio/dissertacao-ufba-2016/blob/master/revisao-estruturada/filter}}
escrito em linguagem Perl especialmente para este estudo.

