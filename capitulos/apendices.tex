\xchapter{Revisão estruturada}
{Este capítulo apresenta as detalhes e números da revisão estruturada.}
\label{apendice-revisao-estruturada}

\section{Edições das conferências revisadas}
\label{edicoes-conferencias}

\subsection{SBES - Brazilian Symposium on Software Engineering}

Link para a conferência no IEEE Xplore: http://ieeexplore.ieee.org/servlet/opac?punumber=1800181
Link para o repositório BDBComp: http://www.lbd.dcc.ufmg.br/bdbcomp/servlet/PesquisaEvento?evento=SBES

1987 main
1988 main
1989 main
1990 main
1991 main
1992 main
1993 main
1994 main
1995 main
1996 main
1997 main
1998 main
1999 main + tools OK
2000 ???? - mail p chair <acas@cin.ufpe.br> sem resposta 
2001 main + tools OK
2002 main + tools OK
2003 main - mail p chair <ferramentas-sbes2003@unifor.br> sem resposta
2004 main + tools OK
2005 main + tools OK
2006 main + tools OK
2007 main + tools OK
2008 main + tools OK
2009 main + tools OK
2010 main + tools OK
2011 main + tools OK
2012 main + tools OK
2013 main + tools OK
2014 main + tools OK
2015 main + tools OK

wget -r -np -l 2 -A pdf http://www.lbd.dcc.ufmg.br/colecoes/sbes/

\subsection{SCAM - Source Code Analysis and Manipulation Working Conference}

\begin{itemize}
  \item SCAM 2001 - {\small http://ieeexplore.ieee.org/xpl/mostRecentIssue.jsp?punumber=7667}
  \item SCAM 2002 - {\small http://ieeexplore.ieee.org/xpl/mostRecentIssue.jsp?punumber=6494367}
  \item SCAM 2003 - {\small http://ieeexplore.ieee.org/xpl/mostRecentIssue.jsp?punumber=8773}
  \item SCAM 2004 - {\small http://ieeexplore.ieee.org/xpl/mostRecentIssue.jsp?punumber=9523}
  \item SCAM 2005 - {\small http://ieeexplore.ieee.org/xpl/mostRecentIssue.jsp?punumber=10344}
  \item SCAM 2006 - {\small http://ieeexplore.ieee.org/xpl/mostRecentIssue.jsp?punumber=4026839}
  \item SCAM 2007 - {\small http://ieeexplore.ieee.org/xpl/mostRecentIssue.jsp?punumber=4362882}
  \item SCAM 2008 - {\small http://ieeexplore.ieee.org/xpl/mostRecentIssue.jsp?punumber=4637522}
  \item SCAM 2009 - {\small http://ieeexplore.ieee.org/xpl/mostRecentIssue.jsp?punumber=5279860}
  \item SCAM 2010 - {\small http://ieeexplore.ieee.org/xpl/mostRecentIssue.jsp?punumber=5600365}
  \item SCAM 2011 - {\small http://ieeexplore.ieee.org/xpl/mostRecentIssue.jsp?punumber=6063701}
  \item SCAM 2012 - {\small http://ieeexplore.ieee.org/xpl/mostRecentIssue.jsp?punumber=6389882}
  \item SCAM 2013 - {\small http://ieeexplore.ieee.org/xpl/mostRecentIssue.jsp?punumber=6636284}
  \item SCAM 2014 - {\small http://ieeexplore.ieee.org/xpl/mostRecentIssue.jsp?punumber=6970367}
  \item SCAM 2015 - {\small http://ieeexplore.ieee.org/xpl/mostRecentIssue.jsp?punumber=7321933}
\end{itemize}

\subsection{ASE - Automated Software Engineering}

Até o ano de 1996 a conferencia ASE se chamava KBSE - Knowledge-Based Software
Engineering Conference e a partir de 1997 passou a se chamar ASE - Automated
Software Conference.

\begin{itemize}
  \item ASE/KBSE 1991 - {\small http://ieeexplore.ieee.org/xpl/mostRecentIssue.jsp?punumber=5069}
  \item ASE/KBSE 1992 - {\small http://ieeexplore.ieee.org/xpl/mostRecentIssue.jsp?punumber=421}
  \item ASE/KBSE 1993 - {\small http://ieeexplore.ieee.org/xpl/mostRecentIssue.jsp?punumber=921}
  \item ASE/KBSE 1994 - {\small http://ieeexplore.ieee.org/xpl/mostRecentIssue.jsp?punumber=995}
  \item ASE/KBSE 1995 - {\small http://ieeexplore.ieee.org/xpl/mostRecentIssue.jsp?punumber=3518}
  \item ASE/KBSE 1996 - {\small http://ieeexplore.ieee.org/xpl/mostRecentIssue.jsp?punumber=4065}
  \item ASE 1997 - {\small http://ieeexplore.ieee.org/xpl/mostRecentIssue.jsp?punumber=5003}
  \item ASE 1998 - {\small http://ieeexplore.ieee.org/xpl/mostRecentIssue.jsp?punumber=5935}
  \item ASE 1999 - {\small http://ieeexplore.ieee.org/xpl/mostRecentIssue.jsp?punumber=6516}
  \item ASE 2000 - {\small http://ieeexplore.ieee.org/xpl/mostRecentIssue.jsp?punumber=7013}
  \item ASE 2001 - {\small http://ieeexplore.ieee.org/xpl/mostRecentIssue.jsp?punumber=7763}
  \item ASE 2002 - {\small http://ieeexplore.ieee.org/xpl/mostRecentIssue.jsp?punumber=8183}
  \item ASE 2003 - {\small http://ieeexplore.ieee.org/xpl/conhome.jsp?punumber=1000064}
  \item ASE 2004 - {\small http://ieeexplore.ieee.org/xpl/mostRecentIssue.jsp?punumber=9305}
  \item ASE 2005 - {\small http://dl.acm.org/citation.cfm?id=1101908}
  \item ASE 2006 - {\small http://ieeexplore.ieee.org/xpl/mostRecentIssue.jsp?punumber=4019543}
  \item ASE 2007 - {\small http://dl.acm.org/citation.cfm?id=1321631}
  \item ASE 2008 - {\small http://ieeexplore.ieee.org/xpl/mostRecentIssue.jsp?punumber=4639292}
  \item ASE 2009 - {\small http://ieeexplore.ieee.org/xpl/mostRecentIssue.jsp?punumber=5431684}
  \item ASE 2010 - {\small http://dl.acm.org/citation.cfm?id=1858996}
  \item ASE 2011 - {\small http://ieeexplore.ieee.org/xpl/mostRecentIssue.jsp?punumber=6093623}
  \item ASE 2012 - {\small http://ieeexplore.ieee.org/xpl/mostRecentIssue.jsp?punumber=6494367}
  \item ASE 2013 - {\small http://ieeexplore.ieee.org/xpl/mostRecentIssue.jsp?punumber=6684409}
  \item ASE 2014 - {\small http://dl.acm.org/citation.cfm?id=2642937}
  \item ASE 2015 - {\small http://ieeexplore.ieee.org/xpl/mostRecentIssue.jsp?punumber=7371449}
\end{itemize}

\section{Softwares científicos selecionados}
\label{softwares-cientificos}

Resumo dos softwares selecionados na revisão estruturada.

\begin{table}[h]
\caption{Resultados da revisão estruturada para cada edição do SCAM}
\centering
\begin{tabular}{| l | c | c | c |}
  \hline
  Edição & (1) Busca & (2) Filtro & (3) Seleção \\
  \hline
  SCAM 2001 & 23    & 6         & 1           \\
  SCAM 2002 & 18    & 6         & 5           \\
  SCAM 2003 & 21    & 8         & 3           \\
  SCAM 2004 & 17    & 3         & 1           \\
  SCAM 2005 & 19    & 7         & 1           \\
  SCAM 2006 & 22    & 10        & 7           \\
  SCAM 2007 & 23    & 7         & 2           \\
  SCAM 2008 & 29    & 14        & 2           \\
  SCAM 2009 & 20    & 10        & -           \\
  SCAM 2010 & 21    & 15        & 5           \\
  SCAM 2011 & 21    & 10        & 2           \\
  SCAM 2012 & 22    & 12        & 4           \\
  SCAM 2013 & 24    & 13        & 2           \\
  SCAM 2014 & 36    & 16        & 4           \\
  SCAM 2015 & 30    & 18        & 2           \\
  \hline
  Total     & 346   & 155       & 41          \\
  \hline
\end{tabular}
\label{artigos-do-scam}
\end{table}

\begin{table}[h]
\caption{Resultados da revisão estruturada para cada edição do ASE}
\centering
\begin{tabular}{| l | c | c | c |}
  \hline
  Edição & (1) Busca & (2) Filtro & (3) Seleção \\
  \hline
  ASE 1991 & 28    & -         & -           \\
  ASE 1992 & 25    & -         & -           \\
  ASE 1993 & 21    & -         & -           \\
  ASE 1994 & 23    & -         & -           \\
  ASE 1995 & 23    & -         & -           \\
  ASE 1996 & 15    & -         & -           \\
  ASE 1997 & 47    & 1         & -           \\
  ASE 1998 & 44    & 4         & -           \\
  ASE 1999 & 50    & -         & -           \\
  ASE 2000 & 44    & 2         & -           \\
  ASE 2001 & 68    & 7         & 2           \\
  ASE 2002 & 46    & 5         & -           \\
  ASE 2003 & 54    & 5         & 2           \\
  ASE 2004 & 68    & 7         & -           \\
  ASE 2005 & 79    & 9         & 1           \\
  ASE 2006 & 61    & 12        & 2           \\
  ASE 2007 & 102   & 19        & 5           \\
  ASE 2008 & 90    & 18        & 5           \\
  ASE 2009 & 89    & 19        & 11          \\
  ASE 2010 & 87    & 20        & 3           \\
  ASE 2011 & 112   & 28        & 5           \\
  ASE 2012 & 68    & 16        & 5           \\
  ASE 2013 & 90    & 28        & 7           \\
  ASE 2014 & 100   & 39        & 7           \\
  ASE 2015 & 99    & 42        & 7           \\
  \hline
  Total    & 1533  & 281       & 62          \\
  \hline
\end{tabular}
\label{artigos-do-ase}
\end{table}

\begin{table}[h]
{\scriptsize
\caption{Resumo dos 60 softwares selecionados na revisão estruturada}
\centering
\begin{tabular}{| l | l |}
  \hline
  Software       & Descrição                            \\
  \hline
  2LS            & análise de terminação de programas   \\
  AccessAnalysis & cálculo de métricas IGAT e IGAM      \\
  APIExample     & documentação de API Java             \\
  BEG            & verificação de modelos Java          \\
  ccJava         & linguagem orientada a aspectos       \\
  CIVL           & verificação de programa concorrente  \\
  CodeBoost      & transformação source-to-source C++   \\
  composite      & verificação de modelos               \\
  CPA+           & análise configurável de programa     \\
  CSeq           & transformação source-to-source       \\
  DCR            & ?                                    \\
  DDVerify       & checagem de modelos                  \\
  Derailer       & localização de falhas de segurança   \\
  Diagnosys      & construção de interfaces de debug    \\
  DOMPLETION     & sugestão de código javascript        \\
  e-munity       & verificação de segurança             \\
  EJB            & criação de diagramas de sequência    \\
  error-prone    & localização de bugs                  \\
  ESBMC          & verificação de modelos               \\
  ETXL           & transformação de código              \\
  FaultBuster    & refatoração de code smells           \\
  Flowgen        & criação de grafos UML                \\
  GRT            & geração automática de testes         \\
  GUIZMO         & inferência de layout                 \\
  GumTree        & análise e comparação de mudanças     \\
  HUSACCT        & ?                                    \\
  Indus          & biblioteca de program slicing        \\
  JastAdd        & descrição e gramática de atributos   \\
  JFlow          & transformação source-to-source       \\
  JstereoCode    & detecção de esteriótipos Java        \\
  Jtop           & gestão de casos de teste             \\
  Kiasan/Bogor   & verificação de modelos               \\
  Loopfrog       & verificação de modelos               \\
  Lotrack        & análise estática de configuração     \\
  MPAnalyzer     & análise de padrões                   \\
  MSP            & criação de modelo de acesso a memória\\
  mygcc          & verificação de programas C           \\
  PARSEWeb       & sugestão de uso de código e bibliotecas \\
  PAT            & ambiente de teste automático         \\
  PHP AiR        & ?                                    \\
  protopurity    & análise de impacto                   \\
  Pseudogen      & transformação source-to-pseudo-source\\
  PtYasm         & verificação de modelos               \\
  PuMoC          & verificação de modelos               \\
  PYTHIA         & criação automática de casos de teste \\
  ReAssert       & localização de falhas em testes      \\
  Rêve           & verificação de regressão             \\
  RRFinder       & mineração de liberação de recursos   \\
  Sapid/XML      & representação intermediária em XML   \\
  SonarQubePlug-in & SourceMeter plugin para análise Java \\
  SPARTA         & segurança e detecção de malwares     \\
  srcML          & transformação source-to-source       \\
  SWAT           & teste automático para aplicação web  \\
  TACLE          & type-analysis e visualizaçao de call-graph \\
  TEBA           & transformação source-to-source       \\
  TestEra        & geração automática de testes         \\
  Vdiff          & visualização de diferença de código  \\
  WALA           & análise de {\it bytecode} Java       \\
  Wrangler       & refatoração de código Erlang         \\
  XOgastan       & transformação source-to-source       \\
  \hline
\end{tabular}
\label{resumo-softwares}
}
\end{table}

\begin{table}[h]
\caption{Resumo dos 37 softwares selecionados e disponíveis para obtenção}
\centering
\begin{tabular}{| l | l |}
  \hline
  Software        & Página web \\
  \hline
  2LS             & \url{http://svn.cprover.org/wiki/doku.php?id=2ls_for_program_analysis} \\
  AccessAnalysis  & \url{http://accessanalysis.sourceforge.net} \\
  CIVL            & \url{http://vsl.cis.udel.edu/civl} \\
  CodeBoost       & \url{http://codeboost.org} \\
  composite       & \url{http://www.cs.ucsb.edu/~bultan/composite} \\
  CSeq            & \url{http://users.ecs.soton.ac.uk/gp4/cseq/files/cseq-0.5.zip} \\
  Derailer        & \url{http://people.csail.mit.edu/jnear/derailer} \\
  DOMPLETION      & \url{https://github.com/saltlab/dompletion} \\
  EJB             & \url{https://www.dropbox.com/s/glhg8any43lccgm/EJB.zip} \\
  e-munity        & \url{http://sourceforge.net/p/emunity/code/ci/master/tree} \\
  error-prone     & \url{http://code.google.com/p/error-prone} \\
  FaultBuster     & \url{http://www.sed.inf.u-szeged.hu/FaultBuster} \\
  Flowgen         & \url{https://github.com/jlopezvi/Flowgen} \\
  GUIZMO          & \url{http://modelum.es/trac/guizmo} \\
  GumTree         & \url{https://github.com/jrfaller/gumtree} \\
  HUSACCT         & \url{http://husacct.github.io/HUSACCT} \\
  Indus           & \url{http://indus.projects.cis.ksu.edu} \\
  JastAdd         & \url{http://jastadd.cs.lth.se/web} \\
  JFlow           & \url{http://vazexqi.github.io/JFlow} \\
  Jtop            & \url{http://code.google.com/p/pku-jtop} \\
  Kiasan/Bogor    & \url{http://bogor.projects.cs.ksu.edu/manual} \\
  Loopfrog        & \url{http://verify.inf.usi.ch/content/loopfrog} \\
  Lotrack         & \url{https://github.com/MaxLillack/Lotrack} \\
  MPAnalyzer      & \url{https://github.com/YoshikiHigo/MPAnalyzer} \\
  mygcc           & \url{http://mygcc.free.fr/} \\
  PHP AiR         & \url{https://github.com/cwi-swat/php-analysis} \\
  protopurity     & \url{https://github.com/jensnicolay/jipda/tree/scam2015/protopurity} \\
  Pseudogen       & \url{http://ahclab.naist.jp/pseudogen} \\
  PtYasm          & \url{http://www.cs.toronto.edu/~tomhart/ptyasm} \\
  ReAssert        & \url{http://mir.cs.illinois.edu/reassert} \\
  SonarQubePlugin & \url{http://github.com/FrontEndART/SonarQube-plug-in} \\
  SPARTA          & \url{http://types.cs.washington.edu/sparta} \\
  srcML           & \url{http://www.sdml.info/projects/srcml/trunk} \\
  TACLE           & \url{http://presto.cse.ohio-state.edu/tacle} \\
  TEBA            & \url{http://tebasaki.jp/src} \\
  WALA            & \url{http://wala.sourceforge.net/wiki/index.php/Main\_Page} \\
  Wrangler        & \url{http://www.cs.kent.ac.uk/projects/wrangler/Home.html} \\
  \hline
\end{tabular}
\label{resumo-softwares-disponiveis}
\end{table}

\section{Reproducibilidade do estudo}
\label{reproducibilidade-do-estudo}

Os artigos analisados na revisão estruturada estão todos documentados arquivo
{\it
dataset/dataset.ods}\footnote{\url{http://github.com/joenio/dissertacao-ufba-2016/blob/master/dataset/dataset.ods}},
uma planilha no formato aberto {\it Open Document Format for Office
Applications}\footnote{\url{http://www.oasis-open.org/committees/office}}.

Nesta planilha está documentada cada etapa da revisão estruturada, indicando em
cada artigo analisado qual o estado do mesmo, se foi ou não incluído na
execução da atividade.  Nesta planilha é possível encontrar também o nome de
cada ferramenta e uma caracterização completa.

O script utilizado na segunda atividade da revisão estruturada -- {\it (2)
Filtro} -- também está neste mesmo repositório no arquivo {\it
dataset/revisao-estruturada/filter}\footnote{\url{http://github.com/joenio/dissertacao-ufba-2016/blob/master/revisao-estruturada/filter}}
escrito em linguagem Perl especialmente para este estudo.

A maior parte das atividades de pesquisa, reuniões de orientação e comunicação
realizadas neste estudo estão também documentadas em {\it issues} neste
repositório e na wiki do grupo de pesquisa aSide.

\begin{itemize}
  \item \url{http://wiki.dcc.ufba.br/Aside/Orientacao2014JoenioCosta}
  \item \url{https://github.com/joenio/dissertacao-ufba-2016/issues?q=}
\end{itemize}
