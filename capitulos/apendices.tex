\xchapter{Reproducibilidade do estudo}
{}
\label{reproducibilidade-do-estudo}

%Definimos em linhas gerais utilizar arquivos em texto plano para armazenar
%os dados coletados visando a facilidade e transparência ao acesso aos dados
%por

Este trabalho de pesquisa, incluindo textos, imagens, dados e códigos estão
disponíveis publicamente sob a
licença CC BY-SA 4.0\footnote{\url{http://creativecommons.org/licenses/by-sa/4.0}}
(Licença Atribuição-CompartilhaIgual 4.0 Internacional Creative Commons)
no seguinte endereço:

\begin{itemize}
  \item \url{https://github.com/joenio/dissertacao-ufba-2016}
\end{itemize}

A maior parte das atividades de comunicação e reuniões de orientação realizadas
durante a pesquisa estão documentadas em {\it issues} no repositório acima ou na wiki do
grupo de pesquisa aSide, em:

\begin{itemize}
  \item \url{http://wiki.dcc.ufba.br/Aside/Orientacao2014JoenioCosta}
\end{itemize}

O texto foi escrito em \LaTeX, códigos, scripts e templates foram desenvolvidos
em Perl, os dados foram coletados em arquivos de texto plano nos formatos CSV,
YAML e BibTeX. Alguns elementos do repositório serão detalhados a seguir, a Tabela
\ref{arquivos-repositorio} apresenta uma visão geral.

% versão de cada software utilizado, ambiente, Debian, Perl, Analizo
% libre office, Evince, YAML, BibTeX, navegador, etc

\begin{table}[h]
\caption{Organização de arquivos e pastas do repositório.}
\centering
\begin{tabular}{ l p{10cm} }
  \hline
  Arquivo ou pasta            & Descrição           \\
  \hline
  \texttt{bibliografia.bib}   & Arquivo BibTeX com as referências bibliográficas utilizadas no texto. \\
  \texttt{bin/}               & Pasta contendo os scripts desenvolvidos para coleta, transformação e análise de dados. \\
  \texttt{capitulos/}         & Arquivos \LaTeX \ com o conteúdo/texto de cada capítulo da dissertação. \\
  \texttt{dataset/}           & Dados coletados durante os estudos apresentados nos Capítulos \ref{estudo1}, \ref{estudo2} e \ref{estudo3}. \\
  \texttt{dissertacao.tex}    & Arquivo \LaTeX \ principal com inclusão dos capítulos, título e resumo do texto. \\
  \texttt{documents/}         & Documentos gerados pelos templates com apresentação dos dados coletados. \\
  \texttt{imagens/}           & Imagens, gráficos e demais elementos visuais utilizados no texto. \\
  \texttt{lib/}               & Diretório contendo implementação de métodos para leitura análise e transformação dos dados utilizado nos scripts. \\
  \texttt{Makefile}           & Conjunto de regras para execução dos scripts e compilação do código fonte \LaTeX \ em PDF.                 \\
  \texttt{README.md}          & Documento com descrição do repositório, informaçoes de contato, e instruções para compilação dos fontes \LaTeX \ e execução dos scripts. \\
  \texttt{templates/}         & Templates (modelos) para leitura dos dados coletados e transformação em documentos. \\
  \texttt{ufbathesis.cls}     & Arquivo de estilo do \LaTeX \ com definições e padrão de formatação deacordo as regras de publicação da UFBA. \\
  \hline
\end{tabular}
\label{arquivos-repositorio}
\end{table}

\section{Organização do repositório}

\subsection{\texttt{bin/}}

\begin{description}
  \item [\texttt{bin/cache}] Script utilizado para agregar todos os dados
  de todos os projetos num arquivo único em formato YML em \texttt{cache/dataset.yml}.

%agregar todos os dados coletados de todos os projetos a partir dos arquivos
%\texttt{software.yml}, \texttt{acm.bib}, \texttt{ieee.bib}, \texttt{paper.bib},
%\texttt{references.yml} e \texttt{search.yml}.
%O script \texttt{bin/render} lê este arquivo \texttt{cache/dataset.yml},
%carrega os dados em memória e passa estes dados como parâmetro para os arquivos
%templates.

%  \item [\texttt{bin/filter-papers}] (refatorar, renomear)
%O script utilizado na segunda atividade da revisão estruturada -- {\it (2)
%Filtro} -- também está neste mesmo repositório no arquivo {\it
%dataset/revisao-estruturada/filter}\footnote{\url{http://github.com/joenio/dissertacao-ufba-2016/blob/master/revisao-estruturada/filter}}
%escrito em linguagem Perl especialmente para este estudo.


  \item [\texttt{bin/ids}]
    Script utilizado para criar o campo \texttt{id} com valor autoincremento
    para cada referência no arquivo \texttt{documents/references.bib}

  \item [\texttt{bin/merge}]
    Combina os arquivos \texttt{acm.bib}, \texttt{ieee.bib} e
    \texttt{paper.bib} de cada projeto em um único arquivo no formato BibTeX
    removendo duplicidades dos resultados.

  \item [\texttt{bin/render}]
    Este script lê o arquivo \texttt{cache/dataset.yml}, carrega os dados em
    memória e passa estes dados como parâmetro para os arquivos templates
    localizados em \texttt{templates/}.

  \item [\texttt{bin/run-analizo}] 
    Script utilizado para automatizar a execução da ferramenta {\it analizo
    metrics} para coletar o número de módulos de cada lançamento foi
    desenvolvido.

\end{description}

\subsection{\texttt{dataset/}}

%(falar da planilha de revisao de literatura do estudo1,
%apresentar a estrutura de diretórios e arquivos para cada software)

\begin{description}

  \item [\texttt{dataset/literature-review.ods}]
    Planilha LibreOffice Calc\footnote{\url{https://www.libreoffice.org}} com
    os dados coletados no estudo do Capítulo \ref{estudo1} resultando na
    seleção de \SoftwareCount \ projetos de software acadêmico.

Nesta planilha está documentada cada etapa da revisão estruturada, indicando em
cada artigo analisado qual o estado do mesmo, se foi ou não incluído na
execução da atividade.  Nesta planilha é possível encontrar também o nome de
cada ferramenta e uma caracterização completa.

  \item [\texttt{dataset/software/}]
    Estrutura de diretório utilizado para coleta de dados em arquivos do tipo
    YAML e BibTeX, cada projeto de software selecionado possui um diretório
    nesta estrutura.

  \item [\texttt{dataset/software/<nome>/software.yml}]
    Arquivo contendo os dados básicos de cada software, como: nome, descrição, url,
    licença, disponibilidade para download, entre outros dados.

  \item [\texttt{dataset/software/<nome>/references.yml}]
    Arquivo YAML para armazenar dados coletados sobre as menções a cada software
    acadêmico.

  \item [\texttt{dataset/software/<nome>/releases.yml}]
    Arquivo para armazenar os dados de cada lançamento do software cotendo o
    número da versão, a data do lançamento e a url para download do software
    ne versão específica.

  \item [\texttt{dataset/software/<nome>/paper.bib}]
    Arquivo BibTeX contendo os metadados do artigo inicial onde o software foi
    selecionado.

  \item [\texttt{dataset/software/<nome>/search/acm.bib}]
    Metadados dos artigos obtidos como resultado da busca na base ACM.

  \item [\texttt{dataset/software/<nome>/search/ieee.bib}]
    Metadados dos artigos obtidos como resultado da busca na base IEEE.

\end{description}

\subsection{\texttt{lib/Dissertacao.pm}}

A maior parte da lógica dos scripts foi implementada no arquivo
\texttt{lib/Dissertacao.pm} com objetivo de reduzir repetição de código e
proporcionar reuso.

\subsection{\texttt{templates/}}

O resultado de cada arquivo de template é armazenado num arquivo de mesmo nome
no diretório \texttt{documents/} sem a extenção \texttt{.epl}.

\begin{description}

  \item [\texttt{templates/references.csv.epl}]
    Cria o arquivo \texttt{documents/references.csv} com todos os artigos e
    projetos, indicando quando há relação entre eles através de menção marcada
    com um \texttt{x} sempre que houver menção de um artigo para um software.

  \item [\texttt{templates/search-strings.csv.epl}]
    Gera um documento CSV com todas as strings de busca no arquivo
    \texttt{documents/search-strings.csv}.

  \item [\texttt{templates/dataset.csv.epl}]
    Este template gera o arquivo \texttt{documents/dataset.csv} com um resumo de
    dados coletados durante todos os estudos da pesquisa, incluindo número de lançamentos,
    número de menções, número de menções por tipo, métrica de código fonte da última
    versão, entre outros dados para cada software estudado.

  \item [\texttt{templates/metrics.csv.epl}]
    Este arquivo agrega as métricas de código fonte coletadas para cada lançamento
    de cada software numa tabela em formato CSV.

%dataset-by-year.csv.epl
%life-cycle.csv.epl
%mentions-timeline.csv.epl

\end{description}

\section{Detalhes de implementação}

% TODO
% documentar instalação das dependencias do script para filtro
% documentar a instaação do sloccount

Todos os scripts e templates foram desenvolvidos utilizando a linguagem de
programação Perl\footnote{\url{http://perl.org}} com o auxílio dos seguintes
módulos CPAN\footnote{\url{http://cpan.org}}:

\begin{itemize}
  \item Devel::CheckBin \url{http://metacpan.org/pod/Devel::CheckBin}
  \item List::Util \url{http://metacpan.org/pod/List::Util}
  \item Modern::Perl \url{http://metacpan.org/pod/Modern::Perl}
  \item Mojo::Template \url{http://metacpan.org/pod/Mojo::Template}
  \item Text::BibTeX \url{http://metacpan.org/pod/Text::BibTeX}
  \item YAML \url{http://metacpan.org/pod/YAML}
  \item YAML::XS \url{http://metacpan.org/pod/YAML::XS}
\end{itemize}

%Apresentar e detalhar os arquivos scam-links.md e ase-link.md (citado no estudo1:preparacao).

%de cada artigo é realizada com o auxílio da funcionalidade de busca
%do leitor de pdf utilizado neste estudo\footnote{Utilizamos software Evince v3.22.1}
%utilizado para leitura dos artigos, com o auxílio da busca encontramos cada
%ocorrência ao nome do software tomando nota a confirmação sobre a menção
%encontrada.

%Implementamos scripts para transformar os dados da busca em formato BibTeX para
%arquivos no formato YML pois a coleta e inspeção dos artigos serão registrados
%em arquivos YML para cada projeto de software. Estes scripts recebem como entrada
%os resultados da busca de todos os projetos e geram como saída os arquivos \texttt{references.bib}
%e \texttt{references.yml} para cada projeto.
