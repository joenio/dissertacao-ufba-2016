\xchapter{Reproducibilidade do estudo}
{Este capítulo apresenta ....}
\label{reproducibilidade-do-estudo}

% TODO
% documentar instalação das dependencias do script para filtro
% documentar a instaação do sloccount
% descrever o formato YAML utilizado para caracterização dos projetos

%produz um resultado indicando todas as linguagens e
%quanto do código total é escrito em cada uma delas.

Apresentar detalhes de implementação, execução e uso do script bin/run-analizo, bem como detalhes para instalação do Analizo.

Apresentar e detalhar os arquivos scam-links.md e ase-link.md (citado no estudo1:preparacao).

Os artigos analisados na revisão estruturada estão todos documentados arquivo
{\it
dataset/dataset.ods}\footnote{\url{http://github.com/joenio/dissertacao-ufba-2016/blob/master/dataset/dataset.ods}},
uma planilha no formato aberto {\it Open Document Format for Office
Applications}\footnote{\url{http://www.oasis-open.org/committees/office}}.

Nesta planilha está documentada cada etapa da revisão estruturada, indicando em
cada artigo analisado qual o estado do mesmo, se foi ou não incluído na
execução da atividade.  Nesta planilha é possível encontrar também o nome de
cada ferramenta e uma caracterização completa.

O script utilizado na segunda atividade da revisão estruturada -- {\it (2)
Filtro} -- também está neste mesmo repositório no arquivo {\it
dataset/revisao-estruturada/filter}\footnote{\url{http://github.com/joenio/dissertacao-ufba-2016/blob/master/revisao-estruturada/filter}}
escrito em linguagem Perl especialmente para este estudo.

A maior parte das atividades de pesquisa, reuniões de orientação e comunicação
realizadas neste estudo estão também documentadas em {\it issues} neste
repositório e na wiki do grupo de pesquisa aSide.

\begin{itemize}
  \item \url{http://wiki.dcc.ufba.br/Aside/Orientacao2014JoenioCosta}
  \item \url{https://github.com/joenio/dissertacao-ufba-2016/issues}
\end{itemize}
