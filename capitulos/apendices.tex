\xchapter{Revisão estruturada}
{Este capítulo apresenta as detalhes e números da revisão estruturada.}
\label{apendice-revisao-estruturada}

\section{Softwares científicos selecionados}
\label{softwares-cientificos}

Resumo dos softwares selecionados na revisão estruturada.

\begin{table}[h]
\caption{Resultados da revisão estruturada para cada edição do SCAM}
\centering
\begin{tabular}{| l | c | c | c |}
  \hline
  Edição & (1) Busca & (2) Filtro & (3) Seleção \\
  \hline
  SCAM 2001 & 23    & 6         & 1           \\
  SCAM 2002 & 18    & 6         & 5           \\
  SCAM 2003 & 21    & 8         & 3           \\
  SCAM 2004 & 17    & 3         & 1           \\
  SCAM 2005 & 19    & 7         & 1           \\
  SCAM 2006 & 22    & 10        & 7           \\
  SCAM 2007 & 23    & 7         & 2           \\
  SCAM 2008 & 29    & 14        & 2           \\
  SCAM 2009 & 20    & 10        & -           \\
  SCAM 2010 & 21    & 15        & 5           \\
  SCAM 2011 & 21    & 10        & 2           \\
  SCAM 2012 & 22    & 12        & 4           \\
  SCAM 2013 & 24    & 13        & 2           \\
  SCAM 2014 & 36    & 16        & 4           \\
  SCAM 2015 & 30    & 18        & 2           \\
  \hline
  Total     & 346   & 155       & 41          \\
  \hline
\end{tabular}
\label{artigos-do-scam}
\end{table}

\begin{table}[h]
\caption{Resultados da revisão estruturada para cada edição do ASE}
\centering
\begin{tabular}{| l | c | c | c |}
  \hline
  Edição & (1) Busca & (2) Filtro & (3) Seleção \\
  \hline
  ASE 1991 & 28    & -         & -           \\
  ASE 1992 & 25    & -         & -           \\
  ASE 1993 & 21    & -         & -           \\
  ASE 1994 & 23    & -         & -           \\
  ASE 1995 & 23    & -         & -           \\
  ASE 1996 & 15    & -         & -           \\
  ASE 1997 & 47    & 1         & -           \\
  ASE 1998 & 44    & 4         & -           \\
  ASE 1999 & 50    & -         & -           \\
  ASE 2000 & 44    & 2         & -           \\
  ASE 2001 & 68    & 7         & 2           \\
  ASE 2002 & 46    & 5         & -           \\
  ASE 2003 & 54    & 5         & 2           \\
  ASE 2004 & 68    & 7         & -           \\
  ASE 2005 & 79    & 9         & 1           \\
  ASE 2006 & 61    & 12        & 2           \\
  ASE 2007 & 102   & 19        & 5           \\
  ASE 2008 & 90    & 18        & 5           \\
  ASE 2009 & 89    & 19        & 11          \\
  ASE 2010 & 87    & 20        & 3           \\
  ASE 2011 & 112   & 28        & 5           \\
  ASE 2012 & 68    & 16        & 5           \\
  ASE 2013 & 90    & 28        & 7           \\
  ASE 2014 & 100   & 39        & 7           \\
  ASE 2015 & 99    & 42        & 7           \\
  \hline
  Total    & 1533  & 281       & 62          \\
  \hline
\end{tabular}
\label{artigos-do-ase}
\end{table}

\section{Reproducibilidade do estudo}
\label{reproducibilidade-do-estudo}


% grep -R "download_available: yes" dataset/software/*/software.yml | wc -l
% 36

% grep -R "download_available: no" dataset/software/*/software.yml | wc -l
% 24

% TODO
% documentar instalação das dependencias do script para filtro
% documentar a instaação do sloccount
% descrever o formato YAML utilizado para caracterização dos projetos

%produz um resultado indicando todas as linguagens e
%quanto do código total é escrito em cada uma delas.

Os artigos analisados na revisão estruturada estão todos documentados arquivo
{\it
dataset/dataset.ods}\footnote{\url{http://github.com/joenio/dissertacao-ufba-2016/blob/master/dataset/dataset.ods}},
uma planilha no formato aberto {\it Open Document Format for Office
Applications}\footnote{\url{http://www.oasis-open.org/committees/office}}.

Nesta planilha está documentada cada etapa da revisão estruturada, indicando em
cada artigo analisado qual o estado do mesmo, se foi ou não incluído na
execução da atividade.  Nesta planilha é possível encontrar também o nome de
cada ferramenta e uma caracterização completa.

O script utilizado na segunda atividade da revisão estruturada -- {\it (2)
Filtro} -- também está neste mesmo repositório no arquivo {\it
dataset/revisao-estruturada/filter}\footnote{\url{http://github.com/joenio/dissertacao-ufba-2016/blob/master/revisao-estruturada/filter}}
escrito em linguagem Perl especialmente para este estudo.

A maior parte das atividades de pesquisa, reuniões de orientação e comunicação
realizadas neste estudo estão também documentadas em {\it issues} neste
repositório e na wiki do grupo de pesquisa aSide.

\begin{itemize}
  \item \url{http://wiki.dcc.ufba.br/Aside/Orientacao2014JoenioCosta}
  \item \url{https://github.com/joenio/dissertacao-ufba-2016/issues}
\end{itemize}


% **ATENÇÃO** não editar este arquivo!
%
% o conteúdo deste arquivo é gerado pelo script bin/softwares-summary e pelo
% template capitulos/softwares-summary.tex.epl

\xchapter{Dados dos softwares acadêmicos}{Este capítulo apresenta
os dados coletados para cada software acadêmico selecionado na
revisão estruturada}
\label{softwares-summary}

\section{2LS - 2nd order Logic Solving}

Análise de terminação para programas C usando resumo interprocedural baseado em modelos
publicado no  ,
disponibilizado em \url{http://svn.cprover.org/wiki/doku.php?id=2ls for program analysis},
disponível,
foss
sob uma licença BSD License.

Software com lançamentos Occasionally,
7 versões lançadas
em ,
escrito em C++,
uma busca por citações no {\bf IEEE Xplore} por
\texttt{}
e no {\bf ACM} por
\texttt{}
retornou
0 resultados,
nenhum faz referência ao software.



\section{AccessAnalysis}

Cálculo de métricas IGAT e IGAM
publicado no  ,
disponibilizado em \url{http://accessanalysis.sourceforge.net},
disponível,
foss
sob uma licença Eclipse Public License.

Software considerado obsoleto,
4 versões lançadas
em ,
escrito em Java,
uma busca por citações no {\bf IEEE Xplore} por
\texttt{}
e no {\bf ACM} por
\texttt{}
retornou
0 resultados,
nenhum faz referência ao software.



\section{APIExample}

Extração de informações de API Java e documentação automática com exemplos
publicado no  ,
disponibilizado em \url{http://www.apiexample.com},
mas apesar do endereço estar acessível na data 
Wed Aug  9 03:18:56 UTC 2017
não é possível encontrar download do software no site.

Software sem informações sobre lançamentos ou releases,
uma busca por citações no {\bf IEEE Xplore} por
\texttt{}
e no {\bf ACM} por
\texttt{}
retornou
0 resultados,
nenhum faz referência ao software.



\section{BEG - Bandera environment generator}

Criação automática de ambientes para verificação de modelos Java
publicado no  ,
disponibilizado em \url{http://bandera.projects.cs.ksu.edu/},
mas apesar do endereço estar acessível na data 
Wed Aug  9 03:19:39 UTC 2017
não é possível encontrar download do software no site.

Software sem informações sobre lançamentos ou releases,
uma busca por citações no {\bf IEEE Xplore} por
\texttt{}
e no {\bf ACM} por
\texttt{}
retornou
0 resultados,
nenhum faz referência ao software.



\section{ccJava - Class-based Crosscutting Language for Java}

Linguagem orientada a aspectos
publicado no  ,
disponibilizado em \url{http://posl.minnie.ai.kyutech.ac.jp/},
inacessível.

Software sem informações sobre lançamentos ou releases,
uma busca por citações no {\bf IEEE Xplore} por
\texttt{}
e no {\bf ACM} por
\texttt{}
retornou
0 resultados,
nenhum faz referência ao software.



\section{CIVL - Concurrency intermediate verification language}

Framework para verificação de programas concorrentes
publicado no  ,
disponibilizado em \url{http://vsl.cis.udel.edu/civl/},
disponível,
foss
sob uma licença GNU General Public License.

Software com lançamentos Frequently,
36 versões lançadas
em ,
escrito em C,
uma busca por citações no {\bf IEEE Xplore} por
\texttt{}
e no {\bf ACM} por
\texttt{}
retornou
0 resultados,
nenhum faz referência ao software.



\section{CodeBoost}

Transformação source-to-source para otimização de programas C++
publicado no  ,
disponibilizado em \url{http://codeboost.org},
disponível,
foss
sob uma licença GNU General Public License.

Software com lançamentos Occasionally,
134 versões lançadas
em ,
escrito em C,
uma busca por citações no {\bf IEEE Xplore} por
\texttt{}
e no {\bf ACM} por
\texttt{}
retornou
0 resultados,
nenhum faz referência ao software.



\section{CSL - Composite Symbolic Library}

Verificação de modelos
publicado no  ,
disponibilizado em \url{http://www.cs.ucsb.edu/~bultan/composite/},
disponível,
gratis
sem uma licença definida.

Software sem informações sobre lançamentos ou releases,
escrito em C,
uma busca por citações no {\bf IEEE Xplore} por
\texttt{}
e no {\bf ACM} por
\texttt{}
retornou
0 resultados,
nenhum faz referência ao software.



\section{CPA+ - Configurable program analysis with dynamic precision adjustment}

Análise configurável de programa com ajuste dinâmico de precisão
publicado no  ,
disponibilizado em \url{http://www.cs.sfu.ca/~dbeyer/blast_cpaplus/},
inacessível.

Software sem informações sobre lançamentos ou releases,
uma busca por citações no {\bf IEEE Xplore} por
\texttt{}
e no {\bf ACM} por
\texttt{}
retornou
0 resultados,
nenhum faz referência ao software.



\section{CSeq}

Transformação source-to-source para programas C concorrentes
publicado no  ,
disponibilizado em \url{http://users.ecs.soton.ac.uk/gp4/cseq/files/cseq-0.5.zip},
disponível,
foss
sob uma licença BSD License.

Software sem informações sobre lançamentos ou releases,
escrito em C,
uma busca por citações no {\bf IEEE Xplore} por
\texttt{}
e no {\bf ACM} por
\texttt{}
retornou
0 resultados,
nenhum faz referência ao software.



\section{DDVerify}

Verificação de Linux drivers através de checagem de modelos
publicado no  ,
disponibilizado em \url{http://www.verify.ethz.ch/ddverify},
inacessível.

Software sem informações sobre lançamentos ou releases,
uma busca por citações no {\bf IEEE Xplore} por
\texttt{}
e no {\bf ACM} por
\texttt{}
retornou
0 resultados,
nenhum faz referência ao software.



\section{Derailer}

Localização de falhas de segurança em aplicações web
publicado no  ,
disponibilizado em \url{http://people.csail.mit.edu/jnear/derailer},
disponível,
foss
sob uma licença GNU General Public License.

Software considerado obsoleto,
2 versões lançadas
em ,
escrito em Ruby,
uma busca por citações no {\bf IEEE Xplore} por
\texttt{}
e no {\bf ACM} por
\texttt{}
retornou
0 resultados,
nenhum faz referência ao software.



\section{Diagnosys}

Construção de interfaces de debug para o kernel Linux
publicado no  ,
disponibilizado em \url{http://momentum.labri.fr/projects/diagnosys},
inacessível.

Software sem informações sobre lançamentos ou releases,
uma busca por citações no {\bf IEEE Xplore} por
\texttt{}
e no {\bf ACM} por
\texttt{}
retornou
0 resultados,
nenhum faz referência ao software.



\section{DOMPLETION}

Sugestão de código javascript
publicado no  ,
disponibilizado em \url{https://github.com/saltlab/dompletion},
disponível,
gratis
sem uma licença definida.

Software sem informações sobre lançamentos ou releases,
escrito em Javascript,
uma busca por citações no {\bf IEEE Xplore} por
\texttt{}
e no {\bf ACM} por
\texttt{}
retornou
0 resultados,
nenhum faz referência ao software.



\section{DRC - Dangling Reference Checker}

Análise estática para detecção de referências inválidas em código dinâmico PHP
publicado no  ,
disponibilizado em \url{http://home.engineering.iastate.edu/~hungnv/Research/DRC},
inacessível.

Software sem informações sobre lançamentos ou releases,
uma busca por citações no {\bf IEEE Xplore} por
\texttt{}
e no {\bf ACM} por
\texttt{}
retornou
0 resultados,
nenhum faz referência ao software.



\section{e-munity}

Verificação de segurança
publicado no  ,
disponibilizado em \url{http://sourceforge.net/p/emunity/code/ci/master/tree/},
disponível,
gratis
sem uma licença definida.

Software sem informações sobre lançamentos ou releases,
escrito em C,
uma busca por citações no {\bf IEEE Xplore} por
\texttt{}
e no {\bf ACM} por
\texttt{}
retornou
0 resultados,
nenhum faz referência ao software.



\section{EJB}

(EJB Interceptor Analyzer) Criação de diagramas de sequência
publicado no  ,
disponibilizado em \url{https://www.dropbox.com/s/glhg8any43lccgm/EJB.zip},
disponível,
gratis
sem uma licença definida.

Software sem informações sobre lançamentos ou releases,
escrito em Java,
uma busca por citações no {\bf IEEE Xplore} por
\texttt{}
e no {\bf ACM} por
\texttt{}
retornou
0 resultados,
nenhum faz referência ao software.



\section{Error Prone}

Localização de bugs em código Java construído em cima do compilador javac
publicado no  ,
disponibilizado em \url{http://code.google.com/p/error-prone},
disponível,
foss
sob uma licença Apache License.

Software com lançamentos Frequently,
22 versões lançadas
em ,
escrito em Java,
uma busca por citações no {\bf IEEE Xplore} por
\texttt{}
e no {\bf ACM} por
\texttt{}
retornou
0 resultados,
nenhum faz referência ao software.



\section{ESBMC - Efficient SMT-Based Context-Bounded Model Checker}

Verificação de modelos
publicado no  ,
disponibilizado em \url{http://users.ecs.soton.ac.uk/lcc08r/esbmc/},
inacessível.

Software sem informações sobre lançamentos ou releases,
uma busca por citações no {\bf IEEE Xplore} por
\texttt{}
e no {\bf ACM} por
\texttt{}
retornou
0 resultados,
nenhum faz referência ao software.



\section{ETXL}

Transformação de código
publicado no  ,
disponibilizado em \url{http://www.cs.queensu.ca/home/thurston/etxl},
inacessível.

Software sem informações sobre lançamentos ou releases,
uma busca por citações no {\bf IEEE Xplore} por
\texttt{}
e no {\bf ACM} por
\texttt{}
retornou
0 resultados,
nenhum faz referência ao software.



\section{FaultBuster}

Refatoração de code smells
publicado no  ,
disponibilizado em \url{http://www.sed.inf.u-szeged.hu/FaultBuster},
disponível,
gratis
sob uma licença demo.

Software sem informações sobre lançamentos ou releases,
uma busca por citações no {\bf IEEE Xplore} por
\texttt{}
e no {\bf ACM} por
\texttt{}
retornou
0 resultados,
nenhum faz referência ao software.



\section{Flowgen}

Criação automática de grafos UML
publicado no  ,
disponibilizado em \url{https://github.com/jlopezvi/Flowgen},
disponível,
foss
sob uma licença GNU General Public License.

Software sem informações sobre lançamentos ou releases,
escrito em Python,
uma busca por citações no {\bf IEEE Xplore} por
\texttt{}
e no {\bf ACM} por
\texttt{}
retornou
0 resultados,
nenhum faz referência ao software.



\section{GRT - Guided Random Testing}

Geração automática de testes
publicado no  ,
disponibilizado em \url{http://www.sites.google.com/site/grtprojectut/download},
mas apesar do endereço estar acessível na data 
Wed Aug  9 03:21:53 UTC 2017
não é possível encontrar download do software no site.

Software sem informações sobre lançamentos ou releases,
uma busca por citações no {\bf IEEE Xplore} por
\texttt{}
e no {\bf ACM} por
\texttt{}
retornou
0 resultados,
nenhum faz referência ao software.



\section{GUIZMO}

Inferência de layout
publicado no  ,
disponibilizado em \url{http://modelum.es/trac/guizmo/},
disponível,
foss
sob uma licença Apache License.

Software sem informações sobre lançamentos ou releases,
escrito em Java,
uma busca por citações no {\bf IEEE Xplore} por
\texttt{}
e no {\bf ACM} por
\texttt{}
retornou
0 resultados,
nenhum faz referência ao software.



\section{GumTree}

Comparação de mudanças
publicado no  ,
disponibilizado em \url{https://github.com/jrfaller/gumtree},
disponível,
foss
sob uma licença GNU Lesser General Public License.

Software com lançamentos Occasionally,
3 versões lançadas
em ,
escrito em Java,
uma busca por citações no {\bf IEEE Xplore} por
\texttt{}
e no {\bf ACM} por
\texttt{}
retornou
0 resultados,
nenhum faz referência ao software.



\section{HUSACCT - HU Software Architecture Compliance Checking Tool}

verificação de conformidade arquitetural
publicado no  ,
disponibilizado em \url{http://husacct.github.io/HUSACCT},
disponível,
foss
sob uma licença Affero General Public License.

Software com lançamentos Frequently,
22 versões lançadas
em ,
escrito em Java,
uma busca por citações no {\bf IEEE Xplore} por
\texttt{}
e no {\bf ACM} por
\texttt{}
retornou
0 resultados,
nenhum faz referência ao software.



\section{Indus}

Biblioteca de program slicing
publicado no  ,
disponibilizado em \url{http://indus.projects.cis.ksu.edu},
disponível,
foss
sob uma licença Eclipse Public License.

Software com lançamentos ?,
36 versões lançadas
em ,
escrito em Java,
uma busca por citações no {\bf IEEE Xplore} por
\texttt{}
e no {\bf ACM} por
\texttt{}
retornou
0 resultados,
nenhum faz referência ao software.



\section{JastAdd}

Análise de código-fonte através da descrição de atributos via gramática de atributos (AG)
publicado no  ,
disponibilizado em \url{http://jastadd.cs.lth.se/web},
disponível,
foss
sob uma licença BSD License.

Software com lançamentos Frequently,
24 versões lançadas
em ,
escrito em Java,
uma busca por citações no {\bf IEEE Xplore} por
\texttt{}
e no {\bf ACM} por
\texttt{}
retornou
0 resultados,
nenhum faz referência ao software.



\section{JFlow}

Transformação source-to-source
publicado no  ,
disponibilizado em \url{http://vazexqi.github.io/JFlow/},
disponível,
foss
sob uma licença Illinois/NCSA Open Source License.

Software considerado obsoleto,
5 versões lançadas
em ,
escrito em Java,
uma busca por citações no {\bf IEEE Xplore} por
\texttt{}
e no {\bf ACM} por
\texttt{}
retornou
0 resultados,
nenhum faz referência ao software.



\section{JstereoCode}

Detecção de esteriótipos Java
publicado no  ,
disponibilizado em \url{http://www.cs.wayne.edu/~severe/revenge/},
mas apesar do endereço estar acessível na data 
Wed Aug  9 03:22:02 UTC 2017
não é possível encontrar download do software no site.

Software sem informações sobre lançamentos ou releases,
uma busca por citações no {\bf IEEE Xplore} por
\texttt{}
e no {\bf ACM} por
\texttt{}
retornou
0 resultados,
nenhum faz referência ao software.



\section{Jtop}

Gestão de casos de teste
publicado no  ,
disponibilizado em \url{http://code.google.com/p/pku-jtop/},
mas apesar do endereço estar acessível na data 
Wed Aug  9 03:22:03 UTC 2017
não é possível encontrar download do software no site.

Software sem informações sobre lançamentos ou releases,
uma busca por citações no {\bf IEEE Xplore} por
\texttt{}
e no {\bf ACM} por
\texttt{}
retornou
0 resultados,
nenhum faz referência ao software.



\section{Bogor/Kiasan}

Verificação de modelos
publicado no  ,
disponibilizado em \url{http://bogor.projects.cs.ksu.edu/manual/},
disponível,
foss
sob uma licença SAnToS Laboratory Open Academic License.

Software sem informações sobre lançamentos ou releases,
escrito em Java,
uma busca por citações no {\bf IEEE Xplore} por
\texttt{}
e no {\bf ACM} por
\texttt{}
retornou
0 resultados,
nenhum faz referência ao software.



\section{Loopfrog}

Verificação de modelos
publicado no  ,
disponibilizado em \url{http://verify.inf.usi.ch/content/loopfrog},
disponível,
gratis
sem uma licença definida.

Software sem informações sobre lançamentos ou releases,
uma busca por citações no {\bf IEEE Xplore} por
\texttt{}
e no {\bf ACM} por
\texttt{}
retornou
0 resultados,
nenhum faz referência ao software.



\section{Lotrack}

Análise estática de configuração
publicado no  ,
disponibilizado em \url{https://github.com/MaxLillack/Lotrack},
disponível,
gratis
sem uma licença definida.

Software sem informações sobre lançamentos ou releases,
escrito em Java,
uma busca por citações no {\bf IEEE Xplore} por
\texttt{}
e no {\bf ACM} por
\texttt{}
retornou
0 resultados,
nenhum faz referência ao software.



\section{MPAnalyzer}

Análise de padrões disponível
publicado no  ,
disponibilizado em \url{https://github.com/YoshikiHigo/MPAnalyzer},
disponível,
gratis
sem uma licença definida.

Software sem informações sobre lançamentos ou releases,
escrito em Java,
uma busca por citações no {\bf IEEE Xplore} por
\texttt{}
e no {\bf ACM} por
\texttt{}
retornou
0 resultados,
nenhum faz referência ao software.



\section{MSP}

Construção de modelo formal de acesso a memória
publicado no  ,
disponibilizado em \url{http://icps.u-strasbg.fr/software/msp},
mas apesar do endereço estar acessível na data 
Wed Aug  9 03:22:06 UTC 2017
não é possível encontrar download do software no site.

Software sem informações sobre lançamentos ou releases,
uma busca por citações no {\bf IEEE Xplore} por
\texttt{}
e no {\bf ACM} por
\texttt{}
retornou
0 resultados,
nenhum faz referência ao software.



\section{mygcc}

Verificação de programas C
publicado no  ,
disponibilizado em \url{http://mygcc.free.fr},
disponível,
foss
sob uma licença GNU General Public License.

Software considerado obsoleto,
5 versões lançadas
em ,
escrito em C,
uma busca por citações no {\bf IEEE Xplore} por
\texttt{}
e no {\bf ACM} por
\texttt{}
retornou
0 resultados,
nenhum faz referência ao software.



\section{PARSEWeb}

Query para apoio e sugestão de reuso de bibliotecas
publicado no  ,
disponibilizado em \url{http://ase.csc.ncsu.edu/parseweb},
inacessível.

Software sem informações sobre lançamentos ou releases,
uma busca por citações no {\bf IEEE Xplore} por
\texttt{}
e no {\bf ACM} por
\texttt{}
retornou
0 resultados,
nenhum faz referência ao software.



\section{PAT - Puzzle-Based Automatic Testing}

Ambiente de teste automático
publicado no  ,
disponibilizado em \url{http://pat.cse.ust.hk:8080},
inacessível.

Software sem informações sobre lançamentos ou releases,
uma busca por citações no {\bf IEEE Xplore} por
\texttt{}
e no {\bf ACM} por
\texttt{}
retornou
0 resultados,
nenhum faz referência ao software.



\section{PHP AiR}

Um framework para análise de código PHP escrito em Rascal
publicado no  ,
disponibilizado em \url{https://github.com/cwi-swat/php-analysis},
disponível,
gratis
sem uma licença definida.

Software com lançamentos ?,
4 versões lançadas
em ,
escrito em Rascal,
uma busca por citações no {\bf IEEE Xplore} por
\texttt{}
e no {\bf ACM} por
\texttt{}
retornou
0 resultados,
nenhum faz referência ao software.



\section{protopurity}

Análise de impacto
publicado no  ,
disponibilizado em \url{https://github.com/jensnicolay/jipda/tree/scam2015/protopurity},
disponível,
gratis
sem uma licença definida.

Software sem informações sobre lançamentos ou releases,
escrito em Javascript,
uma busca por citações no {\bf IEEE Xplore} por
\texttt{}
e no {\bf ACM} por
\texttt{}
retornou
0 resultados,
nenhum faz referência ao software.



\section{Pseudogen}

Transformação de código-fonte em pseudo-código
publicado no  ,
disponibilizado em \url{http://ahclab.naist.jp/pseudogen},
disponível,
gratis
sem uma licença definida.

Software sem informações sobre lançamentos ou releases,
escrito em Python,
uma busca por citações no {\bf IEEE Xplore} por
\texttt{}
e no {\bf ACM} por
\texttt{}
retornou
0 resultados,
nenhum faz referência ao software.



\section{PtYasm}

Verificação de modelos
publicado no  ,
disponibilizado em \url{http://www.cs.toronto.edu/~tomhart/ptyasm},
disponível,
gratis
sem uma licença definida.

Software sem informações sobre lançamentos ou releases,
escrito em Java,
uma busca por citações no {\bf IEEE Xplore} por
\texttt{}
e no {\bf ACM} por
\texttt{}
retornou
0 resultados,
nenhum faz referência ao software.



\section{PuMoC}

Verificação de modelos
publicado no  ,
disponibilizado em \url{http://www.liafa.jussieu.fr/~song/PuMoC},
mas apesar do endereço estar acessível na data 
Wed Aug  9 03:22:21 UTC 2017
não é possível encontrar download do software no site.

Software sem informações sobre lançamentos ou releases,
uma busca por citações no {\bf IEEE Xplore} por
\texttt{}
e no {\bf ACM} por
\texttt{}
retornou
0 resultados,
nenhum faz referência ao software.



\section{PYTHIA}

Criação automática de casos de teste
publicado no  ,
disponibilizado em \url{http://salt.ece.ubc.ca/software/pythia/},
inacessível.

Software sem informações sobre lançamentos ou releases,
uma busca por citações no {\bf IEEE Xplore} por
\texttt{}
e no {\bf ACM} por
\texttt{}
retornou
0 resultados,
nenhum faz referência ao software.



\section{ReAssert}

Localização de falhas em testes e refatoração
publicado no  ,
disponibilizado em \url{http://mir.cs.illinois.edu/reassert},
disponível,
foss
sob uma licença Illinois/NCSA Open Source License.

Software considerado obsoleto,
5 versões lançadas
em ,
escrito em Java,
uma busca por citações no {\bf IEEE Xplore} por
\texttt{}
e no {\bf ACM} por
\texttt{}
retornou
0 resultados,
nenhum faz referência ao software.



\section{Rêve}

Verificação de regressão
publicado no  ,
disponibilizado em \url{http://formal.iti.kit.edu/improve},
mas apesar do endereço estar acessível na data 
Wed Aug  9 03:22:42 UTC 2017
não é possível encontrar download do software no site.

Software sem informações sobre lançamentos ou releases,
uma busca por citações no {\bf IEEE Xplore} por
\texttt{}
e no {\bf ACM} por
\texttt{}
retornou
0 resultados,
nenhum faz referência ao software.



\section{RRFinder}

Mineração de especificação de liberação de recursos
publicado no  ,
disponibilizado em \url{http://sa.seforge.org/RRFinder/},
inacessível.

Software sem informações sobre lançamentos ou releases,
uma busca por citações no {\bf IEEE Xplore} por
\texttt{}
e no {\bf ACM} por
\texttt{}
retornou
0 resultados,
nenhum faz referência ao software.



\section{Sapid/XML}

Representação intermediária de código Java usando XML ao invés de AST
publicado no  ,
disponibilizado em \url{http://www.jtool.org},
inacessível.

Software sem informações sobre lançamentos ou releases,
uma busca por citações no {\bf IEEE Xplore} por
\texttt{}
e no {\bf ACM} por
\texttt{}
retornou
0 resultados,
nenhum faz referência ao software.



\section{Sonar Qube Plug-in}

Extende o SourceMeter com análise de código Java com o uso do SonarQube
publicado no  ,
disponibilizado em \url{http://github.com/FrontEndART/SonarQube-plug-in},
disponível,
gratis
sob uma licença FrontEndART Software Ltd.

Software com lançamentos Frequently,
4 versões lançadas
em ,
escrito em Java,
uma busca por citações no {\bf IEEE Xplore} por
\texttt{}
e no {\bf ACM} por
\texttt{}
retornou
0 resultados,
nenhum faz referência ao software.



\section{SPARTA - Static Program Analysis for Reliable Trusted Apps}

Segurança pra detecção de malware
publicado no  ,
disponibilizado em \url{http://types.cs.washington.edu/sparta},
disponível,
gratis
sem uma licença definida.

Software com lançamentos Occasionally,
14 versões lançadas
em ,
escrito em Java,
uma busca por citações no {\bf IEEE Xplore} por
\texttt{}
e no {\bf ACM} por
\texttt{}
retornou
0 resultados,
nenhum faz referência ao software.



\section{srcML}

Transformação source-to-source
publicado no  ,
disponibilizado em \url{},
disponível,
foss
sob uma licença GNU General Public License.

Software com lançamentos Occasionally,
14 versões lançadas
em ,
escrito em C++,
uma busca por citações no {\bf IEEE Xplore} por
\texttt{}
e no {\bf ACM} por
\texttt{}
retornou
0 resultados,
nenhum faz referência ao software.



\section{SWAT - Search based Web Application Tester}

Teste automático para aplicação web
publicado no  ,
disponibilizado em \url{http://www.cs.ucl.ac.uk/staff/nalshahw/swat},
inacessível.

Software sem informações sobre lançamentos ou releases,
uma busca por citações no {\bf IEEE Xplore} por
\texttt{}
e no {\bf ACM} por
\texttt{}
retornou
0 resultados,
nenhum faz referência ao software.



\section{TACLE - Type Analysis and CalL graph construction for Eclipse}

Análise de tipo (Type Analysis) e construção e visualizaçao de grafos de chamada (Call Graph)
publicado no  ,
disponibilizado em \url{http://presto.cse.ohio-state.edu/tacle},
disponível,
gratis
sem uma licença definida.

Software sem informações sobre lançamentos ou releases,
escrito em Java,
uma busca por citações no {\bf IEEE Xplore} por
\texttt{}
e no {\bf ACM} por
\texttt{}
retornou
0 resultados,
nenhum faz referência ao software.



\section{TEBA}

Transformação source-to-source
publicado no  ,
disponibilizado em \url{http://tebasaki.jp/src},
disponível,
gratis
sem uma licença definida.

Software com lançamentos Occasionally,
21 versões lançadas
em ,
escrito em Perl,
uma busca por citações no {\bf IEEE Xplore} por
\texttt{}
e no {\bf ACM} por
\texttt{}
retornou
0 resultados,
nenhum faz referência ao software.



\section{TestEra}

Geração automática de testes
publicado no  ,
disponibilizado em \url{http://www.mit.edu/~sarfraz/testera},
inacessível.

Software sem informações sobre lançamentos ou releases,
uma busca por citações no {\bf IEEE Xplore} por
\texttt{}
e no {\bf ACM} por
\texttt{}
retornou
0 resultados,
nenhum faz referência ao software.



\section{Vdiff}

Visualização de diferença de código-fonte
publicado no  ,
disponibilizado em \url{http://web.cs.ucla.edu/~miryung/software/vdiff/web/index.html},
mas apesar do endereço estar acessível na data 
Wed Aug  9 03:23:18 UTC 2017
não é possível encontrar download do software no site.

Software sem informações sobre lançamentos ou releases,
uma busca por citações no {\bf IEEE Xplore} por
\texttt{}
e no {\bf ACM} por
\texttt{}
retornou
0 resultados,
nenhum faz referência ao software.



\section{WALA}

Análise estática de bytecode Java
publicado no  ,
disponibilizado em \url{http://wala.sourceforge.net/wiki/index.php/Main_Page},
disponível,
foss
sob uma licença Eclipse Public License.

Software com lançamentos Occasionally,
37 versões lançadas
em ,
escrito em Java,
uma busca por citações no {\bf IEEE Xplore} por
\texttt{}
e no {\bf ACM} por
\texttt{}
retornou
0 resultados,
nenhum faz referência ao software.



\section{Wrangler}

Refatoração de código Erlang
publicado no  ,
disponibilizado em \url{http://www.cs.kent.ac.uk/projects/wrangler/Home.html},
disponível,
foss
sob uma licença BSD License.

Software com lançamentos Occasionally,
34 versões lançadas
em ,
escrito em Erlang,
uma busca por citações no {\bf IEEE Xplore} por
\texttt{}
e no {\bf ACM} por
\texttt{}
retornou
0 resultados,
nenhum faz referência ao software.



\section{XOgastan}

Transformação source-to-source
publicado no  ,
disponibilizado em \url{http://web.ing.unisannio.it/villano/students/masone},
inacessível.

Software sem informações sobre lançamentos ou releases,
uma busca por citações no {\bf IEEE Xplore} por
\texttt{}
e no {\bf ACM} por
\texttt{}
retornou
0 resultados,
nenhum faz referência ao software.





%
% **ATENÇÃO** não editar este arquivo!
%
% o conteúdo deste arquivo é gerado pelo script bin/render-template e pelo
% template templates/softwares-data-table.tex.epl

\xchapter{Dados resumidos}{Este capítulo ...}

\begin{table}[H]
\caption{?????}
\centering
\begin{tabular}{| l | c | c | c | c | c | c |}
  \hline
  Software & Versões & Citações & [cita] & [usa] & [contribui] & [cria] \\
  \hline
  2ls
  &
  7
  &
  1
  &
  0
  &
  0
  &
  0
  &
  1
  \\
  accessanalysis
  &
  4
  &
  2
  &
  1
  &
  0
  &
  0
  &
  1
  \\
  apiexample
  &
  0
  &
  4
  &
  3
  &
  0
  &
  0
  &
  1
  \\
  beg
  &
  0
  &
  9
  &
  4
  &
  3
  &
  1
  &
  1
  \\
  ccjava
  &
  0
  &
  5
  &
  2
  &
  0
  &
  2
  &
  1
  \\
  civl
  &
  36
  &
  6
  &
  5
  &
  0
  &
  0
  &
  1
  \\
  codeboost
  &
  134
  &
  15
  &
  14
  &
  0
  &
  0
  &
  1
  \\
  composite
  &
  0
  &
  6
  &
  2
  &
  2
  &
  1
  &
  1
  \\
  cpa+
  &
  0
  &
  5
  &
  0
  &
  0
  &
  4
  &
  1
  \\
  cseq
  &
  0
  &
  5
  &
  1
  &
  2
  &
  1
  &
  1
  \\
  ddverify
  &
  0
  &
  3
  &
  2
  &
  0
  &
  0
  &
  1
  \\
  derailer
  &
  2
  &
  2
  &
  1
  &
  0
  &
  0
  &
  1
  \\
  diagnosys
  &
  0
  &
  1
  &
  0
  &
  0
  &
  0
  &
  1
  \\
  dompletion
  &
  0
  &
  2
  &
  1
  &
  0
  &
  0
  &
  1
  \\
  drc
  &
  0
  &
  5
  &
  4
  &
  0
  &
  0
  &
  1
  \\
  e-munity
  &
  0
  &
  1
  &
  0
  &
  0
  &
  0
  &
  1
  \\
  ejb
  &
  0
  &
  3
  &
  1
  &
  1
  &
  0
  &
  1
  \\
  error-prone
  &
  22
  &
  2
  &
  0
  &
  1
  &
  0
  &
  1
  \\
  esbmc
  &
  0
  &
  42
  &
  16
  &
  18
  &
  7
  &
  1
  \\
  etxl
  &
  0
  &
  1
  &
  0
  &
  0
  &
  0
  &
  1
  \\
  faultbuster
  &
  0
  &
  1
  &
  0
  &
  0
  &
  0
  &
  1
  \\
  flowgen
  &
  0
  &
  3
  &
  2
  &
  0
  &
  0
  &
  1
  \\
  grt
  &
  0
  &
  9
  &
  4
  &
  2
  &
  3
  &
  0
  \\
  guizmo
  &
  0
  &
  1
  &
  0
  &
  0
  &
  0
  &
  1
  \\
  gumtree
  &
  3
  &
  18
  &
  5
  &
  10
  &
  2
  &
  1
  \\
  husacct
  &
  22
  &
  7
  &
  3
  &
  2
  &
  0
  &
  2
  \\
  indus
  &
  36
  &
  4
  &
  0
  &
  3
  &
  0
  &
  1
  \\
  jastadd
  &
  24
  &
  43
  &
  24
  &
  15
  &
  3
  &
  1
  \\
  jflow
  &
  5
  &
  7
  &
  5
  &
  1
  &
  0
  &
  1
  \\
  jstereocode
  &
  0
  &
  8
  &
  2
  &
  5
  &
  0
  &
  1
  \\
  jtop
  &
  0
  &
  2
  &
  0
  &
  1
  &
  0
  &
  1
  \\
  kiasan
  &
  0
  &
  16
  &
  10
  &
  0
  &
  5
  &
  1
  \\
  loopfrog
  &
  0
  &
  5
  &
  2
  &
  2
  &
  0
  &
  1
  \\
  lotrack
  &
  0
  &
  2
  &
  1
  &
  0
  &
  0
  &
  1
  \\
  mpanalyzer
  &
  0
  &
  1
  &
  0
  &
  0
  &
  0
  &
  1
  \\
  msp
  &
  0
  &
  2
  &
  1
  &
  0
  &
  0
  &
  1
  \\
  mygcc
  &
  5
  &
  7
  &
  3
  &
  1
  &
  1
  &
  2
  \\
  parseweb
  &
  0
  &
  24
  &
  22
  &
  1
  &
  0
  &
  1
  \\
  pat
  &
  0
  &
  2
  &
  1
  &
  0
  &
  0
  &
  1
  \\
  php-air
  &
  4
  &
  9
  &
  1
  &
  5
  &
  2
  &
  1
  \\
  protopurity
  &
  0
  &
  1
  &
  0
  &
  0
  &
  0
  &
  1
  \\
  pseudogen
  &
  0
  &
  1
  &
  0
  &
  0
  &
  0
  &
  1
  \\
  ptyasm
  &
  0
  &
  2
  &
  0
  &
  0
  &
  2
  &
  0
  \\
  pumoc
  &
  0
  &
  2
  &
  1
  &
  0
  &
  0
  &
  1
  \\
  pythia
  &
  0
  &
  2
  &
  0
  &
  0
  &
  1
  &
  1
  \\
  reassert
  &
  5
  &
  13
  &
  9
  &
  3
  &
  0
  &
  1
  \\
  reve
  &
  0
  &
  1
  &
  0
  &
  0
  &
  0
  &
  1
  \\
  rrfinder
  &
  0
  &
  3
  &
  2
  &
  0
  &
  0
  &
  1
  \\
  sapid-xml
  &
  0
  &
  5
  &
  1
  &
  2
  &
  1
  &
  1
  \\
  sonarqube-plugin
  &
  4
  &
  1
  &
  0
  &
  0
  &
  0
  &
  1
  \\
  sparta
  &
  14
  &
  4
  &
  1
  &
  2
  &
  0
  &
  1
  \\
  srcml
  &
  14
  &
  40
  &
  14
  &
  24
  &
  1
  &
  1
  \\
  swat
  &
  0
  &
  4
  &
  1
  &
  2
  &
  0
  &
  1
  \\
  tacle
  &
  0
  &
  3
  &
  1
  &
  1
  &
  0
  &
  1
  \\
  teba
  &
  21
  &
  1
  &
  0
  &
  0
  &
  0
  &
  1
  \\
  testera
  &
  0
  &
  23
  &
  17
  &
  5
  &
  0
  &
  1
  \\
  vdiff
  &
  0
  &
  5
  &
  4
  &
  0
  &
  0
  &
  1
  \\
  wala
  &
  37
  &
  11
  &
  5
  &
  5
  &
  0
  &
  1
  \\
  wrangler
  &
  34
  &
  33
  &
  16
  &
  10
  &
  6
  &
  1
  \\
  xogastan
  &
  0
  &
  5
  &
  4
  &
  0
  &
  0
  &
  1
  \\
  \hline
\end{tabular}
\end{table}


