\xchapter{Reproducibilidade do estudo}
{}
\label{reproducibilidade-do-estudo}

Este trabalho de pesquisa, incluindo textos, imagens, dados e códigos estão
disponíveis publicamente sob a
licença CC BY-SA 4.0\footnote{\url{http://creativecommons.org/licenses/by-sa/4.0}}
(Licença Atribuição-CompartilhaIgual 4.0 Internacional Creative Commons)
no seguinte endereço:

\begin{itemize}
  \item \url{https://github.com/joenio/dissertacao-ufba-2016}
\end{itemize}

A maior parte das atividades de comunicação e reuniões de orientação realizadas
durante a pesquisa estão documentadas em {\it issues} no repositório acima ou na wiki do
grupo de pesquisa aSide, em:

\begin{itemize}
  \item \url{http://wiki.dcc.ufba.br/Aside/Orientacao2014JoenioCosta}
\end{itemize}

O texto foi escrito em \LaTeX, códigos, scripts e templates foram desenvolvidos
em Perl, os dados foram coletados em arquivos de texto plano nos formatos CSV,
YAML e BibTeX. Alguns elementos do repositório serão detalhados a seguir, a Tabela
\ref{arquivos-repositorio} apresenta uma visão geral.

% versão de cada software utilizado, ambiente, Debian, Perl, Analizo
% libre office, Evince, YAML, BibTeX, navegador, etc

\begin{table}[h]
\caption{Organização de arquivos e pastas do repositório.}
\centering
\begin{tabular}{ l p{10cm} }
  \hline
  Arquivo ou pasta            & Descrição           \\
  \hline
  \texttt{bibliografia.bib}   & Arquivo BibTeX com as referências bibliográficas utilizadas no texto. \\
  \texttt{bin/}               & Pasta contendo os scripts desenvolvidos para coleta, transformação e análise de dados. \\
  \texttt{capitulos/}         & Arquivos \LaTeX \ com o conteúdo/texto de cada capítulo da dissertação. \\
  \texttt{dataset/}           & Dados coletados durante os estudos apresentados nos Capítulos \ref{estudo1}, \ref{estudo2} e \ref{estudo3}. \\
  \texttt{dissertacao.tex}    & Arquivo \LaTeX \ principal com inclusão dos capítulos, título e resumo do texto. \\
  \texttt{documents/}         & Documentos gerados pelos templates com apresentação dos dados coletados. \\
  \texttt{imagens/}           & Imagens, gráficos e demais elementos visuais utilizados no texto. \\
  \texttt{lib/}               & ...                 \\
  \texttt{Makefile}           & Conujunto de regras para execução dos scripts e compilação do código fonte \LaTeX \ em PDF.                 \\
  \texttt{README.md}          & Documento com descrição do repositório, informaçoes de contato, e instruções para compilação dos fontes \LaTeX \ e execução dos scripts. \\
  \texttt{templates/}         & Templates (modelos) para leitura dos dados coletados e transformação em documentos. \\
  \texttt{ufbathesis.cls}     & Arquivo de estilo do \LaTeX \ com definições e padrão de formatação deacordo as regras de publicação da UFBA. \\
  \hline
\end{tabular}
\label{arquivos-repositorio}
\end{table}

\section{Organização do repositório}

\subsection{\texttt{bin/}}

(detalhes de implementação e descrição de cada script, instruções de uso)

\begin{description}
  \item [\texttt{bin/cache}] Script para ...
  \item [\texttt{bin/chart-dataset}] (remover este script)
  \item [\texttt{bin/filter-papers}] (refatorar, renomear)
  \item [\texttt{bin/ids}] ...
  \item [\texttt{bin/merge}] ...
  \item [\texttt{bin/render}] ...
  \item [\texttt{bin/run-analizo}] ...
\end{description}

\subsection{\texttt{dataset/}}

(falar da planilha de revisao de literatura do estudo1,
apresentar a estrutura de diretórios e arquivos para cada software)

\begin{description}
  \item [\texttt{dataset/software/}] Estrutura de diretório utilizado para coleta de dados em arquivos do tipo YAML e BibTeX.
  \item [\texttt{dataset/literature-review.ods}] Planilha LibreOffice Calc\footnote{\url{https://www.libreoffice.org}} ...

Os artigos analisados na revisão estruturada estão todos documentados arquivo
{\it
dataset/dataset.ods}\footnote{\url{http://github.com/joenio/dissertacao-ufba-2016/blob/master/dataset/dataset.ods}},
uma planilha no formato aberto {\it Open Document Format for Office
Applications}\footnote{\url{http://www.oasis-open.org/committees/office}}.

%Os dados extraídos são armazenados no arquivo \texttt{references.yml} de
%cada projeto, sendo atualizado com a inclusão de um novo campo indicando o tipo
%de menção do artigo ao software, \texttt{mention\_type},

%para utilizar na etapa de análise. Implementamos também templates para gerar arquivos em
%formato TeX para posterior inclusão na seção de coleta onde apresentaremos os dados
%coletados.

%Os dados armazenados em arquivos
%YAML e BibTeX são carregados em estruturas em memória através de um script
%escrito em linguagem Perl e repassados para templates processados para exibir
%os dados neste texto, em arquivos TeX, CSV, etc. A Figura \ref{estudo2-fluxograma}
%apresenta o fluxo de coleta, análise e transformação dos dados.

Nesta planilha está documentada cada etapa da revisão estruturada, indicando em
cada artigo analisado qual o estado do mesmo, se foi ou não incluído na
execução da atividade.  Nesta planilha é possível encontrar também o nome de
cada ferramenta e uma caracterização completa.

%Esta anotação foi feita no arquivo \texttt{references.yml} de cada projeto, nele
%foi adicionado um novo registro para cada artigo que faz menção e notas sobre a
%menção foi adicionada.


%Os scripts \texttt{bin/merge},  foram utilizados para gerar o
%\texttt{documents/references.bib} a partir dos arquivos \texttt{acm.bib},
%\texttt{ieee.bib} e \texttt{paper.bib} de cada projeto.


%Uma tabela no formato CSV com todos os artigos e projetos, indicando quando há
%relação entre eles através de menção pode ser encontrada no repositório desta
%dissertação no arquivo \texttt{documents/references.csv}, neste arquivo todos
%os artigos e todos os projetos são representados e a relação entre eles é
%marcada com um \texttt{x} sempre que houver menção de um artigo para um
%software.

%Merge único de todos os arquivos BibTeX, incluindo o arquivo \texttt{paper.bib} de
%cada projeto gerado no estudo anterior apresentado no Capítulo \ref{estudo1}, um
%arquivo \texttt{documents/references.bib} foi criado contendo todas as referências
%a todos os projetos, referências únicas.


\end{description}

\subsection{\texttt{templates/}}

% TODO
% documentar instalação das dependencias do script para filtro
% documentar a instaação do sloccount
% descrever o formato YAML utilizado para caracterização dos projetos

%produz um resultado indicando todas as linguagens e
%quanto do código total é escrito em cada uma delas.

%detalhes sobre o formato YAML, estrutura utilizada para armazenamento e
%instalação estão documentados no Apêndice \ref{reproducibilidade-do-estudo}.

%Detalhes sobre o funcionamento do script de fitro, código fonte e instruções
%de uso podem ser consultados no Apêndice \ref{reproducibilidade-do-estudo}.

%Instruções de uso do sloccount e como ele foi utilizado neste estudo está
%documentado no Apêndice \ref{reproducibilidade-do-estudo}.

%com estas palavras num script, instalamos todas as suas
%dependências de execução, os detalhes de instalação e forma de uso deste script
%e suas dependências são documentados no Apêndice
%\ref{reproducibilidade-do-estudo}.

%O formato desta planilha , sobre a estrutura de pastas utilizadas
%para armazenar os artigos, e onde encontrar o arquivo utilizado neste estudo
%para a coleta destes dados pode ser consultado em detalhes no Apêndice
%\ref{reproducibilidade-do-estudo}.

%Todos estes scripts estão disponíveis no repositório desta dissertação e foram desenvolvidos
%utilizando a a linguagem de
%programação Perl\footnote{\url{http://perl.org}} com o auxílio dos
%módulos Modern::Perl\footnote{\url{http://metacpan.org/pod/Modern::Perl}},
%YAML\footnote{\url{http://metacpan.org/pod/YAML}},
%Mojo::Template\footnote{\url{http://metacpan.org/pod/Mojo::Template}},
%Text::BibTeX\footnote{\url{http://metacpan.org/pod/Text::BibTeX}} e
%List::Util\footnote{\url{http://metacpan.org/pod/List::Util}}.
%A maior parte da lógica foi implementada no arquivo
%\texttt{lib/Dissertacao.pm} com objetivo de reduzir repetição de código e
%proporcionar reuso.

%Um documento
%CSV com todas as strings de busca pode ser encontrado no repositório desta
%dissertaçao no arquivo \texttt{documents/search-strings.csv}.

%Para auxiliar a definição da identificação para cada artigo implementamos o script \texttt{bin/ids}, este script


Apresentar detalhes de implementação, execução e uso do script bin/run-analizo, bem como detalhes para instalação do Analizo.

%\texttt{references.yml} de cada projeto onde iremos anotar e coletar os dados
%da triagem, além do template utilizado como modelo para estes arquivos
%implementamos também 

%cada artigo presente no arquivo \texttt{documents/references.bib} será
%inspecionado, os dados coletados neste passo serão armazenados no arquivo
%\texttt{is\_software\_mentioned} deve ser criado com os valores \texttt{yes} ou
%\texttt{no} indicando se o artigo menciona ou não o projeto.

%Implementamos scripts para transformar os dados da busca em formato BibTeX para
%arquivos no formato YML pois a coleta e inspeção dos artigos serão registrados
%em arquivos YML para cada projeto de software. Estes scripts recebem como entrada
%os resultados da busca de todos os projetos e geram como saída os arquivos \texttt{references.bib}
%e \texttt{references.yml} para cada projeto.

%no arquivo
%\texttt{references.yml} no campo \texttt{review} sobre como o software é
%mencionado naquele artigo.

%para cada projeto, a Listagem \ref{references-yml} apresenta um exemplo do
%trecho do YML para armazenar a extração dessas informações.


Apresentar e detalhar os arquivos scam-links.md e ase-link.md (citado no estudo1:preparacao).


O script utilizado na segunda atividade da revisão estruturada -- {\it (2)
Filtro} -- também está neste mesmo repositório no arquivo {\it
dataset/revisao-estruturada/filter}\footnote{\url{http://github.com/joenio/dissertacao-ufba-2016/blob/master/revisao-estruturada/filter}}
escrito em linguagem Perl especialmente para este estudo.

%de cada artigo é realizada com o auxílio da funcionalidade de busca
%do leitor de pdf utilizado neste estudo\footnote{Utilizamos software Evince v3.22.1}
%utilizado para leitura dos artigos, com o auxílio da busca encontramos cada
%ocorrência ao nome do software tomando nota a confirmação sobre a menção
%encontrada.

