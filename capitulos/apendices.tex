\xchapter{Revisão estruturada}
{Este capítulo apresenta as detalhes e números da revisão estruturada.}
\label{apendice-revisao-estruturada}

\section{Edições das conferências revisadas}
\label{edicoes-conferencias}

\subsection{SCAM - Source Code Analysis and Manipulation Working Conference}

\begin{itemize}
  \item SCAM 2001 - {\small http://ieeexplore.ieee.org/xpl/mostRecentIssue.jsp?punumber=7667}
  \item SCAM 2002 - {\small http://ieeexplore.ieee.org/xpl/mostRecentIssue.jsp?punumber=6494367}
  \item SCAM 2003 - {\small http://ieeexplore.ieee.org/xpl/mostRecentIssue.jsp?punumber=8773}
  \item SCAM 2004 - {\small http://ieeexplore.ieee.org/xpl/mostRecentIssue.jsp?punumber=9523}
  \item SCAM 2005 - {\small http://ieeexplore.ieee.org/xpl/mostRecentIssue.jsp?punumber=10344}
  \item SCAM 2006 - {\small http://ieeexplore.ieee.org/xpl/mostRecentIssue.jsp?punumber=4026839}
  \item SCAM 2007 - {\small http://ieeexplore.ieee.org/xpl/mostRecentIssue.jsp?punumber=4362882}
  \item SCAM 2008 - {\small http://ieeexplore.ieee.org/xpl/mostRecentIssue.jsp?punumber=4637522}
  \item SCAM 2009 - {\small http://ieeexplore.ieee.org/xpl/mostRecentIssue.jsp?punumber=5279860}
  \item SCAM 2010 - {\small http://ieeexplore.ieee.org/xpl/mostRecentIssue.jsp?punumber=5600365}
  \item SCAM 2011 - {\small http://ieeexplore.ieee.org/xpl/mostRecentIssue.jsp?punumber=6063701}
  \item SCAM 2012 - {\small http://ieeexplore.ieee.org/xpl/mostRecentIssue.jsp?punumber=6389882}
  \item SCAM 2013 - {\small http://ieeexplore.ieee.org/xpl/mostRecentIssue.jsp?punumber=6636284}
  \item SCAM 2014 - {\small http://ieeexplore.ieee.org/xpl/mostRecentIssue.jsp?punumber=6970367}
  \item SCAM 2015 - {\small http://ieeexplore.ieee.org/xpl/mostRecentIssue.jsp?punumber=7321933}
\end{itemize}

\subsection{ASE - Automated Software Engineering}

Até o ano de 1996 a conferencia ASE se chamava KBSE - Knowledge-Based Software
Engineering Conference e a partir de 1997 passou a se chamar ASE - Automated
Software Conference.

\begin{itemize}
  \item ASE/KBSE 1991 - {\small http://ieeexplore.ieee.org/xpl/mostRecentIssue.jsp?punumber=5069}
  \item ASE/KBSE 1992 - {\small http://ieeexplore.ieee.org/xpl/mostRecentIssue.jsp?punumber=421}
  \item ASE/KBSE 1993 - {\small http://ieeexplore.ieee.org/xpl/mostRecentIssue.jsp?punumber=921}
  \item ASE/KBSE 1994 - {\small http://ieeexplore.ieee.org/xpl/mostRecentIssue.jsp?punumber=995}
  \item ASE/KBSE 1995 - {\small http://ieeexplore.ieee.org/xpl/mostRecentIssue.jsp?punumber=3518}
  \item ASE/KBSE 1996 - {\small http://ieeexplore.ieee.org/xpl/mostRecentIssue.jsp?punumber=4065}
  \item ASE 1997 - {\small http://ieeexplore.ieee.org/xpl/mostRecentIssue.jsp?punumber=5003}
  \item ASE 1998 - {\small http://ieeexplore.ieee.org/xpl/mostRecentIssue.jsp?punumber=5935}
  \item ASE 1999 - {\small http://ieeexplore.ieee.org/xpl/mostRecentIssue.jsp?punumber=6516}
  \item ASE 2000 - {\small http://ieeexplore.ieee.org/xpl/mostRecentIssue.jsp?punumber=7013}
  \item ASE 2001 - {\small http://ieeexplore.ieee.org/xpl/mostRecentIssue.jsp?punumber=7763}
  \item ASE 2002 - {\small http://ieeexplore.ieee.org/xpl/mostRecentIssue.jsp?punumber=8183}
  \item ASE 2003 - {\small http://ieeexplore.ieee.org/xpl/conhome.jsp?punumber=1000064}
  \item ASE 2004 - {\small http://ieeexplore.ieee.org/xpl/mostRecentIssue.jsp?punumber=9305}
  \item ASE 2005 - {\small http://dl.acm.org/citation.cfm?id=1101908}
  \item ASE 2006 - {\small http://ieeexplore.ieee.org/xpl/mostRecentIssue.jsp?punumber=4019543}
  \item ASE 2007 - {\small http://dl.acm.org/citation.cfm?id=1321631}
  \item ASE 2008 - {\small http://ieeexplore.ieee.org/xpl/mostRecentIssue.jsp?punumber=4639292}
  \item ASE 2009 - {\small http://ieeexplore.ieee.org/xpl/mostRecentIssue.jsp?punumber=5431684}
  \item ASE 2010 - {\small http://dl.acm.org/citation.cfm?id=1858996}
  \item ASE 2011 - {\small http://ieeexplore.ieee.org/xpl/mostRecentIssue.jsp?punumber=6093623}
  \item ASE 2012 - {\small http://ieeexplore.ieee.org/xpl/mostRecentIssue.jsp?punumber=6494367}
  \item ASE 2013 - {\small http://ieeexplore.ieee.org/xpl/mostRecentIssue.jsp?punumber=6684409}
  \item ASE 2014 - {\small http://dl.acm.org/citation.cfm?id=2642937}
  \item ASE 2015 - {\small http://ieeexplore.ieee.org/xpl/mostRecentIssue.jsp?punumber=7371449}
\end{itemize}

\section{Softwares científicos selecionados}
\label{softwares-cientificos}

Resumo dos softwares selecionados na revisão estruturada.

\begin{table}[h]
\caption{Resultados da revisão estruturada para cada edição do SCAM}
\centering
\begin{tabular}{| l | c | c | c |}
  \hline
  Edição & (1) Busca & (2) Filtro & (3) Seleção \\
  \hline
  SCAM 2001 & 23    & 6         & 1           \\
  SCAM 2002 & 18    & 6         & 5           \\
  SCAM 2003 & 21    & 8         & 3           \\
  SCAM 2004 & 17    & 3         & 1           \\
  SCAM 2005 & 19    & 7         & 1           \\
  SCAM 2006 & 22    & 10        & 7           \\
  SCAM 2007 & 23    & 7         & 2           \\
  SCAM 2008 & 29    & 14        & 2           \\
  SCAM 2009 & 20    & 10        & -           \\
  SCAM 2010 & 21    & 15        & 5           \\
  SCAM 2011 & 21    & 10        & 2           \\
  SCAM 2012 & 22    & 12        & 4           \\
  SCAM 2013 & 24    & 13        & 2           \\
  SCAM 2014 & 36    & 16        & 4           \\
  SCAM 2015 & 30    & 18        & 2           \\
  \hline
  Total     & 346   & 155       & 41          \\
  \hline
\end{tabular}
\label{artigos-do-scam}
\end{table}

\begin{table}[h]
\caption{Resultados da revisão estruturada para cada edição do ASE}
\centering
\begin{tabular}{| l | c | c | c |}
  \hline
  Edição & (1) Busca & (2) Filtro & (3) Seleção \\
  \hline
  ASE 1991 & 28    & -         & -           \\
  ASE 1992 & 25    & -         & -           \\
  ASE 1993 & 21    & -         & -           \\
  ASE 1994 & 23    & -         & -           \\
  ASE 1995 & 23    & -         & -           \\
  ASE 1996 & 15    & -         & -           \\
  ASE 1997 & 47    & 1         & -           \\
  ASE 1998 & 44    & 4         & -           \\
  ASE 1999 & 50    & -         & -           \\
  ASE 2000 & 44    & 2         & -           \\
  ASE 2001 & 68    & 7         & 2           \\
  ASE 2002 & 46    & 5         & -           \\
  ASE 2003 & 54    & 5         & 2           \\
  ASE 2004 & 68    & 7         & -           \\
  ASE 2005 & 79    & 9         & 1           \\
  ASE 2006 & 61    & 12        & 2           \\
  ASE 2007 & 102   & 19        & 5           \\
  ASE 2008 & 90    & 18        & 5           \\
  ASE 2009 & 89    & 19        & 11          \\
  ASE 2010 & 87    & 20        & 3           \\
  ASE 2011 & 112   & 28        & 5           \\
  ASE 2012 & 68    & 16        & 5           \\
  ASE 2013 & 90    & 28        & 7           \\
  ASE 2014 & 100   & 39        & 7           \\
  ASE 2015 & 99    & 42        & 7           \\
  \hline
  Total    & 1533  & 281       & 62          \\
  \hline
\end{tabular}
\label{artigos-do-ase}
\end{table}

\begin{table}[h]
{\scriptsize
\caption{Resumo das ferramentas selecionadas}
\centering
\begin{tabular}{| l | l | l | l |}
  \hline
  Ferramenta     & Descrição                            & Ferramenta       & Descrição                            \\
  \hline
  2LS            & análise de terminação de programas   & JstereoCode      & detecção de esteriótipos Java        \\
  AccessAnalysis & cálculo de métricas IGAT e IGAM      & JSysDG           & construção de grafo de dependência   \\
  ADiMat         & transformação source-to-source       & Jtop             & gestão de casos de teste             \\
  AGENDA         & teste de banco de dados relacional   & Kiasan/Bogor     & verificação de modelos               \\
  AMNESIA        & detecção de ataques SQL injection    & LIBCROOS         & localização de bugs                  \\
  AMOEBA         & correção de erros de modelagem       & Loopfrog         & verificação de modelos               \\
  APIExample     & documentação de API Java             & Lotrack          & análise estática de configuração     \\
  BEG            & verificação de modelos Java          & MAGIC            & program slicing                      \\
  BEST           & predição de violação de concorrência & MemSafe          & detecta violação de acesso a memória \\
  CAWDOR         & defesa e deteção de worms            & MPAnalyzer       & análise de padrões                   \\
  CBFA           & detecção de concerns                 & MSP              & criação de modelo de acesso a memória \\
  ccJava         & linguagem orientada a aspectos       & mygcc            & verificação de programas C           \\
  CIVL           & verificação de programa concorrente  & Parfait          & localização de bugs                  \\
  ClassSplitter  & particionamento de classes           & PARSEWeb         & sugestão de uso de código e bibliotecas \\
  CodeBoost      & transformação source-to-source C++   & PHP AiR          & ?                                    \\
  CodeHow        & busca e pesquisa em código           & PAT              & ambiente de teste automático         \\
  composite      & verificação de modelos               & ProgramCutter    & particionamento automático           \\
  CONBOL         & criação automática de teste          & protopurity      & análise de impacto                   \\
  COPES          & detecção de falhas de permissão      & Pseudogen        & transformação source-to-pseudo-source \\
  CPA+           & análise configurável de programa     & PtrTracker       & detecção de bugs                     \\
  CPPCHECKER     & detecção de inconsistência em macros & PtYasm           & verificação de modelos               \\
  CSeq           & transformação source-to-source       & PuMoC            & verificação de modelos               \\
  DCR            & ?                                    & PYTHIA           & criação automática de casos de teste \\
  DDVerify       & checagem de modelos                  & ReAssert         & localização de falhas em testes      \\
  Decor          & detecção de defeitos de design       & Relda            & localização de vazamento de recurso  \\
  Derailer       & localização de falhas de segurança   & Rêve             & verificação de regressão             \\
  Diagnosys      & construção de interfaces de debug    & ReWeb            & refatoração de aplicações Web        \\
  DMS            & transformação de programas           & RRFinder         & mineração de liberação de recursos   \\
  Doc2Spec       & inferência de especificação de API   & SAFEWApp         & análise de performance javascript    \\
  DOMPLETION     & sugestão de código javascript        & Sapid/XML        & representação intermediária em XML   \\
  Drails         & detecção de erros de runtime         & SCATR            & mapeamento baseado em código         \\
  EJB            & criação de diagramas de sequência    & Scoria           & localização de violação arquitetural \\
  ELAN           & inspeção de resultados de defeitos   & Smack            & checagem de memória                  \\
  EMBER          & cálculo de métricas de código fonte  & SonarQubePlug-in & SourceMeter plugin para análise Java \\
  e-munity       & verificação de segurança             & SPARTA           & segurança e detecção de malwares     \\
  error-prone    & localização de bugs                  & srcML            & transformação source-to-source       \\
  ESBMC          & verificação de modelos               & SUDS             & detecção de bugs                     \\
  ETXL           & transformação de código              & SWAT             & teste automático para aplicação web  \\
  FaultBuster    & refatoração de code smells           & SymCrash         & geração de casos de teste            \\
  Flowgen        & criação de grafos UML                & TACLE            & type-analysis e visualizaçao de call-graph \\
  GASR           & query lógica para AspectJ            & TEBA             & transformação source-to-source       \\
  Goanna         & checagem de modelos                  & Templar          & visualização de código               \\
  GRT            & geração automática de testes         & TestEra          & geração automática de testes         \\
  GUIZMO         & inferência de layout                 & Tikanga          & detecção de violação                 \\
  GumTree        & análise e comparação de mudanças     & UMGAR            & geração de modelos UML               \\
  HaRe           & refatoração de código funcional      & VADA             & análise de dependencia de variáveis  \\
  HUSACCT        & ?                                    & Vdiff            & visualização de diferença de código  \\
  iMaus          & detecção de código duplicado         & WAIVE+           & validação de parâmetros HTTP-request \\
  Impendulo      & mineração de repositórios            & WALA             & análise de {\it bytecode} Java       \\
  Indus          & biblioteca de program slicing        & Wrangler         & refatoração de código Erlang         \\
  ISIS4J         & identificação de concerns            & X-DEVELOP        & refatoração                          \\
  JastAdd        & descrição e gramática de atributos   & XOgastan         & transformação source-to-source       \\
  JFlow          & transformação source-to-source       &                  &                                      \\
  \hline
\end{tabular}
\label{resumo-softwares}
}
\end{table}

\section{Reproducibilidade do estudo}
\label{reproducibilidade-do-estudo}

Os artigos analisados na revisão estruturada estão todos documentados arquivo
{\it
dataset/dataset.ods}\footnote{\url{http://github.com/joenio/dissertacao-ufba-2016/blob/master/dataset/dataset.ods}},
uma planilha no formato aberto {\it Open Document Format for Office
Applications}\footnote{\url{http://www.oasis-open.org/committees/office}}.

Nesta planilha está documentada cada etapa da revisão estruturada, indicando em
cada artigo analisado qual o estado do mesmo, se foi ou não incluído na
execução da atividade.  Nesta planilha é possível encontrar também o nome de
cada ferramenta e uma caracterização completa.

O script utilizado na segunda atividade da revisão estruturada -- {\it (2)
Filtro} -- também está neste mesmo repositório no arquivo {\it
dataset/revisao-estruturada/filter}\footnote{\url{http://github.com/joenio/dissertacao-ufba-2016/blob/master/revisao-estruturada/filter}}
escrito em linguagem Perl especialmente para este estudo.

A maior parte das atividades de pesquisa, reuniões de orientação e comunicação
realizadas neste estudo estão também documentadas em {\it issues} neste
repositório e na wiki do grupo de pesquisa aSide.

\begin{itemize}
  \item \url{http://wiki.dcc.ufba.br/Aside/Orientacao2014JoenioCosta}
  \item \url{https://github.com/joenio/dissertacao-ufba-2016/issues?q=}
\end{itemize}
