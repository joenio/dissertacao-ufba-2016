\xchapter{Caracterização das ferramentas}
{Este capítulo apresenta a caracterização inicial das ferramentas selecionadas a partir da revisão estruturada.}
\label{caracterizacao-ferramentas}

\section{AccessAnalysis}

AccessAnalysis é um plugin do Eclipse de análise estática 
para cálculo das métricas IGAT e IGAM
publicadas no artigo ``AccessAnalysis — A Tool for Measuring the
Appropriateness of Access Modifiers in Java Systems'' do SCAM 2012,
disponível em \url{http://accessanalysis.sourceforge.net}. O código-fonte
utilizado em nosso estudo obtido no site da ferramenta foi o
\texttt{AccessAnalysis-1.2-src.zip}. Características da ferramenta:

\begin{description}

  \item {\it Lançamentos ({\it Releases}) - quantos lançamentos por ano:}
    \begin{table}[h!]
      \centering
      \begin{tabular}{| l | l |}
        \hline
        Ano  & Lançamentos    \\
        \hline
        2012 & 1.2, 1.1, 1.0  \\
        2010 & 0.17           \\
        \hline
      \end{tabular}
    \end{table}
    \begin{itemize}
      \item Obsoleta 0 vezes ao ano - intervalo entre novos lançamentos é maior que 1 ano
    \end{itemize}

  \item {\it Linguagem de programação - em qual linguagem a ferramenta é escrita:}
    \begin{itemize}
      \item Java
    \end{itemize}

\end{description}

\section{error-prone}

error-prone é uma ferramenta de localização de bugs construída em cima do
compilador {\it javac} publicada no artigo ``Building Useful Program Analysis
Tools Using an Extensible Java Compiler'' do SCAM 2012 disponível em
\url{http://code.google.com/p/error-prone}. O código-fonte utilizado em nosso
estudo obtido no site da ferramenta foi o \texttt{error-prone-2.0.9.tar.gz}.
Características da ferramenta:

\begin{description}

  \item {\it Lançamentos ({\it Releases}) - quantos lançamentos por ano:}
    \begin{table}[h!]
      \centering
      \begin{tabular}{| l | l |}
        \hline
        Ano  & Lançamentos                                          \\
        \hline
        2016 & 2.0.9, 2.0.8                                         \\
        2015 & 2.0.7, 2.0.6, 2.0.5, 2.0.4, 2.0.3, 2.0.2, 2.0.1, 2.0 \\
        \hline
      \end{tabular}
    \end{table}
    \begin{itemize}
      \item Frequentemente $>=$ 3 vezes ao ano - novas versões da ferramenta são lançadas 3 ou mais vezes por ano
    \end{itemize}

  \item {\it Linguagem de programação - em qual linguagem a ferramenta é escrita:}
    \begin{itemize}
      \item Java
    \end{itemize}

\end{description}

\section{Indus}

Indus é uma biblioteca de {\it program
slicing}\footnote{http://en.wikipedia.org/wiki/Program\_slicing} publicada no
artigo ``An Overview of the Indus Framework for Analysis and Slicing of
Concurrent Java Software'' do SCAM 2006, disponível em
\url{http://indus.projects.cis.ksu.edu}.  O projeto está organizado em três
módulos, os seguintes arquivos, contendo o código-fonte dos três módulos,
foram copiados localmente para análise:
\texttt{indus.indus-src-20091220.zip},
\texttt{indus.javaslicer-src-20091220.zip} e
\texttt{indus.staticanalyses-src-20070305.zip}. Características da ferramenta:

\begin{description}

  \item {\it Lançamentos ({\it Releases}) - quantos lançamentos por ano:}
    \begin{table}[h!]
      \centering
      \begin{tabular}{| l | l |}
        \hline
        Ano  & Lançamentos                              \\
        \hline
        2007 & 0.8.1, 0.8.3.10, 0.8.3.11, 0.8.3.12, 0.8.3.15, 0.8.3.1, 0.8.3.6, 0.8.3.7, 0.8.3, 0.8 \\
        2006 & 0.7.0, 0.7.1, 0.7.2.2, 0.7.2             \\
        2005 & 0.6.1, 0.6.2, 0.6.3, 0.6.4.1, 0.6.4, 0.5 \\
        2004 & 0.3, 0.2, 0.1, 0.1a                      \\
        \hline
      \end{tabular}
    \end{table}
    \begin{itemize}
      \item Obsoleta $0$ vezes ao ano - intervalo entre novos lançamentos é maior que 1 ano
    \end{itemize}

  \item {\it Linguagem de programação - em qual linguagem a ferramenta é escrita:}
    \begin{itemize}
      \item Java
    \end{itemize}

\end{description}

\section{InputTracer}

InputTracer é uma ferramenta de análise dinâmica de binários x86 em Linux
publicado no artigo ``InputTracer: A Data-flow Analysis Tool for Manual
Program Comprehension of x86 Binaries'' do SCAM 2012 disponível em:
\url{http://www.ida.liu.se/divisions/adit/security/InputTracer}. O
código-fonte utilizado em nosso estudo obtido no site da ferramenta foi o
\texttt{valgrind-inputtracer.tar.gz}.  Características da ferramenta:

\begin{description}

  \item {\it Lançamentos ({\it Releases}) - quantos lançamentos por ano:}
    \begin{table}[h!]
      \centering
      \begin{tabular}{| l | l |}
        \hline
        Ano  & Lançamentos                       \\
        \hline
        2011 & 3.6.1                             \\
        2010 & 3.6.0                             \\
        2009 & 3.5.0, 3.4.1, 3.4.0               \\
        2008 & 3.3.1                             \\
        2007 & 3.3.0, 3.2.3, 3.2.2               \\
        2006 & 3.2.1, 3.2.0, 3.1.1               \\
        2005 & 3.1.0, 3.0.1, 3.0.0, 2.4.1, 2.4.0 \\
        2004 & 2.2.0, 2.2.0, 2.1.2, 2.1.1        \\
        2003 & 2.1.0, 2.0.0, 1.9.6, 1.9.5        \\
        \hline
      \end{tabular}
    \end{table}
    \begin{itemize}
      \item Obsoleta $0$ vezes ao ano - intervalo entre novos lançamentos é maior que 1 ano
    \end{itemize}

  \item {\it Linguagem de programação - em qual linguagem a ferramenta é escrita:}
    \begin{itemize}
      \item C
    \end{itemize}

\end{description}

\section{JastAdd}

JastAdd é um sistema para análise de código-fonte através da descrição de
atributos via gramática de atributos (AG) publicado no artigo ``Extending
Attribute Grammars with Collection Attributes – Evaluation and Applications''
do SCAM 2007 disponível em \url{http://jastadd.cs.lth.se/web}. O código-fonte
utilizado em nosso estudo obtido no site da ferramenta foi o
\texttt{jastadd2-src.zip}. Características da ferramenta:

\begin{description}

  \item {\it Lançamentos ({\it Releases}) - quantos lançamentos por ano:}
    \begin{table}[h!]
      \centering
      \begin{tabular}{| l | l |}
        \hline
        Ano  & Lançamentos                              \\
        \hline
        2016 & 2.2.2, 2.2.1, 2.2.1, 2.2.0               \\
        2015 & 2.1.13, 2.1.12, 2.1.11                   \\
        2014 & 2.1.10, 2.1.9, 2.1.8, 2.1.7              \\
        2013 & 2.1.6, 2.1.5, 2.1.4, 2.1.3, 2.1.2, 2.1.1 \\
        \hline
      \end{tabular}
    \end{table}
    \begin{itemize}
      \item Frequentemente $>=$ 3 vezes ao ano - novas versões da ferramenta são lançadas 3 ou mais vezes por ano
    \end{itemize}

  \item {\it Linguagem de programação - em qual linguagem a ferramenta é escrita:}
    \begin{itemize}
      \item Java
    \end{itemize}

\end{description}

\section{Sonar Qube Plug-in}

Sonar Qube Plug-in é um plugin para o SourceMeter que extende a análise de
código Java com o uso do SonarQube publicado no artigo ``SourceMeter SonarQube
plug-in'' do SCAM 2014 disponível em:
\url{http://github.com/FrontEndART/SonarQube-plug-in}. O código-fonte
utilizado em nosso estudo obtido no site da ferramenta foi o
\texttt{SonarQube-plug-in-master.zip}. Características da ferramenta:

\begin{description}

  \item {\it Lançamentos ({\it Releases}) - quantos lançamentos por ano:}
    \begin{table}[h!]
      \centering
      \begin{tabular}{| l | l |}
        \hline
        Ano  & Lançamentos       \\
        \hline
        2016 & 8.0               \\
        2015 & 7.0.5, 7.0.4, 7.0 \\
        \hline
      \end{tabular}
    \end{table}
    \begin{itemize}
      \item Frequentemente $>=$ 3 vezes ao ano - novas versões da ferramenta são lançadas 3 ou mais vezes por ano
    \end{itemize}

  \item {\it Linguagem de programação - em qual linguagem a ferramenta é escrita:}
    \begin{itemize}
      \item Java
    \end{itemize}

\end{description}

\section{srcML}

srcML é um formato texto para representação de código-fonte e um conjunto de
ferramentas de transformação {\it source-to-source} publicada no artigo
``Lightweight Transformation and Fact Extraction with the srcML Toolkit'' do
SCAM 2011 disponível em
\url{http://www.sdml.info/projects/srcml/trunk}\footnote{este endereço
retornou "not found" em contato com os autores por email indicaram que o
projeto foi movido para http://www.srcML.org}. O código-fonte utilizado em
nosso estudo obtido no site da ferramenta foi o \texttt{srcML-src.tar.gz}.
Características da ferramenta:

\begin{description}

  \item {\it Lançamentos ({\it Releases}) - quantos lançamentos por ano:}
    \begin{table}[h!]
      \centering
      \begin{tabular}{| l | l |}
        \hline
        Ano  & Lançamentos                                                     \\
        \hline
        2015 & 0.9.5                                                           \\
        2014 & 0.8.0, Trunk 19109c, Trunk 19109b, Trunk 19109                  \\
        2013 & Trunk 17088                                                     \\
        2012 & Trunk 13990, Trunk 13953, Trunk 13925, Trunk 13528, Trunk 12359 \\
        2011 & Trunk 8007, Trunk 7990, Trunk 7481                              \\
        \hline
      \end{tabular}
    \end{table}
    \begin{itemize}
      \item Ocasionalmente $<$ 3 vezes ao ano - novas versões da ferramenta são lançadas menos que 3 vezes ao ano
    \end{itemize}

  \item {\it Linguagem de programação - em qual linguagem a ferramenta é escrita:}
    \begin{itemize}
      \item C++
    \end{itemize}

\end{description}

\section{TACLE}

TACLE é um plugin do Eclipse para análise de tipo ({\it Type Analysis}) e
construção de visualizaçao de grafos de chamada ({\it Call Graph}) publicado
no artigo ``Estimating the Run-Time Progress of a Call Graph Construction
Algorithm'' do SCAM 2006 disponível em
\url{http://presto.cse.ohio-state.edu/tacle}\footnote{este link está
indisponível, por email os autores indicaram o endereço
http://web.cse.ohio-state.edu/~rountev/presto/tacle/TACLE\_Download/tacle.html}.
O código-fonte utilizado em nosso estudo obtido no site da ferramenta foi o
\texttt{tacle\_1\_2\_1\_src.zip}. Características da ferramenta:

\begin{description}

  \item {\it Lançamentos ({\it Releases}) - quantos lançamentos por ano:}
    \begin{table}[h!]
      \centering
      \begin{tabular}{| l | l |}
        \hline
        Ano  & Lançamentos  \\
        \hline
        2006 & 1.2.1, 1.2.0 \\
        2005 & 1.1.0, 1.0.0 \\
        \hline
      \end{tabular}
    \end{table}
    \begin{itemize}
      \item Obsoleta $0$ vezes ao ano - intervalo entre novos lançamentos é maior que 1 ano
    \end{itemize}

  \item {\it Linguagem de programação - em qual linguagem a ferramenta é escrita:}
    \begin{itemize}
      \item Java
    \end{itemize}

\end{description}

\section{WALA}

WALA é uma ferramenta de análise estática para {\it bytecode} Java publicado
no artigo ``Effective Static Analysis to Find Concurrency Bugs In Java'' do
SCAM 2010 disponível em
\url{http://wala.sourceforge.net/wiki/index.php/Main_Page}. O código-fonte
utilizado em nosso estudo obtido no site da ferramenta foi o
\texttt{WALA-R\_1.3.8.tar.gz}. Características da ferramenta:

\begin{description}

  \item {\it Lançamentos ({\it Releases}) - quantos lançamentos por ano:}
    \begin{table}[h!]
      \centering
      \begin{tabular}{| l | l |}
        \hline
        Ano  & Lançamentos                        \\
        \hline
        2016 & 1.3.9                              \\
        2015 & 1.3.8, 1.3.7                       \\
        2013 & 1.3.6, 1.3.5                       \\
        2012 & 1.3.4, 1.3.3                       \\
        2011 & 1.3.2                              \\
        2010 & 1.3.1                              \\
        2009 & 1.3, 1.2.2                         \\
        2008 & 1.2.1, 1.2, 1.1.3, 1.1.2           \\
        2007 & 1.1.1, 1.1, 1.0.04, 1.0.03, 1.0.02 \\
        2006 & 1.0                                \\
        \hline
      \end{tabular}
    \end{table}
    \begin{itemize}
      \item Ocasionalmente $<$ 3 vezes ao ano - novas versões da ferramenta são lançadas menos que 3 vezes ao ano
    \end{itemize}

  \item {\it Linguagem de programação - em qual linguagem a ferramenta é escrita:}
    \begin{itemize}
      \item Java
    \end{itemize}

\end{description}

\section{Clang Static Analyzer}

O {\it Clang Static Analyzer} é uma ferramenta de análise de código-fonte para
localização de bugs em códigos C, C++, e Objective-C disponível em
\url{http://clang-analyzer.llvm.org}. É distribuído junto ao código do próprio
projeto Clang\footnote{http://clang.llvm.org} e em nosso estudo utilizamos o
código em \texttt{cfe-3.7.1.src.tar.xz}. Características da ferramenta:

\begin{description}

  \item {\it Lançamentos ({\it Releases}) - quantos lançamentos por ano:}
    \begin{table}[h!]
      \centering
      \begin{tabular}{| l | l |}
        \hline
        Ano  & Lançamentos                              \\
        \hline
        2016 & 3.8.0, 3.7.1                             \\
        2015 & 3.7.0, 3.6.2, 3.6.1, 3.6.0, 3.5.2, 3.5.1 \\
        2014 & 3.5.0, 3.4.2, 3.4.1, 3.4                 \\
        2013 & 3.3                                      \\
        2012 & 3.2, 3.1                                 \\
        2011 & 3.0, 2.9                                 \\
        2010 & 2.8, 2.7                                 \\
        2009 & 2.6, 2.5                                 \\
        2008 & 2.4, 2.3, 2.2                            \\
        2007 & 2.1, 2.0                                 \\
        2006 & 1.9, 1.8, 1.7                            \\
        2005 & 1.6, 1.5                                 \\
        2004 & 1.4, 1.3, 1.2                            \\
        2003 & 1.1, 1.0                                 \\
        \hline
      \end{tabular}
    \end{table}
    \begin{itemize}
      \item Frequentemente $>=$ 3 vezes ao ano - novas versões da ferramenta são lançadas 3 ou mais vezes por ano
    \end{itemize}

  \item {\it Linguagem de programação - em qual linguagem a ferramenta é escrita:}
    \begin{itemize}
      \item C++
    \end{itemize}

\end{description}

\section{Closure Compiler}

{\it Closure Compiler} é um compilador que traduz código JavaScript em outro
JavaScript melhor e mais otimizado, está disponível em
\url{https://developers.google.com/closure/compiler}\footnote{O código fonte do
Closure Compiler pode ser obtido em:
http://github.com/google/closure-compiler} e foi utilizado em nosso estudo o
seguinte lançamento
\texttt{closure-compiler-closure-compiler-parent-v20160619.tar.gz}.
Características da ferramenta:

\begin{description}

  \item {\it Lançamentos ({\it Releases}) - quantos lançamentos por ano:}
    \begin{table}[h!]
      \centering
      \begin{tabular}{| l | l |}
        \hline
        Ano  & Lançamentos                              \\
        \hline
        2016 & v20160619, v20160517, v20160315, v20160208, v20160125 \\
        2015 & v20151216, v20151015, v20150920, v20150901, v20150729, v20150609, \\
             & v20150505, v20150315, v20150126 \\
        2014 & v20141215, v20141120, v20141023, v20140923, v20140814, v20140730, \\
             & v20140625, v20140508, v20140407, v20140303, v20140110 \\
        2013 & v20131118, v20131014, v20130823, v20130722, v20130603, v20130411, \\
             & v20130227, v20110811, v20110322, v20110405, v20110119, v20111003, \\
             & v20111114, v20112023, v20120305, v20120430, v20120711, v20121212, \\
             & v20120917 \\
        \hline
      \end{tabular}
    \end{table}
    \begin{itemize}
      \item Frequentemente $>=$ 3 vezes ao ano - novas versões da ferramenta são lançadas 3 ou mais vezes por ano
    \end{itemize}

  \item {\it Linguagem de programação - em qual linguagem a ferramenta é escrita:}
    \begin{itemize}
      \item Java
    \end{itemize}

\end{description}

\section{Cppcheck}

Ferramenta de análise estática de código C/C++ para checagem de vazamento de
memória, erros de alocação, entre outras falhas. Disponível em
\url{http://sourceforge.net/projects/cppcheck}. Em nosso estudo utilizamos o
código em \texttt{cppcheck-1.72.tar.bz2}. Características da ferramenta:

\begin{description}

  \item {\it Lançamentos ({\it Releases}) - quantos lançamentos por ano:}
    \begin{table}[h!]
      \centering
      \begin{tabular}{| l | l |}
        \hline
        Ano  & Lançamentos                                                \\
        \hline
        2016 & 1.74, 1.73, 1.72                                           \\
        2015 & 1.71, 1.70, 1.69 1.68                                      \\
        2014 & 1.67, 1.66, 1.65, 1.64, 1.63                               \\
        2013 & 1.62, 1.61, 1.60.1, 1.60, 1.59, 1.58                       \\
        2012 & 1.57, 1.56, 1.55, 1.54, 1.53                               \\
        2011 & 1.52, 1.51, 1.50, 1.49, 1.48, 1.47                         \\
        2010 & 1.46.1, 1.46, 1.45, 1.44, 1.43, 1.42, 1.41, 1.40           \\
        2009 & 1.39, 1.38, 1.37, 1.36, 1.35, 1.34, 1.33, 1.32, 1.31, 1.30 \\
        \hline
      \end{tabular}
    \end{table}
    \begin{itemize}
      \item Frequentemente $>=$ 3 vezes ao ano - novas versões da ferramenta são lançadas 3 ou mais vezes por ano
    \end{itemize}

  \item {\it Linguagem de programação - em qual linguagem a ferramenta é escrita:}
    \begin{itemize}
      \item C++
    \end{itemize}

\end{description}

\section{CQual}

Ferramenta de análise de typo ({\it type-based analysis}) que fornece um
mecanismo leve e prático para especificação e verificação de propriedades de
programas C. Disponível em \url{http://www.cs.umd.edu/~jfoster/cqual}. Em
nosso estudo utilizamos o código em \texttt{cqual-0.981.tar.gz}.
Características da ferramenta:

\begin{description}

  \item {\it Lançamentos ({\it Releases}) - quantos lançamentos por ano:}
    \begin{table}[h!]
      \centering
      \begin{tabular}{| l | l |}
        \hline
        Ano  & Lançamentos  \\
        \hline
        2004 & 0.981, 0.991 \\
        2003 & 0.98, 0.99   \\
        \hline
      \end{tabular}
    \end{table}
    \begin{itemize}
      \item Obsoleta $0$ vezes ao ano - intervalo entre novos lançamentos é maior que 1 ano
    \end{itemize}

  \item {\it Linguagem de programação - em qual linguagem a ferramenta é escrita:}
    \begin{itemize}
      \item C
    \end{itemize}

\end{description}

\section{FindBugs}

Uma ferramenta para localização de bugs em código Java disponível em
\url{http://findbugs.sourceforge.net}. Em nosso estudo utilizamos o código em
\texttt{findbugs-3.0.1-source.zip}. Características da ferramenta:

\begin{description}

  \item {\it Lançamentos ({\it Releases}) - quantos lançamentos por ano:}
    \begin{table}[h!]
      \centering
      \begin{tabular}{| l | l |}
        \hline
        Ano  & Lançamentos                \\
        \hline
        2015 & 3.0.1                      \\
        2014 & 3.0.0                      \\
        2013 & 2.0.3                      \\
        2012 & 2.0.2, 2.0.1               \\
        2011 & 2.0.0                      \\
        2009 & 1.3.9, 1.3.8               \\
        2008 & 1.3.7, 1.3.6, 1.3.5, 1.3.4 \\
        2007 & 1.2.1                      \\
        \hline
      \end{tabular}
    \end{table}
    \begin{itemize}
      \item Ocasionalmente $<$ 3 vezes ao ano - novas versões da ferramenta são lançadas menos que 3 vezes ao ano
    \end{itemize}

  \item {\it Linguagem de programação - em qual linguagem a ferramenta é escrita:}
    \begin{itemize}
      \item Java
    \end{itemize}

\end{description}

\section{FindSecurityBugs}

Plugin do FindBugs para auditoria de segurança em aplicações web Java,
disponível em \url{http://find-sec-bugs.github.io}.  O código-fonte utilizado
em nosso estudo obtido no site da ferramenta foi o
\texttt{findsecbugs-plugin-1.4.5-sources.jar}. Características da ferramenta:

\begin{description}

  \item {\it Lançamentos ({\it Releases}) - quantos lançamentos por ano:}
    \begin{table}[h!]
      \centering
      \begin{tabular}{| l | l |}
        \hline
        Ano  & Lançamentos                                     \\
        \hline
        2016 & 1.4.6, 1.4.5                                    \\
        2015 & 1.4.4, 1.4.3, 1.4.2, 1.4.1, 1.4.0, 1.3.1, 1.3.0 \\
        2014 & 1.2.1                                           \\
        2013 & 1.2.0, 1.1.0                                    \\
        2012 & 1.0.0                                           \\
        \hline
      \end{tabular}
    \end{table}
    \begin{itemize}
      \item Frequentemente $>=$ 3 vezes ao ano - novas versões da ferramenta são lançadas 3 ou mais vezes por ano
    \end{itemize}

  \item {\it Linguagem de programação - em qual linguagem a ferramenta é escrita:}
    \begin{itemize}
      \item Java
    \end{itemize}

\end{description}

\section{Jlint}

Uma ferramenta para verificaçao de código Java em busca de bugs,
inconsistências e problemas de sincronização disponível em
\url{http://sourceforge.net/projects/jlint}.  O código-fonte utilizado em
nosso estudo obtido no site da ferramenta foi o \texttt{jlint-3.1.2.zip}.
Características da ferramenta:

\begin{description}

  \item {\it Lançamentos ({\it Releases}) - quantos lançamentos por ano:}
    \begin{table}[h!]
      \centering
      \begin{tabular}{| l | l |}
        \hline
        Ano  & Lançamentos \\
        \hline
        2011 & 3.1.2       \\
        2010 & 3.1.1       \\
        2006 & 3.1         \\
        2004 & 3.0         \\
        \hline
      \end{tabular}
    \end{table}
    \begin{itemize}
      \item Obsoleta $0$ vezes ao ano - intervalo entre novos lançamentos é maior que 1 ano
    \end{itemize}

  \item {\it Linguagem de programação - em qual linguagem a ferramenta é escrita:}
    \begin{itemize}
      \item C++
    \end{itemize}

\end{description}

\section{Pixy}

Ferramenta de análise estática de código PHP para verificação de
vulnerabilidades de segurança. Disponível em
\url{http://github.com/oliverklee/pixy}. O código-fonte utilizado em nosso
estudo obtido no site da ferramenta foi o \texttt{pixy-master.zip}.
Características da ferramenta:

\begin{description}

  \item {\it Lançamentos ({\it Releases}) - quantos lançamentos por ano:}
    \begin{table}[h!]
      \centering
      \begin{tabular}{| l | l |}
        \hline
        Ano  & Lançamentos \\
        \hline
        2012 & 3.0.3       \\
        \hline
      \end{tabular}
    \end{table}
    \begin{itemize}
      \item Obsoleta $0$ vezes ao ano - intervalo entre novos lançamentos é maior que 1 ano
    \end{itemize}

  \item {\it Linguagem de programação - em qual linguagem a ferramenta é escrita:}
    \begin{itemize}
      \item Java
    \end{itemize}

\end{description}

\section{PMD}

Ferramenta de análise de código-fonte para localização falhas comuns de
programação com suporte a várias linguagens, disponível em
\url{http://pmd.github.io}.  O código-fonte utilizado em nosso estudo obtido
no site da ferramenta foi o \texttt{pmd-src-5.4.1.zip}. Características da
ferramenta:

\begin{description}

  \item {\it Lançamentos ({\it Releases}) - quantos lançamentos por ano:}
    \begin{table}[h!]
      \centering
      \begin{tabular}{| l | l |}
        \hline
        Ano  & Lançamentos                                                   \\
        \hline
        2016 & 5.5.0, 5.4.2, 5.3.7                                           \\
        2015 & 5.4.1, 5.4.0, 5.3.6, 5.3.5, 5.3.4, 5.3.3, 5.3.2, 5.3.1, 5.3.0 \\
        2014 & 5.2.3, 5.2.2, 5.2.1, 5.2.0, 5.1.3, 5.1.2, 5.1.1, 5.1.0        \\
        2013 & 5.0.5, 5.0.4, 5.0.3, 5.0.2                                    \\
        2012 & 5.0.1, 5.0.0                                                  \\
        2011 & 4.3.0, 4.2.6                                                  \\
        2009 & 4.2.5                                                         \\
        \hline
      \end{tabular}
    \end{table}
    \begin{itemize}
      \item Frequentemente $>=$ 3 vezes ao ano - novas versões da ferramenta são lançadas 3 ou mais vezes por ano
    \end{itemize}

  \item {\it Linguagem de programação - em qual linguagem a ferramenta é escrita:}
    \begin{itemize}
      \item Java
    \end{itemize}

\end{description}

\section{RATS}

Ferramenta de análise estática para auditoria de segurança disponível em
\url{http://code.google.com/archive/p/rough-auditing-tool-for-security}. O
código-fonte utilizado em nosso estudo obtido no site da ferramenta foi o
\texttt{rats-2.4.tgz}. Características da ferramenta:

\begin{description}

  \item {\it Lançamentos ({\it Releases}) - quantos lançamentos por ano:}
    \begin{table}[h!]
      \centering
      \begin{tabular}{| l | l |}
        \hline
        Ano  & Lançamentos \\
        \hline
        2013 & 2.4         \\
        2009 & 2.3         \\
        ??   & 1.5         \\
        \hline
      \end{tabular}
    \end{table}
    \begin{itemize}
      \item Obsoleta $0$ vezes ao ano - intervalo entre novos lançamentos é maior que 1 ano
    \end{itemize}

  \item {\it Linguagem de programação - em qual linguagem a ferramenta é escrita:}
    \begin{itemize}
      \item C
    \end{itemize}

\end{description}

\section{Smatch}

Ferramenta de análise estática para detecção de erros no Kernel disponível em
\url{http://smatch.sourceforge.net}. O código-fonte utilizado em nosso estudo
obtido no site da ferramenta foi o \texttt{smatch.git}. Características da
ferramenta:

\begin{description}

  \item {\it Lançamentos ({\it Releases}) - quantos lançamentos por ano:}
    \begin{table}[h!]
      \centering
      \begin{tabular}{| l | l |}
        \hline
        Ano  & Lançamentos      \\
        \hline
        2015 & 1.60             \\
        2013 & 1.59, 1.58, 1.57 \\
        2012 & 1.56             \\
        2010 & 1.55, 1.54       \\
        2009 & 1.53, 1.52, 1.51 \\
        \hline
      \end{tabular}
    \end{table}
    \begin{itemize}
      \item Ocasionalmente $<$ 3 vezes ao ano - novas versões da ferramenta são lançadas menos que 3 vezes ao ano
    \end{itemize}

  \item {\it Linguagem de programação - em qual linguagem a ferramenta é escrita:}
    \begin{itemize}
      \item C
    \end{itemize}

\end{description}

\section{Splint}

Splint is a tool for statically checking C programs for security
vulnerabilities and coding mistakes Ferramenta para verificação de programas
por vulnerabilidades de segurança e erros de código. Disponível em
\url{http://www.splint.org}. O código-fonte utilizado em nosso estudo obtido
no site da ferramenta foi o \texttt{splint-3.1.2.src.tgz}. Características da
ferramenta:

\begin{description}

  \item {\it Lançamentos ({\it Releases}) - quantos lançamentos por ano:}
    \begin{table}[h!]
      \centering
      \begin{tabular}{| l | l |}
        \hline
        Ano  & Lançamentos                        \\
        \hline
        2007 & 3.1.2                              \\
        2003 & 3.1.0                              \\
        ??   & 3.0.1.6                            \\
        2002 & 3.0.1.5                            \\
        ??   & 3.0.1.4, 3.0.1, 3.0.0.19, 3.0.0.18 \\
        ??   & 3.0.0.17, 3.0.0.15, 3.0.0.14       \\
        2001 & 3.0.0.13                           \\
        \hline
      \end{tabular}
    \end{table}
    \begin{itemize}
      \item Obsoleta $0$ vezes ao ano - intervalo entre novos lançamentos é maior que 1 ano
    \end{itemize}

  \item {\it Linguagem de programação - em qual linguagem a ferramenta é escrita:}
    \begin{itemize}
      \item C
    \end{itemize}

\end{description}

\section{UNO}

Uma ferramenta de análise de código-fonte para detecção de defeitos.
Disponível em \url{http://spinroot.com/uno}. O código-fonte utilizado em nosso
estudo obtido no site da ferramenta foi o \texttt{uno\_v213.tar.gz}.
Características da ferramenta:

\begin{description}

  \item {\it Lançamentos ({\it Releases}) - quantos lançamentos por ano:}
    \begin{table}[h!]
      \centering
      \begin{tabular}{| l | l |}
        \hline
        Ano  & Lançamentos                       \\
        \hline
        2007 & 2.13, 2.12, 2.11                  \\
        2006 & 2.9-2.10                          \\
        2005 & 2.8, 2.7, 2.6, 2.5, 2.4           \\
        2004 & 2.3, 2.2, 2.1, 2.0, 1.7           \\
        2003 & 1.8, 1.6, 1.5, 1.4, 1.3, 1.2, 1.1 \\
        \hline
      \end{tabular}
    \end{table}
    \begin{itemize}
      \item Obsoleta $0$ vezes ao ano - intervalo entre novos lançamentos é maior que 1 ano
    \end{itemize}

  \item {\it Linguagem de programação - em qual linguagem a ferramenta é escrita:}
    \begin{itemize}
      \item C
    \end{itemize}

\end{description}

\section{WAP}

Ferramenta para análise estática de código-fonte e mineraçao de dados para
detectar e corrigir vulnerabilidades em aplicações web. Disponível em
\url{http://awap.sourceforge.net}. O código-fonte utilizado em nosso estudo
obtido no site da ferramenta foi o \texttt{wap-2.1.tar.gz}. Características da
ferramenta:

\begin{description}

  \item {\it Lançamentos ({\it Releases}) - quantos lançamentos por ano:}
    \begin{table}[h!]
      \centering
      \begin{tabular}{| l | l |}
        \hline
        Ano  & Lançamentos                                 \\
        \hline
        2015 & 2.1, 2.0.5, 2.0.4, 2.0.3, 2.0.2, 2.0.1, 2.0 \\
        \hline
      \end{tabular}
    \end{table}
    \begin{itemize}
      \item Frequentemente $>=$ 3 vezes ao ano - novas versões da ferramenta são lançadas 3 ou mais vezes por ano
    \end{itemize}

  \item {\it Linguagem de programação - em qual linguagem a ferramenta é escrita:}
    \begin{itemize}
      \item Java
    \end{itemize}

\end{description}

\xchapter{Análise exploratória dos valores das métricas}
{Este capítulo apresenta uma análise exploratória e interpretação dos valores das métricas coletadas para cada ferramenta.}
\label{analise-metricas}

\subsection{Conexões aferentes de uma classe (ACC)}

ACC é um valor parcial de uma das métricas MOOD (Metrics for Object Oriented
Design) \cite{Brito1994} e mede o nível de acoplamento de uma classe. O
cálculo é feito através do número de classes que fazem referência a um outra
por meio de métodos ou atributos.

As Tabelas \ref{metrica-acc} e \ref{metrica-acc-industria} apresentam os
valores da métrica ACC para as ferramentas da academia e da indústria,
respectivamente.

%% begin.rcode metrica-acc, fig.align='center', results="asis"
% table = percentis_by_project("acc")
% total_modules = metric_by_project("total_modules")
% table = add_column(table, total_modules, colname = "classes")
% knitr_latex_table(table, "percentis da métrica ACC para as ferramentas da academia", "metrica-acc")
%% end.rcode

Duas ferramentas, accessanalysis e error-prone, tiveram valor 0 no percentil
75, é estranho que 75\% das classes destas 2 ferramentas, ambas escritas em
Java, não façam acesso através de métodos ou atributos a nenhuma outra classe
do mesmo sistema.

%% begin.rcode metrica-acc-industria, fig.align='center', results="asis"
% table = percentis_by_nist_project("acc")
% total_modules = metric_by_nist_project("total_modules")
% table = add_column(table, total_modules, colname = "classes")
% knitr_latex_table(table, "percentis da métrica ACC para as ferramentas da indústria", "metrica-acc-industria")
%% end.rcode

As ferramentas cqual e uno tiveram valores no percentil 75\% bem acima das
demais ferramentas, 24 e 34, respectivamente. Onde entre todas as outras o
maior valor foi 8.

\subsection{Média de complexidade ciclomática por método (ACCM)}

ACCM contabiliza o número de caminhos independentes que métodos de uma classe
pode seguir em sua execução. O cálculo é feito a partir do número de
estruturas condicionais encontrados nos métodos de um programa.

As Tabelas \ref{metrica-accm} e \ref{metrica-accm-industria} apresentam os
valores da métrica ACCM para as ferramentas da academia e da indústria,
respectivamente.

%% begin.rcode metrica-accm, fig.align='center', results="asis"
% table = percentis_by_project("accm")
% total_modules = metric_by_project("total_modules")
% table = add_column(table, total_modules, colname = "classes")
% knitr_latex_table(table, "percentis da métrica ACCM para as ferramentas da academia", "metrica-accm")
%% end.rcode

Esta métrica apresenta um comportamento sem muitas exceções dentro de cada
percentil, com intervalos entre 1.0 e 2.1 para percentil 75\%, 2.0 e 3.4 para
percentil 90\% e 2.9 e 5.0 para percentil 95\%.

%% begin.rcode metrica-accm-industria, fig.align='center', results="asis"
% table = percentis_by_nist_project("accm")
% total_modules = metric_by_nist_project("total_modules")
% table = add_column(table, total_modules, colname = "classes")
% knitr_latex_table(table, "percentis da métrica ACCM para as ferramentas da indústria", "metrica-accm-industria")
%% end.rcode

Também entre as ferramentas da indústria esta métrica não apresenta variações
muito grandes, com intervalos entre 1.0 e 6.9 para 75\%, 2.0 e 8.9 para
percentil 90\% e entre 4.0 e 15.6 para 95\%.

\subsection{Média do número de linhas de código por método (AMLOC)}

AMLOC é a média do número de linhas dos métodos de um módulo, apenas linhas
com código executável é calculada, comentários e linhas em branco são
desconsideradas do cálculo.

As Tabelas \ref{metrica-amloc} e \ref{metrica-amloc-industria} apresentam a
métrica AMLOC para as ferramentas da academia e da indústria, respectivamente.

%% begin.rcode metrica-amloc, fig.align='center', results="asis"
% table = percentis_by_project("amloc")
% total_modules = metric_by_project("total_modules")
% table = add_column(table, total_modules, colname = "classes")
% knitr_latex_table(table, "percentis da métrica AMLOC para as ferramentas da academia", "metrica-amloc")
%% end.rcode

Os valores para ferramentas da academia não nos chama atenção para nada
especial, com intervalos entre 5.4 e 14.5 para o percentil 75\%, entre 10.7 e
31.5 pra percentil 90\% e entre 15.4 e 51.4 para 95\%.

%% begin.rcode metrica-amloc-industria, fig.align='center', results="asis"
% table = percentis_by_nist_project("amloc")
% total_modules = metric_by_nist_project("total_modules")
% table = add_column(table, total_modules, colname = "classes")
% knitr_latex_table(table, "percentis da métrica AMLOC para as ferramentas da indústria", "metrica-amloc-industria")
%% end.rcode

Para a indústria os intervalos foram entre 7.0 e 36.9 para o percentil 75\%,
entre 16.0 e 62.1 para percentil 90\% e entre 22.0 e 119.7 para 95\%. A
ferramenta chama atenção por apresentr o menor valor no percentil 75\%, 7, e o
maior valor no prcentil 95\%, 119.7.

\subsection{Média do número de parâmetros por método (ANPM)}

ANPM é a média de parâmetros dos métodos de uma classe.

As Tabelas \ref{metrica-anpm} e \ref{metrica-anpm-industria} apresentam a
métrica ANPM para as ferramentas da academia e da indústria, respectivamente.

%% begin.rcode metrica-anpm, fig.align='center', results="asis"
% table = percentis_by_project("anpm")
% total_modules = metric_by_project("total_modules")
% table = add_column(table, total_modules, colname = "classes")
% knitr_latex_table(table, "percentis da métrica ANPM para as ferramentas da academia", "metrica-anpm")
%% end.rcode

%% begin.rcode metrica-anpm-industria, fig.align='center', results="asis"
% table = percentis_by_nist_project("anpm")
% total_modules = metric_by_nist_project("total_modules")
% table = add_column(table, total_modules, colname = "classes")
% knitr_latex_table(table, "percentis da métrica ANPM para as ferramentas da indústria", "metrica-anpm-industria")
%% end.rcode

\subsection{Acoplamento entre objetos (CBO)}

CBO é a recíproca da métrica ACC e mede quantas classes são utilizadas por uma
certa classe.

As Tabelas \ref{metrica-cbo} e \ref{metrica-cbo-industria} apresentam a
métrica CBO para as ferramentas da academia e da indústria, respectivamente.

%% begin.rcode metrica-cbo, fig.align='center', results="asis"
% table = percentis_by_project("cbo")
% total_modules = metric_by_project("total_modules")
% table = add_column(table, total_modules, colname = "classes")
% knitr_latex_table(table, "percentis da métrica CBO para as ferramentas da academia", "metrica-cbo")
%% end.rcode

%% begin.rcode metrica-cbo-industria, fig.align='center', results="asis"
% table = percentis_by_nist_project("cbo")
% total_modules = metric_by_nist_project("total_modules")
% table = add_column(table, total_modules, colname = "classes")
% knitr_latex_table(table, "percentis da métrica CBO para as ferramentas da indústria", "metrica-cbo-industria")
%% end.rcode

\subsection{Profundidade da árvore de herança (DIT)}

DIT mede a profundidade que uma classe se encontra na árvore de herança.

Os intervalos sugeridos são: até 2 (bom); entre 2 e 4 (regular); de 4 em
diante (ruim).

As Tabelas \ref{metrica-dit} e \ref{metrica-dit-industria} apresentam a
métrica DIT para as ferramentas da academia e da indústria, respectivamente.

%% begin.rcode metrica-dit, fig.align='center', results="asis"
% table = percentis_by_project("dit")
% total_modules = metric_by_project("total_modules")
% table = add_column(table, total_modules, colname = "classes")
% knitr_latex_table(table, "percentis da métrica DIT para as ferramentas da academia", "metrica-dit")
%% end.rcode

%% begin.rcode metrica-dit-industria, fig.align='center', results="asis"
% table = percentis_by_nist_project("dit")
% total_modules = metric_by_nist_project("total_modules")
% table = add_column(table, total_modules, colname = "classes")
% knitr_latex_table(table, "percentis da métrica DIT para as ferramentas da indústria", "metrica-dit-industria")
%% end.rcode

\subsection{Ausência de coesão em métodos (LCOM4)}

LCOM4 calcula quantos conjuntos de métodos relacionados existem dentro de uma
classe, isto é, métodos que compartilham utilização de algum atributo ou que
se referenciam.

Os intervalos sugeridos para código C++ e Java são: até 2 (bom); entre 2 e 5
(regular); de 5 em diante (ruim).

As Tabelas \ref{metrica-lcom4} e \ref{metrica-lcom4-industria} apresentam a
métrica LCOM4 para as ferramentas da academia e da indústria, respectivamente.

%% begin.rcode metrica-lcom4, fig.align='center', results="asis"
% table = percentis_by_project("lcom4")
% total_modules = metric_by_project("total_modules")
% table = add_column(table, total_modules, colname = "classes")
% knitr_latex_table(table, "percentis da métrica LCOM4 para as ferramentas da academia", "metrica-lcom4")
%% end.rcode

%% begin.rcode metrica-lcom4-industria, fig.align='center', results="asis"
% table = percentis_by_nist_project("lcom4")
% total_modules = metric_by_nist_project("total_modules")
% table = add_column(table, total_modules, colname = "classes")
% knitr_latex_table(table, "percentis da métrica LCOM4 para as ferramentas da indústria", "metrica-lcom4-industria")
%% end.rcode

\subsection{Número de linhas de código (LOC)}

LOC é a medida mais comum para o tamanho de um software, conta o número linhas
executáveis excluindo linhas em branco e comentários.

Os intervalos sugeridos para o LOC de uma classe (Java e C++) são: até 70
(bom); entre 70 e 130 (regular); de 130 em diante (ruim).

As Tabelas \ref{metrica-loc} e \ref{metrica-loc-industria} apresentam a
métrica LOC para as ferramentas da academia e da indústria, respectivamente.

%% begin.rcode metrica-loc, fig.align='center', results="asis"
% table = percentis_by_project("loc")
% total_modules = metric_by_project("total_modules")
% table = add_column(table, total_modules, colname = "classes")
% knitr_latex_table(table, "percentis da métrica LOC para as ferramentas da academia", "metrica-loc")
%% end.rcode

%% begin.rcode metrica-loc-industria, fig.align='center', results="asis"
% table = percentis_by_nist_project("loc")
% total_modules = metric_by_nist_project("total_modules")
% table = add_column(table, total_modules, colname = "classes")
% knitr_latex_table(table, "percentis da métrica LOC para as ferramentas da indústria", "metrica-loc-industria")
%% end.rcode

\subsection{Número de atributos (NOA)}

NOA contabiliza o número de atributos de uma classe.

As Tabelas \ref{metrica-noa} e \ref{metrica-noa-industria} apresentam a
métrica NOA para as ferramentas da academia e da indústria, respectivamente.

%% begin.rcode metrica-noa, fig.align='center', results="asis"
% table = percentis_by_project("noa")
% total_modules = metric_by_project("total_modules")
% table = add_column(table, total_modules, colname = "classes")
% knitr_latex_table(table, "percentis da métrica NOA para as ferramentas da academia", "metrica-noa")
%% end.rcode

%% begin.rcode metrica-noa-industria, fig.align='center', results="asis"
% table = percentis_by_nist_project("noa")
% total_modules = metric_by_nist_project("total_modules")
% table = add_column(table, total_modules, colname = "classes")
% knitr_latex_table(table, "percentis da métrica NOA para as ferramentas da indústria", "metrica-noa-industria")
%% end.rcode

\subsection{Número de filhos (NOC)}

NOC é o número total de flhos de uma classe.

As Tabelas \ref{metrica-noc} e \ref{metrica-noc-industria} apresentam a
métrica NOC para as ferramentas da academia e da indústria, respectivamente.

%% begin.rcode metrica-noc, fig.align='center', results="asis"
% table = percentis_by_project("noc")
% total_modules = metric_by_project("total_modules")
% table = add_column(table, total_modules, colname = "classes")
% knitr_latex_table(table, "percentis da métrica NOC para as ferramentas da academia", "metrica-noc")
%% end.rcode

%% begin.rcode metrica-noc-industria, fig.align='center', results="asis"
% table = percentis_by_nist_project("noc")
% total_modules = metric_by_nist_project("total_modules")
% table = add_column(table, total_modules, colname = "classes")
% knitr_latex_table(table, "percentis da métrica NOC para as ferramentas da indústria", "metrica-noc-industria")
%% end.rcode

\subsection{Número de métodos (NOM)}

NOM indica o tamanho das classes em termos das suas operações implementadas.

As Tabelas \ref{metrica-nom} e \ref{metrica-nom-industria} apresentam a
métrica NOM para as ferramentas da academia e da indústria, respectivamente.

%% begin.rcode metrica-nom, fig.align='center', results="asis"
% table = percentis_by_project("nom")
% total_modules = metric_by_project("total_modules")
% table = add_column(table, total_modules, colname = "classes")
% knitr_latex_table(table, "percentis da métrica NOM para as ferramentas da academia", "metrica-nom")
%% end.rcode

%% begin.rcode metrica-nom-industria, fig.align='center', results="asis"
% table = percentis_by_nist_project("nom")
% total_modules = metric_by_nist_project("total_modules")
% table = add_column(table, total_modules, colname = "classes")
% knitr_latex_table(table, "percentis da métrica NOM para as ferramentas da indústria", "metrica-nom-industria")
%% end.rcode

\subsection{Número de atributos públicos (NPA)}

NPA mede o encapsulamento entre classes.

Os intervalos sugeridos para Java e C++ são: até 1 (bom); entre 1 e 9
(regular); de 9 em diante (ruim).

As Tabelas \ref{metrica-npa} e \ref{metrica-npa-industria} apresentam a
métrica NPA para as ferramentas da academia e da indústria, respectivamente.

%% begin.rcode metrica-npa, fig.align='center', results="asis"
% table = percentis_by_project("npa")
% total_modules = metric_by_project("total_modules")
% table = add_column(table, total_modules, colname = "classes")
% knitr_latex_table(table, "percentis da métrica NPA para as ferramentas da academia", "metrica-npa")
%% end.rcode

%% begin.rcode metrica-npa-industria, fig.align='center', results="asis"
% table = percentis_by_nist_project("npa")
% total_modules = metric_by_nist_project("total_modules")
% table = add_column(table, total_modules, colname = "classes")
% knitr_latex_table(table, "percentis da métrica NPA para as ferramentas da indústria", "metrica-npa-industria")
%% end.rcode

\subsection{Número de métodos públicos (NPM)}

NPM indica o tamanho da ``interface'' da classe.

Os intervalos sugeridos para Java e C++ são: até 10 (bom); entre 10 e 40
(regular); de 40 em diante (ruim).

As Tabelas \ref{metrica-npm} e \ref{metrica-npm-industria} apresentam a
métrica NPA para as ferramentas da academia e da indústria, respectivamente.

%% begin.rcode metrica-npm, fig.align='center', results="asis"
% table = percentis_by_project("npm")
% total_modules = metric_by_project("total_modules")
% table = add_column(table, total_modules, colname = "classes")
% knitr_latex_table(table, "percentis da métrica NPM para as ferramentas da academia", "metrica-npm")
%% end.rcode

%% begin.rcode metrica-npm-industria, fig.align='center', results="asis"
% table = percentis_by_nist_project("npm")
% total_modules = metric_by_nist_project("total_modules")
% table = add_column(table, total_modules, colname = "classes")
% knitr_latex_table(table, "percentis da métrica NPM para as ferramentas da indústria", "metrica-npm-industria")
%% end.rcode

\subsection{Resposta para uma classe (RFC)}

RFC conta o número de métodos que podem ser executados a partir de uma
mensagem enviada a um objeto dessa classe.

As Tabelas \ref{metrica-rfc} e \ref{metrica-rfc-industria} apresentam a
métrica RFC para as ferramentas da academia e da indústria, respectivamente.

%% begin.rcode metrica-rfc, fig.align='center', results="asis"
% table = percentis_by_project("rfc")
% total_modules = metric_by_project("total_modules")
% table = add_column(table, total_modules, colname = "classes")
% knitr_latex_table(table, "percentis da métrica RFC para as ferramentas da academia", "metrica-rfc")
%% end.rcode

%% begin.rcode metrica-rfc-industria, fig.align='center', results="asis"
% table = percentis_by_nist_project("rfc")
% total_modules = metric_by_nist_project("total_modules")
% table = add_column(table, total_modules, colname = "classes")
% knitr_latex_table(table, "percentis da métrica RFC para as ferramentas da indústria", "metrica-rfc-industria")
%% end.rcode

\subsection{Complexidade estrutural (SC)}

SC é medida através da combinação das métricas de acoplamento (CBO) e coesão
(LCOM4).

As Tabelas \ref{metrica-sc} e \ref{metrica-sc-industria} apresentam a
métrica SC para as ferramentas da academia e da indústria, respectivamente.

%% begin.rcode metrica-sc, fig.align='center', results="asis"
% table = percentis_by_project("sc")
% total_modules = metric_by_project("total_modules")
% table = add_column(table, total_modules, colname = "classes")
% knitr_latex_table(table, "percentis da métrica SC para as ferramentas da academia", "metrica-sc")
%% end.rcode

%% begin.rcode sumario-sc, fig.align='center', results="asis"
% table = percentis_by_project("sc")
% table = table[-1:-5,]
% table = table[-4,]
% xt = xtable(summary(t(table)), caption="resumo da métrica SC nos percentis 75, 90 e 95 para as ferramentas da academia")
% print(xt, table.placement="H")
%% end.rcode

%% begin.rcode metrica-sc-industria, fig.align='center', results="asis"
% table = percentis_by_nist_project("sc")
% total_modules = metric_by_nist_project("total_modules")
% table = add_column(table, total_modules, colname = "classes")
% knitr_latex_table(table, "percentis da métrica SC para as ferramentas da indústria", "metrica-sc-industria")
%% end.rcode

%% begin.rcode sumario-sc-industria, fig.align='center', results="asis"
% table = percentis_by_nist_project("sc")
% table = table[-1:-5,]
% table = table[-4,]
% xt = xtable(summary(t(table)), caption="resumo da métrica SC nos percentis 75, 90 e 95 para as ferramentas da indústria")
% print(xt, table.placement="H")
%% end.rcode

\xchapter{Histogramas}{}

%% begin.rcode histograma-acc, fig.align='center', results="asis"
% histograma('acc', 'histograma da métrica ACC para todas as ferramentas')
%% end.rcode

%% begin.rcode histograma-accm, fig.align='center', results="asis"
% histograma('accm', 'histograma da métrica ACCM para todas as ferramentas')
%% end.rcode

%% begin.rcode histograma-cbo, fig.align='center', results="asis"
% histograma('cbo', 'histograma da métrica CBO para todas as ferramentas')
%% end.rcode

%% begin.rcode histograma-loc, fig.align='center', results="asis"
% histograma('loc', 'histograma da métrica LOC para todas as ferramentas')
%% end.rcode

%% begin.rcode histograma-amloc, fig.align='center', results="asis"
% histograma('amloc', 'histograma da métrica AMLOC para todas as ferramentas')
%% end.rcode

%% begin.rcode histograma-anpm, fig.align='center', results="asis"
% histograma('anpm', 'histograma da métrica ANPM para todas as ferramentas')
%% end.rcode

%% begin.rcode histograma-dit, fig.align='center', results="asis"
% histograma('dit', 'histograma da métrica DIR para todas as ferramentas')
%% end.rcode

%% begin.rcode histograma-rfc, fig.align='center', results="asis"
% histograma('rfc', 'histograma da métrica RFC para todas as ferramentas')
%% end.rcode

%% begin.rcode histograma-noa, fig.align='center', results="asis"
% histograma('noa', 'histograma da métrica NOA para todas as ferramentas')
%% end.rcode

%% begin.rcode histograma-noc, fig.align='center', results="asis"
% histograma('noc', 'histograma da métrica NOC para todas as ferramentas')
%% end.rcode

%% begin.rcode histograma-npm, fig.align='center', results="asis"
% histograma('npm', 'histograma da métrica NPM para todas as ferramentas')
%% end.rcode

%% begin.rcode histograma-nom, fig.align='center', results="asis"
% histograma('nom', 'histograma da métrica NOM para todas as ferramentas')
%% end.rcode

%% begin.rcode histograma-sc, fig.align='center', results="asis"
% histograma('sc', 'histograma da métrica SC para todas as ferramentas')
%% end.rcode

