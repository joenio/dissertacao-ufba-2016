Qual a pergunta geral de pesquisa??:

O software acadêmico de análise estática de código fonte desenvolvido nos últimos anos é sustentável? Qual o impacto de sua sustentabilidade técnica na reprodutibilidade da pesquisa em ES que o utiliza?

\section{Objetivos}

O objetivo geral deste trabalho é analisar a influência da sustentabilidade técnica de software acadêmico de análise estática sobre a reprodutibilidade de estudos científicos na área de Engenharia de Software.

São objetivos específicos desta pesquisa:

\begin{enumerate}
  \item Definir atributos para caracterização da sustentabilidade técnica de software acadêmico de análise estática. \textcolor{red}{Joenio, para mim, manutenibilidade é uma das dimensões de sustentabilidade.}
  \item Definir atributos para caracterização da reproducibilidade de software acadêmico de análise estática.
  \item Caracterizar a sustentabilidade técnica de software acadêmico  de análise estática.
  \item Caracterizar a reproducibilidade software acadêmico  de análise estática.
  \item Interpretar o impacto da sustentabilidade técnica do software acadêmico de análise estática na reprodutibilidade de seus estudos.
\end{enumerate}

\section{Metodologia de trabalho}

O impacto que a sustentabilidade do
software acadêmico de análise estática causa em estudos, realizado através de uma pesquisa sobre medidas para sustentabilidade técnica. Uma sub-característica importante é a manutenabilidade de software.

O trabalho está organizado em dois estudos distintos.
O primeiro estudo tem como objetivo selecionar software  acadêmico  de análise estática e avaliar sua sustentabilidade técnica. 
%, e o segundo, com o objetivo de medir a manutenabilidade. 
O segundo estudo tem como objetivo investigar a relação entre sustentabilidade do software acadêmico e a reproducibilidade das pesquisas que o utilizam.

Ao final, a análise dos dados coletados trarão uma perspectiva sobre o impacto da sustentabilidade do software acadêmico na reprodutibilidade dos seus estudos.

A seleção do software acadêmico foi realizada através de um procedimento inspirado no mapeamento sistemático de literatura, chamado de revisão estruturada, composto de atividades para seleção de artigos e coleta de informações sobre software  acadêmico. Essa revisão analisou o histórico de 25 edições da conferência ASE (Nome completo, URL) e 15 edições da conferência SCAM (Nome completo, URL).

As informações coletadas incluem nome, descrição e o endereço onde cada software está disponível, normalmente página web ou repositório de código fonte. Estes
endereços foram verificados para confirmar se os softwares estão, de fato,
disponíveis no local indicado.

Em seguida, os softwares foram avaliados em relação à disponibilidade de código fonte e à licença utilizada. Essas informações, e as demais coletadas até aqui, foram distribuídas cronologicamente, e interpretadas numa perspectiva histórica sobre a sustentabilidade técnica dos softwares acadêmicos de análise estática.

Cada software acadêmico selecionado, com código fonte disponível, foi avaliado em relação a sua manutenabilidade através da métrica de complexidade estrutural. A coleta dessa métrica para cada software foi realizada pelo Analizo, uma suíte
de ferramentas para análise de código fonte, e está sendo considerado como um
indicador de manutenabilidade.

Um conjunto de softwares de análise estática da indústria foi incluído nesta etapa, todos os dados coletados para os softwares acadêmicos foram também coletados para este novo conjunto. Esses softwares foram então caracterizados em relação à frequencia de lançamentos, linguagem de programação e o tipo de entrada suportado.

Todas estas características foram comparadas entre sí, por exemplo, softwares com maior frequencia de lançamentos, escritos na mesma linguagem de
programação, apresentam maior complexidade estrutural? Eles são da academia ou da indústria? Software desenvolvido na indústria estudado apresenta melhor manutenabilidade do que o software acadêmico?

Essas perguntas serão respondidas através de uma análise exploratória dos dados, essa análise apresenta também uma perspectiva evolutiva de alguns softwares, aqueles com maior frequencia de lançamentos foram selecionados para
esta avaliação.

% manutenibilidade pode ser vista como uma dimensão de sustentabilidade técnica?

(continua...)

No segundo estudo, a relação entre sustentabilidade técnica do software acadêmico e a reproducibilidade de estudos que o utilizaram foi investigada. 


Perguntas:
\begin{enumerate}
\item o software foi comparado com outro software?
\item o software foi usado em outras pesquisas? do mesmo autor? do mesmo grupo? de outros grupos de pesquisa?
\item o software foi modificado para incluir novas funcionalidades ou reengenharia?  mesmo autor? do mesmo grupo? de outros grupos de pesquisa?

\end{enumerate}


