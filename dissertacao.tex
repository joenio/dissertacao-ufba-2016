\documentclass[12pt]{article}
\usepackage[utf8]{inputenc}
\usepackage[brazil]{babel}
\usepackage{fancyvrb}
\usepackage[alf]{abntcite}
%\bibliographystyle{abnt-alf}
\usepackage[top=2cm,left=0.5in,right=0.5in,bottom=2cm]{geometry}

\title{
  Ferramentas desenvolvidas durante pesquisas em engenharia de software:
  Uma análise histórica
}
\author{Joenio Marques da Costa\\
  {\small Universidade Federal da Bahia (UFBA)} \\
  {\small joenio@colivre.coop.br}
}
\date{\today}

\begin{document}

\maketitle

\section{Objetivos}

Entender como as ferramentas desenvolvidas durante pesquisas em engenharia de
software são criadas, disponibilizadas e como são mantidas ao longo do tempo.

\section{Fundamentação teórica}

Softwares são utilizados em praticamente todas as áreas do conhecimento humano
e têm exercido um papel essencial em nossa sociedade, nós dependemos cada vez
mais das características e serviços oferecidos por sistemas computadorizados
\cite{Mafra2006} e isto evidencia a necessidade de produzir softwares de
qualidade.

Não é raro perceber projetos de software com problemas em termos de
funcionalidades faltando, custos acima do esperado, prazos não cumpridos ou
qualidade abaixo do desejado.  Diante destes e de outros problemas surge a
engenharia de software, uma disciplina centrada no desenvolvimento intensivo
de sistemas de software \cite{Wohlin2012} através de uma abordagem
sistemática, disciplinada, e quantificável para o desenvolvimento, operação e
manutenção \cite{SWEBOK2014} de software. Ao fazer uso de métodos científicos,
especialmente estudos empíricos, pesquisadores e estudiosos transformam a
engenharia de software em uma disciplina mais científica e controlável,
criando meios para avaliar e validar métodos, técnicas, linguagens e
ferramentas.

Assim, através de estudos empíricos, pesquisadores da área tem realizado cada
vez mais estudos fazendo o número de publicações crescer significantemente nas
últimas décadas \cite{Stol2015}. Este aumento constante no número de
publicações desperta a atenção para a necessidade de verificar a validade de tais estudos,
saber se os resultados de um certo estudo é verdadeiro ou não é um ponto
central em pesquisa científica e entre as várias formas de verificação a
replicação é constantemente citada como um importante meio para atingir tal
objetivo \cite{Almqvist2006}.

Apesar do conceito replicação
estar usualmente associado à experimentação argumenta-se que ele deva ser
expandido para incluir também, ao menos, estudos de caso e surveys
\cite{Mantyla2010}. Um dos primeiros artigos discutindo replicação
em engenharia de software foi publicado por Basili et al.
\cite{Basili1986} e sugere replicação não apenas como uma escolha mas como um
possível "próximo passo" a ser tomado após o experimento original ser
concluído.

Diante a importancia da replicação ao validar estudos nota-se que é de
fundamental importancia que as ferramentas de software utilizadas e
desenvolvidas no estudo original estejam ainda disponíveis e em funcionamento
\cite{Kon11}. É bastante comum pesquisas de diversas áreas,
especialmente engenharia de software, propor novas ferramentas de
software ou apenas criar ferramentas como forma de apoio ao estudo, estas
ferramentas serão chamadas aqui de {\it ferramentas de
pesquisa} e serão objeto de pesquisa deste trabalho. Este termo {\it
ferramentas de pesquisa} foi também utilizado por Portillo\cite{Portillo12}
designinando o mesmo significado.

Estas {\it ferramentas de pesquisa} são produtos de software e precisam ser
avaliados a partir de métodos científicos adequados.
revisão sistemática de literatura se mostra um
excelente método para isto por ser um meio de avaliar e interpretar pesquisas
relevantes sobre uma data questão em particular, tópico área, ou fenômeno de
interesse \cite{Kitchenham2006}.

Muitos pesquisadores argumentam que para fazer progressos no campo da
engenharia de software empírica, resultados de vários experimentos e da fato
de outros estudos empíricos como surveys e estudos de caso devem ser
combinados, um método para combiná-los é a meta-análise estatistica
\cite{Almqvist2006}.

Com o aumento de estudos empíricos surge a necessidade de agregar evidências
de multiplos estudos, isto pode dar respostas a questões impossíveis de serem
respondidas com estudos inidivuais. Isto pode ser feito através da coleção e
síntese de evidências empíricas através de revisão sistemática de literatura e
devem ser feito com rigor científico.

Quando um conjunto de estudos empíricos são colecionados sobre um tópico, a
síntese ou agregação entra em cena. Síntese baseada em métodos estatísticos
são referenciados como meta-análise. Exemplos de meta-análise em engenharia de
software inclue métodos de detecção de defeitos, métodos ágeis, e programação
em par. Se os procedimentos da meta-análise não são apliváveis, síntese
descritiva deve ser utilizada. Ela inclue visualização e tabulação dos dados e
estatística descritiva dos dados. Quanto mais ampla a questão guiando a
revisão de literatura, mais métodos qualitativos são necessários para sua
síntese, Cruzes e Dyba \cite{Cruzes2011} apresentam uma visão geral de métodos
qualitativos de síntese, entre eles síntese temática.


\section{Metodologia}

Primeiramente será feito uma revisão sobre estudos secundários, especialmente
revisões sistemáticas, estes estudos visando identificar, avaliar e interpretar os
resultados relevantes a um determinado tópico de pesquisa, fenômeno de
interesse ou questão de pesquisa;

A partir daí será realizada uma revisão sistemática de literatura de artigos
da área de engenharia de software sobre publicação de ferramentas visando
caracterizar, agregar e sintetizar informações a partir das seguintes
perguntas:

\begin{itemize}
  \item Quais ferramentas de pesquisa foram publicadas durante pesquisas em
    engenharia de software e como são licenciadas e disponibilizadas?
  \item Como as ferramentas de pesquisa publicadas são desenvolvidas em termos
    de métodos e processos recomendados pela engenharia de software?
  \item Há colaboracão entre os pesquisadores e universidades no
    desenvolvimento das ferramentas de pesquisa publicadas?
  \item As ferramentas publicadas são avaliadas internamente e externamente?
    Se sim, como são avaliadas?
\end{itemize}

Com os dados em mãos, uma síntese será feita, com base em agregação
quantitativa usando meta-análise ou qualitativa através de síntese temática ou
descritiva será utilizada.

\section{Resultados esperados}

Caracterização e avaliação de ferramentas de pesquisa, ferramentas de software
desenvolvidas durante pesquisas em engenharia de software. Combinação de
resultados através de coleção e síntese quantitativa e qualitativa.

\bibliography{bibliografia}

\end{document}
