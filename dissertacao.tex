\documentclass[qual, classic, a4paper]{ufbathesis}
\usepackage[utf8]{inputenc}
\usepackage[brazil]{babel}
\usepackage{fancyvrb}
\usepackage[alf]{abntex2cite}
\usepackage{graphicx}
\usepackage{subfigure}
\usepackage{longtable}
\usepackage{subfig}
\DeclareGraphicsExtensions{.pdf}
\usepackage{multicol}

%\date{08 de Julho de 2016}
\adviser[f]{Profa. Dra. Christina von Flach G. Chavez}
\coadviser{Prof. Dr. Paulo Roberto Miranda Meirelles}

\title{
  Caracterização da complexidade estrutural em ferramentas de análise estática
  de código-fonte
}

\author{Joenio Marques da Costa\\
  {\small joenio@joenio.me}
}

\begin{document}
%\frontpage
%\frontmatter
%\presentationpage

\tableofcontents

\chapter{Introdução}

\section{Objetivos}

\section{Contribuições esperadas}

\chapter{Análise estática de código-fonte}

\section{Anatomia da análise de código-fonte}

\subsection{Extração de dados}

\subsection{Representação interna}

\subsection{Análise da representação interna}

\chapter{Métricas de código-fonte}

\section{Complexidade estrutural}

\chapter{Metodologia}

\section{Trabalhos relacionados}

\section{Hipóteses}

\section{Planejamento do estudo}

\subsection{Seleção de métricas}

\subsection{Seleção de ferramentas de análise estática}

\subsection{Revisão estruturada}

\section{Coleta de dados}

\subsection{Ferramentas da academia}

\subsection{Ferramentas da indústria}

\section{Análise de dados}

\subsection{Caracterização dos artigos}

\subsection{Caracterização das ferramentas}

\subsection{Distribuição dos valores das métricas}

\subsection{Cálculo de distância e modelo de aproximação}

\chapter{Caracterização dos artigos}

\chapter{Caracterização das ferramentas}

\section{Resultados}

\chapter{Evolução de uma ferramenta da análise estática}

\chapter{Conclusão}

\section{Limitações do trabalho}

\section{Trabalhos futuros}

\backmatter
\bibliography{bibliografia}
\appendix
\end{document}

% vim: filetype=tex
