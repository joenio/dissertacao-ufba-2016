\documentclass[12pt]{article}
\usepackage[utf8]{inputenc}
\usepackage[brazil]{babel}
\usepackage{fancyvrb}
\usepackage[alf]{abntcite}
%\bibliographystyle{abnt-alf}
\usepackage[top=2cm,left=0.5in,right=0.5in,bottom=2cm]{geometry}

\title{
  Ferramentas desenvolvidas durante pesquisas em engenharia de software:
  Uma análise histórica
}
\author{Joenio Marques da Costa\\
  {\small Universidade Federal da Bahia (UFBA)} \\
  {\small joenio@colivre.coop.br}
}
\date{\today}

\begin{document}

\maketitle

\section{Objetivos}

\section{Fundamentação teórica}

Softwares são utilizados em praticamente todas as áreas do conhecimento humano
e têm exercido um papel essencial em nossa sociedade, nós dependemos cada vez
mais das características e serviços oferecidos por sistemas computadorizados
\cite{Mafra2006}, isto evidencia a necessidade de preocupar-se com a sua
qualidade visto que muitos projetos enfrentam desafios em
termos de funcionalidades faltando, custos acima do esperado ou prazos não
cumpridos.

Diante destes desafios surge a engenharia de software, uma disciplina centrada
no desenvolvimento intensivo de sistemas de software \cite{Wohlin2012} através
da aplicação de uma abordagem sistemática, disciplinada, e quantificável para
o desenvolvimento, operação e manutenção \cite{SWEBOK2014} de software. Ao
fazer uso de métodos científicos, especialmente estudos empíricos,
pesquisadores e estudiosos transformam a engenharia de software em uma
disciplina mais científica e controlável. Possibilitando meios para
avaliar e validar novos métodos, técnicas, linguagens e ferramentas.

O foco em estudos empíricos tem crescido significantemente nas últimas décadas
\cite{Stol2015} e isto tem impacto natural no crescimento de estudos
utilizando tais métodos, sejam eles, surveys, estudos de caso, experimentos,
revisões sistemáticas de literatura ou outro. Este aumento constante no número
de publicações evidencia a necessidade de averiguar a validade de tais
estudos, saber se os resultados de um certo estudo é verdadeiro ou não é um
ponto central em qualquer pesquisa científica e a replicação é um importante
meio para atingir tal objetivo \cite{Almqvist2006}.

Apesar do conceito replicação de estudos empíricos em engenharia de software
estar usualmente associado à experimentação argumenta-se que ele deve ser
extendido para incluir também ao menos estudos de caso e surveys
\cite{Mantyla2010}. Um dos primeiros artigos discutindo replicação de
experimentos em engenharia de software foi publicado por Basili et al.
\cite{Basili1986} e sugere replicação não apenas como uma escolha mas como um
possível "próximo passo" a ser tomado após o experimento original ser
concluído.

Um dos requisitos para viabilizar replicação de estudos, seja experimentação,
estudo de caso ou surveys, é que as ferramentas de software utilizadas durante
o estudo original estejam disponíveis e em funcionamento, a disponibilidade
das ferramentas é um requisito fudamental para a replicação de estudos
\cite{Kon11}. Estas ferramentas de software usualmente são desenvolvidas para
apoiar a pesquisa ou são o resultado da própria pesquisa. Estas ferramentas
serão chamadas neste trabalho de ferramentas de pesquisa, este mesmo termo foi
utilizado também por Portillo\cite{Portillo12}, e são o objeto de pesquisa
deste trabalho.

Neste sentido, visando primariamente a capacidade de replicação, nota-se a
necessidade de avaliar as ferramentas de pesquisa diante métodos científicos
apropriados para tal e a revisão sistemática de literatura se mostra um
excelente método para isto por ser um meio de avaliar e interpretar pesquisas
relevantes sobre uma data questão em particular, tópico área, ou fenômeno de
interesse \cite{Kitchenham2006}.

Muitos pesquisadores argumentam que para fazer progressos no campo da
engenharia de software empírrica, resultados de vários experimentos e da fato
de outros estudos empíricos como surveys e estudos de caso devem ser
combinados, um método para combiná-los é a meta-análise estatistica
\cite{Almqvist2006}.

Com o aumento de estudos empíricos surge a necessidade de agregar evidências
de multiplos estudos, isto pode dar respostas a questões impossíveis de serem
respondidas com estudos inidivuais. Isto pode ser feito através da coleção e
síntese de evidências empíricas através de revisão sistemática de literatura e
devem ser feito com rigor científico.

Quando um conjunto de estudos empíricos são colecionados sobre um tópico, a
síntese ou agregação entra em cena. Síntese baseada em métodos estatísticos
são referenciados como meta-análise. Exemplos de meta-análise em engenharia de
software inclue métodos de detecção de defeitos, métodos ágeis, e programação
em par. Se os procedimentos da meta-análise não são apliváveis, síntese
descritiva deve ser utilizada. Ela inclue visualização e tabulação dos dados e
estatística descritiva dos dados. Quanto mais ampla a questão guiando a
revisão de literatura, mais métodos qualitativos são necessários para sua
síntese.  Cruzes and Dyba present an overview of qualitative synthesis methods
\cite{Cruzes11}.

\section{Metodologia}

\section{Resultados esperados}

\bibliography{bibliografia}

\end{document}
