\documentclass[12pt]{article}
\usepackage[utf8]{inputenc}
\usepackage[brazil]{babel}
\usepackage{fancyvrb}
\usepackage[alf]{abntex2cite}
\usepackage[top=2cm,left=0.5in,right=0.5in,bottom=2cm]{geometry}

\title{
  Caracterização de Softwares Científicos no Contexto de Engenharia de
  Software Experimental
}
\author{Joenio Marques da Costa\\
  {\small Universidade Federal da Bahia (UFBA)} \\
  {\small joenio@colivre.coop.br}
}
\date{\today}

% sugestão do professor Ecivaldo sobre a ordem para escrever a dissertação
% 01 Problemática
% 02 Problemas (+hipóteses se houver)
% 03 Objetivos (Geral e Específicos)
% 04 Metodologia
% 05 Revisão sistemática
% 06 Resultados parciais (se houver)
% 07 Considerações parciais
% 08 Problemática (processo ciclico)
% 09 Resumo/Abstract
% 10 Título
% 11 Revisão

% feedback durante apresentacao 01/out
% * caracterizar se a ferramenta tem manuais, isso inviabiliza o uso
% * reduzir escopo no Brasil? (chris: ser internacional eh bom)
% * reduzir escopo em domínio específico? Ex: recuperacao arquitetural
%     ** a disponibilidade das ferramentas eh muito dificil, os links estao offline
%     ** arcade, so tem um aluno mexendo, mas n sabe ensinar a ferramenta, a ferramenta tem mto recurso
% * archview eh uma ferramenta que recebe atencao de forma esporatica, faz e larga lá
% trabalho futuro
% * pegar pesquisas em desenvolvimento com criacao de ferramentas e convidar
%   para seguir boas praticas para evitar os problemas que estou mostrando

\begin{document}

\maketitle

\section{Resumo do projeto}

Ciência Aberta é um movimento que tem por objetivo tornar a pesquisa
científica, seus dados e sua disseminação acessíveis à todos os interessados,
dentre as várias iniciativas deste movimento, destaca-se a preocupação com a
reprodutibilidade dos resultados de pesquisas de forma independente e aberta,
visto que a maioria dos componentes necessários para a reprodução dos
resultados de uma pesquisa -- por exemplo, códigos fonte e dados -- não são
publicados.

Em diversas linhas de pesquisa da Computação e, em especial, em Engenharia de
Software, é bastante comum que novos sistemas de software sejam desenvolvidos,
tais sistemas costumam ser utilizados como meio para atingir os resultados da
pesquisa ou, em alguns casos, são o próprio fim do estudo realizado. Neste
trabalho, tais sistemas ou ferramentas de software são nosso objeto de
pesquisa e serão chamados de {\it software científico}.

Seja como meio ou como fim, tais {\it softwares científicos} são produtos de
software e, tal como propõe a Engenharia de Software Experimental, precisam
ser avaliadas com uso de métodos científicos adequados. Neste contexto, é de
fundamental importância que os softwares utilizados ou desenvolvidos durante
estudos estejam disponíveis e em funcionamento.

%Nas últimas décadas, o foco em estudos empíricos na área de
%Engenharia de Software tem crescido, resultando no uso crescente de métodos
%científicos como surveys, estudos de caso, experimentos e revisões
%sistemáticas de literatura.  A replicação desses estudos empíricos podem, e
%devem, ser realizados, de modo a averiguar a sua validade e aumentar o nível
%de confiança em seus resultados. 

Neste trabalho de mestrado, será realizada uma revisão sistemática de
literatura, a partir de artigos da área de Engenharia de Software que tratam
de publicação de "softwares científicos", visando coletar e caracterizar
alguns atributos de tais softwares para prover recomendações iniciais para o
seu desenvolvimento e adoção em ampla escala em estudos empíricos em
Engenharia de Software.

\section{Objetivos}

O objetivo principal deste trabalho é melhorar a compreensão sobre as
ferramentas de software para análise de código fonte desenvolvidos durante
pesquisas em Engenharia de Software -- aqui denominados "software científico"
\ -- por meio de sua caracterização no contexto de estudos empíricos em
Engenharia de Software.

São objetivos específicos deste trabalho:

\begin{itemize}
  \item Realizar uma revisão sistemática, com base em artigos publicados nas
    principais conferências e periódicos da área de Engenharia de Software,
    sobre "software científico";
  \item Caracterizar os "softwares científicos" coletados em termos de
    atributos relacionados ao seu desenvolvimento;
  \item Caracterizar os "software científicos" coletados em termos de
    atributos relacionados à sua distribuição;
  \item Caracterizar os "software científicos" coletados  em termos de
    atributos relacionados ao seu uso em estudos empíricos, sejam primários ou
    replicações;
  \item Sintetizar informações sobre "softwares científicos" coletados;
  \item Prover um conjunto inicial de recomendações para facilitar a adoção de
    "softwares científicos" em estudos empíricos em Engenharia de Software. 
\end{itemize}

\section{Fundamentação teórica}

\subsection{Ciência Aberta}

Ciência Aberta é um movimento que tem por objetivo tornar a pesquisa
científica, seus dados e sua disseminação acessíveis à todos os interessados,
sejam amadores ou profissionais \cite{WikipediaOpenScience}. Sua principal
motivação está em possibilitar a reprodução dos resultados de pesquisas e em
garantir transparência das metodologias utilizadas, isto aumenta o impacto
social das pesquisas e gera economia de tempo e dinheiro para os pesquisadores
e para as instituições \cite{Nesta2010}.

Este movimento é guiado por princípios básicos de transparência,
acessibilidade e reusabilidade universais, disseminadas via ferramentas
online, ele é dividido em quatro grandes áreas: (1) Open Access, (2) Open
Data, (3) Open Source e (4) Open Reproducible Research. Dentre elas destaca-se
a Open Reproducible Research por preocupar-se com a reprodutibilidade dos
resultados de pesquisas de forma independente \cite{Stodden2009} e aberta, no
entanto, esta área tem recebido ainda pouca atenção da comunidade de pesquisa
\cite{Nancy2015} \cite{Grand2010Open} apesar do aumento geral do interesse
pelas práticas da Ciência Aberta \cite{Grand2010}.

Enquanto pesquisadores publicam artigos descrevendo e divulgando seus
resultados, é raro que façam o mesmo com toda a produção gerada durante a
pesquisa. A maioria dos componentes necessários para a reprodução dos
resultados de uma pesquisa -- por exemplo, códigos fonte e dados -- usualmente
permanecem não publicados. Este é um problema sério já que um dos fundamentos
da ciência é que novas descobertas sejam reproduzidas antes de serem
consideradas parte da base de conhecimento \cite{Stodden2009}.

Neste sentido, \citeonline{Prlic2012} dão dicas para o desenvolvimento aberto de software
científico e citam que disponibilizar o código criado durante pesquisas não
apenas aumenta o impacto como também se torna essencial para outros
reproduzirem os resultados encontrados. Eles citam ainda que manutenabilidade
e disponibilidade do software após a publicação é o maior problema enfrentado
pelos pesquisadores que desenvolvem tais softwares, e é aí que a
participação no desenvolvimento aberto desde o início pode trazer maior
benefício.

Dentro deste contexto, e considerando que pesquisas em engenharia de software
produzem bastante softwares científicos, surge a preocupação de avaliar tais
softwares em termos de sua manutenabilidade e disponibilidade a partir de
métodos científicos adequados.

\subsection{Engenharia de Software}

Sistemas de software são utilizados em praticamente todas as áreas do
conhecimento humano e têm exercido um papel essencial em nossa sociedade
\cite{Mafra2006}. A dependência crescente de serviços oferecidos por tais
sistemas evidencia a necessidade de produzir software de qualidade,
contornando os  desafios relacionados a funcionalidades incompletas ou
incorretas, custos acima do esperado ou prazos não cumpridos.

Diante destes desafios, surge a Engenharia de Software, uma disciplina
centrada no desenvolvimento de sistemas de software \cite{Wohlin2012} através
de uma abordagem sistemática, disciplinada, e quantificável para o
desenvolvimento, operação e manutenção \cite{SWEBOK2014}.

Nas últimas décadas, o foco em estudos empíricos na área de Engenharia de
Software tem crescido significantemente \cite{Stol2015}, resultando no uso
crescente de métodos como surveys, estudos de caso, experimentos e revisões
sistemáticas de literatura. Através destes estudos empíricos, pesquisadores
transformam a Engenharia de Software em uma disciplina mais científica e
controlável -- a  Engenharia de Software Experimental -- provendo meios para
avaliar e validar métodos, técnicas, linguagens e ferramentas.

O crescimento no número de pesquisas e publicações em Engenharia de Software
Experimental desperta a atenção para a necessidade de verificar a validade dos
estudos empíricos realizados -- um ponto central em qualquer pesquisa
científica. A validade de um estudo empírico deve ser averiguada com o intuito
de aumentar o nível de confiança em seus resultados, replicação costuma ser
citado como um importante meio para atingir tal objetivo \cite{Almqvist2006}.

Um dos primeiros artigos discutindo replicação de experimentos em Engenharia
de Software foi publicado por Basili et al. \cite{Mantyla2010} e sugere
replicação não apenas como uma escolha, mas como um possível "próximo passo" a
ser tomado após o experimento original ser concluído. Apesar do conceito
replicação de estudos empíricos em Engenharia de Software estar usualmente
associado à experimentação, argumenta-se que ele deve ser estendido para
incluir ao menos estudos de caso e surveys \cite{Basili1986}.

Em diversas linhas de pesquisa da Computação e, em especial, em Engenharia de
Software, é bastante comum que novos sistemas de software sejam desenvolvidos,
tais sistemas costumam ser utilizados como meio para atingir os resultados da
pesquisa ou, em alguns casos, são o próprio fim do estudo realizado. Neste
trabalho, tais ferramentas de software são nosso objeto de pesquisa e serão
chamados de "software científico" \ -- \citeonline{Portillo12} utiliza o
termo "research tool" para designar este mesmo tipo de software.

Softwares científicos são produtos de software e, em geral, precisam ser
avaliados com uso de métodos científicos adequados, e, é de fundamental
importância que estejam disponíveis e em funcionamento \cite{Kon2011}.

%Reviewing, criticizing, and improving code is easier for readers who can run
%the code themselves. Use of languages, libraries, systems, and tools which are
%widely available is strongly recommended \cite{McCormick2014}
%\cite{Barnes2013}.
%
%Os tópicos "irreproducibility studies", "reproducibility guidelines",
%"reproducibility testing" e "webometrics" são possíveis "gaps" dentro da
%agenda da Ciência Aberta\cite{Nancy2015}.

\subsubsection{Análise de Código Fonte}

O rápido crescimento no número de softwares nas últimas décadas leva a uma
crescente demanda por mecanismos e ferramentas de apoio à compreensão de,
desenvolvedores necessitam entender em profundade a implementação de um
determinado software antes de realizar atividades de correção ou refatoração
de forma eficiente \cite{Kirkov2010}.

Isto é evidenciado ao perceber que a complexidade dos softwares vem crescendo
à cada dia \cite{Kirkov2010}, tornando assim, extremamente útil a existência
de ferramentas de análise automática de código-fonte, tais ferramentas
auxiliam os desenvolvedores e engenheiros à compreender a implementação de um
determinado software de forma profunda e abrangente.

Essas ferramentas geram modelos de alto nível representando entidades,
relacionamentos, métricas, características ou outra informação qualquer
extraída diretamente do código-fonte, e permitem que sejam geradas
visualizações em diversos níveis de abstração representando o código-fonte.

% Dentre as inúmeras sub-áreas da engenharia de software, este trabalho irá
% focar do domínio de ferramentas de análise de código-fonte, esta escolha tem
% por base o domínio do pesquisador nesta área e a escolha de um domínio é
% necessário para reduzir o escopo e viabilizar a revisão de estudos e
% ferramentas de um domínio específico, do contrário o trabalho seria muito
% extenso e a revisão sistemática seria inviável dentro do prazo de um trabalho
% de pesquisa num mestrado.

% baixar artigo da rede ufba:
% Source Code Analysis: A Road Map
% http://dl.acm.org/citation.cfm?id=1254713

\subsection{Revisão Sistemática e Meta-análise}

Muitos pesquisadores argumentam que, para se obter progressos em uma
determinada área do conhecimento, os resultados de vários experimentos e
outros estudos empíricos -- surveys e estudos de caso, por exemplo -- devem
ser combinados. Quando um conjunto de estudos empíricos é coletado sobre um
tópico, a síntese ou agregação entra em cena.

Quando os procedimentos da meta-análise, ou seja, síntese baseada em métodos
estatísticos \cite{Almqvist2006}, não são aplicáveis, a síntese descritiva
deve ser utilizada. Esta inclui visualização, tabulação e estatística
descritiva dos dados. Quanto mais ampla é a questão guiando a revisão de
literatura, mais métodos qualitativos são necessários para sua síntese.
\citeonline{Cruzes2011} apresentam uma visão geral de métodos qualitativos de
síntese, entre eles a síntese temática.

Adicionalmente, com o aumento na adoção de estudos empíricos em Engenharia de
Software, surge a necessidade de agregar evidências de múltiplos estudos
relacionados, de modo a obter respostas a questões impossíveis de serem
respondidas com os estudos individuais. A coleta e síntese de evidências
empíricas podem ser realizadas com rigor científico, por meio de Revisão
Sistemática da Literatura \cite{Kitchenham2007}. A Revisão Sistemática da
Literatura  é um meio de avaliar e interpretar pesquisas relevantes -- estudos
primários -- sobre uma dada questão em particular, tópico, área, ou fenômeno
de interesse.

\section{Metodologia}

Primeiramente será feita uma revisão sobre estudos secundários -- em especial,
revisões sistemáticas -- relacionados ao uso e desenvolvimento de "softwares
científicos".

Em seguida, será realizada uma revisão sistemática de literatura, com base nas
recomendações encontradas em \cite{Kitchenham2007}, a partir de artigos da
área de Engenharia de Software que tratam de publicação de "softwares
científicos" para análise de código fonte, visando caracterizar seus atributos
a partir das seguites perguntas:

\begin{itemize}
  \item Quais são os "softwares científicos" associados a pesquisas em
    engenharia de software?
  \item Como os "softwares científicos" publicados nestes estudos são
    licenciados e distribuídos?
  \item Como os "softwares científicos" são desenvolvidos, em termos de
    métodos e processos recomendados pela engenharia de software?
  \item Há colaboração entre pesquisadores de universidades distintas no
    desenvolvimento dos "softwares científicos" publicados?
  \item Os "softwares científicos" são avaliados internamente e externamente?
    Se sim, como são avaliados?
\end{itemize}

A partir das informações coletadas na revisão sistemática será feita agregação
e síntese dos dados, utilizando métodos quantitativos e qualitativos, com
objetivo de identificar temas recorrentes e problemas comuns, além de elaborar
conclusão a respeito dos diversos estudos analisados na revisão sistemática.

Após a caracterização realizada, espera-se propor um conjunto preliminar de
recomendações para desenvolvimento e adoção de "softwares científicos" para a
comunidade acadêmica de Engenharia de Software.

\section{Resultados esperados}

\begin{itemize}
  \item Caracterização dos atributos das ferramentas de software desenvolvidas
    durante pesquisas em engenharia de software, chamadas aqui de \textit{"softwares científicos"};
  \item Síntese dos resultados e lições a respeito de temas recorrentes e
    problemas comuns no desenvolvimento e publicação de \textit{"softwares
      científicos"};
  \item Conjunto preliminar de recomendações para desenvolvimento e adoção de
    \textit{"softwares científicos"} para a comunidade acadêmica de Engenharia de
    Software;
  \item Artigos científicos publicados;
  \item Dissertação de mestrado.
\end{itemize}

\section{Atividades e metas}

\subsection{Meta 1: Realizar revisão sistemática}

Atividades:

\begin{itemize}
  \item Pesquisar estudos secundários sobre o tema;
  \item Identificar fontes de dados: bibliotecas digitais; anais em
    conferências sobre ferramentas;
  \item Levantar e selecionar dados/papers sobre ferramentas de software
  \item Iniciar e documentar análise dos papers selecionados
  \item Agregar e sintetizar informações encontradas
\end{itemize}

\subsection{Meta 2: Divulgar resultados}

Atividades:

\begin{itemize}
  \item Apresentar qualificação de mestrado
  \item Elaborar artigo científico sobre caracterização de ferramentas de pesquisa publicadas no Brasil
  \item Elaborar artigo científico com lições a respeito de desenvolvimento e publicação de ferramentas de pesquisa
  \item Elaborar artigo científico com recomendações para desenvolvimento e adoção de ferramentas de pesquisa
  \item Elaborar dissertação de mestrado
  \item Apresentar dissertação de mestrado
\end{itemize}

\bibliography{bibliografia}

\end{document}
