\documentclass[12pt]{article}
\usepackage[utf8]{inputenc}
\usepackage[brazil]{babel}
\usepackage{fancyvrb}
\usepackage[alf]{abntcite}
\usepackage[top=2cm,left=0.5in,right=0.5in,bottom=2cm]{geometry}

\title{
  Caracterização de Ferramentas de Pesquisa no Contexto de Engenharia de
  Software Experimental
}
\author{Joenio Marques da Costa\\
  {\small Universidade Federal da Bahia (UFBA)} \\
  {\small joenio@colivre.coop.br}
}
\date{\today}

% sugestão do professor Ecivaldo sobre a ordem para escrever a dissertação
% 01 Problemática
% 02 Problemas (+hipóteses se houver)
% 03 Objetivos (Geral e Específicos)
% 04 Metodologia
% 05 Revisão sistemática
% 06 Resultados parciais (se houver)
% 07 Considerações parciais
% 08 Problemática (processo ciclico)
% 09 Resumo/Abstract
% 10 Título
% 11 Revisão

\begin{document}

\maketitle

\section{Resumo do projeto}

Em diversas linhas de pesquisa da Computação e, em especial, em Engenharia de
Software, é bastante comum que novos sistemas de software sejam desenvolvidos,
tais sistemas costumam ser utilizados como meio para atingir os resultados da
pesquisa ou, em alguns casos, são o próprio fim do estudo realizado. Neste
trabalho, tais sistemas ou ferramentas de software são nosso objeto de
pesquisa e serão chamados de {\it ferramentas de pesquisa}.

Seja como meio ou como fim, tais {\it ferramentas de pesquisa} são produtos de
software e, tal como propõe a Engenharia de Software Experimental, precisam
ser avaliadas com uso de métodos científicos adequados.

Neste contexto, é de fundamental importância
que as ferramentas de pesquisa utilizadas ou desenvolvidas durante estudos
estejam disponíveis e em funcionamento.

%Nas últimas décadas, o foco em estudos empíricos na área de
%Engenharia de Software tem crescido, resultando no uso crescente de métodos
%científicos como surveys, estudos de caso, experimentos e revisões
%sistemáticas de literatura.  A replicação desses estudos empíricos podem, e
%devem, ser realizados, de modo a averiguar a sua validade e aumentar o nível
%de confiança em seus resultados. 

Neste trabalho de mestrado, será realizada uma revisão sistemática de
literatura, a partir de artigos da área de Engenharia de Software que tratam
de publicação de ferramentas de pesquisa, visando coletar e caracterizar
alguns atributos de ferramentas de pesquisa e prover recomendações para seu
desenvolvimento e adoção em ampla escala em estudos empíricos em Engenharia de
Software.

\section{Objetivos}

O objetivo principal deste trabalho é melhorar a compreensão sobre as
ferramentas de software desenvolvidas durante pesquisas em Engenharia de
Software -- aqui denominadas “ferramentas de pesquisa” -- por meio de sua
caracterização no contexto de estudos empíricos em Engenharia de Software. 

São objetivos específicos deste trabalho:

\begin{itemize}
  \item Realizar uma revisão sistemática, com base em artigos publicados nas
    principais conferências e periódicos da área de Engenharia de Software,
    sobre “ferramentas de pesquisa”;
  \item Caracterizar as “ferramentas de pesquisa” coletadas em termos de
    atributos relacionados ao seu desenvolvimento;
  \item Caracterizar  “ferramentas de pesquisa” coletadas em termos de
    atributos relacionados à sua distribuição;
  \item Caracterizar “ferramentas de pesquisa” coletadas  em termos de
    atributos relacionados ao seu uso em estudos empíricos, sejam primários ou
    replicações;
  \item Sintetizar informações sobre “ferramentas de pesquisa” coletadas;
  \item Prover um conjunto inicial de recomendações para facilitar a adoção de
    “ferramentas de pesquisa”  em estudos empíricos em Engenharia de Software. 
\end{itemize}

\section{Fundamentação teórica}

Sistemas de software são utilizados em praticamente todas as áreas do
conhecimento humano e têm exercido um papel essencial em nossa sociedade
\cite{Mafra2006}. A dependência crescente de serviços oferecidos por tais
sistemas evidencia a necessidade de produzir software de qualidade e contornar
os  desafios relacionados a sua funcionalidade (incompleta ou incorreta),
custos acima do esperado ou prazos não cumpridos.

Diante destes desafios, surge a Engenharia de Software, uma disciplina
centrada no desenvolvimento de sistemas de software \cite{Wohlin2012} através
da aplicação de uma abordagem sistemática, disciplinada, e quantificável para
o desenvolvimento, operação e manutenção \cite{SWEBOK2014}.

Nas últimas décadas, o foco em estudos empíricos na área de Engenharia de
Software tem crescido significantemente \cite{Stol2015}, resultando no uso
crescente de métodos como surveys, estudos de caso, experimentos e revisões
sistemáticas de literatura. Ao fazer uso de estudos empíricos, pesquisadores
transformam a Engenharia de Software em uma disciplina mais científica e
controlável -- a  Engenharia de Software Experimental -- provendo meios para
avaliar e validar novos métodos, técnicas, linguagens e ferramentas.

O crescimento no número de pesquisas e publicações em Engenharia de Software
Experimental desperta a atenção para a necessidade de verificar a validade dos
estudos empíricos realizados -- um ponto central em qualquer pesquisa
científica. A validade de um estudo empírico  deve ser averiguada, para
aumentar o nível de confiança em seus resultados. A replicação é um importante
meio para atingir tal objetivo \cite{Almqvist2006}.

Um dos primeiros artigos discutindo replicação de experimentos em Engenharia
de Software foi publicado por Basili et al. \cite{Mantyla2010} e sugere
replicação não apenas como uma escolha, mas como um possível "próximo passo" a
ser tomado após o experimento original ser concluído. Apesar do conceito
replicação de estudos empíricos em Engenharia de Software estar usualmente
associado à experimentação, argumenta-se que ele deve ser estendido para
incluir ao menos estudos de caso e surveys \cite{Basili1986}.

Em diversas linhas de pesquisa da Computação e, em especial, em Engenharia de
Software, é bastante comum que ferramentas de software sejam desenvolvidas
para apoiar a pesquisa ou sejam o resultado da própria pesquisa. Neste
trabalho, tais ferramentas de software são nosso objeto de pesquisa e serão
chamadas de “ferramentas de pesquisa” -- termo utilizado também por Portillo
\cite{Portillo12}.

Ferramentas de pesquisa são produtos de software e, em geral, precisam ser
avaliados com uso de métodos científicos adequados. Diante da importância da
replicação para a validação de estudos empíricos, é de fundamental importância
que as ferramentas de pesquisa utilizadas ou desenvolvidas no estudo original
estejam disponíveis e em funcionamento \cite{Kon2011}.

\subsection{Revisão Sistemática e Meta-análise}

Muitos pesquisadores argumentam que, para se obter progressos em uma
determinada área do conhecimento, os resultados de vários experimentos e
outros estudos empíricos (surveys e estudos de caso, por exemplo) devem ser
combinados. Quando um conjunto de estudos empíricos é coletado sobre um
tópico, a síntese ou agregação entra em cena. Síntese baseada em métodos
estatísticos é referenciada como meta-análise \cite{Almqvist2006}.

Se os procedimentos da meta-análise não são aplicáveis, a síntese descritiva
deve ser utilizada. Esta inclui visualização e tabulação dos dados e
estatística descritiva dos dados. Quanto mais ampla é a questão guiando a
revisão de literatura, mais métodos qualitativos são necessários para sua
síntese. Cruzes e Dyba \cite{Cruzes2011} apresentam uma visão geral de métodos
qualitativos de síntese, entre eles síntese temática.

Adicionalmente, com o aumento na adoção de estudos empíricos em Engenharia de
Software, surge a necessidade de agregar evidências de múltiplos estudos
relacionados, de modo a obter respostas a questões impossíveis de serem
respondidas com os estudos individuais. A coleta e síntese de evidências
empíricas podem ser realizadas com rigor científico, por meio de Revisão
Sistemática da Literatura \cite{Kitchenham2007}. A Revisão Sistemática da
Literatura  é um meio de avaliar e interpretar pesquisas relevantes (estudos
primários) sobre uma data questão em particular, tópico, área, ou fenômeno de
interesse.

\section{Metodologia}

Primeiramente será feita uma revisão sobre estudos secundários (em especial,
revisões sistemáticas) relacionados ao uso e desenvolvimento de ferramentas de
pesquisa.
  
Em seguida, será realizada uma revisão sistemática de literatura, com base nas
recomendações encontradas em \cite{Kitchenham2007}, a partir
de artigos da área de Engenharia de Software que tratam de publicação de
ferramentas de pesquisa, visando caracterizar seus atributos a partir de
perguntas, por exemplo:

\begin{itemize}
  \item Quais são as ferramentas de pesquisa associadas a pesquisas em
    engenharia de software?
  \item Como as ferramentas de pesquisa publicadas nestes estudos são
    licenciadas e distribuídas?
  \item Como as ferramentas de pesquisa são desenvolvidas, em termos de
    métodos e processos recomendados pela engenharia de software?
  \item Há colaboração entre pesquisadores de universidades distintas no
    desenvolvimento das ferramentas de pesquisa publicadas?
  \item As ferramentas publicadas são avaliadas internamente e externamente?
    Se sim, como são avaliadas?
\end{itemize}

A partir das informações coletadas na revisão sistemática serão feitas a
agregação e a síntese dos dados, utilizando métodos quantitativos e
qualitativos, com objetivo de identificar temas recorrentes e problemas
comuns, além de elaborar conclusão a respeito dos diversos estudos analisados
na revisão sistemática.

Após a caracterização realizada, espera-se propor um conjunto preliminar de
recomendações para desenvolvimento e adoção de ferramentas de pesquisa para a
comunidade acadêmica de Engenharia de Software.

\section{Resultados esperados}

\begin{itemize}
  \item Caracterização dos atributos das ferramentas de software desenvolvidas
    durante pesquisas em engenharia de software, chamadas aqui de {\it “ferramentas
    de pesquisa”};
  \item Síntese dos resultados e lições a respeito de temas recorrentes e
    problemas comuns no desenvolvimento e publicação de {\it “ferramentas de
    pesquisa”};
  \item Conjunto preliminar de recomendações para desenvolvimento e adoção de
    ferramentas de pesquisa para a comunidade acadêmica de Engenharia de
    Software;
  \item Artigos científicos publicados;
  \item Dissertação de mestrado.
\end{itemize}


\section{Atividades e metas}

\subsection{Meta 1: Realizar revisão sistemática}

Atividades:

\begin{itemize}
  \item Pesquisar estudos secundários sobre o tema;
  \item Identificar fontes de dados: bibliotecas digitais; anais em
    conferências sobre ferramentas;
  \item Levantar e selecionar dados/papers sobre ferramentas de software
  \item Iniciar e documentar análise dos papers selecionados
  \item Agregar e sintetizar informações encontradas
\end{itemize}

\subsection{Meta 2: Divulgar resultados}

Atividades:

\begin{itemize}
  \item Apresentar qualificação de mestrado
  \item Elaborar artigo científico sobre caracterização de ferramentas de pesquisa publicadas no Brasil
  \item Elaborar artigo científico com lições a respeito de desenvolvimento e publicação de ferramentas de pesquisa
  \item Elaborar artigo científico com recomendações para desenvolvimento e adoção de ferramentas de pesquisa
  \item Elaborar dissertação de mestrado
  \item Apresentar dissertação de mestrado
\end{itemize}

\bibliography{bibliografia}

\end{document}
