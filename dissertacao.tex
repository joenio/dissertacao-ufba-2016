\documentclass[12pt]{article}
\usepackage[utf8]{inputenc}
\usepackage[brazil]{babel}
\usepackage{fancyvrb}
\usepackage[alf]{abntcite}
%\bibliographystyle{abnt-alf}
\usepackage[top=2cm,left=0.5in,right=0.5in,bottom=2cm]{geometry}

\title{Ciclo de vida de ferramentas desenvolvidas durante pesquisas em
        engenharia de software: Uma revisão sistemática de literatura
}
\author{Joenio Marques da Costa\\
  {\small Universidade Federal da Bahia (UFBA)} \\
  {\small joenio@colivre.coop.br}
}
\date{\today}


\begin{document}

\maketitle

% O projeto deve, portanto, especificar os objetivos da pesquisa, apresentar a
% justificativa de sua realização, definir a modalidade de pesquisa e determinar
% os procedimentos de coleta e análise de dados. Deve, ainda, esclarecer
% acerca do cronograma a ser seguido no desenvolvimento da pesquisa e
% proporcionar a indicação dos recursos humanos, financeiros e materiais
% necessários para assegurar o êxito da pesquisa.

\section{Problema}

%* o problema claramente formulado
%
% O que  um problema?
% Questão não solvida e que é objeto de discussão, em qualquer domínio do conhecimento;
% Um problema é de natureza científica quando envolve variáveis que podem ser tidas como testáveis;
% Exemplos:
% "Em que medida a escolaridade determina a preferência político-partidária?"
% "A desnutrição determina o rebaixamento intelectual?"
% Como formular um problema?
% (a) O problema deve ser formulado como pergunta;
% (b) O problema deve ser claro e preciso;
% (c) O problema deve ser empírico;
% (d) O problema deve ser suscetível de solução; 
% (e) O problema deve ser delimitado a uma dimensão viável

Os pesquisadores de engenharia de software encontram dificuldades em utilizar
ferramentas desenvolvidas pela própria comunidade acadêmica em suas pesquisas,
estas dificuldades geram como consequência ao menos dois problemas:

\begin{itemize}
\item Duplicação de esforço\\
        {\it o pesquisador não adota ferramenta existente e desenvolve sua
        própria solução}
\item Dificuldade em replicar estudos\\
        {\it o pesquisador não consegue reproduzir pesquisas anteriores,
        algo necessário para aumentar a validade externa dos estudos}
\end{itemize}

Com o objetivo de entender os fatores que geram tais dificuldades, será feita
uma caracterização das ferramentas desenvolvidas nos últimos 15 anos de
pesquisas em engenharia de software a partir das seguintes questões:

\begin{itemize}
\item Quando foi lançada publicamente?
\item Quando recebeu sua última atualização?
\item Qual pesquisador desenvolveu?
\item Qual grupo de pesquisa desenvolveu?
\item Dá suporte para quais sistemas operacionais?
\item Em qual linguagem de programação foi escrita?
\item Em quais momentos foi utilizada após o lançamento?
\item Qual pesquisador utilizou após o lançamento?
\item Quais grupos de pesquisa utilizaram após o lançamento?
\item Qual problema resolve, qual o domínio?
\item É utilizada fora do seu grupo de pesquisa?
\item É utilizada fora da comunidade acadêmica?
\item É adotada pela indústria?
\item É derivada ou é extensão de alguma outra ferramenta?
\item Tem cobertura de testes?
\item Qual o nível da cobertura de testes?
\item Tem documentação?
\item Em quais idiomas a documentação está disponível?
%(fiz anotações nos artigos sobre outras questões, consultar e completar esta lista aqui)
\end{itemize}

Esta caracterização permitirá entender alguns aspectos do ciclo de vida de
ferramentas desenvolvidas em pesquisas de engenharia de software nos últimos
15 anos.

%, bem como servir de apoio para demonstrar, verificar ou testar algumas
%das hipóteses listadas abaixo a respeito de quais fatores contribuem para a
%dificuldade dos pesquisadores na utilização destas ferramentas.
%\begin{itemize}
%\item Dificuldade em obter a ferramenta, site fora do ar, link quebrado, etc
%\item Problemas com licenciamento, não-livre, etc
%\item Setup e configurações complexas
%\item Ego individual faz o pesquisador desenvolver sua própria ferramenta
%\item Ferramentas sem qualidade, os critérios utilizados ao revisar
%        os papers deveriam ser também aplicados às ferramentas
%\item Curva de aprendizado difícil, torna-se mais fácil fazer algo "home made"
%\end{itemize}

% \section{Coleta e análse dos dados}
% %* o plano de coleta e análise dos dados

\section{Metodologia}

\begin{itemize}
\item Levantar revisões sistemáticas de literatura de engenharia de software\\
        {\it pode haver algum estudo anterior com a mesma abordagem e que
        responda ao problema citado aqui}
\item Levantar revisões sistemáticas de literatura sobre ferramentas de
        engenharia de software\\
        {\it pode existir estudos similares ou complementares ao problema
        levantado aqui}
\item Iniciar revisão sistemática de literatura para responder ao problema
        definido\\
        {\it buscar estudos dos últimos 15 anos de pesquisas em engenharia de
        software}
\end{itemize}

\section{Cronograma}

{\it pendente}

% \bibliography{bibliografia}

\end{document}
