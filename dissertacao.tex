\documentclass[12pt]{article}
\usepackage[utf8]{inputenc}
\usepackage[brazil]{babel}
\usepackage{fancyvrb}
\usepackage[alf]{abntcite}
%\bibliographystyle{abnt-alf}
\usepackage[top=2cm,left=0.5in,right=0.5in,bottom=2cm]{geometry}

\title{???
}
\author{Joenio Marques da Costa\\
  {\small Universidade Federal da Bahia (UFBA)} \\
  {\small joenio@colivre.coop.br}
}
\date{\today}

\begin{document}

\maketitle

Softwares são utilizados em praticamente todas as áreas do conhecimento humano
e têm exercido um papel essencial em nossa sociedade, nós dependemos das
características e serviços oferecidos por sistemas computadorizados
\cite{Mafra2006} e isto pôe o software num lugar de destaque evidenciando a
importancia em preocupar-se com a qualidade dos softwares, como eles são
contruídos, distribuídos e mantidos.

A atividade de desenvolvimento de software é um processo altamente criativo e
humano, equipes podem por muito pequenas ou programadores individuais até
equipes com centenas de pessoas distribuídas ao redor do globo. O rápido
crescimento da área e a enorme quantidade de softwares desenvolvidos ao longo
dos anos significa também que muitos projetos tem problemas em termos de
funcionalidades faltando, custos acima do esperado, prazos não cumpridos ou
qualidade ruim.

O termo engenharia de software surgiu a partir destes desafios com intenção
resolvêlos focando em desenvolvimento intensivo de sistemas de software
\cite{Wohlin2012} através da aplicação sistemática, disciplinada, e
quantificável de uma abordagem para o desenvolvimento, operação e manutenção
de software \cite{SWEBOK2014}. A engenharia de software através de métodos
científicos, especialmente estudos empíricos, podem ser aplicados a todas as 3
abordagens (sistemática, disciplinada e quantificável), e transformar a
engenharia de software em uma atividade mais científica e como estudos
empíricos tem um papel importante neste quesito.

Dentre os métodos empíricos, os experimentos são ferramentas importantes para
todo engenheiro de software que está avaliando e escolhendo diferentes
métodos, técnicas, linguagens e ferramentas. A aplicação destes métodos é
fudamental para ter mais controle sobre os softwares desenvolvidos ao invés de
soluções basedas em marketing e convicção. Esta tem sido uma necessidade
crescente avaliar e validar novas propostas... Estudos empíricos incluem
"surveys", "estudos de caso", "revisão sistemática de literatura" além de
"experimentos".

Novos métodos, técnicas, linguagens e ferramentas devem não só sugeridas,
publicadas e divulgadas, mas é crucial avaliar novas invenções e propostas em
comparação com as já existentes. Experimentação provê esta oportunidade e deve
ser usada de acordo, em outras palavras nós devemos usar métodos e estratégias
disponíveis quando conduzimos poesquisa em engenharia de software.

Pesquisas em engenharia de software tem crescido ao longo do tempo e o
amadurecimento é visível em termos de X e Y, estas pesquisas geram além dos
resultados científicos relatados em papers, relatórios, etc. Geram ferramentas
ou produtos de software, estes softwares resultados de estudos e pesquisas na
área podem ser criadas como forma de apoio ou podem ser o resultado final da
pesquisa, seja como for estas ferramentas podem também ser consideradas como
produção da comunidade de engenharia de software e precisam também serem
avalias assim como são avaliados os resultados da indústria.

Estes softwares desenvolvidos durante pesquisas serão chamados aqui de
ferramentas de pesquisa, este termos será utilizado para evitar confusão com
outras ferramentas de software criadas fora do contexto acadêmico. Assim como
é evidenciado a necessidade de avaliar ferramentas desenvolvidas pelos
praticantes de engenharia de software estas ferramentas de pesquisa também
precisam ser avaliadas, etc...

Conhecer como eles são desenvolvidos, publicados e mantidos ao longo do tempo
é também crucial para entender o seguinte problema: pesquisadores enfrentam
dificuldades na utilização de ferramentas de pesquisa.

Agregando evidências de estudos empíricos, com o aumento de estudos empíricos
surge a necessidade de agregar evidências de multiplos estudos, isto pode dar
respostas a questões não possíveis de serem respondidas com estudos
inidivuais. A coleção e síntese de evidências empíricas deve ser feito com
rigor científico também e isto pode ser feito revisão sistemática de
literatura.

Revisões sistemáticas são estudos secundários e foi proposto uma metodologia
para engenharia de software por kichetman and bla... 

When a set of empirical studies is collected on a topic, the synthesis or
aggregation takes place. Syntheses based on statistical methods are referred to as
meta-analysis. Examples of meta-analyses in software engineering include defect
detection methods [74, 121], agile methods [46], and pair programming [73].
If the meta-analysis procedures do not apply, descriptive synthesis has to be used.
These include visualization and tabulation of data and descriptive statistics of the
data [96]. The broader research question for a literature review, the more qualitative
methods are needed for its synthesis. Cruzes and Dyb ̊a present an overview of
qualitative synthesis methods [39].

\bibliography{bibliografia}

\end{document}
